%voor handouts:
%\documentclass[handout]{beamer}
\documentclass[aspectratio=169]{beamer}
\usepackage{ifthen}
\usepackage{graphicx}
\usepackage{booktabs}
\usepackage{tabularx}
\usepackage{algorithm}
%\usepackage{algorithmx}
%\usepackage{cwpuzzle}
\usepackage{algpseudocode}
\usepackage[english]{babel}
%\usepackage{qtree}
\usepackage{amssymb}
\usepackage{amsmath,amsthm}
\usepackage{subcaption}
\usepackage{mathrsfs}
%\usepackage{eurosym}
\usepackage{soul}
\usepackage{fontawesome}
\usepackage{etoolbox}
\usepackage{soul}
\usepackage{multicol}
\usepackage{tikz}
\usepackage{xstring}
\usepackage{pgfplots}
\usepackage{etoolbox}
\usepackage{ifthen}
\usetikzlibrary{arrows,shapes}
\usetikzlibrary{positioning, calc}

\definecolor{hospital}{RGB}{0,0,255}
\definecolor{patient}{RGB}{255,0,0}

\tikzstyle{flow_node}=[circle,draw=black,minimum size=15pt,inner sep=0pt]
\tikzstyle{flow_s}=[fill=green!20]
\tikzstyle{flow_subtask}=[fill=green!80]
\tikzstyle{flow_task}=[fill=red!80]
\tikzstyle{flow_t}=[fill=blue!20]
\tikzstyle{flow_edge}=[->, ultra thick]
\tikzstyle{flow_capacity}=[fill=white, inner sep = 1pt]
\tikzstyle{flow_edge_mincut}=[draw=red]
\tikzstyle{flow_edge_fullcap}=[draw=red]
\tikzstyle{flow_mincut}=[dashed, thick]
\tikzstyle{flow_supply}=[fill=green!80]
\tikzstyle{flow_demand}=[fill=red!80]

\newcommand\ifMaxFlow[4]{
	\begingroup
		\pgfmathsetmacro{\flow}{#1}
		\pgfmathsetmacro{\cap}{#2}
		\pgfmathparse{ifthenelse(\flow == \cap,1,0)}
		\ifdim\pgfmathresult pt= 1 pt
			 #3
			\else
			 #4
		\fi
	\endgroup
}

\newcommand\ifSomeFlow[3]{
	\begingroup
		\pgfmathsetmacro{\flow}{#1}
		\pgfmathparse{ifthenelse(\flow > 0,1,0)}
		\ifdim\pgfmathresult pt= 1 pt
			 #2
			\else
			 #3
		\fi
	\endgroup
}


\usepackage[duration=105,lastminutes=15]{../common/pdfpcnotes}

\makeatletter
\let\@@magyar@captionfix\relax
\makeatother

\newcommand{\bigO}{O}

\usetheme[width=0mm]{PaloAlto}
\setbeamertemplate{navigation symbols}
{\ifthenelse{\not\equal{\thepage}{1}}
  {\insertframenumber}
  {}
}
\setbeamertemplate{footline}%
{\ifthenelse{\not\equal{\thepage}{1}}%
  {\color{gray!30!white}{\tiny \copyright 2018 TU Delft}}% \hfill\insertframenumber/\inserttotalframenumber}
  {}
}

\newenvironment<>{problemblock}[1]{%
  \begin{actionenv}#2%
      \def\insertblocktitle{Problem: #1}%
      \par%
      \mode<presentation>{%
        \setbeamercolor{block title}{fg=white,bg=orange!80!black}
       \setbeamercolor{block body}{fg=black,bg=orange!40}
       \setbeamercolor{itemize item}{fg=orange!20!black}
       \setbeamertemplate{itemize item}[triangle]
     }%
      \usebeamertemplate{block begin}}
    {\par\usebeamertemplate{block end}\end{actionenv}}

\newenvironment<>{questionblock}[1]{%
  \begin{actionenv}#2%
      \def\insertblocktitle{Question: #1}%
      \par%
      \mode<presentation>{%
        \setbeamercolor{block title}{fg=white,bg=cyan!80!black}
       \setbeamercolor{block body}{fg=black,bg=cyan!30}
       \setbeamercolor{itemize item}{fg=cyan!20!black}
       \setbeamertemplate{itemize item}[triangle]
     }%
      \usebeamertemplate{block begin}}
    {\par\usebeamertemplate{block end}\end{actionenv}}

\newenvironment<>{answerblock}[1]{%
  \begin{actionenv}#2%
      \def\insertblocktitle{Answer: #1}%
      \par%
      \mode<presentation>{%
        \setbeamercolor{block title}{fg=white,bg=green!80!black}
       \setbeamercolor{block body}{fg=black,bg=green!30}
       \setbeamercolor{itemize item}{fg=green!20!black}
       \setbeamertemplate{itemize item}[triangle]
     }%
      \usebeamertemplate{block begin}}
    {\par\usebeamertemplate{block end}\end{actionenv}}


\title[Algorithms \& Datastructures]{TI1520TW Algorithms \& Datastructures}
\subtitle{\color{cyan} \textbf{Tutorial 4: Trees!}}
\author{Stefan Hugtenburg\\ {\tiny{\qquad~~\copyright 2019 TU Delft}}}
\institute{CSE Teaching Team | EEMCS | TU Delft}
\titlegraphic{\includegraphics[height=1cm]{../common/logo.pdf}~\hfill~\includegraphics[height=1cm]{../common/cc-by-nc-sa.png}}
\date{2018-2019 Q3}

\begin{document}

\frame{\titlepage}

\section{Introduction}

\begin{frame}
	\frametitle{You are here.}
	\begin{block}{The course so far}
		\begin{itemize}
			\item Time and Space Complexity
			\item Array-based vs Linked Lists
		\end{itemize}
	\end{block}
	\pause
	\begin{exampleblock}{Today's content}
		\begin{itemize}
			\item Stacks \& Queues
			\item Example: Parsing LaTeX
			\item An experiment (of great scientific value)
		\end{itemize}
	\end{exampleblock}
	\pause
	\begin{block}{The future}
		\begin{itemize}
			\item Next time: Sorting
			\item Trees, Graphs...
			\item P vs NP, and much more!
		\end{itemize}
	\end{block}
\end{frame}


\begin{frame}
	\frametitle{Remarks/reminders}
	\begin{itemize}
		\item Collegerama people are hard to get hold off...
			\pause
		\item I will try to upload handout versions of my slide too. Less scrolling for you :)
			\pause
		\item Thank you for your feedback on the lab assignments of last week, we will use them to improve the lab for next
			year!
	\end{itemize}
	
\end{frame}


\begin{frame}
	\frametitle{Given a traversal\dots}
	
	\begin{problemblock}{In-order}
		Given an in-order traversal of a tree that returns the sequence:\\
		1, 2, 7, 12, 3, 5, 42, -2, 0, 18.\\
		Give a binary tree of height 4 that produces this in-order traversal.
	\end{problemblock}
\end{frame}

\begin{frame}
	\frametitle{Counting leafs}
	\framesubtitle{Based on C-8.34 in the book}
	\begin{problemblock}{Something fun?}
		Consider a tree with $n_l$ internal nodes and $n_E$ leafs. Show that if every internal node has exactly 3 children,
		then $n_E = 2n_l + 1$.
	\end{problemblock}
\end{frame}

\begin{frame}
	\frametitle{Removing an item}
	\begin{problemblock}{Let's remove items from an AVL-tree}
		\begin{itemize}
			\item Let's draw an AVL-tree.
			\item Let's remove an item.
			\item Let's fix the damage.
		\end{itemize}
	\end{problemblock}
\end{frame}

\frame{\titlepage}

\end{document}
