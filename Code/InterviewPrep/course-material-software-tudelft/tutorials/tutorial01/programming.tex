\section{Programming part}
\label{sec:programming_part}


\begin{frame}
	\frametitle{List comprehensions}
	\framesubtitle{5 minutes}
	\begin{problemblock}{Lists!}
		Create the list \texttt{[0,2,6,12,20,30,42,56,72,90]} using a list-comprehension.
	\end{problemblock}
\end{frame}

\begin{frame}
	\frametitle{Duplicate detection}
	\framesubtitle{5 minutes}
	\begin{problemblock}{SELECT DISTINCT(id) FROM user}
		\begin{itemize}
			\item Given a list, check if the items are all distinct. 
			\item In other words check if the list has duplicates.
		\end{itemize}	
	\end{problemblock}
\end{frame}

\begin{frame}
	\frametitle{MinMax function}
	\framesubtitle{5 minutes}

	\begin{problemblock}{Minmaxing \st{life} lists}
		Implement a function that returns both the minimum and maximum of a list.
	\end{problemblock}
\end{frame}

\begin{frame}
	\frametitle{MinMax function, using recursion}
	\framesubtitle{10 minutes}

	\begin{problemblock}{Minmaxing \st{life} lists recursively}
		\begin{itemize}
			\item Implement a function that returns both the minimum and maximum of a list.
			\item But this time, it should be recursive!
		\end{itemize}
	\end{problemblock}
\end{frame}
