%voor handouts:
%\documentclass[handout]{beamer}
\documentclass[aspectratio=169]{beamer}
\usepackage{ifthen}
\usepackage{graphicx}
\usepackage{booktabs}
\usepackage{tabularx}
\usepackage{algorithm}
%\usepackage{algorithmx}
%\usepackage{cwpuzzle}
\usepackage{algpseudocode}
\usepackage[english]{babel}
%\usepackage{qtree}
\usepackage{amssymb}
\usepackage{amsmath,amsthm}
\usepackage{subcaption}
\usepackage{mathrsfs}
%\usepackage{eurosym}
\usepackage{soul}
\usepackage{fontawesome}
\usepackage{etoolbox}
\usepackage{soul}
\usepackage{multicol}
\usepackage{tikz}
\usepackage{xstring}
\usepackage{pgfplots}
\usepackage{etoolbox}
\usepackage{ifthen}
\usetikzlibrary{arrows,shapes}
\usetikzlibrary{positioning, calc}

\definecolor{hospital}{RGB}{0,0,255}
\definecolor{patient}{RGB}{255,0,0}

\tikzstyle{flow_node}=[circle,draw=black,minimum size=15pt,inner sep=0pt]
\tikzstyle{flow_s}=[fill=green!20]
\tikzstyle{flow_subtask}=[fill=green!80]
\tikzstyle{flow_task}=[fill=red!80]
\tikzstyle{flow_t}=[fill=blue!20]
\tikzstyle{flow_edge}=[->, ultra thick]
\tikzstyle{flow_capacity}=[fill=white, inner sep = 1pt]
\tikzstyle{flow_edge_mincut}=[draw=red]
\tikzstyle{flow_edge_fullcap}=[draw=red]
\tikzstyle{flow_mincut}=[dashed, thick]
\tikzstyle{flow_supply}=[fill=green!80]
\tikzstyle{flow_demand}=[fill=red!80]

\newcommand\ifMaxFlow[4]{
	\begingroup
		\pgfmathsetmacro{\flow}{#1}
		\pgfmathsetmacro{\cap}{#2}
		\pgfmathparse{ifthenelse(\flow == \cap,1,0)}
		\ifdim\pgfmathresult pt= 1 pt
			 #3
			\else
			 #4
		\fi
	\endgroup
}

\newcommand\ifSomeFlow[3]{
	\begingroup
		\pgfmathsetmacro{\flow}{#1}
		\pgfmathparse{ifthenelse(\flow > 0,1,0)}
		\ifdim\pgfmathresult pt= 1 pt
			 #2
			\else
			 #3
		\fi
	\endgroup
}


\usepackage[duration=105,lastminutes=15]{../common/pdfpcnotes}

\makeatletter
\let\@@magyar@captionfix\relax
\makeatother

\newcommand{\bigO}{O}

\usetheme[width=0mm]{PaloAlto}
\setbeamertemplate{navigation symbols}
{\ifthenelse{\not\equal{\thepage}{1}}
  {\insertframenumber}
  {}
}
\setbeamertemplate{footline}%
{\ifthenelse{\not\equal{\thepage}{1}}%
  {\color{gray!30!white}{\tiny \copyright 2018 TU Delft}}% \hfill\insertframenumber/\inserttotalframenumber}
  {}
}

\newenvironment<>{problemblock}[1]{%
  \begin{actionenv}#2%
      \def\insertblocktitle{Problem: #1}%
      \par%
      \mode<presentation>{%
        \setbeamercolor{block title}{fg=white,bg=orange!80!black}
       \setbeamercolor{block body}{fg=black,bg=orange!40}
       \setbeamercolor{itemize item}{fg=orange!20!black}
       \setbeamertemplate{itemize item}[triangle]
     }%
      \usebeamertemplate{block begin}}
    {\par\usebeamertemplate{block end}\end{actionenv}}

\newenvironment<>{questionblock}[1]{%
  \begin{actionenv}#2%
      \def\insertblocktitle{Question: #1}%
      \par%
      \mode<presentation>{%
        \setbeamercolor{block title}{fg=white,bg=cyan!80!black}
       \setbeamercolor{block body}{fg=black,bg=cyan!30}
       \setbeamercolor{itemize item}{fg=cyan!20!black}
       \setbeamertemplate{itemize item}[triangle]
     }%
      \usebeamertemplate{block begin}}
    {\par\usebeamertemplate{block end}\end{actionenv}}

\newenvironment<>{answerblock}[1]{%
  \begin{actionenv}#2%
      \def\insertblocktitle{Answer: #1}%
      \par%
      \mode<presentation>{%
        \setbeamercolor{block title}{fg=white,bg=green!80!black}
       \setbeamercolor{block body}{fg=black,bg=green!30}
       \setbeamercolor{itemize item}{fg=green!20!black}
       \setbeamertemplate{itemize item}[triangle]
     }%
      \usebeamertemplate{block begin}}
    {\par\usebeamertemplate{block end}\end{actionenv}}


\title[Algorithms \& Datastructures]{TI1520TW Algorithms \& Datastructures}
\subtitle{\color{cyan} \textbf{Tutorial 5: Maps and Graphs}}
\author{Stefan Hugtenburg\\ {\tiny{\qquad~~\copyright 2019 TU Delft}}}
\institute{CSE Teaching Team | EEMCS | TU Delft}
\titlegraphic{\includegraphics[height=1cm]{../common/logo.pdf}~\hfill~\includegraphics[height=1cm]{../common/cc-by-nc-sa.png}}
\date{2018-2019 Q3}

\begin{document}

\frame{\titlepage}

\section{Introduction}

\begin{frame}
	\frametitle{Inserting in a growing hashmap}

	\begin{questionblock}{Growing maps}
		We are adding the following keys in the following order. Assume the initial size of the map is $2$ and it grows
		every time it is ``at least half full''. We use separate chaining with the hash function $f(k) = 3k \mod \texttt{map.size}$.\\
		$1, 8, 2, 37 , 3, 513, 4138732, 19, 27, 4, 17, 9, 89$
	\end{questionblock}
	
\end{frame}

\begin{frame}
	\frametitle{Graph things}
	\framesubtitle{Excercise 14.16 from the book}
		\begin{columns}
			\column{0.605\textwidth}
			\begin{questionblock}{Graphs}
				Let G be an undirected graph whose vertices are the integers 1 through 8, and let the adjacent vertices of each
				vertex be given by the table to the right.
				Assume that, in a traversal of G, the adjacent vertices of a given vertex are returned in the same order as they are
				listed in the table above.
				\begin{enumerate}
					\item 
						Draw G.
					\item
						Give the sequence of vertices of G visited using a DFS traversal
						starting at vertex 1.
					\item
						Give the sequence of vertices visited using a BFS traversal starting
						at vertex 1.
				\end{enumerate}
	\end{questionblock}
			\column{0.405\textwidth}
	\begin{tabular}{c | c}
		vertex & adjacent vertices\\
		\midrule
		1  & (2, 3, 4)\\
		2  & (1, 3, 4)\\
		3  & (1, 2, 4)\\
		4  & (1, 2, 3, 6)\\
		5  & (6, 7, 8)\\
		6  & (4, 5, 7)\\
		7  & (5, 6, 8)\\
		8  & (5, 7)
	\end{tabular}
		\end{columns}
\end{frame}

\begin{frame}
	\frametitle{On the topic of DFS}
	\framesubtitle{Exercise 14.14 from the book}
	\begin{questionblock}{DFS on complete graphs}
		A simple undirected graph is complete if it contains an edge between every
pair of distinct vertices. What does a depth-first search tree of a complete
graph look like?
	\end{questionblock}
	
\end{frame}

\frame{\titlepage}

\end{document}
