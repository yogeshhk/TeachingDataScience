\subsection{Homeworks}
\label{sub:homeworks}

There will be a total of 6 sets of homework exercises. Homeworks are not mandatory, but there will still be lab sessions
every week. During these lab sessions, students can visit a TA and receive feedback on their assignments. Each
assignment consists of both implementation and analysis questions, unless indicated otherwise.

\subsection*{Assignment 1}
\label{sub:assignment_1}

This homework assignment is mostly about complexity analysis of iterative code and it concerns
\cref{lo:notation,lo:runtime} of the course and has a deadline in week 2. There are no implementation exercises about
new material this week, but there are some more exercises to practice with the material from Introduction to Python
Programming.

\subsection*{Assignment 2}
\label{sub:assignment_2}

This homework assignment is mostly about recursion and features the first implementation exercises. It concerns
\cref{lo:runtime,lo:space,lo:hash} of the course and has a deadline in week 3.

\subsection*{Assignment 3}
\label{sub:assignment_3}

This homework assignment is mostly about sorting, stacks, and queues and concerns \cref{lo:sort,lo:stack,lo:queue} of the
course and has a deadline in week 4.

\subsection*{Assignment 4}
\label{sub:assignment_4}

This homework assignment is mostly about priorityqueues, heaps, and trees and it concerns \cref{lo:heap,lo:tree} of
the course and has a deadline in week 6.

\subsection*{Assignment 5}
\label{sub:assignment_5}

This homework assignment is mostly about search trees, maps, and sets, it concerns \cref{lo:pqueue,lo:map} of the course
and has a deadline in week 8.

\subsection*{Assignment 6}
\label{sub:assignment_6}

This homework assignment is mostly about graphs, it concerns the new \cref{lo:graph,lo:prove} of the course and has a
deadline in week 9.



