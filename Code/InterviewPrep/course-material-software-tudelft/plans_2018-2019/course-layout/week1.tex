\subsection{Week 1}
\label{sub:week_1}

\subsubsection{Lecture 1: Counting operations \& big-Oh analysis}
\label{sub:lecture_1}

This lecture introduces the course material and presents a course overview. 

\hfill\\
\textbf{Learning Objectives:}\\
We start with \cref{lo:runtime}
of the course, more specifically after this lecture the student is able to:
\begin{itemize}
	\item count primitive operations in iterative pieces of code.
	\item derive a total expression in terms of primitives for a piece of code.
	\item describe the notion of big-Oh in complexity theory.
	\item prove a function is of a certain big-Oh complexity.
	\item order a set of functions based on their complexity.
\end{itemize}

\hfill\\
\textbf{Intermezzo:}\\
\improvement{Look this up concretely, I remember something about some significant increase in speed in some ruby method
that the standard library uses a lot}

\hfill\\
\textbf{Lecture plan:}\\
\begin{description}
	\item[5 min] Introduction and what do students expect from the course?
	\item[5 min] Why is ADS cool/important?
	\item[3 min] What will I teach them? (L.O.s in plain language)
	\item[5 min] Short survey to audience: what do they want to learn, what not? Why choose this subject?
	\item[12 min] Course set-up: exams, labs (TAs), tutorials, lectures, weblab, e-mail address, weekly assignments, the
		book.
	\item[15 min] Python recap (interactive), maps, sets, lists, classes.
	\item Break time
	\item[5 min] Intermezzo
	\item[5 min] Counting primitives in for-loop and nested for-loop.
	\item[5 min] Counting in nested with i= 0 to n, j = i to n.
	\item[5 min] Comparing runtimes experimentally (constant, linear, quadratic, exponential, factorial).
	\item[3 min] Formal notation for big-Oh.
	\item[7 min] A big-Oh proof for one of the examples studied earlier.
	\item[1 min] The notion of polynomial run time.
	\item[5 min] Revisit the previous code examples.
	\item[5 min] The notion of tightest bound and comparing functions.
	\item[4 min] Short survey about the lecture: what was unclear, what was good, what could be better?
\end{description}

\subsubsection{Lecture 2: Complexity notation, space complexity \& an introduction to recursion}
\label{sub:lecture_2}

\hfill\\
\textbf{Learning Objectives:}\\
This lecture concerns \cref{lo:notation,lo:runtime,lo:space} of the course, more specifically after this lecture the
student is able to:
\begin{itemize}
	\item describe the notion of $\Theta$, $\Omega$ in complexity theory.
	\item derive the right big-Oh notation for an iterative piece of code.
	\item describe the time and space complexity of some common python functions.
	\item derive the space complexity of algorithms.
	\item derive a recurrence relation for time and space from a recursive piece of code.
	\item prove the run time complexity of a recursive piece of code using repeated unfolding and induction.
\end{itemize}

\hfill\\
\textbf{Intermezzo:}\\
The big-Oh race of matrix multiplication.

\hfill\\
\textbf{Lecture problem:}\\
Searching in a sorted list. (Only second half of the lecture)

\hfill\\
\textbf{Lecture plan:}\\
\begin{description}
	\item[5 min] Notes on the admin related side \& questions from students.
	\item[5 min] Recap of last lecture, reminder of big-Oh and one example of it.
	\item[10 min] Python standards like \texttt{in} for a list, \texttt{range}, and list comprehension.
		\unsure{Maybe other functions/standards are more suitable?}
	\item[5 min] big-$\Theta$ and big-$\Omega$: why care about those?
	\item[3 min] Introduction to space complexity.
	\item[5 min] What happens with memory under the hood of python? Heap vs stack (short and not too technical)
	\item[7 min] Code examples, including: why we ignore the input.
	\item[5 min] Relations between time and space.
	\item Break time
	\item[5 min] Intermezzo
	\item[2 min] Introduction to today's problem.
	\item[3 min] Introduction to recursion.
	\item[8 min] Binary search: how does it work, what is the rec. equation for run time and space?
	\item[7 min] What is such a recurrence equation? Generalise to recursion trees (fibonacci?)
	\item[15 min] Solving a rec. equation. First repeatedly unfold, then prove by induction. Do this for BS.
	\item[5 min] Short survey about the lecture: what was unclear, what was good, what could be better?
\end{description}

\subsubsection{Tutorial 1: Complexity notation, space complexity}
\label{ssub:tutorial_1_complexity_notation_space_complexity}

\hfill\\
\textbf{Plan:}\\
\begin{description}
	\item[10 min] Exercise C1.18 from the book.
	\item[10 min] Exercise C1.15 from the book.
	\item[10 min] Iterative minmax function that returns both.
	\item[15 min] Recursive minmax function that returns both.
	\item Break time
	\item[10 min] Prove something is $\Theta(n^2)$.
	\item[5 min] Prove that $\log n$ is $O(n^c)$ for any $c > 0$.\unsure{Not sure if I want this, might not be needed for
		math students.}
	\item[5 min] Analyse iterative minmax.
	\item[10 min] Analyse recursive minmax.
	\item[10 min] Space complexity of iterative and recursive minmax.
	\item[5 min] Short survey about the tutorial: what was unclear, what was good, what could be better?
\end{description}

\newpage
\subsubsection{Assignment 1: Deadline Sunday before week 2}
\label{ssub:assignment_1}
\unsure{Should be about 7 hours of work. Currently planned: 6 hours. For the first week this may be okay.}

\hfill\\
\textbf{Analysis:}\\
\begin{description}
	\item[10 min] Order sets of functions by big-Oh, determine if functions are $\Theta(f(n))$ or not.
	\item[] Give T(n) and determine big-Oh. \unsure{Look into ways to auto-grade without making it MC.}
		\begin{description}
			\item[5 min] Search in $O(n)$ list.
			\item[5 min] Cartesian product in $O(n^2)$ sets.
			\item[5 min] Matrix multiplication in $O(n^3)$.
			\item[10 min] Recursive fibonacci (impossible to do properly at the level of this course, but we can do it by
				hinting that T(n-2) is O(T(n-1))).
			\item[10 min] Recursive product.
			\item[15 min] Code fragment 4.10 from the book. Answer is $O(n)$ is already given.
		\end{description}
	\item[30 min] 4 rec. equations for which closed form and big-Oh should be determined.
	\item[15 min] 5 MC (exam-level) questions.
	\item[15 min] Big-Oh proof.
\end{description}

\hfill\\
\textbf{Implementation:}\\
\begin{description}
	\item[45 min] Basic OOP-skills in python. \idea{Points and triangles?}
	\item[15 min] IndexOf method for lists.
	\item[20 min] Dict$<$Key, Set$>$ find all keys for which a certain $v$ is in the set of values.
	\item[20 min] Something with exception handling/throwing.
	\item[20 min] Something that requires a tuple of return values.
	\item[20 min] Binary search.
	\item[40 min] Towers of Hanoi (C4.14).
	\item[30 min] Exercise C4.18.
	\item[30 min] Exercise C4.19.
\end{description}

\hfill\\
\textbf{Extra challenges:}\\
\begin{description}
	\item[40 min] FPC: number of zeroes in factorial \unsure{Maybe?}
\end{description}
