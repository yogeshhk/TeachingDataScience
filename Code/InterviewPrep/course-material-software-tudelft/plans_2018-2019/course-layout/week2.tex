\subsection{Week 2}
\label{sub:week_2}

\subsubsection{Lecture 3: Recursion \& the Master Method}
\label{sub:lecture_3}

\hfill\\
\textbf{Learning Objectives:}\\
This lecture concerns \cref{lo:runtime,lo:space} of the course, more specifically after this lecture the student is able
to:
\begin{itemize}
	\item describe the solution to solving the closest pair of points problem.
	\item derive a recurrence relation for time from a recursive piece of code.
	\item prove the run time complexity of a recursive piece of code using the master method.
\end{itemize}

\hfill\\
\textbf{Intermezzo:}\\
\improvement{Maybe something map reduce?}

\hfill\\
\textbf{Lecture problem:}\\
Closest pair of points.

\hfill\\
\textbf{Lecture plan:}\\
\begin{itemize}
	\item[5 min] Notes on the admin related side \& questions from students.
	\item[5 min] Recap of last lecture, reminder of recurrence equations etc.
	\item[35 min] Closest-pair of points problem and solution.
	\item Break time
	\item[5 min] Intermezzo
	\item[10 min] The master method: how does it work?
	\item[15 min] Examples of applying the master method.
	\item[10 min] Back to closest-pair of points.
	\item[5 min] Short survey about the lecture: what was unclear, what was good, what could be better?
\end{itemize}

\newpage
\subsubsection{Lecture 4: Equality, and lists}
\label{sub:lecture_4}

\hfill\\
\textbf{Learning Objectives:}\\

This lecture concerns \cref{lo:hash,lo:list} of the course, more specifically after this lecture the student is able
to:
\begin{itemize}
	\item describe the differences between linked lists and position-based lists.
	\item describe the implementation of linked lists and arraylists.
	\item select the right type of list given a use case.
	\item explain the need for equality function for a custom class.
\end{itemize}

\hfill\\
\textbf{Intermezzo:}\\
\unsure{Something vaguely related to lists?}

\hfill\\
\textbf{Lecture problem:}\\
A queue (linked list) like in the supermarket and maybe goods in the supermarket that are like an array? \unsure{needs
work}

\hfill\\
\textbf{Lecture plan:}\\
\begin{itemize}
	\item[5 min] Notes on the admin related side \& questions from students.
	\item[5 min] Recap of last lecture, master method for recursion.
	\item[5 min] Introduction to today's topic: lists and different use cases.
	\item[15 min] Analysing the python list implementation: array-based. Adding, removing, searching.
	\item[10 min] Amortised run time when growing an array list.
	\item[5 min] Introduction to alternative implemention: linked-list.
	\item Break time
	\item[5 min] Intermezzo
	\item[15 min] An alternative implementation: singly and doubly linked-list. Adding, removing, searching.
	\item[5 min] Space and time considerations for different types of lists.
	\item[5 min] Back to our problem of today and solving it.
	\item[10 min] Putting custom objects into such lists, the need for equality functions.
	\item[5 min] Short survey about the lecture: what was unclear, what was good, what could be better?
\end{itemize}

\newpage
\subsubsection{Tutorial 2: Recursion \& lists}
\label{ssub:tutorial_1_complexity_notation_space_complexity}

\hfill\\
\textbf{Plan:}\\
\begin{itemize}
	\item[15 min] Exercise P4.24 from the book.
	\item[10 min] Exercise C1.15 from the book.
	\item[10 min] Exercise R7.1 from the book.
	\item[10 min] Exercise C7.28 from the book.
	\item Break time
	\item[10 min] Implementing equals functions.
	\item[15 min] Practice with the master method (3 exercises).
	\item[10 min] Analyse algorithm that uses linked-lists.
	\item[5 min] Given a use case, decide between LL and Array-based.
	\item[5 min] Short survey about the tutorial: what was unclear, what was good, what could be better?
\end{itemize}


\newpage
\subsubsection{Assignment 2: Deadline Sunday before week 3}
\label{ssub:assignment_2}

\hfill\\
\textbf{Analysis:}\\
\begin{description}
	\item[45 min] 5 rec. equations to solve using the master method.
	\item[15 min] 2 rec. equations to solve using repeated unfolding.
	\item[10 min] 1 rec. equations to solve using repeated unfolding + induction.
	\item[] Give T(n) and determine big-Oh. \unsure{Look into ways to auto-grade without making it MC.}
		\begin{description}
			\item[10 min] 2 simple recursions.
			\item[10 min] More complicated recursion.
			\item[15 min] Something with array-based lists.
			\item[15 min] Same thing, but with with linked-lists.
		\end{description}
	\item[15 min] 5 MC (exam-level) questions.
	\item[15 min] Big-Omega proof.
\end{description}

\hfill\\
\textbf{Implementation:}\\
\begin{description}
	\item[60 min] Closest-pair of points.
	\item[45 min] Given implementation of singly-linked list, make it doubly-linked.
	\item[90 min] Circularly-linked list (not covered in lecture, so needs to be built up well!)
	\item[30 min] \hfill\\\unsure{Some problem that needs solving using only a list (maybe something with a simple unsorted queue) that
		favours a linked-list over an array-based one.}
	\item[45 min] \hfill\\\unsure{Maybe something like the stuff explained in 7.6?}
\end{description}
