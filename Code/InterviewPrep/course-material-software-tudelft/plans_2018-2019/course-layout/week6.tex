\subsection{Week 6}
\label{sub:week_6}

\subsubsection{Lecture 9: Hashsets and maps}
\label{sub:lecture_9}

\hfill\\
\textbf{Learning Objectives:}\\
This lecture concerns \cref{lo:map,lo:set} of the course, more specifically after this lecture the student is able
to:
\begin{itemize}
	\item list the requirements a hash functions should fulfill.
	\item implement an equality and hash function for a custom python class.
	\item describe the operations of the \texttt{dict} datastructure in python.
	\item describe the interface of a map.
	\item describe the differences between a hashset and a hashmap.
	\item describe how hash conflicts are handled by these datastructures.
\end{itemize}

\hfill\\
\textbf{Intermezzo:}\\
The importance of good hash funtions, link to assignment

\hfill\\
\textbf{Lecture problem:}\\
Zesje: matching students to submissions.

\hfill\\
\textbf{Lecture plan:}\\
\begin{itemize}
	\item[5 min] Notes on the admin related side \& questions from students.
	\item[5 min] Recap of last lecture, sorted trees.
	\item[5 min] Introduce problem of matching objects to objects. We all know about dictionaries already, but...
	\item[5 min] Properties of hash functions (one or two examples)
	\item[5 min] Implementing a hash function in python.
	\item[20 min] What does a dict do in python: hash conflicts, internal data structures etc.
	\item Break time
	\item[10 min] Intermezzo
	\item[5 min] Sets in python: maps without keys.
	\item[5 min] Solving the problem of zesje.
	\item[20 min] Sorted maps and multimaps
	\item[5 min] Short survey about the lecture: what was unclear, what was good, what could be better?
\end{itemize}

\newpage
\subsubsection{Lecture 10: Introduction to graphs}
\label{sub:lecture_8}

\hfill\\
\textbf{Learning Objectives:}\\
\begin{itemize}
	\item define basic graph components (vertices, edges, directedness).
	\item prove certain properties about graphs.
	\item describe the operation of DFS and BFS.
	\item select the right traversal algorithm for a given use case.
	\item describe how DFS and BFS can be used to detect cycles in a graph.
\end{itemize}

\hfill\\
\textbf{Intermezzo:}\\
Graph databases

\hfill\\
\textbf{Lecture problem:}\\
The Dutch rail network: finding a route from Haarlem to Delft.

\hfill\\
\textbf{Lecture plan:}\\
\begin{itemize}
	\item[5 min] Notes on the admin related side \& questions from students.
	\item[5 min] Recap of trees and maps.
	\item[5 min] Introduce the NS problem.
	\item[15 min] Different methods of storing graphs: pros and cons.
	\item[15 min] DFS (recursion), DFS (stack), BFS (queue).
	\item Break time
	\item[5 min] Intermezzo
	\item[5 min] Solving the NS problem.
	\item[5 min] Transitive closure.
	\item[15 min] \unsure{DAG: properties (prove some, iirc there are bounds on number of edges).}
	\item[10 min] Topological ordering
	\item[5 min] Short survey about the lecture: what was unclear, what was good, what could be better?
\end{itemize}

\newpage
\subsubsection{Tutorial 5: Maps and graphs}
\label{ssub:tutorial_5}

\hfill\\
\textbf{Plan:}\\
\begin{itemize}
	\item[10 min] Implement hash functions for some custom classes.
	\item[15 min] Implement graph structure using Adjacency Map.
	\item[20 min] Implement BFS.
	\item Break time
	\item[5 min] Exercise R10.4.
	\item[5 min] Exercise R10.6.
	\item[10 min] Exercise R10.17.
	\item[5 min] Exercise R14.11.
	\item[10 min] Exercise R14.16.
	\item[5 min] Exercise R14.14.
	\item[5 min] Short survey about the tutorial: what was unclear, what was good, what could be better?
\end{itemize}


\newpage
\subsubsection{Assignment 5: Deadline Sunday before week 7}
\label{ssub:assignment_5}

\hfill\\
\textbf{Analysis:}\\
\begin{description}
	\item[] Give T(n) and determine big-Oh. \unsure{Look into ways to auto-grade without making it MC.}
		\begin{description}
			\item[5 min] 1 simple recursions.
			\item[10 min] More complicated recursion.
			\item[15 min] Tree algorithm.
			\item[10 min] Something that uses a map.
			\item[45 min] Graph algorithms. \idea{Maybe number of connected components?}
		\end{description}
	\item[15 min] 5 MC (exam-level) questions.
	\item[10 min] 2 use cases with the question: what data structure would you use and why?
	\item[5 min] 1 rec. equations to solve using the master method.
	\item[90 min] Implement different hash functions and do an experimental evaluation of runtimes (one with many
		collisions, one with some, one with few. We should provide the hash functions in English).
\end{description}

\hfill\\
\textbf{Implementation:}\\
\begin{description}
	\item[15 min] Implement some basic hash functions for custom classes.
	\item[30 min] Implement a Graph structure using an adjacency matrix.
	\item[30 min] Implement DFS.
	\item[30 min] Implement Topological Ordering.
	\item[30 min] \hfill\\\unsure{Fun graph problem that can be solved using BFS/DFS.}
	\item[30 min] Implement a tree algorithm.
\end{description}
