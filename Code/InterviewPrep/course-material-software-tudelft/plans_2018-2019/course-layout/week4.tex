\subsection{Week 4}
\label{sub:week_4}

\subsubsection{Lecture 7: Trees}
\label{sub:lecture_7}

\hfill\\
\textbf{Learning Objectives:}\\

This lecture concerns \cref{lo:tree} of the course, more specifically after this lecture the student is able
to:
\begin{itemize}
	\item describe the interface of a tree.
	\item construct basic tree-traversal algorithms (like post-order).
	\item analyse the run time complexity of basic tree-traversal algorithms.
\end{itemize}

\hfill\\
\textbf{Intermezzo:}\\
\unsure{Something fun with trees}

\hfill\\
\textbf{Lecture problem:}\\
Parsing mathematical expressions.

\hfill\\
\textbf{Lecture plan:}\\
\begin{itemize}
	\item[5 min] Notes on the admin related side \& questions from students.
	\item[5 min] Recap of last lecture, reminder of different types sorting.
	\item[5 min] Introduce problem of parsing math, relate back to recursion trees.
	\item[5 min] Tree properties: parents, children, ancestors, descendants, root.
	\item[5 min] Tree implementation: linked structure.
	\item[10 min] Tree ADT: methods like: size, leafs, height.
	\item[10 min] A binary tree.
	\item Break time
	\item[5 min] Intermezzo
	\item[10 min] Alternative implementation: array-based.
	\item[10 min] Tree-traversals (one implemented, the others described only).
	\item[5 min] Solving the parsing problem (what traversal order do we need?)
	\item[10 min] Introduction to heap-structure: binary tree with special properties.
	\item[5 min] Short survey about the lecture: what was unclear, what was good, what could be better?
\end{itemize}

\newpage
\subsubsection{Lecture 8: Heaps \& Searchtrees}
\label{sub:lecture_8}

\hfill\\
\textbf{Learning Objectives:}\\

This lecture concerns \cref{lo:tree,lo:heap} of the course, more specifically after this lecture the student is able
to:
\begin{itemize}
	\item describe the properties of a balanced tree.
	\item construct a binary search tree by repeatedly applying the insert operation.
	\item analyse the run time complexity of binary search tree algorithms.
\end{itemize}

\hfill\\
\textbf{Intermezzo:}\\
Databases and their use of B-trees.

\hfill\\
\textbf{Lecture problem:}\\
\improvement{Fun story that requires a priority queue.}

\hfill\\
\textbf{Lecture plan:}\\
\begin{itemize}
	\item[5 min] Notes on the admin related side \& questions from students.
	\item[5 min] Recap of trees and heaps.
	\item[5 min] Introduce the PQ problem.
	\item[15 min] Up and down bubbling.
	\item[10 min] Analysing the priorityqueue we just created.
	\item[5 min] Solve our PQ problem.
	\item Break time
	\item[5 min] Intermezzo
	\item[5 min] Problem with sorted tree.
	\item[15 min] Introduce notion of binary search tree: insertions/deletions.
	\item[15 min] Extend to Multiway search tree.
	\item[5 min] Short survey about the lecture: what was unclear, what was good, what could be better?
\end{itemize}

\newpage
\subsubsection{Tutorial 4: Trees}
\label{ssub:tutorial_4}

\hfill\\
\textbf{Plan:}\\
\begin{itemize}
	\item[10 min] Implement something like: countLeaves() for a tree.
	\item[10 min] Implement something like: depth() for a tree.
	\item[15 min] Implement searching for an item in a binary search tree.
	\item[10 min] Implement searching for an item in a multiway search tree.
	\item Break time
	\item[5 min] Exercise R8.16.
	\item[10 min] Exercise R8.34.
	\item[10 min] Exercise C8.43.
	\item[5 min] Questions about insertion/deletion from heaps.
	\item[5 min] Exercise R11.16.
	\item[5 min] Exercise R11.21a.
	\item[5 min] Short survey about the tutorial: what was unclear, what was good, what could be better?
\end{itemize}


\newpage
\subsubsection{Assignment 4: Deadline Sunday before week 5}
\label{ssub:assignment_4}

\hfill\\
\textbf{Analysis:}\\
\begin{description}
	\item[] Give T(n) and determine big-Oh. \unsure{Look into ways to auto-grade without making it MC.}
		\begin{description}
			\item[5 min] 1 simple recursions.
			\item[10 min] More complicated recursion.
			\item[10 min] Sorting algorithm D.
			\item[45 min] Tree algorithms!
		\end{description}
	\item[15 min] 5 MC (exam-level) questions.
	\item[10 min] 2 use cases with the question: what data structure would you use and why?
	\item[10 min] 2 rec. equations to solve using the master method.
	\item[10 min] 1 rec. equations to solve using repeated unfolding + induction.
	\item[10 min] Big-Omega proof.
\end{description}

\hfill\\
\textbf{Implementation:}\\
\begin{description}
	\item[45 min] Tree traversals.
	\item[45 min] Implement some heap bubbling.
	\item[90 min] Implement a (2,4)-tree.
	\item[30 min] Extend it to a (3,5)-tree.
	\item[30 min] \hfill\\\unsure{Some problem that needs solving using a PQ, maybe a nice greedy problem? We can give
		most of the idea away, students should just implement it.}
	\item[30 min] Implement selection-sort.
	\item[30 min] Implement radix-sort.
\end{description}
