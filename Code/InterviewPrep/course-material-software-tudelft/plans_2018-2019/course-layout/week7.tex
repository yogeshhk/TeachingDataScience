\subsection{Week 7}
\label{sub:week_7}

\subsubsection{Lecture 11: Hashsets and maps}
\label{sub:lecture_11}

\hfill\\
\textbf{Learning Objectives:}\\
This lecture concerns \cref{lo:graph} of the course, more specifically after this lecture the student is able
to:
\begin{itemize}
	\item apply and analyse Dijkstra's algorithm.
	\item prove the correctness of Dijkstra's algorithm.
	\item apply and anlyse the A* algorithm.
\end{itemize}

\hfill\\
\textbf{Intermezzo:}\\
Routing in your phone.

\hfill\\
\textbf{Lecture problem:}\\
NS part 2: shortest path from Haarlem to Delft.

\hfill\\
\textbf{Lecture plan:}\\
\begin{itemize}
	\item[5 min] Notes on the admin related side \& questions from students.
	\item[5 min] Recap of last lecture, graphs.
	\item[5 min] Introduce problem of NS.
	\item[15 min] Dijkstra's algorithm: what does it do?
	\item[10 min] Proof by induction: correctness
	\item[5 min] Runtime analysis.
	\item Break time
	\item[5 min] Intermezzo
	\item[10 min] Improving the algorithm: PQ instead of O(n) lookup.
	\item[15 min] Improving(??) the algorithm: adding a heuristic (A*).
	\item[10 min] Discuss differences between them, harder to give bounds, pros/cons.
	\item[5 min] Short survey about the lecture: what was unclear, what was good, what could be better?
\end{itemize}

\newpage
\subsubsection{Lecture 12: MSTs}
\label{sub:lecture_12}

\hfill\\
\textbf{Learning Objectives:}\\
This lecture concerns \cref{lo:graph} of the course, more specifically after this lecture the student is able
to:
\begin{itemize}
	\item describe the properties of a Minimum Spanning Tree (MST).
	\item apply Prims algorithm.
	\item describe and implement the Union-Find datastructure.
	\item apply Kruskals algorithm.
\end{itemize}

\hfill\\
\textbf{Intermezzo:}\\
\unsure{No clue yet}

\hfill\\
\textbf{Lecture problem:}\\
Routing, minimal connections to connect everyone.

\hfill\\
\textbf{Lecture plan:}\\
\begin{itemize}
	\item[5 min] Notes on the admin related side \& questions from students.
	\item[5 min] Recap of shortest path (dijkstra)
	\item[5 min] Introduce the routing problem.
	\item[5 min] Introduce the notion of an MST, what is it?
	\item[10 min] Prove cycle/cut properties.
	\item[15 min] Prim's algorithm: based on Dijkstra. 
	\item Break time
	\item[5 min] Intermezzo
	\item[5 min] Solving the routing problem.
	\item[10 min] Introduce alternative method: Kruskal.
	\item[5 min] But we need Union-Find to compete with Prim.
	\item[15 min] Explain union-find, wrap up Kruskal.
	\item[5 min] Short survey about the lecture: what was unclear, what was good, what could be better?
\end{itemize}

\newpage
\subsubsection{Tutorial 6: Shortest path and MST}
\label{ssub:tutorial_6}

\hfill\\
\textbf{Plan:}\\
\begin{itemize}
	\item[25 min] Implement Union-Find.
	\item[20 min] Implement some heuristics for A* and compare how they work out. (A* impl. should be given)
	\item Break time
	\item[15 min] Apply Dijkstra and A* to some graph.
	\item[5 min] Given use-case come up with some heuristics for A*.
	\item[5 min] Find the MST of some graph.
	\item[5 min] Apply Union-Find given some order of operations.
	\item[10 min] New use-case in which UF is usefull.
	\item[5 min] Short survey about the tutorial: what was unclear, what was good, what could be better?
\end{itemize}


\newpage
\subsubsection{Assignment 6: Deadline Sunday before week 8}
\label{ssub:assignment_6}

\hfill\\
\textbf{Analysis:}\\
\begin{description}
	\item[] Give T(n) and determine big-Oh. \unsure{Look into ways to auto-grade without making it MC.}
		\begin{description}
			\item[5 min] 1 simple recursions.
			\item[10 min] More complicated recursion.
			\item[15 min] Tree algorithm.
			\item[10 min] Something that uses a set.
			\item[20 min] Graph traversal algorithm.
			\item[25 min] Some fancy/weird implemnetation of an MST/Shortest Path algorithm. Maybe All-pair-shortest-path?
		\end{description}
	\item[15 min] 5 MC (exam-level) questions.
	\item[10 min] 2 use cases with the question: what data structure would you use and why?
	\item[5 min] 1 rec. equations to solve using the master method.
	\item[90 min] Implement different heuristics for A* functions and do an experimental evaluation of runtimes (We should
		provide the heuristics in English).
\end{description}

\hfill\\
\textbf{Implementation:}\\
\begin{description}
	\item[60 min] Implement Dijsktra.
	\item[60 min] Implement Prim.
	\item[60 min] Implement Kruskal.
	\item[45 min] \hfill\\\unsure{Fun graph problem that can be solved using Dijkstra, but with a twist. Like AD has}
	\item[30 min] Implement some graph manipulation algorithm (adding/removing edges).
\end{description}
