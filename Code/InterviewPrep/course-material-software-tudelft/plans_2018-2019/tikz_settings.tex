\usepackage{tikz}
\usepackage{xstring}
\usepackage{pgfplots}
\usepackage{etoolbox}
\usepackage{amssymb}
\usetikzlibrary{arrows,shapes}
\usetikzlibrary{positioning, calc}

\tikzstyle{skill}=[draw=black,minimum size=25pt,inner sep=5pt, rectangle split, anchor = north]
\tikzstyle{skill_basic}=[rectangle split parts= 2, rectangle split part fill = {blue!30,
blue!20}, rectangle split part align = {center, left}, skill]
\tikzstyle{skill_test}=[skill, rectangle split parts=3, rectangle split part fill = {blue!30, blue!20,red!20}]
\tikzstyle{skill_material}=[skill, rectangle split parts=4, rectangle split part fill = {blue!30,
blue!20,yellow!20,red!20}]
\tikzstyle{skill_unlock}=[skill, rectangle split parts=4, rectangle split part fill = {blue!30, blue!20,red!20,green!30}]

\tikzstyle{optional}=[fill=gray!30]

\tikzstyle{glue_edge}=[ultra thick]
\tikzstyle{edge}=[glue_edge, ->]

\newcommand{\drawskill}[4][]{
	\node[#1, skil_basic] (#2) {#3 \nodepart{second} #4};
}

\newcommand{\drawskilltest}[5][]{
	\node[#1, skill_test] (#2) {#3 \nodepart{second} $\diamond$ #4 \nodepart{third} $\circ$ #5};
}

\newcommand{\drawskillmaterialtest}[6][]{
	\node[#1, skill_material] (#2) {#3 \nodepart{second} $\diamond$ #4 \nodepart{third} $\to$ #5 \nodepart{fourth}
	$\circ$ #6};
}


