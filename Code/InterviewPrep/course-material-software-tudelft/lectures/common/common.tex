\usepackage{ifthen}
\usepackage{graphicx}
\usepackage{booktabs}
\usepackage{tabularx}
\usepackage{algorithm}
%\usepackage{algorithmx}
%\usepackage{cwpuzzle}
\usepackage{algpseudocode}
\usepackage[english]{babel}
%\usepackage{qtree}
\usepackage{amssymb}
\usepackage{amsmath,amsthm}
\usepackage{subcaption}
\usepackage{mathrsfs}
%\usepackage{eurosym}
\usepackage{soul}
\usepackage{fontawesome}
\usepackage{etoolbox}
\usepackage{soul}
\usepackage{multicol}
\usepackage{listings}
\usepackage{color}

\definecolor{mygreen}{rgb}{0,0.6,0}
\definecolor{mygray}{rgb}{0.5,0.5,0.5}
\definecolor{mymauve}{rgb}{0.58,0,0.82}
\usepackage{tikz}
\usepackage{xstring}
\usepackage{pgfplots}
\usepackage{etoolbox}
\usepackage{ifthen}
\usetikzlibrary{arrows,shapes}
\usetikzlibrary{positioning, calc}

\definecolor{hospital}{RGB}{0,0,255}
\definecolor{patient}{RGB}{255,0,0}

\tikzstyle{flow_node}=[circle,draw=black,minimum size=15pt,inner sep=0pt]
\tikzstyle{flow_s}=[fill=green!20]
\tikzstyle{flow_subtask}=[fill=green!80]
\tikzstyle{flow_task}=[fill=red!80]
\tikzstyle{flow_t}=[fill=blue!20]
\tikzstyle{flow_edge}=[->, ultra thick]
\tikzstyle{flow_capacity}=[fill=white, inner sep = 1pt]
\tikzstyle{flow_edge_mincut}=[draw=red]
\tikzstyle{flow_edge_fullcap}=[draw=red]
\tikzstyle{flow_mincut}=[dashed, thick]
\tikzstyle{flow_supply}=[fill=green!80]
\tikzstyle{flow_demand}=[fill=red!80]

\newcommand\ifMaxFlow[4]{
	\begingroup
		\pgfmathsetmacro{\flow}{#1}
		\pgfmathsetmacro{\cap}{#2}
		\pgfmathparse{ifthenelse(\flow == \cap,1,0)}
		\ifdim\pgfmathresult pt= 1 pt
			 #3
			\else
			 #4
		\fi
	\endgroup
}

\newcommand\ifSomeFlow[3]{
	\begingroup
		\pgfmathsetmacro{\flow}{#1}
		\pgfmathparse{ifthenelse(\flow > 0,1,0)}
		\ifdim\pgfmathresult pt= 1 pt
			 #2
			\else
			 #3
		\fi
	\endgroup
}


\usepackage[duration=105,lastminutes=15]{../common/pdfpcnotes}

\makeatletter
\let\@@magyar@captionfix\relax
\makeatother

\newcommand{\bigO}{O}

\usetheme[width=0mm]{PaloAlto}
\setbeamertemplate{navigation symbols}
{\ifthenelse{\not\equal{\thepage}{1}}
  {\insertframenumber}
  {}
}
\setbeamertemplate{footline}%
{\ifthenelse{\not\equal{\thepage}{1}}%
  {\color{gray!30!white}{\tiny \copyright 2018 TU Delft}}% \hfill\insertframenumber/\inserttotalframenumber}
  {}
}

\newenvironment<>{problemblock}[1]{%
  \begin{actionenv}#2%
      \def\insertblocktitle{Problem: #1}%
      \par%
      \mode<presentation>{%
        \setbeamercolor{block title}{fg=white,bg=orange!80!black}
       \setbeamercolor{block body}{fg=black,bg=orange!40}
       \setbeamercolor{itemize item}{fg=orange!20!black}
       \setbeamertemplate{itemize item}[triangle]
     }%
      \usebeamertemplate{block begin}}
    {\par\usebeamertemplate{block end}\end{actionenv}}

\newenvironment<>{questionblock}[1]{%
  \begin{actionenv}#2%
      \def\insertblocktitle{Question: #1}%
      \par%
      \mode<presentation>{%
        \setbeamercolor{block title}{fg=white,bg=cyan!80!black}
       \setbeamercolor{block body}{fg=black,bg=cyan!30}
       \setbeamercolor{itemize item}{fg=cyan!20!black}
       \setbeamertemplate{itemize item}[triangle]
     }%
      \usebeamertemplate{block begin}}
    {\par\usebeamertemplate{block end}\end{actionenv}}

\newenvironment<>{answerblock}[1]{%
  \begin{actionenv}#2%
      \def\insertblocktitle{Answer: #1}%
      \par%
      \mode<presentation>{%
        \setbeamercolor{block title}{fg=white,bg=green!80!black}
       \setbeamercolor{block body}{fg=black,bg=green!30}
       \setbeamercolor{itemize item}{fg=green!20!black}
       \setbeamertemplate{itemize item}[triangle]
     }%
      \usebeamertemplate{block begin}}
    {\par\usebeamertemplate{block end}\end{actionenv}}

% Fix a bug in lstlinebgrd
% https://tex.stackexchange.com/questions/451532/recent-issues-with-lstlinebgrd-package-with-listings-after-the-latters-updates

\makeatletter
\let\old@lstKV@SwitchCases\lstKV@SwitchCases
\def\lstKV@SwitchCases#1#2#3{}
\makeatother
\usepackage{lstlinebgrd}
\makeatletter
\let\lstKV@SwitchCases\old@lstKV@SwitchCases

\lst@Key{numbers}{none}{%
    \def\lst@PlaceNumber{\lst@linebgrd}%
    \lstKV@SwitchCases{#1}%
    {none:\\%
     left:\def\lst@PlaceNumber{\llap{\normalfont
                \lst@numberstyle{\thelstnumber}\kern\lst@numbersep}\lst@linebgrd}\\%
     right:\def\lst@PlaceNumber{\rlap{\normalfont
                \kern\linewidth \kern\lst@numbersep
                \lst@numberstyle{\thelstnumber}}\lst@linebgrd}%
    }{\PackageError{Listings}{Numbers #1 unknown}\@ehc}}
\makeatother

\lstset{ 
  backgroundcolor=\color{white},   % choose the background color; you must add \usepackage{color} or \usepackage{xcolor}; should come as last argument
  basicstyle=\footnotesize\ttfamily,        % the size of the fonts that are used for the code
  breakatwhitespace=false,         % sets if automatic breaks should only happen at whitespace
  breaklines=true,                 % sets automatic line breaking
  captionpos=b,                    % sets the caption-position to bottom
  commentstyle=\color{mygreen},    % comment style
  deletekeywords={...},            % if you want to delete keywords from the given language
	escapeinside={\%*}{*\%},          % if you want to add LaTeX within your code
  extendedchars=true,              % lets you use non-ASCII characters; for 8-bits encodings only, does not work with UTF-8
  frame=single,	                   % adds a frame around the code
  keepspaces=true,                 % keeps spaces in text, useful for keeping indentation of code (possibly needs columns=flexible)
  keywordstyle=\color{blue},       % keyword style
  language=Python,                 % the language of the code
	literate={\ \ }{{\ }}1,					 % Replace two spaces with just one
  morekeywords={*,...},            % if you want to add more keywords to the set
  numbers=left,                    % where to put the line-numbers; possible values are (none, left, right)
  numbersep=5pt,                   % how far the line-numbers are from the code
  numberstyle=\tiny\color{mygray}, % the style that is used for the line-numbers
  rulecolor=\color{black},         % if not set, the frame-color may be changed on line-breaks within not-black text (e.g. comments (green here))
  showspaces=false,                % show spaces everywhere adding particular underscores; it overrides 'showstringspaces'
  showstringspaces=false,          % underline spaces within strings only
  showtabs=false,                  % show tabs within strings adding particular underscores
  stepnumber=1,                    % the step between two line-numbers. If it's 1, each line will be numbered
  stringstyle=\color{mymauve},     % string literal style
  tabsize=2,   	                   % sets default tabsize to 2 spaces
	emptylines=*1
}
\lstset{
	defaultdialect=[LaTeX]TeX
}

%%%%%%%%%%%%%%%%%%%%%%%%%%%%%%%%%%%%%%%%%%%%%%%%%%%%%%%%%%%%%%%%%%%%%%%%%%%%%%%%
% Hacks for highlighting lines in listings, thanks to Robbert Krebbers
%%%%%%%%%%%%%%%%%%%%%%%%%%%%%%%%%%%%%%%%%%%%%%%%%%%%%%%%%%%%%%%%%%%%%%%%%%%%%%%%

\makeatletter
\newlength\beamerleftmargin
\setlength\beamerleftmargin{\Gm@lmargin}
\makeatother

\makeatletter
\newcount\bt@rangea
\newcount\bt@rangeb

\newcommand\btIfInRange[2]{%
    \global\let\bt@inrange\@secondoftwo%
    \edef\bt@rangelist{#2}%
    \foreach \range in \bt@rangelist {%
        \afterassignment\bt@getrangeb%
        \bt@rangea=0\range\relax%
        \pgfmathtruncatemacro\result{ ( #1 >= \bt@rangea) && (#1 <= \bt@rangeb) }%
        \ifnum\result=1\relax%
            \breakforeach%
            \global\let\bt@inrange\@firstoftwo%
        \fi%
    }%
    \bt@inrange%
}
\newcommand\bt@getrangeb{%
    \@ifnextchar\relax%
        {\bt@rangeb=\bt@rangea}%
        {\@getrangeb}%
}
\def\@getrangeb-#1\relax{%
    \ifx\relax#1\relax%
        \bt@rangeb=100000%   \maxdimen is too large for pgfmath
    \else%
        \bt@rangeb=#1\relax%
    \fi%
}

\newcommand<>{\btLstHL}[1]{%
  \only#2{\btIfInRange{\value{lstnumber}}{#1}%
    {\color{blue!30}}%
    {\def\lst@linebgrdcmd####1####2####3{}}% define as no-op
  }%
}%
\makeatother

