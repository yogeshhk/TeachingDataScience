\section{Intermezzo}
\label{sec:intermezzo}

\begin{frame}
	\frametitle{Intermezzo: On the topic of hashing}
	\framesubtitle{\url{https://softwareengineering.stackexchange.com/a/145633/331066}}

	\begin{itemize}
		\item There are many standard hashing algorithms.
					\pause
		\item All with their own methods.
					\pause
		\item A user on StackExchange did a brilliant comparison:
			\begin{itemize}
				\item There are some that perform great on strings, but not on numbers.
				\item And also the reverse!
					\pause
				\item Some are faster with more collisions, others slower but with fewer collisions.
			\end{itemize}
	\end{itemize}
\end{frame}

\begin{frame}
	\frametitle{Zipcodes and hashing}
	\framesubtitle{\url{https://blogs.msdn.microsoft.com/ericlippert/2003/09/19}}
	\begin{itemize}
		\item A hash function should spread a set of similar keys uniformly over the range.
			\pause
		\item A developer creating the C\# Dictionary class, made a small mistake in the hash function used.
			\pause
		\item The result? \url{msn.com} got very slow!
			\pause
		\item For some reason, they had a dictionary to store every zip code in the US. Which resulted in many collisions
			due to the faulty hash function.
	\end{itemize}
\end{frame}

\begin{frame}
	\frametitle{My own little experiment}
	\framesubtitle{See GitLab for the full code}
	\begin{columns}
		\column{0.455\textwidth}
			
	\lstinputlisting[firstline=42, lastline=54]{code/hashexp.py}
	\pause
		\column{0.455\textwidth}
	\lstinputlisting[firstline=57, lastline=69]{code/hashexp.py}
			
	\pause
	\end{columns}
	\pause
	\lstinputlisting[firstline=72]{code/hashexp.py}
\end{frame}


