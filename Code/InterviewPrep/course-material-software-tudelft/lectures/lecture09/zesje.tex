\section{Use case for dicts}%
\label{sec:use_case_for_dicts}

\begin{frame}
	\frametitle{Zesje}
	\framesubtitle{\url{https://gitlab.kwant-project.org/zesje/zesje}}
\begin{center}
	\includegraphics[width=\textwidth]{figures/zesje.png}\\
	\hspace*{15pt}\hbox{\scriptsize Screenshot by: \thinspace{\itshape Stefan Hugtenburg}}\\
	\hspace*{15pt}\hbox{\scriptsize Software by: \thinspace{\itshape Anton Akhmerov et al.}}
\end{center}	
\end{frame}

\begin{frame}
	\frametitle{Matching students to exams}

	\begin{problemblock}{Grading exams}
		We have $n$ exams belonging to students. We want to efficiently access the exam of a student based on his/her
		student number.
	\end{problemblock}

	\pause
		\begin{alertblock}{Array-Based List}
			Array-based lists allow for $\Theta(1)$ look-ups of data if you know their index.\\
			\pause
			But in this case that would require an array of size approximately 50.000, since student numbers are pretty big.\\
			We could translate them to a lower range, but still there would a lot of wasted space!
		\end{alertblock}	
	
		\pause
			\begin{exampleblock}{A dictionary}
				Of course we should use a \texttt{dict} in Python. \\
				\pause
				But how do they work? Why are they better?
			\end{exampleblock}	
\end{frame}

