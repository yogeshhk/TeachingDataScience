\section{Use case: Stacks}
\label{sec:use_case_stacks}


\begin{frame}
	\frametitle{Use case: Parsing \LaTeX}
	\begin{columns}
		\column{0.655\textwidth}
		\lstinputlisting[language=TeX]{parseslide.tex}
		\pause
		\column{0.355\textwidth}
		\begin{questionblock}{Testing syntax}
			How can I test this slide will compile?
		\end{questionblock}
		\pause
		\begin{answerblock}{Stacks!}
			By using a Stack!
		\end{answerblock}
	\end{columns}
\end{frame}


\begin{frame}
	\frametitle{An example}
	\begin{columns}
		\column{0.655\textwidth}
		\lstinputlisting[language=TeX,
		 linebackgroundcolor={%
    \btLstHL<1>{1}% on slide 1, highlight lines 1-3
    \btLstHL<2>{3}% on slide 2, highlight lines 6 and 9
    \btLstHL<3>{8}%
    \btLstHL<4>{10}%
    \btLstHL<5>{12}%
    \btLstHL<6>{14}%
    \btLstHL<7>{15}%
    \btLstHL<8>{16}%
  }
		]{parseslide.tex}
		\column{0.355\textwidth}
		\section{Use case: Stacks}
\label{sec:use_case_stacks}


\begin{frame}
	\frametitle{Use case: Parsing \LaTeX}
	\begin{columns}
		\column{0.655\textwidth}
		\lstinputlisting[language=TeX]{parseslide.tex}
		\pause
		\column{0.355\textwidth}
		\begin{questionblock}{Testing syntax}
			How can I test this slide will compile?
		\end{questionblock}
		\pause
		\begin{answerblock}{Stacks!}
			By using a Stack!
		\end{answerblock}
	\end{columns}
\end{frame}


\begin{frame}
	\frametitle{An example}
	\begin{columns}
		\column{0.655\textwidth}
		\lstinputlisting[language=TeX,
		 linebackgroundcolor={%
    \btLstHL<1>{1}% on slide 1, highlight lines 1-3
    \btLstHL<2>{3}% on slide 2, highlight lines 6 and 9
    \btLstHL<3>{8}%
    \btLstHL<4>{10}%
    \btLstHL<5>{12}%
    \btLstHL<6>{14}%
    \btLstHL<7>{15}%
    \btLstHL<8>{16}%
  }
		]{parseslide.tex}
		\column{0.355\textwidth}
		\section{Use case: Stacks}
\label{sec:use_case_stacks}


\begin{frame}
	\frametitle{Use case: Parsing \LaTeX}
	\begin{columns}
		\column{0.655\textwidth}
		\lstinputlisting[language=TeX]{parseslide.tex}
		\pause
		\column{0.355\textwidth}
		\begin{questionblock}{Testing syntax}
			How can I test this slide will compile?
		\end{questionblock}
		\pause
		\begin{answerblock}{Stacks!}
			By using a Stack!
		\end{answerblock}
	\end{columns}
\end{frame}


\begin{frame}
	\frametitle{An example}
	\begin{columns}
		\column{0.655\textwidth}
		\lstinputlisting[language=TeX,
		 linebackgroundcolor={%
    \btLstHL<1>{1}% on slide 1, highlight lines 1-3
    \btLstHL<2>{3}% on slide 2, highlight lines 6 and 9
    \btLstHL<3>{8}%
    \btLstHL<4>{10}%
    \btLstHL<5>{12}%
    \btLstHL<6>{14}%
    \btLstHL<7>{15}%
    \btLstHL<8>{16}%
  }
		]{parseslide.tex}
		\column{0.355\textwidth}
		\section{Use case: Stacks}
\label{sec:use_case_stacks}


\input{parseslide.tex}

\begin{frame}
	\frametitle{An example}
	\begin{columns}
		\column{0.655\textwidth}
		\lstinputlisting[language=TeX,
		 linebackgroundcolor={%
    \btLstHL<1>{1}% on slide 1, highlight lines 1-3
    \btLstHL<2>{3}% on slide 2, highlight lines 6 and 9
    \btLstHL<3>{8}%
    \btLstHL<4>{10}%
    \btLstHL<5>{12}%
    \btLstHL<6>{14}%
    \btLstHL<7>{15}%
    \btLstHL<8>{16}%
  }
		]{parseslide.tex}
		\column{0.355\textwidth}
		\input{figures/tikz/parsingtex.tex}\\
		\only<-8>{
		The stack of open \textit{tags}.\\
	}
		\only<9->{
			The code is done and my stack is empty $\to$ Correct TeX.
		}
	\end{columns}
	
\end{frame}

\begin{frame}
	\frametitle{Incorrect Tex}
	\begin{columns}
		\column{0.655\textwidth}
		\lstinputlisting[language=TeX,
		 linebackgroundcolor={%
    \btLstHL<1>{1}% on slide 1, highlight lines 1-3
    \btLstHL<2>{3}% on slide 2, highlight lines 6 and 9
    \btLstHL<3>{5}%
  }
		]{parseerror.tex}
		\column{0.355\textwidth}
		\input{figures/tikz/parsingerror.tex}\\
		The stack of open \textit{tags}.\\
		\only<3->{
			We encounter the wrong closing tag $\to$ Incorrect TeX.
		}
	\end{columns}
\end{frame}

\begin{frame}
	\frametitle{Our TeX parsing algorithm}
	\begin{columns}
		\column{0.655\textwidth}
	\begin{algorithmic}
		\State tagStack $\gets$ empty stack.
		\While{there is TeX}
		\pause
			\If{opening tag}
				\State tagStack.push(tagname)
			\Else
		\pause
			\If{closing tag != tagStack.top()}
				\State \Return False
			\Else
				\State tagStack.pop()
			\EndIf
			\EndIf
		\EndWhile
		\pause
		\State \Return tagStack.size() == 0
	\end{algorithmic}
	\pause
		\column{0.405\textwidth}
			\begin{exampleblock}{Tag based languages}
				This basic syntax checkers, works for any tag-based language!
				\pause
				\begin{itemize}
					\item (La)TeX
					\item HTML
					\item XML
					\item But even for some basics of languages like Java.
				\end{itemize}
				\pause See also \url{https://youtu.be/QZOLb0xHB_Q} to see someone else explain the same thing :)
			\end{exampleblock}	
			
	\end{columns}
\end{frame}
\\
		\only<-8>{
		The stack of open \textit{tags}.\\
	}
		\only<9->{
			The code is done and my stack is empty $\to$ Correct TeX.
		}
	\end{columns}
	
\end{frame}

\begin{frame}
	\frametitle{Incorrect Tex}
	\begin{columns}
		\column{0.655\textwidth}
		\lstinputlisting[language=TeX,
		 linebackgroundcolor={%
    \btLstHL<1>{1}% on slide 1, highlight lines 1-3
    \btLstHL<2>{3}% on slide 2, highlight lines 6 and 9
    \btLstHL<3>{5}%
  }
		]{parseerror.tex}
		\column{0.355\textwidth}
		\begin{tikzpicture}[
  node distance=0.2em,
  stackframe/.style={font=\small,draw=structure,thick,fill=structure!0.1,text width=8em},
	every label/.style={right,font=\scriptsize\tt},
]
\onslide<1->{\node[stackframe,onslide=<1>{draw=alert}] (frame) {
  frame
};}

\onslide<2->{\node[stackframe,above=of frame,onslide=<2-3>{draw=alert}] (columns) {
		columns
};}

\end{tikzpicture}
\\
		The stack of open \textit{tags}.\\
		\only<3->{
			We encounter the wrong closing tag $\to$ Incorrect TeX.
		}
	\end{columns}
\end{frame}

\begin{frame}
	\frametitle{Our TeX parsing algorithm}
	\begin{columns}
		\column{0.655\textwidth}
	\begin{algorithmic}
		\State tagStack $\gets$ empty stack.
		\While{there is TeX}
		\pause
			\If{opening tag}
				\State tagStack.push(tagname)
			\Else
		\pause
			\If{closing tag != tagStack.top()}
				\State \Return False
			\Else
				\State tagStack.pop()
			\EndIf
			\EndIf
		\EndWhile
		\pause
		\State \Return tagStack.size() == 0
	\end{algorithmic}
	\pause
		\column{0.405\textwidth}
			\begin{exampleblock}{Tag based languages}
				This basic syntax checkers, works for any tag-based language!
				\pause
				\begin{itemize}
					\item (La)TeX
					\item HTML
					\item XML
					\item But even for some basics of languages like Java.
				\end{itemize}
				\pause See also \url{https://youtu.be/QZOLb0xHB_Q} to see someone else explain the same thing :)
			\end{exampleblock}	
			
	\end{columns}
\end{frame}
\\
		\only<-8>{
		The stack of open \textit{tags}.\\
	}
		\only<9->{
			The code is done and my stack is empty $\to$ Correct TeX.
		}
	\end{columns}
	
\end{frame}

\begin{frame}
	\frametitle{Incorrect Tex}
	\begin{columns}
		\column{0.655\textwidth}
		\lstinputlisting[language=TeX,
		 linebackgroundcolor={%
    \btLstHL<1>{1}% on slide 1, highlight lines 1-3
    \btLstHL<2>{3}% on slide 2, highlight lines 6 and 9
    \btLstHL<3>{5}%
  }
		]{parseerror.tex}
		\column{0.355\textwidth}
		\begin{tikzpicture}[
  node distance=0.2em,
  stackframe/.style={font=\small,draw=structure,thick,fill=structure!0.1,text width=8em},
	every label/.style={right,font=\scriptsize\tt},
]
\onslide<1->{\node[stackframe,onslide=<1>{draw=alert}] (frame) {
  frame
};}

\onslide<2->{\node[stackframe,above=of frame,onslide=<2-3>{draw=alert}] (columns) {
		columns
};}

\end{tikzpicture}
\\
		The stack of open \textit{tags}.\\
		\only<3->{
			We encounter the wrong closing tag $\to$ Incorrect TeX.
		}
	\end{columns}
\end{frame}

\begin{frame}
	\frametitle{Our TeX parsing algorithm}
	\begin{columns}
		\column{0.655\textwidth}
	\begin{algorithmic}
		\State tagStack $\gets$ empty stack.
		\While{there is TeX}
		\pause
			\If{opening tag}
				\State tagStack.push(tagname)
			\Else
		\pause
			\If{closing tag != tagStack.top()}
				\State \Return False
			\Else
				\State tagStack.pop()
			\EndIf
			\EndIf
		\EndWhile
		\pause
		\State \Return tagStack.size() == 0
	\end{algorithmic}
	\pause
		\column{0.405\textwidth}
			\begin{exampleblock}{Tag based languages}
				This basic syntax checkers, works for any tag-based language!
				\pause
				\begin{itemize}
					\item (La)TeX
					\item HTML
					\item XML
					\item But even for some basics of languages like Java.
				\end{itemize}
				\pause See also \url{https://youtu.be/QZOLb0xHB_Q} to see someone else explain the same thing :)
			\end{exampleblock}	
			
	\end{columns}
\end{frame}
\\
		\only<-8>{
		The stack of open \textit{tags}.\\
	}
		\only<9->{
			The code is done and my stack is empty $\to$ Correct TeX.
		}
	\end{columns}
	
\end{frame}

\begin{frame}
	\frametitle{Incorrect Tex}
	\begin{columns}
		\column{0.655\textwidth}
		\lstinputlisting[language=TeX,
		 linebackgroundcolor={%
    \btLstHL<1>{1}% on slide 1, highlight lines 1-3
    \btLstHL<2>{3}% on slide 2, highlight lines 6 and 9
    \btLstHL<3>{5}%
  }
		]{parseerror.tex}
		\column{0.355\textwidth}
		\begin{tikzpicture}[
  node distance=0.2em,
  stackframe/.style={font=\small,draw=structure,thick,fill=structure!0.1,text width=8em},
	every label/.style={right,font=\scriptsize\tt},
]
\onslide<1->{\node[stackframe,onslide=<1>{draw=alert}] (frame) {
  frame
};}

\onslide<2->{\node[stackframe,above=of frame,onslide=<2-3>{draw=alert}] (columns) {
		columns
};}

\end{tikzpicture}
\\
		The stack of open \textit{tags}.\\
		\only<3->{
			We encounter the wrong closing tag $\to$ Incorrect TeX.
		}
	\end{columns}
\end{frame}

\begin{frame}
	\frametitle{Our TeX parsing algorithm}
	\begin{columns}
		\column{0.655\textwidth}
	\begin{algorithmic}
		\State tagStack $\gets$ empty stack.
		\While{there is TeX}
		\pause
			\If{opening tag}
				\State tagStack.push(tagname)
			\Else
		\pause
			\If{closing tag != tagStack.top()}
				\State \Return False
			\Else
				\State tagStack.pop()
			\EndIf
			\EndIf
		\EndWhile
		\pause
		\State \Return tagStack.size() == 0
	\end{algorithmic}
	\pause
		\column{0.405\textwidth}
			\begin{exampleblock}{Tag based languages}
				This basic syntax checkers, works for any tag-based language!
				\pause
				\begin{itemize}
					\item (La)TeX
					\item HTML
					\item XML
					\item But even for some basics of languages like Java.
				\end{itemize}
				\pause See also \url{https://youtu.be/QZOLb0xHB_Q} to see someone else explain the same thing :)
			\end{exampleblock}	
			
	\end{columns}
\end{frame}
