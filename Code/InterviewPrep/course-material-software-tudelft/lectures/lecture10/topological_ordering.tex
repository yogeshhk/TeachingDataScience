\section{Topological Ordering}%
\label{sec:topological_ordering}

\begin{frame}
	\frametitle{Topological Ordering}
	
	\begin{center}
		\includegraphics[width=0.68\textwidth]{figures/topology.jpeg}\\
		\hspace*{15pt}\hbox{\scriptsize Image By:\thinspace{\itshape Tama66}}
		% https://pixnio.com/objects/geography-map-earth-globe-object-education-topology#
	\end{center}
\end{frame}

\begin{frame}
	\frametitle{Jobs to do!}
	\framesubtitle{Remember we saw this as a use case of the pre-order traversal in the binary tree.}
	\begin{problemblock}{Tons of things to do}
		I have a bunch of things to do, that depend on each other. My question is, what order can I do them in so that all
		dependencies are met?
	\end{problemblock}
	\pause
	\begin{columns}
		\column{0.455\textwidth}
		\begin{exampleblock}{For example...}
			I need to do the following things:
			\begin{itemize}
				\item Create an exam
		\pause
				\item Have the exam proof read
				\item Print the exam
		\pause
				\item Make the answer sheet
				\item Print the answer sheet
		\pause
				\item Bring everything to location
			\end{itemize}
		\end{exampleblock}	
		\column{0.455\textwidth}
		\pause
			\begin{block}{Lets draw that}
				Lets make that into a graph!\\
				Tasks become vertices.\\
				And dependencies become edges.\\
			\end{block}	
	\end{columns}
\end{frame}

\begin{frame}
	\frametitle{Now what do we need?}
	
		\begin{block}{A topological ordering}
			A topological ordering of the vertices $V$ in a \textit{directed} graph, is an ordering $v_1, \dots, v_n$ such
			that $e=(v_i,v_j)$ with $i < j$ for all $e \in E$.
		\end{block}	

		\pause
		\begin{block}{DAGs and topological orderings}
			A graph $G$ has a topological ordering if and only if it is a DAG.
		\end{block}	
		\pause
		\begin{proof}
			(first part) Suppose $G$ has a topological ordering.\\
			\pause
			We now use a proof by contradiction. Assume there is a cycle (i.e. it is no DAG). That means there is a cycle
			$(v_x, v_y), \dots (v_w, v_x)$. But if it has a topological ordering, then $x < y < w < x$, which is clearly not
			possible.\\
			\pause
			(second part) A little harder... Let's make an algorithm!
		\end{proof}
\end{frame}

\begin{frame}
	\frametitle{The idea}
	\begin{overlayarea}{\textwidth}{\textheight}
		\begin{block}{The idea}
			\begin{itemize}
				\item 
					If $G$ is acyclic, there must be some vertex without incoming edges.\\
				\item<3->
					We can start with this one, add it to the topological ordering and remove its outgoing edges.\\
				\item<4->
					Now repeat!
			\end{itemize}
		\end{block}	
		\only<2>{
			\begin{questionblock}{Why?}
				Why must this be the case?
			\end{questionblock}
		}
	\end{overlayarea}
\end{frame}

\begin{frame}
	\frametitle{Lets code it!}
	
	\begin{columns}
		\column{0.555\textwidth}
		{
	\small
	\begin{algorithmic}
		\Function{Topo}{$G$}
		\State order $\gets$ empty list
		\State q $\gets$ empty queue
		\pause
		\State $v \gets$ a random vertex with $\mathit{indeg}(v) = 0$.
		\State q.enqueue($v$)
		\pause
		\While{q is not empty}
		\State $v \gets$ q.dequeue()
		\State order.append($v$)
		\pause
		\For{each outgoing edge $e=(v,u)$ of $v$}
		\State $G$.remove(e)
		\pause
		\If{$\mathit{indeg}(u) == 0$}
		\State q.enqueue($u$)
		\pause
		\EndIf
		\EndFor
		\EndWhile
		\EndFunction
	\end{algorithmic}
}
			
		\column{0.405\textwidth}
			\begin{block}{Run time}
				\begin{itemize}
					\item If we implement this efficiently\dots
						\pause
					\item Then finding the first vertex is $\Theta(|V|)$
						\pause
					\item We then consider all vertices at most once: $\Theta(|V|)$
						\pause
					\item We also remove all edges at most once (which our map-based implementation can do in expected
						$\Theta(1)$): $\Theta(|E|)$.
					\item All-in-all: $\Theta(|V| + |E|)$.
				\end{itemize}	
			\end{block}	
			
	\end{columns}
\end{frame}

\begin{frame}
	\frametitle{One closing remark}
	
		\begin{questionblock}{One important difference}
			Both DFS/BFS and the topological ordering give an order of the nodes.\\
			\pause
			But is there a difference? (Other than the order in which they are returned?)
		\end{questionblock}	

		\pause
		\begin{answerblock}{Yes!}
			DFS/BFS require a starting node! And give different results (not necessarily the whole graph) when starting from
			different nodes.\\
			\pause
			Topological orderings are Graph properties, not depending on a specific vertex!
		\end{answerblock}
\end{frame}
