\chapter{\basisDimTitle}\label{basisdimension}

\label{sec:dimension}
\label{dimension}
In chapter~\ref{linearind}, the notions of   a linearly independent set of vectors in a vector space $V$, and of a set of vectors that span $V$ were established; any set of vectors that span $V$ can be reduced to some minimal collection of linearly independent vectors; such a minimal set is called a \emph{basis} of the subspace $V$.  

\begin{definition}
Let $V$ be a vector space.  Then a set $S$ is a {\bf basis}\index{Basis} for $V$ if $S$ is linearly independent and $V=\spa S$.


If $S$ is a basis of $V$ and $S$ has only finitely many elements, then we say that $V$ is {\bf finite-dimensional}.  The number of vectors in $S$ is the {\bf dimension}\index{Dimension} of~$V$.
\end{definition}

Suppose $V$ is a \emph{finite-dimensional}\index{Vector space!finite dimensional} vector space, and $S$ and $T$ are two different bases for $V$.  One might worry that $S$ and $T$ have a different number of vectors; then we would have to talk about the dimension of $V$ in terms of the basis $S$ or in terms of the basis $T$.  Luckily this isn't what happens.
Later in this chapter, we will show that $S$ and $T$ must have the same number of vectors.  This means that the dimension of a vector space is basis-independent.  In fact, dimension is a very important  characteristic of a vector space.% $V$.

\begin{example}
$P_n(t)$ (polynomials in $t$ of degree $n$ or less) has a basis $\{1,t,\ldots , t^n \}$, since every vector in this space is a sum
\[
a^0\,1+a^1\,t+\cdots +a^n\,t^n, \qquad a^i\in \Re\, ,
\]
so $P_n(t)=\spa \{1,t,\ldots , t^n \}$.  This set of vectors is linearly independent;  If the polynomial $p(t)=c^01+c^1t+\cdots +c^nt^n=0$, then $c^0=c^1=\cdots =c^n=0$, so $p(t)$ is the zero polynomial.  
Thus $P_n(t)$ is finite dimensional, and $\dim P_n(t)=n+1$.
\end{example}



\begin{theorem}\label{uniqvec}
Let $S=\{v_1, \ldots, v_n \}$  be a basis for a vector space $V$.  Then every vector $w \in V$ can be written \emph{uniquely} as a linear combination of vectors in the basis $S$:
\[
w=c^1v_1+\cdots + c^nv_n.
\]
\end{theorem}

\begin{proof}
Since $S$ is a basis for $V$, then $\spa S=V$, and so there exist constants~$c^i$ such that $w=c^1v_1+\cdots + c^nv_n$.

Suppose there exists a second set of constants $d^i$ such that 
$$w=d^1v_1+\cdots + d^nv_n\, .$$  Then
\begin{eqnarray*}
0_V&=&w-w\\
&=&c^1v_1+\cdots + c^nv_n-d^1v_1-\cdots - d^nv_n \\[1mm]
&=&(c^1-d^1)v_1+\cdots + (c^n-d^n)v_n. \\
\end{eqnarray*}
If it occurs exactly once that $c^i\neq d^i$, then the equation reduces to $0=(c^i-d^i)v_i$, which is a contradiction since the vectors $v_i$ are assumed to be non-zero.

If we have more than one $i$ for which $c^i\neq d^i$, we can use this last equation to write one of the vectors in $S$ as a linear combination of other vectors in $S$, which contradicts the assumption that $S$ is linearly independent.  Then for every $i$, $c^i=d^i$.
\end{proof}

\Videoscriptlink{basis_and_dimension_thm.mp4}{Proof Explanation}{basis_and_dimension_thm}

\begin{remark}
This theorem is the one that makes bases so useful--they allow us to convert abstract vectors into column vectors.
By ordering the set $S$ we obtain $B=(v_1,\ldots,v_n)$ and can write
$$
w=(v_1,\ldots,v_n) \ccolvec{c^1\\ \vdots\\ c^n }=\ccolvec{c^1\\ \vdots\\ c^n }_B\, .
$$
Remember that in general it makes no sense to drop the subscript $B$ on the column vector on the right--most vector spaces  are not made from  columns of numbers!
\end{remark}

\Videoscriptlink{eigenvectors_and_eigenvalues_matrix.mp4}{Worked Example}{scripts_eigenvalseigenvects_matrix}

Next, we would like to establish a method for determining whether a collection of vectors forms a basis for $\Re^n$.  But first, we need to show that any two bases for a finite-dimensional vector space has the same number of vectors.

\begin{lemma}\label{mlessn}
If $S=\{v_1, \ldots, v_n \}$ is a basis for a vector space $V$ and $T=\{w_1, \ldots, w_m \}$ is a linearly independent set of vectors in $V$, then $m\leq n$.
\end{lemma}

%\begin{figure}
%\begin{center}
%\includegraphics[scale=.28]{\basisDimPath/indep_span.jpg}
%\end{center}
%\end{figure}

The idea of the proof is to start with the set $S$ and replace vectors in $S$ one at a time with vectors from $T$, such that after each replacement we still have a basis for $V$.

%\begin{center}\href{\webworkurl ReadingHomework17/1/}{Reading homework: problem \ref{basisdimension}.1}\end{center}
\Reading{BasisAndDimension}{1}

\begin{proof}
Since $S$ spans $V$, then the set $\{w_1, v_1, \ldots, v_n \}$ is linearly dependent.  Then we can write $w_1$ as a linear combination of the $v_i$; using that equation, we can express one of the $v_i$ in terms of $w_1$ and the remaining $v_j$ with~$j\neq i$.  Then we can discard one of the $v_i$ from this set to obtain a linearly independent set that still spans $V$.  Now we need to prove that $S_1$ is a basis; we must show that $S_1$ is linearly independent and that $S_1$ spans $V$.

The set $S_1=\{w_1, v_1, \ldots, v_{i-1}, v_{i+1},\ldots, v_n \}$ is linearly independent:  By the previous theorem, there was a unique way to express $w_1$ in terms of the set~$S$.  Now, to obtain a contradiction, suppose there is some $k$ and constants~$c^i$ such that
\[
v_k = c^0w_1+c^1v_1+\cdots + c^{i-1}v_{i-1} + c^{i+1}v_{i+1} + \cdots + c^nv_n.
\]
Then replacing $w_1$ with its expression in terms of the collection $S$ gives a way to express the vector $v_k$ as a linear combination of the vectors in $S$, which contradicts the linear independence of $S$.  On the other hand, we cannot express $w_1$ as a linear combination of the vectors in $\{v_j | j\neq i\}$, since the expression of $w_1$ in terms of $S$ was unique, and had a non-zero coefficient for the vector $v_i$.  Then no vector in $S_1$ can be expressed as a combination of other vectors in $S_1$, which demonstrates that $S_1$ is linearly independent.

The set $S_1$ spans $V$:  For any $u\in V$, we can express $u$ as a linear combination of vectors in $S$.  But we can express $v_i$ as a linear combination of vectors in the collection $S_1$; rewriting $v_i$ as such allows us to express $u$ as a linear combination of the vectors in $S_1$. Thus $S_1$ is a basis of $V$ with $n$ vectors.

We can now iterate this process, replacing one of the $v_i$ in $S_1$ with $w_2$, and so on.  If $m\leq n$, this process ends with the set $S_m=\{w_1,\ldots, w_m$, $v_{i_1},\ldots,v_{i_{n-m}}  \}$, which is fine.

Otherwise, we have $m>n$, and the set $S_n=\{w_1,\ldots, w_n \}$ is a basis for~$V$.  But we still have some vector 
$w_{n+1}$  in $T$ that is not in $S_n$.  Since $S_n$ is a basis, we can write $w_{n+1}$ as a combination of the vectors in $S_n$, which contradicts the linear independence of the set $T$.  Then it must be the case that $m\leq n$, as desired.
\end{proof}

\Videoscriptlink{basis_and_dimension_example.mp4}{Worked Example}{basis_and_dimension_example}

\begin{corollary}\label{corsame}
For a finite-dimensional vector space $V$, any two bases for~$V$ have the same number of vectors.
\end{corollary}

\begin{proof}
Let $S$ and $T$ be two bases for $V$.  Then both are linearly independent sets that span $V$.  Suppose $S$ has $n$ vectors and $T$ has $m$ vectors.  Then by the previous lemma, we have that $m\leq n$.  But (exchanging the roles of $S$ and $T$ in application of the lemma) we also see that $n\leq m$.  Then $m=n$, as desired.
\end{proof}

%\begin{center}\href{\webworkurl ReadingHomework17/2/}{Reading homework: problem \ref{basisdimension}.2}\end{center}
\Reading{BasisAndDimension}{2}

\section{Bases in $\Re^n$.}

In review question~\ref{stdbasis}, chapter~\ref{linearind} you checked that
\[
\Re^n = \spa \left\{ \colvec{1\\0\\ \vdots \\ 0}, 
\colvec{0\\1\\ \vdots \\ 0}, \ldots, \colvec{0\\0\\ \vdots \\ 1}\right\},
\]
and that this set of vectors is linearly independent. (If you didn't do that problem, check this before reading any further!)  So this set of vectors is a basis for~$\Re^n$, and $\dim \Re^n=n$.  This basis is often called the \emph{standard} or \emph{canonical basis}\index{Standard basis}\index{Canonical basis|seealso{Standard basis}} for $\Re^n$.  The vector with a one in the $i$th position and zeros everywhere else is written 
$e_i$. (You could also view it as the function $\{1,2,\ldots,n\}\to {\mathbb R}$ where $e_i(j)=1$ if $i=j$ and $0$ if $i\neq j$.)  It points in the direction of the $i$th coordinate axis, and has unit length.  In multivariable calculus classes, this basis is often written $\{ \hat i,\hat j,\hat k \}$ for $\Re^3$. 

%\begin{figure}
%\begin{center}
%\includegraphics[scale=.32]{\basisDimPath/canonical.jpg}
%\end{center}
%\end{figure}

Note that it is often convenient to order  basis elements, so rather than writing a set
of vectors, we would write a list. This is called an ordered basis. For example, the canonical ordered basis for ${\mathbb R^n}$ is $(e_1,e_2,\ldots,e_n)$. The possibility to reorder basis vectors is not the only way in which bases are non-unique.

\begin{remark}[Bases are not unique.]
While there exists a unique way to express a vector in terms of any particular basis, bases themselves are far from unique.
For example, both of the sets 
\[
\left\{ \colvec{1\\0}, \colvec{0\\1} \right\} \text{ and }
\left\{ \colvec{1\\1}, \colvec{1\\-1} \right\}
\]
are bases for $\Re^2$.  Rescaling any vector in one of these sets is already enough to show that~$\Re^2$ has infinitely many bases.  But even if we require that all of the basis vectors have unit length, it turns out that there are still infinitely many bases for $\Re^2$ (see review question~\ref{lotsofbases}).
\end{remark}


To see whether a set of vectors $S=\{v_1, \ldots, v_m \}$ is a basis for~$\Re^n$, we have to check that the elements  are linearly independent and that they span~$\Re^n$.  From the previous discussion, we also know that $m$ must equal $n$, so lets assume~$S$ has $n$ vectors.
If $S$ is linearly independent, then there is no non-trivial solution of the equation
\[
0 = x^1v_1+\cdots + x^nv_n.
\]
Let $M$ be a matrix whose columns are the vectors $v_i$ and $X$ the column vector with entries $x^i$.  Then the above equation is equivalent to requiring that there is a unique solution to \[MX=0\, .\]

To see if $S$ spans $\Re^n$, we take an arbitrary vector $w$ and solve the linear system
\[
w=x^1v_1+\cdots + x^nv_n
\]
in the unknowns $x^i$.  For this, we need to find a unique solution for the linear system $MX=w$.  

Thus, we need to show that $M^{-1}$ exists, so that 
\[
X=M^{-1}w
\]
is the unique solution we desire.  Then we see that $S$ is a basis for $\mathbb{R}^n$ if and only if $\det M\neq 0$.




\begin{theorem}
Let $S=\{v_1, \ldots, v_m \}$ be a collection of vectors in $\Re^n$.  Let~$M$ be the matrix whose columns are the vectors in $S$.  Then $S$ is a basis for $V$ if and only if $m$ is the dimension of $V$ and 
\[
\det M \neq 0.
\]
\end{theorem}

\begin{remark}
Also observe that  $S$ is a basis if and only if ${\rm RREF}(M)=I$.
\end{remark}

\begin{example}
Let 
\[
S=\left\{ \colvec{1\\0}, \colvec{0\\1} \right\} \text{ and }
T=\left\{ \colvec{1\\1}, \colvec{1\\-1} \right\}.
\]
Then set $M_S=\begin{pmatrix}
1 & 0\\
0 & 1\\
\end{pmatrix}$.  Since $\det M_S=1\neq 0$, then $S$ is a basis for $\Re^2$.\\

\noindent
Likewise, set $M_T=\begin{pmatrix}
1 & 1\\
1 & -1\\
\end{pmatrix}$.  Since $\det M_T=-2\neq 0$, then $T$ is a basis for $\Re^2$.
\end{example}


\section{Matrix of a Linear Transformation (Redux)}\index{Matrix of a linear transformation}

Not only do bases allow us to describe arbitrary vectors as column vectors, they also permit linear transformations
 to be expressed as matrices. This is a very powerful tool for computations, which is covered in chapter~\ref{Matrices} and
 reviewed again here.
 
Suppose we have a linear transformation $L \colon V\rightarrow W$ and 
ordered input and output bases $E=(e_1, \ldots, e_n)$ and $F=(f_1, \ldots, f_m)$ for $V$ and $W$ respectively (of course, these need not be the standard basis--in all likelihood $V$ is {\it not} ${\mathbb R}^n$). 
Since for each $e_j$, $L(e_j)$ is a vector in $W$, there exist unique  numbers~$m^i_j$ such that
\[
L(e_j)=f_1m^1_j + \cdots + f_mm^m_j =(f_1,\ldots, f_m) \ccolvec{m^1_j\\\vdots\\m^m_j}\, .
%= \sum_{i=1}^m f_iM^i_j
%=\rowvec{f_1 & f_2 & \cdots & f_m}\colvec{M^1_j \\[1mm] M^2_j \\[1mm] \vdots \\[1mm] M^m_j}\, .
\]
%We've written the $M^i_j$ on the right side of the $f$'s to agree with our previous notation for matrix multiplication.  
%We have an ``up-hill rule'' where the matching indices for the multiplied objects run up and to the right, like so:~$f_iM^i_j$.
The number $m^i_j$ is the $i$th component of $L(e_j)$ in the basis $F$, while the $f_i$ are vectors (note that if $\alpha$ is a scalar, and $v$ a vector, $\alpha v=v\alpha$, we have used the latter---rather uncommon---notation in the above formula).
The numbers $m^i_j$ naturally form a matrix whose $j$th column is the column vector displayed above. 
%we can see that the $j$th column of $M$ is the coefficients of $L(e_j)$ in the basis $F$.
%The most efficient way to see this is through Einstein notation:
%by linearity, for any vector $v$ in $V$ 
%\begin{eqnarray*}
%Lv=L(e_iv^i)=L(e_i)v^i=f_i M^i_j v^j
%\\
%\end{eqnarray*}
%\begin{eqnarray*}
%L(v) & = & L( v^1e_1 + v^2e_2 + \cdots + v^ne_n) \\[2mm]
%     & = & v^1L(e_1) + v^2L(e_2) + \cdots + v^nL(e_n)\\[2mm]
%     &=&\rowvec{L(e_1) & L(e_2) & \cdots & L(e_n)}\colvec{v^1 \\ v^2 \\ \vdots \\ v^n}\, .
%\end{eqnarray*}
%This is a vector in $W$.  Let's compute its components in $W$.
%
%%%%%%%%%%%%%%%%%%%
Indeed, if $$v=e_1v^1+\cdots+e_n v^n\, ,$$
Then
\begin{eqnarray*}
L(v) & = & L( v^1e_1 + v^2e_2 + \cdots + v^ne_n) \\[1mm]
     & = & v^1L(e_1) + v^2L(e_2) + \cdots + v^nL(e_n) 
     \: = \: \sum_{j=1}^m L(e_j) v^j \\
     & = & \sum_{j=1}^m ( f_1 m^1_j+ \cdots + f_mm^m_j) v^j 
     \: = \: \sum_{i=1}^n f_i \left[ \sum_{j=1}^m M^i_jv^j \right]\\
     &=&\rowvec{f_1 & f_2 & \cdots & f_m}
             \left(\!\begin{array}{cccc}m^1_1&m^1_2&\cdots &m^1_n\\ m^2_1 & m^2_2 && \\
                                              \vdots &&\ddots&\vdots\\ m^m_1 &&\cdots & m^m_n\end{array}\!\right)\colvec{v^1 \\ v^2 \\ \vdots \\ v^n}
\end{eqnarray*}
%The last equality above comes from the definition of matrix multiplication. 
In the column vector-basis notation this equality looks familiar: 
\[
L\ccolvec{v^1\\ \vdots \\ v^n}_E 
%\stackrel{L}{\mapsto}
=
\left(
\left(\!\begin{array}{ccc}
m^1_1 & \ldots & m^1_n \\
\vdots & & \vdots \\
m^m_1 & \ldots & m^m_n \\
\end{array}\!\right)
\ccolvec{v^1\\ \vdots \\ v^n}
\right)_F.
\]
The array of numbers $M=(m^i_j)$ is called the matrix of 
$L$ in the input and output bases $E$ and $F$ for $V$ and $W$, respectively. 
This matrix will change if we change either of the bases. 
Also observe that the columns of $M$ are computed by examining $L$ acting on each basis vector in $V$ expanded in the 
basis vectors of $W$.
%
%An efficient procedure for finding the matrix for a linear operator in particular bases for its domain and target is to calculate the way the linear operator acts on the basis vectors from the domain one at a time. This allows you to read off the columns of the matrix as
%$$
%L(e_j)=
%\rowvec{f_1 & f_2 & \cdots & f_m}\ccolvec{M^1_j \\[1mm] M^2_j \\[1mm] \vdots \\[1mm] M^m_j} .
%$$
\begin{example}

\noindent
Let $L \colon P_1(t) \mapsto P_1(t)$, such that $L(a+bt)=(a+b)t$.  Since $V=P_1(t)=W$, let's choose the same ordered basis $B=(1-t, 1+t )$ for $V$ and $W$.
\begin{eqnarray*}
L(1-t)&=&(1-1)t=\ 0\ =(1-t)\cdot 0 + (1+t)\cdot 0=
\rowvec{1-t, 1+t}\colvec{0\\0} \\[2mm]
L(1+t)&=&(1+1)t=2t\ =(1-t)\cdot -1 + (1+t)\cdot 1=
\rowvec{1-t, 1+t}\colvec{-1\\1}\\[2mm]
&\Rightarrow& 
L\colvec {a\\b }_B= 
\left( 
\begin{pmatrix}
0 & -1 \\
0 & 1 \\
\end{pmatrix}
\colvec{a\\b}
\right)_B.
\end{eqnarray*}
\end{example}


%
%\begin{example}
%Consider a linear transformation $$L \colon \Re^2\rightarrow \Re^2\, .$$  Suppose we know that $L\colvec{1\\0}=\colvec{a\\c}$ and $L\colvec{0\\1}=\colvec{b\\d}$.  Then, because of linearity, we can determine what $L$ does to any vector $\colvec{x\\y}$:
%
%\[
%L\colvec{x\\y}=L\left(x\colvec{1\\0}+y\colvec{0\\1}\right)=xL\colvec{1\\0}+yL\colvec{0\\1}=x\colvec{a\\c}+y\colvec{b\\d}=\colvec{ax+by\\cx+dy}.
%\]
%Now notice that for any vector $\colvec{x\\y}$, we have 
%
%\[
%\begin{pmatrix}
%a & b \\
%c & d \\
%\end{pmatrix}
%\colvec{x\\y}=\colvec{ax+by\\cx+dy}=L\colvec{x\\y}.
%\]
%Then the matrix $\begin{pmatrix}
%a & b \\
%c & d \\
%\end{pmatrix}$ acts by matrix multiplication in the same way that~$L$ does.  This is the  \emph{matrix of $L$} in the standard (ordered) \emph{basis} $\left(\colvec{1\\0}, \colvec{0\\1} \right)$.
%\end{example}

When the vector space is~${\mathbb R^n}$ and the standard basis is used,  the problem of finding the matrix of a 
linear transformation will seem almost trivial. It is worthwhile working through it once in the above language though.


\begin{example}
Any vector in $\Re^n$ can be written as a linear combination of the \emph{standard (ordered) basis}\index{Standard basis} $(e_1,\dots e_n)$.  
The vector $e_i$ has a one in the $i$th position, and zeros everywhere else.  {\it I.e.}
$$
e_1=\colvec{1\\ 0\\ \vdots \\0}\, ,\quad e_2=\colvec{0\\ 1\\ \vdots \\0}\, ,\ldots,\quad e_n=\colvec{0\\ 0\\ \vdots \\1 }\, .
$$
Then to find the matrix of any linear transformation $L \colon \Re^n \rightarrow \Re^n$, it suffices to know what $L(e_i)$ is for every $i$.  

For any matrix $M$, observe that $Me_i$ is equal to the $i$th column of $M$.  Then if the $i$th column of $M$ equals $L(e_i)$ for every $i$, then $Mv=L(v)$ for every $v\in \Re^n$.  Then the matrix representing $L$ in the standard basis is just the matrix whose $i$th column is~$L(e_i)$. 

For example, if 
$$
L\colvec{1\\0\\0}=\colvec{1\\4\\7}\, ,\quad
L\colvec{0\\1\\0}=\colvec{2\\5\\8}\, ,\quad
L\colvec{0\\0\\1}=\colvec{3\\6\\9}\, ,
$$
then the matrix of $L$ in the standard basis is simply
$$
\begin{pmatrix}1&2&3\\4&5&6\\7&8&9\end{pmatrix}\, .
$$
Alternatively, this information would often be presented as
$$
L\colvec{x\\y\\z}=\ccolvec{x+2y+3z\\4x+5y+6z\\7x+8y+9z}\, .
$$
You could either rewrite this  as 
$$
L\colvec{x\\y\\z}=\begin{pmatrix}1&2&3\\4&5&6\\7&8&9\end{pmatrix}\colvec{x\\y\\z}\, ,
$$
to immediately learn the matrix of $L$, or taking a more circuitous route:
\begin{eqnarray*}
L\colvec{x\\y\\z}&=&L\left[x\colvec{1\\0\\0}+y\colvec{0\\0\\1}+z
\colvec{0\\0\\1}\right]\\[2mm]&=&
x\colvec{1\\4\\7}+y\colvec{2\\5\\8}+z
\colvec{3\\6\\9}\:=\:\begin{pmatrix}1&2&3\\4&5&6\\7&8&9\end{pmatrix}\colvec{x\\y\\z}\, .
\end{eqnarray*}
\end{example}

%\section*{References}
%Hefferon, Chapter Two, Section II: Linear Independence
%\\
%Hefferon, Chapter Two, Section III.1: Basis
%\\
%Beezer, Chapter VS, Section B, Subsections B-BNM
%\\
%Beezer, Chapter VS, Section D, Subsections D-DVS
%\\
%Wikipedia:
%\begin{itemize}
%\item \href{http://en.wikipedia.org/wiki/Linear_independence}{Linear Independence}
%\item \href{http://en.wikipedia.org/wiki/Basis_(linear_algebra)}{Basis}
%\end{itemize}
%

\section{Review Problems}

{\bf Webwork:} 
\begin{tabular}{|c|c|}
\hline
Reading Problems & 
 \hwrref{BasisAndDimension}{1},\hwrref{BasisAndDimension}{2}\\
Basis checks&  \hwref{BasisAndDimension}{3},\hwref{BasisAndDimension}{4}\\
 Computing column vectors &  \hwref{BasisAndDimension}{5},\hwref{BasisAndDimension}{6}\\
  \hline
\end{tabular}



\begin{enumerate}

\item While performing  Gaussian elimination on these augmented matrices write the full system of equations describing the new rows in terms of the old rows above each equivalence symbol as in  \hyperlink{Keeping track of EROs with equations between rows}{Example}~\ref{Rsystem}. 
$$
\begin{amatrix}{2} 
2 & 2 & 10 \\
1 & 2 & 8 \\
\end{amatrix}
,~
\begin{amatrix}{3} 
1 & 1 & 0 & 5 \\
1 & 1 & \!\!-1& 11 \\
-1 & 1 & 1 & -5 \\ 
\end{amatrix}
$$

%%%%%%%%%%%%%%%%%%%

\item Solve the vector equation by applying ERO matrices to each side of the equation to perform elimination. Show each matrix explicitly as in \hyperlink{Undoing}{Example~\ref{slowly}}.

\begin{eqnarray*}
\begin{pmatrix}
3	&6 	&2 \\ %-3
5 	&9 	&4 \\ %1
2	&4	&2 \\ %0
\end{pmatrix} 
\begin{pmatrix}
 x \\ 
y \\
z 
\end{pmatrix} 
=
\begin{pmatrix}
-3 \\ 
1  \\
0  \\
\end{pmatrix} 
\end{eqnarray*}

%%%%%%%%%%%%%%%%%%%

\item Solve this vector equation by finding the inverse of the matrix through $(M|I)\sim (I|M^{-1})$ and then applying $M^{-1}$ to both sides of the equation. 
\begin{eqnarray*}
\begin{pmatrix}
2	&1 	&1 \\ %9
1 	&1 	&1 \\ %6
1	&1	&2 \\ %7
\end{pmatrix} 
\begin{pmatrix}
 x \\ 
y \\
z 
\end{pmatrix} 
=
\begin{pmatrix}
9 \\ 
6  \\
7  \\
\end{pmatrix} 
\end{eqnarray*}


%%%%%%%%%%%%%%%%%%%

\item Follow the method of  \hyperlink{elldeeeww}{Examples~\ref{factorize} and~\ref{factorizes}} to find the $LU$ and $LDU$ factorization of 
\begin{eqnarray*}
\begin{pmatrix}
3	&3 	&6 \\ %0 %2
3 	&5 	&2 \\ %1 %1
6	&2	&5 \\ %0 %1
\end{pmatrix} .
\end{eqnarray*}



%%%%%%%%%%%%%%%%%%%%

\item 
Multiple matrix equations with the same matrix can be solved simultaneously. 
\begin{enumerate}
\item Solve both systems by performing elimination on just one augmented matrix.
\begin{eqnarray*}
\begin{pmatrix}
2	&-1 	&-1 \\ %0 %2
-1 	&1 	&1 \\ %1 %1
1	&-1	&0 \\ %0 %1
\end{pmatrix} 
\begin{pmatrix}
 x \\ 
y \\
z 
\end{pmatrix} 
=
\begin{pmatrix}
0\\ 
1  \\
0  \\
\end{pmatrix} 
,~
\begin{pmatrix}
2	&-1 	&-1 \\ %0 %2
-1 	&1 	&1 \\ %1 %1
1	&-1	&0 \\ %0 %1
\end{pmatrix} 
\begin{pmatrix}
 a \\ 
b \\
c 
\end{pmatrix} 
=
\begin{pmatrix}
2\\ 
1  \\
1  \\
\end{pmatrix} 
\end{eqnarray*}
\item Give an interpretation of the columns of $M^{-1}$ in $(M|I)\sim (I|M^{-1})$ in terms of solutions to certain systems of linear equations.
\end{enumerate}

%%%%%%%%%%%%%%%%%%%%%%%%

\item How can you convince your fellow students to never make this mistake?
\begin{eqnarray*}
\begin{amatrix}{3} 
1 & 0 & 2 & 3 \\ 
0 & 1 & 2& 3 \\
2 & 0 & 1 & 4 \\
\end{amatrix} 
& 
\stackrel{R_1'=R_1+R_2}{
\stackrel{R_2'=R_1-R_2}{ 
\stackrel{\ R_3'= R_1+2R_2}{\sim}}}
&
\begin{amatrix}{3} 
1 & 1 & 4 & 6 \\
1 & \!\!-1 & 0& 0 \\
1 & 2 & 6 & 9 
\end{amatrix}
\end{eqnarray*}

\item Is $LU$ factorization of a matrix unique?  Justify your answer.


\item[$\infty$.] If you randomly create a matrix by picking numbers out of the blue, it will probably be difficult to perform elimination or factorization; fractions and large numbers will probably be involved. To invent simple problems it is better to start with a simple answer:
\begin{enumerate}
\item Start with any augmented matrix in RREF. Perform EROs to make most of the components non-zero. Write the result on a separate piece of paper and give it to your friend. Ask that friend to find RREF of the augmented matrix you gave them. Make sure they get the same augmented matrix you started with.  
\item Create  an upper triangular matrix $U$ and a lower triangular matrix~$L$ with only $1$s on the diagonal. Give the result to a friend to factor into $LU$ form. 
\item Do the same with an $LDU$ factorization. 
\end{enumerate}
\end{enumerate}

\phantomnewpage



\newpage

