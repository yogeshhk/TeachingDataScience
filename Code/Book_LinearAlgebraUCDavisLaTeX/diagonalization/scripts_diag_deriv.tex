
\subsection*{Non Diagonalizable Example}

%%%Insert this to get the typewriter font so it looks like a real movie script
{\ttfamily
\fontdimen2\font=0.4em
\fontdimen3\font=0.2em
\fontdimen4\font=0.1em
\fontdimen7\font=0.1em
\hyphenchar\font=`\-


\hypertarget{scripts_diagonalization_derivative}{First recall that} \hyperlink{derivative_linear}{the derivative operator is linear} and that we can write it as the matrix
\[
\frac{d}{dx} =
\begin{pmatrix}
0 & 1 & 0 & 0 & \cdots \\
0 & 0 & 2 & 0 & \cdots \\
0 & 0 & 0 & 3 & \cdots \\
\vdots & \vdots & \vdots & \vdots & \ddots
\end{pmatrix}.
\]
We note that this transforms into an infinite Jordan cell with eigenvalue 0 or
\[
\begin{pmatrix}
0 & 1 & 0 & 0 & \cdots \\
0 & 0 & 1 & 0 & \cdots \\
0 & 0 & 0 & 1 & \cdots \\
\vdots & \vdots & \vdots & \vdots & \ddots
\end{pmatrix}
\]
which is in the basis $\{n^{-1} x^n \}_n$ (where for $n = 0$, we just have 1). Therefore we note that $1$ (constant polynomials) is the only eigenvector with eigenvalue $0$ for polynomials since they have finite degree, and so the derivative is not diagonalizable. Note that we are ignoring infinite cases for simplicity, but if you want to consider infinite terms such as convergent series or all formal power series where there is no conditions on convergence, there are many eigenvectors. Can you find some? This is an example of how things can change in infinite dimensional spaces.

For a more finite example, consider the space $\mathbb{P}^{\mathbb{C}}_3$ of complex polynomials of degree at most 3, and recall that the derivative $D$ can be written as
\[
D = \begin{pmatrix}
0 & 1 & 0 & 0 \\
0 & 0 & 2 & 0 \\
0 & 0 & 0 & 3 \\
0 & 0 & 0 & 0
\end{pmatrix}.
\]
You can easily check that the only eigenvector is $1$ with eigenvalue $0$ since $D$ always lowers the degree of a polynomial by 1 each time it is applied. Note that this is a nilpotent matrix since $D^4 = 0$, but the only nilpotent matrix that is ``diagonalizable'' is the $0$ matrix.

} % Closing bracket for font

%\newpage
