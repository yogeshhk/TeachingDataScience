\documentclass[12pt]{article}

%To do list:
%%1) Fix numbering of reading problems from the reordering, e.g. Lecture 20, q1-->Lecture 18, q1
%%2) Rename webwork sets according to materials not lecture number
%%3) Move images into directories. Here
%%4) Webwork q5 HW6 (in MAT 67 ordering, say <,> for vector notation.
%%5) Problem 6, HW6 has a bug when the eigenvalue is 0.


%%%%%Some commands to make online notes...
\newcommand{\Lecture}{Module}
\newcommand{\moduletitle}[1]{\newpage\thispagestyle{empty}\begin{center}{\LARGE #1}\end{center}}
\newcommand{\modulesection}[1]{\thispagestyle{empty}\section{#1}}
\newcommand{\modulesubsection}[1]{\thispagestyle{empty}\subsection{#1}}
\newcommand{\modulecomment}[1]{}
\newcommand{\moduleobjectives}[1]{#1}
\newcommand{\inputProblems}[1]{}
\newcommand{\moduleinputProblems}[1]{\input{#1}}
\newcommand{\References}[1]{}
\newcommand{\moduletext}[2]{#2}

%use this if you need to show labels, but turn it off afterwards.
%\usepackage{showkeys}
%\usepackage{multimedia}
\usepackage{makeidx}
\usepackage{nopageno} 

\usepackage{comment}
\usepackage{shadow}
\usepackage[english]{babel}
\usepackage{blindtext}
\usepackage{graphicx, color}
\usepackage{tikz}
\DeclareGraphicsRule{*}{mps}{*}{}
\usepackage{linalgjh}
\usepackage{multirow}
\usepackage[colorlinks=true,linkcolor=blue,urlcolor=red,pdfnewwindow=true]{hyperref}
\usepackage[toc]{glossaries}
\usepackage{textcomp}
% Added to work with my enumerates, and I feel that it allows us more flexibility
%  to select how we want to number our items - TCS 1/9/12
\usepackage{enumerate}

\usepackage{linalgdir} % Get the directory and title information

%Change the next line as appropriate for your course.
\newcommand{\webworkurl}{http://webwork.math.ucdavis.edu/webwork2/MAT22A-Waldron-Winter-2012/}
%\newcommand{\webworkurl}{http://webwork.math.ucdavis.edu/webwork2/LinearAlgebra/}
\newcommand{\videourl}{http://math.ucdavis.edu/~linear/videos/}

% This command works by specifying the following (in order):
%  1 - the filename (with extension)
%  2 - the text describing the video
%  3 - The hyperlink reference
% For example we have
% \videoscriptlink{gaussian_elimination_more_background.mp4}{Augmented Matrix Notation~\ref{ge4} and~\ref{ge5}}{script_gaussian_elimination_more}
\newcommand{\videoscriptlink}[3]{
\begin{center}
\href{\videourl #1}{\raisebox{-.4cm}{\includegraphics[scale=.075]{take1.jpg}}} \hspace{1cm}\scalebox{1.2}{\tt #2}
\hspace{1cm} \hyperlink{#3}{\raisebox{-.3cm}{\includegraphics[scale=.08]{script.jpg}}}
\end{center}
} % videoscriptlink definition
% I changed all videos I could find to this command - TCS 1/9/12

%%%This links to reading homeworks:
\newcommand{\reading}[2]{\begin{center}
\href{\webworkurl ReadingHomework#1/#2/}{
\raisebox{-3mm}{\includegraphics[scale=.1]{glasses.jpg}}\hspace{2mm}Reading homework: problem #1.#2}\hspace{8mm}\end{center}}

%%%This links to Mark's videos
\newcommand{\markvideolink}[2]{
\begin{center}
\href{\videourl #1}{\raisebox{-.4cm}{\includegraphics[scale=.075]{take1.jpg}}} \hspace{1cm}\scalebox{1.2}{\tt #2}
\hspace{1cm} {\raisebox{-.3cm}{\includegraphics[scale=.2]{mark.jpg}}}
\end{center}
}


%%%This links to Kat's videos
\newcommand{\katvideolink}[2]{
\begin{center}
\href{\videourl #1}{\raisebox{-.4cm}{\includegraphics[scale=.075]{take1.jpg}}} \hspace{1cm}\scalebox{1.2}{\tt #2}
\hspace{1cm} {\raisebox{-.3cm}{\includegraphics[scale=.2]{kat.jpg}}}
\end{center}
}



%A simple URL macro
\newcommand{\URL}[1]{\begin{center}\href{#1}{\tt\scriptsize #1}\end{center}}
\newcommand{\URLS}[2]{\begin{center}\href{#1}{\tt\scriptsize #2}\end{center}}




\def\nn{\nonumber}

\newtheorem{theorem}{Theorem}[section]
\newtheorem{lemma}[theorem]{Lemma}
\newtheorem{proposition}[theorem]{Proposition}
\newtheorem{corollary}[theorem]{Corollary}


\newenvironment{definition}[1][Definition]{\begin{trivlist}
\item[\hskip \labelsep {\bfseries #1}]}{\end{trivlist}}
\newenvironment{example}[1][Example]{\small \sf \begin{trivlist}
\item[\hskip \labelsep {\bfseries #1}]}{\end{trivlist}}
\newenvironment{remark}[1][Remark]{\small \sf \begin{trivlist}
\item[\hskip \labelsep {\bfseries #1}]}{\end{trivlist}}

\DeclareMathOperator{\tr}{tr}
\DeclareMathOperator{\rref}{RREF}

% sideremark
\def\sideremark#1{\ifvmode\leavevmode\fi\vadjust{\vbox to0pt{\vss
\hbox to 0pt{\hskip\hsize\hskip1em
\vbox{\hsize3cm\tiny\raggedright\pretolerance10000
 \noindent #1\hfill}\hss}\vbox to8pt{\vfil}\vss}}}


\newcommand{\edz}[1]{\sideremark{#1}}
\def\idx#1{{\em #1\/}} % ****

\newcommand{\1}{{\rm 1\hspace*{-0.4ex}%
\rule{0.1ex}{1.52ex}\hspace*{0.2ex}}}

\DeclareMathOperator{\cofactor}{cofactor}
\DeclareMathOperator{\spa}{span}
\DeclareMathOperator{\nul}{null}
\newcommand{\phantomnewpage}{}

\makeindex

\makeglossaries


%------------------------------------------------
% TODO LIST
% 0) We moved sections 17,18,19 and 24.
% 1) Reorder WebWork to match section reordering.
% 2) Read through, check for dependency problems.
% 3) Fu's list of corrections
% 4) notes2: Worked example with no solutions
% 
% 
% 


\begin{document}

\pagestyle{plain}

%\newpage

%Testing the video


%\movie[width=8cm,height=10cm]{The Terminator}{videos/vector_spaces_example.mp4}

%\newpage

\tableofcontents


\chapter{Least squares and Singular Values}
\label{sec:leastsquaresSVD}
\index{Least squares}

Consider the linear algebraic equation $L(x)=v$, where $L \colon U\stackrel{\text{linear}}{-\!\!\!-\!\!\!\longrightarrow}W$ and $v\in W$ are known while $x$ is unknown. As we have seen, this system may have 
one solution, no solutions, or infinitely many solutions.  
But if $v$ is not in the range of $L$ there will {\it never} be any solutions for $L(x)=v$.
\vspace{-.1cm}
\begin{center}
\includegraphics[scale=.24]{notinimage.jpg}
\end{center} 
\vspace{-1.8cm}
However, for many applications we do not need an exact solution of the system; instead, we may only need the best approximation possible.  

\begin{quote}
``My work always tried to unite the Truth with the Beautiful, but when I had to choose one or the other, I usually chose the Beautiful.'' 

\vspace{-2mm}
\hspace{7cm}-- Hermann Weyl.
\end{quote}

If the vector space $W$ has a notion of lengths of vectors, we can try to find $x$ that minimizes $||L(x)-v||$.
\begin{center}
\includegraphics[scale=.24]{minimize.jpg}
\end{center} 
This method has many applications, such as when trying to fit a (perhaps linear) function to a ``noisy'' set of observations.  For example, suppose we measured the position of a bicycle on a racetrack once every five seconds.  Our observations won't be exact, but so long as the observations are right on average, we can figure out a best-possible linear function of position of the bicycle in terms of time.

Suppose $M$ is the matrix for the linear function $L:U \to W$ in some bases for $U$ and $W$. The vectors~$v$ and~$x$ are represented by column vectors $V$ and $X$ in these bases.  Then we need to approximate
\[
MX-V\approx 0\, .
\]

Note that if $\dim U=n$ and $\dim W=m$ then $M$ can be represented by an $m\times n$ matrix and $x$ and $v$ as vectors in $\Re^n$ and $\Re^m$, respectively. Thus, we can write $W=L(U)\oplus L(U)^\perp$.  Then we can uniquely write $v=v^\parallel + v^\perp$, with $v^\parallel \in L(U)$ and $v^\perp \in L(U)^\perp$.  



Thus we should solve $L(u)=v^\parallel$.  In components, $v^\perp$ is just $V-MX$, and is the part we will eventually wish to minimize.  

In terms of $M$, recall that $L(V)$ is spanned by the columns of $M$.  (In the standard basis, the columns of $M$ are $Me_1$, 
$\ldots$, $Me_n$.)  Then $v^\perp$ must be perpendicular to the columns of $M$.  \textit{i.e.}, $M^T(V-MX)=0$, or
\[
M^TMX = M^TV.
\]
Solutions of $M^TMX = M^TV$ for $X$ are called \emph{least squares}\index{Least squares!solutions} solutions to $MX=V$.  
Notice that any solution $X$ to $MX=V$ is a least squares solution.  However, the converse is often false.  In fact, the equation $MX=V$ may have no solutions at all, but still have least squares solutions to $M^TMX = M^TV$.

Observe that since $M$ is an $m\times n$ matrix, then $M^T$ is an $n\times m$ matrix.  Then $M^TM$ is an $n\times n$ matrix, and is symmetric, since $(M^TM)^T=M^TM$.  Then, for any vector $X$, we can evaluate $X^TM^TMX$ to obtain a number.  This is a very nice number, though!  It is just the length $|MX|^2 = (MX)^T(MX)=X^TM^TMX$.

%\href{\webworkurl ReadingHomework25/1/}{Reading homework: problem 25.1}
\Reading{LeastSquares}{1}

Now suppose that $\ker L=\{0\}$, so that the only solution to $MX=0$ is $X=0$. (This need not mean that $M$ is invertible because $M$ is an $n\times m $ matrix, so not necessarily square.) 
However the square matrix $M^TM$ {\it is} invertible. To see this, suppose there was a vector $X$ such that 
$M^T M X=0$. Then it would follow that $X^T M^T M X = |M X|^2=0$. In other words the vector $MX$ would have zero length, so could only be the zero vector. But we are assuming that $\ker L=\{0\}$ so $MX=0$ implies $X=0$. Thus the kernel of $M^TM$ is $\{0\}$ so this matrix is invertible.
So, in this case, the least squares solution (the $X$ that solves $M^TMX=MV$) is unique, and is equal to 
\[
X = (M^TM)^{-1}M^TV.
\]
In a nutshell, this is the least squares method:

\begin{itemize}
\item Compute $M^TM$ and $M^TV$.
\item Solve $(M^TM)X=M^TV$ by Gaussian elimination.
\end{itemize}


\begin{example}
Captain Conundrum\index{Captain Conundrum} falls off of the leaning tower of Pisa and makes three (rather shaky) measurements of his velocity at three different times.

\begin{center}
\begin{tabular}{c|c}
$t$ s & $v $ m/s \\ \hline
$1$ & $11$ \\
$2$ & $19$ \\
$3$ & $31$
\end{tabular}
\end{center}

Having taken some calculus\footnote{In fact, he is a \emph{Calculus Superhero}\index{Calculus Superhero}.}, he believes that his data are best approximated by a straight line
\[
v = at+b.
\]
Then he should find $a$ and $b$ to best fit the data.
\begin{eqnarray*}
11 &=& a\cdot 1 + b \\
19 &=& a\cdot 2 + b \\
31 &=& a\cdot 3 + b.
\end{eqnarray*}
As a system of linear equations, this becomes:

\[
\begin{pmatrix}
1 & 1 \\
2 & 1 \\
3 & 1 \\
\end{pmatrix}
\colvec{a\\b} \stackrel{?}{=}
\colvec{11\\19\\31}.
\]
There is likely no actual straight line solution, so instead solve $M^TMX=M^TV$.

\[
\begin{pmatrix}
1 & 2 & 3 \\
1 & 1 & 1 \\
\end{pmatrix}
\begin{pmatrix}
1 & 1 \\
2 & 1 \\
3 & 1 \\
\end{pmatrix} \colvec{a\\b}
= 
\begin{pmatrix}
1 & 2 & 3 \\
1 & 1 & 1 \\
\end{pmatrix}
\colvec{11\\19\\31}.
\]
This simplifies to 

\[
\begin{amatrix}{2}
14 & 6 & 142 \\
6 & 3 & 61
\end{amatrix}
\sim
\begin{amatrix}{2}
1 & 0 & 10 \\
0 & 1 & \frac{1}{3}
\end{amatrix}.
\]
Thus, the least-squares fit is the line

\[
v = 10\ t + \frac{1}{3}\, .
\]
Notice that this equation implies that Captain Conundrum accelerates towards Italian soil at 10 m/s$^2$ (which is an excellent
approximation to reality) and that he started at a downward velocity of $\frac13$ m/s (perhaps somebody gave him a shove...)!

\end{example}

\section{Projection Matrices}
We have seen that even if $MX=V$ has no solutions $M^TMX=M^T V$ does have solutions. One way to think about this is, since the codomain of $M$ is the direct sum 
$$ \text{codom M}=\text{ran} M \oplus \ker M^T$$ 
there is a unique way to write  $V=V_r+V_k$ with $V_k\in \ker M^T$ and $V_r\in \text{ran }\, M$, and it is clear that $Mx=V$ only has a solution of 
$V\in \text{ran}\, M \Leftrightarrow V_k=0$. If not, then the closest thing to a solution of $MX=V$ is a solution to $MX=V_r$. We learned to find solutions to this in the previous subsection of this book. 

But here is another question, how can we determine what $V_r$ is given $M$ and $V$? The answer is simple; suppose $X$ is a solution to $MX=V_r$. Then
$$  MX=V_r 
\implies M^TMx=M^T V_r 
\implies M^TMx=M^T (V_r + 0) $$ 
$$
\implies M^TMx=M^T (V_r+V_k)
\implies M^TMx=M^T V 
\implies X=(M^TM)^{-1} M^T V 
$$
if indeed $M^TM$ is invertible. Since, by assumption, $X$ is a solution \\
\begin{center}
\shabox{ $M(M^TM)^{-1} M^T\, V =V_r. $}
\end{center}
That is, the matrix which projects $V$ onto its $\text{ran} \, M$ part is $M(M^TM)^{-1} M^T$. 

\begin{example} To project $\colvec{1\\1\\1}$ onto $\spa \left\{    \colvec{ 1\\1\\0}, \colvec{1\\-1\\0 }  \right\} = \text{ran} 
\begin{pmatrix}
 1& 1  \\
1 & -1  \\
0 & 0 
\end{pmatrix}
 $ 
  multiply by the matrix 
 $$
\begin{pmatrix}
 1& 1  \\
1 & -1  \\
0 & 0 
\end{pmatrix}
\left [ 
 \begin{pmatrix}
 1& 1 &0 \\
1 & -1 &0 
\end{pmatrix}
\begin{pmatrix}
 1& 1  \\
1 & -1  \\
0 & 0 
\end{pmatrix}
 \right]^{-1}
 \begin{pmatrix}
 1& 1 &0 \\
1 & -1 &0 
\end{pmatrix}
 $$
 $$
=\begin{pmatrix}
 1& 1  \\
1 & -1  \\
0 & 0 
\end{pmatrix}
 \begin{pmatrix}
 2& 0  \\
0 & 2  
\end{pmatrix}^{-1}
 \begin{pmatrix}
 1& 1 &0 \\
1 & -1 &0 
\end{pmatrix} 
$$
$$
=\frac12 \begin{pmatrix}
 1& 1  \\
1 & -1  \\
0 & 0 
\end{pmatrix}
 \begin{pmatrix}
 1& 1 &0 \\
1 & -1 &0 
\end{pmatrix} 
=
\frac12 \begin{pmatrix}
 2 & 0 &0 \\
0 & 2  &0\\
0 & 0 &0
\end{pmatrix}. 
$$

This gives 
$$\frac12 \begin{pmatrix}
 2 & 0 &0 \\
0 & 2  &0\\
0 & 0 &0
\end{pmatrix}
\colvec{1\\1\\1 } = \colvec{1\\1\\0} .$$
\end{example}



\section{Singular Value Decomposition}

Suppose 
$$
L:V\tolinear W\, .
$$
It is unlikely that $\dim V=:n=m:=\dim W$ so a $m\times n$ matrix $M$ of $L$ in bases for $V$ and $W$ will not be square.
Therefore there is no eigenvalue problem  we can use to uncover a preferred basis. However, if the vector spaces $V$ and 
$W$ both have inner products, there does exist an analog of the eigenvalue problem, namely the singular values of $L$.

Before giving the details of the powerful technique known as the singular value decomposition, we note that it is an 
excellent example of what Eugene Wigner called the ``Unreasonable Effectiveness of Mathematics'':
\begin{quote}{\scriptsize
There is a story about two friends who were classmates in high school, talking about their jobs. One of them became a statistician
and was working on population trends. He showed a reprint to his former classmate.
The reprint started, as usual with the Gaussian distribution and the statistician explained
to his former classmate the meaning of the symbols for the actual population and so on. His classmate
was a bit incredulous and was not quite sure whether the statistician was pulling his leg. ``How can you 
know that?'' was his query. ``And what is this symbol here?'' ``Oh,'' said the statistician, this is ``$\pi$.''
``And what is that?'' ``The ratio of the circumference of the circle to its diameter.'' ``Well, now
you are pushing your joke too far,'' said the classmate, ``surely the population has nothing to do with the 
circumference of the circle.''


Eugene Wigner, Commun. Pure and Appl. Math. {\bf XIII}, 1 (1960).
}
\end{quote}
Whenever we mathematically model a system, any ``canonical quantities'' 
(those that  %on which we can all agree and 
do not
depend on any choices we make for calculating them) will correspond to important features of the system. For examples, the eigenvalues
of the eigenvector equation you found in review question~\ref{stringval}, chapter~\ref{eigenvalseigenvects} encode the notes and harmonics that a guitar string can play! 

Singular values appear in many linear algebra applications, especially those involving very large data sets such as statistics and signal processing. 

Let us focus on the $m\times n$ matrix $M$ of a linear transformation $L:V\to W$ written in orthonormal bases for the input and outputs of $L$ (notice, the existence of these othonormal bases is predicated on having inner products for $V$ and $W$).
Even though the matrix $M$ is not square, both the matrices $M M^T$ and $M^T M$ are square and symmetric! 
In terms of linear transformations $M^T$ is the matrix of a linear transformation 
$$
L^*:W\tolinear V\, .
$$
Thus $LL^*:W\to W$ and $L^*L:V\to V$ and both have eigenvalue problems.
Moreover,  as is shown  in Chapter~\ref{symmetricmatrices},  both $L^*L$ and $LL^*$ have orthonormal bases of eigenvectors, and
 both $MM^T$ and $M^TM$ can be diagonalized. 
 
Next, let us make a simplifying assumption, namely $\ker L=\{0\}$. This is not necessary, but will make some of our computations simpler.
Now suppose we have found an orthonormal basis $(u_1,\ldots , u_n)$ for $V$ composed of eigenvectors for $L^*L$. That is 
$$
L^*L u_i= \lambda_i u_i\, .
$$
Then multiplying by $L$ gives 
$$
L L^* L u_i = \lambda_i L u_i\, .
$$
{\it I.e.}, $L u_i$ is an eigenvector of $L L^*$.
The vectors $(Lu_1,\ldots, Lu_n)$ are linearly independent, because $\ker L=\{0\}$ (this is where we use our simplifying assumption, but you can 
try and extend our analysis to the case where it no longer holds). 

Lets compute the angles between and lengths of these vectors. 
For that we express the vectors $u_i$ in the bases used to compute the matrix $M$ of $L$. Denoting these column vectors by $U_i$ we then compute
$$
(MU_i)\cdot (MU_j)=U_i^T M^T M U_j = \lambda_j \, U_i^T U_j=\lambda_j \, U_i\cdot U_j = \lambda_j \delta_{ij}\, .
$$
We see that  vectors $(Lu_1,\ldots, Lu_n)$ are orthogonal but not orthonormal. Moreover, the length of $Lu_i$ is $\sqrt{\lambda_i}$.
Normalizing gives the orthonormal and linearly independent ordered set
$$
\left(\frac{Lu_1}{\sqrt{\lambda_1}},\ldots,\frac{Lu_n}{\sqrt{\lambda_n}}\right).
$$

In general, this cannot be a basis for $W$ 
since $\ker L=\{0\},~\dim L(V)=\dim V,$
and in turn $\dim V\leq \dim W$, so $n\leq m$. 

However,  it is a subset of the eigenvectors of $LL^*$ so there is an orthonormal basis of eigenvectors of $LL^*$ of the form 
$$
O'=\left(\frac{Lu_1}{\sqrt{\lambda_1}},\ldots,\frac{Lu_n}{\sqrt{\lambda_n}},v_{n+1},\ldots,v_{m}\right)=:(v_1,\ldots,v_m)\, .
$$
Now lets compute the matrix of $L$ with respect to the orthonormal basis $O=(u_1,\ldots,u_n)$ for $V$ and the orthonormal basis~$O'=(v_1,\ldots,v_m)$ for~$W$. As usual, our starting point is the computation of $L$ acting on the input basis vectors;
\begin{eqnarray*}
LO=\big(Lu_1,\ldots, Lu_n\big)&=&
\big(\sqrt{\lambda_1}\,  v_1,\ldots,\sqrt{\lambda_n}\,  v_n\big)\\[2mm]&=&\big(v_1,\ldots,v_m\big)
%\begin{pmatrix}
%\sqrt{\lambda_1}&\mc0&\cdots&\mc0&0&\cdots&0\\[1mm]
%\mc0&\sqrt{\lambda_2}&\cdots&\mc0&0&\cdots&0\\
%\mc{\vdots}&\mc\vdots&\ddots&\mc\vdots&\mc\vdots&&\mc\vdots\\[1mm]
%\mc0&\mc0&\cdots&\sqrt{\lambda_n}&0&\cdots&0\\
%\end{pmatrix}
\begin{pmatrix}
\sqrt{\lambda_1}&\mc0&\cdots&\mc0\\[1mm]
\mc0&\sqrt{\lambda_2}&\cdots&\mc0\\
\mc{\vdots}&\mc\vdots&\ddots&\mc\vdots\\[1mm]
\mc0&\mc0&\cdots&\sqrt{\lambda_n}\\[1mm]
\mc 0 & \mc 0& \cdots &\mc 0\\
\mc{\vdots}&\mc\vdots&&\mc\vdots\\
\mc 0 & \mc 0& \cdots &\mc 0
\end{pmatrix}\, .
\end{eqnarray*}
The result is very close to diagonalization; the numbers $\sqrt{\lambda_i}$ along the leading diagonal are called the singular values of $L$.

\begin{example} Let the matrix of a linear transformation be
$$
M=\begin{pmatrix}
\frac12&\frac12\\[1mm]-1&1\\[1mm]-\frac12&-\frac12
\end{pmatrix}\, .
$$
Clearly $\ker M=\{0\}$ while
$$
M^TM=\begin{pmatrix}\frac32&-\frac12\\[2mm]-\frac12&\frac32\end{pmatrix}
$$
which has eigenvalues and eigenvectors
$$
 \lambda=1\, ,\,  u_1:=\colvec{\frac{1}{\sqrt2}\\[2mm]\frac{1}{\sqrt2}}; \qquad
\lambda=2\, ,\,  u_2:=\colvec{\frac{1}{\sqrt2}\\[2mm]-\frac{1}{\sqrt2}}\,\, .
$$
so our orthonormal input basis is $$O=\left(\colvec{\frac{1}{\sqrt2}\\[2mm]\frac{1}{\sqrt2}},\colvec{\frac{1}{\sqrt2}\\[2mm]-\frac{1}{\sqrt2}}\right)\, .
$$
These are called the {\it right singular vectors}\index{Right singular vector} of $M$.
The vectors 
$$
M u_1= \colvec{\frac1{\sqrt{2}}\\[1mm]\mc{\ \ 0}\\-\frac1{\sqrt{2}}}\mbox{ and }
M u_2=\ccolvec{0\\[1mm]-\sqrt{2}\\[1mm]0}
$$
are eigenvectors of 
$$M M^T=\begin{pmatrix}\frac12&\ 0&\!-\frac12\\0&2&0\\-\frac12&0&\frac12\end{pmatrix}$$ 
with eigenvalues $1$ and $2$, respectively. The third eigenvector (with eigenvalue~$0$) of $MM^T$ is 
$$v_3=\colvec{\frac1{\sqrt{2}}\\[1mm]\mc{\ \ 0}\\ \frac1{\sqrt{2}}}\, .$$
The eigenvectors $Mu_1$ and $Mu_2$ are necessarily orthogonal, dividing them by their lengths we obtain the {\it left singular vectors}\index{Left singular vectors} and in turn  our orthonormal output basis
$$
O'=\left(\colvec{\frac1{\sqrt{2}}\\[1mm]\mc{\ \ 0}\\-\frac1{\sqrt{2}}},\ccolvec{0\\[1mm]-1\\[1mm]0},\colvec{\frac1{\sqrt{2}}\\[1mm]\mc{\ \ 0}\\\frac1{\sqrt{2}}}\right)\, .
$$
The new matrix~$M'$ of the linear transformation given by $M$ with respect to the bases $O$ and $O'$ is
$$
M'=\begin{pmatrix}
1&0\\0&\sqrt{2}\\0&0
\end{pmatrix}\, ,
$$
so the singular values are $1,\sqrt{2}$. 

Finally note that arranging the column vectors of $O$ and $O'$ into change of basis matrices
$$
P=\begin{pmatrix}
\frac1{\sqrt{2}}&\frac1{\sqrt{2}}\\[2mm]
\frac1{\sqrt{2}}&-\frac1{\sqrt{2}}
\end{pmatrix}\, ,\qquad
Q=
\begin{pmatrix}
\frac1{\sqrt{2}}&0&\frac1{\sqrt{2}}\\[2mm]
\mc {\ \ 0}&-1&\mc 0\\[2mm]
\!-\frac1{\sqrt{2}}&0&\frac1{\sqrt{2}}
\end{pmatrix}\, ,
$$
we have, as usual,
$$
M'=Q^{-1}MP\, .
$$
\end{example}

Singular vectors and values have a very nice geometric interpretation; they provide an orthonormal bases for the domain and range of $L$
and give the factors by which $L$ stretches the orthonormal input basis vectors. This is depicted below for the example we just computed.
\begin{center}
\includegraphics[scale=.27]{singval.jpg}
\end{center} 



%{\it Congratulations, you have reached the end of these notes! You can test your skills
%on the \hyperref[sample3]{sample final exam}.}
\begin{center}
\shabox{
{\bf \hyperref[sample3]{\begin{tabular}{c}Congratulations, you have reached the end of the book! \\[2mm]
\includegraphics[scale=.15]{final.jpg}\\
Now test your skills on the \hyperref[sample3]{sample final exam}.
%You are now ready to 
%apply for membership in\\
% be a minion of Captain Conundrum's nemesis, 
%The League of Ninjas of Numbers. 
%Now test your skills
%on the sample final exam. 
\end{tabular}
}}}
\end{center}













%\section*{References}
%Hefferon, Chapter Three, Section VI.2: Gram-Schmidt Orthogonalization \\
%Beezer, Part A, Section CF, Subsection DF \\
%Wikipedia:
%\begin{itemize}
%\item \href{http://en.wikipedia.org/wiki/Linear_least_squares}{Linear Least Squares}
%\item \href{http://en.wikipedia.org/wiki/Least_squares}{Least Squares}
%\end{itemize}

\section{Review Problems}

{\bf Webwork:} 
\begin{tabular}{|c|c|}
\hline
Reading Problem & 
 \hwrref{LeastSquares}{1}, 
\\
   \hline
\end{tabular}





\begin{enumerate}

\item While performing  Gaussian elimination on these augmented matrices write the full system of equations describing the new rows in terms of the old rows above each equivalence symbol as in  \hyperlink{Keeping track of EROs with equations between rows}{Example}~\ref{Rsystem}. 
$$
\begin{amatrix}{2} 
2 & 2 & 10 \\
1 & 2 & 8 \\
\end{amatrix}
,~
\begin{amatrix}{3} 
1 & 1 & 0 & 5 \\
1 & 1 & \!\!-1& 11 \\
-1 & 1 & 1 & -5 \\ 
\end{amatrix}
$$

%%%%%%%%%%%%%%%%%%%

\item Solve the vector equation by applying ERO matrices to each side of the equation to perform elimination. Show each matrix explicitly as in \hyperlink{Undoing}{Example~\ref{slowly}}.

\begin{eqnarray*}
\begin{pmatrix}
3	&6 	&2 \\ %-3
5 	&9 	&4 \\ %1
2	&4	&2 \\ %0
\end{pmatrix} 
\begin{pmatrix}
 x \\ 
y \\
z 
\end{pmatrix} 
=
\begin{pmatrix}
-3 \\ 
1  \\
0  \\
\end{pmatrix} 
\end{eqnarray*}

%%%%%%%%%%%%%%%%%%%

\item Solve this vector equation by finding the inverse of the matrix through $(M|I)\sim (I|M^{-1})$ and then applying $M^{-1}$ to both sides of the equation. 
\begin{eqnarray*}
\begin{pmatrix}
2	&1 	&1 \\ %9
1 	&1 	&1 \\ %6
1	&1	&2 \\ %7
\end{pmatrix} 
\begin{pmatrix}
 x \\ 
y \\
z 
\end{pmatrix} 
=
\begin{pmatrix}
9 \\ 
6  \\
7  \\
\end{pmatrix} 
\end{eqnarray*}


%%%%%%%%%%%%%%%%%%%

\item Follow the method of  \hyperlink{elldeeeww}{Examples~\ref{factorize} and~\ref{factorizes}} to find the $LU$ and $LDU$ factorization of 
\begin{eqnarray*}
\begin{pmatrix}
3	&3 	&6 \\ %0 %2
3 	&5 	&2 \\ %1 %1
6	&2	&5 \\ %0 %1
\end{pmatrix} .
\end{eqnarray*}



%%%%%%%%%%%%%%%%%%%%

\item 
Multiple matrix equations with the same matrix can be solved simultaneously. 
\begin{enumerate}
\item Solve both systems by performing elimination on just one augmented matrix.
\begin{eqnarray*}
\begin{pmatrix}
2	&-1 	&-1 \\ %0 %2
-1 	&1 	&1 \\ %1 %1
1	&-1	&0 \\ %0 %1
\end{pmatrix} 
\begin{pmatrix}
 x \\ 
y \\
z 
\end{pmatrix} 
=
\begin{pmatrix}
0\\ 
1  \\
0  \\
\end{pmatrix} 
,~
\begin{pmatrix}
2	&-1 	&-1 \\ %0 %2
-1 	&1 	&1 \\ %1 %1
1	&-1	&0 \\ %0 %1
\end{pmatrix} 
\begin{pmatrix}
 a \\ 
b \\
c 
\end{pmatrix} 
=
\begin{pmatrix}
2\\ 
1  \\
1  \\
\end{pmatrix} 
\end{eqnarray*}
\item Give an interpretation of the columns of $M^{-1}$ in $(M|I)\sim (I|M^{-1})$ in terms of solutions to certain systems of linear equations.
\end{enumerate}

%%%%%%%%%%%%%%%%%%%%%%%%

\item How can you convince your fellow students to never make this mistake?
\begin{eqnarray*}
\begin{amatrix}{3} 
1 & 0 & 2 & 3 \\ 
0 & 1 & 2& 3 \\
2 & 0 & 1 & 4 \\
\end{amatrix} 
& 
\stackrel{R_1'=R_1+R_2}{
\stackrel{R_2'=R_1-R_2}{ 
\stackrel{\ R_3'= R_1+2R_2}{\sim}}}
&
\begin{amatrix}{3} 
1 & 1 & 4 & 6 \\
1 & \!\!-1 & 0& 0 \\
1 & 2 & 6 & 9 
\end{amatrix}
\end{eqnarray*}

\item Is $LU$ factorization of a matrix unique?  Justify your answer.


\item[$\infty$.] If you randomly create a matrix by picking numbers out of the blue, it will probably be difficult to perform elimination or factorization; fractions and large numbers will probably be involved. To invent simple problems it is better to start with a simple answer:
\begin{enumerate}
\item Start with any augmented matrix in RREF. Perform EROs to make most of the components non-zero. Write the result on a separate piece of paper and give it to your friend. Ask that friend to find RREF of the augmented matrix you gave them. Make sure they get the same augmented matrix you started with.  
\item Create  an upper triangular matrix $U$ and a lower triangular matrix~$L$ with only $1$s on the diagonal. Give the result to a friend to factor into $LU$ form. 
\item Do the same with an $LDU$ factorization. 
\end{enumerate}
\end{enumerate}

\phantomnewpage




\newpage



\chapter{Least squares and Singular Values}
\label{sec:leastsquaresSVD}
\index{Least squares}

Consider the linear algebraic equation $L(x)=v$, where $L \colon U\stackrel{\text{linear}}{-\!\!\!-\!\!\!\longrightarrow}W$ and $v\in W$ are known while $x$ is unknown. As we have seen, this system may have 
one solution, no solutions, or infinitely many solutions.  
But if $v$ is not in the range of $L$ there will {\it never} be any solutions for $L(x)=v$.
\vspace{-.1cm}
\begin{center}
\includegraphics[scale=.24]{notinimage.jpg}
\end{center} 
\vspace{-1.8cm}
However, for many applications we do not need an exact solution of the system; instead, we may only need the best approximation possible.  

\begin{quote}
``My work always tried to unite the Truth with the Beautiful, but when I had to choose one or the other, I usually chose the Beautiful.'' 

\vspace{-2mm}
\hspace{7cm}-- Hermann Weyl.
\end{quote}

If the vector space $W$ has a notion of lengths of vectors, we can try to find $x$ that minimizes $||L(x)-v||$.
\begin{center}
\includegraphics[scale=.24]{minimize.jpg}
\end{center} 
This method has many applications, such as when trying to fit a (perhaps linear) function to a ``noisy'' set of observations.  For example, suppose we measured the position of a bicycle on a racetrack once every five seconds.  Our observations won't be exact, but so long as the observations are right on average, we can figure out a best-possible linear function of position of the bicycle in terms of time.

Suppose $M$ is the matrix for the linear function $L:U \to W$ in some bases for $U$ and $W$. The vectors~$v$ and~$x$ are represented by column vectors $V$ and $X$ in these bases.  Then we need to approximate
\[
MX-V\approx 0\, .
\]

Note that if $\dim U=n$ and $\dim W=m$ then $M$ can be represented by an $m\times n$ matrix and $x$ and $v$ as vectors in $\Re^n$ and $\Re^m$, respectively. Thus, we can write $W=L(U)\oplus L(U)^\perp$.  Then we can uniquely write $v=v^\parallel + v^\perp$, with $v^\parallel \in L(U)$ and $v^\perp \in L(U)^\perp$.  



Thus we should solve $L(u)=v^\parallel$.  In components, $v^\perp$ is just $V-MX$, and is the part we will eventually wish to minimize.  

In terms of $M$, recall that $L(V)$ is spanned by the columns of $M$.  (In the standard basis, the columns of $M$ are $Me_1$, 
$\ldots$, $Me_n$.)  Then $v^\perp$ must be perpendicular to the columns of $M$.  \textit{i.e.}, $M^T(V-MX)=0$, or
\[
M^TMX = M^TV.
\]
Solutions of $M^TMX = M^TV$ for $X$ are called \emph{least squares}\index{Least squares!solutions} solutions to $MX=V$.  
Notice that any solution $X$ to $MX=V$ is a least squares solution.  However, the converse is often false.  In fact, the equation $MX=V$ may have no solutions at all, but still have least squares solutions to $M^TMX = M^TV$.

Observe that since $M$ is an $m\times n$ matrix, then $M^T$ is an $n\times m$ matrix.  Then $M^TM$ is an $n\times n$ matrix, and is symmetric, since $(M^TM)^T=M^TM$.  Then, for any vector $X$, we can evaluate $X^TM^TMX$ to obtain a number.  This is a very nice number, though!  It is just the length $|MX|^2 = (MX)^T(MX)=X^TM^TMX$.

%\href{\webworkurl ReadingHomework25/1/}{Reading homework: problem 25.1}
\Reading{LeastSquares}{1}

Now suppose that $\ker L=\{0\}$, so that the only solution to $MX=0$ is $X=0$. (This need not mean that $M$ is invertible because $M$ is an $n\times m $ matrix, so not necessarily square.) 
However the square matrix $M^TM$ {\it is} invertible. To see this, suppose there was a vector $X$ such that 
$M^T M X=0$. Then it would follow that $X^T M^T M X = |M X|^2=0$. In other words the vector $MX$ would have zero length, so could only be the zero vector. But we are assuming that $\ker L=\{0\}$ so $MX=0$ implies $X=0$. Thus the kernel of $M^TM$ is $\{0\}$ so this matrix is invertible.
So, in this case, the least squares solution (the $X$ that solves $M^TMX=MV$) is unique, and is equal to 
\[
X = (M^TM)^{-1}M^TV.
\]
In a nutshell, this is the least squares method:

\begin{itemize}
\item Compute $M^TM$ and $M^TV$.
\item Solve $(M^TM)X=M^TV$ by Gaussian elimination.
\end{itemize}


\begin{example}
Captain Conundrum\index{Captain Conundrum} falls off of the leaning tower of Pisa and makes three (rather shaky) measurements of his velocity at three different times.

\begin{center}
\begin{tabular}{c|c}
$t$ s & $v $ m/s \\ \hline
$1$ & $11$ \\
$2$ & $19$ \\
$3$ & $31$
\end{tabular}
\end{center}

Having taken some calculus\footnote{In fact, he is a \emph{Calculus Superhero}\index{Calculus Superhero}.}, he believes that his data are best approximated by a straight line
\[
v = at+b.
\]
Then he should find $a$ and $b$ to best fit the data.
\begin{eqnarray*}
11 &=& a\cdot 1 + b \\
19 &=& a\cdot 2 + b \\
31 &=& a\cdot 3 + b.
\end{eqnarray*}
As a system of linear equations, this becomes:

\[
\begin{pmatrix}
1 & 1 \\
2 & 1 \\
3 & 1 \\
\end{pmatrix}
\colvec{a\\b} \stackrel{?}{=}
\colvec{11\\19\\31}.
\]
There is likely no actual straight line solution, so instead solve $M^TMX=M^TV$.

\[
\begin{pmatrix}
1 & 2 & 3 \\
1 & 1 & 1 \\
\end{pmatrix}
\begin{pmatrix}
1 & 1 \\
2 & 1 \\
3 & 1 \\
\end{pmatrix} \colvec{a\\b}
= 
\begin{pmatrix}
1 & 2 & 3 \\
1 & 1 & 1 \\
\end{pmatrix}
\colvec{11\\19\\31}.
\]
This simplifies to 

\[
\begin{amatrix}{2}
14 & 6 & 142 \\
6 & 3 & 61
\end{amatrix}
\sim
\begin{amatrix}{2}
1 & 0 & 10 \\
0 & 1 & \frac{1}{3}
\end{amatrix}.
\]
Thus, the least-squares fit is the line

\[
v = 10\ t + \frac{1}{3}\, .
\]
Notice that this equation implies that Captain Conundrum accelerates towards Italian soil at 10 m/s$^2$ (which is an excellent
approximation to reality) and that he started at a downward velocity of $\frac13$ m/s (perhaps somebody gave him a shove...)!

\end{example}

\section{Projection Matrices}
We have seen that even if $MX=V$ has no solutions $M^TMX=M^T V$ does have solutions. One way to think about this is, since the codomain of $M$ is the direct sum 
$$ \text{codom M}=\text{ran} M \oplus \ker M^T$$ 
there is a unique way to write  $V=V_r+V_k$ with $V_k\in \ker M^T$ and $V_r\in \text{ran }\, M$, and it is clear that $Mx=V$ only has a solution of 
$V\in \text{ran}\, M \Leftrightarrow V_k=0$. If not, then the closest thing to a solution of $MX=V$ is a solution to $MX=V_r$. We learned to find solutions to this in the previous subsection of this book. 

But here is another question, how can we determine what $V_r$ is given $M$ and $V$? The answer is simple; suppose $X$ is a solution to $MX=V_r$. Then
$$  MX=V_r 
\implies M^TMx=M^T V_r 
\implies M^TMx=M^T (V_r + 0) $$ 
$$
\implies M^TMx=M^T (V_r+V_k)
\implies M^TMx=M^T V 
\implies X=(M^TM)^{-1} M^T V 
$$
if indeed $M^TM$ is invertible. Since, by assumption, $X$ is a solution \\
\begin{center}
\shabox{ $M(M^TM)^{-1} M^T\, V =V_r. $}
\end{center}
That is, the matrix which projects $V$ onto its $\text{ran} \, M$ part is $M(M^TM)^{-1} M^T$. 

\begin{example} To project $\colvec{1\\1\\1}$ onto $\spa \left\{    \colvec{ 1\\1\\0}, \colvec{1\\-1\\0 }  \right\} = \text{ran} 
\begin{pmatrix}
 1& 1  \\
1 & -1  \\
0 & 0 
\end{pmatrix}
 $ 
  multiply by the matrix 
 $$
\begin{pmatrix}
 1& 1  \\
1 & -1  \\
0 & 0 
\end{pmatrix}
\left [ 
 \begin{pmatrix}
 1& 1 &0 \\
1 & -1 &0 
\end{pmatrix}
\begin{pmatrix}
 1& 1  \\
1 & -1  \\
0 & 0 
\end{pmatrix}
 \right]^{-1}
 \begin{pmatrix}
 1& 1 &0 \\
1 & -1 &0 
\end{pmatrix}
 $$
 $$
=\begin{pmatrix}
 1& 1  \\
1 & -1  \\
0 & 0 
\end{pmatrix}
 \begin{pmatrix}
 2& 0  \\
0 & 2  
\end{pmatrix}^{-1}
 \begin{pmatrix}
 1& 1 &0 \\
1 & -1 &0 
\end{pmatrix} 
$$
$$
=\frac12 \begin{pmatrix}
 1& 1  \\
1 & -1  \\
0 & 0 
\end{pmatrix}
 \begin{pmatrix}
 1& 1 &0 \\
1 & -1 &0 
\end{pmatrix} 
=
\frac12 \begin{pmatrix}
 2 & 0 &0 \\
0 & 2  &0\\
0 & 0 &0
\end{pmatrix}. 
$$

This gives 
$$\frac12 \begin{pmatrix}
 2 & 0 &0 \\
0 & 2  &0\\
0 & 0 &0
\end{pmatrix}
\colvec{1\\1\\1 } = \colvec{1\\1\\0} .$$
\end{example}



\section{Singular Value Decomposition}

Suppose 
$$
L:V\tolinear W\, .
$$
It is unlikely that $\dim V=:n=m:=\dim W$ so a $m\times n$ matrix $M$ of $L$ in bases for $V$ and $W$ will not be square.
Therefore there is no eigenvalue problem  we can use to uncover a preferred basis. However, if the vector spaces $V$ and 
$W$ both have inner products, there does exist an analog of the eigenvalue problem, namely the singular values of $L$.

Before giving the details of the powerful technique known as the singular value decomposition, we note that it is an 
excellent example of what Eugene Wigner called the ``Unreasonable Effectiveness of Mathematics'':
\begin{quote}{\scriptsize
There is a story about two friends who were classmates in high school, talking about their jobs. One of them became a statistician
and was working on population trends. He showed a reprint to his former classmate.
The reprint started, as usual with the Gaussian distribution and the statistician explained
to his former classmate the meaning of the symbols for the actual population and so on. His classmate
was a bit incredulous and was not quite sure whether the statistician was pulling his leg. ``How can you 
know that?'' was his query. ``And what is this symbol here?'' ``Oh,'' said the statistician, this is ``$\pi$.''
``And what is that?'' ``The ratio of the circumference of the circle to its diameter.'' ``Well, now
you are pushing your joke too far,'' said the classmate, ``surely the population has nothing to do with the 
circumference of the circle.''


Eugene Wigner, Commun. Pure and Appl. Math. {\bf XIII}, 1 (1960).
}
\end{quote}
Whenever we mathematically model a system, any ``canonical quantities'' 
(those that  %on which we can all agree and 
do not
depend on any choices we make for calculating them) will correspond to important features of the system. For examples, the eigenvalues
of the eigenvector equation you found in review question~\ref{stringval}, chapter~\ref{eigenvalseigenvects} encode the notes and harmonics that a guitar string can play! 

Singular values appear in many linear algebra applications, especially those involving very large data sets such as statistics and signal processing. 

Let us focus on the $m\times n$ matrix $M$ of a linear transformation $L:V\to W$ written in orthonormal bases for the input and outputs of $L$ (notice, the existence of these othonormal bases is predicated on having inner products for $V$ and $W$).
Even though the matrix $M$ is not square, both the matrices $M M^T$ and $M^T M$ are square and symmetric! 
In terms of linear transformations $M^T$ is the matrix of a linear transformation 
$$
L^*:W\tolinear V\, .
$$
Thus $LL^*:W\to W$ and $L^*L:V\to V$ and both have eigenvalue problems.
Moreover,  as is shown  in Chapter~\ref{symmetricmatrices},  both $L^*L$ and $LL^*$ have orthonormal bases of eigenvectors, and
 both $MM^T$ and $M^TM$ can be diagonalized. 
 
Next, let us make a simplifying assumption, namely $\ker L=\{0\}$. This is not necessary, but will make some of our computations simpler.
Now suppose we have found an orthonormal basis $(u_1,\ldots , u_n)$ for $V$ composed of eigenvectors for $L^*L$. That is 
$$
L^*L u_i= \lambda_i u_i\, .
$$
Then multiplying by $L$ gives 
$$
L L^* L u_i = \lambda_i L u_i\, .
$$
{\it I.e.}, $L u_i$ is an eigenvector of $L L^*$.
The vectors $(Lu_1,\ldots, Lu_n)$ are linearly independent, because $\ker L=\{0\}$ (this is where we use our simplifying assumption, but you can 
try and extend our analysis to the case where it no longer holds). 

Lets compute the angles between and lengths of these vectors. 
For that we express the vectors $u_i$ in the bases used to compute the matrix $M$ of $L$. Denoting these column vectors by $U_i$ we then compute
$$
(MU_i)\cdot (MU_j)=U_i^T M^T M U_j = \lambda_j \, U_i^T U_j=\lambda_j \, U_i\cdot U_j = \lambda_j \delta_{ij}\, .
$$
We see that  vectors $(Lu_1,\ldots, Lu_n)$ are orthogonal but not orthonormal. Moreover, the length of $Lu_i$ is $\sqrt{\lambda_i}$.
Normalizing gives the orthonormal and linearly independent ordered set
$$
\left(\frac{Lu_1}{\sqrt{\lambda_1}},\ldots,\frac{Lu_n}{\sqrt{\lambda_n}}\right).
$$

In general, this cannot be a basis for $W$ 
since $\ker L=\{0\},~\dim L(V)=\dim V,$
and in turn $\dim V\leq \dim W$, so $n\leq m$. 

However,  it is a subset of the eigenvectors of $LL^*$ so there is an orthonormal basis of eigenvectors of $LL^*$ of the form 
$$
O'=\left(\frac{Lu_1}{\sqrt{\lambda_1}},\ldots,\frac{Lu_n}{\sqrt{\lambda_n}},v_{n+1},\ldots,v_{m}\right)=:(v_1,\ldots,v_m)\, .
$$
Now lets compute the matrix of $L$ with respect to the orthonormal basis $O=(u_1,\ldots,u_n)$ for $V$ and the orthonormal basis~$O'=(v_1,\ldots,v_m)$ for~$W$. As usual, our starting point is the computation of $L$ acting on the input basis vectors;
\begin{eqnarray*}
LO=\big(Lu_1,\ldots, Lu_n\big)&=&
\big(\sqrt{\lambda_1}\,  v_1,\ldots,\sqrt{\lambda_n}\,  v_n\big)\\[2mm]&=&\big(v_1,\ldots,v_m\big)
%\begin{pmatrix}
%\sqrt{\lambda_1}&\mc0&\cdots&\mc0&0&\cdots&0\\[1mm]
%\mc0&\sqrt{\lambda_2}&\cdots&\mc0&0&\cdots&0\\
%\mc{\vdots}&\mc\vdots&\ddots&\mc\vdots&\mc\vdots&&\mc\vdots\\[1mm]
%\mc0&\mc0&\cdots&\sqrt{\lambda_n}&0&\cdots&0\\
%\end{pmatrix}
\begin{pmatrix}
\sqrt{\lambda_1}&\mc0&\cdots&\mc0\\[1mm]
\mc0&\sqrt{\lambda_2}&\cdots&\mc0\\
\mc{\vdots}&\mc\vdots&\ddots&\mc\vdots\\[1mm]
\mc0&\mc0&\cdots&\sqrt{\lambda_n}\\[1mm]
\mc 0 & \mc 0& \cdots &\mc 0\\
\mc{\vdots}&\mc\vdots&&\mc\vdots\\
\mc 0 & \mc 0& \cdots &\mc 0
\end{pmatrix}\, .
\end{eqnarray*}
The result is very close to diagonalization; the numbers $\sqrt{\lambda_i}$ along the leading diagonal are called the singular values of $L$.

\begin{example} Let the matrix of a linear transformation be
$$
M=\begin{pmatrix}
\frac12&\frac12\\[1mm]-1&1\\[1mm]-\frac12&-\frac12
\end{pmatrix}\, .
$$
Clearly $\ker M=\{0\}$ while
$$
M^TM=\begin{pmatrix}\frac32&-\frac12\\[2mm]-\frac12&\frac32\end{pmatrix}
$$
which has eigenvalues and eigenvectors
$$
 \lambda=1\, ,\,  u_1:=\colvec{\frac{1}{\sqrt2}\\[2mm]\frac{1}{\sqrt2}}; \qquad
\lambda=2\, ,\,  u_2:=\colvec{\frac{1}{\sqrt2}\\[2mm]-\frac{1}{\sqrt2}}\,\, .
$$
so our orthonormal input basis is $$O=\left(\colvec{\frac{1}{\sqrt2}\\[2mm]\frac{1}{\sqrt2}},\colvec{\frac{1}{\sqrt2}\\[2mm]-\frac{1}{\sqrt2}}\right)\, .
$$
These are called the {\it right singular vectors}\index{Right singular vector} of $M$.
The vectors 
$$
M u_1= \colvec{\frac1{\sqrt{2}}\\[1mm]\mc{\ \ 0}\\-\frac1{\sqrt{2}}}\mbox{ and }
M u_2=\ccolvec{0\\[1mm]-\sqrt{2}\\[1mm]0}
$$
are eigenvectors of 
$$M M^T=\begin{pmatrix}\frac12&\ 0&\!-\frac12\\0&2&0\\-\frac12&0&\frac12\end{pmatrix}$$ 
with eigenvalues $1$ and $2$, respectively. The third eigenvector (with eigenvalue~$0$) of $MM^T$ is 
$$v_3=\colvec{\frac1{\sqrt{2}}\\[1mm]\mc{\ \ 0}\\ \frac1{\sqrt{2}}}\, .$$
The eigenvectors $Mu_1$ and $Mu_2$ are necessarily orthogonal, dividing them by their lengths we obtain the {\it left singular vectors}\index{Left singular vectors} and in turn  our orthonormal output basis
$$
O'=\left(\colvec{\frac1{\sqrt{2}}\\[1mm]\mc{\ \ 0}\\-\frac1{\sqrt{2}}},\ccolvec{0\\[1mm]-1\\[1mm]0},\colvec{\frac1{\sqrt{2}}\\[1mm]\mc{\ \ 0}\\\frac1{\sqrt{2}}}\right)\, .
$$
The new matrix~$M'$ of the linear transformation given by $M$ with respect to the bases $O$ and $O'$ is
$$
M'=\begin{pmatrix}
1&0\\0&\sqrt{2}\\0&0
\end{pmatrix}\, ,
$$
so the singular values are $1,\sqrt{2}$. 

Finally note that arranging the column vectors of $O$ and $O'$ into change of basis matrices
$$
P=\begin{pmatrix}
\frac1{\sqrt{2}}&\frac1{\sqrt{2}}\\[2mm]
\frac1{\sqrt{2}}&-\frac1{\sqrt{2}}
\end{pmatrix}\, ,\qquad
Q=
\begin{pmatrix}
\frac1{\sqrt{2}}&0&\frac1{\sqrt{2}}\\[2mm]
\mc {\ \ 0}&-1&\mc 0\\[2mm]
\!-\frac1{\sqrt{2}}&0&\frac1{\sqrt{2}}
\end{pmatrix}\, ,
$$
we have, as usual,
$$
M'=Q^{-1}MP\, .
$$
\end{example}

Singular vectors and values have a very nice geometric interpretation; they provide an orthonormal bases for the domain and range of $L$
and give the factors by which $L$ stretches the orthonormal input basis vectors. This is depicted below for the example we just computed.
\begin{center}
\includegraphics[scale=.27]{singval.jpg}
\end{center} 



%{\it Congratulations, you have reached the end of these notes! You can test your skills
%on the \hyperref[sample3]{sample final exam}.}
\begin{center}
\shabox{
{\bf \hyperref[sample3]{\begin{tabular}{c}Congratulations, you have reached the end of the book! \\[2mm]
\includegraphics[scale=.15]{final.jpg}\\
Now test your skills on the \hyperref[sample3]{sample final exam}.
%You are now ready to 
%apply for membership in\\
% be a minion of Captain Conundrum's nemesis, 
%The League of Ninjas of Numbers. 
%Now test your skills
%on the sample final exam. 
\end{tabular}
}}}
\end{center}













%\section*{References}
%Hefferon, Chapter Three, Section VI.2: Gram-Schmidt Orthogonalization \\
%Beezer, Part A, Section CF, Subsection DF \\
%Wikipedia:
%\begin{itemize}
%\item \href{http://en.wikipedia.org/wiki/Linear_least_squares}{Linear Least Squares}
%\item \href{http://en.wikipedia.org/wiki/Least_squares}{Least Squares}
%\end{itemize}

\section{Review Problems}

{\bf Webwork:} 
\begin{tabular}{|c|c|}
\hline
Reading Problem & 
 \hwrref{LeastSquares}{1}, 
\\
   \hline
\end{tabular}





\begin{enumerate}

\item While performing  Gaussian elimination on these augmented matrices write the full system of equations describing the new rows in terms of the old rows above each equivalence symbol as in  \hyperlink{Keeping track of EROs with equations between rows}{Example}~\ref{Rsystem}. 
$$
\begin{amatrix}{2} 
2 & 2 & 10 \\
1 & 2 & 8 \\
\end{amatrix}
,~
\begin{amatrix}{3} 
1 & 1 & 0 & 5 \\
1 & 1 & \!\!-1& 11 \\
-1 & 1 & 1 & -5 \\ 
\end{amatrix}
$$

%%%%%%%%%%%%%%%%%%%

\item Solve the vector equation by applying ERO matrices to each side of the equation to perform elimination. Show each matrix explicitly as in \hyperlink{Undoing}{Example~\ref{slowly}}.

\begin{eqnarray*}
\begin{pmatrix}
3	&6 	&2 \\ %-3
5 	&9 	&4 \\ %1
2	&4	&2 \\ %0
\end{pmatrix} 
\begin{pmatrix}
 x \\ 
y \\
z 
\end{pmatrix} 
=
\begin{pmatrix}
-3 \\ 
1  \\
0  \\
\end{pmatrix} 
\end{eqnarray*}

%%%%%%%%%%%%%%%%%%%

\item Solve this vector equation by finding the inverse of the matrix through $(M|I)\sim (I|M^{-1})$ and then applying $M^{-1}$ to both sides of the equation. 
\begin{eqnarray*}
\begin{pmatrix}
2	&1 	&1 \\ %9
1 	&1 	&1 \\ %6
1	&1	&2 \\ %7
\end{pmatrix} 
\begin{pmatrix}
 x \\ 
y \\
z 
\end{pmatrix} 
=
\begin{pmatrix}
9 \\ 
6  \\
7  \\
\end{pmatrix} 
\end{eqnarray*}


%%%%%%%%%%%%%%%%%%%

\item Follow the method of  \hyperlink{elldeeeww}{Examples~\ref{factorize} and~\ref{factorizes}} to find the $LU$ and $LDU$ factorization of 
\begin{eqnarray*}
\begin{pmatrix}
3	&3 	&6 \\ %0 %2
3 	&5 	&2 \\ %1 %1
6	&2	&5 \\ %0 %1
\end{pmatrix} .
\end{eqnarray*}



%%%%%%%%%%%%%%%%%%%%

\item 
Multiple matrix equations with the same matrix can be solved simultaneously. 
\begin{enumerate}
\item Solve both systems by performing elimination on just one augmented matrix.
\begin{eqnarray*}
\begin{pmatrix}
2	&-1 	&-1 \\ %0 %2
-1 	&1 	&1 \\ %1 %1
1	&-1	&0 \\ %0 %1
\end{pmatrix} 
\begin{pmatrix}
 x \\ 
y \\
z 
\end{pmatrix} 
=
\begin{pmatrix}
0\\ 
1  \\
0  \\
\end{pmatrix} 
,~
\begin{pmatrix}
2	&-1 	&-1 \\ %0 %2
-1 	&1 	&1 \\ %1 %1
1	&-1	&0 \\ %0 %1
\end{pmatrix} 
\begin{pmatrix}
 a \\ 
b \\
c 
\end{pmatrix} 
=
\begin{pmatrix}
2\\ 
1  \\
1  \\
\end{pmatrix} 
\end{eqnarray*}
\item Give an interpretation of the columns of $M^{-1}$ in $(M|I)\sim (I|M^{-1})$ in terms of solutions to certain systems of linear equations.
\end{enumerate}

%%%%%%%%%%%%%%%%%%%%%%%%

\item How can you convince your fellow students to never make this mistake?
\begin{eqnarray*}
\begin{amatrix}{3} 
1 & 0 & 2 & 3 \\ 
0 & 1 & 2& 3 \\
2 & 0 & 1 & 4 \\
\end{amatrix} 
& 
\stackrel{R_1'=R_1+R_2}{
\stackrel{R_2'=R_1-R_2}{ 
\stackrel{\ R_3'= R_1+2R_2}{\sim}}}
&
\begin{amatrix}{3} 
1 & 1 & 4 & 6 \\
1 & \!\!-1 & 0& 0 \\
1 & 2 & 6 & 9 
\end{amatrix}
\end{eqnarray*}

\item Is $LU$ factorization of a matrix unique?  Justify your answer.


\item[$\infty$.] If you randomly create a matrix by picking numbers out of the blue, it will probably be difficult to perform elimination or factorization; fractions and large numbers will probably be involved. To invent simple problems it is better to start with a simple answer:
\begin{enumerate}
\item Start with any augmented matrix in RREF. Perform EROs to make most of the components non-zero. Write the result on a separate piece of paper and give it to your friend. Ask that friend to find RREF of the augmented matrix you gave them. Make sure they get the same augmented matrix you started with.  
\item Create  an upper triangular matrix $U$ and a lower triangular matrix~$L$ with only $1$s on the diagonal. Give the result to a friend to factor into $LU$ form. 
\item Do the same with an $LDU$ factorization. 
\end{enumerate}
\end{enumerate}

\phantomnewpage




\newpage



\chapter{Least squares and Singular Values}
\label{sec:leastsquaresSVD}
\index{Least squares}

Consider the linear algebraic equation $L(x)=v$, where $L \colon U\stackrel{\text{linear}}{-\!\!\!-\!\!\!\longrightarrow}W$ and $v\in W$ are known while $x$ is unknown. As we have seen, this system may have 
one solution, no solutions, or infinitely many solutions.  
But if $v$ is not in the range of $L$ there will {\it never} be any solutions for $L(x)=v$.
\vspace{-.1cm}
\begin{center}
\includegraphics[scale=.24]{notinimage.jpg}
\end{center} 
\vspace{-1.8cm}
However, for many applications we do not need an exact solution of the system; instead, we may only need the best approximation possible.  

\begin{quote}
``My work always tried to unite the Truth with the Beautiful, but when I had to choose one or the other, I usually chose the Beautiful.'' 

\vspace{-2mm}
\hspace{7cm}-- Hermann Weyl.
\end{quote}

If the vector space $W$ has a notion of lengths of vectors, we can try to find $x$ that minimizes $||L(x)-v||$.
\begin{center}
\includegraphics[scale=.24]{minimize.jpg}
\end{center} 
This method has many applications, such as when trying to fit a (perhaps linear) function to a ``noisy'' set of observations.  For example, suppose we measured the position of a bicycle on a racetrack once every five seconds.  Our observations won't be exact, but so long as the observations are right on average, we can figure out a best-possible linear function of position of the bicycle in terms of time.

Suppose $M$ is the matrix for the linear function $L:U \to W$ in some bases for $U$ and $W$. The vectors~$v$ and~$x$ are represented by column vectors $V$ and $X$ in these bases.  Then we need to approximate
\[
MX-V\approx 0\, .
\]

Note that if $\dim U=n$ and $\dim W=m$ then $M$ can be represented by an $m\times n$ matrix and $x$ and $v$ as vectors in $\Re^n$ and $\Re^m$, respectively. Thus, we can write $W=L(U)\oplus L(U)^\perp$.  Then we can uniquely write $v=v^\parallel + v^\perp$, with $v^\parallel \in L(U)$ and $v^\perp \in L(U)^\perp$.  



Thus we should solve $L(u)=v^\parallel$.  In components, $v^\perp$ is just $V-MX$, and is the part we will eventually wish to minimize.  

In terms of $M$, recall that $L(V)$ is spanned by the columns of $M$.  (In the standard basis, the columns of $M$ are $Me_1$, 
$\ldots$, $Me_n$.)  Then $v^\perp$ must be perpendicular to the columns of $M$.  \textit{i.e.}, $M^T(V-MX)=0$, or
\[
M^TMX = M^TV.
\]
Solutions of $M^TMX = M^TV$ for $X$ are called \emph{least squares}\index{Least squares!solutions} solutions to $MX=V$.  
Notice that any solution $X$ to $MX=V$ is a least squares solution.  However, the converse is often false.  In fact, the equation $MX=V$ may have no solutions at all, but still have least squares solutions to $M^TMX = M^TV$.

Observe that since $M$ is an $m\times n$ matrix, then $M^T$ is an $n\times m$ matrix.  Then $M^TM$ is an $n\times n$ matrix, and is symmetric, since $(M^TM)^T=M^TM$.  Then, for any vector $X$, we can evaluate $X^TM^TMX$ to obtain a number.  This is a very nice number, though!  It is just the length $|MX|^2 = (MX)^T(MX)=X^TM^TMX$.

%\href{\webworkurl ReadingHomework25/1/}{Reading homework: problem 25.1}
\Reading{LeastSquares}{1}

Now suppose that $\ker L=\{0\}$, so that the only solution to $MX=0$ is $X=0$. (This need not mean that $M$ is invertible because $M$ is an $n\times m $ matrix, so not necessarily square.) 
However the square matrix $M^TM$ {\it is} invertible. To see this, suppose there was a vector $X$ such that 
$M^T M X=0$. Then it would follow that $X^T M^T M X = |M X|^2=0$. In other words the vector $MX$ would have zero length, so could only be the zero vector. But we are assuming that $\ker L=\{0\}$ so $MX=0$ implies $X=0$. Thus the kernel of $M^TM$ is $\{0\}$ so this matrix is invertible.
So, in this case, the least squares solution (the $X$ that solves $M^TMX=MV$) is unique, and is equal to 
\[
X = (M^TM)^{-1}M^TV.
\]
In a nutshell, this is the least squares method:

\begin{itemize}
\item Compute $M^TM$ and $M^TV$.
\item Solve $(M^TM)X=M^TV$ by Gaussian elimination.
\end{itemize}


\begin{example}
Captain Conundrum\index{Captain Conundrum} falls off of the leaning tower of Pisa and makes three (rather shaky) measurements of his velocity at three different times.

\begin{center}
\begin{tabular}{c|c}
$t$ s & $v $ m/s \\ \hline
$1$ & $11$ \\
$2$ & $19$ \\
$3$ & $31$
\end{tabular}
\end{center}

Having taken some calculus\footnote{In fact, he is a \emph{Calculus Superhero}\index{Calculus Superhero}.}, he believes that his data are best approximated by a straight line
\[
v = at+b.
\]
Then he should find $a$ and $b$ to best fit the data.
\begin{eqnarray*}
11 &=& a\cdot 1 + b \\
19 &=& a\cdot 2 + b \\
31 &=& a\cdot 3 + b.
\end{eqnarray*}
As a system of linear equations, this becomes:

\[
\begin{pmatrix}
1 & 1 \\
2 & 1 \\
3 & 1 \\
\end{pmatrix}
\colvec{a\\b} \stackrel{?}{=}
\colvec{11\\19\\31}.
\]
There is likely no actual straight line solution, so instead solve $M^TMX=M^TV$.

\[
\begin{pmatrix}
1 & 2 & 3 \\
1 & 1 & 1 \\
\end{pmatrix}
\begin{pmatrix}
1 & 1 \\
2 & 1 \\
3 & 1 \\
\end{pmatrix} \colvec{a\\b}
= 
\begin{pmatrix}
1 & 2 & 3 \\
1 & 1 & 1 \\
\end{pmatrix}
\colvec{11\\19\\31}.
\]
This simplifies to 

\[
\begin{amatrix}{2}
14 & 6 & 142 \\
6 & 3 & 61
\end{amatrix}
\sim
\begin{amatrix}{2}
1 & 0 & 10 \\
0 & 1 & \frac{1}{3}
\end{amatrix}.
\]
Thus, the least-squares fit is the line

\[
v = 10\ t + \frac{1}{3}\, .
\]
Notice that this equation implies that Captain Conundrum accelerates towards Italian soil at 10 m/s$^2$ (which is an excellent
approximation to reality) and that he started at a downward velocity of $\frac13$ m/s (perhaps somebody gave him a shove...)!

\end{example}

\section{Projection Matrices}
We have seen that even if $MX=V$ has no solutions $M^TMX=M^T V$ does have solutions. One way to think about this is, since the codomain of $M$ is the direct sum 
$$ \text{codom M}=\text{ran} M \oplus \ker M^T$$ 
there is a unique way to write  $V=V_r+V_k$ with $V_k\in \ker M^T$ and $V_r\in \text{ran }\, M$, and it is clear that $Mx=V$ only has a solution of 
$V\in \text{ran}\, M \Leftrightarrow V_k=0$. If not, then the closest thing to a solution of $MX=V$ is a solution to $MX=V_r$. We learned to find solutions to this in the previous subsection of this book. 

But here is another question, how can we determine what $V_r$ is given $M$ and $V$? The answer is simple; suppose $X$ is a solution to $MX=V_r$. Then
$$  MX=V_r 
\implies M^TMx=M^T V_r 
\implies M^TMx=M^T (V_r + 0) $$ 
$$
\implies M^TMx=M^T (V_r+V_k)
\implies M^TMx=M^T V 
\implies X=(M^TM)^{-1} M^T V 
$$
if indeed $M^TM$ is invertible. Since, by assumption, $X$ is a solution \\
\begin{center}
\shabox{ $M(M^TM)^{-1} M^T\, V =V_r. $}
\end{center}
That is, the matrix which projects $V$ onto its $\text{ran} \, M$ part is $M(M^TM)^{-1} M^T$. 

\begin{example} To project $\colvec{1\\1\\1}$ onto $\spa \left\{    \colvec{ 1\\1\\0}, \colvec{1\\-1\\0 }  \right\} = \text{ran} 
\begin{pmatrix}
 1& 1  \\
1 & -1  \\
0 & 0 
\end{pmatrix}
 $ 
  multiply by the matrix 
 $$
\begin{pmatrix}
 1& 1  \\
1 & -1  \\
0 & 0 
\end{pmatrix}
\left [ 
 \begin{pmatrix}
 1& 1 &0 \\
1 & -1 &0 
\end{pmatrix}
\begin{pmatrix}
 1& 1  \\
1 & -1  \\
0 & 0 
\end{pmatrix}
 \right]^{-1}
 \begin{pmatrix}
 1& 1 &0 \\
1 & -1 &0 
\end{pmatrix}
 $$
 $$
=\begin{pmatrix}
 1& 1  \\
1 & -1  \\
0 & 0 
\end{pmatrix}
 \begin{pmatrix}
 2& 0  \\
0 & 2  
\end{pmatrix}^{-1}
 \begin{pmatrix}
 1& 1 &0 \\
1 & -1 &0 
\end{pmatrix} 
$$
$$
=\frac12 \begin{pmatrix}
 1& 1  \\
1 & -1  \\
0 & 0 
\end{pmatrix}
 \begin{pmatrix}
 1& 1 &0 \\
1 & -1 &0 
\end{pmatrix} 
=
\frac12 \begin{pmatrix}
 2 & 0 &0 \\
0 & 2  &0\\
0 & 0 &0
\end{pmatrix}. 
$$

This gives 
$$\frac12 \begin{pmatrix}
 2 & 0 &0 \\
0 & 2  &0\\
0 & 0 &0
\end{pmatrix}
\colvec{1\\1\\1 } = \colvec{1\\1\\0} .$$
\end{example}



\section{Singular Value Decomposition}

Suppose 
$$
L:V\tolinear W\, .
$$
It is unlikely that $\dim V=:n=m:=\dim W$ so a $m\times n$ matrix $M$ of $L$ in bases for $V$ and $W$ will not be square.
Therefore there is no eigenvalue problem  we can use to uncover a preferred basis. However, if the vector spaces $V$ and 
$W$ both have inner products, there does exist an analog of the eigenvalue problem, namely the singular values of $L$.

Before giving the details of the powerful technique known as the singular value decomposition, we note that it is an 
excellent example of what Eugene Wigner called the ``Unreasonable Effectiveness of Mathematics'':
\begin{quote}{\scriptsize
There is a story about two friends who were classmates in high school, talking about their jobs. One of them became a statistician
and was working on population trends. He showed a reprint to his former classmate.
The reprint started, as usual with the Gaussian distribution and the statistician explained
to his former classmate the meaning of the symbols for the actual population and so on. His classmate
was a bit incredulous and was not quite sure whether the statistician was pulling his leg. ``How can you 
know that?'' was his query. ``And what is this symbol here?'' ``Oh,'' said the statistician, this is ``$\pi$.''
``And what is that?'' ``The ratio of the circumference of the circle to its diameter.'' ``Well, now
you are pushing your joke too far,'' said the classmate, ``surely the population has nothing to do with the 
circumference of the circle.''


Eugene Wigner, Commun. Pure and Appl. Math. {\bf XIII}, 1 (1960).
}
\end{quote}
Whenever we mathematically model a system, any ``canonical quantities'' 
(those that  %on which we can all agree and 
do not
depend on any choices we make for calculating them) will correspond to important features of the system. For examples, the eigenvalues
of the eigenvector equation you found in review question~\ref{stringval}, chapter~\ref{eigenvalseigenvects} encode the notes and harmonics that a guitar string can play! 

Singular values appear in many linear algebra applications, especially those involving very large data sets such as statistics and signal processing. 

Let us focus on the $m\times n$ matrix $M$ of a linear transformation $L:V\to W$ written in orthonormal bases for the input and outputs of $L$ (notice, the existence of these othonormal bases is predicated on having inner products for $V$ and $W$).
Even though the matrix $M$ is not square, both the matrices $M M^T$ and $M^T M$ are square and symmetric! 
In terms of linear transformations $M^T$ is the matrix of a linear transformation 
$$
L^*:W\tolinear V\, .
$$
Thus $LL^*:W\to W$ and $L^*L:V\to V$ and both have eigenvalue problems.
Moreover,  as is shown  in Chapter~\ref{symmetricmatrices},  both $L^*L$ and $LL^*$ have orthonormal bases of eigenvectors, and
 both $MM^T$ and $M^TM$ can be diagonalized. 
 
Next, let us make a simplifying assumption, namely $\ker L=\{0\}$. This is not necessary, but will make some of our computations simpler.
Now suppose we have found an orthonormal basis $(u_1,\ldots , u_n)$ for $V$ composed of eigenvectors for $L^*L$. That is 
$$
L^*L u_i= \lambda_i u_i\, .
$$
Then multiplying by $L$ gives 
$$
L L^* L u_i = \lambda_i L u_i\, .
$$
{\it I.e.}, $L u_i$ is an eigenvector of $L L^*$.
The vectors $(Lu_1,\ldots, Lu_n)$ are linearly independent, because $\ker L=\{0\}$ (this is where we use our simplifying assumption, but you can 
try and extend our analysis to the case where it no longer holds). 

Lets compute the angles between and lengths of these vectors. 
For that we express the vectors $u_i$ in the bases used to compute the matrix $M$ of $L$. Denoting these column vectors by $U_i$ we then compute
$$
(MU_i)\cdot (MU_j)=U_i^T M^T M U_j = \lambda_j \, U_i^T U_j=\lambda_j \, U_i\cdot U_j = \lambda_j \delta_{ij}\, .
$$
We see that  vectors $(Lu_1,\ldots, Lu_n)$ are orthogonal but not orthonormal. Moreover, the length of $Lu_i$ is $\sqrt{\lambda_i}$.
Normalizing gives the orthonormal and linearly independent ordered set
$$
\left(\frac{Lu_1}{\sqrt{\lambda_1}},\ldots,\frac{Lu_n}{\sqrt{\lambda_n}}\right).
$$

In general, this cannot be a basis for $W$ 
since $\ker L=\{0\},~\dim L(V)=\dim V,$
and in turn $\dim V\leq \dim W$, so $n\leq m$. 

However,  it is a subset of the eigenvectors of $LL^*$ so there is an orthonormal basis of eigenvectors of $LL^*$ of the form 
$$
O'=\left(\frac{Lu_1}{\sqrt{\lambda_1}},\ldots,\frac{Lu_n}{\sqrt{\lambda_n}},v_{n+1},\ldots,v_{m}\right)=:(v_1,\ldots,v_m)\, .
$$
Now lets compute the matrix of $L$ with respect to the orthonormal basis $O=(u_1,\ldots,u_n)$ for $V$ and the orthonormal basis~$O'=(v_1,\ldots,v_m)$ for~$W$. As usual, our starting point is the computation of $L$ acting on the input basis vectors;
\begin{eqnarray*}
LO=\big(Lu_1,\ldots, Lu_n\big)&=&
\big(\sqrt{\lambda_1}\,  v_1,\ldots,\sqrt{\lambda_n}\,  v_n\big)\\[2mm]&=&\big(v_1,\ldots,v_m\big)
%\begin{pmatrix}
%\sqrt{\lambda_1}&\mc0&\cdots&\mc0&0&\cdots&0\\[1mm]
%\mc0&\sqrt{\lambda_2}&\cdots&\mc0&0&\cdots&0\\
%\mc{\vdots}&\mc\vdots&\ddots&\mc\vdots&\mc\vdots&&\mc\vdots\\[1mm]
%\mc0&\mc0&\cdots&\sqrt{\lambda_n}&0&\cdots&0\\
%\end{pmatrix}
\begin{pmatrix}
\sqrt{\lambda_1}&\mc0&\cdots&\mc0\\[1mm]
\mc0&\sqrt{\lambda_2}&\cdots&\mc0\\
\mc{\vdots}&\mc\vdots&\ddots&\mc\vdots\\[1mm]
\mc0&\mc0&\cdots&\sqrt{\lambda_n}\\[1mm]
\mc 0 & \mc 0& \cdots &\mc 0\\
\mc{\vdots}&\mc\vdots&&\mc\vdots\\
\mc 0 & \mc 0& \cdots &\mc 0
\end{pmatrix}\, .
\end{eqnarray*}
The result is very close to diagonalization; the numbers $\sqrt{\lambda_i}$ along the leading diagonal are called the singular values of $L$.

\begin{example} Let the matrix of a linear transformation be
$$
M=\begin{pmatrix}
\frac12&\frac12\\[1mm]-1&1\\[1mm]-\frac12&-\frac12
\end{pmatrix}\, .
$$
Clearly $\ker M=\{0\}$ while
$$
M^TM=\begin{pmatrix}\frac32&-\frac12\\[2mm]-\frac12&\frac32\end{pmatrix}
$$
which has eigenvalues and eigenvectors
$$
 \lambda=1\, ,\,  u_1:=\colvec{\frac{1}{\sqrt2}\\[2mm]\frac{1}{\sqrt2}}; \qquad
\lambda=2\, ,\,  u_2:=\colvec{\frac{1}{\sqrt2}\\[2mm]-\frac{1}{\sqrt2}}\,\, .
$$
so our orthonormal input basis is $$O=\left(\colvec{\frac{1}{\sqrt2}\\[2mm]\frac{1}{\sqrt2}},\colvec{\frac{1}{\sqrt2}\\[2mm]-\frac{1}{\sqrt2}}\right)\, .
$$
These are called the {\it right singular vectors}\index{Right singular vector} of $M$.
The vectors 
$$
M u_1= \colvec{\frac1{\sqrt{2}}\\[1mm]\mc{\ \ 0}\\-\frac1{\sqrt{2}}}\mbox{ and }
M u_2=\ccolvec{0\\[1mm]-\sqrt{2}\\[1mm]0}
$$
are eigenvectors of 
$$M M^T=\begin{pmatrix}\frac12&\ 0&\!-\frac12\\0&2&0\\-\frac12&0&\frac12\end{pmatrix}$$ 
with eigenvalues $1$ and $2$, respectively. The third eigenvector (with eigenvalue~$0$) of $MM^T$ is 
$$v_3=\colvec{\frac1{\sqrt{2}}\\[1mm]\mc{\ \ 0}\\ \frac1{\sqrt{2}}}\, .$$
The eigenvectors $Mu_1$ and $Mu_2$ are necessarily orthogonal, dividing them by their lengths we obtain the {\it left singular vectors}\index{Left singular vectors} and in turn  our orthonormal output basis
$$
O'=\left(\colvec{\frac1{\sqrt{2}}\\[1mm]\mc{\ \ 0}\\-\frac1{\sqrt{2}}},\ccolvec{0\\[1mm]-1\\[1mm]0},\colvec{\frac1{\sqrt{2}}\\[1mm]\mc{\ \ 0}\\\frac1{\sqrt{2}}}\right)\, .
$$
The new matrix~$M'$ of the linear transformation given by $M$ with respect to the bases $O$ and $O'$ is
$$
M'=\begin{pmatrix}
1&0\\0&\sqrt{2}\\0&0
\end{pmatrix}\, ,
$$
so the singular values are $1,\sqrt{2}$. 

Finally note that arranging the column vectors of $O$ and $O'$ into change of basis matrices
$$
P=\begin{pmatrix}
\frac1{\sqrt{2}}&\frac1{\sqrt{2}}\\[2mm]
\frac1{\sqrt{2}}&-\frac1{\sqrt{2}}
\end{pmatrix}\, ,\qquad
Q=
\begin{pmatrix}
\frac1{\sqrt{2}}&0&\frac1{\sqrt{2}}\\[2mm]
\mc {\ \ 0}&-1&\mc 0\\[2mm]
\!-\frac1{\sqrt{2}}&0&\frac1{\sqrt{2}}
\end{pmatrix}\, ,
$$
we have, as usual,
$$
M'=Q^{-1}MP\, .
$$
\end{example}

Singular vectors and values have a very nice geometric interpretation; they provide an orthonormal bases for the domain and range of $L$
and give the factors by which $L$ stretches the orthonormal input basis vectors. This is depicted below for the example we just computed.
\begin{center}
\includegraphics[scale=.27]{singval.jpg}
\end{center} 



%{\it Congratulations, you have reached the end of these notes! You can test your skills
%on the \hyperref[sample3]{sample final exam}.}
\begin{center}
\shabox{
{\bf \hyperref[sample3]{\begin{tabular}{c}Congratulations, you have reached the end of the book! \\[2mm]
\includegraphics[scale=.15]{final.jpg}\\
Now test your skills on the \hyperref[sample3]{sample final exam}.
%You are now ready to 
%apply for membership in\\
% be a minion of Captain Conundrum's nemesis, 
%The League of Ninjas of Numbers. 
%Now test your skills
%on the sample final exam. 
\end{tabular}
}}}
\end{center}













%\section*{References}
%Hefferon, Chapter Three, Section VI.2: Gram-Schmidt Orthogonalization \\
%Beezer, Part A, Section CF, Subsection DF \\
%Wikipedia:
%\begin{itemize}
%\item \href{http://en.wikipedia.org/wiki/Linear_least_squares}{Linear Least Squares}
%\item \href{http://en.wikipedia.org/wiki/Least_squares}{Least Squares}
%\end{itemize}

\section{Review Problems}

{\bf Webwork:} 
\begin{tabular}{|c|c|}
\hline
Reading Problem & 
 \hwrref{LeastSquares}{1}, 
\\
   \hline
\end{tabular}





\begin{enumerate}

\item While performing  Gaussian elimination on these augmented matrices write the full system of equations describing the new rows in terms of the old rows above each equivalence symbol as in  \hyperlink{Keeping track of EROs with equations between rows}{Example}~\ref{Rsystem}. 
$$
\begin{amatrix}{2} 
2 & 2 & 10 \\
1 & 2 & 8 \\
\end{amatrix}
,~
\begin{amatrix}{3} 
1 & 1 & 0 & 5 \\
1 & 1 & \!\!-1& 11 \\
-1 & 1 & 1 & -5 \\ 
\end{amatrix}
$$

%%%%%%%%%%%%%%%%%%%

\item Solve the vector equation by applying ERO matrices to each side of the equation to perform elimination. Show each matrix explicitly as in \hyperlink{Undoing}{Example~\ref{slowly}}.

\begin{eqnarray*}
\begin{pmatrix}
3	&6 	&2 \\ %-3
5 	&9 	&4 \\ %1
2	&4	&2 \\ %0
\end{pmatrix} 
\begin{pmatrix}
 x \\ 
y \\
z 
\end{pmatrix} 
=
\begin{pmatrix}
-3 \\ 
1  \\
0  \\
\end{pmatrix} 
\end{eqnarray*}

%%%%%%%%%%%%%%%%%%%

\item Solve this vector equation by finding the inverse of the matrix through $(M|I)\sim (I|M^{-1})$ and then applying $M^{-1}$ to both sides of the equation. 
\begin{eqnarray*}
\begin{pmatrix}
2	&1 	&1 \\ %9
1 	&1 	&1 \\ %6
1	&1	&2 \\ %7
\end{pmatrix} 
\begin{pmatrix}
 x \\ 
y \\
z 
\end{pmatrix} 
=
\begin{pmatrix}
9 \\ 
6  \\
7  \\
\end{pmatrix} 
\end{eqnarray*}


%%%%%%%%%%%%%%%%%%%

\item Follow the method of  \hyperlink{elldeeeww}{Examples~\ref{factorize} and~\ref{factorizes}} to find the $LU$ and $LDU$ factorization of 
\begin{eqnarray*}
\begin{pmatrix}
3	&3 	&6 \\ %0 %2
3 	&5 	&2 \\ %1 %1
6	&2	&5 \\ %0 %1
\end{pmatrix} .
\end{eqnarray*}



%%%%%%%%%%%%%%%%%%%%

\item 
Multiple matrix equations with the same matrix can be solved simultaneously. 
\begin{enumerate}
\item Solve both systems by performing elimination on just one augmented matrix.
\begin{eqnarray*}
\begin{pmatrix}
2	&-1 	&-1 \\ %0 %2
-1 	&1 	&1 \\ %1 %1
1	&-1	&0 \\ %0 %1
\end{pmatrix} 
\begin{pmatrix}
 x \\ 
y \\
z 
\end{pmatrix} 
=
\begin{pmatrix}
0\\ 
1  \\
0  \\
\end{pmatrix} 
,~
\begin{pmatrix}
2	&-1 	&-1 \\ %0 %2
-1 	&1 	&1 \\ %1 %1
1	&-1	&0 \\ %0 %1
\end{pmatrix} 
\begin{pmatrix}
 a \\ 
b \\
c 
\end{pmatrix} 
=
\begin{pmatrix}
2\\ 
1  \\
1  \\
\end{pmatrix} 
\end{eqnarray*}
\item Give an interpretation of the columns of $M^{-1}$ in $(M|I)\sim (I|M^{-1})$ in terms of solutions to certain systems of linear equations.
\end{enumerate}

%%%%%%%%%%%%%%%%%%%%%%%%

\item How can you convince your fellow students to never make this mistake?
\begin{eqnarray*}
\begin{amatrix}{3} 
1 & 0 & 2 & 3 \\ 
0 & 1 & 2& 3 \\
2 & 0 & 1 & 4 \\
\end{amatrix} 
& 
\stackrel{R_1'=R_1+R_2}{
\stackrel{R_2'=R_1-R_2}{ 
\stackrel{\ R_3'= R_1+2R_2}{\sim}}}
&
\begin{amatrix}{3} 
1 & 1 & 4 & 6 \\
1 & \!\!-1 & 0& 0 \\
1 & 2 & 6 & 9 
\end{amatrix}
\end{eqnarray*}

\item Is $LU$ factorization of a matrix unique?  Justify your answer.


\item[$\infty$.] If you randomly create a matrix by picking numbers out of the blue, it will probably be difficult to perform elimination or factorization; fractions and large numbers will probably be involved. To invent simple problems it is better to start with a simple answer:
\begin{enumerate}
\item Start with any augmented matrix in RREF. Perform EROs to make most of the components non-zero. Write the result on a separate piece of paper and give it to your friend. Ask that friend to find RREF of the augmented matrix you gave them. Make sure they get the same augmented matrix you started with.  
\item Create  an upper triangular matrix $U$ and a lower triangular matrix~$L$ with only $1$s on the diagonal. Give the result to a friend to factor into $LU$ form. 
\item Do the same with an $LDU$ factorization. 
\end{enumerate}
\end{enumerate}

\phantomnewpage




\newpage



\chapter{Least squares and Singular Values}
\label{sec:leastsquaresSVD}
\index{Least squares}

Consider the linear algebraic equation $L(x)=v$, where $L \colon U\stackrel{\text{linear}}{-\!\!\!-\!\!\!\longrightarrow}W$ and $v\in W$ are known while $x$ is unknown. As we have seen, this system may have 
one solution, no solutions, or infinitely many solutions.  
But if $v$ is not in the range of $L$ there will {\it never} be any solutions for $L(x)=v$.
\vspace{-.1cm}
\begin{center}
\includegraphics[scale=.24]{notinimage.jpg}
\end{center} 
\vspace{-1.8cm}
However, for many applications we do not need an exact solution of the system; instead, we may only need the best approximation possible.  

\begin{quote}
``My work always tried to unite the Truth with the Beautiful, but when I had to choose one or the other, I usually chose the Beautiful.'' 

\vspace{-2mm}
\hspace{7cm}-- Hermann Weyl.
\end{quote}

If the vector space $W$ has a notion of lengths of vectors, we can try to find $x$ that minimizes $||L(x)-v||$.
\begin{center}
\includegraphics[scale=.24]{minimize.jpg}
\end{center} 
This method has many applications, such as when trying to fit a (perhaps linear) function to a ``noisy'' set of observations.  For example, suppose we measured the position of a bicycle on a racetrack once every five seconds.  Our observations won't be exact, but so long as the observations are right on average, we can figure out a best-possible linear function of position of the bicycle in terms of time.

Suppose $M$ is the matrix for the linear function $L:U \to W$ in some bases for $U$ and $W$. The vectors~$v$ and~$x$ are represented by column vectors $V$ and $X$ in these bases.  Then we need to approximate
\[
MX-V\approx 0\, .
\]

Note that if $\dim U=n$ and $\dim W=m$ then $M$ can be represented by an $m\times n$ matrix and $x$ and $v$ as vectors in $\Re^n$ and $\Re^m$, respectively. Thus, we can write $W=L(U)\oplus L(U)^\perp$.  Then we can uniquely write $v=v^\parallel + v^\perp$, with $v^\parallel \in L(U)$ and $v^\perp \in L(U)^\perp$.  



Thus we should solve $L(u)=v^\parallel$.  In components, $v^\perp$ is just $V-MX$, and is the part we will eventually wish to minimize.  

In terms of $M$, recall that $L(V)$ is spanned by the columns of $M$.  (In the standard basis, the columns of $M$ are $Me_1$, 
$\ldots$, $Me_n$.)  Then $v^\perp$ must be perpendicular to the columns of $M$.  \textit{i.e.}, $M^T(V-MX)=0$, or
\[
M^TMX = M^TV.
\]
Solutions of $M^TMX = M^TV$ for $X$ are called \emph{least squares}\index{Least squares!solutions} solutions to $MX=V$.  
Notice that any solution $X$ to $MX=V$ is a least squares solution.  However, the converse is often false.  In fact, the equation $MX=V$ may have no solutions at all, but still have least squares solutions to $M^TMX = M^TV$.

Observe that since $M$ is an $m\times n$ matrix, then $M^T$ is an $n\times m$ matrix.  Then $M^TM$ is an $n\times n$ matrix, and is symmetric, since $(M^TM)^T=M^TM$.  Then, for any vector $X$, we can evaluate $X^TM^TMX$ to obtain a number.  This is a very nice number, though!  It is just the length $|MX|^2 = (MX)^T(MX)=X^TM^TMX$.

%\href{\webworkurl ReadingHomework25/1/}{Reading homework: problem 25.1}
\Reading{LeastSquares}{1}

Now suppose that $\ker L=\{0\}$, so that the only solution to $MX=0$ is $X=0$. (This need not mean that $M$ is invertible because $M$ is an $n\times m $ matrix, so not necessarily square.) 
However the square matrix $M^TM$ {\it is} invertible. To see this, suppose there was a vector $X$ such that 
$M^T M X=0$. Then it would follow that $X^T M^T M X = |M X|^2=0$. In other words the vector $MX$ would have zero length, so could only be the zero vector. But we are assuming that $\ker L=\{0\}$ so $MX=0$ implies $X=0$. Thus the kernel of $M^TM$ is $\{0\}$ so this matrix is invertible.
So, in this case, the least squares solution (the $X$ that solves $M^TMX=MV$) is unique, and is equal to 
\[
X = (M^TM)^{-1}M^TV.
\]
In a nutshell, this is the least squares method:

\begin{itemize}
\item Compute $M^TM$ and $M^TV$.
\item Solve $(M^TM)X=M^TV$ by Gaussian elimination.
\end{itemize}


\begin{example}
Captain Conundrum\index{Captain Conundrum} falls off of the leaning tower of Pisa and makes three (rather shaky) measurements of his velocity at three different times.

\begin{center}
\begin{tabular}{c|c}
$t$ s & $v $ m/s \\ \hline
$1$ & $11$ \\
$2$ & $19$ \\
$3$ & $31$
\end{tabular}
\end{center}

Having taken some calculus\footnote{In fact, he is a \emph{Calculus Superhero}\index{Calculus Superhero}.}, he believes that his data are best approximated by a straight line
\[
v = at+b.
\]
Then he should find $a$ and $b$ to best fit the data.
\begin{eqnarray*}
11 &=& a\cdot 1 + b \\
19 &=& a\cdot 2 + b \\
31 &=& a\cdot 3 + b.
\end{eqnarray*}
As a system of linear equations, this becomes:

\[
\begin{pmatrix}
1 & 1 \\
2 & 1 \\
3 & 1 \\
\end{pmatrix}
\colvec{a\\b} \stackrel{?}{=}
\colvec{11\\19\\31}.
\]
There is likely no actual straight line solution, so instead solve $M^TMX=M^TV$.

\[
\begin{pmatrix}
1 & 2 & 3 \\
1 & 1 & 1 \\
\end{pmatrix}
\begin{pmatrix}
1 & 1 \\
2 & 1 \\
3 & 1 \\
\end{pmatrix} \colvec{a\\b}
= 
\begin{pmatrix}
1 & 2 & 3 \\
1 & 1 & 1 \\
\end{pmatrix}
\colvec{11\\19\\31}.
\]
This simplifies to 

\[
\begin{amatrix}{2}
14 & 6 & 142 \\
6 & 3 & 61
\end{amatrix}
\sim
\begin{amatrix}{2}
1 & 0 & 10 \\
0 & 1 & \frac{1}{3}
\end{amatrix}.
\]
Thus, the least-squares fit is the line

\[
v = 10\ t + \frac{1}{3}\, .
\]
Notice that this equation implies that Captain Conundrum accelerates towards Italian soil at 10 m/s$^2$ (which is an excellent
approximation to reality) and that he started at a downward velocity of $\frac13$ m/s (perhaps somebody gave him a shove...)!

\end{example}

\section{Projection Matrices}
We have seen that even if $MX=V$ has no solutions $M^TMX=M^T V$ does have solutions. One way to think about this is, since the codomain of $M$ is the direct sum 
$$ \text{codom M}=\text{ran} M \oplus \ker M^T$$ 
there is a unique way to write  $V=V_r+V_k$ with $V_k\in \ker M^T$ and $V_r\in \text{ran }\, M$, and it is clear that $Mx=V$ only has a solution of 
$V\in \text{ran}\, M \Leftrightarrow V_k=0$. If not, then the closest thing to a solution of $MX=V$ is a solution to $MX=V_r$. We learned to find solutions to this in the previous subsection of this book. 

But here is another question, how can we determine what $V_r$ is given $M$ and $V$? The answer is simple; suppose $X$ is a solution to $MX=V_r$. Then
$$  MX=V_r 
\implies M^TMx=M^T V_r 
\implies M^TMx=M^T (V_r + 0) $$ 
$$
\implies M^TMx=M^T (V_r+V_k)
\implies M^TMx=M^T V 
\implies X=(M^TM)^{-1} M^T V 
$$
if indeed $M^TM$ is invertible. Since, by assumption, $X$ is a solution \\
\begin{center}
\shabox{ $M(M^TM)^{-1} M^T\, V =V_r. $}
\end{center}
That is, the matrix which projects $V$ onto its $\text{ran} \, M$ part is $M(M^TM)^{-1} M^T$. 

\begin{example} To project $\colvec{1\\1\\1}$ onto $\spa \left\{    \colvec{ 1\\1\\0}, \colvec{1\\-1\\0 }  \right\} = \text{ran} 
\begin{pmatrix}
 1& 1  \\
1 & -1  \\
0 & 0 
\end{pmatrix}
 $ 
  multiply by the matrix 
 $$
\begin{pmatrix}
 1& 1  \\
1 & -1  \\
0 & 0 
\end{pmatrix}
\left [ 
 \begin{pmatrix}
 1& 1 &0 \\
1 & -1 &0 
\end{pmatrix}
\begin{pmatrix}
 1& 1  \\
1 & -1  \\
0 & 0 
\end{pmatrix}
 \right]^{-1}
 \begin{pmatrix}
 1& 1 &0 \\
1 & -1 &0 
\end{pmatrix}
 $$
 $$
=\begin{pmatrix}
 1& 1  \\
1 & -1  \\
0 & 0 
\end{pmatrix}
 \begin{pmatrix}
 2& 0  \\
0 & 2  
\end{pmatrix}^{-1}
 \begin{pmatrix}
 1& 1 &0 \\
1 & -1 &0 
\end{pmatrix} 
$$
$$
=\frac12 \begin{pmatrix}
 1& 1  \\
1 & -1  \\
0 & 0 
\end{pmatrix}
 \begin{pmatrix}
 1& 1 &0 \\
1 & -1 &0 
\end{pmatrix} 
=
\frac12 \begin{pmatrix}
 2 & 0 &0 \\
0 & 2  &0\\
0 & 0 &0
\end{pmatrix}. 
$$

This gives 
$$\frac12 \begin{pmatrix}
 2 & 0 &0 \\
0 & 2  &0\\
0 & 0 &0
\end{pmatrix}
\colvec{1\\1\\1 } = \colvec{1\\1\\0} .$$
\end{example}



\section{Singular Value Decomposition}

Suppose 
$$
L:V\tolinear W\, .
$$
It is unlikely that $\dim V=:n=m:=\dim W$ so a $m\times n$ matrix $M$ of $L$ in bases for $V$ and $W$ will not be square.
Therefore there is no eigenvalue problem  we can use to uncover a preferred basis. However, if the vector spaces $V$ and 
$W$ both have inner products, there does exist an analog of the eigenvalue problem, namely the singular values of $L$.

Before giving the details of the powerful technique known as the singular value decomposition, we note that it is an 
excellent example of what Eugene Wigner called the ``Unreasonable Effectiveness of Mathematics'':
\begin{quote}{\scriptsize
There is a story about two friends who were classmates in high school, talking about their jobs. One of them became a statistician
and was working on population trends. He showed a reprint to his former classmate.
The reprint started, as usual with the Gaussian distribution and the statistician explained
to his former classmate the meaning of the symbols for the actual population and so on. His classmate
was a bit incredulous and was not quite sure whether the statistician was pulling his leg. ``How can you 
know that?'' was his query. ``And what is this symbol here?'' ``Oh,'' said the statistician, this is ``$\pi$.''
``And what is that?'' ``The ratio of the circumference of the circle to its diameter.'' ``Well, now
you are pushing your joke too far,'' said the classmate, ``surely the population has nothing to do with the 
circumference of the circle.''


Eugene Wigner, Commun. Pure and Appl. Math. {\bf XIII}, 1 (1960).
}
\end{quote}
Whenever we mathematically model a system, any ``canonical quantities'' 
(those that  %on which we can all agree and 
do not
depend on any choices we make for calculating them) will correspond to important features of the system. For examples, the eigenvalues
of the eigenvector equation you found in review question~\ref{stringval}, chapter~\ref{eigenvalseigenvects} encode the notes and harmonics that a guitar string can play! 

Singular values appear in many linear algebra applications, especially those involving very large data sets such as statistics and signal processing. 

Let us focus on the $m\times n$ matrix $M$ of a linear transformation $L:V\to W$ written in orthonormal bases for the input and outputs of $L$ (notice, the existence of these othonormal bases is predicated on having inner products for $V$ and $W$).
Even though the matrix $M$ is not square, both the matrices $M M^T$ and $M^T M$ are square and symmetric! 
In terms of linear transformations $M^T$ is the matrix of a linear transformation 
$$
L^*:W\tolinear V\, .
$$
Thus $LL^*:W\to W$ and $L^*L:V\to V$ and both have eigenvalue problems.
Moreover,  as is shown  in Chapter~\ref{symmetricmatrices},  both $L^*L$ and $LL^*$ have orthonormal bases of eigenvectors, and
 both $MM^T$ and $M^TM$ can be diagonalized. 
 
Next, let us make a simplifying assumption, namely $\ker L=\{0\}$. This is not necessary, but will make some of our computations simpler.
Now suppose we have found an orthonormal basis $(u_1,\ldots , u_n)$ for $V$ composed of eigenvectors for $L^*L$. That is 
$$
L^*L u_i= \lambda_i u_i\, .
$$
Then multiplying by $L$ gives 
$$
L L^* L u_i = \lambda_i L u_i\, .
$$
{\it I.e.}, $L u_i$ is an eigenvector of $L L^*$.
The vectors $(Lu_1,\ldots, Lu_n)$ are linearly independent, because $\ker L=\{0\}$ (this is where we use our simplifying assumption, but you can 
try and extend our analysis to the case where it no longer holds). 

Lets compute the angles between and lengths of these vectors. 
For that we express the vectors $u_i$ in the bases used to compute the matrix $M$ of $L$. Denoting these column vectors by $U_i$ we then compute
$$
(MU_i)\cdot (MU_j)=U_i^T M^T M U_j = \lambda_j \, U_i^T U_j=\lambda_j \, U_i\cdot U_j = \lambda_j \delta_{ij}\, .
$$
We see that  vectors $(Lu_1,\ldots, Lu_n)$ are orthogonal but not orthonormal. Moreover, the length of $Lu_i$ is $\sqrt{\lambda_i}$.
Normalizing gives the orthonormal and linearly independent ordered set
$$
\left(\frac{Lu_1}{\sqrt{\lambda_1}},\ldots,\frac{Lu_n}{\sqrt{\lambda_n}}\right).
$$

In general, this cannot be a basis for $W$ 
since $\ker L=\{0\},~\dim L(V)=\dim V,$
and in turn $\dim V\leq \dim W$, so $n\leq m$. 

However,  it is a subset of the eigenvectors of $LL^*$ so there is an orthonormal basis of eigenvectors of $LL^*$ of the form 
$$
O'=\left(\frac{Lu_1}{\sqrt{\lambda_1}},\ldots,\frac{Lu_n}{\sqrt{\lambda_n}},v_{n+1},\ldots,v_{m}\right)=:(v_1,\ldots,v_m)\, .
$$
Now lets compute the matrix of $L$ with respect to the orthonormal basis $O=(u_1,\ldots,u_n)$ for $V$ and the orthonormal basis~$O'=(v_1,\ldots,v_m)$ for~$W$. As usual, our starting point is the computation of $L$ acting on the input basis vectors;
\begin{eqnarray*}
LO=\big(Lu_1,\ldots, Lu_n\big)&=&
\big(\sqrt{\lambda_1}\,  v_1,\ldots,\sqrt{\lambda_n}\,  v_n\big)\\[2mm]&=&\big(v_1,\ldots,v_m\big)
%\begin{pmatrix}
%\sqrt{\lambda_1}&\mc0&\cdots&\mc0&0&\cdots&0\\[1mm]
%\mc0&\sqrt{\lambda_2}&\cdots&\mc0&0&\cdots&0\\
%\mc{\vdots}&\mc\vdots&\ddots&\mc\vdots&\mc\vdots&&\mc\vdots\\[1mm]
%\mc0&\mc0&\cdots&\sqrt{\lambda_n}&0&\cdots&0\\
%\end{pmatrix}
\begin{pmatrix}
\sqrt{\lambda_1}&\mc0&\cdots&\mc0\\[1mm]
\mc0&\sqrt{\lambda_2}&\cdots&\mc0\\
\mc{\vdots}&\mc\vdots&\ddots&\mc\vdots\\[1mm]
\mc0&\mc0&\cdots&\sqrt{\lambda_n}\\[1mm]
\mc 0 & \mc 0& \cdots &\mc 0\\
\mc{\vdots}&\mc\vdots&&\mc\vdots\\
\mc 0 & \mc 0& \cdots &\mc 0
\end{pmatrix}\, .
\end{eqnarray*}
The result is very close to diagonalization; the numbers $\sqrt{\lambda_i}$ along the leading diagonal are called the singular values of $L$.

\begin{example} Let the matrix of a linear transformation be
$$
M=\begin{pmatrix}
\frac12&\frac12\\[1mm]-1&1\\[1mm]-\frac12&-\frac12
\end{pmatrix}\, .
$$
Clearly $\ker M=\{0\}$ while
$$
M^TM=\begin{pmatrix}\frac32&-\frac12\\[2mm]-\frac12&\frac32\end{pmatrix}
$$
which has eigenvalues and eigenvectors
$$
 \lambda=1\, ,\,  u_1:=\colvec{\frac{1}{\sqrt2}\\[2mm]\frac{1}{\sqrt2}}; \qquad
\lambda=2\, ,\,  u_2:=\colvec{\frac{1}{\sqrt2}\\[2mm]-\frac{1}{\sqrt2}}\,\, .
$$
so our orthonormal input basis is $$O=\left(\colvec{\frac{1}{\sqrt2}\\[2mm]\frac{1}{\sqrt2}},\colvec{\frac{1}{\sqrt2}\\[2mm]-\frac{1}{\sqrt2}}\right)\, .
$$
These are called the {\it right singular vectors}\index{Right singular vector} of $M$.
The vectors 
$$
M u_1= \colvec{\frac1{\sqrt{2}}\\[1mm]\mc{\ \ 0}\\-\frac1{\sqrt{2}}}\mbox{ and }
M u_2=\ccolvec{0\\[1mm]-\sqrt{2}\\[1mm]0}
$$
are eigenvectors of 
$$M M^T=\begin{pmatrix}\frac12&\ 0&\!-\frac12\\0&2&0\\-\frac12&0&\frac12\end{pmatrix}$$ 
with eigenvalues $1$ and $2$, respectively. The third eigenvector (with eigenvalue~$0$) of $MM^T$ is 
$$v_3=\colvec{\frac1{\sqrt{2}}\\[1mm]\mc{\ \ 0}\\ \frac1{\sqrt{2}}}\, .$$
The eigenvectors $Mu_1$ and $Mu_2$ are necessarily orthogonal, dividing them by their lengths we obtain the {\it left singular vectors}\index{Left singular vectors} and in turn  our orthonormal output basis
$$
O'=\left(\colvec{\frac1{\sqrt{2}}\\[1mm]\mc{\ \ 0}\\-\frac1{\sqrt{2}}},\ccolvec{0\\[1mm]-1\\[1mm]0},\colvec{\frac1{\sqrt{2}}\\[1mm]\mc{\ \ 0}\\\frac1{\sqrt{2}}}\right)\, .
$$
The new matrix~$M'$ of the linear transformation given by $M$ with respect to the bases $O$ and $O'$ is
$$
M'=\begin{pmatrix}
1&0\\0&\sqrt{2}\\0&0
\end{pmatrix}\, ,
$$
so the singular values are $1,\sqrt{2}$. 

Finally note that arranging the column vectors of $O$ and $O'$ into change of basis matrices
$$
P=\begin{pmatrix}
\frac1{\sqrt{2}}&\frac1{\sqrt{2}}\\[2mm]
\frac1{\sqrt{2}}&-\frac1{\sqrt{2}}
\end{pmatrix}\, ,\qquad
Q=
\begin{pmatrix}
\frac1{\sqrt{2}}&0&\frac1{\sqrt{2}}\\[2mm]
\mc {\ \ 0}&-1&\mc 0\\[2mm]
\!-\frac1{\sqrt{2}}&0&\frac1{\sqrt{2}}
\end{pmatrix}\, ,
$$
we have, as usual,
$$
M'=Q^{-1}MP\, .
$$
\end{example}

Singular vectors and values have a very nice geometric interpretation; they provide an orthonormal bases for the domain and range of $L$
and give the factors by which $L$ stretches the orthonormal input basis vectors. This is depicted below for the example we just computed.
\begin{center}
\includegraphics[scale=.27]{singval.jpg}
\end{center} 



%{\it Congratulations, you have reached the end of these notes! You can test your skills
%on the \hyperref[sample3]{sample final exam}.}
\begin{center}
\shabox{
{\bf \hyperref[sample3]{\begin{tabular}{c}Congratulations, you have reached the end of the book! \\[2mm]
\includegraphics[scale=.15]{final.jpg}\\
Now test your skills on the \hyperref[sample3]{sample final exam}.
%You are now ready to 
%apply for membership in\\
% be a minion of Captain Conundrum's nemesis, 
%The League of Ninjas of Numbers. 
%Now test your skills
%on the sample final exam. 
\end{tabular}
}}}
\end{center}













%\section*{References}
%Hefferon, Chapter Three, Section VI.2: Gram-Schmidt Orthogonalization \\
%Beezer, Part A, Section CF, Subsection DF \\
%Wikipedia:
%\begin{itemize}
%\item \href{http://en.wikipedia.org/wiki/Linear_least_squares}{Linear Least Squares}
%\item \href{http://en.wikipedia.org/wiki/Least_squares}{Least Squares}
%\end{itemize}

\section{Review Problems}

{\bf Webwork:} 
\begin{tabular}{|c|c|}
\hline
Reading Problem & 
 \hwrref{LeastSquares}{1}, 
\\
   \hline
\end{tabular}





\begin{enumerate}

\item While performing  Gaussian elimination on these augmented matrices write the full system of equations describing the new rows in terms of the old rows above each equivalence symbol as in  \hyperlink{Keeping track of EROs with equations between rows}{Example}~\ref{Rsystem}. 
$$
\begin{amatrix}{2} 
2 & 2 & 10 \\
1 & 2 & 8 \\
\end{amatrix}
,~
\begin{amatrix}{3} 
1 & 1 & 0 & 5 \\
1 & 1 & \!\!-1& 11 \\
-1 & 1 & 1 & -5 \\ 
\end{amatrix}
$$

%%%%%%%%%%%%%%%%%%%

\item Solve the vector equation by applying ERO matrices to each side of the equation to perform elimination. Show each matrix explicitly as in \hyperlink{Undoing}{Example~\ref{slowly}}.

\begin{eqnarray*}
\begin{pmatrix}
3	&6 	&2 \\ %-3
5 	&9 	&4 \\ %1
2	&4	&2 \\ %0
\end{pmatrix} 
\begin{pmatrix}
 x \\ 
y \\
z 
\end{pmatrix} 
=
\begin{pmatrix}
-3 \\ 
1  \\
0  \\
\end{pmatrix} 
\end{eqnarray*}

%%%%%%%%%%%%%%%%%%%

\item Solve this vector equation by finding the inverse of the matrix through $(M|I)\sim (I|M^{-1})$ and then applying $M^{-1}$ to both sides of the equation. 
\begin{eqnarray*}
\begin{pmatrix}
2	&1 	&1 \\ %9
1 	&1 	&1 \\ %6
1	&1	&2 \\ %7
\end{pmatrix} 
\begin{pmatrix}
 x \\ 
y \\
z 
\end{pmatrix} 
=
\begin{pmatrix}
9 \\ 
6  \\
7  \\
\end{pmatrix} 
\end{eqnarray*}


%%%%%%%%%%%%%%%%%%%

\item Follow the method of  \hyperlink{elldeeeww}{Examples~\ref{factorize} and~\ref{factorizes}} to find the $LU$ and $LDU$ factorization of 
\begin{eqnarray*}
\begin{pmatrix}
3	&3 	&6 \\ %0 %2
3 	&5 	&2 \\ %1 %1
6	&2	&5 \\ %0 %1
\end{pmatrix} .
\end{eqnarray*}



%%%%%%%%%%%%%%%%%%%%

\item 
Multiple matrix equations with the same matrix can be solved simultaneously. 
\begin{enumerate}
\item Solve both systems by performing elimination on just one augmented matrix.
\begin{eqnarray*}
\begin{pmatrix}
2	&-1 	&-1 \\ %0 %2
-1 	&1 	&1 \\ %1 %1
1	&-1	&0 \\ %0 %1
\end{pmatrix} 
\begin{pmatrix}
 x \\ 
y \\
z 
\end{pmatrix} 
=
\begin{pmatrix}
0\\ 
1  \\
0  \\
\end{pmatrix} 
,~
\begin{pmatrix}
2	&-1 	&-1 \\ %0 %2
-1 	&1 	&1 \\ %1 %1
1	&-1	&0 \\ %0 %1
\end{pmatrix} 
\begin{pmatrix}
 a \\ 
b \\
c 
\end{pmatrix} 
=
\begin{pmatrix}
2\\ 
1  \\
1  \\
\end{pmatrix} 
\end{eqnarray*}
\item Give an interpretation of the columns of $M^{-1}$ in $(M|I)\sim (I|M^{-1})$ in terms of solutions to certain systems of linear equations.
\end{enumerate}

%%%%%%%%%%%%%%%%%%%%%%%%

\item How can you convince your fellow students to never make this mistake?
\begin{eqnarray*}
\begin{amatrix}{3} 
1 & 0 & 2 & 3 \\ 
0 & 1 & 2& 3 \\
2 & 0 & 1 & 4 \\
\end{amatrix} 
& 
\stackrel{R_1'=R_1+R_2}{
\stackrel{R_2'=R_1-R_2}{ 
\stackrel{\ R_3'= R_1+2R_2}{\sim}}}
&
\begin{amatrix}{3} 
1 & 1 & 4 & 6 \\
1 & \!\!-1 & 0& 0 \\
1 & 2 & 6 & 9 
\end{amatrix}
\end{eqnarray*}

\item Is $LU$ factorization of a matrix unique?  Justify your answer.


\item[$\infty$.] If you randomly create a matrix by picking numbers out of the blue, it will probably be difficult to perform elimination or factorization; fractions and large numbers will probably be involved. To invent simple problems it is better to start with a simple answer:
\begin{enumerate}
\item Start with any augmented matrix in RREF. Perform EROs to make most of the components non-zero. Write the result on a separate piece of paper and give it to your friend. Ask that friend to find RREF of the augmented matrix you gave them. Make sure they get the same augmented matrix you started with.  
\item Create  an upper triangular matrix $U$ and a lower triangular matrix~$L$ with only $1$s on the diagonal. Give the result to a friend to factor into $LU$ form. 
\item Do the same with an $LDU$ factorization. 
\end{enumerate}
\end{enumerate}

\phantomnewpage




\newpage


\newpage

\section*{Wikipedia}

\begin{itemize}
\item \href{http://en.wikipedia.org/wiki/System_of_linear_equations}{Systems of Linear Equations}

\item \href{http://en.wikipedia.org/wiki/Row_echelon_form}{Row Echelon Form}
 
\item \href{http://en.wikipedia.org/wiki/Row_echelon_form}{Row Echelon Form}

\item \href{http://en.wikipedia.org/wiki/Elementary_matrix_transformations}{Elementary Matrix Operations}
\end{itemize}


\newpage

\section*{Review Problems}

\section*{Linear Systems}

\moduleinputProblems{\whatIsPath/problems}

\newpage

\section*{Gaussian Elimination}

\moduleinputProblems{\gaussElimPath/problems}

\newpage

\section*{Elementary Row Operations}

\moduleinputProblems{\elemRowOpsPath/problems}

\newpage

\section*{Solution Sets for Systems of Linear Equations}

\moduleinputProblems{\solutionSetsPath/problems}

\newpage

\section{Scripts}


\subsection{\whatIsTitle: $3 \times 3$ Matrix Example}

{\ttfamily
\fontdimen2\font=0.4em
\fontdimen3\font=0.2em
\fontdimen4\font=0.1em
\fontdimen7\font=0.1em
\hyphenchar\font=`\-

\hypertarget{scripts_what_is_linear_algebra_3_3_matrix}{Your friend places a jar} on a table and tells you that there is 65 cents in this jar with 7 coins consisting of quarters, nickels, and dimes, and that there are twice as many dimes as quarters. Your friend wants to know how many nickels, dimes, and quarters are in the jar.

We can translate this into a system of the following linear equations:
\begin{align*}
5n + 10d + 25q & = 65
\\ n + d + q & = 7
\\ d & = 2q
\end{align*}
Now we can rewrite the last equation in the form of $-d + 2q = 0$, and thus express this problem as the matrix equation
\[
\begin{pmatrix}
5 & 10 & 25 \\
1 & 1 & 1 \\
0 & -1 & 2
\end{pmatrix} \begin{pmatrix}n\\d\\q\end{pmatrix} = \begin{pmatrix}65\\7\\0\end{pmatrix}.
\]
or as an \hyperlink{augmented_matrix}{augmented matrix} (see also \hyperlink{script_gaussian_elimination_more}{this script on the notation}).
\[
\begin{pmatrix}
5 & 10 & 25 & \vline & 65\\
1 & 1 & 1 & \vline & 7 \\
0 & -1 & 2 & \vline & 0
\end{pmatrix}
\]
Now to solve it, using our original set of equations and by substitution, we have
\begin{align*}
5n + 20q + 25q = 5n + 45q & = 65
\\ n + 2q + q = n + 3q & = 7
\end{align*}
and by subtracting 5 times the bottom equation from the top, we get
\[
45q - 15q = 30q = 65 - 35 = 30
\]
and hence $q = 1$. Clearly $d = 2$, and hence $n = 7 - 2 - 1 = 4$. Therefore there are four nickels, two dimes, and one quarter.

} % Closing brace for font

\newpage


\subsection*{Hint for Review Problem~\ref{QRprob}}

%%%Insert this to get the typewriter font so it looks like a real movie script
{\ttfamily
\fontdimen2\font=0.4em
\fontdimen3\font=0.2em
\fontdimen4\font=0.1em
\fontdimen7\font=0.1em
\hyphenchar\font=`\-


%%%%put a hypertarget around the opening bit of text
\hypertarget{gram_schmidt_and_orthogonal_complements_hint}{This} video shows you a way to solve problem~\ref{QRprob} 
that's different to the \hyperlink{methodQR}{method} described in the Lecture. The first thing is to think of 
$$
M=\begin{pmatrix}1 & 0 & 2 \\ -1 & 2 & 0 \\ -1 & 2 \ & 2\end{pmatrix}
$$
as a set of 3 vectors 
$$
v_1=\begin{pmatrix}0 \\ -1 \\ -1\end{pmatrix}\, ,\qquad
v_2=\begin{pmatrix}0 \\ 2 \\ -2\end{pmatrix}\, ,\qquad
v_3=\begin{pmatrix}2 \\ 0 \\ 2\end{pmatrix}\, .
$$
Then you need to remember that we are searching for a decomposition
$$M=QR$$ 
where $Q$ is an orthogonal matrix. Thus the upper triangular matrix $R = Q^T M$ and $Q^T Q = I$.
Moreover, orthogonal matrices perform rotations. To see this compare the inner product $u\dotprod v = u^T v$ of vectors $u$ and $v$
with that of $Qu$ and $Qv$:
$$
(Qu)\dotprod (Qv) = (Qu)^T (Qv) = u^T Q^T Q v = u^T v = u\dotprod v\, .
$$  
Since the dot product doesn't change, we learn that $Q$ does not change angles or lengths of vectors.

Now, here's an interesting procedure: rotate $v_1, v_2$ and $v_3$ such that $v_1$ is along the $x$-axis, $v_2$ is in the $xy$-plane.
Then if you put these in a matrix you get something of the form
$$
\begin{pmatrix}a & b & c \\ 0 & d & e \\ 0  & 0 & f\end{pmatrix}
$$
which is exactly what we want for $R$!

Moreover, the vector 
$$
\begin{pmatrix}
a \\ 0 \\ 0
\end{pmatrix}
$$
is the rotated $v_1$ so must have length $||v_1|| = \sqrt{3}$. Thus $a= \sqrt{3}$. 

The rotated $v_2$ is
$$
\begin{pmatrix}
b \\ d \\ 0
\end{pmatrix}
$$
and must have length $||v_2||=2\sqrt{2}$. Also the dot product between  
$$
\begin{pmatrix}
a \\ 0 \\ 0
\end{pmatrix}
\mbox{ and }
\begin{pmatrix}
b \\ d \\ 0
\end{pmatrix}
$$
is $ab$ and
must equal $v_1\dotprod v_2=0$. (That $v_1$ and $v_2$ were orthogonal is just a coincidence here... .) Thus $b=0$.
So now we know most of the matrix $R$
$$
R=\begin{pmatrix}\sqrt{3} & 0 & c \\ 0 & 2\sqrt{2} & e \\ 0  & 0 & f\end{pmatrix}\, .
$$
You can work out the last column using the same ideas. Thus it only remains to compute $Q$ from
$$
Q=M R^{-1}\, .
$$

 
%%%%don't forget to close the bracket so the stuff after your file doesn't look like a movie!
}

%\newpage

%\subsection*{\gaussElimTitle: Augmented Matrix Notation}
\subsection*{Augmented Matrix Notation}

%%%Insert this to get the typewriter font so it looks like a real movie script
{\ttfamily
\fontdimen2\font=0.4em
\fontdimen3\font=0.2em
\fontdimen4\font=0.1em
\fontdimen7\font=0.1em
\hyphenchar\font=`\-


%%%%put a hypertarget around the opening bit of text
\hypertarget{script_gaussian_elimination_more}{Why is the augmented  matrix} 
$$ \left( \begin{array}{cc | c}
1 & 1 & 27 \\
2 & -1 & 0  
\end{array} \right)\, ,
$$
equivalent to the system of equations
\begin{eqnarray*}
 x+y &=& 27\\
 2x - y &=& 0\, ?
\end{eqnarray*}
Well the augmented matrix is just a new notation for the matrix equation
\begin{equation*}
    \begin{pmatrix}
      1             &1  \\
      2             &-1
    \end{pmatrix}
  \colvec{x \\ y}
  =
  \colvec{27 \\ 0}
\end{equation*}
and if you review your matrix multiplication remember that 
\begin{equation*}
    \begin{pmatrix}
      1             &1  \\
      2             &-1
    \end{pmatrix}
  \colvec{x \\ y}
  =
  \colvec{x+ y \\ 2x - y}
\end{equation*}
This means that

\begin{equation*}
  \colvec{x+ y \\ 2x - y}
  =
  \colvec{27 \\ 0}\, ,
\end{equation*}
which is our original equation.

%%%%don't forget to close the bracket so the stuff after your file doesn't look like a movie!
}



\subsection*{Equivalence of Augmented Matrices}

%%%Insert this to get the typewriter font so it looks like a real movie script
{\ttfamily
\fontdimen2\font=0.4em
\fontdimen3\font=0.2em
\fontdimen4\font=0.1em
\fontdimen7\font=0.1em
\hyphenchar\font=`\-


%%%%put a hypertarget around the opening bit of text
\hypertarget{script_gaussian_elimination_background}{Lets think about what it means for the two augmented matrices} 


$$ \left( \begin{array}{cc | c}
1 & 1 & 27 \\
2 & -1 & 0  
\end{array} \right)
\mbox{ and } \left( \begin{array}{cc | c}
1 & 0 & 9 \\
0 & 1 & 18  
\end{array} \right)
$$
to be equivalent:
They are certainly not equal, because they don't match in each component, but since these augmented matrices represent a system, we might want to introduce a new kind of equivalence relation.

Well we could look at the system of linear equations this represents 

\begin{eqnarray*}
 x+y &=& 27\\
 2x - y &=& 0\, 
\end{eqnarray*}
and notice that the solution is $x=9$ and $y=18$. The other augmented matrix represents the system 
\begin{eqnarray*}
 x\ +0 \cdot y &=& 9\\
 0 \cdot x \ +\   \phantom{0 \cdot} y  &=& 18\, 
\end{eqnarray*}
This clearly has the same solution. The first and second system are related in the sense that their solutions are the same. Notice that it is really nice to have the augmented matrix in the second form, because the matrix multiplication can be done in your head.


%%%%don't forget to close the bracket so the stuff after your file doesn't look like a movie!
}

%\newpage

\subsection*{Hint for Review Problem~\ref{QRprob}}

%%%Insert this to get the typewriter font so it looks like a real movie script
{\ttfamily
\fontdimen2\font=0.4em
\fontdimen3\font=0.2em
\fontdimen4\font=0.1em
\fontdimen7\font=0.1em
\hyphenchar\font=`\-


%%%%put a hypertarget around the opening bit of text
\hypertarget{gram_schmidt_and_orthogonal_complements_hint}{This} video shows you a way to solve problem~\ref{QRprob} 
that's different to the \hyperlink{methodQR}{method} described in the Lecture. The first thing is to think of 
$$
M=\begin{pmatrix}1 & 0 & 2 \\ -1 & 2 & 0 \\ -1 & 2 \ & 2\end{pmatrix}
$$
as a set of 3 vectors 
$$
v_1=\begin{pmatrix}0 \\ -1 \\ -1\end{pmatrix}\, ,\qquad
v_2=\begin{pmatrix}0 \\ 2 \\ -2\end{pmatrix}\, ,\qquad
v_3=\begin{pmatrix}2 \\ 0 \\ 2\end{pmatrix}\, .
$$
Then you need to remember that we are searching for a decomposition
$$M=QR$$ 
where $Q$ is an orthogonal matrix. Thus the upper triangular matrix $R = Q^T M$ and $Q^T Q = I$.
Moreover, orthogonal matrices perform rotations. To see this compare the inner product $u\dotprod v = u^T v$ of vectors $u$ and $v$
with that of $Qu$ and $Qv$:
$$
(Qu)\dotprod (Qv) = (Qu)^T (Qv) = u^T Q^T Q v = u^T v = u\dotprod v\, .
$$  
Since the dot product doesn't change, we learn that $Q$ does not change angles or lengths of vectors.

Now, here's an interesting procedure: rotate $v_1, v_2$ and $v_3$ such that $v_1$ is along the $x$-axis, $v_2$ is in the $xy$-plane.
Then if you put these in a matrix you get something of the form
$$
\begin{pmatrix}a & b & c \\ 0 & d & e \\ 0  & 0 & f\end{pmatrix}
$$
which is exactly what we want for $R$!

Moreover, the vector 
$$
\begin{pmatrix}
a \\ 0 \\ 0
\end{pmatrix}
$$
is the rotated $v_1$ so must have length $||v_1|| = \sqrt{3}$. Thus $a= \sqrt{3}$. 

The rotated $v_2$ is
$$
\begin{pmatrix}
b \\ d \\ 0
\end{pmatrix}
$$
and must have length $||v_2||=2\sqrt{2}$. Also the dot product between  
$$
\begin{pmatrix}
a \\ 0 \\ 0
\end{pmatrix}
\mbox{ and }
\begin{pmatrix}
b \\ d \\ 0
\end{pmatrix}
$$
is $ab$ and
must equal $v_1\dotprod v_2=0$. (That $v_1$ and $v_2$ were orthogonal is just a coincidence here... .) Thus $b=0$.
So now we know most of the matrix $R$
$$
R=\begin{pmatrix}\sqrt{3} & 0 & c \\ 0 & 2\sqrt{2} & e \\ 0  & 0 & f\end{pmatrix}\, .
$$
You can work out the last column using the same ideas. Thus it only remains to compute $Q$ from
$$
Q=M R^{-1}\, .
$$

 
%%%%don't forget to close the bracket so the stuff after your file doesn't look like a movie!
}

%\newpage


\subsection{\gaussElimTitle: $3 \times 3$ Example}

{\ttfamily
\fontdimen2\font=0.4em
\fontdimen3\font=0.2em
\fontdimen4\font=0.1em
\fontdimen7\font=0.1em
\hyphenchar\font=`\-

\hypertarget{scripts_gaussian_elimination_3_3_example}{We'll start with the matrix} from the \hyperlink{scripts_what_is_linear_algebra_3_3_matrix}{What is Linear Algebra: $3 \times 3$ Matrix Example} which was
\[
\begin{pmatrix}
5 & 10 & 25 & \vline & 65\\
1 & 1 & 1 & \vline & 7 \\
0 & -1 & 2 & \vline & 0
\end{pmatrix},
\]
and recall the solution to the problem was $n = 4$, $d = 2$, and $q = 1$. So as a matrix equation we have
\[
\begin{pmatrix}1 & 0 & 0 \\ 0 & 1 & 0 \\ 0 & 0 & 1\end{pmatrix} \begin{pmatrix}n \\ d \\ q\end{pmatrix} = \begin{pmatrix}4 \\ 2 \\ 1 \end{pmatrix}
\]
or as an augmented matrix
\[
\begin{pmatrix}
1 & & & \vline & 4 \\
& 1 & & \vline & 2 \\
& & 1 & \vline & 1
\end{pmatrix}
\]

Note that often in diagonal matrices people will either omit the zeros or write in a single large zero. Now
the first matrix is equivalent to the second matrix and is written as
\[
\begin{pmatrix}
5 & 10 & 25 & \vline & 65\\
1 & 1 & 1 & \vline & 7 \\
0 & -1 & 2 & \vline & 0
\end{pmatrix},
\sim
\begin{pmatrix}
1 & & & \vline & 4 \\
& 1 & & \vline & 2 \\
& & 1 & \vline & 1
\end{pmatrix}
\]
since they have the same solutions.


} % Closing brace for the font

\newpage



\subsection*{$2 \times 2$ Example}

%%%Insert this to get the typewriter font so it looks like a real movie script
{\ttfamily
\fontdimen2\font=0.4em
\fontdimen3\font=0.2em
\fontdimen4\font=0.1em
\fontdimen7\font=0.1em
\hyphenchar\font=`\-


\hypertarget{scripts_eigenvalseigenvects_example}{Here is an example of how to find the eigenvalues} and eigenvectors of a $2 \times 2$ matrix.
\[
M = 
\begin{pmatrix}
4 & 2\\
1 & 3\\ 
\end{pmatrix}.
\]
Remember that an eigenvector $v$ with eigenvalue $\lambda$ for $M$ will be a vector such that $Mv = \lambda v$ i.e. $M(v) - \lambda I (v) = \vec{0}$. When we are talking about a nonzero $v$ then this means that $\det (M - \lambda I) = 0$. We will start by finding the eigenvalues that make this statement true. First we compute
 \[
 \det (M - \lambda I) = \det \left(
\begin{pmatrix}
4 & 2\\
1 & 3\\ 
\end{pmatrix} -
\begin{pmatrix}
\lambda & 0\\
0 & \lambda\\ 
\end{pmatrix} \right)
= 
\det 
\begin{pmatrix}
4-\lambda & 2\\
1 & 3-\lambda \\ 
\end{pmatrix} 
\]
so $ \det (M - \lambda I)= (4-\lambda)(3-\lambda ) - 2\cdot1$. We set this equal to zero to find values of $\lambda$ that make this true:
\[
(4-\lambda)(3-\lambda ) - 2\cdot1 = 10-7\lambda +\lambda^2 = (2-\lambda)(5-\lambda) = 0\, .
\]
This means that $\lambda= 2$ and $\lambda= 5$ are solutions. Now if we want to find the eigenvectors that correspond to these values we look at vectors $v$ such that 
\[
\begin{pmatrix}
4-\lambda & 2\\
1 & 3-\lambda \\ 
\end{pmatrix} v = \vec 0 \, .
\]

For $\lambda= 5$
\[
\begin{pmatrix}
4-5 & 2\\
1 & 3-5 \\ 
\end{pmatrix} \colvec{x\\y} =
\begin{pmatrix}
-1 & 2\\
1 & -2 \\ 
\end{pmatrix} \colvec{x\\y}
= \vec 0 \, .
\]
This gives us the equalities $-x +2y = 0$ and $x -2y = 0$ which both give the line $ y = \frac{1}{2}x$. Any point on this line, so for example $\colvec{2\\1}$, is an eigenvector with eigenvalue $\lambda = 5$.

Now lets find the eigenvector for $\lambda = 2$
\[
\begin{pmatrix}
4-2 & 2\\
1 & 3-2 \\ 
\end{pmatrix} \colvec{x\\y} =
\begin{pmatrix}
2 & 2\\
1 & 1 \\ 
\end{pmatrix} \colvec{x\\y}
= \vec 0 ,
\]
which gives the equalities $2x+2y = 0$ and $x+y = 0$. 
(Notice that these equations are not independent of one another, so our eigenvalue must be correct.)
This means any vector $v =  \colvec{x\\y}$ where $y = -x$ , such as $\colvec{1\\-1}$, or any scalar multiple of this vector , {\it i.e.} any vector on the line $y = -x$ is an eigenvector with eigenvalue 2. This solution could be written neatly as
 \[
\lambda_1 = 5, \, v_1= \colvec{2\\1} \, \text{ and } \lambda_2 = 2, \, v_2=\colvec{1\\-1}.
\]
} % Closing bracket for font

%\newpage



\subsection*{Worked Examples of Gaussian Elimination}

{\ttfamily
\fontdimen2\font=0.4em
\fontdimen3\font=0.2em
\fontdimen4\font=0.1em
\fontdimen7\font=0.1em
\hyphenchar\font=`\-

\hypertarget{scripts_elementary_row_operations_worked_examples}{Let us consider that we are} given two systems of equations that give rise to the following two (augmented) matrices:
\begin{align*}
\begin{pmatrix}
2 & 5 & 2 & 0 & \vline & 2 \\
1 & 1 & 1 & 0 & \vline & 1 \\
1 & 4 & 1 & 0 & \vline & 1
\end{pmatrix}
\quad\quad
\begin{pmatrix}
5 & 2 & \vline & 9 \\
0 & 5 & \vline & 10 \\
0 & 3 & \vline & 6
\end{pmatrix}
\end{align*}
and we want to find the solution to those systems. We will do so by doing Gaussian elimination.

For the first matrix we have
\begin{align*}
\begin{pmatrix}
2 & 5 & 2 & 0 & \vline & 2 \\
1 & 1 & 1 & 0 & \vline & 1 \\
1 & 4 & 1 & 0 & \vline & 1
\end{pmatrix}
\overset{R_1 \leftrightarrow R_2}{\sim} &
\begin{pmatrix}
1 & 1 & 1 & 0 & \vline & 1 \\
2 & 5 & 2 & 0 & \vline & 2 \\
1 & 4 & 1 & 0 & \vline & 1
\end{pmatrix}
\\ \overset{R_2 - 2 R_1 ; R_3 - R_1}{\sim} &
\begin{pmatrix}
1 & 1 & 1 & 0 & \vline & 1 \\
0 & 3 & 0 & 0 & \vline & 0 \\
0 & 3 & 0 & 0 & \vline & 0
\end{pmatrix}
\\ \overset{\frac{1}{3}R_2}{\sim} &
\begin{pmatrix}
1 & 1 & 1 & 0 & \vline & 1 \\
0 & 1 & 0 & 0 & \vline & 0 \\
0 & 3 & 0 & 0 & \vline & 0
\end{pmatrix}
\\ \overset{R_1 - R_2 ; R_3 - 3 R_2}{\sim} &
\begin{pmatrix}
1 & 0 & 1 & 0 & \vline & 1 \\
0 & 1 & 0 & 0 & \vline & 0 \\
0 & 0 & 0 & 0 & \vline & 0
\end{pmatrix}
\end{align*}
\begin{enumerate}[1.]
\item We begin by interchanging the first two rows in order to get a 1 in the upper-left hand corner and avoiding dealing with fractions.

\item Next we subtract row 1 from row 3 and twice from row 2 to get zeros in the left-most column.

\item Then we scale row 2 to have a 1 in the eventual pivot.

\item Finally we subtract row 2 from row 1 and three times from row 2 to get it into Reduced Row  Echelon Form.
\end{enumerate}
Therefore we can write $x = 1 - \lambda$, $y = 0$, $z = \lambda$ and $w = \mu$, or in vector form
\[
\begin{pmatrix}x\\y\\z\\w\end{pmatrix} = \begin{pmatrix}1\\0\\0\\0\end{pmatrix} + \lambda \begin{pmatrix}-1\\0\\1\\0\end{pmatrix} + \mu \begin{pmatrix}0\\0\\0\\1\end{pmatrix}.
\]

Now for the second system we have
\begin{align*}
\begin{pmatrix}
5 & 2 & \vline & 9 \\
0 & 5 & \vline & 10 \\
0 & 3 & \vline & 6
\end{pmatrix}
\overset{\frac{1}{5}R_2}{\sim} &
\begin{pmatrix}
5 & 2 & \vline & 9 \\
0 & 1 & \vline & 2 \\
0 & 3 & \vline & 6
\end{pmatrix}
\\ \overset{R_3 - 3 R_2}{\sim} &
\begin{pmatrix}
5 & 2 & \vline & 9 \\
0 & 1 & \vline & 2 \\
0 & 0 & \vline & 0
\end{pmatrix}
\\ \overset{R_1 - 2 R_2}{\sim} &
\begin{pmatrix}
5 & 0 & \vline & 5 \\
0 & 1 & \vline & 2 \\
0 & 0 & \vline & 0
\end{pmatrix}
\\ \overset{\frac{1}{5}R_1}{\sim} &
\begin{pmatrix}
1 & 0 & \vline & 1 \\
0 & 1 & \vline & 2 \\
0 & 0 & \vline & 0
\end{pmatrix}
\end{align*}
We scale the second and third rows appropriately in order to avoid fractions, then subtract the corresponding rows as before. Finally scale the first row and hence we have $x = 1$ and $y = 2$ as a unique solution.

} % Closing brace for the font

%\newpage



\subsection{\elemRowOpsTitle: Explanation of Proof for Theorem~\ref{GJEunique}}

{\ttfamily
\fontdimen2\font=0.4em
\fontdimen3\font=0.2em
\fontdimen4\font=0.1em
\fontdimen7\font=0.1em
\hyphenchar\font=`\-

\hypertarget{scripts_elementary_row_operations_proof}{The first thing to realize is that} 
there are choices in the Gaussian elimination recipe, so maybe that could lead to two different
RREF's and in turn two different solution sets for the same linear system. But that would be weird,
in fact this Theorem says that this can never happen!

Because this proof comes at the end of the section it is often glossed over, but it is a very important result.
Here's a sketch of what happens in the video:
\begin{center}
\includegraphics[scale=.3]{RREF_unique.jpg}
\end{center}

In words: we start with a linear system and convert it to an augmented matrix. Then, because we are studying a uniqueness
statement, we try a proof by contradiction. That is the  method where to show that a statement is true, you try to demonstrate that
the opposite of the statement leads to a contradiction. Here, the opposite statement to the theorem would be to find
two different RREFs for the same system.

Suppose, therefore, that Alice and Bob do find different RREF augmented matrices called $A$ and $B$. 
Then remove all the non-pivot columns  from $A$ and $B$  until you hit the first column that differs. Record that in the last column
and call the results $\widehat A$ and $\widehat B$. Removing columns
does change the solution sets, but it does not ruin row equivalence, so  $\widehat A$ and $\widehat B$ have the same solution sets.

Now, because we left only the pivot columns (plus the first column that differs) we have
$$\hat{A}=\begin{amatrix}{1}
I_N & a\\
0 & 0
\end{amatrix} \mbox{ and } \hat{B}=\begin{amatrix}{1}
I_N & b\\
0 & 0
\end{amatrix}\, ,$$ where $I_N$ is an identity matrix and $a$ and $b$ are column vectors.
Importantly, by assumption,
$$
a\neq b\, .
$$
So if we try to wrote down the solution sets for $\widehat A$ and $\widehat B$ they would be different.
But at all stages, we only performed operations that kept Alice's solution set the same as Bob's.
This is a contradiction so the proof is complete.
} % Closing brace for the font

\newpage


\subsection*{Hint for Review Problem~\ref{QRprob}}

%%%Insert this to get the typewriter font so it looks like a real movie script
{\ttfamily
\fontdimen2\font=0.4em
\fontdimen3\font=0.2em
\fontdimen4\font=0.1em
\fontdimen7\font=0.1em
\hyphenchar\font=`\-


%%%%put a hypertarget around the opening bit of text
\hypertarget{gram_schmidt_and_orthogonal_complements_hint}{This} video shows you a way to solve problem~\ref{QRprob} 
that's different to the \hyperlink{methodQR}{method} described in the Lecture. The first thing is to think of 
$$
M=\begin{pmatrix}1 & 0 & 2 \\ -1 & 2 & 0 \\ -1 & 2 \ & 2\end{pmatrix}
$$
as a set of 3 vectors 
$$
v_1=\begin{pmatrix}0 \\ -1 \\ -1\end{pmatrix}\, ,\qquad
v_2=\begin{pmatrix}0 \\ 2 \\ -2\end{pmatrix}\, ,\qquad
v_3=\begin{pmatrix}2 \\ 0 \\ 2\end{pmatrix}\, .
$$
Then you need to remember that we are searching for a decomposition
$$M=QR$$ 
where $Q$ is an orthogonal matrix. Thus the upper triangular matrix $R = Q^T M$ and $Q^T Q = I$.
Moreover, orthogonal matrices perform rotations. To see this compare the inner product $u\dotprod v = u^T v$ of vectors $u$ and $v$
with that of $Qu$ and $Qv$:
$$
(Qu)\dotprod (Qv) = (Qu)^T (Qv) = u^T Q^T Q v = u^T v = u\dotprod v\, .
$$  
Since the dot product doesn't change, we learn that $Q$ does not change angles or lengths of vectors.

Now, here's an interesting procedure: rotate $v_1, v_2$ and $v_3$ such that $v_1$ is along the $x$-axis, $v_2$ is in the $xy$-plane.
Then if you put these in a matrix you get something of the form
$$
\begin{pmatrix}a & b & c \\ 0 & d & e \\ 0  & 0 & f\end{pmatrix}
$$
which is exactly what we want for $R$!

Moreover, the vector 
$$
\begin{pmatrix}
a \\ 0 \\ 0
\end{pmatrix}
$$
is the rotated $v_1$ so must have length $||v_1|| = \sqrt{3}$. Thus $a= \sqrt{3}$. 

The rotated $v_2$ is
$$
\begin{pmatrix}
b \\ d \\ 0
\end{pmatrix}
$$
and must have length $||v_2||=2\sqrt{2}$. Also the dot product between  
$$
\begin{pmatrix}
a \\ 0 \\ 0
\end{pmatrix}
\mbox{ and }
\begin{pmatrix}
b \\ d \\ 0
\end{pmatrix}
$$
is $ab$ and
must equal $v_1\dotprod v_2=0$. (That $v_1$ and $v_2$ were orthogonal is just a coincidence here... .) Thus $b=0$.
So now we know most of the matrix $R$
$$
R=\begin{pmatrix}\sqrt{3} & 0 & c \\ 0 & 2\sqrt{2} & e \\ 0  & 0 & f\end{pmatrix}\, .
$$
You can work out the last column using the same ideas. Thus it only remains to compute $Q$ from
$$
Q=M R^{-1}\, .
$$

 
%%%%don't forget to close the bracket so the stuff after your file doesn't look like a movie!
}

%\newpage


\subsection*{Planes}

%%%Insert this to get the typewriter font so it looks like a real movie script
{\ttfamily
\fontdimen2\font=0.4em
\fontdimen3\font=0.2em
\fontdimen4\font=0.1em
\fontdimen7\font=0.1em
\hyphenchar\font=`\-


%%%%put a hypertarget around the opening bit of text
\hypertarget{solution_sets_for_systems_of_linear_equations_planes}{Here we want} to describe the mathematics of planes in space.
The video is summarised by the following picture:
\begin{center}
\includegraphics[scale=.2]{plane1eq.jpg}
\end{center}
A plane is often called ${\mathbb R}^2$ because it is spanned by  two coordinates, and space is called ${\mathbb R}^3$ and has three coordinates, 
usually called $(x,y,z)$. The equation for a plane is
$$
ax+by+cz=d\, .
$$
Lets simplify this by calling $V=(x,y,z)$ the vector of unknowns and $N=(a,b,c)$. Using the dot product in ${\mathbb R}^3$
we have
$$
N\dotprod V = d\, .
$$
Remember that when vectors are perpendicular their dot products vanish. {\it I.e.} $U\dotprod V = 0 \Leftrightarrow U \perp V$.
This means that if a vector $V_0$ solves our equation $N\dotprod V =d$, then so too does $V_0+C$ whenever $C$ is perpendicular to $N$.
This is because
$$N\dotprod (V_0+C) = N\dotprod V_0 + N\dotprod C = d + 0 = d\, .$$
But $C$ is ANY vector perpendicular to $N$, so all the possibilities for $C$ span a plane whose normal vector is $N$. Hence we have shown that 
solutions to the equation $ax+by+cz=0$ are a plane with normal vector $N=(a,b,c)$.



%%%%don't forget to close the bracket so the stuff after your file doesn't look like a movie!
}

%\newpage



\subsection*{Pictures and Explanation}

%%%Insert this to get the typewriter font so it looks like a real movie script
{\ttfamily
\fontdimen2\font=0.4em
\fontdimen3\font=0.2em
\fontdimen4\font=0.1em
\fontdimen7\font=0.1em
\hyphenchar\font=`\-


%%%%put a hypertarget around the opening bit of text
\hypertarget{solution_sets_for_systems_of_linear_equations_overview}{This video considers solutions sets}
for linear systems with three unknowns. These are often called $(x,y,z)$ and label points in ${\mathbb R}^3$.
Lets work case by case:

\begin{itemize}
\item If you have no equations at all, then any $(x,y,z)$ is a solution, so the solution set is  all of~${\mathbb R}^3$.
The picture looks a little silly:
\begin{center}
\includegraphics[scale=.15]{all_of_R3.jpg}
\end{center}
\item For a single equation, the solution is a plane. This is explained in this \href{\videourl solution_sets_for_systems_of_linear_equations_planes.mp4}{video}
or the accompanying \hyperlink{solution_sets_for_systems_of_linear_equations_planes}{script}. The picture looks like this:
\begin{center}
\includegraphics[scale=.18]{plane_in_R3.jpg}
\end{center}
\item For two equations, we must look at two planes. These usually intersect along a line, so the solution set will also (usually) be a line:~\begin{center}
\includegraphics[scale=.18]{two_planes_in_R3.jpg}
\end{center}
\item For three equations, most often their intersection will be a single point so the solution will then be unique:
\begin{center}
\includegraphics[scale=.17]{three_planes_in_R3.jpg}
\end{center}
\item Of course stuff can go wrong. Two different looking equations could determine the same plane, or worse equations could be inconsistent.
If the equations are inconsistent, there will be no solutions at all. For example, if you had four equations determining four parallel planes the solution set would be empty. This looks like this:
\begin{center}
\includegraphics[scale=.16]{four_planes_in_R3.jpg}
\end{center}
\end{itemize}



%%%%don't forget to close the bracket so the stuff after your file doesn't look like a movie!
}

%\newpage



\subsection*{$2 \times 2$ Example}

%%%Insert this to get the typewriter font so it looks like a real movie script
{\ttfamily
\fontdimen2\font=0.4em
\fontdimen3\font=0.2em
\fontdimen4\font=0.1em
\fontdimen7\font=0.1em
\hyphenchar\font=`\-


\hypertarget{scripts_eigenvalseigenvects_example}{Here is an example of how to find the eigenvalues} and eigenvectors of a $2 \times 2$ matrix.
\[
M = 
\begin{pmatrix}
4 & 2\\
1 & 3\\ 
\end{pmatrix}.
\]
Remember that an eigenvector $v$ with eigenvalue $\lambda$ for $M$ will be a vector such that $Mv = \lambda v$ i.e. $M(v) - \lambda I (v) = \vec{0}$. When we are talking about a nonzero $v$ then this means that $\det (M - \lambda I) = 0$. We will start by finding the eigenvalues that make this statement true. First we compute
 \[
 \det (M - \lambda I) = \det \left(
\begin{pmatrix}
4 & 2\\
1 & 3\\ 
\end{pmatrix} -
\begin{pmatrix}
\lambda & 0\\
0 & \lambda\\ 
\end{pmatrix} \right)
= 
\det 
\begin{pmatrix}
4-\lambda & 2\\
1 & 3-\lambda \\ 
\end{pmatrix} 
\]
so $ \det (M - \lambda I)= (4-\lambda)(3-\lambda ) - 2\cdot1$. We set this equal to zero to find values of $\lambda$ that make this true:
\[
(4-\lambda)(3-\lambda ) - 2\cdot1 = 10-7\lambda +\lambda^2 = (2-\lambda)(5-\lambda) = 0\, .
\]
This means that $\lambda= 2$ and $\lambda= 5$ are solutions. Now if we want to find the eigenvectors that correspond to these values we look at vectors $v$ such that 
\[
\begin{pmatrix}
4-\lambda & 2\\
1 & 3-\lambda \\ 
\end{pmatrix} v = \vec 0 \, .
\]

For $\lambda= 5$
\[
\begin{pmatrix}
4-5 & 2\\
1 & 3-5 \\ 
\end{pmatrix} \colvec{x\\y} =
\begin{pmatrix}
-1 & 2\\
1 & -2 \\ 
\end{pmatrix} \colvec{x\\y}
= \vec 0 \, .
\]
This gives us the equalities $-x +2y = 0$ and $x -2y = 0$ which both give the line $ y = \frac{1}{2}x$. Any point on this line, so for example $\colvec{2\\1}$, is an eigenvector with eigenvalue $\lambda = 5$.

Now lets find the eigenvector for $\lambda = 2$
\[
\begin{pmatrix}
4-2 & 2\\
1 & 3-2 \\ 
\end{pmatrix} \colvec{x\\y} =
\begin{pmatrix}
2 & 2\\
1 & 1 \\ 
\end{pmatrix} \colvec{x\\y}
= \vec 0 ,
\]
which gives the equalities $2x+2y = 0$ and $x+y = 0$. 
(Notice that these equations are not independent of one another, so our eigenvalue must be correct.)
This means any vector $v =  \colvec{x\\y}$ where $y = -x$ , such as $\colvec{1\\-1}$, or any scalar multiple of this vector , {\it i.e.} any vector on the line $y = -x$ is an eigenvector with eigenvalue 2. This solution could be written neatly as
 \[
\lambda_1 = 5, \, v_1= \colvec{2\\1} \, \text{ and } \lambda_2 = 2, \, v_2=\colvec{1\\-1}.
\]
} % Closing bracket for font

%\newpage


\subsection*{Hint for Review Problem~\ref{QRprob}}

%%%Insert this to get the typewriter font so it looks like a real movie script
{\ttfamily
\fontdimen2\font=0.4em
\fontdimen3\font=0.2em
\fontdimen4\font=0.1em
\fontdimen7\font=0.1em
\hyphenchar\font=`\-


%%%%put a hypertarget around the opening bit of text
\hypertarget{gram_schmidt_and_orthogonal_complements_hint}{This} video shows you a way to solve problem~\ref{QRprob} 
that's different to the \hyperlink{methodQR}{method} described in the Lecture. The first thing is to think of 
$$
M=\begin{pmatrix}1 & 0 & 2 \\ -1 & 2 & 0 \\ -1 & 2 \ & 2\end{pmatrix}
$$
as a set of 3 vectors 
$$
v_1=\begin{pmatrix}0 \\ -1 \\ -1\end{pmatrix}\, ,\qquad
v_2=\begin{pmatrix}0 \\ 2 \\ -2\end{pmatrix}\, ,\qquad
v_3=\begin{pmatrix}2 \\ 0 \\ 2\end{pmatrix}\, .
$$
Then you need to remember that we are searching for a decomposition
$$M=QR$$ 
where $Q$ is an orthogonal matrix. Thus the upper triangular matrix $R = Q^T M$ and $Q^T Q = I$.
Moreover, orthogonal matrices perform rotations. To see this compare the inner product $u\dotprod v = u^T v$ of vectors $u$ and $v$
with that of $Qu$ and $Qv$:
$$
(Qu)\dotprod (Qv) = (Qu)^T (Qv) = u^T Q^T Q v = u^T v = u\dotprod v\, .
$$  
Since the dot product doesn't change, we learn that $Q$ does not change angles or lengths of vectors.

Now, here's an interesting procedure: rotate $v_1, v_2$ and $v_3$ such that $v_1$ is along the $x$-axis, $v_2$ is in the $xy$-plane.
Then if you put these in a matrix you get something of the form
$$
\begin{pmatrix}a & b & c \\ 0 & d & e \\ 0  & 0 & f\end{pmatrix}
$$
which is exactly what we want for $R$!

Moreover, the vector 
$$
\begin{pmatrix}
a \\ 0 \\ 0
\end{pmatrix}
$$
is the rotated $v_1$ so must have length $||v_1|| = \sqrt{3}$. Thus $a= \sqrt{3}$. 

The rotated $v_2$ is
$$
\begin{pmatrix}
b \\ d \\ 0
\end{pmatrix}
$$
and must have length $||v_2||=2\sqrt{2}$. Also the dot product between  
$$
\begin{pmatrix}
a \\ 0 \\ 0
\end{pmatrix}
\mbox{ and }
\begin{pmatrix}
b \\ d \\ 0
\end{pmatrix}
$$
is $ab$ and
must equal $v_1\dotprod v_2=0$. (That $v_1$ and $v_2$ were orthogonal is just a coincidence here... .) Thus $b=0$.
So now we know most of the matrix $R$
$$
R=\begin{pmatrix}\sqrt{3} & 0 & c \\ 0 & 2\sqrt{2} & e \\ 0  & 0 & f\end{pmatrix}\, .
$$
You can work out the last column using the same ideas. Thus it only remains to compute $Q$ from
$$
Q=M R^{-1}\, .
$$

 
%%%%don't forget to close the bracket so the stuff after your file doesn't look like a movie!
}

%\newpage

\newpage

\section*{Webwork Links}

\reading{1}{1}

\reading{1}{1}

\reading{2}{1}

\reading{2}{2}

\reading{3}{1}

\reading{4}{1}

\reading{4}{2}


\begin{center}\href{http://webwork.math.ucdavis.edu/webwork2/MAT22A-Waldron-Winter-2012/Homework0-Background/}{Background homework set}\end{center}

\begin{center}\href{http://webwork.math.ucdavis.edu/webwork2/MAT22A-Waldron-Winter-2012/Homework1-LinearSystems-waldron}{Linear Systems}\end{center}

\begin{center}\href{http://webwork.math.ucdavis.edu/webwork2/MAT22A-Waldron-Winter-2012/Homework2-SolnSets-Vectors-waldron/}{Solution Sets}\end{center}

\end{document}
