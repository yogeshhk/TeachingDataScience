
\subsection*{$2 \times 2$ Example}

%%%Insert this to get the typewriter font so it looks like a real movie script
{\ttfamily
\fontdimen2\font=0.4em
\fontdimen3\font=0.2em
\fontdimen4\font=0.1em
\fontdimen7\font=0.1em
\hyphenchar\font=`\-


\hypertarget{scripts_eigenvalseigenvects_example}{Here is an example of how to find the eigenvalues} and eigenvectors of a $2 \times 2$ matrix.
\[
M = 
\begin{pmatrix}
4 & 2\\
1 & 3\\ 
\end{pmatrix}.
\]
Remember that an eigenvector $v$ with eigenvalue $\lambda$ for $M$ will be a vector such that $Mv = \lambda v$ i.e. $M(v) - \lambda I (v) = \vec{0}$. When we are talking about a nonzero $v$ then this means that $\det (M - \lambda I) = 0$. We will start by finding the eigenvalues that make this statement true. First we compute
 \[
 \det (M - \lambda I) = \det \left(
\begin{pmatrix}
4 & 2\\
1 & 3\\ 
\end{pmatrix} -
\begin{pmatrix}
\lambda & 0\\
0 & \lambda\\ 
\end{pmatrix} \right)
= 
\det 
\begin{pmatrix}
4-\lambda & 2\\
1 & 3-\lambda \\ 
\end{pmatrix} 
\]
so $ \det (M - \lambda I)= (4-\lambda)(3-\lambda ) - 2\cdot1$. We set this equal to zero to find values of $\lambda$ that make this true:
\[
(4-\lambda)(3-\lambda ) - 2\cdot1 = 10-7\lambda +\lambda^2 = (2-\lambda)(5-\lambda) = 0\, .
\]
This means that $\lambda= 2$ and $\lambda= 5$ are solutions. Now if we want to find the eigenvectors that correspond to these values we look at vectors $v$ such that 
\[
\begin{pmatrix}
4-\lambda & 2\\
1 & 3-\lambda \\ 
\end{pmatrix} v = \vec 0 \, .
\]

For $\lambda= 5$
\[
\begin{pmatrix}
4-5 & 2\\
1 & 3-5 \\ 
\end{pmatrix} \colvec{x\\y} =
\begin{pmatrix}
-1 & 2\\
1 & -2 \\ 
\end{pmatrix} \colvec{x\\y}
= \vec 0 \, .
\]
This gives us the equalities $-x +2y = 0$ and $x -2y = 0$ which both give the line $ y = \frac{1}{2}x$. Any point on this line, so for example $\colvec{2\\1}$, is an eigenvector with eigenvalue $\lambda = 5$.

Now lets find the eigenvector for $\lambda = 2$
\[
\begin{pmatrix}
4-2 & 2\\
1 & 3-2 \\ 
\end{pmatrix} \colvec{x\\y} =
\begin{pmatrix}
2 & 2\\
1 & 1 \\ 
\end{pmatrix} \colvec{x\\y}
= \vec 0 ,
\]
which gives the equalities $2x+2y = 0$ and $x+y = 0$. 
(Notice that these equations are not independent of one another, so our eigenvalue must be correct.)
This means any vector $v =  \colvec{x\\y}$ where $y = -x$ , such as $\colvec{1\\-1}$, or any scalar multiple of this vector , {\it i.e.} any vector on the line $y = -x$ is an eigenvector with eigenvalue 2. This solution could be written neatly as
 \[
\lambda_1 = 5, \, v_1= \colvec{2\\1} \, \text{ and } \lambda_2 = 2, \, v_2=\colvec{1\\-1}.
\]
} % Closing bracket for font

%\newpage
