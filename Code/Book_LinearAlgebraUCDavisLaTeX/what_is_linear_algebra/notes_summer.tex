\chapter{\whatIsTitle?}\label{warmup}



Why is linear algebra important and worthy of your time? 
In short, many very abstract and (seemingly) difficult science and math problems %involving unfamiliar objects 
can be translated into concrete linear algebra problems and in turn equations that involve just numbers arranged as matrices. 
This is a beautiful story, but it can not be told quickly. 
It takes time to get a feel for linear algebra's large domain of applicability. 
Before we can tell you the story we must equip you with some basic matrix skills.  
Unfortunately, for most people (including the authors), 
this introductory period often leads to the 
belief that linear algebra consists 
entirely of dull arithmetic matrix exercises. 
This is an instance of not seeing the forest through for the trees; 
in this analogy, explicit matrix computations are the trees (if not the dirt and mud) in the beautiful rainforest known as linear algebra. 

To  save the reader from the feeling that linear algebra in nothing more than trees 
(or worse, dirt and mud!)  
we want to start with a glimpse at the forest. 
That is, we are going to look at some  seemingly difficult and definitely abstract problems. 
The goal is to hint at ``what is out there" in mathematics and at where you will someday find linear algebra a helpful tool; linear algebra is not just about matrices. 
You may see some new math without the motivation of a word problem here.
This can be intimidating, so have a pencil and paper ready so that you can write out some of the ideas yourself. Ready? Lets split the question ``What is Linear Algebra" in two: ``What is algebra?" and  ``What is linear?"\\  


%\vspace{-.6cm}
%\begin{center}
%\includegraphics[width=14cm,height=6.5cm]{\whatIsPath/forest.jpg}
%\end{center}
%\vspace{-1.5cm}


%%%%%%%%%%%%%%%%%%%%%%%%%%%%%%%%
%%%%%%%%%%%%%%%%%%%%%%%%%%%%%%%%

\newpage
\noindent 
\section{What is algebra?}
Algebra is the study of  questions involving operations on objects. 
\begin{example} (Algebra Questions) \\[-.5cm]
\begin{enumerate}
\item What number~$x$ satisfies~$5x=10$?\\[-.8cm]
\item What vector~$v$ from 3 space satisfies~$ \colvec{1 \\ 1\\ 0} \times v = \colvec{0\\1\\1}$,\\[.0cm]
where~$\times$ is the cross product?\\[-.4cm]
\item What polynomial~$g$ satisfies~$\int_{-1}^1 f(x,y)g(y) dy = h(x)$, \\[.2cm]
 where~$f(x,y):=xy$ and~$h(x):=x$?\\[-.5cm]
\item What polynomial~$f$ satisfies~$x\frac{d}{dx} f(x) -2f(x)=0~$
\item What power series~$f$ satisfies~$x\frac{d}{dx} f(x) -2f(x)=0~$?
\item What what function~$f$ satisfies~$f''(x)+f^3(x) =\sqrt{x}~$?
\end{enumerate}
\end{example}


\noindent
There are two things to notice about  this list: First,  the range of operations in algebra seems to go beyond addition, subtraction, multiplication, and division while the range of possible objects goes beyond numbers. [In fact, its best to think of the objects (numbers) being generalized while trying to maintain
the basic properties of addition, subtraction, {\it etc}..., as far as possible.]
%; in fact, we mean only to hint at the variety of objects and operations found in algebra.   
Second, these very different questions all have the same form. To emphasize this we restate them:
\begin{example} (Algebra \hypertarget{AandD}{Questions} Restated) \\[-.5cm]
% Note to the Editor:
% putting all in Ax=b is BAD! They emotionally want x to stand for a number, not a function,
\begin{enumerate}
\item 
What 
number~$x$ satisfies~$Ax=10$ where~$Ax:=5x?$\\[-.6cm]
\item What vector~$v$ satisfies~$Bv = \colvec{0\\1\\1}$% \\
where ~$Bv:=\colvec{1 \\ 1\\ 0} \times v$?\\[-.4cm]
\item  What polynomial~$g$ satisfies~$Cg=h$ where~$Cg$ is the function defined by \\[.3cm]
$Cg(x)=\int_{-1}^1 f(x,y)g(y) dy = h(x)$ where~$f(x,y):=xy,~h(x):=x$?\\[-.5cm]
\item What polynomial~$f$ satisfies~$Df=0$ 
where~$Df$ is the function \\[.3cm]
$Df(x):=x\frac{d}{dx}f(x)-2f(x)$?\\[-5mm]
\item What function~$f$ satisfies~$Ef=g$ where~$Ef:=f''+f^3$ and~$g(x):=\sqrt{x}$?
\end{enumerate}
\end{example}
\noindent
In particular, 
our main object of study will be equations of the form
\begin{center}
\shabox{$Ax=b$}
\end{center}
%\begin{eqnarray*}
%Ax=b
%\end{eqnarray*}
where ~$A$ is a {\it linear operator} while~$x$ and~$b$ are {\it vectors}.
\reading{1}{1}


%%%%%%%%%%%%%%%%%%%%%%%%%%%%%%%%%%%%

%\newpage
\section{Operators: Functions of Functions}
A few comments on jargon and notation are in order before we move on. ~$D$ takes in functions like~$x^4$ and gives out functions like~$4x^4-2x^4=2x^4$. You could say that~$D$ is a function that takes in functions and gives out functions.  
Alternatively, that awful combination of words can be avoided by using the word 
``{\it operator}" instead of ``function";
~$D$ is an operator that takes in functions and gives out functions. 
 Similarly,~$C$ is an operator that takes in, for example,~$x^3$ and gives out ~$\frac12 x$. 
We hope you can understand why we don't always use the 
parentheses notation for arguments of functions when we think of the function as an operator;~$Dg(x)$ looks much better~$D(g)(x)$.

%Video of operators? 

If we are going to think of~$B$ as an operator too we need to understand the objects it takes in and gives out in a new way.
It takes vectors from 3-space to vectors in 3-space. 
Such vectors  
\hypertarget{vecs as fun}{are really functions;} 
a vector from 3-space~$v$ is a function whose domain is just (the set containing only)~$1,2,$ and~$3$. 


\begin{example}
\noindent
The vector~$v=\colvec{v_1\\v_2\\v_3}=\colvec{7\\14\\21}$ is the function that: \\[.4cm]
\begin{center}
\begin{tabular}{l}
Takes in~$1$ and gives out~$v_1=7$\\[1mm]
Takes in~$2$ and gives out~$v_2=14$\\[1mm]
Takes in~$3$ and gives out~$v_3=21$.\\[1mm]
\end{tabular}
\end{center}
\end{example}

\noindent
This is an explicit definition of a function as opposed to of the more familiar algebraic definition. In analogy to defining a function~$f$ with domain all real numbers by the algebraic statement~$f(x)=7x$, we can define the function~$v$ by~$v_i=7i$. 
The operator~$B$ takes in such a  function and gives out another. For example~$$Bv=\colvec{1\\1\\0}\times \colvec{7\\14\\21}=\colvec{21\\-21\\7}.$$


You can also think of vectors from two space as functions with domain (the set containing only)~$1$ and~$2$. Similarly vectors from n-space are functions of (the set containing only)~$1,2,...,n$. This includes the case~$n=1$; we can even think of~$A$ as an operator. 

\reading{1}{2}



Digest what you can of these ideas now, but don't worry if you feel that some of this new stuff is not sinking in immediately. We will revisit all of these ideas. The point here is not to learn about any of the particular operators~$A,B,C,D,E$ but rather to show you that the world of algebra is much bigger than the study of questions involving just exponentiation, multiplication, division, addition and subtraction of numbers. Let your imagination go wild with our examples as inspiration. Imagine all the possible integral operators like~$C$ and differential operators like~$D$. We hope you will conclude that your adventures in algebra are just beginning, and that there are many very difficult problems out there. 
The tools you will learn in linear algebra will help you answer all four of these example questions in an efficient way with the same method. That is the long story we wish to tell. 


%<picture please: flow chart with algebra problem-> linear or not linear, if linear then-> matrix equation

%\newpage
%%%%%%%%%%%%%%%%%%%%%%%%%%%%%%%%%%
%%%%%%%%%%%%%%%%%%%%%%%%%%%%%%%%%%


\section{What is Linear?} 


Students often ask questions like ``how do I know when to use a particular method"? A question like ``how do I know when to use integration by parts?" is hard to  answer, but in contrast, ``when can I use  linear algebra?" is easy to answer: %for the kinds of questions given above; 
{\it when the operator involved is linear.} \\

\noindent
An operator~$E$ is \hypertarget{asked in chapter 1}{said to be linear} if it has two properties:\\[-.5cm]
\begin{enumerate}
\item {\it Homogeneity:} ``Constants can be pulled out"
$$E(cv)=cEv\, .$$ \\[-.5cm]
\item {\it Additivity:} ``The operator can be distributed"
$$E(v+u)=Ev+Eu\, .$$ 
\end{enumerate}




\noindent
A great example is the derivative operator: for any two functions~$f$,~$g$ and any number~$c$ the derivative operator satisfies\\[-.5cm]
\begin{example} (The derivative operator is linear)\\
\begin{enumerate}
\item ~$\frac{d}{dx} (cf)=c\frac{d}{dx} f$, \\[-.5cm]
\item~$\frac{d}{dx}(f+g)=\frac{d}{dx}f+\frac{d}{dx}g$.\\[-.5cm]
\end{enumerate}
\end{example}
\noindent
We now explicitly verify that~$B$ and~$C$ are linear. You will verify that~$A$ and~$D$ are linear in the \hyperlink{ch1probset}{problem set} that ends this chapter.  \\

\begin{example} Verifying that~$B$ is linear:

We need to check that $B$ is linear acting on any vector $v=\colvec{x\\ y\\ z}$ so we compute~$Bv$
using the cross product
$$
Bv=\colvec{1 \\ 1\\ 0} \times \colvec{x \\ y\\ z}=\colvec{z\\ -z\\y-x}\, .
$$
Now, until we know what a scalar multiple $cv$ of the vector $v$ means, the problem makes no sense. 
Here we make (the logical) choice
$$
c\colvec{x \\ y\\ z} := \colvec{cx\\ cy\\ cz}\, .
$$
Now, using this definition and our result for how $B$ acts on an arbitrary vector we find:
$$
B(cv)=B\colvec{cx\\ cy\\ cz}=\colvec{cz\\-cz\\cy-cx}\, .
$$
We hope that this coincides with $cBv$ which also compute using our result for $Bv$ and the definition
for multiplying a vector by a number:
$$
cBv=c\colvec{z\\ -z\\y-x}=\colvec{cz\\ c(-z)\\c(y-x)}=\colvec{cz\\-cz\\cy-cx}\, .
$$
Thankfully, we got the same result for both $B(cv)$ and $cBv$, so the homogeneity property holds.

We still have to check additivity. Again, this makes no sense until we define what the sum $v+u$ of two arbitrary vectors $v=\colvec{x\\y\\z}$ 
and $u=\colvec{x'\\y'\\z'}$. For that we again make a logical choice
$$
\colvec{x\\y\\z}+\colvec{x'\\y'\\z'}:=\colvec{x+x'\\y+y'\\z+z'}\, .
$$
Again, we have to make two computations: $B(u+v)$ (first add then act with $B$) and $Bu+Bv$ (first act with $B$, then add):
$$
B(u+v)=B\colvec{x+x'\\y+y'\\z+z'}=\colvec{z+z'\\-(z+z')\\(y+y')-(x+x')}=\colvec{z+z'\\-z-z'\\y+y'-x-x'}\, ,
$$
$$
Bu+Bv=B\colvec{x\\y\\z}+B\colvec{x'\\y'\\z'}=\colvec{z\\-z\\y-x}+\colvec{z'\\-z'\\y'-x'}=\colvec{z+z'\\-z-z'\\y+y'-x-x'}\, .
$$
Both computations give the same answer so now we know that $B$ also obeys additivity and can conclude that $B$ is linear.
%AI)~$A(cx)=5cx= c5x=cAx$, \\
%AII) A(x+y)=5(x+y)=5x+5y=Ax+Ay\\
%\item~$B(cv)=\colvec{1 \\ 1\\ 0} \times cv
%\!\!\!\!\!\!\!\!\!\!\!\!\!\!\stackrel{\rm \begin{array}{c}\rm\scriptstyle Using\  the\ rule\\[-2mm] \rm\scriptstyle for\ cross\ products\ of\\[-2mm]\rm
%\scriptstyle scalar\ multiples\\\downarrow\\[1mm]\end{array}}= 
%\!\!\!\!\!\!\!\!\!\!\!\!\!\!c\colvec{1 \\ 1\\ 0} \times v=c\, Bv$\, .\\
%\item~$B(u+v)=\colvec{1 \\ 1\\ 0} \times (u+v)
%\!\!\!\!\!\!\!\!\!\!\!\!\!\!\stackrel{\rm \begin{array}{c}\rm\scriptstyle Using\  the\ rule\\[-2mm] \rm\scriptstyle for\ cross\ products\\[-2mm]\rm
%\scriptstyle of\ sums\\\downarrow\\[1mm]\end{array}}
%=\!\!\!\!\!\!\!\!\!\!\!\!\!\!\colvec{1 \\ 1\\ 0} \times u+\colvec{1 \\ 1\\ 0} \times v =Bu+Bv\, .$\\[.5cm]
%\end{enumerate} 
\end{example} 
The above example looks a little silly, this because it is really just checking something you probably already know about cross products of vectors. Namely, if $u$, $v$ and $w$ are vectors and $c$ is a number then
$$
w\times(cv)=c(w\times v)\quad\mbox{and}\quad  w\times(v+u)=w\times v+w\times u\, .
$$
You might also have noticed how important it was to know what $cu$ and $u+v$ actually meant to even solve the problem.
In fact, one of the reasons linear algebra is so powerful, is that it can be applied to any choice for what $cu$ and $u+v$ mean
that happens to obey a few basic rules. You will learn about these in Chapter~\ref{\vectorSpacesPath}.

Here is another example involving linear operators made from integrals where we assume all the usual rules for integrals. Notice how much easier checking linearity is.
\begin{example} Verifying that~$C$ is linear:\
\begin{enumerate}
\item~$C(cg)(x)= \int_{-1}^1f(x,y) cg(y)dy =c\int_{-1}^1f(x,y) g(y)dy =c\,Cg(x)$\\[.2cm]
\item~$C(g_1+g_2)(x) =  \int_{-1}^1f(x,y) [g_1(y)+g_2(y)]dy\\[.2cm]
\phantom{a}~~~~~
=
\int_{-1}^1f(x,y) g_1(y)dy
+\int_{-1}^1f(x,y)g_2(y)dy
=(Cg_1 +Cg_2)(x)$\\%[.4cm]
\end{enumerate}
\end{example}
%DI)~$D(cf)(x)=(x\frac{d}{dx}-2)cf(x)=c(x\frac{d}{dx}-2)f(x)=cDf(x)$\\[.1cm]
%DII)~$D(f_1+f_2)(x)=xf_1'(x)-2f_1(x) +xf_2'(x)-2f_2(x) =(Df_1+Df_2)(x)$.\\
%You now know how to test if an operator is linear, and have some examples.  
\noindent 
One moral of our story is that if~$A$ is a {\it linear} operator while~$b$ and~$x$ are vectors, then~$Ax=b$ is equivalent to a matrix{\it -type} equation. 
That is why we will begin by studying matrices.\\
% We have only hinted at the wide range of possible examples for linear operators.
This moral does {\it not} apply to non-linear operators:
\begin{example} Show that~$E$ is {\it not} linear.\\[-5mm]
\begin{enumerate}  
\item~$E(5f)=5f''+5^3f^3 \neq 5f''+5f^3=5Ef~$\\[-5mm]
\item Not needed.%\\[-5mm]
\end{enumerate}  
The operator~$E$ is not linear because it is not homogeneous.\
\end{example}

Actually,~$E$ is not additive either, but we did not need to show this. Once either one of the linearity properties is broken, the operator is not linear. (Unfortunately, you are a criminal the moment that you break just one law, no matter if you are a law abiding citizen in every other way!)

\videoscriptlink{Homnotadd.mp4}{A Homogeneous Nonlinear Operator}{scripts_H}

\begin{example} Show that operator~$Y$ that acts on numbers by~$Yx=5x+3$ is not linear. \\ 
Lets check additivity first:
$$
Y(x+y)= 5(x+y)+3=(5x+3)+(5x)\neq (5x+3)+(5x+3)=Yx+Yy\, .
$$
This proves linearity, unfortunately homogeneity fails:
$$Y(cx)=5cx+3=c (5x+3) +3 -3c= c\, Yx +3-3c\neq cYx \mbox{ if } c\neq1\, .$$
Notice that the special case $c=1$ holds (and just says $Y(1.x)=1.Yx$), but the homogeneity
requirement must work for {\it any} $c$. 

The operator~$Y$ is neither homogeneous nor additive, and therefore not linear.
\end{example}

This example might surprise you because the graph of the equation~$y=5x+3$ is a line, and what else could linear mean other than ``line-like"? In fact, there is a linear operator associated with this equation: 
the operator~$L$ that acts on two-vectors as 
$$L\colvec{x\\y}=y-5x.$$ 
The line is then the collection of solutions to 
$$L\colvec{x\\y}=10 \Leftrightarrow y-5x=3 \,.$$ 
Similarly, the  operator~$M$ on 3-space with 
$$M\colvec{x\\y\\z}=2x-3y+5z$$ 
is linear. The set of solutions to 
$$M\colvec{x\\y\\z}=10 \Leftrightarrow 2x-3y+5z=10$$ 
form a plane. Lines and planes are {\it level sets} of linear functions. 

\videoscriptlink{Linear_Operator_on_Lines.mp4}{Linear Operators Preserve Lines}{scripts_H}



%It is one thing to be able to tell which operators are linear and which are not, and it is another to have a feel for what linear operators do. Lets look at three simple example problems to get a feel for this. On the way we will introduce the main tool of linear algebra, the idea of a matrix. \\

%%%%%%%%%%%%%%%%%%%%%%%%%%%%%%%%%%%%%%%%%%%%
%%%%%%%%%%%%%%%%%%%%%%%%%%%%%%%%%%%%%%%%%%%%

\noindent
\section{What is a Matrix?}
Matrices are linear operators of a certain kind.
One way to learn about them is by studying {\it systems of  linear equations}. 


\begin{example} 
A room contains~$x$ bags and~$y$ boxes of fruit. 
Each bag contains 2 apples and 4 bananas and each box contains 6 apples and 8 bananas. 
There are 20 apples and 28 bananas in the room. Find~$x$ and~$y$.
\\

%<pic please>\\

\noindent
The values are the numbers~$x$ and~$y$ that simultaneously make both of the following equations true:
\begin{eqnarray*}
	2\, x + 6\, y & =  & 20 \\
	4\, x + 8\, y & = & 28\, .
\end{eqnarray*}
\end{example}
Here we have an example of a \emph{System of Linear Equations}\index{Linear System!concept of}.  It's a collection of equations in which variables are multiplied by constants and summed, and no variables are multiplied together:  There are no powers of variables greater than one (like~$x^2$ or~$b^5$), non-integer or negative powers of variables (like~$y^{-1/2}$ or~$a^{\pi}$), and no places where variables are multiplied together (like~$ab$ or~$xy$).
%\begin{center}\href{\webworkurl ReadingHomework1/1/}{Reading homework: problem 1.1}\end{center}
\reading{1}{3}

%The system of equations above has the feel of multiplication of contents per package by number of packages. 
%What we have is a function that takes in two numbers (number of bags and number of boxes) and gives out two numbers (number of apples and number of bananas.) 
%This function is linear: double the number of bags and the number of boxes and you will double the number of apples and number of bananas. Same for tripling, quadrupling, etc...
%
%An important idea underlies the following observation. 
\noindent
Information about the fruity contents of the room can be stored two ways: 
\begin{enumerate}[(i)]
\item In terms of the number of apples and bananas. 
\item In terms of the number of bags and boxes. 
\end{enumerate}
Intuitively, knowing the information in one form allows you to figure out the information in the other form. 
Going from~(ii) to~(i) is easy: 
If you knew there were~3 bags and~2 boxes it would be easy to calculate the number of apples and bananas, and doing so would have the feel of multiplication (containers times fruit per container). 
In the example above we are required to go the other direction, from~(i) to~(ii). This  feels like the opposite of multiplication, {\it i.e.}, division. Matrix notation will 
make clear what we are ``dividing'' by. 

One of our goals is to learn to solve systems of linear equations with maximal efficiency. 
%This is going to be a generalization of dividing both sides of the system of equations by...something. 
%To show you what we mean, we will now give you an example of an inefficient method of solving the system from example 3:\\
%
%\noindent
%{\bf Inefficient Method}:\\
%Rearrange the first equation into~$x=10-3y$ and substitute the result into the second equation to obtain~$28=4(10-3y)+8y$. This equation has only one unknown; the jargon is ``$x$ has been eliminated". The equation implies that~$y=3$. That result can be ``back substituted" into either of the original equations, for example the first, to obtain 
%$20=2x+6\cdot 3 \Leftrightarrow x=1$.\\
%
%It is easy to get lost when using this method. Especially when dealing with large systems of linear equations, such as 256 equations in 256 variables. It would be nice if we could lay out our work in a way that resonates with the intuition we have built from solving simpler algebra problems, like the familiar procedure
%\begin{eqnarray*}
%3x+4 &=&7\\
%{\rm Subtract}~&4&\\
%3x&=&3\\
%{\rm divide~by}~&3&\\
%x&=&1 \,.
%\end{eqnarray*}
%That is, lets keep the variables on the left hand side and rewrite our {\it system} of equations after each step. \\
%
%
%\noindent
%{\bf More Efficient Method:}\\
%Divide (both sides of) the second equation by 2 to obtain the equivalent system
%\begin{eqnarray*}
%	2\cdot x + 6\cdot y & = & 20 \\
%	2\cdot x + 4\cdot y  & = & 14 \,.
%\end{eqnarray*}
%Subtract the first equation from the second, to get the equivalent system 
%\begin{eqnarray*}
%	2\cdot x + 6\cdot y & =  & 20  \\
%	 0\cdot x - 2\cdot y & = & -6\, .
%\end{eqnarray*}
%Now add three times the second equation to the first
%\begin{eqnarray*}
%	2\cdot x + 0\cdot y & = & 2 \\
%	0\cdot x - 2\cdot y & = & -6\, .
%\end{eqnarray*}
%At this point the result~$x=1,y=3$ is obvious. Further, the elimination of~$y$ from the first equation and elimination of~$x$ from the second is clear as can be. 
%%This is elimination, a key idea in this course. 
%%The idea here is to follow some algorithm to have just one variable with non-zero coefficient in each equation, and to rewrite the {\it system} of linear equations at each step. 
%
%There is a clear shortcoming to this ``efficient method": we need to rewrite too much at each step! The~$x$'s are rewritten in the same place over and over, and similarly for the~$y$'s and the equal signs. 
%Lets work toward reducing the amount of things we rewrite by combining the pair of equations in example 3 into a singe equation using vectors from 2-space. 
We will do that in the Chapter~\ref{\gaussElimPath}. Lets try to use less ink by writing our system of  equations as a single equation between 2-vectors.
To do that we will need rules for multiplying a 2-vector by a scalar and adding pairs of 2-vectors:
$$
c\colvec{x\\y}:=\colvec{cx\\cy}\, \mbox{ and } \colvec{x\\y}+\colvec{x'\\y'}:=\colvec{x+x'\\y+y'}\, .
$$
Now we can write the following:
%First we reduce the number of appearances of the equal sign:
\begin{eqnarray*}
   \left.
\begin{array}{lr}
   	2\, x + 6\, y  =  20 \\
	4\, x + 8\, y  =  28
     \end{array}
   \right\} 
& \Leftrightarrow&    \colvec{ 2x+6y \\ 4x+8y}  =\colvec{ 20\\ 28}\\[2mm] \nn 
%\, .
%\end{eqnarray*}
%Now we can reduce the number of appearances of the symbols~$x$ and~$y$ using vector addition and scalar multiplication:
%\begin{eqnarray*}
%    \colvec{ 2x+6y \\ 4x+8y}  %=\colvec{ 20\\ 28}
&\Leftrightarrow&
   x \colvec{ 2\\ 4} + y \colvec{ 6\\ 8} =\colvec{ 20\\ 28} 
\end{eqnarray*}
This can be made more compact still by introducing an operator which takes in 2-vectors and gives out 2-vectors. This operator is denoted by an array of numbers that  is called a {\it matrix}:
\begin{equation*}
 {\rm {\bf The~operator}}   \begin{pmatrix}
      2     & 6 \\
      4     & 8
    \end{pmatrix}~{\rm {\bf is~defined~by}}
    \begin{pmatrix}
      2     & 6 \\
      4     & 8
    \end{pmatrix}
  \colvec{x \\ y}
  := x \colvec{ 2\\ 4} + y \colvec{ 6\\ 8} \, .
%  \colvec{20 \\ 28}
\end{equation*}
A similar definition applies to matrices with different numbers (and matrices of different sizes) but we will get to that later. 
In terms of a matrix we can concisely restate our fruity problem:
\begin{example}  (This time in purely mathematical language): \\[.2cm]
What vector~$  \colvec{x \\ y}$ satisfies 
$
    \begin{pmatrix}
      2     & 6 \\
      4     & 8
    \end{pmatrix}
  \colvec{x \\ y}
  =   \colvec{20 \\ 28}
$?
\end{example}
%\\
This is of the same~$Ax=b$ form as our opening examples. 
The matrix encodes fruit per container. The equation is roughly fruit per container times number of containers. To solve for fruit we want to \hypertarget{ch1divide}{somehow divide} by the matrix. We learn how to do this later on, but for now juste want to get a feel for matrices.


Another way to think about the above example is to remember the rule for multiplying a matrix times a vector.
If you have forgotten this, you can actually  guess a good rule by making sure the matrix equation is the same as the system of linear equations.
This would require that
$$
 \begin{pmatrix}
      2     & 6 \\
      4     & 8
    \end{pmatrix}
  \colvec{x \\ y}
  :=   \colvec{2x+6y \\ 4x+8y}
$$
Indeed this is an example of
\hypertarget{ch1vecmult}{the general rule} that you have probably seen before
%\begin{center}
%``Turn the vector sideways, and combine it with the rows of the matrix" \end{center}
\begin{equation*}\label{2x2multiplication}
    \begin{pmatrix}
      p     & q  \\
      r      & s
    \end{pmatrix}
  \colvec{x \\ y}
  :=
  \colvec{px+qy \\ rx+sy}
\end{equation*}\\%[.2cm]
\reading{1}{4}
\noindent
A matrix is an example of a \emph{Linear Transformation}\index{Linear Transformation!concept of}, because it takes one vector and turns it into another in a ``linear'' way.
Of course, we can have much larger matrices if our system has more variables:

\videoscriptlink{what_is_linear_algebra_Possibilities.mp4}{Possibilities for Matrices}{script_what_is_linear_algebra_hint}

\noindent
\hypertarget{Matrices are linear operators}{Matrices are linear operators} 
The statement of this for the matrix in our fruity example looks like
\begin{enumerate}
\item~$\begin{pmatrix}
      2             &6 \\
      4            &8
    \end{pmatrix}
   c \colvec{x \\ y} 
   =c  \begin{pmatrix}
      2             &6 \\
      4            &8
    \end{pmatrix}
   \colvec{a \\ b} ~$
\item
$     \begin{pmatrix}
      2             &6 \\
      4            &8
    \end{pmatrix}
   \left[ \colvec{x \\ y} +\colvec{x' \\ y'} \right] 
   = \begin{pmatrix}
      2             &6 \\
      4            &8
    \end{pmatrix}
\colvec{x \\ y}
   +
    \begin{pmatrix}
      2             &6 \\
      4            &8
    \end{pmatrix}
    \colvec{x' \\ y'}
$
\end{enumerate}
These equalities can be already verified using only the rules we introduced so far.
\begin{example} Verify that 
$\begin{pmatrix}
      2             &6 \\
      4            &8
    \end{pmatrix}$
is a linear operator.

\noindent
Homogeneity:
\begin{equation*}
\begin{split}
\begin{pmatrix}
      2             &6 \\
      4            &8
    \end{pmatrix}
   c \colvec{a \\ b} 
 =
\begin{pmatrix}
      2             &6 \\
      4            &8
    \end{pmatrix}
   \colvec{ca \\ cb} 
 &=
  ca \colvec{2 \\ 4} 
+   
     cb \colvec{6 \\ 8} \\[2mm]&
 = \colvec{2ac \\ 4ac} 
+   
      \colvec{6bc \\ 8bc}=  \underline{\colvec{2ac+6bc\\4ac+8bc}} \end{split}\end{equation*}
    \\[.2cm]
which ought (and does) give the same result as
 \begin{equation*}\begin{split} c  \begin{pmatrix}
      2             &6 \\
      4            &8
    \end{pmatrix}
   \colvec{a \\ b} ~
   =
     c\left[ a \colvec{2 \\ 4} 
+   
     b \colvec{6 \\ 8} \right]
   &=c\left[\colvec{2a\\4a}+\colvec{6b\\8b}\right]\\[2mm]&=c\colvec{2a+6b\\4a+8b} =  \underline{\colvec{2ac+6bc\\4ac+8bc} }
     \end{split}\end{equation*}

\vspace{3mm}
\noindent
Additivity:
\begin{equation*}
\begin{split}
     \begin{pmatrix}
      2             &6 \\
      4            &8
    \end{pmatrix}
   \left[ \colvec{a \\ b} +\colvec{c \\ d} \right] 
   &= 
        \begin{pmatrix}
      2             &6 \\
      4            &8
    \end{pmatrix}
    \colvec{a +c\\ b+d}  
   =
        (a+c) \colvec{2 \\ 4} 
        +
         (b+d) \colvec{6 \\ 8}\\[2mm]&
         =
        \colvec{2(a+c) \\ 4(a+c)} 
        +
        \colvec{6(b+d) \\ 8(b+d)}
             =
       \underline{ \colvec{2a+2c +6b+6d\\ 4a+4c+8b+8d} }
       \end{split}\end{equation*}
 which we need to compare to  
\begin{equation*}
\begin{split}
  \begin{pmatrix}
      2             &6 \\
      4            &8
    \end{pmatrix}
\colvec{a \\ b}
   +
    \begin{pmatrix}
      2             &6 \\
      4            &8
    \end{pmatrix}
    \colvec{c \\ d}
&=
a\colvec{2\\4} + b\colvec{6\\ 8} + c\colvec{2\\4} +d\colvec{6\\8}\\[2mm]&
=\colvec{2a\\4a} + \colvec{6b\\ 8b} + \colvec{2c\\4c} +\colvec{6d\\8d}
=\underline{\colvec{2a+2c +6b+6d\\ 4a+4c+8b+8d} }\, .
\end{split}\end{equation*}
\end{example}

%\noindent

So we have come full circle; matrices are just examples of the kinds of linear operators that appear in algebra problems like those in 
Section 1.1. 
Lets call any equation of the form~$Mu=v$ with~$M$ a matrix, and~$u,v$ vectors a {\it matrix equation}\index{matrix equation}. The first part of this course is dedicated to the study of matrix equations. 
The next Chapter is about solving systems of linear equations, or equivalently matrix equations, with maximal efficiency.

%\section{Where are we going?}
%
%Let us reiterate, linear algebra is about more than just matrices. It is about linear operators. 
%While matrices are among the simplest linear operators, they are the most powerful; as hinted at above, complicated and confusing operators can be re-written as matrices as long as they are linear. Before one can see that, one must have a deep understanding of matrices and their structure. This course is mostly about learning that structure.
%
%To hint at what we meant by ``structure of matrices" lets look back on some math you already know.  
%The fundamental theorem of arithmetic says that any integer can be factored into a product of prime numbers (and bio further). 
%The fundamental theorem of algebra says that polynomials over the complex numbers can be factored into first order polynomials (and no further, e.g.~$x^4-1=(x+1)(x-1)(x+1)(x-1)$). 
%The prime numbers are the building blocks of integers and the first order polynomials are the building blocks of polynomials. We will first work toward answering the question, what are the building blocks of matrices? We will discover in chapter 2 that the building blocks are objects called elementary row operations. In Chapter 3 will categorize these objects in detail.

%This example uses augmented matrices which have not been introduced yet!
%\videoscriptlink{what_is_linear_algebra_3_3_matrix.mp4}{A~$3 \times 3$ matrix example}{scripts_what_is_linear_algebra_3_3_matrix}

%This vid uses the old first page of the text and  might be a bit confusing. 
%\videoscriptlink{what_is_linear_algebra_overview.mp4}{Video Overview}{video_what_is_overview}

%I've opted to use the word operator instead of transformation-David
%The matrix is an example of a \emph{Linear Transformation}\index{Linear Transformation!concept of}, because it takes one vector and turns it into another in a linear way:


%\References{
%Hefferon, Chapter One, Section 1\\
%Beezer, Chapter SLE, Sections WILA and SSLE\\
%Wikipedia, \href{http://en.wikipedia.org/wiki/System_of_linear_equations}{Systems of Linear Equations}
%}



\section{Review Problems}



\begin{enumerate}

\item While performing  Gaussian elimination on these augmented matrices write the full system of equations describing the new rows in terms of the old rows above each equivalence symbol as in  \hyperlink{Keeping track of EROs with equations between rows}{Example}~\ref{Rsystem}. 
$$
\begin{amatrix}{2} 
2 & 2 & 10 \\
1 & 2 & 8 \\
\end{amatrix}
,~
\begin{amatrix}{3} 
1 & 1 & 0 & 5 \\
1 & 1 & \!\!-1& 11 \\
-1 & 1 & 1 & -5 \\ 
\end{amatrix}
$$

%%%%%%%%%%%%%%%%%%%

\item Solve the vector equation by applying ERO matrices to each side of the equation to perform elimination. Show each matrix explicitly as in \hyperlink{Undoing}{Example~\ref{slowly}}.

\begin{eqnarray*}
\begin{pmatrix}
3	&6 	&2 \\ %-3
5 	&9 	&4 \\ %1
2	&4	&2 \\ %0
\end{pmatrix} 
\begin{pmatrix}
 x \\ 
y \\
z 
\end{pmatrix} 
=
\begin{pmatrix}
-3 \\ 
1  \\
0  \\
\end{pmatrix} 
\end{eqnarray*}

%%%%%%%%%%%%%%%%%%%

\item Solve this vector equation by finding the inverse of the matrix through $(M|I)\sim (I|M^{-1})$ and then applying $M^{-1}$ to both sides of the equation. 
\begin{eqnarray*}
\begin{pmatrix}
2	&1 	&1 \\ %9
1 	&1 	&1 \\ %6
1	&1	&2 \\ %7
\end{pmatrix} 
\begin{pmatrix}
 x \\ 
y \\
z 
\end{pmatrix} 
=
\begin{pmatrix}
9 \\ 
6  \\
7  \\
\end{pmatrix} 
\end{eqnarray*}


%%%%%%%%%%%%%%%%%%%

\item Follow the method of  \hyperlink{elldeeeww}{Examples~\ref{factorize} and~\ref{factorizes}} to find the $LU$ and $LDU$ factorization of 
\begin{eqnarray*}
\begin{pmatrix}
3	&3 	&6 \\ %0 %2
3 	&5 	&2 \\ %1 %1
6	&2	&5 \\ %0 %1
\end{pmatrix} .
\end{eqnarray*}



%%%%%%%%%%%%%%%%%%%%

\item 
Multiple matrix equations with the same matrix can be solved simultaneously. 
\begin{enumerate}
\item Solve both systems by performing elimination on just one augmented matrix.
\begin{eqnarray*}
\begin{pmatrix}
2	&-1 	&-1 \\ %0 %2
-1 	&1 	&1 \\ %1 %1
1	&-1	&0 \\ %0 %1
\end{pmatrix} 
\begin{pmatrix}
 x \\ 
y \\
z 
\end{pmatrix} 
=
\begin{pmatrix}
0\\ 
1  \\
0  \\
\end{pmatrix} 
,~
\begin{pmatrix}
2	&-1 	&-1 \\ %0 %2
-1 	&1 	&1 \\ %1 %1
1	&-1	&0 \\ %0 %1
\end{pmatrix} 
\begin{pmatrix}
 a \\ 
b \\
c 
\end{pmatrix} 
=
\begin{pmatrix}
2\\ 
1  \\
1  \\
\end{pmatrix} 
\end{eqnarray*}
\item Give an interpretation of the columns of $M^{-1}$ in $(M|I)\sim (I|M^{-1})$ in terms of solutions to certain systems of linear equations.
\end{enumerate}

%%%%%%%%%%%%%%%%%%%%%%%%

\item How can you convince your fellow students to never make this mistake?
\begin{eqnarray*}
\begin{amatrix}{3} 
1 & 0 & 2 & 3 \\ 
0 & 1 & 2& 3 \\
2 & 0 & 1 & 4 \\
\end{amatrix} 
& 
\stackrel{R_1'=R_1+R_2}{
\stackrel{R_2'=R_1-R_2}{ 
\stackrel{\ R_3'= R_1+2R_2}{\sim}}}
&
\begin{amatrix}{3} 
1 & 1 & 4 & 6 \\
1 & \!\!-1 & 0& 0 \\
1 & 2 & 6 & 9 
\end{amatrix}
\end{eqnarray*}

\item Is $LU$ factorization of a matrix unique?  Justify your answer.


\item[$\infty$.] If you randomly create a matrix by picking numbers out of the blue, it will probably be difficult to perform elimination or factorization; fractions and large numbers will probably be involved. To invent simple problems it is better to start with a simple answer:
\begin{enumerate}
\item Start with any augmented matrix in RREF. Perform EROs to make most of the components non-zero. Write the result on a separate piece of paper and give it to your friend. Ask that friend to find RREF of the augmented matrix you gave them. Make sure they get the same augmented matrix you started with.  
\item Create  an upper triangular matrix $U$ and a lower triangular matrix~$L$ with only $1$s on the diagonal. Give the result to a friend to factor into $LU$ form. 
\item Do the same with an $LDU$ factorization. 
\end{enumerate}
\end{enumerate}

\phantomnewpage



\newpage