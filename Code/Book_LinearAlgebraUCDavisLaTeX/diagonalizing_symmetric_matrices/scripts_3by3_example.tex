
\subsection*{$3\times3$ Example}

%%%Insert this to get the typewriter font so it looks like a real movie script
{\ttfamily
\fontdimen2\font=0.4em
\fontdimen3\font=0.2em
\fontdimen4\font=0.1em
\fontdimen7\font=0.1em
\hyphenchar\font=`\-


\hypertarget{scripts_diagonalizing_symmetric_matrices_3by3_example}{Lets diagonalize the} matrix
\[
M = 
\begin{pmatrix}
1 &2 & 0 \\
2 & 1 & 0 \\
0& 0 & 5  \\
\end{pmatrix}
\] 
If we want to diagonalize this matrix, we should be happy to see that it is symmetric, since this means we will have real eigenvalues, which means factoring won't be too hard.  As an added bonus if we have three distinct eigenvalues the eigenvectors we find will automatically be orthogonal, which means that the inverse of the matrix $P$ will be easy to compute. We can start by finding the eigenvalues of this 
\begin{eqnarray*}
\det \begin{pmatrix}
1-\lambda &\mc2 & \mc0 \\
\mc2 & 1 -\lambda& \mc0 \\
\mc0& \mc0 & 5-\lambda  \\
\end{pmatrix}
 &=& (1-\lambda) 
\begin{vmatrix}
1 -\lambda&\mc 0 \\
\mc0 & 5-\lambda  \\
\end{vmatrix}
\\ & & - \, (2) 
 \begin{vmatrix}
2 & \mc0 \\
0& 5-\lambda  \\
\end{vmatrix}
 + 0 \begin{vmatrix}
2 & 1 -\lambda \\
0& \mc0   \\
\end{vmatrix}\\
&=&(1-\lambda)(1-\lambda)(5-\lambda)+ (-2)(2)(5-\lambda) +0\\
&=&(1-2\lambda +\lambda^2)(5-\lambda) + (-2)(2)(5-\lambda)\\
&=&((1-4 )-2\lambda +\lambda^2)(5-\lambda)\\
&=&(-3-2\lambda +\lambda^2)(5-\lambda)\\
&=&(1+\lambda)(3-\lambda)(5-\lambda)\\
\end{eqnarray*}
So we get $\lambda = -1, 3, 5$ as eigenvectors. First find $v_1$ for $\lambda_1 = -1$
\[
(M+I)\colvec{x\\y\\z}
=
 \begin{pmatrix}
2 &2 & 0 \\
2 & 2& 0 \\
0& 0 & 6 \\
\end{pmatrix}
\colvec{x\\y\\z}
= \colvec{0\\0\\0},
\]
implies that $2x+2y= 0$ and $6z = 0$,which means any multiple of  $v_1 = \colvec{1\\-1\\0}$ is an eigenvector with eigenvalue $\lambda_1 = -1$.
Now for $v_2$ with $\lambda_2 = 3$
\[
(M-3I)\colvec{x\\y\\z}
=
 \begin{pmatrix}
-2 &2 & 0 \\
2 & -2& 0 \\
0& 0 & 4 \\
\end{pmatrix}
\colvec{x\\y\\z}
= \colvec{0\\0\\0},
\]
and we can find that that $v_2 = \colvec{1\\1\\0}$ would satisfy  $-2x+2y= 0$,  $2x-2y= 0$ and $4z = 0$.

Now for $v_3$ with $\lambda_3 = 5$
\[
(M-5I)\colvec{x\\y\\z}
=
 \begin{pmatrix}
-4 &2 & 0 \\
2 & -4& 0 \\
0& 0 & 0  \\
\end{pmatrix}
\colvec{x\\y\\z}
= \colvec{0\\0\\0},
\]
Now we want $v_3$ to satisfy $-4 x+ 2y = 0$ and  $2 x-4 y = 0$, which imply $x=y=0$, but since there are no restrictions on the $z$ coordinate we have $v_3 = \colvec{0\\0\\1}$.

Notice that the eigenvectors form an orthogonal basis. We can create an orthonormal basis by rescaling to make them unit vectors. This will help us because if $P = [v_1,v_2,v_3]$ is created from orthonormal vectors then $P^{-1}=P^T$, which means computing $P^{-1}$ should be easy. So lets say 
\[
v_1 = \colvec{\frac{1}{\sqrt{2}}\\-\frac{1}{\sqrt{2}}\\\mc0}, \, v_2 = \colvec{\frac{1}{\sqrt{2}}\\\frac{1}{\sqrt{2}}\\\mc0}, \text{ and } v_3 = \colvec{0\\0\\1}
\]
so we get 
\[P=
\begin{pmatrix}
\frac{1}{\sqrt{2}} & \frac{1}{\sqrt{2}}& 0\\
-\frac{1}{\sqrt{2}} &\frac{1}{\sqrt{2}}& 0\\
\mc0&\mc0 & 1
\end{pmatrix} \text{ and }
P^{-1}=
 \begin{pmatrix}
\frac{1}{\sqrt{2}} & -\frac{1}{\sqrt{2}}& 0\\
\frac{1}{\sqrt{2}} &\frac{1}{\sqrt{2}}& 0\\
\mc0&\mc0 & 1
\end{pmatrix}
\]
So when we compute $D= P^{-1} M P$ we'll get 
\[
 \begin{pmatrix}
\frac{1}{\sqrt{2}} & -\frac{1}{\sqrt{2}}& 0\\
\frac{1}{\sqrt{2}} &\frac{1}{\sqrt{2}}& 0\\
\mc0&\mc0 & 1
\end{pmatrix}
\begin{pmatrix}
1 &2 & 0 \\
2 & 5 & 0 \\
0& 0 & 5  \\
\end{pmatrix}
\begin{pmatrix}
\frac{1}{\sqrt{2}} & \frac{1}{\sqrt{2}}& 0\\
-\frac{1}{\sqrt{2}} &\frac{1}{\sqrt{2}}& 0\\
\mc0&\mc0 & 1
\end{pmatrix}
= \begin{pmatrix}
-1 &0 & 0 \\
0 & 3 & 0 \\
0& 0 & 5  \\
\end{pmatrix}
\]
} % Closing bracket for font


%\newpage
