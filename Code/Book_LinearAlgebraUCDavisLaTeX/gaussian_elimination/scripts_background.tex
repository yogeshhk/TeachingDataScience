\subsection*{Equivalence of Augmented Matrices}

%%%Insert this to get the typewriter font so it looks like a real movie script
{\ttfamily
\fontdimen2\font=0.4em
\fontdimen3\font=0.2em
\fontdimen4\font=0.1em
\fontdimen7\font=0.1em
\hyphenchar\font=`\-


%%%%put a hypertarget around the opening bit of text
\hypertarget{script_gaussian_elimination_background}{Lets think about what it means for the two augmented matrices} 


$$ \left( \begin{array}{cc | c}
1 & 1 & 27 \\
2 & -1 & 0  
\end{array} \right)
\mbox{ and } \left( \begin{array}{cc | c}
1 & 0 & 9 \\
0 & 1 & 18  
\end{array} \right)
$$
to be equivalent:
They are certainly not equal, because they don't match in each component, but since these augmented matrices represent a system, we might want to introduce a new kind of equivalence relation.

Well we could look at the system of linear equations this represents 

\begin{eqnarray*}
 x+y &=& 27\\
 2x - y &=& 0\, 
\end{eqnarray*}
and notice that the solution is $x=9$ and $y=18$. The other augmented matrix represents the system 
\begin{eqnarray*}
 x\ +0 \cdot y &=& 9\\
 0 \cdot x \ +\   \phantom{0 \cdot} y  &=& 18\, 
\end{eqnarray*}
This clearly has the same solution. The first and second system are related in the sense that their solutions are the same. Notice that it is really nice to have the augmented matrix in the second form, because the matrix multiplication can be done in your head.


%%%%don't forget to close the bracket so the stuff after your file doesn't look like a movie!
}

%\newpage