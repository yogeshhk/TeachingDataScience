
\subsection{\gaussElimTitle: $3 \times 3$ Example}

{\ttfamily
\fontdimen2\font=0.4em
\fontdimen3\font=0.2em
\fontdimen4\font=0.1em
\fontdimen7\font=0.1em
\hyphenchar\font=`\-

\hypertarget{scripts_gaussian_elimination_3_3_example}{We'll start with the matrix} from the \hyperlink{scripts_what_is_linear_algebra_3_3_matrix}{What is Linear Algebra: $3 \times 3$ Matrix Example} which was
\[
\begin{pmatrix}
5 & 10 & 25 & \vline & 65\\
1 & 1 & 1 & \vline & 7 \\
0 & -1 & 2 & \vline & 0
\end{pmatrix},
\]
and recall the solution to the problem was $n = 4$, $d = 2$, and $q = 1$. So as a matrix equation we have
\[
\begin{pmatrix}1 & 0 & 0 \\ 0 & 1 & 0 \\ 0 & 0 & 1\end{pmatrix} \begin{pmatrix}n \\ d \\ q\end{pmatrix} = \begin{pmatrix}4 \\ 2 \\ 1 \end{pmatrix}
\]
or as an augmented matrix
\[
\begin{pmatrix}
1 & & & \vline & 4 \\
& 1 & & \vline & 2 \\
& & 1 & \vline & 1
\end{pmatrix}
\]

Note that often in diagonal matrices people will either omit the zeros or write in a single large zero. Now
the first matrix is equivalent to the second matrix and is written as
\[
\begin{pmatrix}
5 & 10 & 25 & \vline & 65\\
1 & 1 & 1 & \vline & 7 \\
0 & -1 & 2 & \vline & 0
\end{pmatrix},
\sim
\begin{pmatrix}
1 & & & \vline & 4 \\
& 1 & & \vline & 2 \\
& & 1 & \vline & 1
\end{pmatrix}
\]
since they have the same solutions.


} % Closing brace for the font

\newpage
