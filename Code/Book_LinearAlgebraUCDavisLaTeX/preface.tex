
\section*{Preface}

This ``book'' grew out of a series of twenty five lecture notes for a sophomore linear algebra class
taught at the University of California, Davis. The audience was primarily engineering students and
students of pure sciences, some of whom may go on to major in mathematics.
It was motivated by the lack of a book that 
 taught students basic structures of linear algebra without overdoing mathematical rigor
 or becoming a mindless exercise in crunching recipes at the cost of fundamental understanding.
In particular we wanted a book that was suitable for all students, not just math majors, that focussed
on concepts and developing the ability to think in terms of abstract structures in order to address the 
dizzying array of seemingly disparate applications that can all actually be addressed with linear algebra methods.

In addition we had practical concerns. We wanted to offer students a online version of the book for free, both because
we felt it our academic duty to do so, but also because we could seamlessly link an online book to a myriad of other 
resources--in particular WeBWorK exercises and videos. We also wanted to make the LaTeX source available to other instructors
so they could easily customize the material to fit their own needs. Finally, we wanted to restructure the way the course 
was taught, by getting the students to direct most of their effort at more difficult problems where they had to think through
concepts,  present well-thought out logical arguments and learn to turn word problems into ones where the usual array of linear algebra recipes could take over.

\subsection*{How to Use the Book}

At the end of each chapter there is a set of review questions. Our students found these very difficult, mostly because they 
did not know where to begin, rather than needing a clever trick. We designed them this way to ensure that students grappled with 
basic concepts. Our main aim was for students to master these problems, so that we could ask similar high caliber problems on midterm and
final examinations. This meant that we did have to direct resources to grading some of these problems. For this we used two tricks.
First we asked students to hand in more problems than we could grade, and then secretly selected a subset for grading.
Second, because there are more review questions than what an individual student could handle, we split the class into groups of three
or four and assigned the remaining problems to them for grading. Teamwork is a skill our students will need in the workplace;
also it really enhanced their enjoyment of mathematics.

Learning math is like learning to play a violin--many ``technical exercises'' are
necessary before you can really make music! Therefore, each chapter has a set of dedicated WeBWorK ``skills problems'' where students can test that they have mastered basic linear algebra skills. The beauty of WeBWorK is that students get instant feedback and
problems can be randomized, which means that although students are working on the same types of problem, they cannot simply 
tell each other the answer. Instead, we encourage them to explain to one another how to do the WeBWorK exercises. 
Our experience is that this way, students can mostly figure out how to do the WeBWorK problems among themselves, freeing up discussion groups and office hours for weightier issues.  Finally, we really wanted our students to carefully read the book. Therefore, each chapter has several very simple WeBWorK ``reading problems''. These appear as links at strategic places. They are very simple problems that can 
answered rapidly if a student has read the preceding text.

\subsection*{The Material}

We believe the entire book can be taught in twenty five fifty minute lectures to a sophomore audience that has been exposed to a one year 
calculus course. Vector calculus is useful, but not necessary preparation for this book, which attempts to be self-contained.
Key concepts are presented multiple times, throughout the book, often first in a more intuitive setting, and then again 
in a definition, theorem, proof style later on. We do not aim for students to become agile mathematical proof writers, but we 
do expect them to be able to show and explain why key results hold. We also often use the review exercises to let students discover
key results for themselves; before they are presented again in detail later in the book. 

Linear algebra courses run the risk of becoming a conglomeration of learn-by-rote recipes involving arrays filled with numbers.
In the modern computer era, understanding these recipes, why they work, and what they are for is more important than ever.
Therefore, we believe it is crucial to change the students' approach to mathematics right from the beginning of the course. Instead of
them asking us ``what do I do here?'', we want them to ask ``why would I do that?'' This means that students need to start to think in terms of
abstract structures. In particular, they need to rapidly become conversant in sets and functions--the first WeBWorK set will help 
them brush up these~skills.

There is no best order to teach a linear algebra course. The book has been written such that instructors can reorder the chapters (using the LaTeX source) in any (reasonable) order and still have a consistent text. We hammer the notions of abstract vectors and linear transformations hard and  early, while at the same time giving students the basic matrix skills necessary to perform computations.
Gaussian elimination is followed directly by an ``exploration chapter'' on the simplex algorithm to open students minds to problems 
beyond standard linear systems ones. Vectors in ${\mathbb R}^n$ and general vector spaces are presented back to back so that students
are not stranded with the idea that vectors are just ordered lists of numbers. To this end, we also labor the notion of all functions from a set to 
the real numbers. In the same vein linear transformations and matrices are presented hand in hand. Once students see that a linear map
is specified by its action on a limited set of inputs, they can already understand what a basis is.
All the while students are studying linear systems and their solution sets, so after determinants are introduced right after  matrices.
This material can proceed rapidly since elementary matrices were already introduced with Gaussian elimination. Only then is a careful 
discussion of spans, linear independence and dimension given to ready students for a thorough treatment of eigenvectors and
diagonalization. The dimension formula therefore appears quite late, since we prefer not to elevate rote computations of column and
row spaces to a pedestal. The book ends with applications--least squares and singular values. These are a fun way to end any lecture course.
It would also be quite easy to spend any extra time on systems of differential equations and simple Fourier transform problems.

\newpage 
\noindent
One possible distribution of twenty five fifty minute lectures might be:
\begin{center}
\begin{tabular}{lr}
Chapter & Lectures\\ \hline
What is Linear Algebra?&1\\
Systems of Linear Equations&3\\
The Simplex Method&1\\
Vectors in Space, $n$-Vectors&1\\
Vector Spaces&1\\
Linear Transformations&1\\
Matrices&3\\
Determinants&2\\
Subspaces and Spanning Sets&1\\
Linear Independence&1\\
Basis and Dimension&1\\
Eigenvalues and Eigenvectors&2\\
Diagonalization&1\\
Orthonormal Bases and Complements&2\\
Diagonalizing Symmetric Matrices&1\\
Kernel, Range, Nullity, Rank&1\\ 
Least Squares and Singular Values&1\\ \hline\end{tabular}
\end{center}

Creating this book has taken the labor of many people. Special  thanks are due to Katrina Glaeser  and Travis Scrimshaw
for shooting many of the videos and LaTeXing their scripts. Rohit Thomas wrote many of the WeBWorK problems. Bruno 
Nachtergaele and Anne Schilling provided inspiration for creating a free resource for all students of linear algebra.
Dan Comins helped with technical aspects. A University of California online pilot grant helped fund the graduate students
who worked on the project. Most of all we thank our students who found many errors in the book and taught us how to teach this material!

Finally, we admit the book's many shortcomings: clumsy writing, low quality artwork and low-tech video material. We welcome
anybody who wishes to contribute new material---WeBWorK problems, videos, pictures---to make this resource a better one and are glad to hear of any typographical errors,
mathematical fallacies, or simply ideas how to improve the book.

\vspace{1cm}

\noindent
David, Tom, and Andrew


%These linear algebra lecture notes are designed to be presented as twenty five, fifty minute lectures
%suitable for sophomores likely to use the material for applications but still requiring
%a solid foundation in this fundamental branch of mathematics. The main idea of the course is
%to emphasize the concepts of vector spaces and linear transformations as mathematical
%structures that can be used to model the world around us. Once ``persuaded'' of this truth, students
%learn explicit skills such as Gaussian elimination and diagonalization in order that vectors and linear transformations
%become calculational tools, rather than abstract mathematics.
%
%In practical terms, the course aims to produce students who can perform computations with large linear systems
%while at the same time understand the concepts behind these techniques. Often-times when a problem can be
%reduced to one of linear algebra it is ``solved''. These notes 
%do not devote much space to applications (there are already a plethora of textbooks with titles involving some
%permutation of the words ``linear'', ``algebra'' and ``applications''). Instead, they attempt to explain the fundamental
%concepts carefully enough that students will realize for their own selves when the particular application they encounter
%in future studies is ripe for a solution via linear algebra.
%
%There are relatively few worked examples or illustrations in these notes, this material is instead covered by a series
%of ``linear algebra how-to videos''. They can be viewed by clicking on the take one icon~\raisebox{-.2cm}{\includegraphics[scale=.05]{take1.jpg}}.
%The \hyperlink{scripts}{``scripts''} for these movies are found at the end of the notes if students prefer to read this material in a traditional format
%and can be easily reached via the script icon~\raisebox{-.2cm}{\includegraphics[scale=.05]{script.jpg}}. Watch an introductory video below:
%
%\videoscriptlink{intro.mp4}{Introductory Video}{intro}
%
%The notes are designed to be used in conjunction with a set of online homework exercises which help the students read the
%lecture notes and learn 
%basic linear algebra skills. Interspersed among the lecture notes are links to simple online problems that
%test whether students are actively reading the notes. In addition there are two sets of sample midterm problems with solutions as well
%as a sample final exam. 
%There are also a set of ten online assignments which are usually collected weekly.
%The first assignment is designed to ensure familiarity with some basic mathematic notions (sets, functions, logical quantifiers and basic methods of proof). The remaining nine assignments are devoted to the usual matrix and vector gymnastics expected from 
%any sophomore linear algebra class. These exercises are all available at
%
%\begin{quote}
%\href{\webworkurl}{\webworkurl}
%\end{quote}
%
%\noindent
%Webwork is an open source, online homework system which originated at the University of Rochester. It can efficiently check whether a student
%has answered an explicit, typically computation-based, problem correctly. The problem sets chosen to accompany these
%notes could contribute roughly  20\% of a student's grade, and ensure that basic computational skills are mastered.
%Most students rapidly realize that it is best to print out the Webwork assignments and solve them on paper before 
%entering the answers online. Those who do not tend to fare poorly on midterm examinations. We have found that there 
%tend to be relatively few questions from students in office hours about the Webwork assignments. Instead, by assigning 20\%
%of the grade to written assignments drawn from  problems chosen randomly from the review exercises at the end of each lecture, the
%student's focus was primarily on understanding ideas. They range from simple tests of understanding of the material in the lectures
%to more difficult problems, all of them require thinking, rather than blind application of mathematical ``recipes''. 
%Office hour questions reflected this and offered an excellent chance
%to give students tips how to present written answers in a way that would convince the person grading their work that they deserved full credit!
%
%Each lecture concludes with references to the comprehensive online textbooks of Jim Hefferon and Rob Beezer:
%
%\begin{quote}
%\href{http://joshua.smcvt.edu/linearalgebra/}{http://joshua.smcvt.edu/linearalgebra/}\\[5mm]
%\href{http://linear.ups.edu/index.html}{http://linear.ups.edu/index.html}
%\end{quote}
%and the notes are also hyperlinked to Wikipedia where students can rapidly access further details and background material
%for many of the concepts. Videos of linear algebra lectures are available online from at least two sources:
%\begin{itemize}
%\item The Khan Academy, \\ \href{http://www.khanacademy.org/?video#Linear Algebra}{http://www.khanacademy.org/?video\#Linear Algebra}
%\item MIT OpenCourseWare, Professor Gilbert Strang, \\ \href{http://ocw.mit.edu/courses/mathematics/18-06-linear-algebra-spring-2010/video-lectures/}{http://ocw.mit.edu/courses/mathematics/18-06-linear-algebra-spring\\ -2010/video-lectures/}
%\end{itemize}
%There are also an array of useful commercially available texts. A non-exhaustive list includes
%\begin{itemize}
%\item ``Introductory Linear Algebra, An Applied First Course'', B. Kolman and D. Hill, Pearson 2001.
%\item ``Linear Algebra and Its Applications'', David C. Lay, Addison--Weseley 2011.
%\item ``Introduction to Linear Algebra'', Gilbert Strang, Wellesley Cambridge Press 2009.
%\item ``Linear Algebra Done Right'', S. Axler, Springer 1997.
%\item ``Algebra and Geometry'', D. Holten and J. Lloyd, CBRC, 1978.
%\item ``Schaum's Outline of Linear Algebra'', S. Lipschutz and M. Lipson, McGraw-Hill 2008.
%\end{itemize}
%A good strategy is to find your favorite among these in the University Library.
%
%There are many, many useful online math resources. A partial list is given in Appendix~\ref{othersources}.
%
%Students have also started contributing to these notes. Click \hyperlink{student_creations}{here} to see some of their work.
%
%There are many ``cartoon'' type images for the important theorems and formal\ae\ . In a classroom with a projector, a useful technique for instructors
%is to project these using a computer. They provide a colorful relief for  students from (often illegible) scribbles on a blackboard.
%These can be downloaded at:
%
%\begin{center}
%\href{http://math.ucdavis.edu/~linear/lecture_materials}{Lecture Materials}
%\end{center}
%
%There are still many errors in the notes, as well as awkwardly explained concepts. An army of 400 students, Fu Liu, Stephen Pon and Gerry Puckett have already found
%many of them. Rohit Thomas has spent a great deal of time editing these notes and the accompanying webworks and has improved them immeasurably. 
%Katrina Glaeser and Travis Scrimshaw have spent many hours shooting and scripting the how-to videos and taken these notes to a whole new level!
%Anne Schilling shot a great guest video.
%We also thank Captain Conundrum\index{Captain Conundrum} for providing us his solutions to the sample
%midterm and final questions. The review exercises would provide a better survey of what linear algebra really is if there were more ``applied''
%questions.  We welcome your contributions!
%
%\begin{quote}
%Andrew and Tom.
%\end{quote}

 
\newpage
