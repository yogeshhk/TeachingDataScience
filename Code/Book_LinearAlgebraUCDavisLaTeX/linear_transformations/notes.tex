
\chapter{\linTransTitle}
\label{sec:linearTransformation}

The main objects of study in any course in linear algebra are linear functions:

\begin{definition}
A function $L \colon V\rightarrow W$ is {\bf linear} if $V$ and $W$ are vector spaces and 
\begin{center}
\shabox{$
L(ru + sv) = rL(u) + sL(v) $}
\end{center}
 for all $u,v \in V$ and $r,s \in \Re$.
\end{definition}

\Reading{LinearTransformations}{1}
%\begin{center}\href{\webworkurl ReadingHomework7/1/}{Reading homework: problem 7.1}\end{center}


\begin{remark}
We will often refer to linear functions by names like ``linear map''\index{Linear Map}, ``linear operator''\index{Linear Operator} or ``linear transformation''\index{Linear Transformation}. In some contexts
you will also see the name ``homomorphism''\index{Homomorphism} which generally is applied to functions from one kind of set to the same kind of set while respecting any  structures on the sets; linear maps are from vector spaces to vector spaces that respect scalar multiplication and addition, the two structures on vector spaces. It is common to denote a linear function by capital $L$ as a reminder of its linearity, but sometimes we will use just $f$, after all we are just studying very special functions.
\end{remark}

The definition above coincides with the \hyperlink{twopart}{two part} description in Chapter~\ref{warmup};
the case $r=1,s=1$ describes additivity, while  $s=0$ describes homogeneity. 
We are now ready to learn the powerful consequences of linearity.

\section{The Consequence of Linearity}

Now that we have a sufficiently general notion of vector space 
it is time to talk about why linear operators are so special. 
Think about what is required to fully specify a real function of one variable. 
%We typically deal with functions that have simple algebraic descriptions like 
%$f(x)=x^2$ or $g(x)=\ln(x)$. 
%There are many more functions than these. 
%Imagine assigning a random  output to each of the 
%One output for each input.
%from $\Re^\mathbb{N}$. 
One output must be specified for each input. 
That is an infinite amount of information. 

By contrast, even though a linear function can have infinitely many elements in its domain, it is specified by a very small amount of information. 

\begin{example} (One output specifies infinitely many)\\ 
If you know that the function $L$ is linear and that 
$$L\colvec{1\\0}  =\colvec{5\\3}$$ 
then you do not need any more information to figure out 
$$L\colvec{2\\0},~L\colvec{3\\0}~,L\colvec{4\\0},~L\colvec{5\\0} ,{\rm ~\mbox{\it etc}}\ldots, $$ 
because by homogeneity
$$L\colvec{5\\0}=L\left[ 5\colvec{1\\0} \right] = 5 L\colvec{1\\0}=5\colvec{5\\3}=\colvec{25\\15}.$$
In this way an an infinite number of outputs is specified by just one.
\end{example}

\begin{example}(Two outputs in $\mathbb{R}^2$ specifies all outputs)\\
Likewise, if you  know that $L$ is linear and that
$$
L\colvec{1\\0}=\colvec{5\\3} {\rm ~and~} L\colvec{0\\1}= \colvec{2\\2}
$$ 
then you don't need any more information to compute
$$L\colvec{1\\1}$$ because by additivity
$$
L\colvec{1\\1}= L \left[ \colvec{1\\0} + \colvec{0\\1} \right] 
=L\colvec{1\\0} + L \colvec{0\\1} = \colvec{5\\3} +\colvec{2\\2} =\colvec{7\\5}.
$$
In fact, since every vector in $\Re^2$ can be expressed as 
$$
\colvec{x\\y}= x\colvec{1\\0}+y\colvec{0\\1}\, ,
$$ 
we know how $L$ acts on every vector from 
$\Re^2$ by linearity based on just  two pieces of information;
$$
L\colvec{x\\y}
= L \left[ x\colvec{1\\0}+y\colvec{0\\1} \right]
= x L\colvec{1\\0}+y L\colvec{0\\1} 
= x \colvec{5\\3}+ y\colvec{2\\2} =\colvec{5x+2y\\3x+2y}.
$$
Thus, the value of $L$ at infinitely many inputs is completely specified by its value at just two inputs.
(We can see now that $L$ acts in exactly the way the matrix 
$$
\begin{pmatrix}
5&2\\
3&2 \end{pmatrix}
$$
acts on vectors from $\Re^2$.)
\end{example}

\Reading{LinearTransformations}{2}

This is the reason that linear functions are so nice;
they are secretly very simple functions by virtue of two characteristics:
\begin{enumerate}\item They act on vector spaces.
\item They act additively and homogeneously. 
\end{enumerate}


A linear transformation with domain $\Re^3$ is completely specified by the way it acts on the three vectors 
$$
\colvec{1\\0\\0}\, ,\:\colvec{0\\1\\0}\, ,\:\colvec{0\\0\\1}\, .
$$
Similarly, a linear transformation with domain $\Re^n$ is completely specified by its action on the $n$ different $n$-vectors that have exactly one non-zero component, and its matrix form can be read off this information. However, not all linear functions have such nice domains.


%%%%%%%%%%%%%%%%%%%%%%

\section{Linear Functions on Hyperplanes }
It is not always so easy to write a linear operator as a matrix. 
Generally, this will amount to solving a linear systems problem. Examining a linear function whose domain is a hyperplane is instructive.

\begin{example}\label{Vdef} Let $$V=\left\{  c_1\colvec{1\\1\\0} +c_2\colvec{0\\1\\1} \middle| c_1,c_2\in \Re \right\} $$ and consider $L:V\to \Re^3$ be a linear function that obeys 
$$
L\colvec{1\\1\\0 } = \colvec{0\\1\\0},\qquad
L\colvec{0\\1\\1 } = \colvec{0\\1\\0}.
$$
By linearity this specifies the action of $L$ on any vector from $V$ as
$$
L\left[ c_1\colvec{1\\1\\0 } + c_2 \colvec{0\\1\\1} \right]= (c_1+c_2)\colvec{0\\1\\0}.
$$
The domain of $L$ is a plane and its range is the line through the origin in the $x_2$ direction. 


It is not clear how to formulate $L$ as a matrix; 
since
\begin{eqnarray*}
L\ccolvec{c_1\\\!\!c_1+c_2\!\!\\c_2} = 
\begin{pmatrix}
0&0&0\\
1&0&1\\
0&0&0
\end{pmatrix}
\ccolvec{c_1\\\!\!c_1+c_2\!\!\\c_2} =(c_1+c_2)\colvec{0\\1\\0}\, ,
\end{eqnarray*}
{\it or} 
\begin{eqnarray*}
L\ccolvec{c_1\\\!\!c_1+c_2\!\!\\c_2} = 
\begin{pmatrix}
0&0&0\\
0&1&0\\
0&0&0
\end{pmatrix}
\ccolvec{c_1\\\!\!c_1+c_2\!\!\\c_2} =(c_1+c_2)\colvec{0\\1\\0}\, ,
\end{eqnarray*}
you might suspect that  $L$ is equivalent to one of these $3\times3$ matrices. It is not. By the natural domain convention, all $3\times3$ matrices have $\Re^3$ as their domain, and the domain of $L$ is smaller than that. 
When we do realize this $L$ as a matrix it will be as a  $3\times2$ matrix. We can tell because the domain of $L$ is 2 dimensional and the codomain is $3$ dimensional. (You probably already know that the plane has dimension~2, and a line is 1~dimensional, but the careful definition of ``dimension'' takes some work; this is tackled in Chapter~\ref{basisdimension}.) This leads us to write
$$
L\left[ c_1\colvec{1\\1\\0 } + c_2 \colvec{0\\1\\1} \right]=c_1\colvec{0\\1\\0}+c_2\colvec{0\\1\\0}=\begin{pmatrix}0&0\\1&1\\0&0\end{pmatrix}\colvec{c_1\\c_2}\, .
$$
This makes sense, but requires a {\it warning}: The matrix $\begin{pmatrix}0&0\\1&1\\0&0\end{pmatrix}$ specifies $L$  so long as you also provide the information that you are labeling points in the plane $V$
by the two numbers $(c_1,c_2)$.
\end{example} 




%Recall that the key properties of vector spaces are vector addition and scalar multiplication.  Now suppose we have two vector spaces $V$ and $W$ and a map $L$ between them:
%\[
%L \colon V \rightarrow W
%\]
%Now, both $V$ and $W$ have notions of vector addition and scalar multiplication.  It would be ideal if the map $L$ \emph{preserved} these operations.  In other words, if adding vectors and then applying $L$ were the same as applying $L$ to vectors and then adding them.  Likewise, it would be nice if, when multiplying by a scalar, it didn't matter whether we multiplied before or after applying~$L$.  In formulas, this means that for any $u,v \in V$ and $c \in \Re$:
%\begin{eqnarray*}
%L(u+v) &=& L(u)+L(v) \\[2mm]
%L(cv) &= &cL(v)
%\end{eqnarray*}
%
%Combining these two requirements into one equation, we get the definition of a linear function\index{Linear function} or linear transformation\index{Linear transformation}.


%Notice that on the left the addition and scalar multiplication occur in $V$, while on the right the operations occur in $W$.
%This is often called the {\it linearity property}\index{Linearity} of a linear transformation.



%\begin{example}
%Take $L \colon \Re^3\rightarrow \Re^3$ defined by:
%
%\[
%L\colvec{x\\y\\z} = \colvec{ x+y\\y+z\\0 }
%\]
%The domain of $L$ is $\Re^3$. The range is not all of $\Re^3$, but just the plane comprised of vecotrs whose third component is zero. We say that $L$ transforms $\Re^3$ into this plane.\\
%
%We now check linearity.  Call $u = \colvec{x\\y\\z}$ and $v=\colvec{a\\b\\c}$.  
%
%\begin{eqnarray*}
%L(ru + sv) & = & L\left( r \colvec{x\\y\\z} + s \colvec{a\\b\\c} \right) \\
% & = & L\left( \colvec{rx\\ry\\rz} + \colvec{sa\\sb\\sc} \right) \\
% & = & L \colvec{rx+sa\\ry+sb\\rz+sx}  \\
% & = & \colvec{rx+sa+ry+sb\\ry+sb+rz+sx\\0}
%\end{eqnarray*}
%On the other hand,
%
%\begin{eqnarray*}
%rL(u) + sL(v) & = & rL\colvec{x\\y\\z} + sL\colvec{a\\b\\c}\\
% & = & r\colvec{x+y\\y+z\\0} + s\colvec{a+b\\b+c\\0}\\
% & = & \colvec{rx+ry\\ry+rz\\0} + \colvec{sa+sb\\sb+sc\\0}\\
% & = & \colvec{rx+sa+ry+sb\\ry+sb+rz+sx\\0}
%\end{eqnarray*}
%Then the two sides of the linearity requirement are equal, so $L$ is a linear transformation.
%
%We can write the linear transformation $L$ in the previous example using a matrix like so:
%\[
%L\colvec{x\\y\\z} = \colvec{x+y\\y+z\\0} = 
%\begin{pmatrix}
%1 & 1 & 0 \\
%0 & 1 & 1 \\
%0 & 0 & 0 \\
%\end{pmatrix}\colvec{x\\y\\z}
%\]
%
%\begin{center}\href{\webworkurl ReadingHomework7/2/}{Reading homework: problem 7.2}\end{center}
%\end{example}

%\videoscriptlink{linear_transformations_example.mp4}{A linear and non-linear example}{scripts_linear_transformations_example}

\section{Linear Differential Operators}

Your calculus class became much easier when you stopped using the limit definition of the derivative,  learned the power rule, and started using linearity of the derivative operator.

\begin{example}
Let $V$ be the vector space of polynomials of degree 2 or less with standard addition and scalar multiplication;
\[
V := \{a_0\cdot1 + a_1x + a_2 x^2 \, | \,  a_0,a_1,a_2 \in \Re \}
\]
Let $\frac{d}{dx} \colon V\rightarrow V$ be \hypertarget{derivative_linear}{the derivative operator.}  
The following three equations, along with linearity of the derivative operator, allow one to take the derivative of any 2nd degree polynomial:
$$
\frac{d}{dx} 1=0,~\frac{d}{dx}x=1,~\frac{d}{dx}x^2=2x\,. 
$$
In particular
$$
\frac{d}{dx} (a_01 + a_1x + a_2 x^2) = 
 a_0\frac{d}{dx}1 + a_1 \frac{d}{dx} x + a_2 \frac{d}{dx} x^2  
 = 0+a_1+2a_2x.
$$
Thus, the derivative acting any of the infinitely many second order polynomials is determined by its action for just three inputs.
%The full statement of linearity of $\frac{d}{dx}$ is:  
%For 2nd order polynomials $p_1,p_2$ and numbers $r,s$, 
%\[
%\frac{d}{dx}(rp_1 + s p_2) = r \frac{dp_1}{dx} + s \frac{dp_2}{dx} .
%\]
%We can represent a polynomial as a ``semi-infinite vector'', like so:
%\[
%a_0 + a_1x + \cdots + a_nx^n \longleftrightarrow 
%\colvec{ a_0 \\ a_1 \\ \vdots \\ a_n \\ 0 \\ 0 \\ \vdots }
%\]

%Then we have:
%\[
%\frac{d}{dx}(a_0 + a_1x + \cdots + a_nx^n) = a_1 + 2a_2x + \cdots + na_{n}x^{n-1} \\
%\longleftrightarrow 
%\colvec{ a_1 \\ 2a_2 \\ \vdots \\ na_n \\ 0 \\ 0 \\ \vdots }
%\]
%
%One could then write the derivative as an ``infinite matrix'':
%\[
%\frac{d}{dx} \longleftrightarrow 
%\begin{pmatrix}
%0 & 1 & 0 & 0 & \cdots \\
%0 & 0 & 2 & 0 & \cdots \\
%0 & 0 & 0 & 3 & \cdots \\
%\vdots & \vdots & \vdots & \vdots &  \\
%\end{pmatrix}
%\]
\end{example}


\section{Bases (Take 1)} 
The central idea of linear algebra is to exploit the hidden simplicity of linear functions. 
It ends up there is a lot of freedom in how to do this. That freedom is what makes linear algebra powerful.

You saw that a linear operator acting on $\Re^2$ is completely specified by how it acts on the pair of vectors $\colvec{1\\0}$ and $\colvec{0\\1}$. 
In fact, any linear operator acting on $\Re^2$ is also completely specified by how it acts on the pair of vectors $\colvec{1\\1}$ and $\colvec{1\\-1}$.

\begin{example} The linear operator $L$ is a linear operator then it is completely specified \hypertarget{nonstandard r2 basis}{by the two equalities} 
$$
L\colvec{1\\1}= \colvec{2\\4}, {\rm ~and~} L\colvec{1\\-1}=\colvec{6\\8}.
$$ 
This is because any vector $\colvec{x\\y}$ in $\Re^2$ is a sum of multiples of
$\colvec{1\\1}$ and $\colvec{1\\-1}$ which can be calculated via a linear systems problem as follows:
\begin{eqnarray*}&&
\colvec{x\\y}=a\colvec{1\\1}+b\colvec{1\\-1}\\[1mm]
&\Leftrightarrow& 
\begin{pmatrix}
1&1\\
1&-1
\end{pmatrix}
\colvec{a\\b}
=\colvec{x\\y}\\[1mm]
&\Leftrightarrow& 
\begin{amatrix}{2}
1&1&x\\
1&-1&y
\end{amatrix}
\sim \begin{amatrix}{2}
1&0& \frac{x+y}{2}\\
0&1&\frac{x-y}2
\end{amatrix}\\[1mm]
&\Leftrightarrow&
\left\{ 
\begin{array}{l}
a=\frac{x+y}{2}\\
b=\frac{x-y}{2}\, .
\end{array}
\right.
\end{eqnarray*}
Thus
$$
\colvec{x\\[2mm]y}=\frac{x+y}{2}\colvec{1\\[2mm]1}+\frac{x-y}{2}\colvec{1\\[2mm]-1}\, .
$$
We can then calculate how $L$ acts on any vector by first expressing the vector as  a sum of multiples and then applying linearity;
\begin{eqnarray*}
L\colvec{x\\y}
&=&L\left[    \frac{x+y}{2} \colvec{1\\1} + \frac{x-y}{2} \colvec{1\\-1}  \right]\\[1mm]
&=&\frac{x+y}{2} L \colvec{1\\1} + \frac{x-y}{2} L \colvec{1\\-1} \\[2mm]
&=&\frac{x+y}{2} \colvec{2\\4} + \frac{x-y}{2}  \colvec{6\\8} \\[1mm]
&=&\ccolvec{x+y \\ 2(x+y)} +  \colvec{3(x-y)\\4(x-y)}\\[1mm]
&=&\ccolvec{4x-2y \\ 6x-2y}
\end{eqnarray*}
Thus $L$ is completely specified by its value at just two inputs. 
%In fact, we find that $L$ is equivalent to a matrix, 
%$$
%L\colvec{a\\b}=
%\begin{pmatrix}
%4&-2\\
%6&-1
%\end{pmatrix}
%\colvec{a\\b}.
%$$
%but only by virtue of it having domain and targert $\Re^2$.
\end{example}

It should not surprise you to learn there are infinitely many pairs of vectors from $\Re^2$ 
with the property that any vector can be expressed as a linear combination of them; any pair that when used as columns of a matrix gives an invertible matrix works. Such a pair is called a {\it basis}\index{basis} for $\Re^2$.

Similarly, there are infinitely many triples of vectors with the property that any vector from $\Re^3$ can be expressed as a linear combination of them: these are the triples that used as columns of a matrix give an invertible matrix. Such a triple is called a basis for $\Re^3$.

In a similar spirit, there are infinitely many pairs of vectors with the property that every vector in 
$$V=\left\{  c_1\colvec{1\\1\\0} +c_2\colvec{0\\1\\1} \middle\vert \, c_1,c_2\in \Re \right\} $$ 
can be expressed as a linear combination of them. Some examples are 
$$V=
\left\{ c_1\colvec{1\\1\\0} +c_2\colvec{0\\2\\2}  \middle\vert c_1,c_2\in \Re \right\} 
=\left\{c_1 \colvec{1\\1\\0}+c_2 \colvec{1\\3\\2}  \middle\vert c_1,c_2\in \Re \right\} 
%\\
%=\left\{  c_1\colvec{2\\4\\2} +c_2 \colvec{1\\3\\2}  | c_1,c_2\in \Re \right\} 
$$
Such a pair is a  called a basis for $V$.



%\begin{remark}[Foreshadowing Dimension.]

You probably have some intuitive notion of what dimension\index{Dimension!concept of} means
(the careful mathematical definition is given in chapter~\ref{sec:dimension}).
%Some of the examples of vector spaces we have worked with have been finite dimensional.  (For example, $\Re^n$ will turn out to have dimension $n$.)  
%The polynomial example above is an example of an infinite dimensional vector space.  
Roughly speaking, dimension is the number of independent directions available.  To figure out the dimension of a vector space, I stand at the origin, and pick a direction.  If there are any vectors in my vector space that aren't in that direction, then I choose another direction that isn't in the line determined by the direction I chose.  If there are any vectors in my vector space not in the plane determined by the first two directions, then I choose one of them as my next direction.  In other words, I choose a collection of \emph{independent} vectors in the vector space (independent vectors are defined in Chapter~\ref{linearind}).  
A minimal set of independent vectors is called a {\it basis}\index{basis} (see Chapter~\ref{basisdimension} for the precise definition). 
The number of vectors in my basis is the dimension of the vector space. 
Every vector space has many bases, but all bases for a particular vector space have the same number of vectors. Thus dimension is a well-defined concept. 

The fact that every vector space (over~$\Re$) has infinitely many bases is actually very useful. 
Often a good choice of  basis can reduce the time required to run a calculation in dramatic ways! 

In summary:

\begin{center}
\shabox{A basis is a set of vectors in terms of which it is possible to uniquely express any other vector.}
\end{center}
%For finite dimensional vector spaces, linear transformations can always be represented by matrices.  For that reason, we will start studying matrices intensively in the next few lectures.
%\end{remark}


%\section*{References}
%
%Hefferon, Chapter Three, Section II.  (Note that Hefferon uses the term \emph{homomorphism} for a linear map.  `Homomorphism' is a very general term which in mathematics means `Structure-preserving map.'  A linear map preserves the linear structure of a vector space, and is thus a type of homomorphism.)
%\\[2mm]
%Beezer, Chapter LT, Section LT, sections LT, LTC, and MLT.
%\\
%Wikipedia:
%\begin{itemize}
%\item \href{http://en.wikipedia.org/wiki/Linear_transformation}{Linear Transformation}
%\item \href{http://en.wikipedia.org/wiki/Dimension_(linear_algebra)}{Dimension}
%\end{itemize}
%

\section{Review Problems}

{\bf Webwork:} 
\begin{tabular}{|c|c|}
\hline
Reading problems &
\hwrref{LinearTransformations}{1}, \hwrref{LinearTransformations}{2}\\
Linear? & \hwref{LinearTransformations}{3}\\
Matrix $\times$ vector & \hwref{LinearTransformations}{4}, \hwref{LinearTransformations}{5}\\
Linearity & \hwref{LinearTransformations}{6}, \hwref{LinearTransformations}{7}\\
\hline
\end{tabular}



\begin{enumerate}

\item While performing  Gaussian elimination on these augmented matrices write the full system of equations describing the new rows in terms of the old rows above each equivalence symbol as in  \hyperlink{Keeping track of EROs with equations between rows}{Example}~\ref{Rsystem}. 
$$
\begin{amatrix}{2} 
2 & 2 & 10 \\
1 & 2 & 8 \\
\end{amatrix}
,~
\begin{amatrix}{3} 
1 & 1 & 0 & 5 \\
1 & 1 & \!\!-1& 11 \\
-1 & 1 & 1 & -5 \\ 
\end{amatrix}
$$

%%%%%%%%%%%%%%%%%%%

\item Solve the vector equation by applying ERO matrices to each side of the equation to perform elimination. Show each matrix explicitly as in \hyperlink{Undoing}{Example~\ref{slowly}}.

\begin{eqnarray*}
\begin{pmatrix}
3	&6 	&2 \\ %-3
5 	&9 	&4 \\ %1
2	&4	&2 \\ %0
\end{pmatrix} 
\begin{pmatrix}
 x \\ 
y \\
z 
\end{pmatrix} 
=
\begin{pmatrix}
-3 \\ 
1  \\
0  \\
\end{pmatrix} 
\end{eqnarray*}

%%%%%%%%%%%%%%%%%%%

\item Solve this vector equation by finding the inverse of the matrix through $(M|I)\sim (I|M^{-1})$ and then applying $M^{-1}$ to both sides of the equation. 
\begin{eqnarray*}
\begin{pmatrix}
2	&1 	&1 \\ %9
1 	&1 	&1 \\ %6
1	&1	&2 \\ %7
\end{pmatrix} 
\begin{pmatrix}
 x \\ 
y \\
z 
\end{pmatrix} 
=
\begin{pmatrix}
9 \\ 
6  \\
7  \\
\end{pmatrix} 
\end{eqnarray*}


%%%%%%%%%%%%%%%%%%%

\item Follow the method of  \hyperlink{elldeeeww}{Examples~\ref{factorize} and~\ref{factorizes}} to find the $LU$ and $LDU$ factorization of 
\begin{eqnarray*}
\begin{pmatrix}
3	&3 	&6 \\ %0 %2
3 	&5 	&2 \\ %1 %1
6	&2	&5 \\ %0 %1
\end{pmatrix} .
\end{eqnarray*}



%%%%%%%%%%%%%%%%%%%%

\item 
Multiple matrix equations with the same matrix can be solved simultaneously. 
\begin{enumerate}
\item Solve both systems by performing elimination on just one augmented matrix.
\begin{eqnarray*}
\begin{pmatrix}
2	&-1 	&-1 \\ %0 %2
-1 	&1 	&1 \\ %1 %1
1	&-1	&0 \\ %0 %1
\end{pmatrix} 
\begin{pmatrix}
 x \\ 
y \\
z 
\end{pmatrix} 
=
\begin{pmatrix}
0\\ 
1  \\
0  \\
\end{pmatrix} 
,~
\begin{pmatrix}
2	&-1 	&-1 \\ %0 %2
-1 	&1 	&1 \\ %1 %1
1	&-1	&0 \\ %0 %1
\end{pmatrix} 
\begin{pmatrix}
 a \\ 
b \\
c 
\end{pmatrix} 
=
\begin{pmatrix}
2\\ 
1  \\
1  \\
\end{pmatrix} 
\end{eqnarray*}
\item Give an interpretation of the columns of $M^{-1}$ in $(M|I)\sim (I|M^{-1})$ in terms of solutions to certain systems of linear equations.
\end{enumerate}

%%%%%%%%%%%%%%%%%%%%%%%%

\item How can you convince your fellow students to never make this mistake?
\begin{eqnarray*}
\begin{amatrix}{3} 
1 & 0 & 2 & 3 \\ 
0 & 1 & 2& 3 \\
2 & 0 & 1 & 4 \\
\end{amatrix} 
& 
\stackrel{R_1'=R_1+R_2}{
\stackrel{R_2'=R_1-R_2}{ 
\stackrel{\ R_3'= R_1+2R_2}{\sim}}}
&
\begin{amatrix}{3} 
1 & 1 & 4 & 6 \\
1 & \!\!-1 & 0& 0 \\
1 & 2 & 6 & 9 
\end{amatrix}
\end{eqnarray*}

\item Is $LU$ factorization of a matrix unique?  Justify your answer.


\item[$\infty$.] If you randomly create a matrix by picking numbers out of the blue, it will probably be difficult to perform elimination or factorization; fractions and large numbers will probably be involved. To invent simple problems it is better to start with a simple answer:
\begin{enumerate}
\item Start with any augmented matrix in RREF. Perform EROs to make most of the components non-zero. Write the result on a separate piece of paper and give it to your friend. Ask that friend to find RREF of the augmented matrix you gave them. Make sure they get the same augmented matrix you started with.  
\item Create  an upper triangular matrix $U$ and a lower triangular matrix~$L$ with only $1$s on the diagonal. Give the result to a friend to factor into $LU$ form. 
\item Do the same with an $LDU$ factorization. 
\end{enumerate}
\end{enumerate}

\phantomnewpage



\newpage
