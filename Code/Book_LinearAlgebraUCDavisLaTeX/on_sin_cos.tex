\section{Sine and Cosine as an Orthonormal Basis}

\begin{definition}
Let $\Omega \subseteq \mathbb{R}^n$ for some $n$. Let $L_0^p(\Omega)$ denote the space of all continuous functions $f \colon \Omega \rightarrow \mathbb{R}$ (or $\mathbb{C}$) such that if $p < \infty$, then
\[
\left( \int_{\Omega} |f(x)|^p \, dx \right)^{1/p} < \infty,
\]
otherwise $|f(x)| < M$ for some fixed $M$ and all $x \in \Omega$.
\end{definition}
Note that this is a vector space over $\mathbb{R}$ (or $\mathbb{C}$) under addition (in fact it is an \hyperref[algebras]{algebra} under pointwise multiplication) with norm (the length of the vector)
\[
\norm{f}_p = \left( \int_{\Omega} |f(x)|^p \, dx \right)^{1/p}.
\]
For example, the space $L_0^1(\mathbb{R})$ is all absolutely integrable functions. However note that not every differentiable function is contained in $L_0^p(\Omega)$; for example we have
\[
\int_{\mathbb{R}_+} |1|^p \, dx = \int_0^{\infty} dx = \lim_{x \rightarrow \infty} x = \infty.
\]

In particular, we can take $S^1$, the unit circle in $\mathbb{R}^2$, and to turn this into a valid integral, take $\Omega = [0, 2\pi)$ and take functions $f \colon [0, 2\pi] \rightarrow \mathbb{R}$ such that $f(0) = f(2\pi)$ (or more generally for a \emph{periodic} function $f \colon \mathbb{R} \rightarrow \mathbb{R}$ where $f(x) = f(x + 2\pi n)$ for all $n \in \mathbb{Z}$). Additionally we can define an inner product on $\mathcal{H} = L_0^2(S^1)$ by taking
\[
\langle f, g \rangle = \int_0^{2\pi} f(x) g(x) \, dx,
\]
and note that $\langle f, f \rangle = \norm{f}_2^2$. So the natural question to ask is what is a good basis for $\mathcal{H}$? The answer is $\sin(nx)$ and $\cos(nx)$ for all $n \in \mathbb{Z}_{\geq 0}$, and in fact, they are orthogonal. First note that
\begin{align*}
\langle \sin(mx), \sin(nx) \rangle & = \int_0^{2\pi} \sin(mx) \sin(nx) \, dx
\\ & = \int_0^{2\pi} \frac{\cos((m - n)x) - \cos((m + n)x)}{2} \, dx
\end{align*}
and if $m \neq n$, then we have
\[
\langle \sin(mx), \sin(nx) \rangle = \frac{\sin((m - n)x)}{2(m - n)} \biggr\rvert_0^{2\pi} - \frac{\sin((m + n)x)}{2(m + n)} \biggr\rvert_0^{2\pi} = 0.
\]
However if $m = n$, then we have
\[
\langle \sin(mx), \sin(mx) \rangle = \int_0^{2\pi} 1 - \frac{\cos(2mx)}{2} \, dx = 2\pi,
\]
so $\norm{\sin(mx)}_2 = \sqrt{2\pi}$, and similarly we have $\norm{\cos(mx)}_2 = \sqrt{2\pi}$. Finally we have
\begin{align*}
\langle \sin(mx), \cos(nx) \rangle & = \int_0^{2\pi} \sin(mx) \cos(nx) \, dx
\\ & = \int_0^{2\pi} \frac{\sin((m+n)x) + \sin((m-n)x)}{2} \, dx
\\ & = \frac{\cos((m+n)x)}{2(m + n)} \biggr\rvert_0^{2\pi} + \frac{\cos((m-n)x)}{2(m - n)} \biggr\rvert_0^{2\pi} = 0.
\end{align*}
Now it is not immediately apparent that we haven't missed some basis vector, but this is a consequence of the \href{http://en.wikipedia.org/wiki/Stone-Weierstrass_theorem}{Stone-Weierstrauss theorem}. Now only appealing to linear algebra, we have that $e^{inx}$ is a also basis for $L^2(S^1)$ (only over $\mathbb{C}$ though) since
\begin{align*}
\sin(nx) = \frac{e^{inx} - e^{-inx}}{2i}, & & \cos(nx) = \frac{e^{inx} + e^{inx}}{2}, & & e^{inx} = \cos(nx) + i \sin(nx)
\end{align*}
is a linear change of basis.

\newpage
