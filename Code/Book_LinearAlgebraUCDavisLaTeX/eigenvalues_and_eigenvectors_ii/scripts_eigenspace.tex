
\subsection*{Eigenspaces}

%%%Insert this to get the typewriter font so it looks like a real movie script
{\ttfamily
\fontdimen2\font=0.4em
\fontdimen3\font=0.2em
\fontdimen4\font=0.1em
\fontdimen7\font=0.1em
\hyphenchar\font=`\-


\hypertarget{scripts_eigenvalues_eigenvectors_ii_eigenspaces}{Consider the linear map}
\[
L = \begin{pmatrix}
-4 & 6 & 6 \\
0 & 2 & 0 \\
-3 & 3 &5
\end{pmatrix}.
\]
Direct computation will show that we have
\[
L = Q \begin{pmatrix}
-1 & 0 & 0 \\
0 & 2 & 0 \\
0 & 0 & 2
\end{pmatrix} Q^{-1}
\]
where
\[
Q = \begin{pmatrix}
2 & 1 & 1 \\
0 & 0 & 1 \\
1 & 1 & 0
\end{pmatrix}.
\]
Therefore the vectors
\[
v_1^{(2)} = \begin{pmatrix}1 \\ 0 \\ 1\end{pmatrix} \hspace{20pt} v_2^{(2)} = \begin{pmatrix}1 \\ 1 \\ 0\end{pmatrix}
\]
span the eigenspace $E^{(2)}$ of the eigenvalue 2, and for an explicit example, if we take
\[
v = 2 v_1^{(2)} - v_2^{(2)} = \begin{pmatrix} 1 \\ -1 \\ 2\end{pmatrix}
\]
we have
\[
L v = \begin{pmatrix} 2 \\ -2 \\ 4\end{pmatrix} = 2 v
\]
so $v \in E^{(2)}$. In general, we note the linearly independent vectors $v_i^{(\lambda)}$ with the same eigenvalue $\lambda$ span an eigenspace since for any $v = \sum_i c^i v_i^{(\lambda)}$, we have
\[
Lv = \sum_i c^i Lv_i^{(\lambda)} = \sum_i c^i \lambda v_i^{(\lambda)} = \lambda \sum_i c^i v_i^{(\lambda)} = \lambda v.
\]


} % Closing bracket for font

%\newpage
