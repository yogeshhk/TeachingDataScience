
\chapter{\gramSchmidtTitle}\label{gramschmidt}


Given a vector $v$ and some other vector $u$ not in  $ {\rm span}\, \{v\} $, we can construct a new vector: 
\[
v^\perp:=v-\frac{u\cdot v}{u\cdot u}u.
\]
\begin{center}
\input{vperp.pdftex_t}
\end{center}
This new vector $v^\perp$ is orthogonal to $u$ because 
\[
u\dotprod v^\perp = u\dotprod v - \frac{u\cdot v}{u\cdot u}u\dotprod u = 0.
\]
Hence, $\{u, v^\perp\}$ is an orthogonal basis for $\spa \{u,v\}$.  When $v$ is not parallel to $u$, $v^\perp \neq 0$, and normalizing these vectors we obtain $\left\{\frac{u}{|u|}, \frac{v^\perp}{|v^\perp|} \right\}$, an orthonormal basis for the vector space ${\rm span}\, \{u,v\}$.

Sometimes we write $v = v^\perp + v^\parallel$ where:
\begin{eqnarray*}
v^\perp &=& v-\frac{u\cdot v}{u\cdot u}u \\
v^\parallel &=& \phantom{v-}\frac{u\cdot v}{u\cdot u}u.
\end{eqnarray*}
This is called an \emph{orthogonal decomposition}\index{Orthogonal decomposition} because we have decomposed $v$ into a sum of orthogonal vectors.  This decomposition depends on $u$; if we change the direction of $u$ we change $v^\perp$ and $v^\parallel$.

If $u$, $v$ are linearly independent vectors in $\Re^3$, then the set $\{u, v^\perp, u\times v^\perp \}$ would be an orthogonal basis for $\Re^3$.  This set could then be normalized by dividing each vector by its length to obtain an orthonormal basis.

However, it often occurs that we are interested in vector spaces with dimension greater than $3$, and must resort to craftier means than cross products to obtain an orthogonal basis.
\footnote{Actually, given a set $T$ of $(n-1)$ independent vectors in $n$-space, one can define an analogue of the cross product that will produce a vector orthogonal to the span of $T$, using a method exactly analogous to the usual computation for calculating the cross product of two vectors in $\Re^3$.  This only gets us the \emph{last} orthogonal vector, though; the process in this Section gives a way to get a full orthogonal basis.}
%yeah, and who is going to teach them about the Hodge star in office hours? I say nix this footnote -cherney

Given a third vector $w$, we should first check that $w$ does not lie in the span of $u$ and $v$, \textit{i.e.} check that $u,v$ and $w$ are linearly independent.   We then can define:
\[
w^\perp = w - \frac{u\dotprod w}{u\dotprod u}\,u - \frac{v^\perp\dotprod w}{v^\perp\dotprod v^\perp}\,v^\perp.
\]

We can check that \(u \dotprod w^\perp\) and \(v^\perp \dotprod w^\perp\) are both zero:
\begin{align*}
u \dotprod w^\perp&=u \dotprod \left(w - \frac{u\dotprod w}{u\dotprod u}\,u - \frac{v^\perp\dotprod w}{v^\perp\dotprod v^\perp}\,v^\perp \right)\\&= u\dotprod w - \frac{u \dotprod w}{u \dotprod u}u \dotprod u - \frac{v^\perp \dotprod w}{v^\perp \dotprod v^\perp} u \dotprod v^\perp \\
&=u\dotprod w-u\dotprod w-\frac{v^\perp \dotprod w}{v^\perp \dotprod v^\perp} u \dotprod v^\perp\ =\ 0
\end{align*}
since \(u\) is orthogonal to \(v^\perp\), and
\begin{align*}
v^\perp \dotprod w^\perp&=v^\perp \dotprod \left(w - \frac{u\dotprod w}{u\dotprod u}\,u - \frac{v^\perp\dotprod w}{v^\perp\dotprod v^\perp}\,v^\perp \right)\\ &=v^\perp\dotprod w - \frac{u \dotprod w}{u \dotprod u}v^\perp \dotprod u - \frac{v^\perp \dotprod w}{v^\perp \dotprod v^\perp} v^\perp \dotprod v^\perp \\
&=v^\perp\dotprod w-\frac{u \dotprod w}{u \dotprod u}v^\perp \dotprod u - v^\perp \dotprod w\ =\ 0
\end{align*}
because \(u\) is orthogonal to \(v^\perp\). Since $w^\perp$ is orthogonal to both $u$ and $v^\perp$, we have that $\{u,v^\perp,w^\perp \}$ is an orthogonal basis for $\spa \{u,v,w\}$.

\section{The Gram-Schmidt Procedure}
In fact, given a set $\{v_1, v_2, \ldots \}$ of linearly independent vectors, we can define an orthogonal basis for $\spa \{v_1,v_2, \ldots \}$ consisting of the following vectors:
\begin{eqnarray*}
v_1^\perp&:=&v_1 \\
v_2^\perp &:=& v_2 - \frac{v_1^\perp\cdot v_2}{v_1^\perp\cdot v_1^\perp}\,v_1^\perp \\
v_3^\perp &:=& v_3 - \frac{v_1^\perp\cdot v_3}{v_1^\perp\cdot v_1^\perp}\,v_1^\perp - \frac{v_2^\perp\cdot v_3}{v_2^\perp\cdot v_2^\perp}\,v_2^\perp\\
&\vdots& \\
v_i^\perp%&=&   v_i - \sum_{j<i} \frac{v_j^\perp\cdot v_i}{v_j^\perp\cdot v_j^\perp}\,v_j^\perp \\
 &:=& v_i - \frac{v_1^\perp\cdot v_i}{v_1^\perp\cdot v_1^\perp}\,v_1^\perp -  
 - \frac{v_2^\perp\cdot v_3}{v_2^\perp\cdot v_2^\perp}\,v_2^\perp -\cdots
 - \frac{v_{i-1}^\perp\cdot v_i}{v_{i-1}^\perp\cdot v_{i-1}^\perp}\,v_{i-1}^\perp\\
&\vdots& \\
\end{eqnarray*}
Notice that each $v_i^\perp$ here depends on  $v_j^\perp$ for every $j<i$.  This allows us to inductively/algorithmically build up a linearly independent, orthogonal set of vectors 
$\{v_1^\perp,v_2^\perp, \ldots \}$ 
such that 
$\spa \{v_1^\perp,v_2^\perp, \ldots \}=\spa \{v_1, v_2, \ldots \}$. That is, on orthogonal basis for the latter vector space. This algorithm bears the name \emph{Gram--Schmidt orthogonalization procedure}\index{Gram--Schmidt orthogonalization procedure}\label{GramSchmidt}.

\begin{example}
We'll  obtain an orthogonal basis for $\Re^3$ by appling Gram-Schmidt to the linearly independent set 
$\left\{ v_1=\colvec{1\\1\\0}, v_2=\colvec{1\\1\\1},v_3=\colvec{3\\1\\1} \right\}$.

First, we set $v_1^\perp:=v_1$.  Then:
\begin{eqnarray*}
v_2^\perp&=& \rowvec{1\\1\\1} - \frac{2}{2}\rowvec{1\\1\\0} = \rowvec{0\\0\\1} \\
V_3^\perp&=& \rowvec{3\\1\\1} - \frac{4}{2}\rowvec{1\\1\\0} - \frac{1}{1}\rowvec{0\\0\\1} = \rowvec{1\\-1\\0}. 
\end{eqnarray*}
Then the set
\[
\left\{ \rowvec{1\\1\\0},\rowvec{0\\0\\1},\rowvec{1\\-1\\0}\right\}
\]
is an orthogonal basis for $\Re^3$.  To obtain an orthonormal basis, as always we simply divide each of these vectors by its length, yielding:
\[
\left\{ \rowvec{\frac{1}{\sqrt2}\\\frac{1}{\sqrt2}\\0},\rowvec{0\\0\\1},\rowvec{\frac{1}{\sqrt2}\\\frac{-1}{\sqrt2}\\0}\right\}.
\]
\end{example}

\videoscriptlink{gram_schimdt_and_orthogonal_complements_4by4_example.mp4}{A $4\times4$ Gram Schmidt Example} {scripts_gram_schimdt_and_orthogonal_complements_4by4_example}

\section{$QR$ Decomposition}
In Lecture~\ref{LUdecomp} we learned how to solve linear systems by decomposing a matrix $M$ into 
a product of lower and upper triangular matrices
$$M=LU\, .$$
The Gram--Schmidt procedure suggests another matrix decomposition,
$$M=QR$$ 
where $Q$ is an orthogonal matrix and $R$ is an upper triangular matrix. So-called QR-decompositions\index{QR decomposition}
are useful for solving linear systems, eigenvalue problems and least squares approximations. You can
easily get the idea behind $QR$ decomposition by working through a simple example.

\begin{example}
\hypertarget{methodQR}{Find} the $QR$ decomposition of $$M=\begin{pmatrix}2&-1&1\\1&3&-2\\0&1&-2\end{pmatrix}\, .$$
What we will do is to think of the columns of $M$ as three vectors and use Gram--Schmidt to
build an orthonormal basis from these that will become the columns of the orthogonal matrix $Q$.
We will use the matrix $R$ to record the steps of the Gram--Schmidt procedure in such a way
that the product $QR$ equals $M$. 

To begin with we write
$$
M=\begin{pmatrix}2&-\frac75&1\\[1mm]1&\frac{14}5&-2\\[1mm]0&1&-2\end{pmatrix}
\begin{pmatrix}1&\frac15&0\\[1mm]0&1&0\\[1mm]0&0&1\end{pmatrix}\, .
$$
In the first matrix the first two columns are  orthogonal because we simpy replaced the second column of $M$ by the vector that the Gram--Schmidt
procedure produces from the first two columns of~$M$, namely
$$
\colvec{-\frac75\\[1mm]\frac{14}5\\[1mm]1}=\colvec{-1\\[1mm]3\\[1mm]1}-\frac15
\colvec{ 2 \\[1mm]1\\[1mm]0}\, .
$$
 The matrix on the right is almost the identity
matrix, save the $+\frac15$ in the second entry of the first row, whose effect upon multiplying the
two matrices precisely undoes what we we did to the second column of the first matrix. 

For the third column of $M$ we use Gram--Schmidt to deduce the third orthogonal vector
$$
\colvec{-\frac16\\[1mm]\frac13\\[1mm]-\frac76}=
\colvec{1\\[1mm]-2\\[1mm]-2}
-0
\colvec{ 2 \\[1mm]1\\[1mm]0}
-\frac{-9}{\frac{54}{5}}\colvec{-\frac75\\[1mm]\frac{14}5\\[1mm]1}\, ,
$$
and therefore, using exactly the same procedure write
$$
M=\begin{pmatrix}2&-\frac75&-\frac16\\[1mm]1&\frac{14}5&\frac13\\[1mm]0&1&-\frac76\end{pmatrix}
\begin{pmatrix}1&\frac15&0\\[1mm]0&1&-\frac56\\[1mm]0&0&1\end{pmatrix}\, .
$$
This is not quite the answer because the first matrix is now made of mutually orthogonal column vectors,
but  a {\it bona fide} orthogonal matrix is comprised of {\it orthonormal} vectors. To achieve that we divide
each column of the first matrix by its length and multiply the corresponding row of the second matrix by the same 
amount:
$$
M=\begin{pmatrix}\frac{2\sqrt{5}}{5}&-\frac{7\sqrt{30}}{90}&-\frac{\sqrt{6}}{18}\\[2mm]
\frac{\sqrt{5}}{5}&\frac{7\sqrt{30}}{45}&\frac{\sqrt{6}}{9}\\[2mm]
0&\frac{\sqrt{30}}{18}&-\frac{7\sqrt{6}}{18}\end{pmatrix}
\begin{pmatrix}\sqrt{5}&\frac{\sqrt{5}}{5}&0\\[2mm]
0&\frac{3\sqrt{30}}{5}&-\frac{\sqrt{30}}{2}\\[2mm]
0&0&\frac{\sqrt{6}}{2}\end{pmatrix}=QR\, .
$$
A nice check of this result is to verify that entry $(i,j)$  of the matrix $R$
equals the dot product of the $i$-th column of $Q$ with the $j$-th column of $M$.
(Some people memorize this fact and use it as a recipe for computing $QR$ deompositions.)
{\it A good test of your own understanding is to work out why this is true!}
\end{example}


\videoscriptlink{gram_schimdt_and_orthogonal_complements_qr_example.mp4}{Another $QR$ decomposition example}{scripts_gram_schmidt_and_orthogonal_complements_qr_example}

\section{Orthogonal Complements}

Let $U$ and $V$ be subspaces of a vector space $W$.  We saw as a \hyperref[UcapV]{review exercise} that $U\cap V$ is a subspace of $W$, and that $U\cup V$ was not a subspace.  However, $\spa (U\cup V)$ is certainly a subspace, since the span of \emph{any} subset of a vector space is a subspace.
Notice that all elements of $\spa (U\cup V)$ take the form $u+v$ with $u\in U$ and $v\in V$.  We call the subspace 
\[
U+V:=\spa (U\cup V) = \{u+v | u\in U, v\in V \}
\] 
the \emph{sum}\index{Sum of vectors spaces} of $U$ and $V$.  Here, we are not adding vectors, but vector spaces to produce a new vector space!


\begin{definition}
Given two subspaces $U$ and $V$ of a space $W$ such that $U\cap V=\{0_W\}$, the \emph{direct sum}\index{Direct sum} of $U$ and $V$ is defined as:
\[
U \oplus V = \spa (U\cup V)= \{u+v | u\in U, v\in V \}.
\]
\end{definition}
Notice that when $U\cap V= \{0_W\}$, $U+V=U\oplus V$.


The direct sum has a very nice property.

\begin{theorem}
If $w\in U\oplus V$  then 
%the expression $w=u+v$ is unique.  That is, 
there is only one way to write \(w\) as the sum of a vector in \(U\) and a vector in \(V\).  
\end{theorem}

\begin{proof}
Suppose that $u+v=u'+v'$, with $u,u'\in U$, and $v,v' \in V$.  Then we could express $0=(u-u')+(v-v')$.  Then $(u-u')=-(v-v')$.  Since $U$ and $V$ are subspaces, we have $(u-u')\in U$ and $-(v-v')\in V$.  But since these elements are equal, we also have $(u-u')\in V$.  Since $U\cap V=\{0\}$, then $(u-u')=0$.  Similarly, $(v-v')=0$. Therefore $u=u'$ and  $v=v'$, proving the theorem. 
\end{proof}

\reading{22}{1}
%\begin{center}\href{\webworkurl ReadingHomework22/1/}{Reading homework: problem \ref{gramschmidt}.1}\end{center}

Given a subspace $U$ in $W$, how can we write $W$ as the direct sum of $U$ and \emph{something}? There is not a unique answer to this question as can be seen from this picture of subspaces in $W={\mathbb R}^3$: 
\begin{center}
\includegraphics[scale=.25]{\gramSchmidtPath/direct_sums.jpg}
\end{center}
However, using the inner product, there is a natural candidate $U^\perp$ for this second subspace as shown here:
\begin{center}
\includegraphics[scale=.25]{\gramSchmidtPath/U_perp.jpg}
\end{center}

The general definition is as follows:
\begin{definition}
Given a subspace $U$ of a vector space $W$, define:
\[
U^\perp = \{w\in W | w\dotprod u=0 \text{ for all } u\in U\}.
\]
\end{definition}

The set $U^\perp$ (pronounced ``$U$-perp'') is the set of all vectors in $W$ orthogonal to \emph{every} vector in $U$.  This is also often called the \emph{orthogonal complement}\index{Orthogonal complement} of $U$. Probably by now you may be feeling overwhelmed, it may help to watch this quick overview video:

\videoscriptlink{gram_schmidt_and_orthogonal_complements_theory.mp4}{Overview}{scripts_gram_schmidt_and_orthogonal_complements_theory}



\begin{example}
Consider any plane $P$ through the origin in $\Re^3$.  Then $P$ is a subspace, and $P^\perp$ is the line through the origin orthogonal to $P$.  For example, if $P$ is the $xy$-plane, then
\[
\Re^3=P\oplus P^\perp=\{(x,y,0)| x,y\in \Re \} \oplus \{(0,0,z)| z\in \Re \}.
\]
\end{example}

\begin{theorem}
Let $U$ be a subspace of a finite-dimensional vector space $W$.  Then the set $U^\perp$ is a subspace of $W$, and $W=U\oplus U^\perp$\index{Perp@``Perp''}.
\end{theorem}

\begin{proof}
To see that $U^\perp$ is a subspace, we only need to check closure, which requires a simple check.

We have $U\cap U^\perp=\{0\}$, since if $u\in U$ and $u\in U^\perp$, we have:
\[
u\dotprod u = 0 \Leftrightarrow u=0.
\]

Finally, we show that any vector $w\in W$ is in $U\oplus U^\perp$.  (This is where we use the assumption that $W$ is finite-dimensional.)  Let $e_1, \ldots, e_n$ be an orthonormal basis for $W$.  Set: 
\begin{eqnarray*}
u&=&(w\dotprod e_1)e_1 + \cdots + (w\dotprod e_n)e_n \in U\\
u^\perp&=& w-u
\end{eqnarray*}
It is easy to check that $u^\perp \in U^\perp$ (see the Gram-Schmidt procedure).  Then $w=u+u^\perp$, so $w\in U\oplus U^\perp$, and we are done.
\end{proof}

\reading{22}{2}
%\begin{center}\href{\webworkurl ReadingHomework22/2/}{Reading homework: problem \ref{gramschmidt}.2}\end{center}

\begin{example}
Consider any line \(L\) through the origin in \(\Re^4\). Then \(L\) is a subspace, and \(L^\perp\) is a \(3\)-dimensional subspace orthogonal to \(L\). For example, let \(L\) be the line 
$\spa \{ (1,1,1,1)^T\}$ in \(\Re^4.\) Then \(L^\perp\) is given by
\begin{eqnarray*}
L^\perp&=&\{(x,y,z,w)^T \mid x,y,z,w \in \Re \text{ and } (x,y,z,w)^T \dotprod (1,1,1,1)^T=0\} \\
&=&\{(x,y,z,w)^T \mid x,y,z,w \in \Re \text{ and } x,y,z,w=0\}.
\end{eqnarray*}
It is easy to check that 
$$
\left\{
v_1=\colvec{1\\-1\\0\\0}, v_2=\colvec{1\\0\\-1\\0}, v_3=\colvec{1\\0\\0\\-1} \right \}
$$ 
forms a basis for \(L^\perp\). We use Gram-Schmidt to find an orthogonal basis for \(L^\perp\):

First, we set \(v_1^\perp=v_1\). Then:
\begin{eqnarray*}
v_2^\perp&=&\colvec{1\\0\\-1\\0}-\frac{1}{2}\colvec{1,-1,0,0}
=\colvec{\frac{1}{2}\\ \frac{1}{2} \\-1\\ 0 },\\
v_3^\perp&=&\colvec{ 1\\0\\0\\-1} -\frac{1}{2}\colvec{1\\-1\\0\\0}-\frac{1/2}{3/2}
\colvec{ \frac{1}{2}\\\frac{1}{2}\\-1\\0} =\colvec{ \frac{1}{3}\\\frac{1}{3}\\\frac{1}{3}\\-1}.
\end{eqnarray*}
So the set \[\left\{ (1,-1,0,0)^T, \left(\frac{1}{2},\frac{1}{2},-1,0\right)^T, \left(\frac{1}{3},\frac{1}{3},\frac{1}{3},-1\right)^T \right\} \] is an orthogonal basis for \(L^\perp\).
We find an orthonormal basis for \(L^\perp\) by dividing each basis vector by its length:
\[
\left\{
\left( \frac{1}{\sqrt{2}}, -\frac{1}{\sqrt{2}},0,0 \right)^T,
\left( \frac{1}{\sqrt{6}}, \frac{1}{\sqrt{6}}, -\frac{2}{\sqrt{6}},0 \right)^T,
\left( \frac{\sqrt{3}}{6}, \frac{\sqrt{3}}{6}, \frac{\sqrt{3}}{6}, -\frac{\sqrt{3}}{2} \right)^T
\right\}.
\]
Moreover, we have
\[
\Re^4=L \oplus L^\perp = \{(c,c,c,c)^T \mid c \in \Re\} \oplus \{(x,y,z,w)^T \mid x,y,z,w \in \Re \text{ and } x+y+z+w=0\}.
\]
\end{example}

Notice that for any subspace $U$, the subspace $(U^\perp)^\perp$ is just $U$ again.  As such, $\perp$ is an involution on the set of subspaces of a vector space.

%\section*{References}
%Hefferon, Chapter Three, Section VI.2: Gram-Schmidt Orthogonalization
%\\
%Beezer, Chapter V, Section O, Subsection GSP
%\\
%Wikipedia:
%\begin{itemize}
%\item \href{http://en.wikipedia.org/wiki/Gram_schmidt}{Gram-Schmidt Process}
%\item \href{http://en.wikipedia.org/wiki/QR_decomposition}{QR Decomposition}
%\item \href{http://en.wikipedia.org/wiki/Orthonormal_basis}{Orthonormal Basis}
%\item \href{http://en.wikipedia.org/wiki/Direct_sum}{Direct Sum}
%\end{itemize}
%

\section{Review Problems}



\begin{enumerate}

\item While performing  Gaussian elimination on these augmented matrices write the full system of equations describing the new rows in terms of the old rows above each equivalence symbol as in  \hyperlink{Keeping track of EROs with equations between rows}{Example}~\ref{Rsystem}. 
$$
\begin{amatrix}{2} 
2 & 2 & 10 \\
1 & 2 & 8 \\
\end{amatrix}
,~
\begin{amatrix}{3} 
1 & 1 & 0 & 5 \\
1 & 1 & \!\!-1& 11 \\
-1 & 1 & 1 & -5 \\ 
\end{amatrix}
$$

%%%%%%%%%%%%%%%%%%%

\item Solve the vector equation by applying ERO matrices to each side of the equation to perform elimination. Show each matrix explicitly as in \hyperlink{Undoing}{Example~\ref{slowly}}.

\begin{eqnarray*}
\begin{pmatrix}
3	&6 	&2 \\ %-3
5 	&9 	&4 \\ %1
2	&4	&2 \\ %0
\end{pmatrix} 
\begin{pmatrix}
 x \\ 
y \\
z 
\end{pmatrix} 
=
\begin{pmatrix}
-3 \\ 
1  \\
0  \\
\end{pmatrix} 
\end{eqnarray*}

%%%%%%%%%%%%%%%%%%%

\item Solve this vector equation by finding the inverse of the matrix through $(M|I)\sim (I|M^{-1})$ and then applying $M^{-1}$ to both sides of the equation. 
\begin{eqnarray*}
\begin{pmatrix}
2	&1 	&1 \\ %9
1 	&1 	&1 \\ %6
1	&1	&2 \\ %7
\end{pmatrix} 
\begin{pmatrix}
 x \\ 
y \\
z 
\end{pmatrix} 
=
\begin{pmatrix}
9 \\ 
6  \\
7  \\
\end{pmatrix} 
\end{eqnarray*}


%%%%%%%%%%%%%%%%%%%

\item Follow the method of  \hyperlink{elldeeeww}{Examples~\ref{factorize} and~\ref{factorizes}} to find the $LU$ and $LDU$ factorization of 
\begin{eqnarray*}
\begin{pmatrix}
3	&3 	&6 \\ %0 %2
3 	&5 	&2 \\ %1 %1
6	&2	&5 \\ %0 %1
\end{pmatrix} .
\end{eqnarray*}



%%%%%%%%%%%%%%%%%%%%

\item 
Multiple matrix equations with the same matrix can be solved simultaneously. 
\begin{enumerate}
\item Solve both systems by performing elimination on just one augmented matrix.
\begin{eqnarray*}
\begin{pmatrix}
2	&-1 	&-1 \\ %0 %2
-1 	&1 	&1 \\ %1 %1
1	&-1	&0 \\ %0 %1
\end{pmatrix} 
\begin{pmatrix}
 x \\ 
y \\
z 
\end{pmatrix} 
=
\begin{pmatrix}
0\\ 
1  \\
0  \\
\end{pmatrix} 
,~
\begin{pmatrix}
2	&-1 	&-1 \\ %0 %2
-1 	&1 	&1 \\ %1 %1
1	&-1	&0 \\ %0 %1
\end{pmatrix} 
\begin{pmatrix}
 a \\ 
b \\
c 
\end{pmatrix} 
=
\begin{pmatrix}
2\\ 
1  \\
1  \\
\end{pmatrix} 
\end{eqnarray*}
\item Give an interpretation of the columns of $M^{-1}$ in $(M|I)\sim (I|M^{-1})$ in terms of solutions to certain systems of linear equations.
\end{enumerate}

%%%%%%%%%%%%%%%%%%%%%%%%

\item How can you convince your fellow students to never make this mistake?
\begin{eqnarray*}
\begin{amatrix}{3} 
1 & 0 & 2 & 3 \\ 
0 & 1 & 2& 3 \\
2 & 0 & 1 & 4 \\
\end{amatrix} 
& 
\stackrel{R_1'=R_1+R_2}{
\stackrel{R_2'=R_1-R_2}{ 
\stackrel{\ R_3'= R_1+2R_2}{\sim}}}
&
\begin{amatrix}{3} 
1 & 1 & 4 & 6 \\
1 & \!\!-1 & 0& 0 \\
1 & 2 & 6 & 9 
\end{amatrix}
\end{eqnarray*}

\item Is $LU$ factorization of a matrix unique?  Justify your answer.


\item[$\infty$.] If you randomly create a matrix by picking numbers out of the blue, it will probably be difficult to perform elimination or factorization; fractions and large numbers will probably be involved. To invent simple problems it is better to start with a simple answer:
\begin{enumerate}
\item Start with any augmented matrix in RREF. Perform EROs to make most of the components non-zero. Write the result on a separate piece of paper and give it to your friend. Ask that friend to find RREF of the augmented matrix you gave them. Make sure they get the same augmented matrix you started with.  
\item Create  an upper triangular matrix $U$ and a lower triangular matrix~$L$ with only $1$s on the diagonal. Give the result to a friend to factor into $LU$ form. 
\item Do the same with an $LDU$ factorization. 
\end{enumerate}
\end{enumerate}

\phantomnewpage



\newpage

