\chapter{\luDecompTitle}
\label{LUdecomp}

Certain matrices are easier to work with than others.  In this section, we will see how to write any square\footnote{The case where $M$ is not square is dealt with at the end of the lecture.} matrix $M$ as the product of two simpler matrices.  We will write $$M=LU\, ,$$ where:
\begin{itemize}
\item $L$ is \emph{lower triangular}\index{Lower triangular matrix}.  This means that all entries above the main diagonal are zero.  In notation,
$L=(l^i_j)$ with $l^i_j=0$ for all $j>i$.
\[L=\begin{pmatrix}
l^1_1 & 0 & 0 & \cdots \\
l^2_1 & l^2_2 & 0 & \cdots \\
l^3_1 & l^3_2 & l^3_3 & \cdots \\
\vdots & \vdots & \vdots & \ddots \\
\end{pmatrix}
\]

\item $U$ is \emph{upper triangular}\index{Upper triangular matrix}.  This means that all entries below the main diagonal are zero.  In notation,
$U=(u^i_j)$ with $u^i_j=0$ for all $j<i$.
\[U=\begin{pmatrix}
u^1_1 & u^1_2 & u^1_3 & \cdots \\
0 & u^2_2 & u^2_3 & \cdots \\
0 & 0 & u^3_3 & \cdots \\
\vdots & \vdots & \vdots & \ddots \\
\end{pmatrix}
\]
\end{itemize}
$M=LU$ is called an \emph{$LU$ decomposition}\index{LU@$LU$ decomposition} of $M$.

This is a useful trick for  computational reasons; it is much easier to compute the inverse of an upper or lower triangular matrix than general matrices.  Since inverses are useful for solving linear systems, this makes solving any linear system associated to the matrix much faster as well.  The determinant---a very important quantity associated with any square matrix---is very easy to compute for triangular matrices.

\begin{example}
Linear systems associated to upper triangular matrices are very easy to solve by back substitution.
\[
\begin{amatrix}{2}
a & b & 1 \\
0 & c & e \\
\end{amatrix} \ \Rightarrow \ y=\frac{e}{c}\, , \quad x=\frac{1}{a}\left(1-\frac{be}{c}\right)
\]

\[
\begin{amatrix}{3}
1 & 0 & 0 & d \\
a & 1 & 0 & e \\
b & c & 1 & f \\
\end{amatrix} \Rightarrow x=d\, , \qquad y=e-ad\, , \qquad z=f-bd-c(e-ad)
\]
For lower triangular matrices, \emph{back} substitution\index{Back substitution} gives a quick solution; for upper triangular matrices, \emph{forward} substitution\index{Forward substitution} gives the solution.
\end{example}





\section{Using $LU$ Decomposition to Solve Linear Systems}

Suppose we have $M=LU$ and want to solve the system
\[
MX=LUX=V.
\]

\begin{itemize}
\item{Step 1:} Set $W=\colvec{u\\v\\w}=UX$.  

\item{Step 2:} Solve the system $LW=V$.  This should be simple by forward substitution since $L$ is lower triangular.  Suppose the solution to $LW=V$ is $W_0$.  

\item{Step 3:} Now solve the system $UX=W_0$.  This should be easy by backward substitution, since $U$ is upper triangular.  The solution to this system is the solution to the original system.
\end{itemize}
We can think of this as using the matrix $L$ to perform row operations on the matrix $U$ in order to solve the system; this idea also appears in the  study of determinants.

%\href{\webworkurl ReadingHomework11/1/}{Reading homework: problem 11.1}
\reading{11}{1}

\begin{example}
Consider the linear system:
\[
      \begin{linsys}{4}
            6x & +&18y & +&3z         &=& 3  \\[1mm]
            2x & +&12y & +&z	    &=& 19 \\[1mm]
            4x & +&15y & +&3z         &=& 0  
      \end{linsys}
\]

An $LU$ decomposition for the associated matrix $M$ is:
\[
\begin{pmatrix}
6 & 18 & 3 \\
2 & 12 & 1 \\
4 & 15 & 3 
\end{pmatrix} =
\begin{pmatrix}
3 & 0 & 0 \\
1 & 6 & 0 \\
2 & 3 & 1 
\end{pmatrix}
\begin{pmatrix}
2 & 6 & 1 \\
0 & 1 & 0 \\
0 & 0 & 1 
\end{pmatrix}.
\]

\begin{itemize}
\item{Step 1:} \hypertarget{LUproc}{Set} $W=\colvec{u\\v\\w}=UX$.  

\item{Step 2:} Solve the system $LW=V$:

\[
\begin{pmatrix}
3 & 0 & 0 \\
1 & 6 & 0 \\
2 & 3 & 1 
\end{pmatrix}
\colvec{u\\v\\w} =
\colvec{3\\19\\0}
\]

By substitution, we get $u=1$, $v=3$, and $w=-11$.  Then 
\[W_0=\colvec{1\\3\\-11}\]

\item{Step 3:} Solve the system $UX=W_0$.  
\[
\begin{pmatrix}
2 & 6 & 1 \\
0 & 1 & 0 \\
0 & 0 & 1 
\end{pmatrix}
\colvec{x\\y\\z} =
\colvec{1\\3\\-11}
\]
Back substitution gives $z=-11, y=3$, and $x=-3$.  

Then $X=\colvec{-3\\3\\-11}$, and we're done.
\end{itemize}
\end{example}

\videoscriptlink{lu_decomposition_using_lu_decomp.mp4}{Using a $LU$ decomposition}{scripts_lu_decomposition_using_lu_example}

%\begin{figure}
\begin{center}
\includegraphics[scale=.3]{\luDecompPath/LU_solution.jpg}
\end{center}
%\end{figure}

\section{Finding an $LU$ Decomposition.}
\label{finding_LU_decomp}
 
For any given matrix, there are actually many different $LU$ decompositions.  However, there is a unique $LU$ decomposition in which the $L$ matrix has ones on the diagonal. In that case $L$ is called a \emph{lower unit triangular matrix}\index{Lower unit triangular matrix}.

To find the $LU$ decomposition, we'll create two sequences of matrices $L_0, L_1, \ldots$ and $U_0, U_1, \ldots$ such that at each step, $L_iU_i=M$.  Each of the $L_i$ will be lower triangular, but only the last $U_i$ will be upper triangular.

Start by setting $L_0=I$ and $U_0=M$, because $L_0U_0=M$. A main concept of this calculation is captured by the following example:

\begin{example}
Consider $$E=\begin{pmatrix}1&0\\\lambda&1\end{pmatrix}\, ,\qquad M=\begin{pmatrix}a&b&c&\cdots\\d&e&f&\cdots\end{pmatrix}\, .$$
Lets compute $EM$
$$
EM=\begin{pmatrix}a&b&c&\cdots\\d+\lambda a&e+\lambda b&f+\lambda c&\cdots\end{pmatrix}\, ,.
$$
Something neat happened here: multiplying $M$ by $E$ performed the row operation $R_2\to R_2+\lambda R-1$ on $M$.
Another interesting fact:
$$
E^{-1}:=\begin{pmatrix}1&0\\-\lambda&1\end{pmatrix}
$$ 
obeys (check this yourself...)
$$
E^{-1} E = 1\, .
$$
Hence $M=E^{-1} E M$ or, writing this out
$$
\begin{pmatrix}a&b&c&\cdots\\d&e&f&\cdots\end{pmatrix}=\begin{pmatrix}1&0\\-\lambda&1\end{pmatrix} \begin{pmatrix}a&b&c&\cdots\\d+\lambda a&e+\lambda b&f+\lambda c&\cdots\end{pmatrix}\, .
$$
Here the matrix on the left is lower triangular, while the matrix on the right has had a row operation performed on it.
\end{example}




\vspace{2mm}
We would like to  use the first row of $U_0$ to zero out the first entry of every row below it.  For our running example, $$U_0=M=\begin{pmatrix}
6 & 18 & 3 \\
2 & 12 & 1 \\
4 & 15 & 3 
\end{pmatrix}\, ,$$ so we would like to perform the row operations $R_2\to R_2 -\frac 13 R_1$ and $R_3\to R_3-\frac 23R_1$.
%so the second row minus $\frac{1}{3}$ of the first row will zero out the first entry in the second row.  Likewise, the third row minus $\frac{2}{3}$ of the first row will zero out the first entry in the third row.
If we perform these row operations on $U_0$ to produce 
$$U_1=\begin{pmatrix}
6 & 18 & 3 \\
0 & 6 & 0 \\
0 & 3 & 1 
\end{pmatrix}\, ,$$
we need to multiply this on the left by a lower triangular matrix $L_1$ so that the product $L_1U_1=M$ still.
The above example shows how to do this:
Set $L_1$ to be the lower triangular matrix whose first column is filled with the minus constants used to zero out the first column of $M$.  Then $$L_1 = \begin{pmatrix}
1 & 0 & 0 \\[1mm]
\frac{1}{3} & 1 & 0 \\[1mm]
\frac{2}{3} & 0 & 1 
\end{pmatrix}\, .$$  
%Set $U_1$ to be the matrix obtained by zeroing out the first column of $M$.  Then $U_1=\begin{pmatrix}
%6 & 18 & 3 \\
%0 & 6 & 0 \\
%0 & 3 & 1 
%\end{pmatrix}$.
By construction $L_1 U_1=M$, but you should compute this yourself as a double check.

Now repeat the process by zeroing the second column of $U_1$ below the diagonal using the second row of $U_1$ using the row operation
$R_3\to R_3-\frac 12 R_2$ to produce
$$U_2=\begin{pmatrix}6&18&3\\0&6&0\\0&0&1\end{pmatrix}\, .$$
The matrix that undoes this row operation is obtained in the same way we found $L_1$ above and is:
$$
\begin{pmatrix}
1&0&0\\
0&1&0\\
0&\frac 12& 0
\end{pmatrix}\, .
$$
Thus our answer for $L_2$ is the product of this matrix with $L_1$, namely
$$
L_2=
\begin{pmatrix}
1 & 0 & 0 \\[1mm]
\frac{1}{3} & 1 & 0 \\[1mm]
\frac{2}{3} & 0 & 1 
\end{pmatrix}\begin{pmatrix}
1&0&0\\
0&1&0\\
0&\frac 12& 0
\end{pmatrix}
=\begin{pmatrix}
1 & 0 & 0 \\[1mm]
\frac{1}{3} & 1 & 0 \\[1mm]
\frac{2}{3} & \frac{1}{2} & 1 
\end{pmatrix}\, .
$$
Notice that it is lower triangular because 

\begin{center}
\textcolor{brown}{THE PRODUCT OF LOWER TRIANGULAR MATRICES IS ALWAYS LOWER TRIANGULAR!}
\end{center}

\noindent
Moreover it is obtained by recording minus the constants used for all our row operations in the appropriate columns (this always works this way).
Moreover, $U_2$ is upper triangular and $M=L_2U_2$, we are done!
Putting this all together we have
$$M=\begin{pmatrix}
6 & 18 & 3 \\
2 & 12 & 1 \\
4 & 15 & 3 
\end{pmatrix}= \begin{pmatrix}
1 & 0 & 0 \\[1mm]
\frac{1}{3} & 1 & 0 \\[1mm]
\frac{2}{3} & \frac{1}{2} & 1 
\end{pmatrix}\begin{pmatrix}
6 & 18 & 3 \\
0 & 6 & 0 \\
0 & 0 & 1 
\end{pmatrix}\, .$$  
%Since $U_2$ is upper-triangular, we're done.  Inserting the new number into $L_1$ to get $L_2$ really is safe: the numbers in the first column don't affect the second column of $U_1$, since the first column of $U_1$ is already zeroed out.

If the matrix you're working with has more than three rows, just continue this process by zeroing out the next column below the diagonal, and repeat until there's nothing left to do.

\videoscriptlink{lu_decomposition_example.mp4}{Another $LU$ decomposition example}{scripts_lu_decomposition_example}

The fractions in the $L$ matrix are admittedly ugly.  For two matrices $LU$, we can multiply one entire column of $L$ by a constant $\lambda$ and divide the corresponding row of $U$ by the same constant without changing the product of the two matrices.  Then:

\begin{eqnarray*}
LU &=& \begin{pmatrix}
1 & 0 & 0 \\[1mm]
\frac{1}{3} & 1 & 0 \\[1mm]
\frac{2}{3} & \frac{1}{2} & 1 
\end{pmatrix}
I
\begin{pmatrix}
6 & 18 & 3 \\
0 & 6 & 0 \\
0 & 0 & 1 
\end{pmatrix} \\
&=&
\begin{pmatrix}
1 & 0 & 0 \\[1mm]
\frac{1}{3} & 1 & 0 \\[1mm]
\frac{2}{3} & \frac{1}{2} & 1 
\end{pmatrix}
\begin{pmatrix}
3 & 0 & 0 \\
0 & 6 & 0 \\
0 & 0 & 1 
\end{pmatrix}
\begin{pmatrix}
\frac{1}{3} & 0 & 0 \\[1mm]
0 & \frac{1}{6} & 0 \\[1mm]
0 & 0 & 1 
\end{pmatrix}
\begin{pmatrix}
6 & 18 & 3 \\
0 & 6 & 0 \\
0 & 0 & 1 
\end{pmatrix} \\
&=&
\begin{pmatrix}
3 & 0 & 0 \\
1 & 6 & 0 \\
2 & 3 & 1 
\end{pmatrix}\begin{pmatrix}
2 & 6 & 1 \\
0 & 1 & 0 \\
0 & 0 & 1 
\end{pmatrix}.
\end{eqnarray*}
The resulting matrix looks nicer, but isn't in standard (lower unit triangular matrix) form.

\reading{11}{2}
%\href{\webworkurl ReadingHomework11/2/}{Reading homework: problem 11.2}

For matrices that are not square, $LU$ decomposition still makes sense.  Given an $m\times n$ matrix $M$, for example we could write $M=LU$ with $L$ a square lower unit triangular matrix, and $U$ a rectangular matrix.  Then $L$ will be an $m\times m$ matrix, and $U$ will be an $m\times n$ matrix (of the same shape as $M$).  From here, the process is exactly the same as for a square matrix.  We create a sequence of matrices $L_i$ and $U_i$ that is eventually the $LU$ decomposition.  Again, we start with $L_0=I$ and $U_0=M$.

\begin{example}
Let's find the $LU$ decomposition of $M=U_0=\begin{pmatrix}
-2 & 1 & 3 \\
-4 & 4 & 1 
\end{pmatrix}$.  Since $M$ is a $2\times 3$ matrix, our decomposition will consist of a $2\times 2$ matrix and a $2\times 3$ matrix.  Then we start with $L_0=I_2=\begin{pmatrix}
1 & 0 \\
0 & 1
\end{pmatrix}$.

The next step is to zero-out the first column of $M$ below the diagonal.  There is only one row to cancel, then, and it can be removed by subtracting $2$ times the first row of $M$ to the second row of $M$.  Then:

\[
L_1=\begin{pmatrix}
1 & 0 \\
2 & 1
\end{pmatrix}, \qquad 
U_1 = \begin{pmatrix}
-2 & 1 & 3 \\
0 & 2 & -5 
\end{pmatrix}
\]
Since $U_1$ is upper triangular, we're done.  With a larger matrix, we would just continue the process.
\end{example}





\section{Block $LDU$ Decomposition}

Let $M$ be a square block matrix with square blocks $X,Y,Z,W$ such that $X^{-1}$ exists.  Then $M$ can be decomposed as a block $LDU$ decomposition, where $D$ is block diagonal, as follows:
\[
M=\begin{pmatrix}
X & Y \\
Z & W
\end{pmatrix}
\]

Then: \[M=\begin{pmatrix}
I &  0 \\
ZX^{-1} & I
\end{pmatrix}\begin{pmatrix}
X & 0 \\
0 & W-ZX^{-1}Y
\end{pmatrix}\begin{pmatrix}
I & X^{-1}Y \\
0 & I
\end{pmatrix}.\]
This can be checked explicitly simply by block-multiplying these three matrices.

\videoscriptlink{lu_decomposition_blocks.mp4}{Block $LDU$ Explanation}{scripts_lu_decomposition_blocks}

\begin{example}
For a $2\times 2$ matrix, we can regard each entry as a block.
\[
\begin{pmatrix}
1 & 2 \\
3 & 4
\end{pmatrix}=
\begin{pmatrix}
1 & 0 \\
3 & 1
\end{pmatrix}
\begin{pmatrix}
1 & 0 \\
0 & -2
\end{pmatrix}
\begin{pmatrix}
1 & 2 \\
0 & 1
\end{pmatrix}
\]
By multiplying the diagonal matrix by the upper triangular matrix, we get the standard $LU$ decomposition of the matrix.
\end{example}


%\section*{References}
%Wikipedia:
%\begin{itemize}
%\item \href{http://en.wikipedia.org/wiki/LU_decomposition}{$LU$ Decomposition}
%\item \href{http://en.wikipedia.org/wiki/Block_LU_decomposition}{Block $LU$ Decomposition}
%\end{itemize}

\section{Review Problems}



\begin{enumerate}

\item While performing  Gaussian elimination on these augmented matrices write the full system of equations describing the new rows in terms of the old rows above each equivalence symbol as in  \hyperlink{Keeping track of EROs with equations between rows}{Example}~\ref{Rsystem}. 
$$
\begin{amatrix}{2} 
2 & 2 & 10 \\
1 & 2 & 8 \\
\end{amatrix}
,~
\begin{amatrix}{3} 
1 & 1 & 0 & 5 \\
1 & 1 & \!\!-1& 11 \\
-1 & 1 & 1 & -5 \\ 
\end{amatrix}
$$

%%%%%%%%%%%%%%%%%%%

\item Solve the vector equation by applying ERO matrices to each side of the equation to perform elimination. Show each matrix explicitly as in \hyperlink{Undoing}{Example~\ref{slowly}}.

\begin{eqnarray*}
\begin{pmatrix}
3	&6 	&2 \\ %-3
5 	&9 	&4 \\ %1
2	&4	&2 \\ %0
\end{pmatrix} 
\begin{pmatrix}
 x \\ 
y \\
z 
\end{pmatrix} 
=
\begin{pmatrix}
-3 \\ 
1  \\
0  \\
\end{pmatrix} 
\end{eqnarray*}

%%%%%%%%%%%%%%%%%%%

\item Solve this vector equation by finding the inverse of the matrix through $(M|I)\sim (I|M^{-1})$ and then applying $M^{-1}$ to both sides of the equation. 
\begin{eqnarray*}
\begin{pmatrix}
2	&1 	&1 \\ %9
1 	&1 	&1 \\ %6
1	&1	&2 \\ %7
\end{pmatrix} 
\begin{pmatrix}
 x \\ 
y \\
z 
\end{pmatrix} 
=
\begin{pmatrix}
9 \\ 
6  \\
7  \\
\end{pmatrix} 
\end{eqnarray*}


%%%%%%%%%%%%%%%%%%%

\item Follow the method of  \hyperlink{elldeeeww}{Examples~\ref{factorize} and~\ref{factorizes}} to find the $LU$ and $LDU$ factorization of 
\begin{eqnarray*}
\begin{pmatrix}
3	&3 	&6 \\ %0 %2
3 	&5 	&2 \\ %1 %1
6	&2	&5 \\ %0 %1
\end{pmatrix} .
\end{eqnarray*}



%%%%%%%%%%%%%%%%%%%%

\item 
Multiple matrix equations with the same matrix can be solved simultaneously. 
\begin{enumerate}
\item Solve both systems by performing elimination on just one augmented matrix.
\begin{eqnarray*}
\begin{pmatrix}
2	&-1 	&-1 \\ %0 %2
-1 	&1 	&1 \\ %1 %1
1	&-1	&0 \\ %0 %1
\end{pmatrix} 
\begin{pmatrix}
 x \\ 
y \\
z 
\end{pmatrix} 
=
\begin{pmatrix}
0\\ 
1  \\
0  \\
\end{pmatrix} 
,~
\begin{pmatrix}
2	&-1 	&-1 \\ %0 %2
-1 	&1 	&1 \\ %1 %1
1	&-1	&0 \\ %0 %1
\end{pmatrix} 
\begin{pmatrix}
 a \\ 
b \\
c 
\end{pmatrix} 
=
\begin{pmatrix}
2\\ 
1  \\
1  \\
\end{pmatrix} 
\end{eqnarray*}
\item Give an interpretation of the columns of $M^{-1}$ in $(M|I)\sim (I|M^{-1})$ in terms of solutions to certain systems of linear equations.
\end{enumerate}

%%%%%%%%%%%%%%%%%%%%%%%%

\item How can you convince your fellow students to never make this mistake?
\begin{eqnarray*}
\begin{amatrix}{3} 
1 & 0 & 2 & 3 \\ 
0 & 1 & 2& 3 \\
2 & 0 & 1 & 4 \\
\end{amatrix} 
& 
\stackrel{R_1'=R_1+R_2}{
\stackrel{R_2'=R_1-R_2}{ 
\stackrel{\ R_3'= R_1+2R_2}{\sim}}}
&
\begin{amatrix}{3} 
1 & 1 & 4 & 6 \\
1 & \!\!-1 & 0& 0 \\
1 & 2 & 6 & 9 
\end{amatrix}
\end{eqnarray*}

\item Is $LU$ factorization of a matrix unique?  Justify your answer.


\item[$\infty$.] If you randomly create a matrix by picking numbers out of the blue, it will probably be difficult to perform elimination or factorization; fractions and large numbers will probably be involved. To invent simple problems it is better to start with a simple answer:
\begin{enumerate}
\item Start with any augmented matrix in RREF. Perform EROs to make most of the components non-zero. Write the result on a separate piece of paper and give it to your friend. Ask that friend to find RREF of the augmented matrix you gave them. Make sure they get the same augmented matrix you started with.  
\item Create  an upper triangular matrix $U$ and a lower triangular matrix~$L$ with only $1$s on the diagonal. Give the result to a friend to factor into $LU$ form. 
\item Do the same with an $LDU$ factorization. 
\end{enumerate}
\end{enumerate}

\phantomnewpage



\newpage
