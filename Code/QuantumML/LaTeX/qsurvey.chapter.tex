



\documentstyle[11pt]{article}



%no page numbering in entire document except first page
\pagestyle{empty}
%no page numbering on the page if placed just after \maketitle
%\thispagestyle{empty}


% \bibliographystyle{unsrt}


\setlength{\topmargin}{-0.5in}
%\setlength{\headheight}{0.0in}
%\setlength{\headsep}{0.0in}

%\setlength{\textheight}{9.1in}
%\setlength{\textwidth}{6.1in}
%\setlength{\oddsidemargin}{0.25in}
%\setlength{\evensidemargin}{0.25in}





%***DOUBLE SPACING ADDED TO CHAPTER
\renewcommand{\baselinestretch}{2}

\begin{document}


 
\renewcommand{\thefootnote}{\fnsymbol{footnote}}
\addtocounter{footnote}{2}

\newcommand{\alg}[1]{\newline\parbox{#1cm}{\ }}


\newtheorem{lemma}{Lemma}[section]
\newtheorem{theorem}{Theorem}[section]
\newtheorem{corollary}{Corollary}[section]
\newtheorem{definition}{Definition}[section]
\newtheorem{proposition}{Proposition}[section]


\def\Proof{\penalty25\medskip\noindent{\bf Proof:}\enskip}
\def\blackslug{\hbox{\hskip 1pt \vrule width 4pt height 8pt depth 
1.5pt
  \hskip 1pt}}
\def\QED{\quad\blackslug\lower 8.5pt\null}


\def\noi{\noindent}
\def\mmexp{{\alpha}}
\def\ss{\scriptstyle}
\def\Tabs#1{\hbox{$\vert {#1} \vert$}}

\def\Rsp#1{\hbox{\bf{R}$^{#1}$}}
\def\Zsp#1{\hbox{\bf{Z}$_r^{#1}$}}
\def\conv{\hbox{\rm conv\/}}
\def\spl{\hbox{\rm spl\/}}
%\def\mod{\hbox{\rm\thinspace\hspace{-5pt} mod\thinspace}}
\def\mod{\hbox{\rm\thinspace mod\thinspace}}
\def\rank{\hbox{\rm rank\/}}
\def\NC{\hbox{${\cal NC}$}}
\def\RNC{\hbox{${\cal RNC}$}~}
\def\pf{\hbox{\rm pf\/}}

\def\Oh{\hbox{$\tilde{O}$}}
\def\ystar{\hbox{$y_{\star}$}}
\def\Tangb#1#2{\hbox{$\langle #1, #2 \rangle$}}



\title{{\LARGE\sf 
Chapter 3: Quantum Computing
%Quantum Information Processing: Algorithms, Technologies and Challenges
}}

\author{
John H. Reif \\
Department of  Computer Science\\ Duke University
\thanks{Surface address: Department of  Computer Science, 
Duke University, Durham, NC 27708-0129.
 E-mail: reif@cs.duke.edu.  
This work has been supported by grants from NSF CCF-0432038 and CCF-0523555. 
%Grants NSF/DARPA CCR-9725021, CCR-96-33567, NSF IRI-9619647, ARO contract DAAH-04-96-1-0448, and ONR contract  N00014-99-1-0406.
%This material was presented in an invited talk at the Data Compression Conference (DCC'98), Snowbird, UT, March, 1999.
A postscript 
%document containing a bibliography of over  $385$ reference
%papers in quantum computation 
preprint of this Chapter is available online 
%on the web
 at URL:
 http://www.cs.duke.edu/$\sim$reif/paper/qsurvey/qsurvey.chapter.pdf.
%The views and conclusions contained in this document 
%are those of the author and should
%not be interpreted as representing
% the official policies, either expressed or implied
%of the Advanced Research Projects Agency, NSF, ONR 
%or the U.S. government.
}}

\date{}
\maketitle

%no page numbering on this first page if placed just after \maketitle
\thispagestyle{empty}

%%\newpage

%   {sec-intro}
%   {subsec-assump}
%   {subsec-motiv}
%   {subsec-results}
%   {subsec-prev results}
%   {subsec-Organization}
%   
%   
%   {sec-ack}


\setcounter{section}{3}
\setcounter{subsection}{-1}
\subsection{Summary}
{\em Quantum Computation (QC)} is a type of computation where
unitary and measurement operations are executed on linear superpositions of
basis states.
This paper provides a brief introduction to QC. We begin with a discussion of basic models for QC such as quantum TMs, quantum gates and circuits and related complexity results.
We then discuss a number of topics in quantum information theory,
including bounds for quantum communication and  I/O complexity,  methods for quantum data compression. and quantum error correction
(that is, techniques for decreasing decoherence errors in QC),
Furthermore, we enumerate a number of methodologies and technologies for doing
 QC.  Finally, we discuss resource bounds for QC including bonds for processing time, energy and volume, particularly emphasizing challenges in determining volume bounds for observation apperatus.


\subsection{Introduction} \label{sec-introduction}

\subsubsection{Reversible Computations}

{\em  Reversible Computations} are computations where each state
transformation
 is a reversible function, so that any computation can be reversed without loss of information. 
Landauer [1] showed that irreversible computations must generate heat in the computing process,
 and that reversible computations have the property that if executed slowly enough,
 they (in the limit) can consume no energy in an adiabatic computation.
 Bennett [2] (also see Bennett, Landauer [3], Landauer [4],  Toffoli [5] ) 
showed that any computing machine (e.g., an abstract machine such as a Turing Machine) 
can be transformed to do reversible computations.
 Bennett's reversibility construction required extra space to store information 
to insure reversibility; Li, Vitanyi [6] give trade-offs between
 time and space in the resulting reversible machine.
An innovative technique due to Bennett [2,7] can be used to make reversible functions bijective,
as required for quantum computations.
 Given a  bijective function $f$, suppose we can reversibly compute in time $T(x)$
a bijective function $f$ and it's inverse $f^{-1}$ using auxiliary registers for storage of
the input. He proves in time $O(T(n))$ we can also reversibly compute the bijective mapping:
 $(x,0,0) \rightarrow (f(x),0,0)$
without use of auxiliary registers for storage of the input.

\subsubsection{An Introduction to Quantum Computation}

Computations and methods not making  use of quantum mechanics
 will be termed {\it classical}.
In contrast, {\it Quantum Computation (QC)}
applies quantum mechanics to do  computation.
 A single molecule (or collection of particles and/or atoms) 
may have a number $n$ of degrees of freedom known as {\em qubits}.
Associated with each fixed setting $X$
  of the $n$ qubits to  Boolean values is a {\em basis state}  denoted $|a>$. 
%Let $S$ be the set of all basis states.

Quantum mechanics allows for a linear superposition (also termed an {\it entangled quantum state})
of these basis states to
exist simultaneously. 
Each basis state $|a>$ of the superposition 
is assigned a given complex amplitude $\alpha$;
this is denoted $\alpha |a>$. 
{\it Unitary transformations} are reversible operations on the
superpositions which can be represented by unitary matrices
$A$ 
(e.g., permutation matrices, rotation matrices, and the matrices of  Fourier transforms) 
where $AA^T=I.$
%,(e.g., 
%a class of
%unitary matrices, known as {\it permutation matrices,}
% have exactly one $1$ on every row and column, with all other entries $0$).
The
sum of the squares of the magnitudes of the amplitudes of all basis states is $1$.
%The   quantum  superposition of basis states is denoted $\sum_{\alpha(a)|a \in S>$.
This sum remains invariant
due to the application of a 
unitary transformations.
The Hilbert space $H_n$ is the set of all possible such linear superpositions.




QC is a method of computation where
various operations can be executed on these superpositions:
\begin{itemize}
\item {\em unitary operations}, and
\item {\em observation operations}, which allow for the (strong) measurement of each qubit,
providing a mapping from the current superposition
to a superposition where the measured
 qubit is assigned a  Boolean value with probability
 given by the square of the amplitude of the qubit in its original superposition.
\end{itemize}

{\it Elementary unitary operations} that suffice for any quantum computation over qubits
 (see [8] and [9]) include a
conditional form of  the conditional XOR operation $\oplus$,  the Boolean operation NOT, and a constant Boolean
operation yielding $0$. The time bound for a quantum computations is defined to be the
number  of such elementary unitary operations.

\subsubsection{Surveys of QC}

 The following are reviews and surveys have been made of QC:
Bennett [10], Barenco [11],
 Benio [12], Brassard [13,14], Haroche, Raimond [15], 
Brassard   [16], Preskill [17], Scarani [18], Steane [19], 
Vedral, Plenio [20]. Also, Taubes [21] and Gershenfeld, Chuang [22] give 
popular press descriptions of QC. The following are texts quantum computing: 
\begin{itemize}
\item  Overviews: [23, 24, 25, 26].
\item Quantum information processing: [27, 28, 29, 30, 31, 32, 33, 34]. \item Quantum cryptography: [35] ,36, 37].
\item Quantum coding theory:[38, 39].
\item Quantum algorithms: [40].
\item  Experimental implementation of quantum computation: [41, 42, 43, 44, 45].
\end{itemize}

\subsubsection{Initial Work in QC}

Feynman [46,47] and Benioff [48]  were the first to suggest 
the use of quantum mechanical principles for doing computation.
 Deutsch and 
 Jozsa [49] give the first example of a quantum algorithm that gave a rapid solution
 of an example problem, where the problem (for a given a black box function) 
 is not quickly solvable by any deterministic conventional computing machine. 
But their problem could be  quickly solved using randomization.
Bernstein and
 Vazirani [50] then provided the first example of  a fast quantum algorithm
for a problem that could not be quickly solved by 
conventional computing machines even using randomization.
(Also 
see Costantini, Smeraldi [51] for a generalization of Deutsch's example
 and see Collins et al [52] for a simplified  Deutsch-Jozsa algorithm,
and see Jozsa [53,54,55] for further work in quantum computation and complexity.)
\subsubsection{Organization of this
Paper}

In this Section \ref{sec-introduction} we have introduced  QC.
In Section \ref{sec-models} we introduce formal quantum computing models and in  Section \ref{sec-complexity}
we discuss
quantum complexity classes.
Next we overview key topics concerning quantum information processing: in Section \ref{sec-communication}  we discuss bounds for
quantum communication, then next in Section \ref{sec-compress} we discuss methods for quantum errorless
compression,  in Section \ref{sec-codes} we discuss
methods for quantum error coding, and in Section
\ref{sec-cryptography} we describe methods for quantum cryptography. In Section
\ref{sec-applications} we discuss further algorithmic applications of QC.
In Section
\ref{sec-technologies} we enumerate various technologies  for
doing QC.
In Section \ref{sec-resources} we 
review of the resource bounds of quantum computing as compared
with the resources required by classical methods for computation.
In Section \ref{sec-conclusion} we conclude the paper.
In Appendix
\ref{sec-observation} we discuss the challenge of providing volume bounds for observation apparatus when doing QC.


\subsection{Quantum Computing Models}  \label{sec-models}
\noindent
{\bf $\bullet$ Quantum TMs and other Automata.}
 Deutsch [56] gave the first formal description of a quantum computer, 
known as a {\it quantum TM}. The tape contents of the TM are qubits. 
 {\it Quantum configurations} of the QTM are superpositions of (classical) TM configurations. 
A transition of the QTM is a unitary mapping on quantum configurations of the QTM. 
Thus, a computation of the QTM is a unitary mapping from the initial quantum
configuration
 to the final quantum configuration.
Various papers generalize machines and  automata to the quantum case.
 Moore, Crutchfield [57] propose quantum finite-state and push-down automata, 
and regular and context-free grammars, and they generalize several formal language and automata theorems, 
e.g. pumping lemmas, closure properties, rational and algebraic generating functions, 
and Greibach normal form. Kondacs and Watrous [58] partially characterize the 
power of quantum finite state automata. Dunlavey [59] gives a space-efficient
 simulation of a deterministic finite state machine (FSM) on a quantum computer
(using Grover's search algorithm discussed below). Watrous [60] investigates quantum cellular automata 
 and
% D�rr
D\"{u}rr et al [61,62] give decision procedures for unitary linear (one dimensional) 
quantum cellular automata.

\noindent
{\bf $\bullet$ Quantum Gates.}
A set of Boolean gates are {\it universal} if any Boolean operation on arbitrarily many bits can be expressed as compositions of these gates.
Toffoli [5] defined an extended XOR 3-bit gate
(which is an XOR gate condition on one of the inputs and 
is known as the {\it Toffoli gate}) and showed that this gate, in combination with certain 1-bit gates, is universal.
A set of quantum qubit gates are {\it universal} for Boolean computations for QC if any unitary operation on arbitrarily many qubits can be expressed as compositions of these gates.
Deutsch defined the extended quantum XOR 3-qubit gate
(known as the Deutsch-Toffoli gate) and proved this gate, in combination with certain one qubit gates, is universal.
Barenco [63], Sleator et al [64], Barenco et al [65], and DiVincenzo [66] proved the 2-qubit XOR gates with certain 1-qubit gates can implement the Deutsch-Toffoli gate, so are universal for QC (also see Smolin and DiVincenzo [67], 
DiVincenzo et al [68, 69], Poyatos et al [70], Mozyrsky et al [71,72,73]). Lloyd [74] then proved that almost any 2-qubit quantum logic gate (with certain 1-qubit gates) is universal for QC.
Monroe et al [75], DiVincenz et al [76] gave experimental demonstrations of
quantum gates. 
[77] defined a quantum computing model
known as a {\it quantum gate array} which allows
 execution of a (possibly cyclic) sequence of quantum gates, where each input is a
qubit, and each gate computes a unitary transformation.

\noindent
{\bf $\bullet$ Quantum Circuits.} Yao [78] restricted
the concept to (acyclic) {\it quantum circuits} which are a generalization of
Boolean logic circuits for quantum gates. 
 It suffices that a quantum circuit use only these universal gates. 
Yao [78] proved that QTM computations are equivalent to uniform quantum 
 circuit families. 
 Bernstein and Vazirani [50] showed that quantum gates of only  logarithmic accuracy suffice for polynomial time quantum circuits.
Aharonov et al [79] discusses a generalization of quantum circuits to
allow mixed states, where measurements can be done in the middle of the
computation, and showed that such quantum circuits are equivalent in
 computational power to standard quantum circuits. 
This generalized an earlier result of Bernstein and Vazirani [50] 
that showed
that all
observation operations can be pushed to the end of the computation,
 by repeated use of a quantum XOR gate construction.
 Aharonov et al [80] considered  a adiabatic model of quantum computation
and showed it is equivalent to standard quantum computation.

\noindent
{\bf $\bullet$ Computer Simulations of QC.}
Obenland, Despain [81, 82, 83] have given efficient
 computer simulations of QC, including errors and decoherence, and Cerf, S. E. Koonin
[84] have given Monte Carlo simulations of QC.




\subsection{Complexity Bounds for QC}
\label{sec-complexity}

\subsubsection{Quantum Complexity Classes and Structural Complexity}

Berthiaume, Brassard [85] survey open QC structural complexity problems 
(also see Berthiaume [86]).
QC can clearly execute deterministic and randomized computations with no slow down. 
P (NP, QP, respectively) are the class of problems solved by deterministic 
(nondeterministic, quantum, respectively) polynomial time computations.
Thus QP is the quantum analog
of the time efficient class P. It is not known if QP contains NP, that is if QC can
solve NP search problems in polynomial time. It is also not known whether QP is a
superset of P, nor if there are any problems QC can solve in polynomial time that are
not in P (but this is true given quantum oracles; see
Berthiaume, Brassard [87,88], Machta [89], van Dam [90, 91] 
for complexity bounds for computing with quantum oracles). 

\subsubsection{Bounded Precision QC}

Let {\it BQP} be the class of polynomial time quantum
 computations that are computed within bounded error.
 Most of the algorithms we will mention
 (such as Shor's) are in the class BQP. 
 [50] showed that BQP computations
 can be done using unitary operations with a fixed irrational rotation.
 Adleman et al [92] improved this to show that 
BQP can be computed using only unitary operations with rational rotations, and that BQP
 is in the class PSPACE of polynomial space computations of (classical) TMs.
  Practical implementations of QC most likely will need to be done via unitary transitions within 
some modest amplitude precision. Bernstein,
Vazirani [50] proved that BQP computations running in time $T$ can be done with  unitary operations specified by only $O(\log T)$ bits of precision.

\subsubsection{ Quantum Parallel Complexity Classes}

Let NC (QNC, respectively) be the class of (quantum, respectively)
 circuits with polynomial size and polylogorithmic depth. Thus QNC is the quantum analog
of the processor efficient parallel class NC.
Moore, Nilsson [93] define QNC and show various problems are in QNC,
for example they show that the quantum Fourier transform can be parallelized 
to linear depth and polynomial size.

\subsection{\bf Bounds on Measurement, Sensing, and Communication}
\label{sec-communication}

\subsubsection{Lower Bounds on Quantum Communication.}

Cleve et al [94] prove linear lower bounds for the quantum 
communication complexity of the inner product function, and give
 a reduction from the quantum information theory problem to the problem of quantum
 computation of the inner product. Knill, Laflamme [95] characterize the
 communication complexity of one qubit.

\subsubsection{Interaction-Free  Quantum Measurement}

 A method for {\it (nearly) interaction-free measurement
(IFM)} specifies
 the design of a quantum optical sensing
 system that is able to determine with arbitrarily high
 likelihood 
if an
obstructing body has been inserted 
into the system,
without moving or modifying its optical components; moreover, 
In the case that the obstructing body is present, IFM 
 uses at most an arbitrarily small multiplicative factor
 of the input intensity 
to do the sensing. 
Kwiat et al [96] 
 (also see [97])
have given a method 
 for IFM which 
does repeated rounds of measurement to affect small phase changes that
eventually determine (via the quantum Zeno effect) 
whether an obstructing body has been inserted.
Kwiat et al [97] assert  their method can be applied to sensing tasks such as photography, but
 the use of their method for IMF has major practical limitations, since 
 if the obstructing body has not been inserted, then 
the amount of sensing can be quite large.

\subsubsection{Interaction-Free Quantum Sensing}

Reif [98] defines
 {\it (nearly) interaction-free sensing (IFS)} similarly to
IFM,
 except an upper bound is imposed on both the intensity
 to do the sensing
(which again is an arbitrarily small multiplicative factor
 of the input intensity)
 whether or not the obstructing body is present. 
A quantum optical method for IFS (but not IFM)
may be used to do I/O
with bandwidth reduced by an arbitrarily small multiplicative factor
of the bandwidth required for 
classical (e.g., conventional optical or electronic) I/O methods
Reif [98] proves there is no method for IFS with unitary 
transformations, and so concludes
I/O bandwidth can not be significantly reduced by such quantum methods for sensing.
(Also see Holevo [99],  Fuchs and  Caves [100] for proof 
 that  quantum methods can not increase the bandwidth for
transmission of classical information.)

\subsection{Quantum Compression} \label{sec-compress}

\noindent
{\bf Summary.}
Although as noted above,  quantum methods can not increase the bandwidth for
transmission of classical information, still in certain cases entangled
 states can be compressed  to fewer qubits. This quantum compression could have important applications in practice, where the number of usable qubits is very limited.
Schumacher [101] considered compression  and decompression
of a noiseless source of $n$ quantum bits (qubits), each sampled independently from a
given mixed  state quantum ensemble. For such a quantum source,
the compression factor obtainable by classical information
theory is limited by the  Shannon entropy, which  in general
(except in the case where the quantum ensemble has only orthogonal  states)  is less than the
quantum compression factor given by the von Neumann
entropy. 
In particular, Schumacher [101] proved a {\it quantum noiseless coding theorem}
 that states that the source's von Neumann entropy is the number of qubits
 per source state which is necessary and sufficient to asymptotically 
(in the limit of large code-block size) encode the output of the source with arbitrarily high fidelity.
The quantum noiseless coding  of Schumacher 
has asymptotically optimal
fidelity and size; the resulting
 compressed number of qubits
 can be far fewer than in the classical case.





\noindent
{\bf Shannon Entropy and the Limitations of Classical Methods for
Noiseless Compression.}
Suppose   $n$ characters from a finite alphabet $\Sigma$  are each sampled independently
over some probability distribution $p$.
%where the probability of $0$ is $p(i)$ for $i$ $ = $ $0,1.$
 In classical information theory,
the Shannon entropy of each character is $H_S(p)$ $ = $ $ -\sum_{a \in \Sigma}^{1}p(a) \log p(a)$.
A  string 
of these $n$ bits may be losslessly  compressed to a bit string of mean length
$H_S(p)n$. 


\noindent
{\bf The von Neumann entropy and Quantum  Noiseless Compression.}
Following Schumacher [101], we assume there is a finite quantum state ensemble $(S, p)$
which is a {\it mixed state} consisting of a finite number of qubit states $S $ $ = $ $
\{|a_0>,\dots,|a_{|S|-1}>\}$, where each
$|a_i>
\in S$ has  probability
$p_i$.  The compressor is assumed to act on blocks of $n$ qubits (so is a block compressor),
and is assumed to know  this underlying ensemble $(S, p).$
The {\it density matrix}  of   $(S, p)$ is an $|S| \times |S|$ matrix 
$\rho $ $ = $ $ \sum_{i=0}^{|S|-1}p_i |a_i><a_i|.$
The {\it von Neumann entropy} (see [102,101]) corresponding to  $(S, p)$ is  $H_{VN}(\rho)
$ $ = $ $ - Tr(\rho log \rho).$ In general, the Shannon entropy $H_S(p)$ is greater than or equal
to the von Neumann entropy. These entropies are equal only when the states in S are mutually
orthogonal.


 
 \noindent
{\bf An  Example:} Consider a slightly more complex example of a source consisting of
a sequence of $n$ photons polarized randomly,
 with equal probability of phase  $0$ or phase angle $\theta.$  (e.g.,  As   a very simple  example of a source with low von Neumann
entropy, consider $N$ photons polarized randomly, 
 equiprobably at 0 or 1. )
In this case, the states are $S $ $ = $ $ \{|a_0>,|a_1>\},$ where  
the first state $|a_0> = |0>,$ corresponds to phase  $0$, and the other state  $ |a_1>
=
\cos
\theta|0>+\sin
\theta |1>$ corresponds to phase angle $\theta$,  and the probabilities are both $p(0)$
$ =$
$ p(1) $$ =$
$ {\frac {1}{2}}$. The density matrix is  $\rho $
$ =$
$ {\frac {1}{2}} |a_0><a_0|$ $ + $ ${\frac {1}{2}} |a_1><a_1|$
$= {\frac {1}{2}} (|0><0|$ $ + $ $ (\cos \theta|0>+\sin \theta |1>)(\cos
\theta<0|+\sin \theta <1|)$ $)$
$= {\frac {1}{2}}$ $ ((1+\cos^2 \theta)|0><0|$
 $ + $ $ \cos \theta \sin \theta |0><1|$
$ + $ $ \cos \theta \sin \theta|1><0|$
$ + $ $ \sin^2 \theta|1><1|$ $)$
 which  has $2 \times 2$
matrix form  
$ {\frac {1}{2}}$ $\left[\begin{array}{cc} 1+\cos^2 \theta & \cos \theta \sin \theta \\
\cos \theta \sin \theta & \sin^2 \theta   \end {array}
 \right]$
over the basis  vector
$\left[\begin{array}{c} |0>  \\ |1>\end {array}\right].$
Then we can find an appropriate $\beta$ which gives a change of basis with new basis  states
$|0'> = |0>$ and
$ |1'> =
\cos
\beta |0>+\sin
\beta |1>,$ providing a diagonal  density matrix  $\rho' $ 
$ = {\frac {1}{2}}$ $\left[\begin{array}{cc} (1+\cos^2 \theta)+\cos \theta \sin \theta \tan
\beta &
0
\\ 0 & \cos \theta \sin \theta+\sin^2 \theta \tan
\beta  \end {array}
\right]$
over the basis  vector
$\left[\begin{array}{c} |0'>  \\ |1'>\end {array}\right].$
Although this source has  high Shannon entropy $H_S(p), $ 
%$ = $ $ 1$,
 it will have 
low von Neumann entropy $H_{VN}(\rho)$ in the case of a small magnitude phase angle $\theta.$
However, note  that the entropies are the same in the special case where $\theta = \pi/2,$
so the states $ |a_0> = |0>$, $ |a_1> = |1>$ are orthogonal and the density matrix is  simply
the diagonal matrix
$\rho$
$ =$
$ {\frac {1}{2}} ( |0><0|$ $ + $ $ |1><1|) $ which  has a diagonal  density matrix  
$\rho = $ $\left[\begin{array}{cc} {\frac {1}{2}} & 0 \\0 &   {\frac {1}{2}} \end {array}
\right]$ over the basis  vector
$\left[\begin{array}{c} |0>  \\ |1>\end {array}\right].$

 




For technical reasons, the unitary compression and decompression mappings
need to preserve the number of bits (some of which are ignored).
 An {\it $n$-to-$n'$ quantum
compressor} is a unitary transformation that maps $n$-qubit strings to $n$-qubit strings;
the first $n'$ qubits that are output by the compressor  are taken as the compressed
version of its input, and the remaining $n -  n'$ qubits are discarded. An {\it $n'$-to-$n$
decompressor}  is a unitary transformation that maps $n$-qubit strings to $n$-qubit strings;
the first $n'$ qubits input to the decompressor are  the compressed version of  the
uncompressed $n$ qubits, and the remaining $n -  n'$ qubits are all $0$.
The {\it source} to the compression scheme is assumed to be a sequence of $n$ qubits
sampled independently from  $(S, p)$. The {\it observed output} is the result of  first
compressing the input qubits, then decompressing them, and finally measurement of the
result (over a basis containing the $n$ inputs)
The {\it compression rate} is $n/n'$ and the {\it compression factor} is $n'/n$.
The {\it fidelity} of the compression scheme is the
probability the observed output  is equal to the original input (that is the probability
that the original qubits are correctly recovered,  from the compressed qubits). The  goal
here is a quantum compression with both a high fidelity and a high  compression 
rate.

\noindent
{\bf Example (Continued):}
Consider again the   example of a source consisting of a sequence of $n$ photons
polarized randomly,
 with equal probability of phase  $0$ or phase angle $\theta$.
If $\theta$ has small magnitude, then a quantum encoder can compress these
 photons into an entangled state using just a few
photons.
 Furthermore, a quantum decoder can the recover  $n$ photons with the original
distribution 
(with arbitrarily high
 fidelity for large $n$) from these compressed photons.


\noindent
{\bf Schumacher Quantum  Noiseless Compression.}
Schumacher
 [101] gave a {\it quantum noiseless coding theorem} which provided
asymptotically optimal noiseless compression of a
sequence of qubits   independently sampled from
 a finite quantum state ensemble $(S, p)$.
% far fewer than in the classical case.
The quantum noiseless coding theorem states that for any
$\epsilon,
\delta > 0$ and sufficiently large $n$, (i) there is an $n$-to-$n'$ quantum compression
scheme with fidelity at least $1-\epsilon$ and compression to length $n' $ $ \le $ $
n(H_{VN}(\rho)$ $ + $ $\delta)$, and (ii)
 any $n$-to-$n''$ quantum compression scheme which gives compression to length $n'' $ $ \le $ $
n(H_{VN}(\rho)-\delta)$, has 
 fidelity $ < 1-\epsilon.$
That is, in the limit of large code-block size, the source's von Neumann entropy
$H_{VN}(\rho)$ is  asymptotically the number of qubits
 per source state which is necessary and sufficient to 
 encode the output of the
 source with arbitrarily high fidelity.
Given a known finite quantum state ensemble $(S, p)$,
Schumacher's compression scheme assumes
a known basis for which  the density matrix $\rho$ is diagonal, with non-increasing values along the diagonal.




The proof of the Schumacher quantum noiseless coding theorem and its refinements
 by Jozsa and  Schumacher [103] and  H.Szeto [104] 
 make use of the existence of a  {\it typical subspace}
$\Lambda$
 (see [103]) within a Hilbert space of $n$ qubits  over a source of
von Neumann entropy $H_{VN} (\rho)$.
The  typical subspace $\Lambda$ has  dimension $ \le $ $ 2^{nH_{VN}(\rho)}$
and with high probability,
 a sample of $n$ qubits has an almost unit
projection onto $\Lambda$. 
The
Schumacher compressor  simply
 transposes (via a permutation mapping) the subspace $\Lambda$  into the Hilbert space of a smaller block of
$nH_{VN}(\rho)$ qubits.
These proofs are not completely constructive. 

 Bennett [105]
gave a  constructive presentation of the Schumacher  compression.
He observes that the Schumacher  compression can be done by a
unitary mapping to a  basis for which  the density matrix $\rho$ is diagonal
(in certain simple cases the
density matrix $\rho$  is already  diagonal, e.g., when the input is a set of $n$ identical qubits)
followed by certain combinatorial computation which we will call the
{\it Schumacher compression function SCHUMACHER}.
The Schumacher compression function SCHUMACHER simply orders the basis states first by the number of ones 
(from smallest to largest) that are in the binary expansion of the bits
and then refines this order by a lexical sort of the
the binary expansion of the bits.
That is, all strings with $i$ ones are mapped before all strings with $i$ $ + $ $1$ ones,
and those strings with the same number of ones are lexically ordered.
Note that for any given value $X$ of the qubits, this transformation SCHUMACHER$(X)$ is simply
 a deterministic mapping from an $n$ bit sequence to a $n'$ bit sequence
defined by a combinatorial computation. 
In particular, given an $n$ bit binary string $X$,
 the transformation SCHUMACHER$(X)$ is the number of $n$ bit
strings so ordered before
$X$. It is easy to show that SCHUMACHER$(X)$ is a permutation. 
Since it is a permutation,  it is a bijective function which is
uniquely reversible, and also is a unitary transformation.
 To insure that the overall transformation (for all the states)
is a quantum computation, it is essential that the
 transformation SCHUMACHER$(X)$ be done using only reversible, quantum-coherent elementary
operations. (Bennett et al [106]  gave a polynomial time  quantum algorithm for the related
problem of extraction of only classical information from a quantum noiseless coding.)
Cleve, DiVincenzo [107] then developed  the
first polynomial time algorithm for  Schumacher noiseless compression of $n$ qubits.
In particular,  they explicitly
computed the bijective function SCHUMACHER$(X)$ and it's reverse using
$O(n^3)$ reversible,  elementary unitary operations.
Up to then, this was the fastest previous algorithm for
the Schumacher
encoding and decoding functions.
Recently Reif and Chakraborty [108] gave a time efficient
algorithm for asymptotically optimal noiseless quantum  compression and decompression,
costing only $ O(n (\log^4 n )\log \log n)$ elementary quantum operations.
This modified Schumacher encoding requires the evaluation of various combinatorial sums,
for which Reif and Chakraborty provides efficient recursive,
reversible quantum algorithms. 
The coding of   [108] employed a modified Schumacher encoding that was still asymptotically optimal
in fidelity and size.

The Schumacher quantum noiseless coding theorem assumes the compressor  knows the
source. Jozsa, et al [109] recently gave a generalization of the Schumacher  compression to
the case where the compressor does not know the source, thus providing the first
asymptotically optimal  universal algorithm for quantum  compression.
Also,  Braustein, et al [110] have recently given a fast algorithm for an quantum analog of
Huffman coding, but do not provide a proof that this coding gives 
asymptotically optimal noiseless quantum  compression (that is, reaches the von Neumann entropy),
as  provided by Schumacher compression.
 ( [104] assumes the  compressor  knows the source, but can be 
(extended to a  asymptotically optimal  universal algorithm for quantum  compression
where the compressor does not know the source,
using the techniques of Jozsa, et al [109].)

\subsection{Quantum Error Correcting Codes}
\label{sec-codes}

\subsubsection{Quantum Coding Theory.}

The qubit can be defined in quantum information theory as the amount of information that can be carried in a quantum
 system with two basis states, e.g. the internal degree of freedom of a polarized photon. The qubit is thus fundamental unit of quantum channel capacity.
Nielsen [111]  (Svozil [112,113], Holevo [114], Knill, Laflamme [115,116], Ohya [117], develop a theory of quantum error-correcting codes and quantum information theory), e.g., they give the definition of {\it quantum mutual entropy} for an entangled state. Buhrman et al [[118], Adami, Cerf [119] 
 contrast quantum information theory with classical information theory. 
Quantum channel capacity has been investigated for noisy channels (DiVincenzo, et al [120], Holevo [121], Barnum et al [122]), very noisy channels (Shor, Smolin [123]), and 
quantum erasure channels (Bennett et al [124]).
Fuchs [125]  showed that nonorthogonal quantum states maximize classical information capacity.
(Also, Helstrom [126,127] defines a quantum theory of information detection,
and Fuchs [128] defines a related quantum theory of information distinguishability.)

\subsubsection{Decoherence Errors in QC.}

 {\it Quantum decoherence} is the gradual
introduction of errors of amplitude in the quantum superposition of basis states.
 All known experimental implementations of QC suffer from the gradual decoherence of entangled states. 
The rate of decoherence per step of QC
 depends on the specific technology implementing QC.
A significant property of Shor's algorithm
 is that the precision of the amplitudes in the superpositions
 need be only a polynomial number of bits.
Although the addition of decoherence errors in the amplitudes may at first not 
have a major 
effect on the QC, the affect of the errors may accumulate over time
and completely destroy the computation.
%Consequently,
%there have been developed a techniques for correcting decoherence errors in QC. 
 Researchers have dealt with
decoherence errors by extending classical error correction techniques to quantum analogs.
Generally, there is assumed a decoherence error model where the errors introduced are
assumed to be uniform random with bounded magnitude, independently for each qubit.

\subsubsection{Quantum Codes.}

 Shor [129]  and Steane [130] gave the first techniques for reducing quantum decoherence, 
by the addition of 
extra qubits which are then projected via observation operations to 
eliminate errors in the superposition.
Calderbank, Shor [131] and Steane [132] then proved that 
QC can be done with bounded decoherence error, assuming the error correction mechanism is without error itself.
Bennett et al [133], Laflamme [134] gave the first optimal
 5-qubit codes, leading to asymptotically optimal (for large code blocks)
 quantum error correction codes. Shor [135]  and Kitaev [136,137] extended these techniques to do fault tolerant quantum computation on quantum networks, in the presence of bounded decoherence error, even if the error correction mechanism also suffers from error decoherence errors.
A final innovation (Gottesman et al [138], Aharonov, Ben-O [139], Knill et al [140,141]) was concatenated versions of the above quantum codes that 
allow for arbitrarily long QC in the presence of arbitrary (i.e., not necessarily random) 
 decoherence error below a fixed constant threshold. 
Current bounds on this threshold are very small, 
and it seems likely (although it is not yet known) 
 they can be increased to above the decoherance 
error bounds of experimental techniques for QC. 

Also see the texts [38,39]  on quantum coding theory.

\subsection{Quantum Cryptography} \label{sec-cryptography}

Here we overview Quantum cryptography; also see the following texts: [35,36,37] .
  \subsubsection{Quantum Keys}

Bennett et al [142] and Bennett  and Brassard [143] gave the first methods for quantum cryptography
 using qubits as keys, which were proved to be secure (see however the remarks at the end of this subsection)  against certain types of attacks.
 Surveys of quantum cryptography are given in Bennett, Brassard, Ekert [144], 
Brassard [145],  Bennett, Brassard [146], Brassard [147] , and Gisin 148] . Ozhigov [149] 
 gives a protocol for security of information in quantum databases. Hruby [149]
 discusses further methods for quantum cryptography. Bennett et al [150] ,
 Hughes et al [151]  describes experiments of quantum cryptography, including optical fibers.
 
Bennett et al [152] gave a protocol for quantum oblivious transfer.
 Mayers [153] gives quantum oblivious transfer and key distribution protocols and
 Mayers [154] extends the protocols to noisy channels. Lo, Chau [155] 
 give a quantum key distribution protocol which is unconditionally secure
 over arbitrarily long distance. 

Brassard, Cr�peau [156] gave quantum bit commitment and quantum 
coin tossing protocols. Brassard et al [157] gives quantum bit commitment 
scheme provably unbreakable by both parties. Yao [158] proved quantum protocols
 secure against coherent measurements. Brassard et al [159] shows how to defeat
 classical bit commitments with a quantum computer.
 Chau, Lo [160]  gives further methods for qubit commitment.
Cr�peau et al [161]  gives protocols for quantum oblivious mutual identification.

{\bf Is quantum cryptography actually unbreakable? } Unfortunately, some of the methods for quantum cryptography that are claimed to be unbreakable, can in fact be broken by sidestepping assumptions assumed in the proofs of their security. 
For example, to break the well known quantum cryptography method of  Bennett  and Brassard [143], Brandt [162] provided a method (experimentally demonstrated by Kim et al [163]) that exploited entanglement of momentum with the phase of photons, making observations of the momentum portions to infer transmitted phases. It is not clear what other prior results in  quantum cryptography  could be broken by similar techniques, which places the field of quantum cryptography  in some doubt.
(Also, Lo, Chau [164] have recently argued that quantum bit commitment and 
ideal quantum coin tossing are impossible in certain cases that are not 
covered in the above results.)

\subsubsection{ Distributed Quantum Networks}

Future hardware will have to be fast, scalable, and highly parallelizable.
A {\it quantum network} is a network of QCs executing over a spatially distributed network, where quantum entanglement is distributed among distant nodes in the quantum network. Thus, using {\em distributed entanglement}, a quantum network distributes the parts of an entangled state to various processors, which can to act on the parts independently. Pellizzari [165]  proposes quantum networks using optical
fibers, and Cirac, Zoller et al [166], and Bose, Vedral [167] show state transfer distribution can be
done among distant nodes. For example, [166] use a cavity QED device that traps atoms in multiple
cavities and exchanges photons between the cavities to establish the distributed entanglement.
 Various basic difficulties were overcome:
%\begin {itemize}

\noindent
{\em -- How can one do state transfer distribution?}
Bennett et al [168, 169] , Brassard [170]  developed a technique known as 
{\it teleportation} to transmit arbitrary input states with perfect fidelity.
 It does this by separating the input state into classical and quantum components.
 The input can then be reconstructed from these components with perfect fidelity. 

\noindent
{\em -- How can one cope with communication errors and attenuation in a quantum network?}
Wootters, Zurek [171]  proved that a single quantum cannot be cloned. 
 (note: Buzek, Hillery [172]  recently claimed a universal optimal cloning
 of qubits and quantum registers in a distributed quantum network,
 but this seem inconsistent with the no-coning theorem).
That no-cloning theorem implies that once a signal becomes attenuated in a an optical fiber communication channel,
 then it cannot in general be amplified.
 Hence it would at first appear that communication and quantum network links may be limited to distances of the order of the attenuation length in the fiber.
 However, the range of quantum communication could be extended using {\em quantum repeaters} that do quantum error correction, restoring the quantum signal without reading the quantum information.
 Ekert, Huelga et al [173]
 %, Knill, Laflamme, Zurek [KLZ96] 
extend the techniques of distributed quantum computation to noisy channels, and showed that for quantum memories and quantum communication, a state can be transmitted over arbitrary distances with bounded error, provided a minimum gate accuracy can be achieved which is a constant factor of this error. 

\subsection{Other Algorithmic Applications of QC}
\label{sec-applications}
%{\bf Applications of QC with Efficient Algorithms.}
%QC a number of problems with efficient algorithms.
 The 
early literature in QC provided some examples of QC algorithms for problems constructed 
for the reasonable purpose of showing that QC can solve some problems more efficiently than conventional
sequential computing models. 
Later, quantum algorithms were developed for variety of useful applications.
Also see the texts on quantum algorithms: [40, 27, 28, 29, 30, 31, 32, 33, 34].



\noindent
{\bf $\bullet$ Quantum Fourier Transforms.}
Drutsch, Jozsa [49] gave an $O(n)$ time quantum algorithm 
for creating a uniform superposition of all possible values of $n$ bits, which is a {\it quantum Fourier transform} over the finite field of size $2$. 
Simon [174] used this quantum Fourier transform to gave an efficient time quantum algorithm for determining whether a function over a finite domain is invariant under some XOR-mask.
This provided the one of the first examples of a quantum algorithm that efficiently solves an interesting problem that is costly
 for classical computation. 
Brassard, Hoyer [175] gave improvements to Simon's algorithm.
There have been a number of efficient quantum algorithms for extensions of the quantum Fourier transform:
% to other related problems in commutative groups and number theory:
to the approximate quantum Fourier transform (Coppersmith [176]),
over various domains (Griffiths, Niu [177], Hoyer [178]),
over symmetric groups (Beals [179]),
over certain non-abelian groups (Pueschel, Roetteler, Bet [180]),
Vedral, Barenco, Ekert  [181] give efficient quantum networks for elementary arithmetic operations,
 using the quantum Fourier transform.
Grigoriev [182]  used the quantum Fourier transform to test shift-equivalence of polynomials.

\noindent
{\bf $\bullet$ Quantum Factoring.}
The most notable algorithmic result in QC 
to date is the quantum algorithm of Shor [183].184]
 (also see a review of the algorithm is given by Ekert and Jozsa [185]) for 
discrete logarithm and integer factorization in polynomial time 
(with modest amplitude precision).
%Shor's algorithm uses
% a reduction to determining the period with a 
%integer ring, and continued fractions.
%In landmark work, Shor [Sho94, Sho97]
%gave a polynomial time quantum algorithm for discrete logarithms
% and integer factoring. 
 Shor's algorithm uses an efficient reduction (due to Miller [186])
from integer factoring to the problem of approximately computing the period (length
 of a orbit) within an integer ring.
Shor approximates the period  by repeated the use of a quantum Fourier transform over an integer ring
%(also see Coppersmith [Cop94])
 and greatest common divisor computations.
 %and solved the latter problem by a quantum Fourier transform over an integer ring.
 There has been considerable further work on Shor's quantum factoring algorithm: Zalka [187] 
improved the time complexity, 
Beckman et al [188]  describe it's execution on quantum networks with small size and depth, 
Obenland, Despain [189], Plenio, Knight [190]  consider the feasibility of
 executing Shor's quantum factoring algorithm on various quantum computer architectures
(the latter provide somewhat pessimistic lower bounds for the factorization time of large 
numbers on a quantum computer in the presence of decoherance errors.)
[191]  describes a 7 qubit  demonstration of Shor's factorization algorithm using nuclear magnetic resonance. 
Kitaev [192]  gave an independent derivation of Shor's factoring result
using a reduction to find an abelian stabilizer.



%% Discussions of its practical implementation appear in \cite{}.
%To factor a number of about 200 bits, the number of qubits is?? and the number of steps is??




\noindent
{\bf $\bullet$ Quantum Search.}
Another significant efficient QC algorithmic result is the algorithm of Grover [193], 
 which
 searches within a data base of size $N$ in time $\sqrt{N}$
(An interesting property of the Grover's algorithm for search is its similarity to the 
quantum Zeno affect technique for quantum measurement Kwiat et al [96,97].
In particular, the algorithm also uses $O(\sqrt{N})$ stages
of unitary operations, each quite similar to a stage of the quantum Zeno sensing method.)
Grover refined his result to require only a single query [194],
and to use almost any unitary transformation [195],
Zalka [196]  showed Grover's algorithm can not be further asymptotically sped up and so is optimal for data base search, and Pati [197]  gave further improvements to the bounds.
Biron et al [198],
extended Grover's algorithm to arbitrary initial amplitude distribution. Cockshott [199]  gave fast quantum algorithms for executing more general operations on relational databases, and Benjamin, Johnson [200]  discuss the use of
Grover's algorithm and related quantum algorithms for other data processing problems.
 Farhi et al [201] showed that Grover's algorithm could not be extend to quickly
 determine parity of $N$ bits; in particular they
showed that any quantum algorithm for parity takes at least $N/2$ steps.
Meyer [202] and Terhal \& Smolin [203] propose quantum search algorithms that do not to require entanglement.
Brassard et al [204,205] combine the algorithmic techniques
 of Grover and Shor to give a fast quantum algorithm for approximately counting
 (i.e., finding the number of matches in a database).

While Grover's algorithm is clearly an improvement over 
linear sequential search in a data base, 
it appears less impressive in the case of
an explicitly defined data base which needs to be stored in volume $N$.
Parallel computation
 can do
search in a data base of size $N$ in time at most polylogarithmic with $N$
(that is, in time $O(\log^{O(1)} N)$)
by relatively straightforward use of parallel search
Moreover, Grover's algorithm may not have a clear advantage even in the case of
an implicitly defined data base, 
which does not need to be stored, but instead can be constructed on the fly
 (e.g., that arising from NP search methods). In this case,
Grover's search algorithm can be used to speed up 
combinatorial search within a domain of size $N$ to a time bound of $O(\sqrt{N})$,
(Hogg [206], Hogg, 
Yanik [207]  investigate similar quantum search techniques for local and other combinatorial search problems), 
and in this case 
Grover's algorithm appears to require only volume logarithmic in the search space size $N$.
%but takes a large time $\sqrt{N}$, where as 
In contrast, parallel computation takes volume  linear in the
combinatorial search space, but takes just time polylogorithmic in the
search space.


\noindent
{\bf $\bullet$ Quantum Simulations in Physics.}
The first application proposed for QC (Feynman [46]) was for simulating quantum physics. 
In principle, quantum computers provide universal quantum simulation of any quantum mechanical physical system (Lloyd [208], Zalka[209], Boghosian [210])). Proposed QC simulations of quantum mechanical systems include:
many-body systems (Wiesner [211] ),
many-body Fermi systems (Abrams, Lloyd [212]),
multiparticle (ballistic) evolution (Benioff [213]),
 quantum lattice-gas models (Boghosian, Taylor [214]), Meyer [215,216] ),
Ising spin glasses (Lidar, Biham [217]),
the thermal rate constant (Lidar, Wang [218],
quantum chaos (Schack [219]).








\noindent
{\bf $\bullet$ Quantum Learning.}
QC may have some interesting applications the learning theory and related problems. Bshouty, Jackson
[220] describe learning Boolean formulas in disjunctive normal form (DNF) over the uniform distribution
of inputs, using a quantum example oracle, and  Ventura, Martinez [221] describe a QC learning algorithm
for learning DNF
 using a classical example oracle.
Also, Yu, Vlasov [222] describe image recognition using QC,
 Tucci [223] investigates quantum bayesian networks, and 
Ventura, Martinez [224] describe a quantum associative memory,




\noindent
{\bf $\bullet$ Quantum Robotics.}
Benioff [225] considers a distributed QC system with mobile 
{\it quantum robots} that can carry out carrying out measurements 
and physical experiments on the environment, and as an example gives an algorithm
 for the problem of measuring the distance between a quantum robot and a particle
 on a 1D space lattice. Hogg [206] proposes the use of distributed QC to allow small-scale sensors and actuators to be controlled in a distributed manner. Further discussion of the applications of QC are given by Landauer [226,227].





\noindent
{\bf $\bullet$ Winding Up Quantum Clocks.}
The precision of atomic clocks are limited by the spontaneous decay lifetimes of excited atomic states. An interesting application of QC proposed by Huelga [228] (also see Bollinger et al [229]) is to extend these lifetimes by using quantum error correcting codes to inhibit the spontaneous decay. A similar idea can be used for improving the precision of frequency standards and interferometers. 

\noindent
{\bf $\bullet$ Quantum Strategies.}
Meyer [230, 231]  has proposed a class of generalized games that allow for quantum  strategies which he proves provide an improvement over conventional mixed (randomized) strategies  for certain games.


\subsection{Possible Technologies  for Doing QC} \label{sec-technologies}
Here we overview various experimental implementations of quantum computation; also see the texts: [41, 42, 43, 44, 45].
As noted previously, any QC can be realized by a {\em universal} set of gates consisting of
the 2-qubit XOR operation along with some 1-qubit operations.
There are two basic approaches known to do QC:

\noindent
{\bf (A) Micromolecular QC.} Here QC on $n$ qubits is executed 
using $n$ individual atoms, ions or photons, and each qubit is generally encoded
using the quantized states of each individual atom, ion or photon.
The readout (observation operation) is by measurement of the 
(eigen) state of each individual atom, ion or photon.
In the following we enumerate a number of proposed micromolecular QC methods:
 
\noindent
{\bf $\bullet$ Quantum Dots.} Burkard [232], Loss et al [233],
 Meekhof et al [234] describe the use of coupled quantum dots to do QC.
 ([80] proposes quantum computation using Cooper pairs.) 

\noindent
{\bf $\bullet$  Ion Trap QC.} Cirac, Zoller [235,236], James [237]
 proposed using a linear array of cold trapped ions
(the ions are trapped by electromagnetic fields)
whose energy states
are  used to store  the qubits 
 (also, vibrational modes between consecutive ions also can be used to store states of qubits).
The coupling of the qubits is by electrostatic repulsion between the ions.
Unitary transitions on superpositions can be executed via an associated array of
lasers, each of which pulses a distinct ion; these
 induce electric dipole moments that determine the transitions.
A group at the National Institute of Standards at Boulder, CO (Meekhof et al [234], Wineland et al [238,239] , King et al [240], Turchette et al [241]) and a group at Los Alamos (Hughes [242], Hughes et al [243], James [244]) have experimentally demonstrated trapped ion QC.
 These and other researchers have addressed various key
 issues associated with quantum computation with trapped ions: 

\noindent
{\bf $-$} Deterministic entanglement of two trapped ions (Turchette et al [241] ),

\noindent
{\bf $-$} decoherence bounds (Hughes et al [245] and Plenio, Knight [246] ),

\noindent
{\bf $-$} measurement and state preparation, i.e., initialization of the collective motion of the trapped ions
(Schneider et al [247]  and King et al [240]),

\noindent
{\bf $-$} 
coherent quantum-state manipulation of trapped atomic ions (Wineland et al [238,239]),

\noindent
{\bf $-$} 
heating of the quantum ground state of trapped ions (James [248]) 
and quantum computation with ``hot'' trapped ions (Schneider et al [249]).


\noindent
{\bf $\bullet$ Cavity QED.} A group at Cal Tech (Turchette [250]) 
have experimentally demonstrated the use of trapped photons
 in a cavity QED system to execute 2-qubit XOR gates
and thus in principle can do universal QC.
The qubits are encoded by the circular polarization of photons.
 interacting.
The XOR unitary transitions on superpositions
 can be executed by resonance between interacting photons in the cavity;
The coupling of qubits is via resonance between interacting photons 
using a Cesium atom also in the cavity, and the coupling is tuned by the spacing of mirrors in the cavity.

\noindent
{\bf $\bullet$ Photonics.} Various groups Chuang et al 
[[251, 252]  , Torma, Stenholm [253]  have experimentally demonstrated
 QC using optical systems where qubits are encoded by photon phases and universal 
quantum gates are implemented by optical components consisting of beamsplitters 
and phase shifters as well as 
%(in the case of [TS96])
nonlinear media 
 (also see the linear optics QC proposed by Adami, Cerf [254]).

\noindent
{\bf $\bullet$ Heteropolymer.} This is a polymer consisting of a
linear array of atoms,
each of which can be either in a ground or excited energy state.
Teich et al [255]  first proposed classical (without quantium superpositions) molecular computations using heteropolymer. Later Lloyd [256] extended the use of heteropolymers to QC, using the energy states
 to store the state of the qubits.
The coupling of qubits may be via electric dipole moments which causes energy shifts on adjacent atoms.
Unitary transitions on superpositions can be executed via pulses of a laser at particular frequencies;
 these
induce electric dipole moments that determine the transitions.

\noindent
{\bf $\bullet$ Nuclear Spin.} DiVincenzo [257] Wei et al [258, 259] proposed the use of nuclear spin to do QC; see the remarks following the discussion of Bulk QC.

\noindent
{\bf $\bullet$ Quantum Propagation Delays.}
Castagnoli [260]  proposed to do QC using retarded and advanced propagation
 of particles through various media.



Of these, Ion Trap QC,
 Cavity QED QC,
and Photonics have been experimentally demonstrated up to
a very small number of qubits (about 3 bits).
 The apparent intention of such micromolecular methods for QC is to
have an apparatus for storing qubits and executing unitary operations
(but not necessarily executing observation operations) 
which requires only volume linear in the 
number of qubits. 
One difficulty (addressed by Kak [261], Murao et al [262]) is
 {\em purification of the initial state}:  
if the state of a QC is initially in an entangled state, 
and each of the quantum gate transformations introduces phase 
uncertainty during the QC, then effect of these perturbations may accumulate
 to make the output to the QC incorrect.
A more basic difficulty for these micromolecular methods is that
they all use experimental technology that is not well established as might be;
in particular their approaches each involve containment of 
atomic size objects (such as individual atoms, ions or photons)
and manipulations of 
their states. A further difficulty of the micromolecular methods for QC is that
 apparatus for the
 observation operation,
for even if observation is approximated, 
seems to require volume growing exponential with the number of qubits,
as described earlier in this paper.

 \noindent
{\bf (B) Bulk (or NMR) QC.} Nuclear magnetic resonance (NMR) spectroscopy is an imaging technology using the spin
of the nuclei of a large collection of atoms. Bulk QC is executed on 
a macroscopic volume containing, in solution
a large number of identical molecules, each of which
 encodes all the qubits. 
The molecule can be chosen so that it has $n$ distinct quantized spins modes (e.g., each of the 
 $n$ nuclei may have a distinct quantized spins). Each of the $n$ qubits is encoded by one of these spin modes of the molecule.
The coupling of qubits is via spin-spin coupling between pairs of distinct nuclei.
Unitary operations such as XOR can be executed by radio frequency (RF) pulses at
 resonance frequencies determined in part by this spin-spin coupling
between pairs of nuclei (and also by the chemical structure of the molecule).
Bulk QC was independently proposed by  Cory, Fahmy, Havel [263] and Gershenfeld, Chuang [264, 22].
 Also see Berman et al [265] and the proposal of 
Wei et al [259]  for doing NMR QC on doped crystals rather than in solutions, and see
Kane [266]  for another solid state NMR architecture for quantum computing using  silicon.
 
\noindent
{\bf $\bullet$  Bulk QC} was experimentally tested (Jones et al [267]) and applied to demonstrate the following tasks: 
quantum search  (Jones  [268]),
approximate quantum counting (Jones, Mosca [269])
Deutsch's problem (Jones, Mosca [270]),
Deutsch-Jozsa algorithm on 3 qubits (Linden, Barjat, Freeman [271]).

\noindent
{\bf $\bullet$ Advantages of Bulk QC:} (i) it can use well established NMR technology
and in particular macroscopic devices,
 The main advantages are (ii) the long time duration until decoherence (due to a low coupling with the
environment) and (iii) it currently scales to more qubits than other proposed technologies for QC.





\noindent
{\bf $\bullet$ Disadvantages of Bulk QC:} 
A possible disadvantage of Bulk QC is that it appears to allow only a {\em weak} measurement
of the ensemble average which does not provide a quantum state reduction;
that is the weak measurement does not alter (at least by much) the superposition of states.
Another disadvantage of Bulk QC is that 
it may require, for a variety of reasons, macroscopic volumes, 
and in particular volumes which grow exponential with the number of qubits.
Macroscopic volumes may be required for measurement via conventional means.
%as described in the introduction.
However, known quantum algorithms can still be executed even in this case (e.g., see Gershenfeld, Chuang
[264, 22]). So the lack of strong measurement is not a major disadvantage.

Also, Bulk QC requires the
 initialization to close to a pure state.
 If Bulk QC is done at room temperature, the initialization methods of
 Cory, Fahmy, Havel [263] (using logical labeling) and 
 Gershenfeld, Chuang [264, 22](using spatial averaging) yield a pseudo-pure state,
 where the number of molecules actually in the pure
 state drops exponentially as $1/c^n$ with the number $n$ of qubits,
for some constant $c$ (as noted by Warren [272]). 
If we approximate the resulting measurement error
by a normal distribution, the measurement error is (with high likelihood) at least 
 a multiplicative factor of $1-c'/\sqrt{N}$,
for some constant $c'$.
To overcome this measurement error, we need 
$1/c^n > c'/\sqrt{N},$ and so 
we require that the volume be at least $N > (c^n/c')^2.$
Hence, for the output of the Bulk QC  to be (weakly) measured, 
the volume (the number $N$ molecules) of Bulk QC  needs to grow exponentially with the number $n$ of qubits.
Recently, there have been various other proposed methods for initialization to a pure state:

\noindent
{\bf $\bullet$} Barnes [273] proposes the use of very low temperatures,

\noindent
{\bf $\bullet$} Gershenfeld, Chuang [264, 22] suggest the use of gradient fields.

\noindent
{\bf $\bullet$} Knill et al [274]  suggest a randomization technique they call temporal averaging.

\noindent
{\bf $\bullet$} Recent work of Schulman, Vazirani [275]  provides polynomial volume for initialization,
with the assumption of an exponential decrease in spin-spin correlations
with the distance between the nuclei located within a molecule
(in particular, they assume  that the statistical correlation between and two bits
 on  a molecule falls of exponentially with the distance between these bits).
Although their methods may provide a solution in practice,
 known inter-atomic interactions such as the
spin-spin correlations are generally considered to be governed by potential force laws which decrease
by inverse polynomial powers rather than by an exponential decrease.
% (Nevertheless, )

It has not yet been experimentally established which of 
these pure state initialization methods scale to a large number of qubits without large volume.


(Note: Some physicists feel that it has not been clearly established whether:
(a) NMR is actually a quantum phenomenon with quantum superposition of basis states, or
(b) if NMR just mimics a quantum phenomenon and is actually just classical parallelism,
 where 
the quantum superposition of basis states is encoded using multiple molecules
 where each molecule is in a distinct basis state.
If the latter is true with each molecule is in a distinct basis state,
 then (see Williams and Clearwater [23]) the volume may grow 
grow exponentially with the number $n$ of qubits, since each basis state
 may need to be stored by at least one molecule, and
the number of basis states can be $2^n$.
Also, even if each molecule is in some partially mixed quantum state
(see Zyczkowski et al [276]), the volume may still need to grow very large.)

In summary, some possible disadvantages of Bulk QC that may make it difficulty to scale are
 (i) the inability to do observation (strong measurement with quantum state reduction), 
(ii) the difficulty to do even a weak measurement without the use of exponential volume, 
(iii) difficulty (possibly now resolved) to obtain pure initial states without the 
use of exponential volume, (iv) the possibility 
 that Bulk QC is not a quantum phenomena at all (an unresolved controversy within physics), 
and so may require use of exponential volume.


It is interesting to consider whether NNR can be scaled
 down from the macroscopic to molecular level.
 DiVincenzo [257], Wei et al [258, 259] propose 
doing QC using the nuclear spins of atoms or electrons
 in a single trapped molecule. The main advantages are (i) small volume and 
 (ii) the long time duration until decoherence (an advantage shared with NMR).
 The key difficulty of this approach is
 the measurement of the state of each spin, which does not appear to be feasible by
the mechanical techniques for detection of magnetic
 resonance usual used in NMR, which can only do detection of the spin for large ensembles of atoms.

%Next  we discuss resource bounds for QC, including 
%time, volume, and energy.

\subsection{Resource Bounds} \label{sec-resources}
In this paper, we have discussed many applications of quantum computation
which provide advantages over classical methods of computation.
Certain of the applications of QC (e.g., quantum cryptography) require only a small or constant
number of qubits, where as other applications (e.g., factoring and data base search) 
require a large number of qubits and moreover require an observation operation
at least as the final step of the QC.
For these
advantages to be practical, we need to  determine that there are no  unfeasible large resources
required by QC.
 Hence we complete the
paper with a review of the resource bounds of quantum computing as compared
with the resources required by classical methods for computation.
In particular, we will conclude that for the advantages of QC (with a large number of qubits) to
be practical for applications 
requiring a large number of qubits,
there needs to be determined (theoretical and practically)  bounds on the
volume required of observation operations. This seems to us a major missing element
in the field of QC.


The energy consumption, processing rate, and volume,  are all
 important resources to consider
 in  computing devices.
Conventional (classical) electronic supercomputers  of the size of a work station  operate 
in the range of 
$ 10^{-9}$ Joules per operation, at up to 
about $50$ giga-ops  per second, with memory of about $10$ to $100$  giga-bytes, and
in a volume of about 10 cm$^2$. The volume scales as the number of bits of storage.


\noindent
{\bf $\bullet$ Energy Bounds for QC.}
The conventional linear model of QC allows only unitary state transformations 
and so by definition is reversible (with the possible exception of the 
observation operation which does quantum state reduction).
 Benioff [48] noted that as a consequence of the reversibility of 
the unitary state transformations of QC, these  transformations dissipate no energy.
But this does not consider (i) the precision of the amplitudes to be preserved nor (ii) the expected time duration required to drive the operation to completion.
Gea-Banacloche [277] and Ozawa [278]   independently derived lower bounds on the energy needed to execute,  within a given precision of the amplitudes and in a given time, an elementary qubit  logical operation on a quantum computer. 
They derived energy lower bounds depending inversely on the time duration for the operation and on the precision of the amplitudes to be preserved. 
Hence, for polynomial time quantum computations requiring  polynomial relative precision, the lower bounds  on energy  are polynomial, though the constant factors could be a limitation for practical implementations.
(Recall that Bernstein,
Vazirani [50] proved that BQP computations  can be done with  unitary operations specified by only logarithmic bits of precision, which corresponds to relative precision $\epsilon$ where $\epsilon > 1/n^{O(1)}.$)
These energy lower bound  results  were stated to be independent of the nature of the physical system encoding the qubits, and under what the authors claimed was normal circumstances  in a wide variety of conditions for implementations of quantum computers. Nevertheless, the matter is still appears not to becompletely resolved, since there may  be physical implementations of quantum computers that do not abide by their assumptions. Energy bounds for the 
quantum qubit logical operations  require better understanding and study, particularly with respect to their dependance on  the technology used.

\noindent
{\bf $\bullet$ Processing Rate of QC.}
The rate of execution unitary operations in QC depend largely on the implementation technology
(see  Section
\ref{sec-technologies});
certain techniques can execute unitary operations in  microseconds (e.g., Bulk NMR) and
some might execute 
at microsecond or even picosecond rates (e.g., photonic techniques for NMR)
The time duration to do observation can also be very short, but
may be highly dependant on the size of the measuring apparatus and on the 
required precision (see the below discussion on the observation operation and its volume).



\noindent
{\bf $\bullet$ Volume Bounds for QC.}
We now consider (perhaps more closely 
than usual in the quantum literature)
the volume bounds of QC.
Potentially, the modest volume bounds of QC may be the one significant advantage 
over classical methods for computation. 
Due to the {\it quantum parallelism} (i.e., the  superposition of the basis states 
allow each basis state to
exist in parallel), the volume would at {\it appear} to be no more that the number of qubits.
This may be true, but there are a number of substantial issues that need to be carefully considered.
Recall the observation operation both provides a measurement of a qubit with a resulting 
 state reduction. 
However, the QC literature has not yet carefully considered the volume bounds
for the observation operation and as we shall see, it is not yet at all clear what the volume is required. 
In spite of major works on the mathematical and physical foundations of quantum observation,
% are not clearly understood,
 the precise nature of quantum state reduction via a strong quantum measurement remains somewhat of 
a mystery.
Two distinct approaches to the mathematical and physical foundations of observation have been developed:

\noindent
  (a) The {\it Copenhagen Formulation,} where the observation is simply
{\it an assumed  basic operation} and is considered to be done by a
 macroscopic measuring device, and


\noindent
  (b) The {\it  Von Neumann Formulation} [102, Chapter 4: Macroscopic Measurement], which views
the measuring apparatus as well as the quantum system measured as both
part of a quantum system.   
Hence the evolution of the system (and
resulting experimental predictions) can be distinct from that predicted by the Copenhagen formulation 
of observation
 (which does not take this into account since the
measuring apparatus is assume in their formulation to be very large).

See Cerf and Adami [18]
for a comparison  the Copenhagen and Von Neumann formulations
and see  Hay and Peres [279] for an example of this difference.
In summary, the Copenhagen  and the von Neumann formulations for
observation differ in the assumed context (macroscopic or microscopic
measurement apparatus). (Note: Attempts to  rectify the difference between the  Copenhagen  and the von Neumann formulation for observation
are given in  Hay and Peres [279]  and in Zurek [280], but it appears not yet resolved.)
The Copenhagen formulation for observation is
generally  used in the context of  quantum physics experiments which 
use  macroscopic  measurement apparatus. 
However, the Copenhagen formulation does not seem to be applicable in the context
 of a microscopic  measurement apparatus,
which is so small that it is subject to quantum effects (and thus is within a unitary quantum system).
So the Copenhagen formulation for observation
may  not appropriate for molecular size QC.
Although the  von Neumann formulation of
observation is not relevant to the vast majority of physics experiments
(since  their experiments generally use large measuring apparatus and
small number of degrees of freedom (qubits)),  nevertheless the von Neumann
formulation for observation appears to be appropriate for molecular size QC.
It is possible that  the volume for quantum observation apparatus grows very quickly
 with the number of qubits in the von Neumann formulation.
In particular, no one has proved a upper bound
on the  volume for quantum observation (as a function of the number of qubits)
assuming the  von Neumann formulation.
See the Appendix
%Reif [Rei98] 
for a further discussion of the problem of determining volume bounds for the observation operation
in QC.


\noindent

\subsection{Conclusion and Acknowledgments} \label{sec-conclusion} 
We have overviewed the field of quantum computing and surveyed its major algorithmic results and applications as well as physical implementations and their limitations. Some of  issues such as energy costs as well as volume of observation apparatus for quantum computing are still unresolved, and the later are further addressed in the Appendix.

{\bf Acknowledgments.} We would like to thank G. Brassard for his clear
 explanation of numerous results in the field of QC. Also, I would like to thank
 P. Shor and U. Vazirani for references and illuminating discussions on quantum computation, and in particular
on the issue of volume bounds for quantum observation.


\noindent
\subsection{Appendix: Volume of Observation Apperatus for Quantum Computing} \label{sec-observation}
Here we discuss the challenge of providing volume bounds for observation apperatus when doing QC.


\subsubsection{A Potentially Fallacious Proof of Small Volume} 

First we note that one might be tempted to give a constructive proof,
that observation can be done on $n$ qubits in small volume,  along the following lines:

\noindent
 (i) {\it Basis Step.} We begin with a simple, well established experimental method
for observation of a single qubit in small quantum system with say $n_0$ qubits, for a constant $n_0$.
There are many other examples of  experimentally verified  methods for observation,
using macroscopic  measurement apparatus.
(For example, a number of proposed QC architectures (e.g., the Cirac and Zoller [232,233] proposed ion trap QC 
and Kane's [262]  silicon-based NMR QC) give specific descriptions of
 measuring apparatus that have been experimentally verified
for observation of a single qubit within a quantum computing systems with a constant number of qubits. 
While their measuring apparatus is macroscopic, it still must
 have just some finite volume.




\noindent
(ii)  {\it Inductive Step.}
However, then we just scale up by using the same  experimental apparatus
to do observation on each of $n$ qubits (that is,  repeating the observation for each of the other qubits).
This seems to result in a small volume (perhaps even liner size)
apparatus for observation.

\noindent
The potential fallacy of this line of argument is that:

(a) In the basis step, the experiments of [232,262] did not provide bounds on the errors (or fidelity)  of the measurement 
 as a function of the volume of the  measuring apparatus. 

(b) The inductive step fails to take into account  quantum effects involving 
both the  measuring apparatus and the $n$ qubits, as might be predicted
 by the von Neumann formulation of quantum measurement in the case where the
measuring apparatus is so small that it is subject to quantum effects.

 That is, there needs to be given, in addition to the 
experimental description (which is only  established for $n_0$ qubits):

\noindent
(iii) {\it A mathematical
analysis of the quantum effects (in the context of a closed unitary system)
 involving the measuring apparatus as the number $n$ of qubits grows large.}
In particular, there need to be determined bounds on the errors (or fidelity)  of the measurement 
 as a function of the size of the  measuring apparatus.

Without this crucial final element, the proof is certainly not complete.
Since the observation operation is not reversible,
such a  proof (in the context of a closed unitary system) seem unlikely to be obtainable.


\subsubsection{Possible Experimental Demonstrations of Measurement:}  Another approach
% to complete the proof,
 would be to experimentally test a proposed small volume apparatus for observation
on $n$ qubits for moderate size $n$ (say, in the rang of a few hundred,
which is  required for a nontrivial factoring computation).
But the experimental evidence of the volume bounds for observation is unclear,
 since the QC experiments have not yet been scaled to large or even moderate numbers (say dozens) of qubits,
and there are few if any physics experiments for this case.
(Shnirman, Schoen [123] describe the use of a single-electron transistor to perform quantum measurements, 
D'Helon, Milburn [281] describe quantum measurements with quantum computers, 
and Ozawa [282] describes methods for nondestructive (known as {\it nondemolition}) quantum measurements of certain quantum computations.)

Hence, at this time that there appears to be {\it neither a mathematical proof
nor an experimental demonstration} (for even a moderately large  number of qubits $n$) {\it that observation
can be done in small volume} (in a closed quantum system).
Thus at this time, there is no evidence (either  mathematical 
or  experimental) that QC using measurement scales to large numbers of qubits with
small volume.


We first consider a number of related questions concerning measurement
 and quantum state reduction:
%\begin{enumerate}

\noindent
\subsubsection{ Is a Quantum Observation Instantaneous?}
 It appears not. Brune et al [283]  describe the progressive decoherence of the meter in a quantum measurement.

\noindent
\subsubsection{Is an Observation Always Reversible?}
 It appears the answer be both no (in a narrow mathematical sense of a state reduction),
 yes (for small closed state spaces), 
and no (in a practical sense for entanglements in a large state space):
\begin{itemize}
\item By the strict mathematical definition of the state reduction due to observation,
in general an observation  is  not reversible.
Under what conditions is a measurement reversible in the strict mathematical sense? That is,
when can we measure classical information from a quantum source
 (yielding a set of pure states with their probabilities with a reduction of quantum entropy),
 but later be able to reverse this process to regenerate the entangled source state ?
 Bennett et al [106]  show that this is possible in the very special case where
%if and only if
 the source states can be  partitioned into two or more mutually orthogonal subsets. 
(Other necessary and sufficient conditions for measurements to be reversible have been proved
in Bennett, et al [106]  and
Chuang, Yamamoto [284] 
 describe how to regenerate a qubit if it has observable error.)
\item There is experimental evidence that the physical execution of some
 reductions via measurement are in fact reversible (at least in very small closed systems).
 Mabuchi, Zoller [285] have observed inversions of quantum jumps in very small quantum-optical systems under continuous observation, and Ueda [286] compares the notions of mathematical and physical reversibility.
 \item On the other hand, in the case of entanglements in a large state space, even if a measurement
 is in principle reversible in a closed system due the reversible nature
 of the diffusion process,
 the likelihood of such a reverse to the original state, within a moderate
 (say polynomial in $n$) time duration, appears to drop
%vanishingly
exponentially  with the number of qubits $n$. 
Gottfield [287] Diosi, Lukacs [288]  (also see Pearle [289,290]) explain quantum
state vector reduction via strong measurement as a physical process, e.g, state  diffusion into the 
atoms of the measurement apparatus.
% or into the environment. 
%Generalizing the rapidly mixing markov systems in probability theory,
This  diffusion due to reduction may be modeled by a  system similar to a
rapidly mixing markov system in probability theory,
 which seems to provide a very low  (dropping exponentially with  $n$)
likelihood  for reversibility within a polynomial time duration.
(Others have modeled measurement by a nonlinear interactions with the environment,
which are irreversible.)

\end{itemize}

\subsubsection{Avoiding Observation Operations ?} 

An alternative approach is to completely avoid observation operations on the basis that the
 observation operation is not actually essential to many quantum computations.
%is to show that the reduction operator 
%is not actually essential to many quantum computations.
%would be to entirely eliminate the observation operation from QC.
(This seems somewhat surprising,
 given the extensive use of the observation operation in the QC 
literature for both algorithms and 
quantum error correction.)
Bernstein and
 Vazirani [50] (by showing that any given  observation operation can be delayed to future steps
by use of the  using XOR operation) proved that all observation operations can be delayed to the final step
of a quantum computation. 
 For a small $\epsilon > 0,$
let some particular qubit (of the linear superposition of basis states) be  {\em
$\epsilon$-near classic} if had the qubit been observed,
 the measured value would be a fixed value (either be 0 or 1) with $\epsilon$ probability.
Suppose the output of a QC consists of the observation of a subset $S$ of the qubits; 
the resulting reduced superposition will be
 termed the {\em output superposition}.
Bernstein and
 Vazirani  [50] and Brassard et al [172,201] observe that 
any QC  can be repeated to insure the output qubits are $\epsilon$-near
classic in
the final output superposition
 after the repetitions.
Note that if a QC with bounded
 amplitude precision 
is reduced by an observation,
the output qubits yield the correct value
 with high likelihood.
Hence we may consider simply not doing the observation reduction to a  basis state in the final step;
in place of this (reduced) output superposition we simply
output the non-reduced  quantum state superposition of the QC that exists just prior to the final  observation step.
This alternative approach can entirely eliminate the observation operation
 from many quantum computations, 
and so provides small volume, but has the drawback of providing a non-classic
output consisting of a non-reduced  quantum state superposition.
The potential difficulty with this approach is as follows:
if this (non-reduced quantum state superposition) output is then processed by a classical computing machine,
it may propagate unwanted quantum effects
to the classical computing machine.



%(Also see Ueda [Ued97] for a comparison between 
%strict mathematical (logical)
%reversibility and physical reversibility in quantum measurement.)


\noindent
\subsubsection{Approximate Observation Operations?}
% Still Appear to require Large Volume.} 
An approach to this difficulty is to only do
 the observation operation approximately within accuracy $\epsilon$; 
this may suffice for many QC applications.
However, even if the observation operation is done $\epsilon$-approximately by unitary operations, 
it appears to require a number of additional qubits $n'$ growing exponentially with 
the $n$, the original number of qubits of the QC.
In fact, we know of no upper bound on $n'$ better than $ 2^n\log(1/\epsilon)$. 


%Again, what are the volume bounds for the observation operation as a function
% of the number $n$ of qubits?
\noindent
\subsubsection{ Why the Volume Required by Observation Apperatus May Not Be Small.}
We next consider whether
it is
 reasonable to expect that a mathematical proof
(or such experimental demonstrations) of small volume quantum observation will ever be done.
We provide an informal argument 
(it should be emphasized that the following is not a formal proof in any sense)
that 
even an $\epsilon$-approximate observation can not be done in polynomial time using small volume,
where $\epsilon$ is the inverse of a polynomial.
Since for $n$ qubits, the size of the basis state space grows as $2^n$ in the general case,
 it seems reasonable to assume 
 (e.g., where the physics of the strong measurement is modeled by a diffusion
process [287,288,289]  that is rapidly mixing) that the likelihood of reversibility within polynomial time bounds 
 %vanishingly
  drops exponentially with the number $n$ of qubits.
Thus, 
in the context of polynomial time computations,
the $\epsilon$-approximate observation is assumed 
irreversible with high likelihood.

The argument will hinge on the assumption, made by conventional formulations of quantum computation, that quantum computation (including both unitary qubit operations as well as the non-unitary observation (or projection) operation) can be exectuted for any given number $n$ qubits, which makes an implicit assumption that both unitary and non-unitary operations can be executed at any scale.

 Let us also assume that the number of qubits $n$ is
small (at most a few hundreds). For sake of contradiction, 
let us for the moment suppose that (i)
 quantum computing scales to
 at least moderate size (say a few tens of thousands of qubits), 
and (ii) an $\epsilon$-approximate observation operation can be
 done on one of $n$ qubits 
by a
 microscopic measuring device of size $n' = n^c$,
 for a constant $c$,
%(say at most $2$), 
and operating within time polynomial in $n$. 
Since $n$ is small,
the measuring device is surely of
 %micromolecular
sufficiently small size so that it's physics is 
 consistent with established quantum physics
(for observe that if quantum computing is to scale to at least moderate size $n'$,
then surely quantum effects need to hold
for molecules of size $n'$).
% this is true assuming that quantum affects hold at micromolecular scales;
This implies we need to view 
the apparatus for the observation as executing polynomial time
 unitary quantum computation, which is reversible,
so the reverse of the observation also executes in quantum polynomial time. 
 Hence we have an apparent contradiction, since we have assumed the 
$\epsilon$-approximate observation is not 
reversible in polynomial time.
%The reader should note that this is not a formal proof, but
(Note. This
argument does not require that the governing physical laws shift at some definite
size from a  quantum-mechanical paradigm to a classical paradigm; instead
the argument  requires that if the quantum-mechanical paradigm is valid
 at size $n$ then it also is valid at some what larger size $n' = n^c$.)

Due to informal nature of this argument,
it only provides partial evidence that (with the above assumption), 
 QC with the observation operation does not scale to a large number of qubits
within small volumes, and in particular
 that a polynomial time $\epsilon$-approximate observation operation requires 
very large volume
%bounded less than quadratically with $n$,
and can not be done at the micromolecular scale 
for moderate large $n$.
%a manner consistent with established quantum physics.
We feel our above argument is far too informal to provide a resolution of the issue.
It remains a major open problem in QC to
{\it provide a  formal proof that either (i) there is large volume required for observation or (ii)  there is not.} 








 
 % 
% \noindent
% {\large {\bf References}}

%  A postscript document containing a bibliography of over  
% $385$ reference papers in quantum computation  is available online 
%on the web
%  at 
%  http://www.cs.duke.edu/$\sim$ reif/paper/altcomp.ps.

%\noindent
% [Rei99] Reif, J.H.,
%\vspace{5pt}\noindent
%[Rei99] J.H. Reif, 
%Quantum Information Processing: Compression, Coding, and Related
%Computations,  
%Full version of the
%extended abstract appearing in these proceedings.



%\vspace{15pt}
%\noindent
%Note: Due to page length constraints, the references in
% have been removed from the  extended abstract
% appearing in these proceedings.
%A postscript preprint of the full paper is online  at 
%http://www.cs.duke.edu/$\sim$ reif/paper/paper.html/qsurvey.ps.
%The full paper [Rei99]  has approximately   $385$
% references to QC. 

%\newpage

%______________________________________________________
\vspace{20pt}
\noindent 
 {\large {\bf References}}

{\small 

\noindent
{\bf  Note: } QCQC 98 is an acronym for:
Proc. of 1st NASA Workshop on Quantum
 Computation and Quantum Communication (QCQC 98), Springer-Verlag, (Feb. 1998).


% [Lan61]
\vspace{5pt}\noindent
[1] R. Landauer. Irreversability and heat generation in the computing process, IBM Journal of Research Development, 5(183), (1961).

%[Ben73]
\vspace{5pt}\noindent
[2] Bennett, C. H., Logical reversibility of computations. IBM Journal of Res. Develop., 17:525-532, (1973). 

 %[BL85]
\noindent
[3] Bennett, C.H., R. Landauer, Physical limits of computation, Scientific American, page 48, (July 1985). 
 


%[Lan96]
\vspace{5pt}\noindent
[4] Landauer, R., The physical nature of information. Phys. Lett. A 217, 188, (1996).


%[Tof80]
\vspace{5pt}\noindent
[5] T. Toffoli, in Automata, Languages and Programming, Eds. J. W. de Bakker and J. van Leeuwen (Springer-Verlag, New York, p. 632 (1980). 

%[LV96] 
\vspace{5pt}\noindent
[6] M. Li, P. Vitanyi, Reversibility and Adiabatic Computation: Trading Time and Space for Energy, (Online preprint quant-ph/9703022), Proc. Royal Society of London, Series A, 452(1996), 769-789.


%[Ben89]
\vspace{5pt}\noindent
[7] Bennett, C. H., Time/space trade-offs for reversible computation, SIAM J. Comput. 18, 766 (1989). 


%[BBC+95]
\vspace{5pt}\noindent
[8] Barenco, A., C. H. Bennett, R. Cleve, D. P. D. P. DiVincenzo, Phys. Rev. A 51, 1015 (1995). 

%[BD95a] 
\noindent
[9] Bennett, and D. P. DiVincenzo, Progress Towards Quantum Computation, Nature, (October 1995).
 

%[BD95b]
\noindent
[10] Bennett, C.H., D. P. DiVincenzo, Quantum Computing: Towards an Engineering Era?, Nature, Vol. 377, (1995).


%[Bar96] 
\vspace{5pt}\noindent
[11] A. Barenco, Quantum Physics and Computers, (Online preprint quant-ph/9612014), Contemp.Phys. 37 (1996) 375. 

%[Ben96] 
\vspace{5pt}\noindent
[12] P. Benioff, Quantum Ballistic Evolution in Quantum Mechanics: Application to Quantum Computers, (Online preprint quant-ph/9605022), to be published in Phys. Rev. A (1996). 

%Bra96 
\noindent
[13] Brassard, G., New Trends in Quantum Computing, (Online preprint quant-ph/9602014), 13th Symposium on Theoretical Aspects of Computer Science, Grenoble, Lecture Notes in Computer Science, Springer-Verlag, (Feb. 1996).

%\bibitem[B 98]{Bra98}
\vspace{5pt}\noindent
[14] Brassard, G., New horizons in quantum information processing,
Proceedings of the 25th Colloquim on Automata, Languages, and Programming, 
Aalborg, Denmark, (May 1998).


%[HR96] 
\vspace{5pt}\noindent
[15] Haroche, S. and Raimond, J. M. Quantum computing: dream or nightmare? Phys. Today 49 (8), 51, (1996)

%[Bra97] 
\vspace{5pt}\noindent
[16] G. Brassard, Quantum information processing: The good, the bad and the ugly. In Burton, S. Kaliski Jr., editor, Advances in Cryptology -- CRYPTO '97, volume 1294 of Lecture Notes in Computer Science, pages 337-341, (August 1997).

%[Pre97a] 
\vspace{5pt}\noindent
[17] J. Preskill, Quantum Computing: Pro and Con, (Online preprint quant-ph/9705032), submitted to Proc. Roy. Soc. Lond. A (1997).

%[Sca98]
\vspace{5pt}\noindent
[18] V. Scarani, Quantum Computing, (Online preprint quant-ph/9804044), Accepted for publ. in American Journal of Physics, (1998).

%[Ste98] 
\vspace{5pt}\noindent
[19] A. Steane, Quantum Computing, (Online preprint quant-ph/9708022), Reports on Progress in Physics (1998).

%[VP98] 
\vspace{5pt}\noindent
[20] V. Vedral, Martin B. Plenio, Basics of Quantum Computation, (Online preprint quant-ph/9802065), invited basic review article for Progress in Quantum Electronics, (1998).


%[Tau96] 
\vspace{5pt}\noindent
[21] Taubes, All Together for Quantum Computing, Science, Vol. 273, (1996).

%[GC98] 
\vspace{5pt}\noindent
[22] N. Gershenfeld and I. L. Chuang
Quantum Computing with Molecules, Scientific American, 278(6), pp. 66-71, (June 1998).


%[WC97]
\vspace{5pt}\noindent
[23] C.P. Williams and S.H. Clearwater, Explorations in Quantum Computing, Springer-Verlag, Yew York, (1997).


%[BC02]
\vspace{5pt}\noindent
[24] R. Brylinski and G. Chen, Mathematics of Quantum Computation, Computational Mathematics Series, CRC Press Inc  Feb 2002).

%[Gr99]
\vspace{5pt}\noindent
[25] J. Gruska, Quantum Computing, Advanced topics in computer science series, Osborne/McGraw-Hill,U.S. (Mar 1999).


%[Hi03] 
\vspace{5pt}\noindent
[26] M. Hirvensalo,  Quantum Computing (Natural Computing Series), Springer-Verlag Berlin and Heidelberg GmbH \& Co. (Dec 2003).	

%[BL05] 
\vspace{5pt}\noindent
[27] T. Beth and Gerd Leuchs, Quantum Information Processing, Wiley VC, 2nd edition (Mar 2005).


%[Bo99]
\vspace{5pt}\noindent
[28] M. Brooks, Quantum Computing and Communications, Springer-Verlag London Ltd (May 1999).

%[IB05] 
\vspace{5pt}\noindent
[29] S. Imre and F. Balazs, Quantum Computing for Communications, John Wiley and Sons Ltd (Jan 2005).

%[Le03]
\vspace{5pt}\noindent
[30] G. Leuchs, Quantum Information Technology, Wiley-VCH (Mar 2003).



%[MPZ01] 
\vspace{5pt}\noindent
[31] C. Macchiavello, G.M. Palma, and A. Zeilinger , Quantum Computation and Quantum Information Theory, World Scientific Publishing (Jan 2001).

%[NC00] 
\vspace{5pt}\noindent
[32] M.l A. Nielsen and I. L. Chuang, Quantum Computation and Quantum Information, Cambridge Series on Information \& the Natural Sciences, Cambridge Press, (Oct 2000).

%[SW07]
\vspace{5pt}\noindent
[33] W. P. Schleich and H. Walther , Elements of Quantum Information, Wiley-VCH (Jan 2007).

%[Ve06] 
\vspace{5pt}\noindent
[34] V. Vedral , Introduction to Quantum Information Science, Oxford Graduate Texts Series,  Oxford University Press (Sep 2006).


%[As06]  
\vspace{5pt}\noindent
[35]  G. Van Assche,  Quantum Cryptography and Secret-Key Distillation, Cambridge University Press (July, 2006).

%[BE+07] 
\vspace{5pt}\noindent
[36] D. Bouwmeester, A. K. Ekert, and A. Zeilinger  The Physics of Quantum Information: Quantum Cryptography, Quantum Teleportation, Quantum Computation, Springer Verlag, Berlin, German,  (Jan , 2007),

%[Se05] 
\vspace{5pt}\noindent
[37] A.V. Sergienko, Quantum Communications and Cryptography  by  American Physical Society, Mach Zehnder, CRC(Nov, 2005).


%[EH+05] 
\vspace{5pt}\noindent
[38] D. Evans, J. J. Holt, C. Jones, and K. Klintworth, Coding Theory and Quantum Computing, Contemporary Mathematics Series, American Mathematical Society (Sep 2005).

%[Fr08] 
\vspace{5pt}\noindent
[39] G. Frank, Quantum Error Correction and Fault Tolerant Quantum Computing, CRC Press Inc (Mar 2008).

%[Pi99] 
\vspace{5pt}\noindent
[40] Arthur O. Pittenger, An Introduction to Quantum Computing Algorithms, Progress in Computer Science and Applied Logic (PCS),  Birkhauser, (Nov, 1999).


%[CC06] 
\vspace{5pt}\noindent
[41] G. Chen, D. A. Church, B. Englert, and C. Henkel, Quantum Computing Devices: Principles, Designs, and Analysis, Chapman \& Hall, CRC Applied Mathematics \& Nonlinear Science, (Sep 2006).

%[Cl02]
\vspace{5pt}\noindent
[42] R. Clark (Editor), Experimental Implementation of Quantum Computation, :IOS Press,US (Dec 2002).


%[MC06] 
\vspace{5pt}\noindent
[43] T. Metodi and F. Chong, Quantum Computing for Computer Architects, Synthesis Lectures on Computer Architecture, Morgan \& Claypool Publishers (Nov 2006).


%[KS02] 
\vspace{5pt}\noindent
[44] B. E. Kane and T. W. Sigmon, Quantum Dot Devices and Computing, SPIE Society of Photo-Optical Instrumentation Engi (April 2002).





%[LR+03] 
\vspace{5pt}\noindent
[45] A. Leggett, B. Ruggiero, and P. Silvestrini, Quantum Computing and Quantum Bits in Mesoscopic Systems, Kluwer Academic / Plenum Publishers (Dec 2003).


%[Fey82] 
\vspace{5pt}\noindent
[46] R. P. Feynman. Simulating physics with computers. International Journal of Theoretical Physics, 21(6/7): pp. 467-488, (1982).

%[Fey86] 
\vspace{5pt}\noindent
[47] R. P. Feynman. Quantum mechanical computers. Foundation of Physics, 16(6):507-531, (1986).


%[Ben82] 
\vspace{5pt}\noindent
[48] Benioff, P. Quantum mechanical models of Turing machines that dissipate no energy. Phys. Rev. Lett. 48, 1581, (1982).

%[DJ92]
\vspace{5pt}\noindent
[49] D. Deutsch and R. Jozsa. Rapid solutions of problems by quantum computation. In Proceedings of the Royal Society, London, volume A439, pages 553-558, (1992).


%[BV93]
\vspace{5pt}\noindent
[50] E. Burnstein and U. Vazirani, Quantum Complexity Theory, Proc. 25th Annual ACM Symposium on Theory of Computing, Association for Computing Machinery, New York, NY, 1993, pp. 11�20. Published in SIAM Journal on Computing, 26(5):1411-1473, (October 1997).

%[CS97] 
\vspace{5pt}\noindent
[51] G. Costantini, F. Smeraldi, A Generalization of Deutsch's Example, (Online preprint quant-ph/9702020), (1997).


%[CKH98] 
\vspace{5pt}\noindent
[52] D. Collins, K. W. Kim, W. C. Holton, Deutsch-Jozsa algorithm as a test of quantum computation, (Online preprint quant-ph/9807012), Approved for publication in Phys Rev A, (1998).

%[Joz96] 
\vspace{5pt}\noindent
[53] Jozsa, Proc. R. Soc. Lond. A 435, 563 (1996). 



%[Joz97] 
\vspace{5pt}\noindent
[54] R. Jozsa, Entanglement and Quantum Computation, (Online preprint quant-ph/9707034), Geometric Issues in the Foundations of Science, ed. S. Huggett et. al., (1997).

%[Joz98] 
\vspace{5pt}\noindent
[55] R. Jozsa, Quantum Effects in Algorithms, (Online preprint quant-ph/9805086), QCQC 98, (Feb. 1998).


%[Deu85b]
\vspace{5pt}\noindent
[56] D. Deutsch. Quantum theory, the Church-Turing principle and the universal quantum computer. Proceedings of the Royal Society, London, A400:97-117, (1985).

%[MC97] 
\vspace{5pt}\noindent
[57] C. Moore, J. P. Crutchfield, Quantum Automata and Quantum Grammars, (Online preprint quant-ph/9707031), (1997).

%[KW97] 
\vspace{5pt}\noindent
[58] A. Kondacs and J. Watrous. On the power of quantum finite state automata. In 38th Annual Symposium on Foundations of Computer Science, pages 66-75, Miami Beach, Florida, (October 1997). IEEE.


%[Dun98]
\vspace{5pt}\noindent
[59] M. R. Dunlavey, Simulation of finite state machines in a quantum computer, (Online preprint quant-ph/9807026), (1998).

%[Wat95] 
\vspace{5pt}\noindent
[60] J. Watrous. On one-dimensional quantum cellular automata. In 36th Annual Symposium on Foundations of Computer Science, pages 528-537, Milwaukee, Wisconsin, (October 1995). IEEE.



%[DTS96] 
\vspace{5pt}\noindent
[61] C. 
%D�rr
D\"{u}rr and H. L� Thanh and M. Santha. A decision procedure for well-formed linear quantum cellular automata. In 13th Annual Symposium on Theoretical Aspects of Computer Science, volume 1046 of Lecture Notes in Computer Science, pages 281-292, Grenoble, France, 22-24 (February 1996). Springer.

%[DS96] 
\vspace{5pt}\noindent
[62] C. D\"{u}rr , M. Santha, A decision procedure for unitary linear quantum cellular automata, (Online preprint quant-ph/9604007), In 37th Annual Symposium on Foundations of Computer Science, pages 38-45, Burlington, Vermont, (October 1996). IEEE.

%[Bar95]
\vspace{5pt}\noindent
[63] A. Barenco, A Universal Two--Bit Gate for Quantum Computation, (Online preprint quant-ph/9505016), (1995).

%[DMS+95] 
\vspace{5pt}\noindent
[64] D. P. DiVincenzo, N. Margolus, P. Shor, T.
Sleator, J. Smolin and H. Weinfurter, Elementary gates for quantum computation, submitted to Phys. Rev. A (1995).

%[BBC+95]
\vspace{5pt}\noindent
[65] Barenco, A., C. H. Bennett, R. Cleve, D.P. DiVincenzo, N. Margolus, P. Shor, T. Sleator, J. Smolin, and H. Weinfurter. Elementary gates for quantum computation. (Online preprint quant-ph/9503016), Phys. Rev. A. 52, 3457 (1995). 



%[DiV95]
\vspace{5pt}\noindent
[66] D. P. DiVincenzo. Two-bit gates are universal for quantum computation. Physical Review Letters A, 50(1015), (1995). 

%[SD95] 
\vspace{5pt}\noindent
[67] J. A. Smolin and D. P. DiVincenzo, Five Two-Bit Quantum Gates are Sufficient to Implement the Quantum Fredkin Gate, Phys. Rev. A. 53, 2855 (1995).


%[DiV96]
\vspace{5pt}\noindent
[68] D. P. DiVincenzo, Quantum Gates and Circuits, (Online preprint quant-ph/9705009), Proceedings of the ITP Conference on Quantum Coherence and Decoherence, (December, 1996), submitted Proc. R. Soc. London A.


%[DS98] 
\vspace{5pt}\noindent
[69] D. P. DiVincenzo and J. Smolin, Results on Two-bit gate design for quantum computers, (1998).


%[PCZ96]
\vspace{5pt}\noindent
[70] J. F. Poyatos, J. I. Cirac, P. Zoller, Complete Characterization of a Quantum Process: the Two-Bit Quantum Gate, (Online preprint quant-ph/9611013), Physical Review Letters 08, (Nov 1996).

%[MPH96a] 
\vspace{5pt}\noindent
[71] D. Mozyrsky, V. Privman, S. P. Hotaling, Extended Quantum XOR Gate in Terms of Two-Spin Interactions, (Online preprint quant-ph/9610008), (1996).


%[MPH97] 
\vspace{5pt}\noindent
[72] D. Mozyrsky, V. Privman, S. P. Hotaling, International Journal of Modern Physics B 11, 2207-2215 (1997).

%[MPH98] 
\vspace{5pt}\noindent
[73] D. Mozyrsky, V. Privman, S. P. Hotaling, Design of gates for quantum computation: the three-spin XOR gate in terms of two-spin interactions, (Online preprint uant-ph/9612029), International Journal of Modern Physics B 12 (1998) 591-600.


%[Llo97c] 
\vspace{5pt}\noindent
[74] S. Lloyd, Almost any quantum logic gate is universal, Los Alamos National Laboratory preprint (1997c).


%[MMK95] 
\vspace{5pt}\noindent
[75] Monroe, C., Meekhof, D. M., King, B. E., Itano, W. M. and Wineland, D. J., Demonstration of a fundamental quantum logic gate, Phys. Rev. Lett. 75, 4714, (1995).

%[DVL98]
\vspace{5pt}\noindent
[76] D. P. DiVincenz, D. Loss, Quantum Information is Physical, (Online preprint cond-mat/9710259), to be published in Superlattices and Microstructures, Special Issue on the Occasion of Rolf Landauer's 70th Birthday. (1998).


%[Deu89]
\vspace{5pt}\noindent
[77] D. Deutsch. Quantum computational network. Proceedings of the Royal Society, London, A425:73-90, (1989).

%[Yao93] 
\vspace{5pt}\noindent
[78] A. C.-C. Yao. Quantum circuit complexity, In Proceedings of the 34th IEEE Symposium on Foundations of Computer Science, pages 352-361, Palo Alto, California, (Nov. 1993).
 
 
\noindent
[79] D. Aharonov, A. Kitaev, N. Nisan, (Online preprint quantum Circuits with Mixed States, Proceedings of the Thirtieth Annual ACM Symposium on Theory of Computation (STOC), pages 20-30, 1997 (Online preprint quant-ph/9806029), (1998).
 
%[AD+04] 
\vspace{5pt}\noindent
[80] D. Aharonov, W. van Dam, J. Kempe, Z. Landau, S. Lloyd, and O. Regev. Adiabatic quantum computation is equivalent to standard quantum computation. In Proc. 45th Annual IEEE Symp. on Foundations of Computer Science (FOCS), pages 42�51, 2004. 

%[OD97] 
\vspace{5pt}\noindent
[81] K. M. Obenland, A. M. Despain, Models to Reduce the Complexity of Simulating a Quantum Computer, (Online preprint quant-ph/9712004), (1997).

%[OD98a] 
\vspace{5pt}\noindent
[82] K. M. Obenland, A. M. Despain, Simulating the Effect of Decoherence and Inaccuracies on a Quantum Computer, (Online preprint quant-ph/9804038), QCQC 98, (Feb. 1998).

%[OD98b]
\vspace{5pt}\noindent
[83] K. M. Obenland, A. M. Despain, A Parallel Quantum Computer Simulator, (Online preprint quant-ph/9804039), High Performance Computing, (1998).


%[CK98]
\vspace{5pt}\noindent
[84] N. J. Cerf, S. E. Koonin, Monte Carlo Simulation of Quantum Computation, (Online preprint quant-ph/9703050), Math. and Comp. in Simulation 47 (1998), 143-152.

%[BB92a] 
\noindent
[85] Berthiaume, A. and G. Brassard, The quantum challenge to structural complexity. In Proceedings of the 7th Annual IEEE Conference on Structure in Complexity, pages 132-137, 1992. 


%[Ber95] 
\vspace{5pt}\noindent
[86] A. Berthiaume. L'ordinateur quantique: complexit'e et stabilisation des calculs. PhD thesis, Dept. d'informatique et de recherche operationelle, Universite de Montreal, (1995). 


%[BB92b] 
\vspace{5pt}\noindent
[87] Berthiaume, A. and G. Brassard. Oracle quantum computing, In Proceedings of the Workshop on Physics and Computation - Physcomp '92, pages 195-199. IEEE Press, (October 1992).

%[BB94] 
\vspace{5pt}\noindent
[88] Berthiaume, A. and G. Brassard. Oracle quantum computing, Journal of Modern Optics, 41(12):2521-2535, (1994).


%[MAC98]
\vspace{5pt}\noindent
[89] J. Machta, Phase Information in Quantum Oracle Computing, (Online preprint quant-ph/9805022), (1998).

%[Dam98a]
\vspace{5pt}\noindent
[90] W. van Dam, Two Classical Queries versus One Quantum Query, (Online preprint quant-ph/9806090), (1998).

%[Dam98b] 
\vspace{5pt}\noindent
[91] W. van Dam, (Online preprint quantum Oracle Interrogation: Getting all information for almost half the price, U of Oxford, CWI, (Online preprint quant-ph/9805006), (1998).

%[ADH97] 
\vspace{5pt}\noindent
[92] L. M. Adleman and J. Demarrais and M.-D. A. Huang. Quantum computability. SIAM Journal on Computing, 26(5):1524-1540, (October 1997). 


%[MN98a] 
\vspace{5pt}\noindent
[93] C. Moore, M. Nilsson, Parallel Quantum Computation and Quantum Codes, (Online preprint quant-ph/9808027), (1998).



%[CDN+98] 
\noindent
[94] R. Cleve, W. van Dam, M. Nielsen, A. Tapp, Quantum Entanglement and the Communication Complexity of the Inner Product Function, (Online preprint quant-ph/970801), QCQC 98, (Feb. 1998).
 
%[KL98] 
\vspace{5pt}\noindent
[95] E. Knill, R. Laflamme, On the Power of One Bit of Quantum Information, (Online preprint quant-ph/9802037), (1998).


%[KWHZK 95] 
\vspace{5pt}\noindent
[96] P. G. Kwiat, H. Weinfurter, T. Herzog, A. Zeilinger, and M. A. Kasevich, Interaction-free Measurement, Phys. Rev. Lett., 74, pp 4763-4766, (1995).

%[KWZ 96]
\vspace{5pt}\noindent
[97] P. G. Kwiat, H. Weinfurter, and A. Zeilinger, 
Quantum Seeing in the Dark, Scientific American, (November, 1996).

%[Rei00] 
\vspace{5pt}\noindent
[98] J.H. Reif, On the Impossibility of Interaction-Free Quantum Sensing for Small I/O Bandwidth,  Information and Computation, Jan 2000, pp. 1-20.
(Online preprint at http://www.cs.duke.edu/$\sim$reif/paper/qsense/qsense.pdf)

%\bibitem[H 73]{H73} 
\vspace{5pt}\noindent
[99] A.S. Holevo,
Some estimates of the information transmitted by quantum communication channels, 
Problemy Peredachi Informatsii, Vol.  9,
pp 3--11, (1973).  English translation in Problems of Information Transmission, 
(USSR),  9, pp. 177--183, (1973).
%


%[FC94] \bibitem[FC 94]{FC94} 
\vspace{5pt}\noindent
[100] C. Fuchs and C. Caves,
Ensemble-dependent bounds for accessible information in quantum mechanics, 
 Physics Rev. Lett. Vol. 73,
pp 3047--3050, (1994).


%[Sch95] 
\vspace{5pt}\noindent
[101] B. Schumacher, On quantum coding, Physical Review Letters A 51, 2738 (1995).



%[Neu32]
\vspace{5pt}\noindent
[102] J. von Neumann, Mathematical Foundations of Quantum Mechanics, 
 Springer Verlag, (1932). Reprinted in Princeton Landmarks in Mathematics, Princeton University Press, (1996).


%[JS94]
\vspace{5pt}\noindent
[103] R. Jozsa and B. Schumacher, {\it A new proof of the
quantum noiseless coding theorem}, J. Mod. Optics 41, 2343-2349 (1994).

%[S48] 
\vspace{5pt}\noindent
[104] K. Y. Szeto, {\it  Data Compression of Quantum Code}, preprint
quant-ph/9607010 (July 1996).

%[B95]
\vspace{5pt}\noindent
[105]
C. H. Bennett, {\it Quantum information and computation}, 
Physics Today,
pp. 24-30, (October 1995).

%[BBJ+94] 
\vspace{5pt}\noindent
[106] Bennett, C. H., G. Brassard, Jozsa, Mayers, Peres, Schumacher, and Wootters, 
Reduction of Quantum Entropy by Reversible Extraction of Classical Information,
 in Journal of Modern Optics, (1994).
 
%[CD96] 
\vspace{5pt}\noindent
[107] R. Cleve, D. P. DiVincenzo, Schumacher's quantum data compression as a quantum computation, (Online preprint quant-ph/9603009), (1996).

%[Rei07]
[108]
John H. Reif and Sukhendu Chakraborty, Efficient and Exact Quantum Compression, Journal of Information and Computation, Vol. 205, pp 967-981 (2007).


%[JHH+98]
\vspace{5pt}\noindent
[109] R. Jozsa, M. Horodecki, P. Horodecki,  R. Horodecki, {\it Universal Quantum Data
Compression}, preprint quant-ph/9805017 (May 1998).


%[BFG+95]
\vspace{5pt}\noindent
[110]
S. L. Braustein, C. A. Fuchs, D. Gottesman, H-K. Lo, {\it A quantum analog of Huffman Coding}, 
report no. quant-ph/9805080 (May 1998).

%[Nie96] 
\vspace{5pt}\noindent
[111] M. A. Nielsen, The entanglement fidelity and quantum error correction, (Online preprint quant-ph/9606012), (1996).

%[Svo95]
\vspace{5pt}\noindent
[112] K. Svozil, Quantum algorithmic information theory, (Online preprint quant-ph/9510005), lectures given at the summer school, Chaitin Complexity and Applications, Mangalia, Mangalia, Romania (June 1995).


%[Svo96]
\vspace{5pt}\noindent
[113] K. Svozil. Quantum information theory. The Journal of Universal Computer Science, 2(5):311-346, (May 1996). 


%[Hol97] 
\vspace{5pt}\noindent
[114] A.S.Holevo, Coding Theorems for Quantum Communication Channels, Steklov Mathematical Institute, (Online preprint quant-ph/9708046), (1997). 


%[KL96a]
\vspace{5pt}\noindent
[115] E. Knill, R. Laflamme, Concatenated Quantum Codes, (Online preprint quant-ph/9608012), (1996).

%[KL96b] 
\vspace{5pt}\noindent
[116] E. Knill, R. Laflamme, A Theory of Quantum Error-Correcting Codes, (Online preprint quant-ph/960403), (1996).



%[Ohy98] 
\vspace{5pt}\noindent
[117] M. Ohya, A mathematical foundation of quantum information and quantum computer -on quantum mutual entropy and Entanglement, (Online preprint quant-ph/9808051), (1998).

%[BCW98] 
\vspace{5pt}\noindent
[118] Buhrman, H., R. Cleve, and A. Wigderson, Quantum vs. Classical Communication and Computation, (Online preprint quant-ph/9802040), (1998).


%[AC98b] 
\vspace{5pt}\noindent
[119] C. Adami, N.J. Cerf, What information theory can tell us about quantum reality, (Online preprint quant-ph/9806047), QCQC 98, (Feb. 1998).

%[DSS+95]
\vspace{5pt}\noindent
[120] D. P. DiVincenzo, Peter W. Shor, and John A. Smolin, Quantum channel capacity of very noisy channels, (Online preprint quant-ph/9706061), to appear 1998.


%[Hol96] 
\vspace{5pt}\noindent
[121] Holevo, A. S., The capacity of quantum channel with general signal states. (Online preprint quant-ph/9611023), (1996).


%[BNS+97] 
\vspace{5pt}\noindent
[122] Barnum, H., M. A. Nielsen, and B. Schumacher, Information transmission through a noisy quantum channel. (Online preprint quant-ph/9702049), (1997). 


%[SS98] 
\vspace{5pt}\noindent
[123] A.Shnirman, G. Schoen, Quantum Measurements Performed with a Single-Electron Transistor, (Online preprint cond-mat/9801125), submitted to Phys. Rev. B (1998).


%[BDS97b] 
\vspace{5pt}\noindent
[124] Bennett, C. H., D. P. DiVincenzo and J. A. Smolin, Capacities of quantum erasure channels. (Online preprint quant-ph/9701015), (1997). 


%[Fuc97] 
\vspace{5pt}\noindent
[125]  Fuchs, C., Nonorthogonal quantum states maximize classical information capacity. (Online preprint quant-ph/9703043.), (1997).

%[H97] 
\vspace{5pt}\noindent
[126] C. W. Helstrom. Detection theory and quantum mechanics. Information and Control, 10(3):254-291, (March 1967).
 
%[H98]
 \noindent
[127] C. W. Helstrom. Detection theory and quantum mechanics (II). Information and Control, 13(2):156-171, (August 1968).


%[Fuc96]
\vspace{5pt}\noindent
[128] C. A. Fuchs, Distinguishability and Accessible Information in Quantum Theory, (Online preprint quant-ph/9601020), (1996).

%[Sho95] 
\vspace{5pt}\noindent
[129] P. Shor, Scheme for reducing decoherence in quantum memory. Phys. Rev. A 52, 2493�2496, (1995).

%[Ste96a] 
\vspace{5pt}\noindent
[130] Steane, A. M., Error correcting codes in quantum theory. Phys. Rev. Lett. 77, 793, (1996). 

%[CS95] 
\noindent
[131] Calderbank, A. R., and P. W. Shor, Good quantum error-correcting codes exist, (Online preprint quant-ph/9512032), (December 1995), Phys. Rev. A 54, 1098. 




%[Ste96b]
\vspace{5pt}\noindent
[132] A. Steane, Multiple particle interference and quantum error correction, (Online preprint quant-ph/9601029, Proceedings of the Royal Society of London Ser. A, 452, 2551, (1996).

\noindent
[133] Bennett, C., D. DiVincenzo, J. Smolin and Wootters, W. Mixed state entanglement and quantum error correction. (Online preprint quant-ph/9604024), Phys. Rev. A 54, 3824, (1996).


%[LMP+96]
\vspace{5pt}\noindent
[134] R. Laflamme, C. Miquel, J. P. Paz, and W. H. Zurek, Perfect quantum error correction code, (Online preprint quant-ph/9602019), Phys.Rev. Lett., 77, 198, (1996).



%[Sho96] 
\vspace{5pt}\noindent
\vspace{5pt}\noindent
[135] P. W. Shor, Fault-tolerant quantum computation, (Online preprint quant-ph/9605011), 37th Symposium on Foundations of Computing, Los Alamitos, CA, IEEE Computer Society Press, pp. 56-65 (1996).

%[KY96]
\vspace{5pt}\noindent
[136] Kitaev, A. Yu., Quantum computing: algorithms and error correction, preprint (in Russian), (1996).

%[Kit97] 
\vspace{5pt}\noindent
[137] A. Y. Kitaev, Fault-tolerant quantum computation by anyons, (Online preprint quant-ph/9707021), (1997).



%[GEK+96]
\vspace{5pt}\noindent
[138] Gottesman, D., Evslin, J., Kakade, S. and Preskill, J., (1996). 

%[AB97] 
\noindent
[139] D. Aharonov, M. Ben-Or, Fault Tolerant Quantum Computation with Constant Error, (Online preprint quant-ph/9611025), In Proceedings of the Twenty-Ninth Annual ACM Symposium on Theory of Computing, pages 176-188, El Paso, Texas, (May 1997).

%[KLZ97] 
\vspace{5pt}\noindent
[140] E. Knill, R. Laflamme, W. Zurek, Threshold Accuracy for Quantum Computation, (Online preprint quant-ph/9610011), (1996).

%[KLZ97] 
\vspace{5pt}\noindent
[141] E. Knill, R. Laflamme, W. H. Zurek, Resilient Quantum Computation: Error Models and Thresholds, (Online preprint quant-ph/9702058), (1997). 

%[BBB+82] 
\vspace{5pt}\noindent
[142] C. H. Bennett, G. Brassard, S. Breidbart, S. Wiesner. Quantum cryptography, or unforgeable subway tokens. In D. Chaum and R. L. Rivest and A. T. Sherman, editors, Advances in Cryptology: Proceedings of Crypto 82, pages 267-275, (August 1982). Plenum Press, New York and London, 1983. 

\vspace{5pt}\noindent
[143]   Charles H. Bennett and Giles Brassard, �Quantum cryptography: public key distribution and coin tossing,� in Proceedings of the IEEE International Conference on Computers, Systems, and Signal Processing, Bangalore, India (IEEE, New York, 1984),pp. 175�179. 

%[BBE92] 
\noindent
[144] Bennett, C. H., G. Brassard, and A Ekert. Quantum cryptography. Scientific American, pages 50-57, (October 1992).

%[Bra93]
\vspace{5pt}\noindent
[145] Brassard, G., Cryptology column | quantum cryptography: A bibliography. Sigact News, 24(3):16-20, (1993).

%[BB84b] 
\vspace{5pt}\noindent
[146] C. H. Bennett and G. Brassard. An update on quantum cryptography. In G. R. Blakley and D. Chaum, editors, Advances in Cryptology: Proceedings of CRYPTO 84, volume 196 of Lecture Notes in Computer Science, pages 475-480, (August 1984). Springer-Verlag, 1995.

%[Bra94]
\vspace{5pt}\noindent
[147] Brassard, G., Cryptology Column -- Quantum Computing: The End of Classical Cryptography?, SIGACTN: SIGACT News (ACM Special Interest Group on Automata and Computability Theory), Vol. 25, (1994). 

\vspace{5pt}\noindent
[148] N. Gisin, G. Ribordy, W. Tittel, and H. Zbinden, Quantum cryptography, Rev. Mod. Phys. 74, 145 (2002).


%[Ozh97a] 
\vspace{5pt}\noindent
[149] Y. Ozhigov, Protection of information in quantum qatabases, (Online preprint quant-ph/9712016), (1997).

%[Hru94] 
\vspace{5pt}\noindent
[148] J. Hruby. Q-deformed quantum cryptography. In Alfredo De Santis, editor, Advances in Cryptology -- EUROCRYPT 94, volume 950 of Lecture Notes in Computer Science, pages 468-472. Springer-Verlag, 1995, (May 1994).

%[BBB+92] 
\noindent
[150] C. H. Bennett, F. Bessette, G. Brassard, L. Salvail, and J. Smolin. Experimental quantum cryptography. Journal of Cryptology, 5(1):3-28, 1992.


%[HLM+96] 
\vspace{5pt}\noindent
[151] R. J. Hughes and G. G. Luther and G. L. Morgan and C. G. Peterson and C. Simmons. Quantum cryptography over underground optical fibers. In Neal Koblitz, editor, Advances in Cryptology -- CRYPTO '96, volume 1109 of Lecture Notes in Computer Science, pages 329-342, (August 1996). Springer-Verlag.

%[BBC+91] 
\vspace{5pt}\noindent
[152] C. H. Bennett, G. Brassard, C. Cr�peau and M.-H. Skubiszewska. Practical quantum oblivious transfer. In J. Feigenbaum, editor, Advances in Cryptology --CRYPTO '91, volume 576 of Lecture Notes in Computer Science, pages 351-366, (August 1991). Springer-Verlag, 1992. 


%[May95] 
\vspace{5pt}\noindent
[153] D. Mayers. On the security of the quantum oblivious transfer and key distribution protocols. In Don Coppersmith, editor, Advances in Cryptology -- CRYPTO '95, volume 963 of Lecture Notes in Computer Science, pages 124-135, (August 1995). Springer-Verlag. 

%[May96] 
\vspace{5pt}\noindent
[154] D. Mayers. Quantum key distribution and string oblivious transfer in noisy channels. In Neal Koblitz, editor, Advances in Cryptology -- CRYPTO '96, volume 1109 of Lecture Notes in Computer Science, pages 343-357, (August 1996). Springer-Verlag. 

%[LC98] 
\vspace{5pt}\noindent
[155] H.-K. Lo, H. F. Chau, Quantum Computers Render Quantum Key Distribution Unconditionally Secure Over Arbitrarily Long Distance, (Online preprint quant-ph/9803006), (1998). 

%[BC90] 
\vspace{5pt}\noindent
[156] G. Brassard, C. Cr�peau. Quantum bit commitment and coin tossing protocols. In A. J. Menezes and S. A. Vanstone, editors, Advances in Cryptology -- CRYPTO '90, volume 537 of Lecture Notes in Computer Science, pages 49-61, (August 1990). Springer-Verlag, 1991. 

%[BCJ93] 
\vspace{5pt}\noindent
[157] G. Brassard, C. Cr�peau, R. Jozsa,D. Langlois. A quantum bit commitment scheme provably unbreakable by both parties. In 34th Annual Symposium on Foundations of Computer Science, pages 362-371, Palo Alto, California, (November 1993).


%[Yao95] 
\noindent
[158] A. C.-C. Yao. Q Security of quantum protocols against coherent measurements. In Proceedings of the Twenty-Seventh Annual ACM Symposium on the Theory of Computing, pages 67-75, Las Vegas, Nevada, (May-June 1995).

%[BCM+98]
\vspace{5pt}\noindent
[159] G. Brassard, C. Cr�peau, D. Mayers, L. Salvail, Defeating classical bit commitments with a quantum computer, (Online preprint quant-ph/9806031), (1998). 


%[CL98]
\vspace{5pt}\noindent
[160] H. F. Chau, H.-K. Lo, An Empty Promise With A Quantum Computer, (Online preprint quant-ph/9709053), Fortsch.Phys. 46 (1998) 507-520.


%[CS95] 
\vspace{5pt}\noindent
[161] C. Cr�peau, L.Salvail. Quantum oblivious mutual identification. In L. C. Guillou and J.-J. Quisquater, editors, Advances in Cryptology -- EUROCRYPT 95, volume 921 of Lecture Notes in Computer Science, pages 133-146. Springer-Verlag, (May 1995).

\vspace{5pt}\noindent
[162]  Howard E. E. Brandt, Quantum-cryptographic entangling probe, Phys. Rev. A 71, 042312(14)(2005)


\vspace{5pt}\noindent
[163]   Taehyun Kim, Ingo Stork genannt Wersborg, Franco N. C. Wong, and Jeffrey H. Shapiro, Complete physical simulation of the entangling-probe attack on the Bennett-Brassard 1984 protocol, Phys. Rev. A 75, 042327 (2007). 


%[LC98] 
\vspace{5pt}\noindent
[164] H.K. Lo, H. F. Chau, Why Quantum Bit Commitment And Ideal Quantum Coin Tossing Are Impossible, (Online preprint quant-ph/9711065), Accepted for publication in a special issue of Physica D, 177-187, (1998).





%[Pel97] 
\vspace{5pt}\noindent
[165] T. Pellizzari, Quantum Networking with Optical Fibres, (Online preprint quant-ph/9707001), submitted to PRL (1997).


%[CZ97]
\vspace{5pt}\noindent
[166] Cirac, J. I., Zoller, P., Kimble, H. J. and Mabuchi, H., Quantum state transfer and entanglement distribution among distant nodes in a quantum network. Phys. Rev. Lett. 78, 3221, (1997).

%[BVK97] 
\vspace{5pt}\noindent
[167] S.Bose, V.Vedral, P.L.Knight, A Multiparticle Generalization of Entanglement Swapping, (Online preprint quant-ph/9708004), (1997).


%[BBC93]
\noindent
[168] Bennett, C. H., Brassard, Crepeau, Jozsa, Peres and Wootters, 
Teleporting an Unknown Quantum State via Dual Classical and Einstein-Podolsky-Rosen Channels, PRL 70, 1895 (1993).

%[BBP+96]
\vspace{5pt}\noindent
[169] Bennett, C. H., Brassard, Popescu, Schumacher, Smolin, and Wootters, Purification of Noisy Entanglement and Faithful Teleportation via Noisy Channels, Phys. Rev. Lett. 76, 722 (1996). 
 
 
%[Bra96]
\vspace{5pt}\noindent
[170] G. Brassard, Teleportation as a quantum computation, (Online preprint quant-ph/9605035), Physica D120 (1998) 43-47.


%[WZ82] 
\vspace{5pt}\noindent
[171] Wootters, W. K. and Zurek, W. H., A single quantum cannot be cloned. Nature 299, 802, (1982).


%[BH98] 
\vspace{5pt}\noindent
[172] V. Buzek, M. Hillery, Universal optimal cloning of qubits and quantum registers, (Online preprint quant-ph/9801009), QCQC 98, (Feb. 1998).

%[EHM+98] 
\vspace{5pt}\noindent
[173] A. Ekert, S.F. Huelga, C. Macchiavello, J.I. Cirac, Distributed Quantum Computation over Noisy Channels, (Online preprint quant-ph/9803017), (1998).

%[Sim94] 
\noindent
[174] Simon, D. R., On the power of quantum computation. In Proceedings of the 35th Annual Symposium on Fundamentals of Computer Science. IEEE Press, pp. 116-123, (Nov. 1994).


%[BH97] 
\vspace{5pt}\noindent
[175] G. Brassard, P. Hoyer, An Exact Quantum Polynomial-Time Algorithm for Simon's Problem,
 (Online preprint quant-ph/9704027), 
 Proceedings of the Fifth Israeli Symposium on Theory of Computing and Systems (ISTCS'97),
 (1997).

%[Cop94]
\vspace{5pt}\noindent
[176] D. Coppersmith. 
An approximate fourier transform useful in quantum computing. IBM Research Report RC19642, (1994). 


%[GN96]
\vspace{5pt}\noindent
[177] R. B. Griffiths, C. Niu, Semiclassical Fourier Transform for Quantum Computation, (Online preprint quant-ph/9511007), Phys.Rev.Lett. 76 (1996) 3228-3231.



%[Hoy97] 
\vspace{5pt}\noindent
[178] P. Hoyer, Efficient Quantum Transforms, (Online preprint quant-ph/9702028), (1997).

%[Bea98] 
\vspace{5pt}\noindent
[179] R. Beals. Quantum computation of Fourier transforms over symmetric groups. In Proceedings of the Twenty-Ninth Annual ACM Symposium on Theory of Computing, pages 48-53, El Paso, Texas, 4-6 May (1997).

%[PRB98] 
\vspace{5pt}\noindent
[180] M. Pueschel, M. Roetteler, T. Bet, Fast Quantum Fourier Transforms for a Class of Non-abelian Groups, Universitaet Karlsruhe, (Online preprint quant-ph/9807064), (1998).
 
%[VBE96]
\vspace{5pt}\noindent
[181]  V. Vedral, A. Barenco, A. Ekert, Quantum Networks for Elementary Arithmetic Operations, (Online preprint quant-ph/9511018), (1996).


%[Gri97] 
\vspace{5pt}\noindent
[182] D. Grigoriev. Testing shift-equivalence of polynomials by deterministic, probabilistic and quantum machines. Theoretical Computer Science, 180(1-2):217-228, (June 1997).

%
\vspace{5pt}\noindent
[183]P. W. Shor, Algorithms for quantum computation: discrete logarithms and factoring, Proceedings of the 35th Annual Symposium on Foundations of Computer Science, Santa Fe, NM, (Nov. 1994). 

%
\vspace{5pt}\noindent
[184] P. W. Shor, Polynomial-Time Algorithms for Prime Factorization and Discrete Logarithms on a Quantum Computer, (Online preprint quant-ph/9508027), SIAM J. Computing 26 (1997) 1484.


%[EJ96]
\vspace{5pt}\noindent
[185]  A. Ekert and R. Jozsa. Shor's quantum algorithme for factoring numbers, Review of Modern Physics, (1996).

%[Mil76]
\vspace{5pt}\noindent
[186] G.L. Miller, Reimann's hypothesis and test for primality, 
J. Computer Systes Sci., Vol. 12, pp. 300-317, (1976).


%[Zal98] 
\vspace{5pt}\noindent
[187] C. Zalka, Fast versions of Shor's quantum factoring algorithm, (Online preprint quant-ph/9806084), (1998).


%[BCD+96] 
\vspace{5pt}\noindent
[188] D. Beckman, A. N. Chari, S. Devabhaktuni, J. Preskill, Efficient Networks for Quantum Factoring, (Online preprint quant-ph/9602016), (1996).

%[OD96a] 
\noindent
[189] Obenland, K. and Despain, A. M., Simulation of factoring on a quantum computer architecture. In Proceedings of the 4th Workshop on Physics and Computation, Boston, Nov. (Nov. 1996), Boston: New England Complex Systems Institute.


%[PK96]
\vspace{5pt}\noindent
[190] M. B. Plenio, P. L. Knight, Realistic lower bounds for the factorization time of large numbers on a quantum computer, (Online preprint quant-ph/9512001), Phys. Rev. A 53, 2986 (1996).

%[VS+01]
\vspace{5pt}\noindent
[191] Vandersypen, L. M. K., Steffen, M., Breyta, G., Yannoni, C. S., Sherwood, M. H., \& Chuang, I. L. Nature, 414, 883�887 (2001). (Online preprint  doi:10.1038/414883a).

%[Kit95] 
\vspace{5pt}\noindent
[192] A. Y. Kitaev, Quantum measurements and the Abelian Stabilizer Problem, preprint, (1995). 


%[Gro96]
\noindent
[193] L. K. Grover, A fast quantum mechanical algorithm for database search, Proceedings of the Twenty-Eighth Annual ACM Symposium on Theory of Computing, 212-219, Philadelphia, PA, (May, 1996).

%[Gro97]
\vspace{5pt}\noindent
[194] L. K. Grover, Quantum computers can search arbitrarily large databases by a single query, (Online preprint quant-ph/9706005), (1997). 

%[Gro98] 
\vspace{5pt}\noindent
[195] L. K. Grover, Quantum computers can search rapidly by using almost any transformation, (Online preprint quant-ph/9712011), Phys.Rev.Lett. 80, 4329-4332 (1998).

%[Zal97] 
\vspace{5pt}\noindent
[196] C. Zalka, Grover's quantum searching algorithm is optimal, (Online preprint quant-ph/9711070), (1997).

%[PAT98] 
\vspace{5pt}\noindent
[197] A. K. Pati, Fast quantum search algorithm and Bounds on it, Theory Div. BARC, Mumbai, India, (Online preprint quant-ph/9807067), (1998). 

%[BBB+98] 
\vspace{5pt}\noindent
[198] D. Biron, O.Biham, E.Biham, M. Grassl, D.A. Lidar, Generalized Grover Search Algorithm for Arbitrary Initial Amplitude Distribution, (Online preprint quant-ph/9801066), QCQC 98, (Feb. 1998).

%[Coc97]
\vspace{5pt}\noindent
[199] P. Cockshott, Quantum Relational Databases, (Online preprint quant-ph/9712025), (1997).


%[BJ98] 
\vspace{5pt}\noindent
[200] S. C. Benjamin, N. F. Johnson, Structures for Data Processing in the Quantum Regime, University of Oxford, (Online preprint cond-mat/9802127), (1998).

%[FGG+98] 
\vspace{5pt}\noindent
[201] E. Farhi (MIT), J. Goldstone (MIT), S. Gutmann, M. Sipser, A Limit on the Speed of Quantum Computation in Determining Parity, (Online preprint quant-ph/9802045), (1998).

%[Mey00] 
\vspace{5pt}\noindent
[202] 
D. A. Meyer, "Sophisticated quantum search without entanglement", UCSD preprint ( 2000) .

%[TS98] 
\vspace{5pt}\noindent
[203] B. M. Terhal \& J. A. Smolin, "Single quantum querying of a database", Phys. Rev. A 58 (1998) 1822-1826.

%[BHT98] 
\vspace{5pt}\noindent
[204] Brassard, G., Peter Hoyer, and Alain Tapp, Quantum Counting,
 (Online preprint quant-ph/9805082), (1996).

%[BHT98] \bibitem[BHT 98]{BHT98}
\vspace{5pt}\noindent
[205] Brassard, G., P. Hoyer, and A. Tapp, Quantum Counting,
Proceedings of the 25th Colloquim on Automata, Languages, and Programming, 
Aalborg, Denmark, (May 1998).

%[Hog96]
\vspace{5pt}\noindent
[206] T. Hogg, Quantum Computing and Phase Transitions in Combinatorial Search, (Online preprint quant-ph/9508012), J. of Artificial Intelligence Research 4,91-128 (1996).

%[HY98]
\vspace{5pt}\noindent
[207] T. Hogg, M. Yanik, Local Search Methods for Quantum Computers, (Online preprint quant-ph/9802043), (1998).


%[Llo96]
\vspace{5pt}\noindent
[208] Lloyd, S., Universal quantum simulators. Science 273, 1073, (1996).

%[Zal96a]
\vspace{5pt}\noindent
[209] C. Zalka, Efficient Simulation of Quantum Systems by Quantum Computers, (Online preprint quant-ph/9603026), (1996).

%[Bog98]
\vspace{5pt}\noindent
[210] B. M. Boghosian, Simulating quantum mechanics on a quantum computer, (Online preprint quant-ph/9701019), Physica D120 (1998) 30-42.

%[Wie96] 
\vspace{5pt}\noindent
[211] S. Wiesner, Simulations of Many-Body Quantum Systems by a Quantum Computer, (Online preprint quant-ph/9603028), (1996).


%[AL97] 
\vspace{5pt}\noindent
[212] Abrams, D. S., Lloyd, S. Simulation of many-body fermi systems on a universal quantum computer. (Online preprint quant-ph/9703054.) (1997).

%[Ben96]
\vspace{5pt}\noindent
[213] Benio, P. A., Review of quantum computation. Trends in Statistical Physics by Council of Scientific Information, Trivandrum, India, 1996.

%[BT96]
\vspace{5pt}\noindent
[214] B. M. Boghosian, W. Taylor, Quantum lattice-gas models for the many-body Schrodinger equation, (Online preprint quant-ph/9701016), Sixth International Conference on Discrete Fluid Mechanics, BU, Boston MA, (August 1996).


%[Mey96a] 
\vspace{5pt}\noindent
[215] Meyer, D. A., Quantum mechanics of lattice gas automata I: one particle plane waves and potentials. (Online preprint quant-ph/9611005), (1996)

%[Mey96b] 
\vspace{5pt}\noindent
[216] D. A. Meyer, From quantum cellular automata to quantum lattice gases, (Online preprint quant-ph/9604003), J. Stat. Phys. 85, (1996) 551-574.

%[LB97] 
\vspace{5pt}\noindent
[217] D. A. Lidar, O. Biham, Simulating Ising Spin Glasses on a Quantum Computer, (Online preprint quant-ph/9611038), Phys. Rev. E vol.56 (1997), p.3661.


%[LW98]
\vspace{5pt}\noindent
[218] D. A. Lidar, H. Wang, Calculating the Thermal Rate Constant with Exponential Speed-Up on a Quantum Computer, UC Berkeley, (Online preprint quant-ph/9807009), (1998).

%[Sch97]
\vspace{5pt}\noindent
[219] R. Schack, Using a quantum computer to investigate quantum chaos, (Online preprint quant-ph/9705016), (1997).

%[BJ95]
\vspace{5pt}\noindent
[220] N. H. Bshouty and J. C. Jackson. Learning DNF over the uniform distribution using a quantum example oracle. In Proceedings of the Eighth Annual Conference on Computational Learning Theory, pages 118-127, Santa Cruz, California, (July 1995). ACM Press.

%[VM98c] 
\vspace{5pt}\noindent
[221] D. Ventura, T. Martinez, A Quantum Computational Learning Algorithm, (previously entitled "Quantum Harmonic Sieve: Learning DNF Using a Classical Example Oracle"), (Online preprint ph/9807052), (1998).
 

%
\vspace{5pt}\noindent
[222] A. Yu. Vlasov, Quantum Computations and Images Recognition, (Online preprint quant-ph/9703010), QCM'96, Japan (September 1996).


%[Tuc98]
\vspace{5pt}\noindent
[223] R. R. Tucci, How to Compile A Quantum Bayesian Net, (Online preprint quant-ph/9805016), (1998).

%[VM98b]
\noindent
[224] D. Ventura, T. Martinez, Quantum Associative Memory, (Online preprint quant-ph/9807053), (1998).


%[Ben97] 
\vspace{5pt}\noindent
[225] P. Benioff, Tight Binding Hamiltonians and Quantum Turing Machines, (Online preprint quant-ph/9610026), Phys.Rev.Lett. 78 (1997) 590-593.

%[Lan95]
\vspace{5pt}\noindent
[226] Landauer, R., Is quantum mechanics useful?, Phil. Tran. R. Soc. Lond. 353, 367, (1995).

%[Lan97]
\vspace{5pt}\noindent
[227] Landauer, R. 1997 Is quantum mechanically coherent computation useful?, In Proc. Drexel-4 Symposium on Quantum Nonintegrability-Quantum-Classical Correspondence, Philadelphia, PA, 8 September 1994 (ed. D. H. Feng and B.-L. Hu), Boston: International Press, (1997).

%[HMP+97]
\vspace{5pt}\noindent
[228] Huelga, S. F., Macchiavello, C., Pellizzari, T., Ekert, A. K., Plenio, M. B., and Cirac, J. I. 1997 On the improvement of frequency standards with quantum entanglement. (Online preprint quantph/9707014.)

%[BIW+97] 
\vspace{5pt}\noindent
[229] Bollinger, J. J., Itano, W. M., Wineland, D. J. and Heinzen, D. J. Optical frequency measurements with maximally correlated states. Phys. Rev. A 54, R4649, (1997).


%[Mey99]  
\vspace{5pt}\noindent
[230]  D. A. Meyer, Quantum strategies, Phys. Rev. Lett. 82 (1999) 1052-1055. 



%[Mey01]
\vspace{5pt}\noindent
[231] D. A. Meyer, Quantum Games and Quantum Algorithms, Quantum Computing and Quantum Information Science,  AMS Contemporary Mathematics volume, 2001.Quantum games and quantum algorithms David A. Meyer,  AMS Contemporary Mathematics volume, New York:ACM, (1993) pp. 11-20.


%[BLD98]
\vspace{5pt}\noindent
[232] G. Burkard, D. Loss, D. P. DiVincenzo, Coupled quantum dots as quantum gates, (Online preprint cond-mat/9808026), (1998).


%[LD97] 
\vspace{5pt}\noindent
[233] D. Loss (Basel), D. P. DiVincenzo, Quantum Computation with Quantum Dots, (Online preprint cond-mat/9701055), (1997).

%[Mee96]
\vspace{5pt}\noindent
[234] Meekhof, D. M., Monroe, C., King, B. E., Itano, W. M. and Wineland, D. J. Generation of nonclassical motional states of a trapped atom. Phys. Rev. Lett. 76, 1796 (1996).


%[CZ95] 
\vspace{5pt}\noindent
[235] Cirac, J. I. and Zoller, P. Quantum computations with cold trapped ions, Phys. Rev. Lett. 74, 4091, (1995).

%[PGCZ95] 
\vspace{5pt}\noindent
[236] T. Pellizzari, S. A. Gardiner, J. I. Cirac, and P. Zoller. Decoherence, continuous observation and quantum computing: A cavity QED model, Submitted to Physical Review Letters, (1995).


%[Jam97] 
\vspace{5pt}\noindent
[237] D. F. V. James, Quantum dynamics of cold trapped ions, with application to quantum computation, (Online preprint quant-ph/9702053), (1997).

%[WMM+96] 
\vspace{5pt}\noindent
[238] D. J. Wineland, C. Monroe, D. M. Meekhof, B. E. King, D. Leibfried, W. M. Itano, J. C. Bergquist, D.Berkeland, J. J. Bollinger, J. Miller, Quantum state manipulation of trapped atomic ions, (Online preprint quant-ph/9705022), Proc. Workshop on Quantum Computing, Santa Barbara, CA, (Dec. 1996), Submitted to Proc. Roy. Soc. A.


%[WMI+98]
\vspace{5pt}\noindent
[239] D.J. Wineland, C. Monroe, W.M. Itano, D. Leibfried, B.E. King, D.M. Meekhof, Experimental issues in coherent quantum-state manipulation of trapped atomic ions, (Online preprint quant-ph/9710025), Journal of Research of the National Institute of Standards and Technology 03, pp 259 (1998).


%[KWM98]
\vspace{5pt}\noindent
[240] B. E. King, C. S. Wood, C. J. Myatt, Q. A. Turchette, D. Leibfried, W. M. Itano, C. Monroe, D. J. Wineland, Initializing the Collective Motion of Trapped Ions for Quantum Logic, (Online preprint quant-ph/9803023), (1998). 

%[TWK+98]
\vspace{5pt}\noindent
[241] Q.A. Turchette, C.S. Wood, B.E. King, C.J. Myatt, D. Leibfried, W.M. Itano, C. Monroe, D.J. Wineland, Deterministic entanglement of two trapped ions, Time and Frequency Division, National Institute of Standards and Technology, Boulder, CO, (Online preprint quant-ph/9806012), (1998).

%[Hug97] 
\vspace{5pt}\noindent
[242] R. J. Hughes, Cryptography, Quantum Computation and Trapped Ions, (Online preprint quant-ph/9712054), Submitted to "Philosophical Transactions of the Royal Society," proceedings of the Royal Society Discussion Meeting on "Quantum Computation: Theory and Experiment," London, England, (November 1997).

%[HJG+98]
\vspace{5pt}\noindent
[243] R. J. Hughes, D. F. V. James, J. J. Gomez, M. S. Gulley, M. H. Holzscheiter, P. G. Kwiat, S. K. Lamoreaux, C. G. Peterson, V. D. Sandberg, M. M. Schauer, C. M. Simmons, C. E. Thorburn, D. Tupa, P. Z., The Los Alamos Trapped Ion Quantum Computer Experiment, (Online preprint quant-ph/9708050), Fortsch.Phys. 46 (1998) 329-362.


%[JGH+98] 
\vspace{5pt}\noindent
[244] D. F. V. James, M. S. Gulley, M. H. Holzscheiter, R. J. Hughes, P. G. Kwiat, S. K. Lamoreaux, C. G. Peterson, V. D. Sandberg, M. M. Schauer, C. M. Simmons, D. Tupa, P. Z. Wang, A. G. White, Trapped Ion Quantum Computer Research at Los Alamos, Los Alamos National Laboratory, (Online preprint quant-ph/9807071), QCQC 98, (Feb. 1998).


%[HJK+96] 
\vspace{5pt}\noindent
[245] R. J. Hughes, D. F. V. James, E. H. Knill, R. Laflamme, A. G. Petschek, Decoherence Bounds on Quantum Computation with Trapped Ions, (Online preprint quant-ph/9604026), (1996).

%[PK97] 
\vspace{5pt}\noindent
[246] M.B. Plenio, P.L. Knight, Decoherence limits to quantum computation using trapped ions, (Online preprint quant-ph/9610015), Proc.Roy.Soc.Lond. A453 (1997) 2017-2041.

%[SWM+98] 
\vspace{5pt}\noindent
[247] S. Schneider, H.M. Wiseman, W.J. Munro, G.J. Milburn, Measurement and state preparation via ion trap quantum computing, (Online preprint quant-ph/9709042), Fortsch.Phys. 46, 391-400, (1998).

%[Jam98]
\vspace{5pt}\noindent
[248] D. F. V. James, The theory of heating of the quantum ground state of trapped ions, (Online preprint quant-ph/9804048), Phys.Rev.Lett. 81 (1998) 317-320.

%[SJM98]
\vspace{5pt}\noindent
[249] S. Schneider, D. F.V. James, G. J. Milburn, Method of quantum computation with ``hot'' trapped ions, (Online preprint quant-ph/9808012), submitted to PRL, (1998).

%[THL+95] 
\vspace{5pt}\noindent
[250] Turchette, Q. A., Hood, C. J., Lange, W., Mabuchi, H. and Kimble, H. J. Measurement of conditional phase shifts for quantum logic. Phys. Rev. Lett. 75, 4710, (1995).

%[CY95] 
\vspace{5pt}\noindent
[251] I. L. Chuang, Y. Yamamoto, A Simple Quantum Computer, (Online preprint quant-ph/9505011), Submitted to Physical Review A (1995).

%[CVZ+98]
\vspace{5pt}\noindent
[252] I. L. Chuang, L. M.K. Vandersypen, X.Zhou, D. W. Leung, S. Lloyd, Experimental realization of a quantum algorithm, (Online preprint quant-ph/9801037), Nature, 393, 143-146, (1998). 


%[TS96] 
\vspace{5pt}\noindent
[253] P. Torma, S. Stenholm, Polarization in Quantum Computations, (Online preprint quant-ph/9602021), (1996).

%[AC98a]
\vspace{5pt}\noindent
[254] C. Adami, N.J. Cerf, Quantum computation with linear optics, (Online preprint quant-ph/9806048), QCQC 98, (Feb. 1998). 

%[TOM88]
\vspace{5pt}\noindent
[255] W. Teich, K. Obermeyer, G. Mehler, Structural Basis of Multistationary Quantum Systems, II. Effective Few-Particle Dynamics, Physical Review B, Vol. 37, No 14, pp 8111-8120, (1988).

%
\vspace{5pt}\noindent
[256] S. Lloyd. A potentially realizable quantum computer. Science, 261, 1569 (1993).

%[DiV97b] 
\vspace{5pt}\noindent
[257] D. P. DiVincenzo, Quantum Computation and Spin Physics, (Online preprint cond-mat/9612125), Proceedings of the Annual MMM Meeting, November, 1996, to be published in J. Appl. Phys (1997).

%[WXM98a]
\vspace{5pt}\noindent
[258] H. Wei, X. Xue, S. D. Morgera, Single Molecule Magnetic Resonance and Quantum Computation, (Online preprint quant-ph/9807057), (1998).

%[WXM98b] 
\vspace{5pt}\noindent
\vspace{5pt}\noindent
[259] H. Wei, X. Xue, S. D. Morgera, NMR Quantum Automata in Doped Crystals, (Online preprint quant-ph/9805059), (1998).

 
%[Cas97] 
\vspace{5pt}\noindent
[260] G. Castagnoli, Quantum Computation Based on Retarded and Advanced Propagation, (Online preprint quant-ph/9706019), (1997).



%[Kak98] 
\vspace{5pt}\noindent
[261] S. Kak, On Initializing Quantum Registers and Quantum Gates, Louisiana State University, (Online preprint quant-ph/9805002), (1998).


%[MPP+97] 
\vspace{5pt}\noindent
[262] M. Murao M.B. Plenio, S. Popescu, V. Vedral, P.L. Knight, Multi-Particle Entanglement Purification Protocols, (Online preprint quant-ph/9712045), (1997).

%[CFH96] 
\vspace{5pt}\noindent
[263] Cory, D. G., Fahmy, A. F. and Havel, T. F., Nuclear magnetic resonance spectroscopy: an experimentally accessible paradigm for quantum computing. In Proceedings of the 4th Workshop on Physics and Computation, Boston: New England Complex Systems Institute, (1996).

%[GC97]
\vspace{5pt}\noindent
[264] Gershenfeld, N. and Chuang, I. Bulk spin resonance quantum computation. Science 275, 350, (1997).

%[BDL+98] 
\noindent
[265] G. P. Berman, G. D. Doolen, G. V. Lopez, V. I. Tsifrinovich, Quantum Entangled States and Quasiclassical Dynamics in Macroscopic Spin Systems, (Online preprint quant-ph/9802015), (1998).

%[JHM98]
\vspace{5pt}\noindent
[265] J. A. Jones, R. H. Hansen, M. Mosca, (Online preprint quantum Logic Gates and Nuclear Magnetic Resonance Pulse Sequences, (Online preprint quant-ph/9805070), Submitted to Journal of Magnetic Resonance, (1998).


%[Kan98] 
\vspace{5pt}\noindent
[266] B.E. Kane, A silicon-based nuclear spin quantum computer,
Nature,  393,  133�137, (1998).


%[JMH98] 
\vspace{5pt}\noindent
[267] J. A. Jones, M. Mosca, R. H. Hansen, Implementation of a Quantum Search Algorithm on a Nuclear Magnetic Resonance Quantum Computer, (Online preprint quant-ph/9805069), Nature 393 (1998) 344-346.


%[Jon98]
\vspace{5pt}\noindent
[268] J. A. Jones, Fast Searches with Nuclear Magnetic Resonance Computers, Science 280, 229, (April 1998).

%[JM98a]
\vspace{5pt}\noindent
[269] J. A. Jones, M. Mosca , Approximate quantum counting on an NMR ensemble quantum computer, Submitted to Physical Review Letters, (1998). 

%[JM98b]
\vspace{5pt}\noindent
[270] J. A. Jones, M. Mosca, Implementation of a Quantum Algorithm to Solve Deutsch's Problem on a Nuclear Magnetic Resonance Quantum Computer, (Online preprint quant-ph/9801027), Journal of Chemical Physics, in press (August 1998).

%[LBR98] 
\noindent
[271] N Linden, H Barjat, R Freeman, An implementation of the Deutsch-Jozsa algorithm on a three-qubit NMR quantum computer, (Online preprint quant-ph/9808039), (1998).


%[War95]
\vspace{5pt}\noindent
[272] W.S. Warren. The usefulness if NMR quantum computing. Sience, 277: 1688-1689, (see also response
by N. Gerenfeld and I. Chuang, ibid, pp 1689-90), (1997).


%[Bar98] 
\vspace{5pt}\noindent
[273] S. E. Barnes, Efficient quantum computing on low temperature spin ensembles, (Online preprint quant-ph/9804065), (1998).


%[KCL97] 
\vspace{5pt}\noindent
[274] E. Knill, Is. Chuang, R. Laflamme, Effective Pure States for Bulk Quantum Computation, (Online preprint quant-ph/9706053), (1997).


%[SV98] 
\noindent
[275]  L. J. Schulman, U. Vazirani, Scalable NMR Quantum Computation, (Online preprint quant-ph/9804060), (1998).

%[ZHS98]
\vspace{5pt}\noindent
[276] Zyczkowski, Horodecki, Sanpera, and Lewenstein, On the volume of the set of mixed entangled states, (Online preprint quant-ph/9804024), (1998).

%[Ge02] 
\vspace{5pt}\noindent
[277] J. Gea-Banacloche, Minimum Energy Requirements for Quantum Computation, The American Physical Societyl, Vol 89, No. 12, (18 Nov 2002), pp. 217901-1 - 217901-4.


%[Oz02]
\vspace{5pt}\noindent
[278] M. Ozawa, Conservative Quantum Computing, Phys. Rev. Lett. 89, 057902 (2002). 


%[HP97] 
\vspace{5pt}\noindent
[279] O. Hay, A. Peres, Quantum and classical descriptions of a measuring apparatus,
(Online preprint quant-ph/9712044.), (1997). 

%[Zur91] 
\vspace{5pt}\noindent
[280] W.~H. Zurek. Decoherence and the transition from
quantum to classical,  Physics Today, vol. 44, pp. 36--44, (1991).

%[HM97] 
\vspace{5pt}\noindent
[281] D'Helon, C. and Milburn, G. J. Quantum measurements with a quantum computer. (Online preprint quant-ph/9705014.), (1997).


%[Oza98]
\vspace{5pt}\noindent
[282] M. Ozawa, Quantum Nondemolition Monitoring of Universal Quantum Computers, (Online preprint quant-ph/9704028), to appear in Phys.Rev.Lett. 80 (1998) 631.


%[BHD96] 
\vspace{5pt}\noindent
[283] Brune, M., Hagley, E., Dreyer, J., Maitre, X., Maali, A., Wunerlich, C., Raimond, J. M. and Haroche, S., Observing the progressive decoherence of the meter in a quantum measurement. Phys. Rev. Lett. 77, 4887, (1996).

%[CY96]
\vspace{5pt}\noindent
[284] I. L. Chuang, Y. Yamamoto, Quantum Bit Regeneration, (Online preprint quant-ph/9604031), Phys. Rev. Lett., (May 13, 1996).

%[MZ96] 
\vspace{5pt}\noindent
[285] Mabuchi, H. and Zoller, P., Inversion of quantum jumps in quantum-optical systems under continuous observation, Phys. Rev. Lett. 76, 3108, (1996).

%
\vspace{5pt}\noindent
[286] M. Ueda, Logical Reversibility and Physical Reversibility in Quantum Measurement, (Online preprint quant-ph/9709045), Int. Conf. on Frontiers in Quantum Physics, Kuala Lumpur, Malaysia, Springer-Verlag, (July, 1997).


%[Got66]
\vspace{5pt}\noindent
[287] K. Gottfield, Quantum Mechanics. London: Benjamin-Cummings (reprinted by Addison-Wesley, 1989), ch 4, pp. 165-190, (1966).


%[DL94] 
\vspace{5pt}\noindent
[288] L. Diosi, B. Lukacs, Eds., Stochastic Evolution of Quantum States in Open Quantum Systems and in the Measurement Process, London: World Scientific, (1994).


%[Pea84] 
\vspace{5pt}\noindent
[289] P. Pearle, State vector reduction as a dynamical process, proceedings of SUNY-Albany conference on Fundamental Questions in Quantum Mechanics (ed. A. Inomata, J. Kimball, and L. Roth), (1984). 

%[Pea85] 
\vspace{5pt}\noindent
[290] P. Pearle, Models for reduction, Quantum concepts in space and time, (1985).





} %end small











\end{document}