\providecommand{\slides}{
  \newcommand{\slideshead}{
  \newcommand{\thepage}{\arabic{mypage}}
  %beamer
  \documentclass[t,hyperref={bookmarks=true}]{beamer}
%  \documentclass[t,hyperref={bookmarks=true},aspectratio=169]{beamer}
  \setbeamersize{text margin left=5mm}
  \setbeamersize{text margin right=5mm}
  \usetheme{default}
  \usefonttheme[onlymath]{serif}
  \setbeamertemplate{navigation symbols}{}
  \setbeamertemplate{itemize items}{{\color{black}$\bullet$}}

  \newwrite\keyfile

  %\usepackage{palatino}
  \stdpackages
  \usepackage{multimedia}

  %%% geometry/spacing issues
  %
  \definecolor{bluecol}{rgb}{0,0,.5}
  \definecolor{greencol}{rgb}{0,.6,0}
  %\renewcommand{\baselinestretch}{1.1}
  \renewcommand{\arraystretch}{1.2}
  \columnsep 0mm

  \columnseprule 0pt
  \parindent 0ex
  \parskip 0ex
  %\setlength{\itemparsep}{3ex}
  %\renewcommand{\labelitemi}{\rule[3pt]{10pt}{10pt}~}
  %\renewcommand{\labelenumi}{\textbf{(\arabic{enumi})}}
  \newcommand{\headerfont}{\helvetica{13}{1.5}{b}{n}}
  \newcommand{\slidefont} {\helvetica{10}{1.4}{m}{n}}
  \newcommand{\codefont} {\helvetica{8}{1.2}{m}{n}}
  \renewcommand{\small} {\helvetica{9}{1.4}{m}{n}}
  \renewcommand{\tiny} {\helvetica{8}{1.3}{m}{n}}
  \newcommand{\ttiny} {\helvetica{7}{1.3}{m}{n}}

  %%% count pages properly and put the page number in bottom right
  %
  \newcounter{mypage}
  \newcommand{\incpage}{\addtocounter{mypage}{1}\setcounter{page}{\arabic{mypage}}}
  \setcounter{mypage}{0}
  \resetcounteronoverlays{page}

  \pagestyle{fancy}
  %\setlength{\headsep}{10mm}
  %\addtolength{\footheight}{15mm}
  \renewcommand{\headrulewidth}{0pt} %1pt}
  \renewcommand{\footrulewidth}{0pt} %.5pt}
  \cfoot{}
  \rhead{}
  \lhead{}
  \lfoot{\vspace*{-3mm}\hspace*{-3mm}\helvetica{5}{1.3}{m}{n}{\texttt{github.com/MarcToussaint/AI-lectures}}}
%  \rfoot{{\tiny\textsf{AI -- \topic -- \subtopic -- \arabic{mypage}/\pageref{lastpage}}}}
%  \rfoot{\vspace*{-4.5mm}{\tiny\textsf{\topic\ -- \subtopic\ -- \arabic{mypage}/\pageref{lastpage}}}\hspace*{-4mm}}
  \rfoot{\vspace*{-4.5mm}{\tiny\textsf{\color{gray}\topic\ -- \subtopic\ -- \arabic{mypage}/\pageref{lastpage}}}\hspace*{-4mm}}
  %\lfoot{\raisebox{5mm}{\tiny\textsf{\slideauthor}}}
  %\rfoot{\raisebox{5mm}{\tiny\textsf{\slidevenue{} -- \arabic{mypage}/\pageref{lastpage}}}}
  %\rfoot{~\anchor{30,12}{\tiny\textsf{\thepage/\pageref{lastpage}}}}
  %\lfoot{\small\textsf{Marc Toussaint}}

  \definecolor{grey}{rgb}{.8,.8,.8}
  \definecolor{head}{rgb}{.85,.9,.9}
  \definecolor{blue}{rgb}{.0,.0,.5}
  \definecolor{green}{rgb}{.0,.5,.0}
  \definecolor{red}{rgb}{.8,.0,.0}
  \newcommand{\inverted}{
    \definecolor{main}{rgb}{1,1,1}
    \color{main}
    \pagecolor[rgb]{.3,.3,.3}
  }
  %auto-ignore
  \renewcommand{\a}{\alpha}
  \renewcommand{\b}{\beta}
  \renewcommand{\d}{\delta}
    \newcommand{\D}{\Delta}
    \newcommand{\e}{\epsilon}
    \newcommand{\g}{\gamma}
    \newcommand{\G}{\Gamma}
  \renewcommand{\l}{\lambda}
  \renewcommand{\L}{\Lambda}
    \newcommand{\m}{\mu}
    \newcommand{\n}{\nu}
    \newcommand{\N}{\nabla}
  \renewcommand{\k}{\kappa}
  \renewcommand{\o}{\omega}
  \renewcommand{\O}{\Omega}
    \newcommand{\p}{\phi}
    \newcommand{\ph}{\varphi}
  \renewcommand{\P}{\Phi}
  \renewcommand{\r}{\varrho}
    \newcommand{\s}{\sigma}
  \renewcommand{\S}{\Sigma}
  \renewcommand{\t}{\theta}
    \newcommand{\T}{\Theta}
  %\renewcommand{\v}{\vartheta}
    \newcommand{\x}{\xi}
    \newcommand{\X}{\Xi}
    \newcommand{\Y}{\Upsilon}
    \newcommand{\z}{\zeta}

  \renewcommand{\AA}{{\cal A}}
    \newcommand{\BB}{{\cal B}}
    \newcommand{\CC}{{\cal C}}
    \newcommand{\cc}{{\cal c}}
    \newcommand{\DD}{{\cal D}}
    \newcommand{\EE}{{\cal E}}
    \newcommand{\FF}{{\cal F}}
    \newcommand{\GG}{{\cal G}}
    \newcommand{\HH}{{\cal H}}
    \newcommand{\II}{{\cal I}}
    \newcommand{\KK}{{\cal K}}
    \newcommand{\LL}{{\cal L}}
    \newcommand{\MM}{{\cal M}}
    \newcommand{\NN}{{\cal N}}
    \newcommand{\oNN}{\overline\NN}
    \newcommand{\OO}{{\cal O}}
    \newcommand{\PP}{{\cal P}}
    \newcommand{\QQ}{{\cal Q}}
    \newcommand{\RR}{{\cal R}}
  \renewcommand{\SS}{{\cal S}}
    \newcommand{\TT}{{\cal T}}
    \newcommand{\uu}{{\cal u}}
    \newcommand{\UU}{{\cal U}}
    \newcommand{\VV}{{\cal V}}
    \newcommand{\XX}{{\cal X}}
    \newcommand{\xx}{\mathcal{x}}
    \newcommand{\YY}{{\cal Y}}
    \newcommand{\SOSO}{{\cal SO}}
    \newcommand{\GLGL}{{\cal GL}}

    \newcommand{\Ee}{{\rm E}}

  \newcommand{\NNN}{{\mathbb{N}}}
  \newcommand{\III}{{\mathbb{I}}}
  \newcommand{\ZZZ}{{\mathbb{Z}}}
  %\newcommand{\RRR}{{\mathrm{I\!R}}}
  \newcommand{\RRR}{{\mathbb{R}}}
  \newcommand{\SSS}{{\mathbb{S}}}
  \newcommand{\CCC}{{\mathbb{C}}}
  \newcommand{\DDD}{{\mathbb{D}}}
  \newcommand{\one}{{{\bf 1}}}
  \newcommand{\eee}{\text{e}}

  \newcommand{\NNNN}{{\overline{\cal N}}}

  \renewcommand{\[}{\Big[}
  \renewcommand{\]}{\Big]}
  \renewcommand{\(}{\Big(}
  \renewcommand{\)}{\Big)}
  \renewcommand{\|}{\,|\,}
  \renewcommand{\;}{\,;\,}
  \renewcommand{\=}{\!=\!}
    \newcommand{\<}{\left\langle}
  \renewcommand{\>}{\right\rangle}

  \newcommand{\na}[1][]{{\nabla_{\!\!#1}}}
  \newcommand{\he}[1][]{{\nabla_{\!\!#1}^2}}
  \newcommand{\Prob}{{\rm Prob}}
  \newcommand{\Dir}{{\rm Dir}}
  \newcommand{\Beta}{{\rm Beta}}
  \newcommand{\Bern}{{\rm Bern}}
  \newcommand{\Bin}{{\rm Bin}}
  \newcommand{\Mult}{{\rm Mult}}
  \newcommand{\Aut}{{\rm Aut}}
  \newcommand{\cor}{{\rm cor}}
  \newcommand{\corr}{{\rm corr}}
  \newcommand{\sd}{{\rm sd}}
  \newcommand{\tr}{{\rm tr}}
  \newcommand{\Tr}{{\rm Tr}}
  \newcommand{\rank}{{\rm rank}}
  \newcommand{\diag}{{\rm diag}}
  \newcommand{\dom}{{\rm dom}}
  \newcommand{\id}{{\rm id}}
  \newcommand{\Id}{{\rm\bf I}}
  \newcommand{\Gl}{{\rm Gl}}
  \renewcommand{\th}{\ensuremath{{}^\text{th}} }
  \newcommand{\lag}{\mathcal{L}}
  \newcommand{\inn}{\rfloor}
  \newcommand{\lie}{\pounds}
  \newcommand{\longto}{\longrightarrow}
  \newcommand{\speer}{\parbox{0.4ex}{\raisebox{0.8ex}{$\nearrow$}}}
  \renewcommand{\dag}{ {}^\dagger }
  \newcommand{\blbox}{\rule{1ex}{1ex}}
  \newcommand{\Ji}{J^\sharp}
  \newcommand{\h}{{}^\star}
  \newcommand{\w}{\wedge}
  \newcommand{\too}{\longrightarrow}
  \newcommand{\oot}{\longleftarrow}
  \newcommand{\To}{\Rightarrow}
  \newcommand{\oT}{\Leftarrow}
  \newcommand{\oTo}{\Leftrightarrow}
  \renewcommand{\iff}{~\Longleftrightarrow~}
  \newcommand{\Too}{\;\Longrightarrow\;}
  \newcommand{\oto}{\leftrightarrow}
  \newcommand{\ot}{\leftarrow}
  \newcommand{\ootoo}{\longleftrightarrow}
  \newcommand{\ow}{\stackrel{\circ}\wedge}
  \newcommand{\defeq}{\stackrel{\hspace{0.2ex}{}_\Delta}=}
%  \newcommand{\defeq}{{\overstack\Delta =}}
  \newcommand{\feed}{\nonumber \\}
  \newcommand{\comma}{~,\quad}
  \newcommand{\period}{~.\quad}
  \newcommand{\del}{\partial}
%  \newcommand{\quabla}{\Delta}
  \newcommand{\point}{$\bullet~~$}
  \newcommand{\doubletilde}{ ~ \raisebox{0.3ex}{$\widetilde {}$} \raisebox{0.6ex}{$\widetilde {}$} \!\! }
  \newcommand{\topcirc}{\parbox{0ex}{~\raisebox{2.5ex}{${}^\circ$}}}
  \newcommand{\topdot} {\parbox{0ex}{~\raisebox{2.5ex}{$\cdot$}}}
  \newcommand{\topddot} {\parbox{0ex}{~\raisebox{1.3ex}{$\ddot{~}$}}}
  \newcommand{\sym}{\topcirc}
  \newcommand{\tsum}{\textstyle\sum}
  \newcommand{\st}{\quad\text{s.t.}\quad}

  \newcommand{\half}{\ensuremath{\frac{1}{2}}}
  \newcommand{\third}{\ensuremath{\frac{1}{3}}}
  \newcommand{\fourth}{\ensuremath{\frac{1}{4}}}

  \newcommand{\ubar}{\underline}
  %\renewcommand{\vec}{\underline}
  \renewcommand{\vec}{\boldsymbol}
  %\renewcommand{\_}{\underset}
  %\renewcommand{\^}{\overset}
  %\renewcommand{\*}{{\rm\raisebox{-.6ex}{\text{*}}{}}}
  \renewcommand{\*}{\text{\footnotesize\raisebox{-.4ex}{*}{}}}

  \newcommand{\gto}{{\raisebox{.5ex}{${}_\rightarrow$}}}
  \newcommand{\gfrom}{{\raisebox{.5ex}{${}_\leftarrow$}}}
  \newcommand{\gnto}{{\raisebox{.5ex}{${}_\nrightarrow$}}}
  \newcommand{\gnfrom}{{\raisebox{.5ex}{${}_\nleftarrow$}}}

  %\newcommand{\RND}{{\SS}}
  %\newcommand{\IF}{\text{if }}
  %\newcommand{\AND}{\textsc{and }}
  %\newcommand{\OR}{\textsc{or }}
  %\newcommand{\XOR}{\textsc{xor }}
  %\newcommand{\NOT}{\textsc{not }}

  %\newcommand{\argmax}[1]{{\rm arg}\!\max_{#1}}
  %\newcommand{\argmin}[1]{{\rm arg}\!\min_{#1}}
  \DeclareMathOperator*{\argmax}{argmax}
  \DeclareMathOperator*{\argmin}{argmin}
  \DeclareMathOperator{\sign}{sign}
  \DeclareMathOperator{\acos}{acos}
  \DeclareMathOperator{\unifies}{unifies}
  \DeclareMathOperator{\Span}{span}
  \newcommand{\ortho}{\perp}
  %\newcommand{\argmax}[1]{\underset{~#1}{\text{argmax}}\;}
  %\newcommand{\argmin}[1]{\underset{~#1}{\text{argmin}}\;}
  \newcommand{\ee}[1]{\ensuremath{\cdot10^{#1}}}
  \newcommand{\sub}[1]{\ensuremath{_{\text{#1}}}}
  \newcommand{\up}[1]{\ensuremath{^{\text{#1}}}}
  \newcommand{\kld}[3][{}]{D_{#1}\big(#2\,\big|\!\big|\,#3\big)}
  %\newcommand{\kld}[2]{D\big(#1:#2\big)}
  \newcommand{\sprod}[2]{\big<#1\,,\,#2\big>}
  \newcommand{\End}{\text{End}}
  \newcommand{\txt}[1]{\quad\text{#1}\quad}
  \newcommand{\Over}[2]{\genfrac{}{}{0pt}{0}{#1}{#2}}
  %\newcommand{\mat}[1]{{\bf #1}}
  \newcommand{\arr}[2]{\hspace*{-.5ex}\begin{array}{#1}#2\end{array}\hspace*{-.5ex}}
  \newcommand{\mat}[3][.9]{
    \renewcommand{\arraystretch}{#1}{\scriptscriptstyle{\left(
      \hspace*{-1ex}\begin{array}{#2}#3\end{array}\hspace*{-1ex}
    \right)}}\renewcommand{\arraystretch}{1.2}
  }
  \newcommand{\Mat}[3][.9]{
    \renewcommand{\arraystretch}{#1}{\scriptscriptstyle{\left[
      \hspace*{-1ex}\begin{array}{#2}#3\end{array}\hspace*{-1ex}
    \right]}}\renewcommand{\arraystretch}{1.2}
  }
  \newcommand{\case}[2][ll]{\left\{\arr{#1}{#2}\right.}
  \newcommand{\seq}[1]{\textsf{\<#1\>}}
  \newcommand{\seqq}[1]{\textsf{#1}}
  \newcommand{\floor}[1]{\lfloor#1\rfloor}
  \newcommand{\Exp}[2][]{\text{E}_{#1}\{#2\}}
  \newcommand{\Var}[2][]{\text{Var}_{#1}\{#2\}}
  \newcommand{\cov}[2][]{\text{cov}_{#1}\{#2\}}

%\newcommand{\Exp}[2]{\left\langle{#2}\right\rangle_{#1}}
  \newcommand{\ex}{\setminus}

  \providecommand{\href}[2]{{\color{blue}USE PDFLATEX!}}
  \providecommand{\url}[2]{\href{#1}{{\color{blue}#2}}}
%  \newcommand{\link}[1]{\href{{\protect #1}}{\texttt{\protect #1}}}
  \newcommand{\anchor}[2]{\begin{picture}(0,0)\put(#1){#2}\end{picture}}
  \newcommand{\pagebox}{\begin{picture}(0,0)\put(-3,-23){
    \textcolor[rgb]{.5,1,.5}{\framebox[\textwidth]{\rule[-\textheight]{0pt}{0pt}}}}
    \end{picture}}

  \newcommand{\hide}[1]{
    \begin{list}{}{\leftmargin0ex \rightmargin0ex \topsep0ex \parsep0ex}
       \helvetica{5}{1}{m}{n}
       \renewcommand{\section}{\par SECTION: }
       \renewcommand{\subsection}{\par SUBSECTION: }
       \item[$~~\blacktriangleright$]
       #1%$\blacktriangleleft~~$
       \message{^^JHIDE--Warning!^^J}
    \end{list}
  }
  %\newcommand{\hide}[1]{{\tt[hide:~}{\footnotesize\sf #1}{\tt]}\message{^^JHIDE--Warning!^^J}}
  \newcommand{\Hide}{\renewcommand{\hide}[1]{\message{^^JHIDE--Warning (hidden)!^^J}}}
  \newcommand{\HIDE}{\renewcommand{\hide}[1]{}}
  \newcommand{\fullhide}[1]{\message{^^JHIDE--Warning (hidden)!^^J}}
  \newcommand{\todo}[1]{{\tt[TODO: #1]}\message{^^JTODO--Warning: #1^^J}}
  \newcommand{\Todo}{\renewcommand{\todo}[1]{\message{^^JTODO--Warning (hidden)!^^J}}}
  %\renewcommand{\title}[1]{\renewcommand{\thetitle}{#1}}
  \newcommand{\myauthor}[1]{\author{#1}\newcommand{\theauthor}{#1}}%\@author}
  \newcommand{\mytitle}[1]{\title{#1}\newcommand{\thetitle}{#1}}%\@title}
  \newcommand{\header}{\begin{document}\mytitle\cleardefs}
  \newcommand{\contents}{{\tableofcontents}\renewcommand{\contents}{}}
  \newcommand{\footer}{\small\bibliography{marc,bibs}\end{document}}
  \newcommand{\widepaper}{\usepackage{geometry}\geometry{a4paper,hdivide={25mm,*,25mm},vdivide={25mm,*,25mm}}}
  \newcommand{\moviex}[2]{\movie[externalviewer]{#1}{#2}} %\pdflatex\usepackage{multimedia}
  \newcommand{\rbox}[1]{\fboxrule2mm\fcolorbox[rgb]{1,.85,.85}{1,.85,.85}{#1}}
  \newcommand{\mpage}[2]{{\begin{minipage}{#1\columnwidth}#2\end{minipage}}}
  \newcommand{\redbox}[2]{\fboxrule1mm\fcolorbox[rgb]{1,.7,.7}{1,.7,.7}{\begin{minipage}{#1\columnwidth}\center#2\end{minipage}}}
  \newcommand{\onecol}[2]{
    \begin{minipage}[c]{#1\columnwidth}#2\end{minipage}}
  \newcommand{\twocol}[5][0]{
    \begin{minipage}[c]{#2\columnwidth}#4\end{minipage}\hspace*{#1\columnwidth}%
    \begin{minipage}[c]{#3\columnwidth}#5\end{minipage}}
  \newcommand{\threecol}[7][0]{%
    \begin{minipage}[c]{#2\columnwidth}#5\end{minipage}\hspace*{#1\columnwidth}%
    \begin{minipage}[c]{#3\columnwidth}#6\end{minipage}\hspace*{#1\columnwidth}%
    \begin{minipage}[c]{#4\columnwidth}#7\end{minipage}}
  \newcommand{\threecoltext}[7][c]{
    \begin{minipage}[#1]{#2\textwidth}#5\end{minipage}%
    \begin{minipage}[#1]{#3\textwidth}#6\end{minipage}%
    \begin{minipage}[#1]{#4\textwidth}#7\end{minipage}}
  \newcommand{\threecoltop}[7][0]{%
   \begin{minipage}[t]{#2\columnwidth}#5\end{minipage}\hspace*{#1\columnwidth}%
   \begin{minipage}[t]{#3\columnwidth}#6\end{minipage}\hspace*{#1\columnwidth}%
   \begin{minipage}[t]{#4\columnwidth}#7\end{minipage}}
  \newcommand{\fourcol}[9][0]{%
   \begin{minipage}[c]{#2\columnwidth}#6\end{minipage}\hspace*{#1\columnwidth}%
   \begin{minipage}[c]{#3\columnwidth}#7\end{minipage}\hspace*{#1\columnwidth}%
   \begin{minipage}[c]{#4\columnwidth}#8\end{minipage}\hspace*{#1\columnwidth}%
   \begin{minipage}[c]{#5\columnwidth}#9\end{minipage}}
  \newcommand{\helvetica}[4]{\setlength{\unitlength}{1pt}\fontsize{#1}{#1}\linespread{#2}\usefont{OT1}{phv}{#3}{#4}}
  \newcommand{\helve}[1]{\helvetica{#1}{1.5}{m}{n}}
  \newcommand{\german}{\usepackage[german]{babel}\usepackage[utf8]{inputenc}}

\newcommand{\norm}[1]{|\!|#1|\!|}
\newcommand{\expr}[1]{[\hspace{-.2ex}[#1]\hspace{-.2ex}]}

\newcommand{\Jwi}{J^\sharp_W}
\newcommand{\THi}{T^\sharp_H}
\newcommand{\Jci}{J^\natural_C}
\newcommand{\hJi}{{\bar J}^\sharp}
\renewcommand{\|}{\,|\,}
\renewcommand{\=}{\!=\!}
\newcommand{\myminus}{{\hspace*{-.0pt}\text{\rm -}\hspace*{-.5pt}}}
\newcommand{\myplus}{{\hspace*{-.0pt}\text{\rm +}\hspace*{-.5pt}}}
\newcommand{\1}{{\myminus1}}
\newcommand{\2}{{\myminus2}}
\newcommand{\3}{{\myminus3}}
\newcommand{\mT}{{\text{\rm -}\hspace*{-1pt}\top}}
\newcommand{\po}{{\myplus1}}
\newcommand{\pt}{{\myplus2}}
%\renewcommand{\-}{\myminus}
%\newcommand{\+}{\myplus}
\renewcommand{\T}{{\!\top\!}}
\newcommand{\xT}{{\underline x}}
\newcommand{\uT}{{\underline u}}
\newcommand{\zT}{{\underline z}}
\newcommand{\Sum}{\textstyle\sum}
\newcommand{\Int}{\textstyle\int}
\newcommand{\Prod}{\textstyle\prod}


\newenvironment{centy}{
\vspace{15mm}
\large
\hspace*{5mm}
\begin{minipage}{8cm}\it\color{blue}
}{
\end{minipage}
}

\newcommand{\old}{{\text{old}}}
\newcommand{\new}{{\text{new}}}
\newcommand{\MAP}{{\text{MAP}}}
\newcommand{\ML}{{\text{ML}}}

\newcommand{\redArrow}{\quad\anchor{0,-1}{\includegraphics[scale=.5]{figs/redArrow}}}
\newcommand{\pub}[1]{{\color{green}\helvetica{8}{1.3}{m}{n}#1\\}}
\DeclareMathOperator{\opKL}{KL}
\newcommand{\KL}[2]{\opKL\big(#1\,\big|\!\big|\,#2\big)} %\left(#1 |\!| #2\right)}

\renewcommand{\show}[2][.8]{\centerline{\includegraphics[width=#1\columnwidth]{#2}}}
\newcommand{\showh}[2][.8]{\includegraphics[width=#1\columnwidth]{#2}}
\newcommand{\shows}[2][.8]{\centerline{\includegraphics[scale=#1]{#2}}}
\newcommand{\showhs}[2][.8]{\includegraphics[scale=#1]{#2}}
\newcommand{\mov}[2]{\movie[externalviewer]{{\color{blue}\small #1}}{movies/#2}}
\newcommand{\movex}[2]{\movie[externalviewer]{#1}{#2}} %\pdflatex\usepackage{multimedia}
%\newcommand{\movgb}[1]{\hfill\movie[externalviewer]{\small[movie]}{/home/mtoussai/movies/10-goalDirectedBehavior/#1}}
\newcommand{\movh}[3][loop]{
\movie[#1]{\showh[#2]{movies/#3.png}}{movies/#3.avi}%
\movie[externalviewer]{$\circ$}{movies/#3.avi}
}
\newcommand{\movc}[3][loop]{\centerline{\movh[#1]{#2}{#3}}}
\newcommand{\cen}[1]{\centerline{#1}}

\newcommand{\citing}[1]{
{\color{citcol}\tiny#1\par}
}

\newcommand{\cit}[3]{
\par\smallskip
{\color{greencol}\tiny #1: \emph{#2}. #3 \par}
}

\newcommand{\citurl}[4]{
\par\smallskip
{\color{greencol}\tiny #1: \protect{\href{#4}{\color{blue}{#2.}}} #3 \par}
}

\newcommand{\cito}[3]{
\par\smallskip
{\color{bluecol}\tiny #1: \emph{#2}. #3 \par}
}

\newcommand{\redoMacrosInProof}{
  \renewcommand{\d}{\delta}
%  \renewcommand{\|}{\,|\,}
  \renewcommand{\=}{\!=\!}
}

%% \makeatletter
%% \newenvironment{code}{%
%%   \begin{lrbox}{\@tempboxa}\begin{minipage}{1\columnwidth}\codefont
%% }{
%%   \end{minipage}\end{lrbox}%
%%   \colorbox[rgb]{.95,.95,.95}{\usebox{\@tempboxa}}
%% }\makeatother

\newenvironment{code}{%
\codefont
\begin{shaded}
}{
\end{shaded}
}

%\newcommand{\refeq}[1]{(\ref{#1})}

\usepackage{algorithm}
\usepackage{algpseudocode}
\algrenewcommand{\algorithmicrequire}{\textbf{Input:~~}}
\algrenewcommand{\algorithmicensure}{\textbf{Output:}}
\algrenewcommand{\algorithmiccomment}[1]{\qquad\hfill~\hspace*{-5ex}\textit{// #1}}
\algrenewcommand{\alglinenumber}[1]{\helvetica{6}{1.3}{m}{n}#1:}

\newenvironment{algo}[1][8]{
\quad\begin{minipage}{.8\columnwidth}\helvetica{#1}{1.3}{m}{n}
\medskip\hrule\medskip
\begin{algorithmic}[1]
}{
\end{algorithmic}
\medskip\hrule\medskip
\end{minipage}
}

\usepackage{etoolbox}

%%%%%%%%%%%%%%%%%%%%%%%%%%%%%%%%%%%%%%%%%%%%%%%%%%%%%%%%%%%%%%%%%%%%%%%%%%%%%%%%

\usepackage{multirow}
\usepackage{colortbl}
%\setlength{\jot}{0pt}
%\setlength{\mathindent}{1ex}
\usepackage{empheq}

%%%%%%%%%%%%%%%%%%%%%%%%%%%%%%%%%%%%%%%%%%%%%%%%%%%%%%%%%%%%%%%%%%%%%%%%%%%%%%%

\newcommand{\mypause}{\pause}
%\newcommand{\dom}{{\text{dom}}}
\newcommand{\defi}[1]{\textbf{#1}}
\newcommand{\red}[1]{\emph{\color{red}#1}}
%\newcommand{\ul}{\underline}
\newcommand{\pos}{{\textsf{pos}}}
\newcommand{\eff}{{\textsf{eff}}}
\newcommand{\rot}{{\textsf{rot}}}
\newcommand{\veC}{{\textsf{vec}}}
\newcommand{\quat}{{\textsf{quat}}}
\newcommand{\col}{{\textsf{col}}}
\newcommand{\de}[2]{\frac{\partial #1}{\partial #2}}
\newcommand{\target}{{\text{target}}}
\newcommand{\near}{{\text{near}}}
\newcommand{\qfree}{Q_{\text{free}}}
\renewcommand{\vec}{\boldsymbol}
\newcommand{\lft}{\text{left}}
\newcommand{\rgh}{\text{right}}
\DeclareMathOperator{\real}{real}
\newcommand{\prev}{{\text{prev}}}
\newcommand{\TR}[2]{T_{{#1}\to{#2}}}
\newcommand{\RO}[2]{R_{{#1}\to{#2}}}
\newcommand{\liter}{\helvetica{8}{1.1}{m}{n}\parskip 1ex}
\newcommand{\Fc}{\color{green}F}
\newcommand{\muc}{\color{blue}\mu}
\newcommand{\Astar}{A$^*$}

%for AI course:
\newcommand{\defn}[1]{\textbf{#1}}

%for optimization course:
\newcommand{\adec}{\r_\a^-}
\newcommand{\ainc}{\r_\a^+}
\newcommand{\ldec}{\r_\l^-}
\newcommand{\linc}{\r_\l^+}
\newcommand{\minc}{\r_\m^+}
\newcommand{\mdec}{\r_\m^-}
\newcommand{\lsstop}{\r_{\text{ls}}}


\definecolor{boxcol}{rgb}{.85,.9,.92}
\newcommand{\eqbox}[1]{\centerline{\fboxrule0mm\fcolorbox{boxcol}{boxcol}{#1}}}
\newcommand{\movgb}[1]{\hfill\movie[externalviewer]{\small[movie]}{/home/mtoussai/movies/10-goalDirectedBehavior/#1}}
\newcommand{\demo}[1]{{{\color{blue}[\small #1]}}}

\graphicspath{{../pics-robotics/}{../pics-ML/}{../pics-all/}{../pics-all2/}{../pics-Optim/}}
\DeclareGraphicsExtensions{.pdf,.png,.jpg}

%\usepackage{pdfpages}
%\setbeamercolor{background canvas}{bg=}

\newcommand{\SUM}{\texttt{sum}}
\usepackage{float}

%% prevent pagebreaks before environment
\makeatletter
\newcommand{\NewParNoBreak}[1][\parskip]{\par\vspace*{-\parskip}\vspace*{#1}\nobreak\@afterheading}
\makeatother

%\newcommand{\idx}[2]{\label{IKgn}}

%%%%%%%%%%%%%%%%%%%%%%%%%%%%%%%%%%%%%%%%%%%%%%%%%%%%%%%%%%%%%%%%%%%%%%%%%%%%%%%%



%% \newwrite\tempfile
%% \immediate\openout\tempfile=z.keys.tex

%% \renewcommand{\key}[1]{
%% %%   \addtocounter{mypage}{1}
%% \makeatletter
%% \immediate\write\tempfile{\symbol{`\\}}
%% \makeatother
%%   \immediate\write\tempfile{hyperref[key:#1]{#1(\arabic{mypage})}}
%% %%  % \phantomsection\label{key:#1}
%% %%   %\index{#1@{\hyperref[key:#1]{#1 (\arabic{mysec}:\arabic{mypage})}}|phantom}
%% %%   \addtocounter{mypage}{-1}
%% }


  \graphicspath{{pics/}{../shared/pics/}}

  \title{\course \topic}
  \author{Marc Toussaint}
  \institute{Machine Learning \& Robotics Lab, U Stuttgart}

  \begin{document}

  \rfoot{\vspace*{-5mm}{\tiny
  \textsf{\arabic{mypage}/\pageref{lastpage}}}\hspace*{-4mm}}

  %% title slide!
  \slide{}{
    \thispagestyle{empty}

    \twocol{.27}{.6}{
      \vspace*{-5mm}\hspace*{-15mm}\includegraphics[width=4cm]{\coursepicture}
    }{\center

      \textbf{\fontsize{17}{20}\selectfont \course}

      ~

      %Lecture
      \topic\\

      \vspace{1cm}

      {\tiny~\emph{\keywords}~\\}

      \vspace{1cm}

      Marc Toussaint
      
      University of Stuttgart

      \coursedate

      ~

    }
  }
}

\newcommand{\slide}[2]{
  \slidefont
  \incpage\begin{frame}
  \addcontentsline{toc}{section}{#1}
  \vfill
  {\headerfont #1} \vspace*{-2ex}
  \begin{itemize}\item[]~\\
    #2
  \end{itemize}
  \vfill
  \end{frame}
}

\newenvironment{slidecore}[1]{
  \slidefont\incpage
  \addcontentsline{toc}{section}{#1}
  \vfill
  {\headerfont #1} \vspace*{-2ex}
  \begin{itemize}\item[]~\\
}{
  \end{itemize}
  \vfill
}


\providecommand{\key}[1]{
  \addtocounter{mypage}{1}
% \immediate\write\keyfile{#1}
  \addtocontents{toc}{\hyperref[key:#1]{#1 (\arabic{mypage})}}
%  \phantomsection\label{key:#1}
%  \index{#1@{\hyperref[key:#1]{#1 (\arabic{mysec}:\arabic{mypage})}}|phantom}
  \addtocounter{mypage}{-1}
}

\providecommand{\course}{}

\providecommand{\subtopic}{}

\providecommand{\sublecture}[2]{
  \renewcommand{\subtopic}{#1}
  \slide{#1}{#2}
}

\providecommand{\story}[1]{
~

Motivation: {\tiny #1}\clearpage
}

\newenvironment{items}[1][9]{
\par\setlength{\unitlength}{1pt}\fontsize{#1}{#1}\linespread{1.2}\selectfont
\begin{list}{--}{\leftmargin4ex \rightmargin0ex \labelsep1ex \labelwidth2ex
\topsep0pt \parsep0ex \itemsep3pt}
}{
\end{list}
}

\providecommand{\slidesfoot}{
  \end{document}
}


  \slideshead
}

\providecommand{\exercises}{
  \newcommand{\exerciseshead}{
  \documentclass[10pt,fleqn]{article}
  \stdpackages

  \definecolor{bluecol}{rgb}{0,0,.5}
  \definecolor{greencol}{rgb}{0,.4,0}
  \definecolor{shadecolor}{gray}{0.9}
  \usepackage[
    %    pdftex%,
    %%    letterpaper,
    %    bookmarks,
    %    bookmarksnumbered,
    colorlinks,
    urlcolor=bluecol,
    citecolor=black,
    linkcolor=bluecol,
    %    pagecolor=bluecol,
    pdfborder={0 0 0},
    %pdfborderstyle={/S/U/W 1},
    %%    backref,     %link from bibliography back to sections
    %%    pagebackref, %link from bibliography back to pages
    %%    pdfstartview=FitH, %fitwidth instead of fit window
    pdfpagemode=UseNone, %UseOutlines, %bookmarks are displayed by acrobat
    pdftitle={\course},
    pdfauthor={Marc Toussaint},
    pdfkeywords={}
  ]{hyperref}
  \DeclareGraphicsExtensions{.pdf,.png,.jpg,.eps}

  \renewcommand{\r}{\varrho}
  \renewcommand{\l}{\lambda}
  \renewcommand{\L}{\Lambda}
  \renewcommand{\b}{\beta}
  \renewcommand{\d}{\delta}
  \renewcommand{\k}{\kappa}
  \renewcommand{\t}{\theta}
  \renewcommand{\O}{\Omega}
  \renewcommand{\o}{\omega}
  \renewcommand{\SS}{{\cal S}}
  \renewcommand{\=}{\!=\!}
  %\renewcommand{\boldsymbol}{}
  %\renewcommand{\Chapter}{\chapter}
  %\renewcommand{\Subsection}{\subsection}

  \renewcommand{\baselinestretch}{1.1}
  \geometry{a4paper,headsep=7mm,hdivide={15mm,*,15mm},vdivide={20mm,*,15mm}}

  \fancyhead[L]{\thetitle, \textit{Marc Toussaint}---\coursedate}
  \fancyhead[R]{\thepage}
  \fancyhead[C]{}
  \fancyfoot{}
  \pagestyle{fancy}

  \parindent 0pt
  \parskip 0.5ex

  \newcommand{\codefont}{\helvetica{8}{1.2}{m}{n}}

  %auto-ignore
  \renewcommand{\a}{\alpha}
  \renewcommand{\b}{\beta}
  \renewcommand{\d}{\delta}
    \newcommand{\D}{\Delta}
    \newcommand{\e}{\epsilon}
    \newcommand{\g}{\gamma}
    \newcommand{\G}{\Gamma}
  \renewcommand{\l}{\lambda}
  \renewcommand{\L}{\Lambda}
    \newcommand{\m}{\mu}
    \newcommand{\n}{\nu}
    \newcommand{\N}{\nabla}
  \renewcommand{\k}{\kappa}
  \renewcommand{\o}{\omega}
  \renewcommand{\O}{\Omega}
    \newcommand{\p}{\phi}
    \newcommand{\ph}{\varphi}
  \renewcommand{\P}{\Phi}
  \renewcommand{\r}{\varrho}
    \newcommand{\s}{\sigma}
  \renewcommand{\S}{\Sigma}
  \renewcommand{\t}{\theta}
    \newcommand{\T}{\Theta}
  %\renewcommand{\v}{\vartheta}
    \newcommand{\x}{\xi}
    \newcommand{\X}{\Xi}
    \newcommand{\Y}{\Upsilon}
    \newcommand{\z}{\zeta}

  \renewcommand{\AA}{{\cal A}}
    \newcommand{\BB}{{\cal B}}
    \newcommand{\CC}{{\cal C}}
    \newcommand{\cc}{{\cal c}}
    \newcommand{\DD}{{\cal D}}
    \newcommand{\EE}{{\cal E}}
    \newcommand{\FF}{{\cal F}}
    \newcommand{\GG}{{\cal G}}
    \newcommand{\HH}{{\cal H}}
    \newcommand{\II}{{\cal I}}
    \newcommand{\KK}{{\cal K}}
    \newcommand{\LL}{{\cal L}}
    \newcommand{\MM}{{\cal M}}
    \newcommand{\NN}{{\cal N}}
    \newcommand{\oNN}{\overline\NN}
    \newcommand{\OO}{{\cal O}}
    \newcommand{\PP}{{\cal P}}
    \newcommand{\QQ}{{\cal Q}}
    \newcommand{\RR}{{\cal R}}
  \renewcommand{\SS}{{\cal S}}
    \newcommand{\TT}{{\cal T}}
    \newcommand{\uu}{{\cal u}}
    \newcommand{\UU}{{\cal U}}
    \newcommand{\VV}{{\cal V}}
    \newcommand{\XX}{{\cal X}}
    \newcommand{\xx}{\mathcal{x}}
    \newcommand{\YY}{{\cal Y}}
    \newcommand{\SOSO}{{\cal SO}}
    \newcommand{\GLGL}{{\cal GL}}

    \newcommand{\Ee}{{\rm E}}

  \newcommand{\NNN}{{\mathbb{N}}}
  \newcommand{\III}{{\mathbb{I}}}
  \newcommand{\ZZZ}{{\mathbb{Z}}}
  %\newcommand{\RRR}{{\mathrm{I\!R}}}
  \newcommand{\RRR}{{\mathbb{R}}}
  \newcommand{\SSS}{{\mathbb{S}}}
  \newcommand{\CCC}{{\mathbb{C}}}
  \newcommand{\DDD}{{\mathbb{D}}}
  \newcommand{\one}{{{\bf 1}}}
  \newcommand{\eee}{\text{e}}

  \newcommand{\NNNN}{{\overline{\cal N}}}

  \renewcommand{\[}{\Big[}
  \renewcommand{\]}{\Big]}
  \renewcommand{\(}{\Big(}
  \renewcommand{\)}{\Big)}
  \renewcommand{\|}{\,|\,}
  \renewcommand{\;}{\,;\,}
  \renewcommand{\=}{\!=\!}
    \newcommand{\<}{\left\langle}
  \renewcommand{\>}{\right\rangle}

  \newcommand{\na}[1][]{{\nabla_{\!\!#1}}}
  \newcommand{\he}[1][]{{\nabla_{\!\!#1}^2}}
  \newcommand{\Prob}{{\rm Prob}}
  \newcommand{\Dir}{{\rm Dir}}
  \newcommand{\Beta}{{\rm Beta}}
  \newcommand{\Bern}{{\rm Bern}}
  \newcommand{\Bin}{{\rm Bin}}
  \newcommand{\Mult}{{\rm Mult}}
  \newcommand{\Aut}{{\rm Aut}}
  \newcommand{\cor}{{\rm cor}}
  \newcommand{\corr}{{\rm corr}}
  \newcommand{\sd}{{\rm sd}}
  \newcommand{\tr}{{\rm tr}}
  \newcommand{\Tr}{{\rm Tr}}
  \newcommand{\rank}{{\rm rank}}
  \newcommand{\diag}{{\rm diag}}
  \newcommand{\dom}{{\rm dom}}
  \newcommand{\id}{{\rm id}}
  \newcommand{\Id}{{\rm\bf I}}
  \newcommand{\Gl}{{\rm Gl}}
  \renewcommand{\th}{\ensuremath{{}^\text{th}} }
  \newcommand{\lag}{\mathcal{L}}
  \newcommand{\inn}{\rfloor}
  \newcommand{\lie}{\pounds}
  \newcommand{\longto}{\longrightarrow}
  \newcommand{\speer}{\parbox{0.4ex}{\raisebox{0.8ex}{$\nearrow$}}}
  \renewcommand{\dag}{ {}^\dagger }
  \newcommand{\blbox}{\rule{1ex}{1ex}}
  \newcommand{\Ji}{J^\sharp}
  \newcommand{\h}{{}^\star}
  \newcommand{\w}{\wedge}
  \newcommand{\too}{\longrightarrow}
  \newcommand{\oot}{\longleftarrow}
  \newcommand{\To}{\Rightarrow}
  \newcommand{\oT}{\Leftarrow}
  \newcommand{\oTo}{\Leftrightarrow}
  \renewcommand{\iff}{~\Longleftrightarrow~}
  \newcommand{\Too}{\;\Longrightarrow\;}
  \newcommand{\oto}{\leftrightarrow}
  \newcommand{\ot}{\leftarrow}
  \newcommand{\ootoo}{\longleftrightarrow}
  \newcommand{\ow}{\stackrel{\circ}\wedge}
  \newcommand{\defeq}{\stackrel{\hspace{0.2ex}{}_\Delta}=}
%  \newcommand{\defeq}{{\overstack\Delta =}}
  \newcommand{\feed}{\nonumber \\}
  \newcommand{\comma}{~,\quad}
  \newcommand{\period}{~.\quad}
  \newcommand{\del}{\partial}
%  \newcommand{\quabla}{\Delta}
  \newcommand{\point}{$\bullet~~$}
  \newcommand{\doubletilde}{ ~ \raisebox{0.3ex}{$\widetilde {}$} \raisebox{0.6ex}{$\widetilde {}$} \!\! }
  \newcommand{\topcirc}{\parbox{0ex}{~\raisebox{2.5ex}{${}^\circ$}}}
  \newcommand{\topdot} {\parbox{0ex}{~\raisebox{2.5ex}{$\cdot$}}}
  \newcommand{\topddot} {\parbox{0ex}{~\raisebox{1.3ex}{$\ddot{~}$}}}
  \newcommand{\sym}{\topcirc}
  \newcommand{\tsum}{\textstyle\sum}
  \newcommand{\st}{\quad\text{s.t.}\quad}

  \newcommand{\half}{\ensuremath{\frac{1}{2}}}
  \newcommand{\third}{\ensuremath{\frac{1}{3}}}
  \newcommand{\fourth}{\ensuremath{\frac{1}{4}}}

  \newcommand{\ubar}{\underline}
  %\renewcommand{\vec}{\underline}
  \renewcommand{\vec}{\boldsymbol}
  %\renewcommand{\_}{\underset}
  %\renewcommand{\^}{\overset}
  %\renewcommand{\*}{{\rm\raisebox{-.6ex}{\text{*}}{}}}
  \renewcommand{\*}{\text{\footnotesize\raisebox{-.4ex}{*}{}}}

  \newcommand{\gto}{{\raisebox{.5ex}{${}_\rightarrow$}}}
  \newcommand{\gfrom}{{\raisebox{.5ex}{${}_\leftarrow$}}}
  \newcommand{\gnto}{{\raisebox{.5ex}{${}_\nrightarrow$}}}
  \newcommand{\gnfrom}{{\raisebox{.5ex}{${}_\nleftarrow$}}}

  %\newcommand{\RND}{{\SS}}
  %\newcommand{\IF}{\text{if }}
  %\newcommand{\AND}{\textsc{and }}
  %\newcommand{\OR}{\textsc{or }}
  %\newcommand{\XOR}{\textsc{xor }}
  %\newcommand{\NOT}{\textsc{not }}

  %\newcommand{\argmax}[1]{{\rm arg}\!\max_{#1}}
  %\newcommand{\argmin}[1]{{\rm arg}\!\min_{#1}}
  \DeclareMathOperator*{\argmax}{argmax}
  \DeclareMathOperator*{\argmin}{argmin}
  \DeclareMathOperator{\sign}{sign}
  \DeclareMathOperator{\acos}{acos}
  \DeclareMathOperator{\unifies}{unifies}
  \DeclareMathOperator{\Span}{span}
  \newcommand{\ortho}{\perp}
  %\newcommand{\argmax}[1]{\underset{~#1}{\text{argmax}}\;}
  %\newcommand{\argmin}[1]{\underset{~#1}{\text{argmin}}\;}
  \newcommand{\ee}[1]{\ensuremath{\cdot10^{#1}}}
  \newcommand{\sub}[1]{\ensuremath{_{\text{#1}}}}
  \newcommand{\up}[1]{\ensuremath{^{\text{#1}}}}
  \newcommand{\kld}[3][{}]{D_{#1}\big(#2\,\big|\!\big|\,#3\big)}
  %\newcommand{\kld}[2]{D\big(#1:#2\big)}
  \newcommand{\sprod}[2]{\big<#1\,,\,#2\big>}
  \newcommand{\End}{\text{End}}
  \newcommand{\txt}[1]{\quad\text{#1}\quad}
  \newcommand{\Over}[2]{\genfrac{}{}{0pt}{0}{#1}{#2}}
  %\newcommand{\mat}[1]{{\bf #1}}
  \newcommand{\arr}[2]{\hspace*{-.5ex}\begin{array}{#1}#2\end{array}\hspace*{-.5ex}}
  \newcommand{\mat}[3][.9]{
    \renewcommand{\arraystretch}{#1}{\scriptscriptstyle{\left(
      \hspace*{-1ex}\begin{array}{#2}#3\end{array}\hspace*{-1ex}
    \right)}}\renewcommand{\arraystretch}{1.2}
  }
  \newcommand{\Mat}[3][.9]{
    \renewcommand{\arraystretch}{#1}{\scriptscriptstyle{\left[
      \hspace*{-1ex}\begin{array}{#2}#3\end{array}\hspace*{-1ex}
    \right]}}\renewcommand{\arraystretch}{1.2}
  }
  \newcommand{\case}[2][ll]{\left\{\arr{#1}{#2}\right.}
  \newcommand{\seq}[1]{\textsf{\<#1\>}}
  \newcommand{\seqq}[1]{\textsf{#1}}
  \newcommand{\floor}[1]{\lfloor#1\rfloor}
  \newcommand{\Exp}[2][]{\text{E}_{#1}\{#2\}}
  \newcommand{\Var}[2][]{\text{Var}_{#1}\{#2\}}
  \newcommand{\cov}[2][]{\text{cov}_{#1}\{#2\}}

%\newcommand{\Exp}[2]{\left\langle{#2}\right\rangle_{#1}}
  \newcommand{\ex}{\setminus}

  \providecommand{\href}[2]{{\color{blue}USE PDFLATEX!}}
  \providecommand{\url}[2]{\href{#1}{{\color{blue}#2}}}
%  \newcommand{\link}[1]{\href{{\protect #1}}{\texttt{\protect #1}}}
  \newcommand{\anchor}[2]{\begin{picture}(0,0)\put(#1){#2}\end{picture}}
  \newcommand{\pagebox}{\begin{picture}(0,0)\put(-3,-23){
    \textcolor[rgb]{.5,1,.5}{\framebox[\textwidth]{\rule[-\textheight]{0pt}{0pt}}}}
    \end{picture}}

  \newcommand{\hide}[1]{
    \begin{list}{}{\leftmargin0ex \rightmargin0ex \topsep0ex \parsep0ex}
       \helvetica{5}{1}{m}{n}
       \renewcommand{\section}{\par SECTION: }
       \renewcommand{\subsection}{\par SUBSECTION: }
       \item[$~~\blacktriangleright$]
       #1%$\blacktriangleleft~~$
       \message{^^JHIDE--Warning!^^J}
    \end{list}
  }
  %\newcommand{\hide}[1]{{\tt[hide:~}{\footnotesize\sf #1}{\tt]}\message{^^JHIDE--Warning!^^J}}
  \newcommand{\Hide}{\renewcommand{\hide}[1]{\message{^^JHIDE--Warning (hidden)!^^J}}}
  \newcommand{\HIDE}{\renewcommand{\hide}[1]{}}
  \newcommand{\fullhide}[1]{\message{^^JHIDE--Warning (hidden)!^^J}}
  \newcommand{\todo}[1]{{\tt[TODO: #1]}\message{^^JTODO--Warning: #1^^J}}
  \newcommand{\Todo}{\renewcommand{\todo}[1]{\message{^^JTODO--Warning (hidden)!^^J}}}
  %\renewcommand{\title}[1]{\renewcommand{\thetitle}{#1}}
  \newcommand{\myauthor}[1]{\author{#1}\newcommand{\theauthor}{#1}}%\@author}
  \newcommand{\mytitle}[1]{\title{#1}\newcommand{\thetitle}{#1}}%\@title}
  \newcommand{\header}{\begin{document}\mytitle\cleardefs}
  \newcommand{\contents}{{\tableofcontents}\renewcommand{\contents}{}}
  \newcommand{\footer}{\small\bibliography{marc,bibs}\end{document}}
  \newcommand{\widepaper}{\usepackage{geometry}\geometry{a4paper,hdivide={25mm,*,25mm},vdivide={25mm,*,25mm}}}
  \newcommand{\moviex}[2]{\movie[externalviewer]{#1}{#2}} %\pdflatex\usepackage{multimedia}
  \newcommand{\rbox}[1]{\fboxrule2mm\fcolorbox[rgb]{1,.85,.85}{1,.85,.85}{#1}}
  \newcommand{\mpage}[2]{{\begin{minipage}{#1\columnwidth}#2\end{minipage}}}
  \newcommand{\redbox}[2]{\fboxrule1mm\fcolorbox[rgb]{1,.7,.7}{1,.7,.7}{\begin{minipage}{#1\columnwidth}\center#2\end{minipage}}}
  \newcommand{\onecol}[2]{
    \begin{minipage}[c]{#1\columnwidth}#2\end{minipage}}
  \newcommand{\twocol}[5][0]{
    \begin{minipage}[c]{#2\columnwidth}#4\end{minipage}\hspace*{#1\columnwidth}%
    \begin{minipage}[c]{#3\columnwidth}#5\end{minipage}}
  \newcommand{\threecol}[7][0]{%
    \begin{minipage}[c]{#2\columnwidth}#5\end{minipage}\hspace*{#1\columnwidth}%
    \begin{minipage}[c]{#3\columnwidth}#6\end{minipage}\hspace*{#1\columnwidth}%
    \begin{minipage}[c]{#4\columnwidth}#7\end{minipage}}
  \newcommand{\threecoltext}[7][c]{
    \begin{minipage}[#1]{#2\textwidth}#5\end{minipage}%
    \begin{minipage}[#1]{#3\textwidth}#6\end{minipage}%
    \begin{minipage}[#1]{#4\textwidth}#7\end{minipage}}
  \newcommand{\threecoltop}[7][0]{%
   \begin{minipage}[t]{#2\columnwidth}#5\end{minipage}\hspace*{#1\columnwidth}%
   \begin{minipage}[t]{#3\columnwidth}#6\end{minipage}\hspace*{#1\columnwidth}%
   \begin{minipage}[t]{#4\columnwidth}#7\end{minipage}}
  \newcommand{\fourcol}[9][0]{%
   \begin{minipage}[c]{#2\columnwidth}#6\end{minipage}\hspace*{#1\columnwidth}%
   \begin{minipage}[c]{#3\columnwidth}#7\end{minipage}\hspace*{#1\columnwidth}%
   \begin{minipage}[c]{#4\columnwidth}#8\end{minipage}\hspace*{#1\columnwidth}%
   \begin{minipage}[c]{#5\columnwidth}#9\end{minipage}}
  \newcommand{\helvetica}[4]{\setlength{\unitlength}{1pt}\fontsize{#1}{#1}\linespread{#2}\usefont{OT1}{phv}{#3}{#4}}
  \newcommand{\helve}[1]{\helvetica{#1}{1.5}{m}{n}}
  \newcommand{\german}{\usepackage[german]{babel}\usepackage[utf8]{inputenc}}

\newcommand{\norm}[1]{|\!|#1|\!|}
\newcommand{\expr}[1]{[\hspace{-.2ex}[#1]\hspace{-.2ex}]}

\newcommand{\Jwi}{J^\sharp_W}
\newcommand{\THi}{T^\sharp_H}
\newcommand{\Jci}{J^\natural_C}
\newcommand{\hJi}{{\bar J}^\sharp}
\renewcommand{\|}{\,|\,}
\renewcommand{\=}{\!=\!}
\newcommand{\myminus}{{\hspace*{-.0pt}\text{\rm -}\hspace*{-.5pt}}}
\newcommand{\myplus}{{\hspace*{-.0pt}\text{\rm +}\hspace*{-.5pt}}}
\newcommand{\1}{{\myminus1}}
\newcommand{\2}{{\myminus2}}
\newcommand{\3}{{\myminus3}}
\newcommand{\mT}{{\text{\rm -}\hspace*{-1pt}\top}}
\newcommand{\po}{{\myplus1}}
\newcommand{\pt}{{\myplus2}}
%\renewcommand{\-}{\myminus}
%\newcommand{\+}{\myplus}
\renewcommand{\T}{{\!\top\!}}
\newcommand{\xT}{{\underline x}}
\newcommand{\uT}{{\underline u}}
\newcommand{\zT}{{\underline z}}
\newcommand{\Sum}{\textstyle\sum}
\newcommand{\Int}{\textstyle\int}
\newcommand{\Prod}{\textstyle\prod}


\newenvironment{centy}{
\vspace{15mm}
\large
\hspace*{5mm}
\begin{minipage}{8cm}\it\color{blue}
}{
\end{minipage}
}

\newcommand{\old}{{\text{old}}}
\newcommand{\new}{{\text{new}}}
\newcommand{\MAP}{{\text{MAP}}}
\newcommand{\ML}{{\text{ML}}}

\newcommand{\redArrow}{\quad\anchor{0,-1}{\includegraphics[scale=.5]{figs/redArrow}}}
\newcommand{\pub}[1]{{\color{green}\helvetica{8}{1.3}{m}{n}#1\\}}
\DeclareMathOperator{\opKL}{KL}
\newcommand{\KL}[2]{\opKL\big(#1\,\big|\!\big|\,#2\big)} %\left(#1 |\!| #2\right)}

\renewcommand{\show}[2][.8]{\centerline{\includegraphics[width=#1\columnwidth]{#2}}}
\newcommand{\showh}[2][.8]{\includegraphics[width=#1\columnwidth]{#2}}
\newcommand{\shows}[2][.8]{\centerline{\includegraphics[scale=#1]{#2}}}
\newcommand{\showhs}[2][.8]{\includegraphics[scale=#1]{#2}}
\newcommand{\mov}[2]{\movie[externalviewer]{{\color{blue}\small #1}}{movies/#2}}
\newcommand{\movex}[2]{\movie[externalviewer]{#1}{#2}} %\pdflatex\usepackage{multimedia}
%\newcommand{\movgb}[1]{\hfill\movie[externalviewer]{\small[movie]}{/home/mtoussai/movies/10-goalDirectedBehavior/#1}}
\newcommand{\movh}[3][loop]{
\movie[#1]{\showh[#2]{movies/#3.png}}{movies/#3.avi}%
\movie[externalviewer]{$\circ$}{movies/#3.avi}
}
\newcommand{\movc}[3][loop]{\centerline{\movh[#1]{#2}{#3}}}
\newcommand{\cen}[1]{\centerline{#1}}

\newcommand{\citing}[1]{
{\color{citcol}\tiny#1\par}
}

\newcommand{\cit}[3]{
\par\smallskip
{\color{greencol}\tiny #1: \emph{#2}. #3 \par}
}

\newcommand{\citurl}[4]{
\par\smallskip
{\color{greencol}\tiny #1: \protect{\href{#4}{\color{blue}{#2.}}} #3 \par}
}

\newcommand{\cito}[3]{
\par\smallskip
{\color{bluecol}\tiny #1: \emph{#2}. #3 \par}
}

\newcommand{\redoMacrosInProof}{
  \renewcommand{\d}{\delta}
%  \renewcommand{\|}{\,|\,}
  \renewcommand{\=}{\!=\!}
}

%% \makeatletter
%% \newenvironment{code}{%
%%   \begin{lrbox}{\@tempboxa}\begin{minipage}{1\columnwidth}\codefont
%% }{
%%   \end{minipage}\end{lrbox}%
%%   \colorbox[rgb]{.95,.95,.95}{\usebox{\@tempboxa}}
%% }\makeatother

\newenvironment{code}{%
\codefont
\begin{shaded}
}{
\end{shaded}
}

%\newcommand{\refeq}[1]{(\ref{#1})}

\usepackage{algorithm}
\usepackage{algpseudocode}
\algrenewcommand{\algorithmicrequire}{\textbf{Input:~~}}
\algrenewcommand{\algorithmicensure}{\textbf{Output:}}
\algrenewcommand{\algorithmiccomment}[1]{\qquad\hfill~\hspace*{-5ex}\textit{// #1}}
\algrenewcommand{\alglinenumber}[1]{\helvetica{6}{1.3}{m}{n}#1:}

\newenvironment{algo}[1][8]{
\quad\begin{minipage}{.8\columnwidth}\helvetica{#1}{1.3}{m}{n}
\medskip\hrule\medskip
\begin{algorithmic}[1]
}{
\end{algorithmic}
\medskip\hrule\medskip
\end{minipage}
}

\usepackage{etoolbox}

%%%%%%%%%%%%%%%%%%%%%%%%%%%%%%%%%%%%%%%%%%%%%%%%%%%%%%%%%%%%%%%%%%%%%%%%%%%%%%%%

\usepackage{multirow}
\usepackage{colortbl}
%\setlength{\jot}{0pt}
%\setlength{\mathindent}{1ex}
\usepackage{empheq}

%%%%%%%%%%%%%%%%%%%%%%%%%%%%%%%%%%%%%%%%%%%%%%%%%%%%%%%%%%%%%%%%%%%%%%%%%%%%%%%

\newcommand{\mypause}{\pause}
%\newcommand{\dom}{{\text{dom}}}
\newcommand{\defi}[1]{\textbf{#1}}
\newcommand{\red}[1]{\emph{\color{red}#1}}
%\newcommand{\ul}{\underline}
\newcommand{\pos}{{\textsf{pos}}}
\newcommand{\eff}{{\textsf{eff}}}
\newcommand{\rot}{{\textsf{rot}}}
\newcommand{\veC}{{\textsf{vec}}}
\newcommand{\quat}{{\textsf{quat}}}
\newcommand{\col}{{\textsf{col}}}
\newcommand{\de}[2]{\frac{\partial #1}{\partial #2}}
\newcommand{\target}{{\text{target}}}
\newcommand{\near}{{\text{near}}}
\newcommand{\qfree}{Q_{\text{free}}}
\renewcommand{\vec}{\boldsymbol}
\newcommand{\lft}{\text{left}}
\newcommand{\rgh}{\text{right}}
\DeclareMathOperator{\real}{real}
\newcommand{\prev}{{\text{prev}}}
\newcommand{\TR}[2]{T_{{#1}\to{#2}}}
\newcommand{\RO}[2]{R_{{#1}\to{#2}}}
\newcommand{\liter}{\helvetica{8}{1.1}{m}{n}\parskip 1ex}
\newcommand{\Fc}{\color{green}F}
\newcommand{\muc}{\color{blue}\mu}
\newcommand{\Astar}{A$^*$}

%for AI course:
\newcommand{\defn}[1]{\textbf{#1}}

%for optimization course:
\newcommand{\adec}{\r_\a^-}
\newcommand{\ainc}{\r_\a^+}
\newcommand{\ldec}{\r_\l^-}
\newcommand{\linc}{\r_\l^+}
\newcommand{\minc}{\r_\m^+}
\newcommand{\mdec}{\r_\m^-}
\newcommand{\lsstop}{\r_{\text{ls}}}


\definecolor{boxcol}{rgb}{.85,.9,.92}
\newcommand{\eqbox}[1]{\centerline{\fboxrule0mm\fcolorbox{boxcol}{boxcol}{#1}}}
\newcommand{\movgb}[1]{\hfill\movie[externalviewer]{\small[movie]}{/home/mtoussai/movies/10-goalDirectedBehavior/#1}}
\newcommand{\demo}[1]{{{\color{blue}[\small #1]}}}

\graphicspath{{../pics-robotics/}{../pics-ML/}{../pics-all/}{../pics-all2/}{../pics-Optim/}}
\DeclareGraphicsExtensions{.pdf,.png,.jpg}

%\usepackage{pdfpages}
%\setbeamercolor{background canvas}{bg=}

\newcommand{\SUM}{\texttt{sum}}
\usepackage{float}

%% prevent pagebreaks before environment
\makeatletter
\newcommand{\NewParNoBreak}[1][\parskip]{\par\vspace*{-\parskip}\vspace*{#1}\nobreak\@afterheading}
\makeatother

%\newcommand{\idx}[2]{\label{IKgn}}

%%%%%%%%%%%%%%%%%%%%%%%%%%%%%%%%%%%%%%%%%%%%%%%%%%%%%%%%%%%%%%%%%%%%%%%%%%%%%%%%



%% \newwrite\tempfile
%% \immediate\openout\tempfile=z.keys.tex

%% \renewcommand{\key}[1]{
%% %%   \addtocounter{mypage}{1}
%% \makeatletter
%% \immediate\write\tempfile{\symbol{`\\}}
%% \makeatother
%%   \immediate\write\tempfile{hyperref[key:#1]{#1(\arabic{mypage})}}
%% %%  % \phantomsection\label{key:#1}
%% %%   %\index{#1@{\hyperref[key:#1]{#1 (\arabic{mysec}:\arabic{mypage})}}|phantom}
%% %%   \addtocounter{mypage}{-1}
%% }


  \DefineShortVerb{\@}

  \newcounter{solutions}
  \setcounter{solutions}{1}
  \newenvironment{solution}{
    \small
    \begin{shaded}
  }{
    \end{shaded}
  }
  
  \renewcommand{\hat}{\widehat}
  \newcommand{\bbg}{{\bar{\bar g}}}
  \graphicspath{{pics/}{../shared/pics/}}

  \renewcommand{\labelenumi}{{\alph{enumi})}}

  %%%%%%%%%%%%%%%%%%%%%%%%%%%%%%%%%%%%%%%%%%%%%%%%%%%%%%%%%%%%%%%%%%%%%%%%%%%%%%%%


  \mytitle{\course\\Exercise \exnum}
  \myauthor{Marc Toussaint\\ \small\addressUSTT}
  \date{\coursedate}
  
  \begin{document}
  \onecolumn
  \maketitle
}

\newcommand{\exsection}[1]{\section{#1}}

\newcommand{\exerfoot}{
  \end{document}
}

\newenvironment{items}[1][9]{
  \par\setlength{\unitlength}{1pt}\fontsize{#1}{#1}\linespread{1.2}\selectfont
  \begin{list}{--}{\leftmargin4ex \rightmargin0ex \labelsep1ex \labelwidth2ex
      \topsep0pt \parsep0ex \itemsep3pt}
}{
  \end{list}
}

  \exerciseshead
}

\providecommand{\script}{
  \newcommand{\scripthead}{
  \documentclass[9pt,twoside]{article}
  \stdpackages

  \usepackage{palatino}
  \usepackage[envcountsect]{beamerarticle}
  \usepackage{makeidx}
  \makeindex

  \definecolor{bluecol}{rgb}{0,0,.5}
  \definecolor{greencol}{rgb}{0,.4,0}
  \definecolor{shadecolor}{gray}{0.9}
  \usepackage[
    %    pdftex%,
    %%    letterpaper,
    %bookmarks,
    bookmarksnumbered,
    colorlinks,
    urlcolor=bluecol,
    citecolor=black,
    linkcolor=bluecol,
    %    pagecolor=bluecol,
    pdfborder={0 0 0},
    %pdfborderstyle={/S/U/W 1},
    %%    backref,     %link from bibliography back to sections
    %%    pagebackref, %link from bibliography back to pages
    %%    pdfstartview=FitH, %fitwidth instead of fit window
    pdfpagemode=UseOutlines, %bookmarks are displayed by acrobat
    pdftitle={\course},
    pdfauthor={Marc Toussaint},
    pdfkeywords={}
  ]{hyperref}
  \DeclareGraphicsExtensions{.pdf,.png,.jpg,.eps}

  \usepackage{multimedia}
  %\setbeamercolor{background canvas}{bg=}

  \renewcommand{\r}{\varrho}
  \renewcommand{\l}{\lambda}
  \renewcommand{\L}{\Lambda}
  \renewcommand{\b}{\beta}
  \renewcommand{\d}{\delta}
  \renewcommand{\k}{\kappa}
  \renewcommand{\t}{\theta}
  \renewcommand{\O}{\Omega}
  \renewcommand{\o}{\omega}
  \renewcommand{\SS}{{\cal S}}
  \renewcommand{\=}{\!=\!}
  %\renewcommand{\boldsymbol}{}
  %\renewcommand{\Chapter}{\chapter}
  %\renewcommand{\Subsection}{\subsection}

  \renewcommand{\baselinestretch}{1.0}
  \geometry{a5paper,headsep=6mm,hdivide={10mm,*,10mm},vdivide={13mm,*,7mm}}

  \fancyhead[OL,ER]{\course, \textit{Marc Toussaint}}
  \fancyhead[OR,EL]{\thepage}
  \fancyhead[C]{}
  \fancyfoot{}
  \pagestyle{fancy}

%  \setcounter{tocdepth}{3}
  \setcounter{tocdepth}{2}

   \columnsep 6ex
  %  \renewcommand{\familydefault}{\sfdefault}
  \newcommand{\headerfont}{\large}%helvetica{12}{1}{b}{n}}
  \newcommand{\slidefont} {}%\helvetica{9}{1.3}{m}{n}}
  \newcommand{\storyfont} {}
  %  \renewcommand{\small}   {\helvetica{8}{1.2}{m}{n}}
  \renewcommand{\tiny}    {\footnotesize}%helvetica{7}{1.1}{m}{n}}
  \newcommand{\ttiny} {\fontsize{5}{5}\selectfont}
  \newcommand{\codefont}{\fontsize{6}{6}\selectfont}%helvetica{8}{1.2}{m}{n}}

  %auto-ignore
  \renewcommand{\a}{\alpha}
  \renewcommand{\b}{\beta}
  \renewcommand{\d}{\delta}
    \newcommand{\D}{\Delta}
    \newcommand{\e}{\epsilon}
    \newcommand{\g}{\gamma}
    \newcommand{\G}{\Gamma}
  \renewcommand{\l}{\lambda}
  \renewcommand{\L}{\Lambda}
    \newcommand{\m}{\mu}
    \newcommand{\n}{\nu}
    \newcommand{\N}{\nabla}
  \renewcommand{\k}{\kappa}
  \renewcommand{\o}{\omega}
  \renewcommand{\O}{\Omega}
    \newcommand{\p}{\phi}
    \newcommand{\ph}{\varphi}
  \renewcommand{\P}{\Phi}
  \renewcommand{\r}{\varrho}
    \newcommand{\s}{\sigma}
  \renewcommand{\S}{\Sigma}
  \renewcommand{\t}{\theta}
    \newcommand{\T}{\Theta}
  %\renewcommand{\v}{\vartheta}
    \newcommand{\x}{\xi}
    \newcommand{\X}{\Xi}
    \newcommand{\Y}{\Upsilon}
    \newcommand{\z}{\zeta}

  \renewcommand{\AA}{{\cal A}}
    \newcommand{\BB}{{\cal B}}
    \newcommand{\CC}{{\cal C}}
    \newcommand{\cc}{{\cal c}}
    \newcommand{\DD}{{\cal D}}
    \newcommand{\EE}{{\cal E}}
    \newcommand{\FF}{{\cal F}}
    \newcommand{\GG}{{\cal G}}
    \newcommand{\HH}{{\cal H}}
    \newcommand{\II}{{\cal I}}
    \newcommand{\KK}{{\cal K}}
    \newcommand{\LL}{{\cal L}}
    \newcommand{\MM}{{\cal M}}
    \newcommand{\NN}{{\cal N}}
    \newcommand{\oNN}{\overline\NN}
    \newcommand{\OO}{{\cal O}}
    \newcommand{\PP}{{\cal P}}
    \newcommand{\QQ}{{\cal Q}}
    \newcommand{\RR}{{\cal R}}
  \renewcommand{\SS}{{\cal S}}
    \newcommand{\TT}{{\cal T}}
    \newcommand{\uu}{{\cal u}}
    \newcommand{\UU}{{\cal U}}
    \newcommand{\VV}{{\cal V}}
    \newcommand{\XX}{{\cal X}}
    \newcommand{\xx}{\mathcal{x}}
    \newcommand{\YY}{{\cal Y}}
    \newcommand{\SOSO}{{\cal SO}}
    \newcommand{\GLGL}{{\cal GL}}

    \newcommand{\Ee}{{\rm E}}

  \newcommand{\NNN}{{\mathbb{N}}}
  \newcommand{\III}{{\mathbb{I}}}
  \newcommand{\ZZZ}{{\mathbb{Z}}}
  %\newcommand{\RRR}{{\mathrm{I\!R}}}
  \newcommand{\RRR}{{\mathbb{R}}}
  \newcommand{\SSS}{{\mathbb{S}}}
  \newcommand{\CCC}{{\mathbb{C}}}
  \newcommand{\DDD}{{\mathbb{D}}}
  \newcommand{\one}{{{\bf 1}}}
  \newcommand{\eee}{\text{e}}

  \newcommand{\NNNN}{{\overline{\cal N}}}

  \renewcommand{\[}{\Big[}
  \renewcommand{\]}{\Big]}
  \renewcommand{\(}{\Big(}
  \renewcommand{\)}{\Big)}
  \renewcommand{\|}{\,|\,}
  \renewcommand{\;}{\,;\,}
  \renewcommand{\=}{\!=\!}
    \newcommand{\<}{\left\langle}
  \renewcommand{\>}{\right\rangle}

  \newcommand{\na}[1][]{{\nabla_{\!\!#1}}}
  \newcommand{\he}[1][]{{\nabla_{\!\!#1}^2}}
  \newcommand{\Prob}{{\rm Prob}}
  \newcommand{\Dir}{{\rm Dir}}
  \newcommand{\Beta}{{\rm Beta}}
  \newcommand{\Bern}{{\rm Bern}}
  \newcommand{\Bin}{{\rm Bin}}
  \newcommand{\Mult}{{\rm Mult}}
  \newcommand{\Aut}{{\rm Aut}}
  \newcommand{\cor}{{\rm cor}}
  \newcommand{\corr}{{\rm corr}}
  \newcommand{\sd}{{\rm sd}}
  \newcommand{\tr}{{\rm tr}}
  \newcommand{\Tr}{{\rm Tr}}
  \newcommand{\rank}{{\rm rank}}
  \newcommand{\diag}{{\rm diag}}
  \newcommand{\dom}{{\rm dom}}
  \newcommand{\id}{{\rm id}}
  \newcommand{\Id}{{\rm\bf I}}
  \newcommand{\Gl}{{\rm Gl}}
  \renewcommand{\th}{\ensuremath{{}^\text{th}} }
  \newcommand{\lag}{\mathcal{L}}
  \newcommand{\inn}{\rfloor}
  \newcommand{\lie}{\pounds}
  \newcommand{\longto}{\longrightarrow}
  \newcommand{\speer}{\parbox{0.4ex}{\raisebox{0.8ex}{$\nearrow$}}}
  \renewcommand{\dag}{ {}^\dagger }
  \newcommand{\blbox}{\rule{1ex}{1ex}}
  \newcommand{\Ji}{J^\sharp}
  \newcommand{\h}{{}^\star}
  \newcommand{\w}{\wedge}
  \newcommand{\too}{\longrightarrow}
  \newcommand{\oot}{\longleftarrow}
  \newcommand{\To}{\Rightarrow}
  \newcommand{\oT}{\Leftarrow}
  \newcommand{\oTo}{\Leftrightarrow}
  \renewcommand{\iff}{~\Longleftrightarrow~}
  \newcommand{\Too}{\;\Longrightarrow\;}
  \newcommand{\oto}{\leftrightarrow}
  \newcommand{\ot}{\leftarrow}
  \newcommand{\ootoo}{\longleftrightarrow}
  \newcommand{\ow}{\stackrel{\circ}\wedge}
  \newcommand{\defeq}{\stackrel{\hspace{0.2ex}{}_\Delta}=}
%  \newcommand{\defeq}{{\overstack\Delta =}}
  \newcommand{\feed}{\nonumber \\}
  \newcommand{\comma}{~,\quad}
  \newcommand{\period}{~.\quad}
  \newcommand{\del}{\partial}
%  \newcommand{\quabla}{\Delta}
  \newcommand{\point}{$\bullet~~$}
  \newcommand{\doubletilde}{ ~ \raisebox{0.3ex}{$\widetilde {}$} \raisebox{0.6ex}{$\widetilde {}$} \!\! }
  \newcommand{\topcirc}{\parbox{0ex}{~\raisebox{2.5ex}{${}^\circ$}}}
  \newcommand{\topdot} {\parbox{0ex}{~\raisebox{2.5ex}{$\cdot$}}}
  \newcommand{\topddot} {\parbox{0ex}{~\raisebox{1.3ex}{$\ddot{~}$}}}
  \newcommand{\sym}{\topcirc}
  \newcommand{\tsum}{\textstyle\sum}
  \newcommand{\st}{\quad\text{s.t.}\quad}

  \newcommand{\half}{\ensuremath{\frac{1}{2}}}
  \newcommand{\third}{\ensuremath{\frac{1}{3}}}
  \newcommand{\fourth}{\ensuremath{\frac{1}{4}}}

  \newcommand{\ubar}{\underline}
  %\renewcommand{\vec}{\underline}
  \renewcommand{\vec}{\boldsymbol}
  %\renewcommand{\_}{\underset}
  %\renewcommand{\^}{\overset}
  %\renewcommand{\*}{{\rm\raisebox{-.6ex}{\text{*}}{}}}
  \renewcommand{\*}{\text{\footnotesize\raisebox{-.4ex}{*}{}}}

  \newcommand{\gto}{{\raisebox{.5ex}{${}_\rightarrow$}}}
  \newcommand{\gfrom}{{\raisebox{.5ex}{${}_\leftarrow$}}}
  \newcommand{\gnto}{{\raisebox{.5ex}{${}_\nrightarrow$}}}
  \newcommand{\gnfrom}{{\raisebox{.5ex}{${}_\nleftarrow$}}}

  %\newcommand{\RND}{{\SS}}
  %\newcommand{\IF}{\text{if }}
  %\newcommand{\AND}{\textsc{and }}
  %\newcommand{\OR}{\textsc{or }}
  %\newcommand{\XOR}{\textsc{xor }}
  %\newcommand{\NOT}{\textsc{not }}

  %\newcommand{\argmax}[1]{{\rm arg}\!\max_{#1}}
  %\newcommand{\argmin}[1]{{\rm arg}\!\min_{#1}}
  \DeclareMathOperator*{\argmax}{argmax}
  \DeclareMathOperator*{\argmin}{argmin}
  \DeclareMathOperator{\sign}{sign}
  \DeclareMathOperator{\acos}{acos}
  \DeclareMathOperator{\unifies}{unifies}
  \DeclareMathOperator{\Span}{span}
  \newcommand{\ortho}{\perp}
  %\newcommand{\argmax}[1]{\underset{~#1}{\text{argmax}}\;}
  %\newcommand{\argmin}[1]{\underset{~#1}{\text{argmin}}\;}
  \newcommand{\ee}[1]{\ensuremath{\cdot10^{#1}}}
  \newcommand{\sub}[1]{\ensuremath{_{\text{#1}}}}
  \newcommand{\up}[1]{\ensuremath{^{\text{#1}}}}
  \newcommand{\kld}[3][{}]{D_{#1}\big(#2\,\big|\!\big|\,#3\big)}
  %\newcommand{\kld}[2]{D\big(#1:#2\big)}
  \newcommand{\sprod}[2]{\big<#1\,,\,#2\big>}
  \newcommand{\End}{\text{End}}
  \newcommand{\txt}[1]{\quad\text{#1}\quad}
  \newcommand{\Over}[2]{\genfrac{}{}{0pt}{0}{#1}{#2}}
  %\newcommand{\mat}[1]{{\bf #1}}
  \newcommand{\arr}[2]{\hspace*{-.5ex}\begin{array}{#1}#2\end{array}\hspace*{-.5ex}}
  \newcommand{\mat}[3][.9]{
    \renewcommand{\arraystretch}{#1}{\scriptscriptstyle{\left(
      \hspace*{-1ex}\begin{array}{#2}#3\end{array}\hspace*{-1ex}
    \right)}}\renewcommand{\arraystretch}{1.2}
  }
  \newcommand{\Mat}[3][.9]{
    \renewcommand{\arraystretch}{#1}{\scriptscriptstyle{\left[
      \hspace*{-1ex}\begin{array}{#2}#3\end{array}\hspace*{-1ex}
    \right]}}\renewcommand{\arraystretch}{1.2}
  }
  \newcommand{\case}[2][ll]{\left\{\arr{#1}{#2}\right.}
  \newcommand{\seq}[1]{\textsf{\<#1\>}}
  \newcommand{\seqq}[1]{\textsf{#1}}
  \newcommand{\floor}[1]{\lfloor#1\rfloor}
  \newcommand{\Exp}[2][]{\text{E}_{#1}\{#2\}}
  \newcommand{\Var}[2][]{\text{Var}_{#1}\{#2\}}
  \newcommand{\cov}[2][]{\text{cov}_{#1}\{#2\}}

%\newcommand{\Exp}[2]{\left\langle{#2}\right\rangle_{#1}}
  \newcommand{\ex}{\setminus}

  \providecommand{\href}[2]{{\color{blue}USE PDFLATEX!}}
  \providecommand{\url}[2]{\href{#1}{{\color{blue}#2}}}
%  \newcommand{\link}[1]{\href{{\protect #1}}{\texttt{\protect #1}}}
  \newcommand{\anchor}[2]{\begin{picture}(0,0)\put(#1){#2}\end{picture}}
  \newcommand{\pagebox}{\begin{picture}(0,0)\put(-3,-23){
    \textcolor[rgb]{.5,1,.5}{\framebox[\textwidth]{\rule[-\textheight]{0pt}{0pt}}}}
    \end{picture}}

  \newcommand{\hide}[1]{
    \begin{list}{}{\leftmargin0ex \rightmargin0ex \topsep0ex \parsep0ex}
       \helvetica{5}{1}{m}{n}
       \renewcommand{\section}{\par SECTION: }
       \renewcommand{\subsection}{\par SUBSECTION: }
       \item[$~~\blacktriangleright$]
       #1%$\blacktriangleleft~~$
       \message{^^JHIDE--Warning!^^J}
    \end{list}
  }
  %\newcommand{\hide}[1]{{\tt[hide:~}{\footnotesize\sf #1}{\tt]}\message{^^JHIDE--Warning!^^J}}
  \newcommand{\Hide}{\renewcommand{\hide}[1]{\message{^^JHIDE--Warning (hidden)!^^J}}}
  \newcommand{\HIDE}{\renewcommand{\hide}[1]{}}
  \newcommand{\fullhide}[1]{\message{^^JHIDE--Warning (hidden)!^^J}}
  \newcommand{\todo}[1]{{\tt[TODO: #1]}\message{^^JTODO--Warning: #1^^J}}
  \newcommand{\Todo}{\renewcommand{\todo}[1]{\message{^^JTODO--Warning (hidden)!^^J}}}
  %\renewcommand{\title}[1]{\renewcommand{\thetitle}{#1}}
  \newcommand{\myauthor}[1]{\author{#1}\newcommand{\theauthor}{#1}}%\@author}
  \newcommand{\mytitle}[1]{\title{#1}\newcommand{\thetitle}{#1}}%\@title}
  \newcommand{\header}{\begin{document}\mytitle\cleardefs}
  \newcommand{\contents}{{\tableofcontents}\renewcommand{\contents}{}}
  \newcommand{\footer}{\small\bibliography{marc,bibs}\end{document}}
  \newcommand{\widepaper}{\usepackage{geometry}\geometry{a4paper,hdivide={25mm,*,25mm},vdivide={25mm,*,25mm}}}
  \newcommand{\moviex}[2]{\movie[externalviewer]{#1}{#2}} %\pdflatex\usepackage{multimedia}
  \newcommand{\rbox}[1]{\fboxrule2mm\fcolorbox[rgb]{1,.85,.85}{1,.85,.85}{#1}}
  \newcommand{\mpage}[2]{{\begin{minipage}{#1\columnwidth}#2\end{minipage}}}
  \newcommand{\redbox}[2]{\fboxrule1mm\fcolorbox[rgb]{1,.7,.7}{1,.7,.7}{\begin{minipage}{#1\columnwidth}\center#2\end{minipage}}}
  \newcommand{\onecol}[2]{
    \begin{minipage}[c]{#1\columnwidth}#2\end{minipage}}
  \newcommand{\twocol}[5][0]{
    \begin{minipage}[c]{#2\columnwidth}#4\end{minipage}\hspace*{#1\columnwidth}%
    \begin{minipage}[c]{#3\columnwidth}#5\end{minipage}}
  \newcommand{\threecol}[7][0]{%
    \begin{minipage}[c]{#2\columnwidth}#5\end{minipage}\hspace*{#1\columnwidth}%
    \begin{minipage}[c]{#3\columnwidth}#6\end{minipage}\hspace*{#1\columnwidth}%
    \begin{minipage}[c]{#4\columnwidth}#7\end{minipage}}
  \newcommand{\threecoltext}[7][c]{
    \begin{minipage}[#1]{#2\textwidth}#5\end{minipage}%
    \begin{minipage}[#1]{#3\textwidth}#6\end{minipage}%
    \begin{minipage}[#1]{#4\textwidth}#7\end{minipage}}
  \newcommand{\threecoltop}[7][0]{%
   \begin{minipage}[t]{#2\columnwidth}#5\end{minipage}\hspace*{#1\columnwidth}%
   \begin{minipage}[t]{#3\columnwidth}#6\end{minipage}\hspace*{#1\columnwidth}%
   \begin{minipage}[t]{#4\columnwidth}#7\end{minipage}}
  \newcommand{\fourcol}[9][0]{%
   \begin{minipage}[c]{#2\columnwidth}#6\end{minipage}\hspace*{#1\columnwidth}%
   \begin{minipage}[c]{#3\columnwidth}#7\end{minipage}\hspace*{#1\columnwidth}%
   \begin{minipage}[c]{#4\columnwidth}#8\end{minipage}\hspace*{#1\columnwidth}%
   \begin{minipage}[c]{#5\columnwidth}#9\end{minipage}}
  \newcommand{\helvetica}[4]{\setlength{\unitlength}{1pt}\fontsize{#1}{#1}\linespread{#2}\usefont{OT1}{phv}{#3}{#4}}
  \newcommand{\helve}[1]{\helvetica{#1}{1.5}{m}{n}}
  \newcommand{\german}{\usepackage[german]{babel}\usepackage[utf8]{inputenc}}

\newcommand{\norm}[1]{|\!|#1|\!|}
\newcommand{\expr}[1]{[\hspace{-.2ex}[#1]\hspace{-.2ex}]}

\newcommand{\Jwi}{J^\sharp_W}
\newcommand{\THi}{T^\sharp_H}
\newcommand{\Jci}{J^\natural_C}
\newcommand{\hJi}{{\bar J}^\sharp}
\renewcommand{\|}{\,|\,}
\renewcommand{\=}{\!=\!}
\newcommand{\myminus}{{\hspace*{-.0pt}\text{\rm -}\hspace*{-.5pt}}}
\newcommand{\myplus}{{\hspace*{-.0pt}\text{\rm +}\hspace*{-.5pt}}}
\newcommand{\1}{{\myminus1}}
\newcommand{\2}{{\myminus2}}
\newcommand{\3}{{\myminus3}}
\newcommand{\mT}{{\text{\rm -}\hspace*{-1pt}\top}}
\newcommand{\po}{{\myplus1}}
\newcommand{\pt}{{\myplus2}}
%\renewcommand{\-}{\myminus}
%\newcommand{\+}{\myplus}
\renewcommand{\T}{{\!\top\!}}
\newcommand{\xT}{{\underline x}}
\newcommand{\uT}{{\underline u}}
\newcommand{\zT}{{\underline z}}
\newcommand{\Sum}{\textstyle\sum}
\newcommand{\Int}{\textstyle\int}
\newcommand{\Prod}{\textstyle\prod}


\newenvironment{centy}{
\vspace{15mm}
\large
\hspace*{5mm}
\begin{minipage}{8cm}\it\color{blue}
}{
\end{minipage}
}

\newcommand{\old}{{\text{old}}}
\newcommand{\new}{{\text{new}}}
\newcommand{\MAP}{{\text{MAP}}}
\newcommand{\ML}{{\text{ML}}}

\newcommand{\redArrow}{\quad\anchor{0,-1}{\includegraphics[scale=.5]{figs/redArrow}}}
\newcommand{\pub}[1]{{\color{green}\helvetica{8}{1.3}{m}{n}#1\\}}
\DeclareMathOperator{\opKL}{KL}
\newcommand{\KL}[2]{\opKL\big(#1\,\big|\!\big|\,#2\big)} %\left(#1 |\!| #2\right)}

\renewcommand{\show}[2][.8]{\centerline{\includegraphics[width=#1\columnwidth]{#2}}}
\newcommand{\showh}[2][.8]{\includegraphics[width=#1\columnwidth]{#2}}
\newcommand{\shows}[2][.8]{\centerline{\includegraphics[scale=#1]{#2}}}
\newcommand{\showhs}[2][.8]{\includegraphics[scale=#1]{#2}}
\newcommand{\mov}[2]{\movie[externalviewer]{{\color{blue}\small #1}}{movies/#2}}
\newcommand{\movex}[2]{\movie[externalviewer]{#1}{#2}} %\pdflatex\usepackage{multimedia}
%\newcommand{\movgb}[1]{\hfill\movie[externalviewer]{\small[movie]}{/home/mtoussai/movies/10-goalDirectedBehavior/#1}}
\newcommand{\movh}[3][loop]{
\movie[#1]{\showh[#2]{movies/#3.png}}{movies/#3.avi}%
\movie[externalviewer]{$\circ$}{movies/#3.avi}
}
\newcommand{\movc}[3][loop]{\centerline{\movh[#1]{#2}{#3}}}
\newcommand{\cen}[1]{\centerline{#1}}

\newcommand{\citing}[1]{
{\color{citcol}\tiny#1\par}
}

\newcommand{\cit}[3]{
\par\smallskip
{\color{greencol}\tiny #1: \emph{#2}. #3 \par}
}

\newcommand{\citurl}[4]{
\par\smallskip
{\color{greencol}\tiny #1: \protect{\href{#4}{\color{blue}{#2.}}} #3 \par}
}

\newcommand{\cito}[3]{
\par\smallskip
{\color{bluecol}\tiny #1: \emph{#2}. #3 \par}
}

\newcommand{\redoMacrosInProof}{
  \renewcommand{\d}{\delta}
%  \renewcommand{\|}{\,|\,}
  \renewcommand{\=}{\!=\!}
}

%% \makeatletter
%% \newenvironment{code}{%
%%   \begin{lrbox}{\@tempboxa}\begin{minipage}{1\columnwidth}\codefont
%% }{
%%   \end{minipage}\end{lrbox}%
%%   \colorbox[rgb]{.95,.95,.95}{\usebox{\@tempboxa}}
%% }\makeatother

\newenvironment{code}{%
\codefont
\begin{shaded}
}{
\end{shaded}
}

%\newcommand{\refeq}[1]{(\ref{#1})}

\usepackage{algorithm}
\usepackage{algpseudocode}
\algrenewcommand{\algorithmicrequire}{\textbf{Input:~~}}
\algrenewcommand{\algorithmicensure}{\textbf{Output:}}
\algrenewcommand{\algorithmiccomment}[1]{\qquad\hfill~\hspace*{-5ex}\textit{// #1}}
\algrenewcommand{\alglinenumber}[1]{\helvetica{6}{1.3}{m}{n}#1:}

\newenvironment{algo}[1][8]{
\quad\begin{minipage}{.8\columnwidth}\helvetica{#1}{1.3}{m}{n}
\medskip\hrule\medskip
\begin{algorithmic}[1]
}{
\end{algorithmic}
\medskip\hrule\medskip
\end{minipage}
}

\usepackage{etoolbox}

%%%%%%%%%%%%%%%%%%%%%%%%%%%%%%%%%%%%%%%%%%%%%%%%%%%%%%%%%%%%%%%%%%%%%%%%%%%%%%%%

\usepackage{multirow}
\usepackage{colortbl}
%\setlength{\jot}{0pt}
%\setlength{\mathindent}{1ex}
\usepackage{empheq}

%%%%%%%%%%%%%%%%%%%%%%%%%%%%%%%%%%%%%%%%%%%%%%%%%%%%%%%%%%%%%%%%%%%%%%%%%%%%%%%

\newcommand{\mypause}{\pause}
%\newcommand{\dom}{{\text{dom}}}
\newcommand{\defi}[1]{\textbf{#1}}
\newcommand{\red}[1]{\emph{\color{red}#1}}
%\newcommand{\ul}{\underline}
\newcommand{\pos}{{\textsf{pos}}}
\newcommand{\eff}{{\textsf{eff}}}
\newcommand{\rot}{{\textsf{rot}}}
\newcommand{\veC}{{\textsf{vec}}}
\newcommand{\quat}{{\textsf{quat}}}
\newcommand{\col}{{\textsf{col}}}
\newcommand{\de}[2]{\frac{\partial #1}{\partial #2}}
\newcommand{\target}{{\text{target}}}
\newcommand{\near}{{\text{near}}}
\newcommand{\qfree}{Q_{\text{free}}}
\renewcommand{\vec}{\boldsymbol}
\newcommand{\lft}{\text{left}}
\newcommand{\rgh}{\text{right}}
\DeclareMathOperator{\real}{real}
\newcommand{\prev}{{\text{prev}}}
\newcommand{\TR}[2]{T_{{#1}\to{#2}}}
\newcommand{\RO}[2]{R_{{#1}\to{#2}}}
\newcommand{\liter}{\helvetica{8}{1.1}{m}{n}\parskip 1ex}
\newcommand{\Fc}{\color{green}F}
\newcommand{\muc}{\color{blue}\mu}
\newcommand{\Astar}{A$^*$}

%for AI course:
\newcommand{\defn}[1]{\textbf{#1}}

%for optimization course:
\newcommand{\adec}{\r_\a^-}
\newcommand{\ainc}{\r_\a^+}
\newcommand{\ldec}{\r_\l^-}
\newcommand{\linc}{\r_\l^+}
\newcommand{\minc}{\r_\m^+}
\newcommand{\mdec}{\r_\m^-}
\newcommand{\lsstop}{\r_{\text{ls}}}


\definecolor{boxcol}{rgb}{.85,.9,.92}
\newcommand{\eqbox}[1]{\centerline{\fboxrule0mm\fcolorbox{boxcol}{boxcol}{#1}}}
\newcommand{\movgb}[1]{\hfill\movie[externalviewer]{\small[movie]}{/home/mtoussai/movies/10-goalDirectedBehavior/#1}}
\newcommand{\demo}[1]{{{\color{blue}[\small #1]}}}

\graphicspath{{../pics-robotics/}{../pics-ML/}{../pics-all/}{../pics-all2/}{../pics-Optim/}}
\DeclareGraphicsExtensions{.pdf,.png,.jpg}

%\usepackage{pdfpages}
%\setbeamercolor{background canvas}{bg=}

\newcommand{\SUM}{\texttt{sum}}
\usepackage{float}

%% prevent pagebreaks before environment
\makeatletter
\newcommand{\NewParNoBreak}[1][\parskip]{\par\vspace*{-\parskip}\vspace*{#1}\nobreak\@afterheading}
\makeatother

%\newcommand{\idx}[2]{\label{IKgn}}

%%%%%%%%%%%%%%%%%%%%%%%%%%%%%%%%%%%%%%%%%%%%%%%%%%%%%%%%%%%%%%%%%%%%%%%%%%%%%%%%



%% \newwrite\tempfile
%% \immediate\openout\tempfile=z.keys.tex

%% \renewcommand{\key}[1]{
%% %%   \addtocounter{mypage}{1}
%% \makeatletter
%% \immediate\write\tempfile{\symbol{`\\}}
%% \makeatother
%%   \immediate\write\tempfile{hyperref[key:#1]{#1(\arabic{mypage})}}
%% %%  % \phantomsection\label{key:#1}
%% %%   %\index{#1@{\hyperref[key:#1]{#1 (\arabic{mysec}:\arabic{mypage})}}|phantom}
%% %%   \addtocounter{mypage}{-1}
%% }


  \DefineShortVerb{\@}

  \newcounter{solutions}
  \setcounter{solutions}{1}
  \renewenvironment{solution}{
    \small
    \begin{shaded}
  }{
    \end{shaded}
  }

  \graphicspath{{pics/}{../shared/pics/}}

%%%%%%%%%%%%%%%%%%%%%%%%%%%%%%%%%%%%%%%%%%%%%%%%%%%%%%%%%%%%%%%%%%%%%%%%%%%%%%%%

  \mytitle{\course\\Lecture Script}
  \myauthor{Marc Toussaint}
  \date{\coursedate}

  \begin{document}

  %% \vspace*{2cm}

  \maketitle
  %\anchor{100,10}{\includegraphics[width=4cm]{optim}}

%  \vspace*{1cm}

  \emph{This is a direct concatenation and reformatting of all lecture
    slides and exercises from this course, including indexing to help
    prepare for exams.}

  {\tableofcontents}
}

%%%%%%%%%%%%%%%%%%%%%%%%%%%%%%%%%%%%%%%%%%%%%%%%%%%%%%%%%%%%%%%%%%%%%%%%%%%%%%%%

%% \renewcommand{\keywords}{}
%% \newcommand{\topic}{}
%% \renewcommand{\mypause}{}

  \newcounter{mypage}
  \setcounter{mypage}{0}
  \newcounter{mysec}
  \setcounter{mysec}{0}
  \newcommand{\incpage}{\addtocounter{mypage}{1}}
  \newcommand{\incsec} {\addtocounter{mysec}{1}}

\newcommand{\beginTocMinipage}{
  \addtocontents{toc}{\smallskip\noindent\hspace*{.036\columnwidth}}
  \addtocontents{toc}{\protect\begin{minipage}{.914\columnwidth}\small}
}
\newcommand{\closeTocMinipage}{
  \addtocontents{toc}{\protect\end{minipage}}
  \addtocontents{toc}{}
  \addtocontents{toc}{\medskip}
}

\renewcommand{\slides}[1][]{
  \clearpage
  \incsec
  \section{\topic}
  {\small #1}
  \beginTocMinipage
  \setcounter{mypage}{0}
  \smallskip\nopagebreak\hrule\medskip
}

\newcommand{\slidesfoot}{
  \closeTocMinipage
  \bigskip
}

\newcommand{\sublecture}[2]{
  \pagebreak[3]
  \incpage
  \closeTocMinipage
  \subsection{#1}
  {\storyfont #2}
  \beginTocMinipage
  {\hfill\tiny \textsf{\arabic{mysec}:\arabic{mypage}}}\nopagebreak%
  \smallskip\nopagebreak\hrule
}

\newcommand{\key}[1]{
  \pagebreak[2]
  \addtocounter{mypage}{1}
  \addtocontents{toc}{\hyperref[key:#1]{#1 (\arabic{mysec}:\arabic{mypage})}}
  \phantomsection\label{key:#1}
  \index{#1@{\hyperref[key:#1]{#1 (\arabic{mysec}:\arabic{mypage})}}|phantom}
  \addtocounter{mypage}{-1}
}

\newenvironment{slidecore}[1]{
  \incpage
  \subsubsection*{#1}%{\headerfont\noindent\textbf{#1}\\}%
  \vspace{-6ex}%
  \begin{list}{$\bullet$}{\leftmargin4ex \rightmargin0ex \labelsep1ex
    \labelwidth2ex \partopsep0ex \topsep0ex \parsep.5ex \parskip0ex \itemsep0pt}\item[]~\\\nopagebreak%
}{
  \end{list}\nopagebreak%
  {\hfill\tiny \textsf{\arabic{mysec}:\arabic{mypage}}}\nopagebreak%
  \smallskip\nopagebreak\hrule
}

\newcommand{\slide}[2]{
  \begin{slidecore}{#1}
    #2
  \end{slidecore}
}

\newcommand{\exsection}[1]{
  \subsubsection{#1}
}

\renewcommand{\exercises}{
  \subsection{Exercise \exnum}
}

\newcommand{\exerfoot}{
  \bigskip
}

\newcommand{\story}[1]{
  \subsection*{Motivation \& Outline}
  \addtocontents{toc}{\hyperref[mot\arabic{mysec}]{Motivation \& Outline}}
  \phantomsection\label{mot\arabic{mysec}}
  {\storyfont\sf #1}
  \medskip\nopagebreak\hrule
}

\newcounter{savedsection}
\newcommand{\subappendix}{\setcounter{savedsection}{\arabic{section}}\appendix}
\newcommand{\noappendix}{
  \setcounter{section}{\arabic{savedsection}}% restore section number
  \setcounter{subsection}{0}% reset section counter
%  \gdef\@chapapp{\sectionname}% reset section name
  \renewcommand{\thesection}{\arabic{section}}% make section numbers arabic
}

\newenvironment{items}[1][9]{
\par\setlength{\unitlength}{1pt}\fontsize{#1}{#1}\linespread{1.2}\selectfont
\begin{list}{--}{\leftmargin4ex \rightmargin0ex \labelsep1ex \labelwidth2ex
\topsep0pt \parsep0ex \itemsep3pt}
}{
\end{list}
}

  \scripthead
}

\providecommand{\course}{NO COURSE}
\providecommand{\coursepicture}{NO PICTURE}
\providecommand{\coursedate}{NO DATE}
\providecommand{\topic}{NO TOPIC}
\providecommand{\keywords}{NO KEYWORDS}
\providecommand{\exnum}{NO NUMBER}


\providecommand{\stdpackages}{
  \usepackage{amsmath}
  \usepackage{amssymb}
  \usepackage{amsfonts}
  \allowdisplaybreaks
  \usepackage{amsthm}
  \usepackage{eucal}
  \usepackage{graphicx}
  \usepackage{color}
  \usepackage{geometry}
  \usepackage{framed}
%  \usecolor{xcolor}
  \definecolor{shadecolor}{gray}{0.9}
  \setlength{\FrameSep}{3pt}
  \usepackage{fancyvrb}
  \fvset{numbers=left,xleftmargin=5ex}

  \usepackage{multicol} 
  \usepackage{fancyhdr}
}

\providecommand{\addressUSTT}{
  Machine~Learning~\&~Robotics~lab, U~Stuttgart\\\small
  Universit{\"a}tsstra{\ss}e 38, 70569~Stuttgart, Germany
}


\renewcommand{\course}{Robotics}
\renewcommand{\coursepicture}{roboticsLecture}
\renewcommand{\coursedate}{Winter 2014}
\renewcommand{\topic}{Mobile Robotics}

\slides

%\newcommand{\+}{\myplus}
%\renewcommand{\-}{\,\text{-}\,}

%%%%%%%%%%%%%%%%%%%%%%%%%%%%%%%%%%%%%%%%%%%%%%%%%%%%%%%%%%%%%%%%%%%%%%%%%%%%%%%%
\key{Darpa challenge}
\slide{}{

\show[1]{06-darpaGrandChallenge}

{\hfill\tiny\url{http://www.darpa.mil/grandchallenge05/}}

\mov{DARPA Grand Challenge 2005}{10-RoboticsLecture/2005-DARPA_Grand_Challenge_Final_Part_2.avi}


}

%%%%%%%%%%%%%%%%%%%%%%%%%%%%%%%%%%%%%%%%%%%%%%%%%%%%%%%%%%%%%%%%%%%%%%%%%%%%%%%%

\slide{}{

\show[1]{07-DarpaUrbanChallenge}


{\hfill\tiny\url{http://www.darpa.mil/grandchallenge/index.asp}}

\mov{DARPA Grand Urban Challenge
2007}{10-RoboticsLecture/2007-The_Robot_Champion_of_DARPA_s_Urban_Challenge.avi}

}

%%%%%%%%%%%%%%%%%%%%%%%%%%%%%%%%%%%%%%%%%%%%%%%%%%%%%%%%%%%%%%%%%%%%%%%%%%%%%%%%
\slide{}{

\show[1]{2009-Quadcopter-movie.png}

{\hfill\tiny\url{http://www.slawomir.de/publications/grzonka09icra/grzonka09icra.pdf}}

\mov{Quadcopter Indoor Localization}{10-RoboticsLecture/2009-Quadcopter-SLAM.avi}

}

%%%%%%%%%%%%%%%%%%%%%%%%%%%%%%%%%%%%%%%%%%%%%%%%%%%%%%%%%%%%%%%%%%%%%%%%%%%%%%%%

\slide{}{

\show[.9]{10-STAIR}

{\hfill\tiny\url{http://stair.stanford.edu/multimedia.php}}

\mov{STAIR: STanford Artificial Intelligence Robot}{10-RoboticsLecture/10-STAIR_office_stapler.avi}

}

%%%%%%%%%%%%%%%%%%%%%%%%%%%%%%%%%%%%%%%%%%%%%%%%%%%%%%%%%%%%%%%%%%%%%%%%%%%%%%%%
\slide{Types of Robot Mobility}{\center

\showh[.3]{pioneer}
\qquad\qquad\showh[.2]{mars-nasa-opportunity}

~

\showh[.2]{asimoWalk}
\qquad\qquad\showh[.4]{quadcopter}

}

%%%%%%%%%%%%%%%%%%%%%%%%%%%%%%%%%%%%%%%%%%%%%%%%%%%%%%%%%%%%%%%%%%%%%%%%%%%%%%%%
%% \slide{Types of Robot Mobility}{\center

%% \showh[.7]{omnidrive}

%% }

%%%%%%%%%%%%%%%%%%%%%%%%%%%%%%%%%%%%%%%%%%%%%%%%%%%%%%%%%%%%%%%%%%%%%%%%%%%%%%%%

\slide{}{

\item Each type of robot mobility corresponds to a 

\cen{system equation $x_{t\po} = x_t + \tau f(x_t,u_t)$}

%{\small(e.g., see lecture ``Dynamics \& non-holonomic systems'')}

or, if the dynamics are stochastic,

\medskip

\eqbox{\large $P(x_{t\po} \| u_t, x_t) = \NN(x_{t\po} \| x_t + \tau f(x_t,u_t), \S)$}

~

\item We considered control, path finding, and trajectory optimization

~

For this we always assumed to know the state $x_t$ of the robot (e.g., its
posture/position)!

}

%%%%%%%%%%%%%%%%%%%%%%%%%%%%%%%%%%%%%%%%%%%%%%%%%%%%%%%%%%%%%%%%%%%%%%%%%%%%%%%%
\slide{Outline}{

\item PART I:

A core challenge in mobile robotics is \textbf{state
estimation}

$\to$ Bayesian filtering \& smoothing

~~~ particles, Kalman

~

\item PART II:

Another challenge is to \textbf{build a map} while exploring

$\to$ SLAM (simultaneous localization and mapping)

}

%%%%%%%%%%%%%%%%%%%%%%%%%%%%%%%%%%%%%%%%%%%%%%%%%%%%%%%%%%%%%%%%%%%%%%%%%%%%%%%%
\sublecture{State Estimation}

%%%%%%%%%%%%%%%%%%%%%%%%%%%%%%%%%%%%%%%%%%%%%%%%%%%%%%%%%%%%%%%%%%%%%%%%%%%%%%%%
\slide{State Estimation Problem}{

\item Our sensory data does not provide sufficient information to determine
our location.

~

\item Given the local sensor readings $y_t$, the current state $x_t$ (location, position) is
\emph{uncertain}.

~

-- which hallway?
\anchor{100,-55}{\showh[.2]{hallway}}

-- which door exactly?

-- which heading direction?

~

~

}

%%%%%%%%%%%%%%%%%%%%%%%%%%%%%%%%%%%%%%%%%%%%%%%%%%%%%%%%%%%%%%%%%%%%%%%%%%%%%%%%
\slide{State Estimation Problem}{

\twocol{.5}{.5}{
\item What is the probability of being in front of room 154, given we
  see what is shown in the image?

~

\item What is the probability given that we were just in front of room
  156?

~

\item What is the probability given that we were in front of room 156
  and moved 15 meters?
}{
\show{hallway}
}

}

%%%%%%%%%%%%%%%%%%%%%%%%%%%%%%%%%%%%%%%%%%%%%%%%%%%%%%%%%%%%%%%%%%%%%%%%%%%%%%%%
\slide{Recall Bayes' theorem}{

\Large

~

\cen{$P(X|Y) = \frac{P(Y|X)~ P(X)}{P(Y)}$}

~

\cen{$\text{posterior} =
\frac{\text{likelihood} ~\cdot~ \text{prior}}{\text{(normalization)}}$}

}

%%%%%%%%%%%%%%%%%%%%%%%%%%%%%%%%%%%%%%%%%%%%%%%%%%%%%%%%%%%%%%%%%%%%%%%%%%%%%%%%
\key{Applying Bayes' rule to state estimation}
\slide{}{

\item How can we apply this to the\newline
 State Estimation Problem?
\anchor{100,-100}{\showh[.3]{hallway}}


~\mypause

~

~

Using Bayes Rule:

{$P(\text{location} \| \text{sensor})
= \frac{P(\text{sensor} \| \text{location})
  P(\text{location})}{P(\text{sensor})}$}

\show[.6]{localization}

}

%%%%%%%%%%%%%%%%%%%%%%%%%%%%%%%%%%%%%%%%%%%%%%%%%%%%%%%%%%%%%%%%%%%%%%%%%%%%%%%%
\key{Bayes Filter}
\slide{Bayes Filter}{

$x_t$ = state (location) at time $t$

$y_t$ = sensor readings at
  time $t$

$u_{t\1}$ = control command (action, steering, velocity) at
  time $t\1$

~

\item Given the history $y_{0:t}$ and $u_{0:t\1}$, we want to compute the probability
  distribution over the state at time $t$

~

\eqbox{\large $p_t(x_t) := P(x_t \| y_{0:t}, u_{0:t\1})$}

~

\item Generally:

\show[.6]{filtering}


}

%%%%%%%%%%%%%%%%%%%%%%%%%%%%%%%%%%%%%%%%%%%%%%%%%%%%%%%%%%%%%%%%%%%%%%%%%%%%%%%%
\slide{Bayes Filter}{

\vspace*{-5mm}
\begin{align*}
p_t&(x_t) := P(x_t \| y_{0:t}, u_{0:t\1}) \\
\onslide<2->{&= c_t~ P(y_t \| x_t, y_{0:t\1}, u_{0:t\1})~ P(x_t \| y_{0:t\1}, u_{0:t\1})\\}
\onslide<3->{&= c_t~ P(y_t \| x_t)~ P(x_t \| y_{0:t\1}, u_{0:t\1})\\}
\onslide<4->{&= c_t~ P(y_t \| x_t)~ \int_{x_{t\1}} P(x_t,x_{t\1} \| y_{0:t\1}, u_{0:t\1})~ dx_{t\1}\\}
\onslide<5->{&= c_t~ P(y_t \| x_t)~ \int_{x_{t\1}} P(x_t \| x_{t\1}, y_{0:t\1},
u_{0:t\1})~ P(x_{t\1} \| y_{0:t\1}, u_{0:t\1})~ dx_{t\1}\\}
\onslide<6->{&= c_t~ P(y_t \| x_t)~ \int_{x_{t\1}} P(x_t \| x_{t\1}, u_{t\1})~ P(x_{t\1} \| y_{0:t\1}, u_{0:t\1})~ dx_{t\1}\\}
\onslide<7->{&= c_t~ {\color{red}P(y_t \| x_t)}~ \int_{x_{t\1}} {\color{red}P(x_t
  \| u_{t\1}, x_{t\1})}~ p_{t\1}(x_{t\1})~ dx_{t\1}}
\end{align*}

\only<2>{using Bayes rule $P(X|Y,Z) = c~ P(Y|X,Z)~ P(X|Z)$ with some
normalization constant $c_t$}

\only<3>{uses conditional independence of the observation on past observations
and controls}

\only<4>{by definition of the marginal}

\only<5>{by definition of a conditional}

\only<6>{given $x_{t\1}$, $x_t$ depends only on the controls $u_{t\1}$ (Markov Property)}

\only<7>{
\item A Bayes filter updates the posterior belief $p_t(x_t)$ in each
  time step using the:

~~ observation model ~ $P(y_t \| x_t)$

~~ transition model ~ $P(x_t \| u_{t\1}, x_{t\1})$
}

}

%%%%%%%%%%%%%%%%%%%%%%%%%%%%%%%%%%%%%%%%%%%%%%%%%%%%%%%%%%%%%%%%%%%%%%%%%%%%%%%%

\slide{Bayes Filter}{

\cen{\large $p_t(x_t)
~\propto~ \underbrace{P(y_t \| x_t)}_\text{new information}~ \underbrace{\int_{x_{t\1}} P(x_t
  \| u_{t\1}, x_{t\1})~ \underbrace{p_{t\1}(x_{t\1})}_\text{old
    estimate}~ dx_{t\1}}_{\text{predictive estimate } \hat p_t(x_t)}$}

~

1. We have a belief $p_{t\1}(x_{t\1})$ of our previous position 

~

2. We use the motion model to predict the current
position

\cen{$\hat p_t(x_t) \propto \int_{x_{t\1}} P(x_t \| u_{t\1}, x_{t\1})~ p_{t\1}(x_{t\1})~ dx_{t\1}$}

~

3. We integetrate this with the current observation to get a better
belief

\cen{$p_t(x_t) \propto P(y_t \| x_t)~ \hat p_t(x_t)$}

}

%%%%%%%%%%%%%%%%%%%%%%%%%%%%%%%%%%%%%%%%%%%%%%%%%%%%%%%%%%%%%%%%%%%%%%%%%%%%%%%%

\slide{}{

\item Typical transition model $P(x_t \| u_{t\1}, x_{t\1})$ in robotics:

~

\show[.6]{motionModel1}

~

(from \emph{Robust Monte Carlo localization for mobile robots}
Sebastian Thrun, Dieter Fox, Wolfram Burgard, Frank Dellaert)

}

%%%%%%%%%%%%%%%%%%%%%%%%%%%%%%%%%%%%%%%%%%%%%%%%%%%%%%%%%%%%%%%%%%%%%%%%%%%%%%%%
\key{Odometry: Filtering without observations}
\slide{Odometry (``Dead Reckoning''): Filtering without observations}{

\item The predictive distributions $\hat p_t(x_t)$ \emph{without
  integrating observations} (removing the $P(y_t|x_t)$ part from the
  Bayesian filter)

~

\show[.5]{motionModel2}

~

(from \emph{Robust Monte Carlo localization for mobile robots}
Sebastian Thrun, Dieter Fox, Wolfram Burgard, Frank Dellaert)

}

%%%%%%%%%%%%%%%%%%%%%%%%%%%%%%%%%%%%%%%%%%%%%%%%%%%%%%%%%%%%%%%%%%%%%%%%%%%%%%%%
\slide{}{

Again, predictive distributions $\hat p_t(x_t)$ \emph{without
  integrating landmark observations}

~

\show{filtering2}
}

%%%%%%%%%%%%%%%%%%%%%%%%%%%%%%%%%%%%%%%%%%%%%%%%%%%%%%%%%%%%%%%%%%%%%%%%%%%%%%%%

\slide{}{

The Bayes-filtered distributions $p_t(x_t)$ integrating landmark observations

~

\show{filtering3}
}

%%%%%%%%%%%%%%%%%%%%%%%%%%%%%%%%%%%%%%%%%%%%%%%%%%%%%%%%%%%%%%%%%%%%%%%%%%%%%%%%
\slide{Bayesian Filters}{

\item How to represent the belief $p_t(x_t)$:

~

~

\item Gaussian
\anchor{30,-70}{\showh[.3]{gaussian}}

~

~

~

~

\item Particles

\show{particles.png}

}

%%%%%%%%%%%%%%%%%%%%%%%%%%%%%%%%%%%%%%%%%%%%%%%%%%%%%%%%%%%%%%%%%%%%%%%%%%%%%%%%

\slide{Recall: Particle Representation of a Distribution}{

\item Weighed set of $N$ particles $\{ (x^i, w^i) \}_{i=1}^N$

~

\cen{$p(x) \approx q(x) := \sum_{i=1}^N w^i \d(x,x^i)$}

~

~

\show[.6]{particleRepresentation}

}

%%%%%%%%%%%%%%%%%%%%%%%%%%%%%%%%%%%%%%%%%%%%%%%%%%%%%%%%%%%%%%%%%%%%%%%%%%%%%%%%
\key{Particle Filter = Bayes Filter with Particles}
\slide{Particle Filter := Bayesian Filtering with Particles}{

(Bayes Filter: $p_t(x_t) ~\propto~ P(y_t \| x_t) \int_{x_{t\1}}
P(x_t \| u_{t\1}, x_{t\1})~ p_{t\1}(x_{t\1})~ dx_{t\1}$~)

~

\show[.7]{particleFiltering}

1. Start with $N$ particles $\{ (x^i_{t\1}, w^i_{t\1}) \}_{i=1}^N$

2. Resample particles to get $N$ weight-1-particles: $\{ \hat
x^i_{t\1} \}_{i=1}^N$

3. Use motion model to get new ``predictive'' particles $\{ x^i_t
\}_{i=1}^N$

~~~ each $x^i_t \sim P(x_t \| u_{t\1}, \hat x^i_{t\1})$

4. Use observation model to assign new weights $w^i_t \propto P(y_t \| x^i_t)$



}

%%%%%%%%%%%%%%%%%%%%%%%%%%%%%%%%%%%%%%%%%%%%%%%%%%%%%%%%%%%%%%%%%%%%%%%%%%%%%%%%
\slide{}{

\item ``Particle Filter''

~

aka \emph{Monte Carlo Localization} in the mobile robotics community

~

\emph{Condensation Algorithm} in the vision community

~

\item Efficient resampling is important:

Typically ``Residual Resampling'':

~

{\small Instead of sampling directly $\hat n^i \sim
  \text{Multi}(\{Nw_i\})$ set $\hat n^i = \lfloor N w_i \rfloor +
\bar n_i$ with $\bar n_i \sim \text{Multi}(\{Nw_i-\lfloor N w_i \rfloor\})$

Liu \& Chen (1998): Sequential Monte Carlo Methods for Dynamic Systems.

Douc, Capp\'e \& Moulines: Comparison of Resampling Schemes for Particle Filtering.

}


}

%%%%%%%%%%%%%%%%%%%%%%%%%%%%%%%%%%%%%%%%%%%%%%%%%%%%%%%%%%%%%%%%%%%%%%%%%%%%%%%%

\slide{Example:~ Quadcopter Localization}{

\show[1]{2009-Quadcopter-movie.png}

{\hfill\tiny\url{http://www.slawomir.de/publications/grzonka09icra/grzonka09icra.pdf}}

\mov{Quadcopter Indoor Localization}{10-RoboticsLecture/2009-Quadcopter-SLAM.avi}

}

%%%%%%%%%%%%%%%%%%%%%%%%%%%%%%%%%%%%%%%%%%%%%%%%%%%%%%%%%%%%%%%%%%%%%%%%%%%%%%%%
\slide{Typical Pitfall in Particle Filtering}{

\item Predicted particles $\{ x^i_t
\}_{i=1}^N$ have very low observation likelihood $P(y_t \| x^i_t)
\approx 0$

(``particles die over time'')

~

\item Classical solution: generate particles also with other than
  purely forward proposal $P(x_t \| u_{t\1}, x_{t\1})$:

~

-- Choose a proposal that depends on the new observation $y_t$, ideally approximating
$P(x_t \| y_t, u_{t\1}, x_{t\1})$

~

-- Or mix particles sampled directly from $P(y_t \| x_t)$ and from
$P(x_t \| u_{t\1}, x_{t\1})$.

{\tiny(\emph{Robust Monte Carlo
localization for mobile robots.}  Sebastian Thrun, Dieter Fox, Wolfram
Burgard, Frank Dellaert)\\}

}

%%%%%%%%%%%%%%%%%%%%%%%%%%%%%%%%%%%%%%%%%%%%%%%%%%%%%%%%%%%%%%%%%%%%%%%%%%%%%%%%
\key{Kalman Filter = Bayes Filter with Gaussian}
\slide{Kalman filter := Bayesian Filtering with Gaussians}{

Bayes Filter: $p_t(x_t) ~\propto~ P(y_t \| x_t) \int_{x_{t\1}}
P(x_t \| u_{t\1}, x_{t\1})~ p_{t\1}(x_{t\1})~ dx_{t\1}$~

~

\item Can be computed analytically for linear-Gaussian observations
  and transitions:

$P(y_t \| x_t) = \NN(y_t \| C x_t + c, W)$

$P(x_t \| u_{t\1}, x_{t\1}) = \NN(x_t \| A(u_{t\1})~ x_{t\1} + a(u_{t\1}), Q)$

~

\providecommand{\+}{\myplus}
\renewcommand{\-}{\,\text{-}\,}
\tiny

Defition:\\
$\NN(x \| a,A) = \frac{1}{|2\pi A|^{1/2}}~ \exp\{-\half (x\-a)^\T~ A^\1~ (x\-a)\}$

Product:\\
$\NN(x \| a,A)~ \NN(x \| b,B) 
 = \NN(x \| B(A\+B)^\1a + A(A\+B)^\1b ,A(A\+B)^\1B)~~ \NN(a\|b,A+B)$

``Propagation'':\\
$\int_y \NN(x \| a + Fy, A)~ \NN(y \| b, B)~ dy
 = \NN(x \| a + Fb, A+FBF^\T)$

Transformation:\\
$ \NN(F x + f \| a,A)
 = \frac{1}{|F|}~ \NN(x \| ~ F^\1 (a-f),~ F^\1 AF^\mT) $

~

(more identities: see ``Gaussian identities'' {\tiny\url{http://ipvs.informatik.uni-stuttgart.de/mlr/marc/notes/gaussians.pdf}})

}

%%%%%%%%%%%%%%%%%%%%%%%%%%%%%%%%%%%%%%%%%%%%%%%%%%%%%%%%%%%%%%%%%%%%%%%%%%%%%%%%
\key{Kalman filter equations}
\slide{Kalman filter derivation}{

\tiny
\begin{align*}
&\quad p_t(x_t) = \NN(x_t \| s_t, S_t) \\
&\quad P(y_t \| x_t) = \NN(y_t \| C x_t + c, W) \\
&\quad P(x_t \| u_{t\1}, x_{t\1}) = \NN(x_t \| A x_{t\1} + a, Q) \\
p_t&(x_t) \propto
P(y_t \| x_t) \int_{x_{t\1}} P(x_t \| u_{t\1}, x_{t\1})~ p_{t\1}(x_{t\1})~ dx_{t\1} \\
&= \NN(y_t \| C x_t + c, W)~ \int_{x_{t\1}}
 \NN(x_t \| A x_{t\1} + a, Q)~
 \NN(x_{t\1} \| s_{t\1}, S_{t\1})~ dx_{t\1} \\
&= \NN(y_t \| C x_t + c, W)~
 \NN(x_t \| \underbrace{A s_{t\1} + a}_{=:\hat
 s_t},~ \underbrace{Q + A S_{t\1} A^\T}_{=:\hat S_t}) \\
&= \NN(C x_t + c \| y_t , W)~
 \NN(x_t \| \hat s_t, \hat S_t) \\
&= \NN[x_t \| C^\T W^\1(y_t-c) ,~ C^\T W^\1 C] ~
 \NN(x_t \| \hat s_t, \hat S_t) \\
&= \NN(x_t \| s_t, S_t) \cdot\<\text{terms indep. of $x_t$}\> \\
S_t
&= (C^\T W^\1 C+\hat S_t^\1)^\1
 = \hat S_t
 - \underbrace{\hat S_t C^\T(W+C \hat S_t C^\T)^\1}_{\text{``Kalman gain'' $K$}} C \hat S_t \\
s_t
&= S_t [C^\T W^\1(y_t-c) + \hat S_t^\1 \hat s_t]
 = \hat s_t + K(y_t-C \hat s_t-c)
\end{align*}
The second to last line uses the general Woodbury identity.

The last line uses $S_t C^\T W^\1=K$ and $S_t\hat S_t^\1=\Id-K C$

}

%%%%%%%%%%%%%%%%%%%%%%%%%%%%%%%%%%%%%%%%%%%%%%%%%%%%%%%%%%%%%%%%%%%%%%%%%%%%%%%%
\key{Extended KF, Unscented Transform}
\slide{Extended Kalman filter (EKF) and Unscented Transform}{

Bayes Filter: $p_t(x_t) ~\propto~ P(y_t \| x_t) \int_{x_{t\1}}
P(x_t \| u_{t\1}, x_{t\1})~ p_{t\1}(x_{t\1})~ dx_{t\1}$~

~

\item Can be computed analytically for linear-Gaussian observations
  and transitions:

$P(y_t \| x_t) = \NN(y_t \| C x_t + c, W)$

$P(x_t \| u_{t\1}, x_{t\1}) = \NN(x_t \| A(u_{t\1}) x_{t\1} + a(u_{t\1}), Q)$

~

\item If $P(y_t \| x_t)$ or $P(x_t \| u_{t\1}, x_{t\1})$ are not
  linear:

$P(y_t \| x_t) = \NN(y_t \| g(x_t), W)$

$P(x_t \| u_{t\1}, x_{t\1}) = \NN(x_t \| f(x_{t\1},u_{t\1}), Q)$

-- approximate $f$ and $g$ as locally linear (\emph{Extended Kalman Filter})

-- or sample locally from them and reapproximate as Gaussian
(\emph{Unscented Transform})

}

%%%%%%%%%%%%%%%%%%%%%%%%%%%%%%%%%%%%%%%%%%%%%%%%%%%%%%%%%%%%%%%%%%%%%%%%%%%%%%%%

\slide{Bayes smoothing}{

~

\show{filtering}

}

%%%%%%%%%%%%%%%%%%%%%%%%%%%%%%%%%%%%%%%%%%%%%%%%%%%%%%%%%%%%%%%%%%%%%%%%%%%%%%%%

\slide{Bayes smoothing}{

\item Let $\PP=y_{0:t}$ past observations, 
 $\FF=y_{t\po:T}$ future observations
\begin{align*}
P(x_t \| \PP, \FF, u_{0:T})
% &= \frac{P(\FF \| x_t, \PP, u_{0:T})~ P(x_t\|\PP, u_{0:T})}{P(\FF\|\PP, u_{0:T})} \\
&\propto
P(\FF \| x_t, \PP, u_{0:T})~ P(x_t\|\PP, u_{0:T}) \\
&=
\underbrace{P(\FF \| x_t, u_{t:T})}_{=:\b_t(x_t)}~ \underbrace{P(x_t\|\PP, u_{0:t\1})}_{=:p(x_t)}
\end{align*}
{\hspace*{-9mm}\color{red}\emph{Bayesian smoothing fuses a forward filter $p_t(x_t)$ with a backward
``filter'' $\b_t(x_t)$}}

~\mypause

\item Backward recursion (derivation analogous to the Bayesian filter)
\begin{align*}
\b_t(x_t)
&:=
P(y_{t\po:T} \| x_t, u_{t:T}) \\
%% &=
%% \int_{x_{t\po}} P(y_{t\po:T}, x_{t\po} \| x_t, u_{t:T})~ dx_{t\po}
%% \comment{def.\ of marginal}\\
%% &= 
%% \int_{x_{t\po}} P(y_{t\po:T} \| x_{t\po}, x_t, u_{t:T})~ P(x_{t\po} \| x_t, u_{t:T})~ dx_{t\po}
%% \comment{def.\ of conditional}\\
%% &= 
%% \int_{x_{t\po}} P(y_{t\po:T} \| x_{t\po}, u_{t\po:T})~ P(x_{t\po} \| x_t, u_t)~ dx_{t\po}
%% \comment{conditional independenciet} \\
%% &= 
%% \int_{x_{t\po}} P(y_{t\pT:T}, y_{t\po} \| x_{t\po}, u_{t\po:T})~ P(x_{t\po} \| x_t, u_t)~ dx_{t\po}\\
%% &= 
%% \int_{x_{t\po}} P(y_{t\pT:T}\| y_{t\po}, x_{t\po}, u_{t\po:T})~ P(y_{t\po} \| x_{t\po}, u_{t\po:T})~  P(x_{t\po} \| x_t, u_t)~ dx_{t\po}
%% \comment{def.\ of conditional} \\
%% &= 
%% \int_{x_{t\po}} P(y_{t\pt:T}\| x_{t\po}, u_{t\po:T})~ P(y_{t\po} \| x_{t\po})~  P(x_{t\po} \| x_t, u_t)~ dx_{t\po}
%% \comment{conditional independenciet} \\
&= 
\int_{x_{t\po}} \b_{t\po}(x_{t\po})~ P(y_{t\po} \| x_{t\po})~ P(x_{t\po} \| x_t, u_t)~ dx_{t\po}
\end{align*}

}

%%%%%%%%%%%%%%%%%%%%%%%%%%%%%%%%%%%%%%%%%%%%%%%%%%%%%%%%%%%%%%%%%%%%%%%%%%%%%%%%
\sublecture{Simultaneous Localization and Mapping (SLAM)}

%%%%%%%%%%%%%%%%%%%%%%%%%%%%%%%%%%%%%%%%%%%%%%%%%%%%%%%%%%%%%%%%%%%%%%%%%%%%%%%%
\slide{Localization and Mapping}{

\item The Bayesian filter requires an observation model $P(y_t \| x_t)$

~

\item A \textbf{map} is something that provides the
observation model:

\emph{A map tells us for each $x_t$ what the sensor readings $y_t$ might
look like}

}

%%%%%%%%%%%%%%%%%%%%%%%%%%%%%%%%%%%%%%%%%%%%%%%%%%%%%%%%%%%%%%%%%%%%%%%%%%%%%%%%

\slide{Types of maps}{

~

\hspace*{-8mm}\twocol{.5}{.5}{
\textbf{Grid map}

%\show[.5]{murphyGridMap}
\show[.5]{gridMap2}

{\tiny

K. Murphy (1999): \emph{Bayesian map learning\\ in dynamic
environments.}

%% {\hfill\tiny\url{http://www.cs.ubc.ca/~murphyk/Papers/map_nips99.pdf}\\}

Grisetti, Tipaldi, Stachniss, Burgard, Nardi:\\
\emph{Fast and Accurate SLAM with\\
Rao-Blackwellized Particle Filters}

%% {\hfill\tiny\url{http://www.informatik.uni-freiburg.de/~stachnis/pdf/grisetti07jras.pdf}\\}

}

~

\textbf{Laser scan map}

\show[.8]{2009-Quadcopter-movie.png}

}{

\textbf{Landmark map}

%\show{victoria}
%\showh[.4]{FastSLAM-vic1}
\show[.8]{FastSLAM-vic2}


%% {\hfill\tiny\url{http://www-personal.acfr.usyd.edu.au/nebot/victoria_park.htm}}

\mov{Victoria Park data set}{10-RoboticsLecture/vic_park_turn_in_out.mpg}

{\tiny M. Montemerlo, S. Thrun, D. Koller, \& B. Wegbreit (2003): \emph{FastSLAM
2.0: An improved particle filtering algorithm for simultaneous
localization and mapping that provably converges.} IJCAI, 1151--1156.

}

}

}

%%%%%%%%%%%%%%%%%%%%%%%%%%%%%%%%%%%%%%%%%%%%%%%%%%%%%%%%%%%%%%%%%%%%%%%%%%%%%%%%
\key{Definition of a map, definition of SLAM problem}
\slide{Simultaneous Localization and Mapping Problem}{

\item Notation:

$x_t$ = state (location) at time $t$

$y_t$ = sensor readings at time $t$

$u_{t\1}$ = control command (action, steering, velocity) at time $t\1$

$m$ = the map; \emph{formally: a map is the parameters that define $P(y_t \| x_t, m)$}

~

\item Given the history $y_{0:t}$ and $u_{0:t\1}$, we want to compute
the belief over the pose {\color{red}AND THE MAP} $m$ at time $t$

~

\cen{\large $p_t(x_t,m) := P(x_t, m \| y_{0:t}, u_{0:t\1})$}

~

\item We assume to know the:

-- transition model ~ $P(x_t \| u_{t\1}, x_{t\1})$

-- observation model ~ $P(y_t \| x_t, {\color{red}m})$ ~ \emph{(defined
   by the map)}

}

%%%%%%%%%%%%%%%%%%%%%%%%%%%%%%%%%%%%%%%%%%%%%%%%%%%%%%%%%%%%%%%%%%%%%%%%%%%%%%%%

\slide{SLAM: classical ``chicken or egg problem''}{

\item  If we knew the state trajectory $x_{0:t}$ we could efficiently compute the belief over the map
$$P(m \| x_{0:t}, y_{0:t}, u_{0:t\1})$$

~

\item If we knew the map we could use a Bayes filter to compute the belief over the state
$$P(x_t \| m, y_{0:t}, u_{0:t\1})$$

~\mypause

\item SLAM requires to tie state estimation and map building together:

1) Joint inference on $x_t$ and $m$ ~ ($\to$ Kalman-SLAM)

2) Tie a state hypothesis (=particle) to a map hypothesis \\ ~~~
($\to$ particle SLAM)

3) Frame everything as a graph optimization problem ~ ($\to$ graph SLAM)
}

%%%%%%%%%%%%%%%%%%%%%%%%%%%%%%%%%%%%%%%%%%%%%%%%%%%%%%%%%%%%%%%%%%%%%%%%%%%%%%%%
\key{Joint Bayes Filter over state and map (Kalman SLAM)}
\slide{Joint Bayesian Filter over $x$ and $m$}{

\item A (formally) straight-forward approach is the joint
Bayesian filter

~

\cen{$p_t(x_t,m)
~\propto~
 P(y_t \| x_t,m)~
 \int_{x_{t\1}} P(x_t \| u_{t\1}, x_{t\1})~ p_{t\1}(x_{t\1},m)~ dx_{t\1}$}

~

~

But: How represent a belief over high-dimensional $x_t,m$?

}

%%%%%%%%%%%%%%%%%%%%%%%%%%%%%%%%%%%%%%%%%%%%%%%%%%%%%%%%%%%%%%%%%%%%%%%%%%%%%%%%
\slide{Map uncertainty}{

\item In the case the map $m=(\t_1,..,\t_N)$ is a set of $N$ landmarks,
   $\t_j \in\RRR^2$

~

\show[.5]{landmarkUncertainties}

~

\item Use Gaussians to represent the uncertainty of landmark positions

}

%%%%%%%%%%%%%%%%%%%%%%%%%%%%%%%%%%%%%%%%%%%%%%%%%%%%%%%%%%%%%%%%%%%%%%%%%%%%%%%%
\slide{(Extended) Kalman Filter SLAM}{

\item Analogous to Localization with Gaussian for the pose belief
   $p_t(x_t)$

-- But now: joint belief $p_t(x_t,\t_{1:N})$ is $3+2N$-dimensional Gaussian

-- Assumes the map $m=(\t_1,..,\t_N)$ is a set of $N$ landmarks,
   $\t_j \in\RRR^2$

-- Exact update equations (under the Gaussian assumption)

-- Conceptually very simple

~

\item Drawbacks:

-- Scaling (full covariance matrix is $O(N^2)$)

-- Sometimes non-robust (uni-modal, ``data association problem'')

-- Lacks advantages of Particle Filter\\
~~ (multiple hypothesis, more robust to non-linearities)

}

%%%%%%%%%%%%%%%%%%%%%%%%%%%%%%%%%%%%%%%%%%%%%%%%%%%%%%%%%%%%%%%%%%%%%%%%%%%%%%%%
\key{Particle SLAM}
\key{Particles carry their own history, their own map}
\slide{SLAM with particles}{

\cen{\textbf{Core idea: Each particle carries its own map
belief}}

~\mypause

\item Use a conditional representation ``$p_t(x_t,m) = p_t(x_t)~
p_t(m\|x_t)$''

{\small(This notation is flaky... the below is more precise)}

~

\item As for the Localization Problem use particles to represent
the \emph{pose belief} $p_t(x_t)$

Note: Each particle actually ``has a history $x^i_{0:t}$'' --
a whole trajectory!

~

\item For each particle separately, estimate the \emph{map belief} $p^i_t(m)$
conditioned on the particle history $x^i_{0:t}$.

The conditional beliefs $p^i_t(m)$ may be factorized over grid points or
landmarks of the map

~

K. Murphy (1999): \emph{Bayesian map learning in dynamic
environments.}

{\hfill\tiny\url{http://www.cs.ubc.ca/~murphyk/Papers/map_nips99.pdf}\\}

}

%%%%%%%%%%%%%%%%%%%%%%%%%%%%%%%%%%%%%%%%%%%%%%%%%%%%%%%%%%%%%%%%%%%%%%%%%%%%%%%%

\slide{Map estimation for a \protect\emph{given} particle history}{

\item Given $x_{0:t}$ (e.g.\ a trajectory of a particle), what is the
posterior over the map $m$?

~

$\to$ simplified Bayes Filter:
$$p_t(m)
:= P(m \| x_{0:t}, y_{0:t}) \propto P(y_t \| m, x_t)~ p_{t\1}(m)$$

(no transtion model: assumption that map is constant)

~

\item In the case of landmarks (FastSLAM):

$m = (\t_1,..,\t_N)$

$\t_j$ = position of the $j$th landmark, $j\in\{1,..,N\}$

$n_t$ = which landmark we observe at time $t$, ~ $n_t\in\{1,..,N\}$

~

We can use a separate (Gaussian) Bayes Filter for each $\t_j$

{\tiny conditioned on $x_{0:t}$, each $\t_j$ is independent from each
$\t_k$:

$$P(\t_{1:N} \| x_{0:t}, y_{0:n}, n_{0:t} ) = \prod_j P(\t_j \|
x_{0:t}, y_{0:n}, n_{0:t} )$$

}

}

%%%%%%%%%%%%%%%%%%%%%%%%%%%%%%%%%%%%%%%%%%%%%%%%%%%%%%%%%%%%%%%%%%%%%%%%%%%%%%%%

\slide{Particle likelihood in SLAM}{

\item Particle likelihood for Localization Problem:

\cen{$w^i_t = P(y_t \| x^i_t)$}

(determins the new importance weight $w^i_t$

~

\item In SLAM the map is uncertain $\to$ each particle is weighted
with the \emph{expected} likelihood:

\cen{$w^i_t = \int P(y_t \| x^i_t, m)~ p_{t-1}(m)~ d m$}

~

\item In case of landmarks (FastSLAM):

\cen{$w^i_t = \int P(y_t \| x^i_t, \t_{n_t}, n_t)~ p_{t-1}(\t_{n_t})~ d \t_{n_t}$}

~


\item Data association problem (actually we don't know $n_t$):

For each particle separately choose $n^i_t = \argmax_{n_t} w^i_t(n_t)$

}

%%%%%%%%%%%%%%%%%%%%%%%%%%%%%%%%%%%%%%%%%%%%%%%%%%%%%%%%%%%%%%%%%%%%%%%%%%%%%%%%

\slide{Particle-based SLAM summary}{

\item We have a set of $N$ particles $\{ (x^i, w^i) \}_{i=1}^N$ to
represent the pose belief $p_t(x_t)$

~

\item For each particle we have a separate map belief $p^i_t(m)$; in
the case of landmarks, this factorizes in $N$ separate 2D-Gaussians

~

\item Iterate

1. Resample particles to get $N$ weight-1-particles: $\{ \hat
x^i_{t\1} \}_{i=1}^N$

2. Use motion model to get new ``predictive'' particles $\{ x^i_t
\}_{i=1}^N$

{\color{blue}3. Update the map belief $p^i_m(m) \propto P(y_t \| m, x_t)~
p^i_{t\1}(m)$ for each particle}

4. Compute new importance weights $w^i_t \propto \int P(y_t \| x^i_t,
m)~ p_{t-1}(m)~ d m$\\ ~~~ using the observation model {\color{blue}and the map belief}

}

%%%%%%%%%%%%%%%%%%%%%%%%%%%%%%%%%%%%%%%%%%%%%%%%%%%%%%%%%%%%%%%%%%%%%%%%%%%%%%%%

\slide{Demo: Visual SLAM}{

\item Map building from a freely moving camera

~

\show{visualSlam}

}

%%%%%%%%%%%%%%%%%%%%%%%%%%%%%%%%%%%%%%%%%%%%%%%%%%%%%%%%%%%%%%%%%%%%%%%%%%%%%%%%

\slide{Demo: Visual SLAM}{

\item Map building from a freely moving camera

~

-- SLAM has become a big topic in the vision community..


-- features are typically landmarks $\t_{1:N}$ with SURF/SIFT features

-- PTAM (Parallel Tracking and Mapping) parallelizes computations...

~

\mov{PTAM1}{10-RoboticsLecture/ptam1.avi}
\mov{PTAM2}{10-RoboticsLecture/ptam2.avi}
\mov{DTAM}{11-DTAM-Davidson.mp4}
\mov{KinectFusion}{11-KinectFusion-Davidson.mp4}
\mov{LSD-SLAM}{14-LSD-slam.mp4}

~

Klein \& Murray: \emph{Parallel Tracking and Mapping for Small AR Workspaces}
{\tiny\hfill\url{http://www.robots.ox.ac.uk/~gk/PTAM/}}

Newcombe, Lovegrove \& Davison: \emph{DTAM: Dense Tracking and Mapping
in Real-Time} ICCV 2011.
{\tiny\hfill\url{http://www.doc.ic.ac.uk/~rnewcomb/}}

Engel, Sch\"ops \& Cremers: LSD-SLAM: Large-Scale Direct Monocular SLAM, ECCV 2014.
{\tiny\hfill\url{http://vision.in.tum.de/_media/spezial/bib/engel14eccv.pdf}}

}

%%%%%%%%%%%%%%%%%%%%%%%%%%%%%%%%%%%%%%%%%%%%%%%%%%%%%%%%%%%%%%%%%%%%%%%%%%%%%%%%
\key{Graph-based SLAM}
\slide{Alternative SLAM approach: Graph-based}{

\show[.5]{filtering3}


\item Represent the previous trajectory as a graph

-- nodes = estimated
   positions \& observations

-- edges = transition \& step estimation based on scan matching

\item Loop Closing: check if some nodes might coincide $\to$ new edges

\item Classical Optimization:

The whole graph defines an optimization problem: Find poses that
minimize sum of edge \& node errors

}

%%%%%%%%%%%%%%%%%%%%%%%%%%%%%%%%%%%%%%%%%%%%%%%%%%%%%%%%%%%%%%%%%%%%%%%%%%%%%%%%

\slide{Loop Closing Problem}{

(Doesn't explicitly exist in Particle Filter methods: If particles
cover the belief, then ``data association'' solves the ``loop
closing problem'')

\show[.4]{SLAMloopClosing}

}

%%%%%%%%%%%%%%%%%%%%%%%%%%%%%%%%%%%%%%%%%%%%%%%%%%%%%%%%%%%%%%%%%%%%%%%%%%%%%%%%

\slide{Graph-based SLAM}{

\showh[.4]{graph-basedSlam1}
\qquad
\showh[.4]{graph-basedSlam2}

~

{\tiny \emph{Life-long Map Learning for Graph-based
SLAM Approaches in Static Environments}
Kretzschmar, Grisetti, Stachniss}

}

%%%%%%%%%%%%%%%%%%%%%%%%%%%%%%%%%%%%%%%%%%%%%%%%%%%%%%%%%%%%%%%%%%%%%%%%%%%%%%%%

%% \slide{A comment...}{

%% \item It is sometimes said:

%% ``SLAM is \textbf{one of the most fundamental problems} for
%% robots to become truly autonomous.''

%% ~

%% I certainly don't agree!!

%% There are much much deeper problems around in robotics...

%% }

%%%%%%%%%%%%%%%%%%%%%%%%%%%%%%%%%%%%%%%%%%%%%%%%%%%%%%%%%%%%%%%%%%%%%%%%%%%%%%%%

\slide{SLAM code}{\label{lastpage}

\item Graph-based and grid map methods:

\url{http://openslam.org/}

~

~

\item Visual SLAM

see previous links

e.g. \url{http://ewokrampage.wordpress.com/}



}

%%%%%%%%%%%%%%%%%%%%%%%%%%%%%%%%%%%%%%%%%%%%%%%%%%%%%%%%%%%%%%%%%%%%%%%%%%%%%%%%

\slidesfoot


