\providecommand{\slides}{
  \newcommand{\slideshead}{
  \newcommand{\thepage}{\arabic{mypage}}
  %beamer
  \documentclass[t,hyperref={bookmarks=true}]{beamer}
%  \documentclass[t,hyperref={bookmarks=true},aspectratio=169]{beamer}
  \setbeamersize{text margin left=5mm}
  \setbeamersize{text margin right=5mm}
  \usetheme{default}
  \usefonttheme[onlymath]{serif}
  \setbeamertemplate{navigation symbols}{}
  \setbeamertemplate{itemize items}{{\color{black}$\bullet$}}

  \newwrite\keyfile

  %\usepackage{palatino}
  \stdpackages
  \usepackage{multimedia}

  %%% geometry/spacing issues
  %
  \definecolor{bluecol}{rgb}{0,0,.5}
  \definecolor{greencol}{rgb}{0,.6,0}
  %\renewcommand{\baselinestretch}{1.1}
  \renewcommand{\arraystretch}{1.2}
  \columnsep 0mm

  \columnseprule 0pt
  \parindent 0ex
  \parskip 0ex
  %\setlength{\itemparsep}{3ex}
  %\renewcommand{\labelitemi}{\rule[3pt]{10pt}{10pt}~}
  %\renewcommand{\labelenumi}{\textbf{(\arabic{enumi})}}
  \newcommand{\headerfont}{\helvetica{13}{1.5}{b}{n}}
  \newcommand{\slidefont} {\helvetica{10}{1.4}{m}{n}}
  \newcommand{\codefont} {\helvetica{8}{1.2}{m}{n}}
  \renewcommand{\small} {\helvetica{9}{1.4}{m}{n}}
  \renewcommand{\tiny} {\helvetica{8}{1.3}{m}{n}}
  \newcommand{\ttiny} {\helvetica{7}{1.3}{m}{n}}

  %%% count pages properly and put the page number in bottom right
  %
  \newcounter{mypage}
  \newcommand{\incpage}{\addtocounter{mypage}{1}\setcounter{page}{\arabic{mypage}}}
  \setcounter{mypage}{0}
  \resetcounteronoverlays{page}

  \pagestyle{fancy}
  %\setlength{\headsep}{10mm}
  %\addtolength{\footheight}{15mm}
  \renewcommand{\headrulewidth}{0pt} %1pt}
  \renewcommand{\footrulewidth}{0pt} %.5pt}
  \cfoot{}
  \rhead{}
  \lhead{}
  \lfoot{\vspace*{-3mm}\hspace*{-3mm}\helvetica{5}{1.3}{m}{n}{\texttt{github.com/MarcToussaint/AI-lectures}}}
%  \rfoot{{\tiny\textsf{AI -- \topic -- \subtopic -- \arabic{mypage}/\pageref{lastpage}}}}
%  \rfoot{\vspace*{-4.5mm}{\tiny\textsf{\topic\ -- \subtopic\ -- \arabic{mypage}/\pageref{lastpage}}}\hspace*{-4mm}}
  \rfoot{\vspace*{-4.5mm}{\tiny\textsf{\color{gray}\topic\ -- \subtopic\ -- \arabic{mypage}/\pageref{lastpage}}}\hspace*{-4mm}}
  %\lfoot{\raisebox{5mm}{\tiny\textsf{\slideauthor}}}
  %\rfoot{\raisebox{5mm}{\tiny\textsf{\slidevenue{} -- \arabic{mypage}/\pageref{lastpage}}}}
  %\rfoot{~\anchor{30,12}{\tiny\textsf{\thepage/\pageref{lastpage}}}}
  %\lfoot{\small\textsf{Marc Toussaint}}

  \definecolor{grey}{rgb}{.8,.8,.8}
  \definecolor{head}{rgb}{.85,.9,.9}
  \definecolor{blue}{rgb}{.0,.0,.5}
  \definecolor{green}{rgb}{.0,.5,.0}
  \definecolor{red}{rgb}{.8,.0,.0}
  \newcommand{\inverted}{
    \definecolor{main}{rgb}{1,1,1}
    \color{main}
    \pagecolor[rgb]{.3,.3,.3}
  }
  %auto-ignore
  \renewcommand{\a}{\alpha}
  \renewcommand{\b}{\beta}
  \renewcommand{\d}{\delta}
    \newcommand{\D}{\Delta}
    \newcommand{\e}{\epsilon}
    \newcommand{\g}{\gamma}
    \newcommand{\G}{\Gamma}
  \renewcommand{\l}{\lambda}
  \renewcommand{\L}{\Lambda}
    \newcommand{\m}{\mu}
    \newcommand{\n}{\nu}
    \newcommand{\N}{\nabla}
  \renewcommand{\k}{\kappa}
  \renewcommand{\o}{\omega}
  \renewcommand{\O}{\Omega}
    \newcommand{\p}{\phi}
    \newcommand{\ph}{\varphi}
  \renewcommand{\P}{\Phi}
  \renewcommand{\r}{\varrho}
    \newcommand{\s}{\sigma}
  \renewcommand{\S}{\Sigma}
  \renewcommand{\t}{\theta}
    \newcommand{\T}{\Theta}
  %\renewcommand{\v}{\vartheta}
    \newcommand{\x}{\xi}
    \newcommand{\X}{\Xi}
    \newcommand{\Y}{\Upsilon}
    \newcommand{\z}{\zeta}

  \renewcommand{\AA}{{\cal A}}
    \newcommand{\BB}{{\cal B}}
    \newcommand{\CC}{{\cal C}}
    \newcommand{\cc}{{\cal c}}
    \newcommand{\DD}{{\cal D}}
    \newcommand{\EE}{{\cal E}}
    \newcommand{\FF}{{\cal F}}
    \newcommand{\GG}{{\cal G}}
    \newcommand{\HH}{{\cal H}}
    \newcommand{\II}{{\cal I}}
    \newcommand{\KK}{{\cal K}}
    \newcommand{\LL}{{\cal L}}
    \newcommand{\MM}{{\cal M}}
    \newcommand{\NN}{{\cal N}}
    \newcommand{\oNN}{\overline\NN}
    \newcommand{\OO}{{\cal O}}
    \newcommand{\PP}{{\cal P}}
    \newcommand{\QQ}{{\cal Q}}
    \newcommand{\RR}{{\cal R}}
  \renewcommand{\SS}{{\cal S}}
    \newcommand{\TT}{{\cal T}}
    \newcommand{\uu}{{\cal u}}
    \newcommand{\UU}{{\cal U}}
    \newcommand{\VV}{{\cal V}}
    \newcommand{\XX}{{\cal X}}
    \newcommand{\xx}{\mathcal{x}}
    \newcommand{\YY}{{\cal Y}}
    \newcommand{\SOSO}{{\cal SO}}
    \newcommand{\GLGL}{{\cal GL}}

    \newcommand{\Ee}{{\rm E}}

  \newcommand{\NNN}{{\mathbb{N}}}
  \newcommand{\III}{{\mathbb{I}}}
  \newcommand{\ZZZ}{{\mathbb{Z}}}
  %\newcommand{\RRR}{{\mathrm{I\!R}}}
  \newcommand{\RRR}{{\mathbb{R}}}
  \newcommand{\SSS}{{\mathbb{S}}}
  \newcommand{\CCC}{{\mathbb{C}}}
  \newcommand{\DDD}{{\mathbb{D}}}
  \newcommand{\one}{{{\bf 1}}}
  \newcommand{\eee}{\text{e}}

  \newcommand{\NNNN}{{\overline{\cal N}}}

  \renewcommand{\[}{\Big[}
  \renewcommand{\]}{\Big]}
  \renewcommand{\(}{\Big(}
  \renewcommand{\)}{\Big)}
  \renewcommand{\|}{\,|\,}
  \renewcommand{\;}{\,;\,}
  \renewcommand{\=}{\!=\!}
    \newcommand{\<}{\left\langle}
  \renewcommand{\>}{\right\rangle}

  \newcommand{\na}[1][]{{\nabla_{\!\!#1}}}
  \newcommand{\he}[1][]{{\nabla_{\!\!#1}^2}}
  \newcommand{\Prob}{{\rm Prob}}
  \newcommand{\Dir}{{\rm Dir}}
  \newcommand{\Beta}{{\rm Beta}}
  \newcommand{\Bern}{{\rm Bern}}
  \newcommand{\Bin}{{\rm Bin}}
  \newcommand{\Mult}{{\rm Mult}}
  \newcommand{\Aut}{{\rm Aut}}
  \newcommand{\cor}{{\rm cor}}
  \newcommand{\corr}{{\rm corr}}
  \newcommand{\sd}{{\rm sd}}
  \newcommand{\tr}{{\rm tr}}
  \newcommand{\Tr}{{\rm Tr}}
  \newcommand{\rank}{{\rm rank}}
  \newcommand{\diag}{{\rm diag}}
  \newcommand{\dom}{{\rm dom}}
  \newcommand{\id}{{\rm id}}
  \newcommand{\Id}{{\rm\bf I}}
  \newcommand{\Gl}{{\rm Gl}}
  \renewcommand{\th}{\ensuremath{{}^\text{th}} }
  \newcommand{\lag}{\mathcal{L}}
  \newcommand{\inn}{\rfloor}
  \newcommand{\lie}{\pounds}
  \newcommand{\longto}{\longrightarrow}
  \newcommand{\speer}{\parbox{0.4ex}{\raisebox{0.8ex}{$\nearrow$}}}
  \renewcommand{\dag}{ {}^\dagger }
  \newcommand{\blbox}{\rule{1ex}{1ex}}
  \newcommand{\Ji}{J^\sharp}
  \newcommand{\h}{{}^\star}
  \newcommand{\w}{\wedge}
  \newcommand{\too}{\longrightarrow}
  \newcommand{\oot}{\longleftarrow}
  \newcommand{\To}{\Rightarrow}
  \newcommand{\oT}{\Leftarrow}
  \newcommand{\oTo}{\Leftrightarrow}
  \renewcommand{\iff}{~\Longleftrightarrow~}
  \newcommand{\Too}{\;\Longrightarrow\;}
  \newcommand{\oto}{\leftrightarrow}
  \newcommand{\ot}{\leftarrow}
  \newcommand{\ootoo}{\longleftrightarrow}
  \newcommand{\ow}{\stackrel{\circ}\wedge}
  \newcommand{\defeq}{\stackrel{\hspace{0.2ex}{}_\Delta}=}
%  \newcommand{\defeq}{{\overstack\Delta =}}
  \newcommand{\feed}{\nonumber \\}
  \newcommand{\comma}{~,\quad}
  \newcommand{\period}{~.\quad}
  \newcommand{\del}{\partial}
%  \newcommand{\quabla}{\Delta}
  \newcommand{\point}{$\bullet~~$}
  \newcommand{\doubletilde}{ ~ \raisebox{0.3ex}{$\widetilde {}$} \raisebox{0.6ex}{$\widetilde {}$} \!\! }
  \newcommand{\topcirc}{\parbox{0ex}{~\raisebox{2.5ex}{${}^\circ$}}}
  \newcommand{\topdot} {\parbox{0ex}{~\raisebox{2.5ex}{$\cdot$}}}
  \newcommand{\topddot} {\parbox{0ex}{~\raisebox{1.3ex}{$\ddot{~}$}}}
  \newcommand{\sym}{\topcirc}
  \newcommand{\tsum}{\textstyle\sum}
  \newcommand{\st}{\quad\text{s.t.}\quad}

  \newcommand{\half}{\ensuremath{\frac{1}{2}}}
  \newcommand{\third}{\ensuremath{\frac{1}{3}}}
  \newcommand{\fourth}{\ensuremath{\frac{1}{4}}}

  \newcommand{\ubar}{\underline}
  %\renewcommand{\vec}{\underline}
  \renewcommand{\vec}{\boldsymbol}
  %\renewcommand{\_}{\underset}
  %\renewcommand{\^}{\overset}
  %\renewcommand{\*}{{\rm\raisebox{-.6ex}{\text{*}}{}}}
  \renewcommand{\*}{\text{\footnotesize\raisebox{-.4ex}{*}{}}}

  \newcommand{\gto}{{\raisebox{.5ex}{${}_\rightarrow$}}}
  \newcommand{\gfrom}{{\raisebox{.5ex}{${}_\leftarrow$}}}
  \newcommand{\gnto}{{\raisebox{.5ex}{${}_\nrightarrow$}}}
  \newcommand{\gnfrom}{{\raisebox{.5ex}{${}_\nleftarrow$}}}

  %\newcommand{\RND}{{\SS}}
  %\newcommand{\IF}{\text{if }}
  %\newcommand{\AND}{\textsc{and }}
  %\newcommand{\OR}{\textsc{or }}
  %\newcommand{\XOR}{\textsc{xor }}
  %\newcommand{\NOT}{\textsc{not }}

  %\newcommand{\argmax}[1]{{\rm arg}\!\max_{#1}}
  %\newcommand{\argmin}[1]{{\rm arg}\!\min_{#1}}
  \DeclareMathOperator*{\argmax}{argmax}
  \DeclareMathOperator*{\argmin}{argmin}
  \DeclareMathOperator{\sign}{sign}
  \DeclareMathOperator{\acos}{acos}
  \DeclareMathOperator{\unifies}{unifies}
  \DeclareMathOperator{\Span}{span}
  \newcommand{\ortho}{\perp}
  %\newcommand{\argmax}[1]{\underset{~#1}{\text{argmax}}\;}
  %\newcommand{\argmin}[1]{\underset{~#1}{\text{argmin}}\;}
  \newcommand{\ee}[1]{\ensuremath{\cdot10^{#1}}}
  \newcommand{\sub}[1]{\ensuremath{_{\text{#1}}}}
  \newcommand{\up}[1]{\ensuremath{^{\text{#1}}}}
  \newcommand{\kld}[3][{}]{D_{#1}\big(#2\,\big|\!\big|\,#3\big)}
  %\newcommand{\kld}[2]{D\big(#1:#2\big)}
  \newcommand{\sprod}[2]{\big<#1\,,\,#2\big>}
  \newcommand{\End}{\text{End}}
  \newcommand{\txt}[1]{\quad\text{#1}\quad}
  \newcommand{\Over}[2]{\genfrac{}{}{0pt}{0}{#1}{#2}}
  %\newcommand{\mat}[1]{{\bf #1}}
  \newcommand{\arr}[2]{\hspace*{-.5ex}\begin{array}{#1}#2\end{array}\hspace*{-.5ex}}
  \newcommand{\mat}[3][.9]{
    \renewcommand{\arraystretch}{#1}{\scriptscriptstyle{\left(
      \hspace*{-1ex}\begin{array}{#2}#3\end{array}\hspace*{-1ex}
    \right)}}\renewcommand{\arraystretch}{1.2}
  }
  \newcommand{\Mat}[3][.9]{
    \renewcommand{\arraystretch}{#1}{\scriptscriptstyle{\left[
      \hspace*{-1ex}\begin{array}{#2}#3\end{array}\hspace*{-1ex}
    \right]}}\renewcommand{\arraystretch}{1.2}
  }
  \newcommand{\case}[2][ll]{\left\{\arr{#1}{#2}\right.}
  \newcommand{\seq}[1]{\textsf{\<#1\>}}
  \newcommand{\seqq}[1]{\textsf{#1}}
  \newcommand{\floor}[1]{\lfloor#1\rfloor}
  \newcommand{\Exp}[2][]{\text{E}_{#1}\{#2\}}
  \newcommand{\Var}[2][]{\text{Var}_{#1}\{#2\}}
  \newcommand{\cov}[2][]{\text{cov}_{#1}\{#2\}}

%\newcommand{\Exp}[2]{\left\langle{#2}\right\rangle_{#1}}
  \newcommand{\ex}{\setminus}

  \providecommand{\href}[2]{{\color{blue}USE PDFLATEX!}}
  \providecommand{\url}[2]{\href{#1}{{\color{blue}#2}}}
%  \newcommand{\link}[1]{\href{{\protect #1}}{\texttt{\protect #1}}}
  \newcommand{\anchor}[2]{\begin{picture}(0,0)\put(#1){#2}\end{picture}}
  \newcommand{\pagebox}{\begin{picture}(0,0)\put(-3,-23){
    \textcolor[rgb]{.5,1,.5}{\framebox[\textwidth]{\rule[-\textheight]{0pt}{0pt}}}}
    \end{picture}}

  \newcommand{\hide}[1]{
    \begin{list}{}{\leftmargin0ex \rightmargin0ex \topsep0ex \parsep0ex}
       \helvetica{5}{1}{m}{n}
       \renewcommand{\section}{\par SECTION: }
       \renewcommand{\subsection}{\par SUBSECTION: }
       \item[$~~\blacktriangleright$]
       #1%$\blacktriangleleft~~$
       \message{^^JHIDE--Warning!^^J}
    \end{list}
  }
  %\newcommand{\hide}[1]{{\tt[hide:~}{\footnotesize\sf #1}{\tt]}\message{^^JHIDE--Warning!^^J}}
  \newcommand{\Hide}{\renewcommand{\hide}[1]{\message{^^JHIDE--Warning (hidden)!^^J}}}
  \newcommand{\HIDE}{\renewcommand{\hide}[1]{}}
  \newcommand{\fullhide}[1]{\message{^^JHIDE--Warning (hidden)!^^J}}
  \newcommand{\todo}[1]{{\tt[TODO: #1]}\message{^^JTODO--Warning: #1^^J}}
  \newcommand{\Todo}{\renewcommand{\todo}[1]{\message{^^JTODO--Warning (hidden)!^^J}}}
  %\renewcommand{\title}[1]{\renewcommand{\thetitle}{#1}}
  \newcommand{\myauthor}[1]{\author{#1}\newcommand{\theauthor}{#1}}%\@author}
  \newcommand{\mytitle}[1]{\title{#1}\newcommand{\thetitle}{#1}}%\@title}
  \newcommand{\header}{\begin{document}\mytitle\cleardefs}
  \newcommand{\contents}{{\tableofcontents}\renewcommand{\contents}{}}
  \newcommand{\footer}{\small\bibliography{marc,bibs}\end{document}}
  \newcommand{\widepaper}{\usepackage{geometry}\geometry{a4paper,hdivide={25mm,*,25mm},vdivide={25mm,*,25mm}}}
  \newcommand{\moviex}[2]{\movie[externalviewer]{#1}{#2}} %\pdflatex\usepackage{multimedia}
  \newcommand{\rbox}[1]{\fboxrule2mm\fcolorbox[rgb]{1,.85,.85}{1,.85,.85}{#1}}
  \newcommand{\mpage}[2]{{\begin{minipage}{#1\columnwidth}#2\end{minipage}}}
  \newcommand{\redbox}[2]{\fboxrule1mm\fcolorbox[rgb]{1,.7,.7}{1,.7,.7}{\begin{minipage}{#1\columnwidth}\center#2\end{minipage}}}
  \newcommand{\onecol}[2]{
    \begin{minipage}[c]{#1\columnwidth}#2\end{minipage}}
  \newcommand{\twocol}[5][0]{
    \begin{minipage}[c]{#2\columnwidth}#4\end{minipage}\hspace*{#1\columnwidth}%
    \begin{minipage}[c]{#3\columnwidth}#5\end{minipage}}
  \newcommand{\threecol}[7][0]{%
    \begin{minipage}[c]{#2\columnwidth}#5\end{minipage}\hspace*{#1\columnwidth}%
    \begin{minipage}[c]{#3\columnwidth}#6\end{minipage}\hspace*{#1\columnwidth}%
    \begin{minipage}[c]{#4\columnwidth}#7\end{minipage}}
  \newcommand{\threecoltext}[7][c]{
    \begin{minipage}[#1]{#2\textwidth}#5\end{minipage}%
    \begin{minipage}[#1]{#3\textwidth}#6\end{minipage}%
    \begin{minipage}[#1]{#4\textwidth}#7\end{minipage}}
  \newcommand{\threecoltop}[7][0]{%
   \begin{minipage}[t]{#2\columnwidth}#5\end{minipage}\hspace*{#1\columnwidth}%
   \begin{minipage}[t]{#3\columnwidth}#6\end{minipage}\hspace*{#1\columnwidth}%
   \begin{minipage}[t]{#4\columnwidth}#7\end{minipage}}
  \newcommand{\fourcol}[9][0]{%
   \begin{minipage}[c]{#2\columnwidth}#6\end{minipage}\hspace*{#1\columnwidth}%
   \begin{minipage}[c]{#3\columnwidth}#7\end{minipage}\hspace*{#1\columnwidth}%
   \begin{minipage}[c]{#4\columnwidth}#8\end{minipage}\hspace*{#1\columnwidth}%
   \begin{minipage}[c]{#5\columnwidth}#9\end{minipage}}
  \newcommand{\helvetica}[4]{\setlength{\unitlength}{1pt}\fontsize{#1}{#1}\linespread{#2}\usefont{OT1}{phv}{#3}{#4}}
  \newcommand{\helve}[1]{\helvetica{#1}{1.5}{m}{n}}
  \newcommand{\german}{\usepackage[german]{babel}\usepackage[utf8]{inputenc}}

\newcommand{\norm}[1]{|\!|#1|\!|}
\newcommand{\expr}[1]{[\hspace{-.2ex}[#1]\hspace{-.2ex}]}

\newcommand{\Jwi}{J^\sharp_W}
\newcommand{\THi}{T^\sharp_H}
\newcommand{\Jci}{J^\natural_C}
\newcommand{\hJi}{{\bar J}^\sharp}
\renewcommand{\|}{\,|\,}
\renewcommand{\=}{\!=\!}
\newcommand{\myminus}{{\hspace*{-.0pt}\text{\rm -}\hspace*{-.5pt}}}
\newcommand{\myplus}{{\hspace*{-.0pt}\text{\rm +}\hspace*{-.5pt}}}
\newcommand{\1}{{\myminus1}}
\newcommand{\2}{{\myminus2}}
\newcommand{\3}{{\myminus3}}
\newcommand{\mT}{{\text{\rm -}\hspace*{-1pt}\top}}
\newcommand{\po}{{\myplus1}}
\newcommand{\pt}{{\myplus2}}
%\renewcommand{\-}{\myminus}
%\newcommand{\+}{\myplus}
\renewcommand{\T}{{\!\top\!}}
\newcommand{\xT}{{\underline x}}
\newcommand{\uT}{{\underline u}}
\newcommand{\zT}{{\underline z}}
\newcommand{\Sum}{\textstyle\sum}
\newcommand{\Int}{\textstyle\int}
\newcommand{\Prod}{\textstyle\prod}


\newenvironment{centy}{
\vspace{15mm}
\large
\hspace*{5mm}
\begin{minipage}{8cm}\it\color{blue}
}{
\end{minipage}
}

\newcommand{\old}{{\text{old}}}
\newcommand{\new}{{\text{new}}}
\newcommand{\MAP}{{\text{MAP}}}
\newcommand{\ML}{{\text{ML}}}

\newcommand{\redArrow}{\quad\anchor{0,-1}{\includegraphics[scale=.5]{figs/redArrow}}}
\newcommand{\pub}[1]{{\color{green}\helvetica{8}{1.3}{m}{n}#1\\}}
\DeclareMathOperator{\opKL}{KL}
\newcommand{\KL}[2]{\opKL\big(#1\,\big|\!\big|\,#2\big)} %\left(#1 |\!| #2\right)}

\renewcommand{\show}[2][.8]{\centerline{\includegraphics[width=#1\columnwidth]{#2}}}
\newcommand{\showh}[2][.8]{\includegraphics[width=#1\columnwidth]{#2}}
\newcommand{\shows}[2][.8]{\centerline{\includegraphics[scale=#1]{#2}}}
\newcommand{\showhs}[2][.8]{\includegraphics[scale=#1]{#2}}
\newcommand{\mov}[2]{\movie[externalviewer]{{\color{blue}\small #1}}{movies/#2}}
\newcommand{\movex}[2]{\movie[externalviewer]{#1}{#2}} %\pdflatex\usepackage{multimedia}
%\newcommand{\movgb}[1]{\hfill\movie[externalviewer]{\small[movie]}{/home/mtoussai/movies/10-goalDirectedBehavior/#1}}
\newcommand{\movh}[3][loop]{
\movie[#1]{\showh[#2]{movies/#3.png}}{movies/#3.avi}%
\movie[externalviewer]{$\circ$}{movies/#3.avi}
}
\newcommand{\movc}[3][loop]{\centerline{\movh[#1]{#2}{#3}}}
\newcommand{\cen}[1]{\centerline{#1}}

\newcommand{\citing}[1]{
{\color{citcol}\tiny#1\par}
}

\newcommand{\cit}[3]{
\par\smallskip
{\color{greencol}\tiny #1: \emph{#2}. #3 \par}
}

\newcommand{\citurl}[4]{
\par\smallskip
{\color{greencol}\tiny #1: \protect{\href{#4}{\color{blue}{#2.}}} #3 \par}
}

\newcommand{\cito}[3]{
\par\smallskip
{\color{bluecol}\tiny #1: \emph{#2}. #3 \par}
}

\newcommand{\redoMacrosInProof}{
  \renewcommand{\d}{\delta}
%  \renewcommand{\|}{\,|\,}
  \renewcommand{\=}{\!=\!}
}

%% \makeatletter
%% \newenvironment{code}{%
%%   \begin{lrbox}{\@tempboxa}\begin{minipage}{1\columnwidth}\codefont
%% }{
%%   \end{minipage}\end{lrbox}%
%%   \colorbox[rgb]{.95,.95,.95}{\usebox{\@tempboxa}}
%% }\makeatother

\newenvironment{code}{%
\codefont
\begin{shaded}
}{
\end{shaded}
}

%\newcommand{\refeq}[1]{(\ref{#1})}

\usepackage{algorithm}
\usepackage{algpseudocode}
\algrenewcommand{\algorithmicrequire}{\textbf{Input:~~}}
\algrenewcommand{\algorithmicensure}{\textbf{Output:}}
\algrenewcommand{\algorithmiccomment}[1]{\qquad\hfill~\hspace*{-5ex}\textit{// #1}}
\algrenewcommand{\alglinenumber}[1]{\helvetica{6}{1.3}{m}{n}#1:}

\newenvironment{algo}[1][8]{
\quad\begin{minipage}{.8\columnwidth}\helvetica{#1}{1.3}{m}{n}
\medskip\hrule\medskip
\begin{algorithmic}[1]
}{
\end{algorithmic}
\medskip\hrule\medskip
\end{minipage}
}

\usepackage{etoolbox}

%%%%%%%%%%%%%%%%%%%%%%%%%%%%%%%%%%%%%%%%%%%%%%%%%%%%%%%%%%%%%%%%%%%%%%%%%%%%%%%%

\usepackage{multirow}
\usepackage{colortbl}
%\setlength{\jot}{0pt}
%\setlength{\mathindent}{1ex}
\usepackage{empheq}

%%%%%%%%%%%%%%%%%%%%%%%%%%%%%%%%%%%%%%%%%%%%%%%%%%%%%%%%%%%%%%%%%%%%%%%%%%%%%%%

\newcommand{\mypause}{\pause}
%\newcommand{\dom}{{\text{dom}}}
\newcommand{\defi}[1]{\textbf{#1}}
\newcommand{\red}[1]{\emph{\color{red}#1}}
%\newcommand{\ul}{\underline}
\newcommand{\pos}{{\textsf{pos}}}
\newcommand{\eff}{{\textsf{eff}}}
\newcommand{\rot}{{\textsf{rot}}}
\newcommand{\veC}{{\textsf{vec}}}
\newcommand{\quat}{{\textsf{quat}}}
\newcommand{\col}{{\textsf{col}}}
\newcommand{\de}[2]{\frac{\partial #1}{\partial #2}}
\newcommand{\target}{{\text{target}}}
\newcommand{\near}{{\text{near}}}
\newcommand{\qfree}{Q_{\text{free}}}
\renewcommand{\vec}{\boldsymbol}
\newcommand{\lft}{\text{left}}
\newcommand{\rgh}{\text{right}}
\DeclareMathOperator{\real}{real}
\newcommand{\prev}{{\text{prev}}}
\newcommand{\TR}[2]{T_{{#1}\to{#2}}}
\newcommand{\RO}[2]{R_{{#1}\to{#2}}}
\newcommand{\liter}{\helvetica{8}{1.1}{m}{n}\parskip 1ex}
\newcommand{\Fc}{\color{green}F}
\newcommand{\muc}{\color{blue}\mu}
\newcommand{\Astar}{A$^*$}

%for AI course:
\newcommand{\defn}[1]{\textbf{#1}}

%for optimization course:
\newcommand{\adec}{\r_\a^-}
\newcommand{\ainc}{\r_\a^+}
\newcommand{\ldec}{\r_\l^-}
\newcommand{\linc}{\r_\l^+}
\newcommand{\minc}{\r_\m^+}
\newcommand{\mdec}{\r_\m^-}
\newcommand{\lsstop}{\r_{\text{ls}}}


\definecolor{boxcol}{rgb}{.85,.9,.92}
\newcommand{\eqbox}[1]{\centerline{\fboxrule0mm\fcolorbox{boxcol}{boxcol}{#1}}}
\newcommand{\movgb}[1]{\hfill\movie[externalviewer]{\small[movie]}{/home/mtoussai/movies/10-goalDirectedBehavior/#1}}
\newcommand{\demo}[1]{{{\color{blue}[\small #1]}}}

\graphicspath{{../pics-robotics/}{../pics-ML/}{../pics-all/}{../pics-all2/}{../pics-Optim/}}
\DeclareGraphicsExtensions{.pdf,.png,.jpg}

%\usepackage{pdfpages}
%\setbeamercolor{background canvas}{bg=}

\newcommand{\SUM}{\texttt{sum}}
\usepackage{float}

%% prevent pagebreaks before environment
\makeatletter
\newcommand{\NewParNoBreak}[1][\parskip]{\par\vspace*{-\parskip}\vspace*{#1}\nobreak\@afterheading}
\makeatother

%\newcommand{\idx}[2]{\label{IKgn}}

%%%%%%%%%%%%%%%%%%%%%%%%%%%%%%%%%%%%%%%%%%%%%%%%%%%%%%%%%%%%%%%%%%%%%%%%%%%%%%%%



%% \newwrite\tempfile
%% \immediate\openout\tempfile=z.keys.tex

%% \renewcommand{\key}[1]{
%% %%   \addtocounter{mypage}{1}
%% \makeatletter
%% \immediate\write\tempfile{\symbol{`\\}}
%% \makeatother
%%   \immediate\write\tempfile{hyperref[key:#1]{#1(\arabic{mypage})}}
%% %%  % \phantomsection\label{key:#1}
%% %%   %\index{#1@{\hyperref[key:#1]{#1 (\arabic{mysec}:\arabic{mypage})}}|phantom}
%% %%   \addtocounter{mypage}{-1}
%% }


  \graphicspath{{pics/}{../shared/pics/}}

  \title{\course \topic}
  \author{Marc Toussaint}
  \institute{Machine Learning \& Robotics Lab, U Stuttgart}

  \begin{document}

  \rfoot{\vspace*{-5mm}{\tiny
  \textsf{\arabic{mypage}/\pageref{lastpage}}}\hspace*{-4mm}}

  %% title slide!
  \slide{}{
    \thispagestyle{empty}

    \twocol{.27}{.6}{
      \vspace*{-5mm}\hspace*{-15mm}\includegraphics[width=4cm]{\coursepicture}
    }{\center

      \textbf{\fontsize{17}{20}\selectfont \course}

      ~

      %Lecture
      \topic\\

      \vspace{1cm}

      {\tiny~\emph{\keywords}~\\}

      \vspace{1cm}

      Marc Toussaint
      
      University of Stuttgart

      \coursedate

      ~

    }
  }
}

\newcommand{\slide}[2]{
  \slidefont
  \incpage\begin{frame}
  \addcontentsline{toc}{section}{#1}
  \vfill
  {\headerfont #1} \vspace*{-2ex}
  \begin{itemize}\item[]~\\
    #2
  \end{itemize}
  \vfill
  \end{frame}
}

\newenvironment{slidecore}[1]{
  \slidefont\incpage
  \addcontentsline{toc}{section}{#1}
  \vfill
  {\headerfont #1} \vspace*{-2ex}
  \begin{itemize}\item[]~\\
}{
  \end{itemize}
  \vfill
}


\providecommand{\key}[1]{
  \addtocounter{mypage}{1}
% \immediate\write\keyfile{#1}
  \addtocontents{toc}{\hyperref[key:#1]{#1 (\arabic{mypage})}}
%  \phantomsection\label{key:#1}
%  \index{#1@{\hyperref[key:#1]{#1 (\arabic{mysec}:\arabic{mypage})}}|phantom}
  \addtocounter{mypage}{-1}
}

\providecommand{\course}{}

\providecommand{\subtopic}{}

\providecommand{\sublecture}[2]{
  \renewcommand{\subtopic}{#1}
  \slide{#1}{#2}
}

\providecommand{\story}[1]{
~

Motivation: {\tiny #1}\clearpage
}

\newenvironment{items}[1][9]{
\par\setlength{\unitlength}{1pt}\fontsize{#1}{#1}\linespread{1.2}\selectfont
\begin{list}{--}{\leftmargin4ex \rightmargin0ex \labelsep1ex \labelwidth2ex
\topsep0pt \parsep0ex \itemsep3pt}
}{
\end{list}
}

\providecommand{\slidesfoot}{
  \end{document}
}


  \slideshead
}

\providecommand{\exercises}{
  \newcommand{\exerciseshead}{
  \documentclass[10pt,fleqn]{article}
  \stdpackages

  \definecolor{bluecol}{rgb}{0,0,.5}
  \definecolor{greencol}{rgb}{0,.4,0}
  \definecolor{shadecolor}{gray}{0.9}
  \usepackage[
    %    pdftex%,
    %%    letterpaper,
    %    bookmarks,
    %    bookmarksnumbered,
    colorlinks,
    urlcolor=bluecol,
    citecolor=black,
    linkcolor=bluecol,
    %    pagecolor=bluecol,
    pdfborder={0 0 0},
    %pdfborderstyle={/S/U/W 1},
    %%    backref,     %link from bibliography back to sections
    %%    pagebackref, %link from bibliography back to pages
    %%    pdfstartview=FitH, %fitwidth instead of fit window
    pdfpagemode=UseNone, %UseOutlines, %bookmarks are displayed by acrobat
    pdftitle={\course},
    pdfauthor={Marc Toussaint},
    pdfkeywords={}
  ]{hyperref}
  \DeclareGraphicsExtensions{.pdf,.png,.jpg,.eps}

  \renewcommand{\r}{\varrho}
  \renewcommand{\l}{\lambda}
  \renewcommand{\L}{\Lambda}
  \renewcommand{\b}{\beta}
  \renewcommand{\d}{\delta}
  \renewcommand{\k}{\kappa}
  \renewcommand{\t}{\theta}
  \renewcommand{\O}{\Omega}
  \renewcommand{\o}{\omega}
  \renewcommand{\SS}{{\cal S}}
  \renewcommand{\=}{\!=\!}
  %\renewcommand{\boldsymbol}{}
  %\renewcommand{\Chapter}{\chapter}
  %\renewcommand{\Subsection}{\subsection}

  \renewcommand{\baselinestretch}{1.1}
  \geometry{a4paper,headsep=7mm,hdivide={15mm,*,15mm},vdivide={20mm,*,15mm}}

  \fancyhead[L]{\thetitle, \textit{Marc Toussaint}---\coursedate}
  \fancyhead[R]{\thepage}
  \fancyhead[C]{}
  \fancyfoot{}
  \pagestyle{fancy}

  \parindent 0pt
  \parskip 0.5ex

  \newcommand{\codefont}{\helvetica{8}{1.2}{m}{n}}

  %auto-ignore
  \renewcommand{\a}{\alpha}
  \renewcommand{\b}{\beta}
  \renewcommand{\d}{\delta}
    \newcommand{\D}{\Delta}
    \newcommand{\e}{\epsilon}
    \newcommand{\g}{\gamma}
    \newcommand{\G}{\Gamma}
  \renewcommand{\l}{\lambda}
  \renewcommand{\L}{\Lambda}
    \newcommand{\m}{\mu}
    \newcommand{\n}{\nu}
    \newcommand{\N}{\nabla}
  \renewcommand{\k}{\kappa}
  \renewcommand{\o}{\omega}
  \renewcommand{\O}{\Omega}
    \newcommand{\p}{\phi}
    \newcommand{\ph}{\varphi}
  \renewcommand{\P}{\Phi}
  \renewcommand{\r}{\varrho}
    \newcommand{\s}{\sigma}
  \renewcommand{\S}{\Sigma}
  \renewcommand{\t}{\theta}
    \newcommand{\T}{\Theta}
  %\renewcommand{\v}{\vartheta}
    \newcommand{\x}{\xi}
    \newcommand{\X}{\Xi}
    \newcommand{\Y}{\Upsilon}
    \newcommand{\z}{\zeta}

  \renewcommand{\AA}{{\cal A}}
    \newcommand{\BB}{{\cal B}}
    \newcommand{\CC}{{\cal C}}
    \newcommand{\cc}{{\cal c}}
    \newcommand{\DD}{{\cal D}}
    \newcommand{\EE}{{\cal E}}
    \newcommand{\FF}{{\cal F}}
    \newcommand{\GG}{{\cal G}}
    \newcommand{\HH}{{\cal H}}
    \newcommand{\II}{{\cal I}}
    \newcommand{\KK}{{\cal K}}
    \newcommand{\LL}{{\cal L}}
    \newcommand{\MM}{{\cal M}}
    \newcommand{\NN}{{\cal N}}
    \newcommand{\oNN}{\overline\NN}
    \newcommand{\OO}{{\cal O}}
    \newcommand{\PP}{{\cal P}}
    \newcommand{\QQ}{{\cal Q}}
    \newcommand{\RR}{{\cal R}}
  \renewcommand{\SS}{{\cal S}}
    \newcommand{\TT}{{\cal T}}
    \newcommand{\uu}{{\cal u}}
    \newcommand{\UU}{{\cal U}}
    \newcommand{\VV}{{\cal V}}
    \newcommand{\XX}{{\cal X}}
    \newcommand{\xx}{\mathcal{x}}
    \newcommand{\YY}{{\cal Y}}
    \newcommand{\SOSO}{{\cal SO}}
    \newcommand{\GLGL}{{\cal GL}}

    \newcommand{\Ee}{{\rm E}}

  \newcommand{\NNN}{{\mathbb{N}}}
  \newcommand{\III}{{\mathbb{I}}}
  \newcommand{\ZZZ}{{\mathbb{Z}}}
  %\newcommand{\RRR}{{\mathrm{I\!R}}}
  \newcommand{\RRR}{{\mathbb{R}}}
  \newcommand{\SSS}{{\mathbb{S}}}
  \newcommand{\CCC}{{\mathbb{C}}}
  \newcommand{\DDD}{{\mathbb{D}}}
  \newcommand{\one}{{{\bf 1}}}
  \newcommand{\eee}{\text{e}}

  \newcommand{\NNNN}{{\overline{\cal N}}}

  \renewcommand{\[}{\Big[}
  \renewcommand{\]}{\Big]}
  \renewcommand{\(}{\Big(}
  \renewcommand{\)}{\Big)}
  \renewcommand{\|}{\,|\,}
  \renewcommand{\;}{\,;\,}
  \renewcommand{\=}{\!=\!}
    \newcommand{\<}{\left\langle}
  \renewcommand{\>}{\right\rangle}

  \newcommand{\na}[1][]{{\nabla_{\!\!#1}}}
  \newcommand{\he}[1][]{{\nabla_{\!\!#1}^2}}
  \newcommand{\Prob}{{\rm Prob}}
  \newcommand{\Dir}{{\rm Dir}}
  \newcommand{\Beta}{{\rm Beta}}
  \newcommand{\Bern}{{\rm Bern}}
  \newcommand{\Bin}{{\rm Bin}}
  \newcommand{\Mult}{{\rm Mult}}
  \newcommand{\Aut}{{\rm Aut}}
  \newcommand{\cor}{{\rm cor}}
  \newcommand{\corr}{{\rm corr}}
  \newcommand{\sd}{{\rm sd}}
  \newcommand{\tr}{{\rm tr}}
  \newcommand{\Tr}{{\rm Tr}}
  \newcommand{\rank}{{\rm rank}}
  \newcommand{\diag}{{\rm diag}}
  \newcommand{\dom}{{\rm dom}}
  \newcommand{\id}{{\rm id}}
  \newcommand{\Id}{{\rm\bf I}}
  \newcommand{\Gl}{{\rm Gl}}
  \renewcommand{\th}{\ensuremath{{}^\text{th}} }
  \newcommand{\lag}{\mathcal{L}}
  \newcommand{\inn}{\rfloor}
  \newcommand{\lie}{\pounds}
  \newcommand{\longto}{\longrightarrow}
  \newcommand{\speer}{\parbox{0.4ex}{\raisebox{0.8ex}{$\nearrow$}}}
  \renewcommand{\dag}{ {}^\dagger }
  \newcommand{\blbox}{\rule{1ex}{1ex}}
  \newcommand{\Ji}{J^\sharp}
  \newcommand{\h}{{}^\star}
  \newcommand{\w}{\wedge}
  \newcommand{\too}{\longrightarrow}
  \newcommand{\oot}{\longleftarrow}
  \newcommand{\To}{\Rightarrow}
  \newcommand{\oT}{\Leftarrow}
  \newcommand{\oTo}{\Leftrightarrow}
  \renewcommand{\iff}{~\Longleftrightarrow~}
  \newcommand{\Too}{\;\Longrightarrow\;}
  \newcommand{\oto}{\leftrightarrow}
  \newcommand{\ot}{\leftarrow}
  \newcommand{\ootoo}{\longleftrightarrow}
  \newcommand{\ow}{\stackrel{\circ}\wedge}
  \newcommand{\defeq}{\stackrel{\hspace{0.2ex}{}_\Delta}=}
%  \newcommand{\defeq}{{\overstack\Delta =}}
  \newcommand{\feed}{\nonumber \\}
  \newcommand{\comma}{~,\quad}
  \newcommand{\period}{~.\quad}
  \newcommand{\del}{\partial}
%  \newcommand{\quabla}{\Delta}
  \newcommand{\point}{$\bullet~~$}
  \newcommand{\doubletilde}{ ~ \raisebox{0.3ex}{$\widetilde {}$} \raisebox{0.6ex}{$\widetilde {}$} \!\! }
  \newcommand{\topcirc}{\parbox{0ex}{~\raisebox{2.5ex}{${}^\circ$}}}
  \newcommand{\topdot} {\parbox{0ex}{~\raisebox{2.5ex}{$\cdot$}}}
  \newcommand{\topddot} {\parbox{0ex}{~\raisebox{1.3ex}{$\ddot{~}$}}}
  \newcommand{\sym}{\topcirc}
  \newcommand{\tsum}{\textstyle\sum}
  \newcommand{\st}{\quad\text{s.t.}\quad}

  \newcommand{\half}{\ensuremath{\frac{1}{2}}}
  \newcommand{\third}{\ensuremath{\frac{1}{3}}}
  \newcommand{\fourth}{\ensuremath{\frac{1}{4}}}

  \newcommand{\ubar}{\underline}
  %\renewcommand{\vec}{\underline}
  \renewcommand{\vec}{\boldsymbol}
  %\renewcommand{\_}{\underset}
  %\renewcommand{\^}{\overset}
  %\renewcommand{\*}{{\rm\raisebox{-.6ex}{\text{*}}{}}}
  \renewcommand{\*}{\text{\footnotesize\raisebox{-.4ex}{*}{}}}

  \newcommand{\gto}{{\raisebox{.5ex}{${}_\rightarrow$}}}
  \newcommand{\gfrom}{{\raisebox{.5ex}{${}_\leftarrow$}}}
  \newcommand{\gnto}{{\raisebox{.5ex}{${}_\nrightarrow$}}}
  \newcommand{\gnfrom}{{\raisebox{.5ex}{${}_\nleftarrow$}}}

  %\newcommand{\RND}{{\SS}}
  %\newcommand{\IF}{\text{if }}
  %\newcommand{\AND}{\textsc{and }}
  %\newcommand{\OR}{\textsc{or }}
  %\newcommand{\XOR}{\textsc{xor }}
  %\newcommand{\NOT}{\textsc{not }}

  %\newcommand{\argmax}[1]{{\rm arg}\!\max_{#1}}
  %\newcommand{\argmin}[1]{{\rm arg}\!\min_{#1}}
  \DeclareMathOperator*{\argmax}{argmax}
  \DeclareMathOperator*{\argmin}{argmin}
  \DeclareMathOperator{\sign}{sign}
  \DeclareMathOperator{\acos}{acos}
  \DeclareMathOperator{\unifies}{unifies}
  \DeclareMathOperator{\Span}{span}
  \newcommand{\ortho}{\perp}
  %\newcommand{\argmax}[1]{\underset{~#1}{\text{argmax}}\;}
  %\newcommand{\argmin}[1]{\underset{~#1}{\text{argmin}}\;}
  \newcommand{\ee}[1]{\ensuremath{\cdot10^{#1}}}
  \newcommand{\sub}[1]{\ensuremath{_{\text{#1}}}}
  \newcommand{\up}[1]{\ensuremath{^{\text{#1}}}}
  \newcommand{\kld}[3][{}]{D_{#1}\big(#2\,\big|\!\big|\,#3\big)}
  %\newcommand{\kld}[2]{D\big(#1:#2\big)}
  \newcommand{\sprod}[2]{\big<#1\,,\,#2\big>}
  \newcommand{\End}{\text{End}}
  \newcommand{\txt}[1]{\quad\text{#1}\quad}
  \newcommand{\Over}[2]{\genfrac{}{}{0pt}{0}{#1}{#2}}
  %\newcommand{\mat}[1]{{\bf #1}}
  \newcommand{\arr}[2]{\hspace*{-.5ex}\begin{array}{#1}#2\end{array}\hspace*{-.5ex}}
  \newcommand{\mat}[3][.9]{
    \renewcommand{\arraystretch}{#1}{\scriptscriptstyle{\left(
      \hspace*{-1ex}\begin{array}{#2}#3\end{array}\hspace*{-1ex}
    \right)}}\renewcommand{\arraystretch}{1.2}
  }
  \newcommand{\Mat}[3][.9]{
    \renewcommand{\arraystretch}{#1}{\scriptscriptstyle{\left[
      \hspace*{-1ex}\begin{array}{#2}#3\end{array}\hspace*{-1ex}
    \right]}}\renewcommand{\arraystretch}{1.2}
  }
  \newcommand{\case}[2][ll]{\left\{\arr{#1}{#2}\right.}
  \newcommand{\seq}[1]{\textsf{\<#1\>}}
  \newcommand{\seqq}[1]{\textsf{#1}}
  \newcommand{\floor}[1]{\lfloor#1\rfloor}
  \newcommand{\Exp}[2][]{\text{E}_{#1}\{#2\}}
  \newcommand{\Var}[2][]{\text{Var}_{#1}\{#2\}}
  \newcommand{\cov}[2][]{\text{cov}_{#1}\{#2\}}

%\newcommand{\Exp}[2]{\left\langle{#2}\right\rangle_{#1}}
  \newcommand{\ex}{\setminus}

  \providecommand{\href}[2]{{\color{blue}USE PDFLATEX!}}
  \providecommand{\url}[2]{\href{#1}{{\color{blue}#2}}}
%  \newcommand{\link}[1]{\href{{\protect #1}}{\texttt{\protect #1}}}
  \newcommand{\anchor}[2]{\begin{picture}(0,0)\put(#1){#2}\end{picture}}
  \newcommand{\pagebox}{\begin{picture}(0,0)\put(-3,-23){
    \textcolor[rgb]{.5,1,.5}{\framebox[\textwidth]{\rule[-\textheight]{0pt}{0pt}}}}
    \end{picture}}

  \newcommand{\hide}[1]{
    \begin{list}{}{\leftmargin0ex \rightmargin0ex \topsep0ex \parsep0ex}
       \helvetica{5}{1}{m}{n}
       \renewcommand{\section}{\par SECTION: }
       \renewcommand{\subsection}{\par SUBSECTION: }
       \item[$~~\blacktriangleright$]
       #1%$\blacktriangleleft~~$
       \message{^^JHIDE--Warning!^^J}
    \end{list}
  }
  %\newcommand{\hide}[1]{{\tt[hide:~}{\footnotesize\sf #1}{\tt]}\message{^^JHIDE--Warning!^^J}}
  \newcommand{\Hide}{\renewcommand{\hide}[1]{\message{^^JHIDE--Warning (hidden)!^^J}}}
  \newcommand{\HIDE}{\renewcommand{\hide}[1]{}}
  \newcommand{\fullhide}[1]{\message{^^JHIDE--Warning (hidden)!^^J}}
  \newcommand{\todo}[1]{{\tt[TODO: #1]}\message{^^JTODO--Warning: #1^^J}}
  \newcommand{\Todo}{\renewcommand{\todo}[1]{\message{^^JTODO--Warning (hidden)!^^J}}}
  %\renewcommand{\title}[1]{\renewcommand{\thetitle}{#1}}
  \newcommand{\myauthor}[1]{\author{#1}\newcommand{\theauthor}{#1}}%\@author}
  \newcommand{\mytitle}[1]{\title{#1}\newcommand{\thetitle}{#1}}%\@title}
  \newcommand{\header}{\begin{document}\mytitle\cleardefs}
  \newcommand{\contents}{{\tableofcontents}\renewcommand{\contents}{}}
  \newcommand{\footer}{\small\bibliography{marc,bibs}\end{document}}
  \newcommand{\widepaper}{\usepackage{geometry}\geometry{a4paper,hdivide={25mm,*,25mm},vdivide={25mm,*,25mm}}}
  \newcommand{\moviex}[2]{\movie[externalviewer]{#1}{#2}} %\pdflatex\usepackage{multimedia}
  \newcommand{\rbox}[1]{\fboxrule2mm\fcolorbox[rgb]{1,.85,.85}{1,.85,.85}{#1}}
  \newcommand{\mpage}[2]{{\begin{minipage}{#1\columnwidth}#2\end{minipage}}}
  \newcommand{\redbox}[2]{\fboxrule1mm\fcolorbox[rgb]{1,.7,.7}{1,.7,.7}{\begin{minipage}{#1\columnwidth}\center#2\end{minipage}}}
  \newcommand{\onecol}[2]{
    \begin{minipage}[c]{#1\columnwidth}#2\end{minipage}}
  \newcommand{\twocol}[5][0]{
    \begin{minipage}[c]{#2\columnwidth}#4\end{minipage}\hspace*{#1\columnwidth}%
    \begin{minipage}[c]{#3\columnwidth}#5\end{minipage}}
  \newcommand{\threecol}[7][0]{%
    \begin{minipage}[c]{#2\columnwidth}#5\end{minipage}\hspace*{#1\columnwidth}%
    \begin{minipage}[c]{#3\columnwidth}#6\end{minipage}\hspace*{#1\columnwidth}%
    \begin{minipage}[c]{#4\columnwidth}#7\end{minipage}}
  \newcommand{\threecoltext}[7][c]{
    \begin{minipage}[#1]{#2\textwidth}#5\end{minipage}%
    \begin{minipage}[#1]{#3\textwidth}#6\end{minipage}%
    \begin{minipage}[#1]{#4\textwidth}#7\end{minipage}}
  \newcommand{\threecoltop}[7][0]{%
   \begin{minipage}[t]{#2\columnwidth}#5\end{minipage}\hspace*{#1\columnwidth}%
   \begin{minipage}[t]{#3\columnwidth}#6\end{minipage}\hspace*{#1\columnwidth}%
   \begin{minipage}[t]{#4\columnwidth}#7\end{minipage}}
  \newcommand{\fourcol}[9][0]{%
   \begin{minipage}[c]{#2\columnwidth}#6\end{minipage}\hspace*{#1\columnwidth}%
   \begin{minipage}[c]{#3\columnwidth}#7\end{minipage}\hspace*{#1\columnwidth}%
   \begin{minipage}[c]{#4\columnwidth}#8\end{minipage}\hspace*{#1\columnwidth}%
   \begin{minipage}[c]{#5\columnwidth}#9\end{minipage}}
  \newcommand{\helvetica}[4]{\setlength{\unitlength}{1pt}\fontsize{#1}{#1}\linespread{#2}\usefont{OT1}{phv}{#3}{#4}}
  \newcommand{\helve}[1]{\helvetica{#1}{1.5}{m}{n}}
  \newcommand{\german}{\usepackage[german]{babel}\usepackage[utf8]{inputenc}}

\newcommand{\norm}[1]{|\!|#1|\!|}
\newcommand{\expr}[1]{[\hspace{-.2ex}[#1]\hspace{-.2ex}]}

\newcommand{\Jwi}{J^\sharp_W}
\newcommand{\THi}{T^\sharp_H}
\newcommand{\Jci}{J^\natural_C}
\newcommand{\hJi}{{\bar J}^\sharp}
\renewcommand{\|}{\,|\,}
\renewcommand{\=}{\!=\!}
\newcommand{\myminus}{{\hspace*{-.0pt}\text{\rm -}\hspace*{-.5pt}}}
\newcommand{\myplus}{{\hspace*{-.0pt}\text{\rm +}\hspace*{-.5pt}}}
\newcommand{\1}{{\myminus1}}
\newcommand{\2}{{\myminus2}}
\newcommand{\3}{{\myminus3}}
\newcommand{\mT}{{\text{\rm -}\hspace*{-1pt}\top}}
\newcommand{\po}{{\myplus1}}
\newcommand{\pt}{{\myplus2}}
%\renewcommand{\-}{\myminus}
%\newcommand{\+}{\myplus}
\renewcommand{\T}{{\!\top\!}}
\newcommand{\xT}{{\underline x}}
\newcommand{\uT}{{\underline u}}
\newcommand{\zT}{{\underline z}}
\newcommand{\Sum}{\textstyle\sum}
\newcommand{\Int}{\textstyle\int}
\newcommand{\Prod}{\textstyle\prod}


\newenvironment{centy}{
\vspace{15mm}
\large
\hspace*{5mm}
\begin{minipage}{8cm}\it\color{blue}
}{
\end{minipage}
}

\newcommand{\old}{{\text{old}}}
\newcommand{\new}{{\text{new}}}
\newcommand{\MAP}{{\text{MAP}}}
\newcommand{\ML}{{\text{ML}}}

\newcommand{\redArrow}{\quad\anchor{0,-1}{\includegraphics[scale=.5]{figs/redArrow}}}
\newcommand{\pub}[1]{{\color{green}\helvetica{8}{1.3}{m}{n}#1\\}}
\DeclareMathOperator{\opKL}{KL}
\newcommand{\KL}[2]{\opKL\big(#1\,\big|\!\big|\,#2\big)} %\left(#1 |\!| #2\right)}

\renewcommand{\show}[2][.8]{\centerline{\includegraphics[width=#1\columnwidth]{#2}}}
\newcommand{\showh}[2][.8]{\includegraphics[width=#1\columnwidth]{#2}}
\newcommand{\shows}[2][.8]{\centerline{\includegraphics[scale=#1]{#2}}}
\newcommand{\showhs}[2][.8]{\includegraphics[scale=#1]{#2}}
\newcommand{\mov}[2]{\movie[externalviewer]{{\color{blue}\small #1}}{movies/#2}}
\newcommand{\movex}[2]{\movie[externalviewer]{#1}{#2}} %\pdflatex\usepackage{multimedia}
%\newcommand{\movgb}[1]{\hfill\movie[externalviewer]{\small[movie]}{/home/mtoussai/movies/10-goalDirectedBehavior/#1}}
\newcommand{\movh}[3][loop]{
\movie[#1]{\showh[#2]{movies/#3.png}}{movies/#3.avi}%
\movie[externalviewer]{$\circ$}{movies/#3.avi}
}
\newcommand{\movc}[3][loop]{\centerline{\movh[#1]{#2}{#3}}}
\newcommand{\cen}[1]{\centerline{#1}}

\newcommand{\citing}[1]{
{\color{citcol}\tiny#1\par}
}

\newcommand{\cit}[3]{
\par\smallskip
{\color{greencol}\tiny #1: \emph{#2}. #3 \par}
}

\newcommand{\citurl}[4]{
\par\smallskip
{\color{greencol}\tiny #1: \protect{\href{#4}{\color{blue}{#2.}}} #3 \par}
}

\newcommand{\cito}[3]{
\par\smallskip
{\color{bluecol}\tiny #1: \emph{#2}. #3 \par}
}

\newcommand{\redoMacrosInProof}{
  \renewcommand{\d}{\delta}
%  \renewcommand{\|}{\,|\,}
  \renewcommand{\=}{\!=\!}
}

%% \makeatletter
%% \newenvironment{code}{%
%%   \begin{lrbox}{\@tempboxa}\begin{minipage}{1\columnwidth}\codefont
%% }{
%%   \end{minipage}\end{lrbox}%
%%   \colorbox[rgb]{.95,.95,.95}{\usebox{\@tempboxa}}
%% }\makeatother

\newenvironment{code}{%
\codefont
\begin{shaded}
}{
\end{shaded}
}

%\newcommand{\refeq}[1]{(\ref{#1})}

\usepackage{algorithm}
\usepackage{algpseudocode}
\algrenewcommand{\algorithmicrequire}{\textbf{Input:~~}}
\algrenewcommand{\algorithmicensure}{\textbf{Output:}}
\algrenewcommand{\algorithmiccomment}[1]{\qquad\hfill~\hspace*{-5ex}\textit{// #1}}
\algrenewcommand{\alglinenumber}[1]{\helvetica{6}{1.3}{m}{n}#1:}

\newenvironment{algo}[1][8]{
\quad\begin{minipage}{.8\columnwidth}\helvetica{#1}{1.3}{m}{n}
\medskip\hrule\medskip
\begin{algorithmic}[1]
}{
\end{algorithmic}
\medskip\hrule\medskip
\end{minipage}
}

\usepackage{etoolbox}

%%%%%%%%%%%%%%%%%%%%%%%%%%%%%%%%%%%%%%%%%%%%%%%%%%%%%%%%%%%%%%%%%%%%%%%%%%%%%%%%

\usepackage{multirow}
\usepackage{colortbl}
%\setlength{\jot}{0pt}
%\setlength{\mathindent}{1ex}
\usepackage{empheq}

%%%%%%%%%%%%%%%%%%%%%%%%%%%%%%%%%%%%%%%%%%%%%%%%%%%%%%%%%%%%%%%%%%%%%%%%%%%%%%%

\newcommand{\mypause}{\pause}
%\newcommand{\dom}{{\text{dom}}}
\newcommand{\defi}[1]{\textbf{#1}}
\newcommand{\red}[1]{\emph{\color{red}#1}}
%\newcommand{\ul}{\underline}
\newcommand{\pos}{{\textsf{pos}}}
\newcommand{\eff}{{\textsf{eff}}}
\newcommand{\rot}{{\textsf{rot}}}
\newcommand{\veC}{{\textsf{vec}}}
\newcommand{\quat}{{\textsf{quat}}}
\newcommand{\col}{{\textsf{col}}}
\newcommand{\de}[2]{\frac{\partial #1}{\partial #2}}
\newcommand{\target}{{\text{target}}}
\newcommand{\near}{{\text{near}}}
\newcommand{\qfree}{Q_{\text{free}}}
\renewcommand{\vec}{\boldsymbol}
\newcommand{\lft}{\text{left}}
\newcommand{\rgh}{\text{right}}
\DeclareMathOperator{\real}{real}
\newcommand{\prev}{{\text{prev}}}
\newcommand{\TR}[2]{T_{{#1}\to{#2}}}
\newcommand{\RO}[2]{R_{{#1}\to{#2}}}
\newcommand{\liter}{\helvetica{8}{1.1}{m}{n}\parskip 1ex}
\newcommand{\Fc}{\color{green}F}
\newcommand{\muc}{\color{blue}\mu}
\newcommand{\Astar}{A$^*$}

%for AI course:
\newcommand{\defn}[1]{\textbf{#1}}

%for optimization course:
\newcommand{\adec}{\r_\a^-}
\newcommand{\ainc}{\r_\a^+}
\newcommand{\ldec}{\r_\l^-}
\newcommand{\linc}{\r_\l^+}
\newcommand{\minc}{\r_\m^+}
\newcommand{\mdec}{\r_\m^-}
\newcommand{\lsstop}{\r_{\text{ls}}}


\definecolor{boxcol}{rgb}{.85,.9,.92}
\newcommand{\eqbox}[1]{\centerline{\fboxrule0mm\fcolorbox{boxcol}{boxcol}{#1}}}
\newcommand{\movgb}[1]{\hfill\movie[externalviewer]{\small[movie]}{/home/mtoussai/movies/10-goalDirectedBehavior/#1}}
\newcommand{\demo}[1]{{{\color{blue}[\small #1]}}}

\graphicspath{{../pics-robotics/}{../pics-ML/}{../pics-all/}{../pics-all2/}{../pics-Optim/}}
\DeclareGraphicsExtensions{.pdf,.png,.jpg}

%\usepackage{pdfpages}
%\setbeamercolor{background canvas}{bg=}

\newcommand{\SUM}{\texttt{sum}}
\usepackage{float}

%% prevent pagebreaks before environment
\makeatletter
\newcommand{\NewParNoBreak}[1][\parskip]{\par\vspace*{-\parskip}\vspace*{#1}\nobreak\@afterheading}
\makeatother

%\newcommand{\idx}[2]{\label{IKgn}}

%%%%%%%%%%%%%%%%%%%%%%%%%%%%%%%%%%%%%%%%%%%%%%%%%%%%%%%%%%%%%%%%%%%%%%%%%%%%%%%%



%% \newwrite\tempfile
%% \immediate\openout\tempfile=z.keys.tex

%% \renewcommand{\key}[1]{
%% %%   \addtocounter{mypage}{1}
%% \makeatletter
%% \immediate\write\tempfile{\symbol{`\\}}
%% \makeatother
%%   \immediate\write\tempfile{hyperref[key:#1]{#1(\arabic{mypage})}}
%% %%  % \phantomsection\label{key:#1}
%% %%   %\index{#1@{\hyperref[key:#1]{#1 (\arabic{mysec}:\arabic{mypage})}}|phantom}
%% %%   \addtocounter{mypage}{-1}
%% }


  \DefineShortVerb{\@}

  \newcounter{solutions}
  \setcounter{solutions}{1}
  \newenvironment{solution}{
    \small
    \begin{shaded}
  }{
    \end{shaded}
  }
  
  \renewcommand{\hat}{\widehat}
  \newcommand{\bbg}{{\bar{\bar g}}}
  \graphicspath{{pics/}{../shared/pics/}}

  \renewcommand{\labelenumi}{{\alph{enumi})}}

  %%%%%%%%%%%%%%%%%%%%%%%%%%%%%%%%%%%%%%%%%%%%%%%%%%%%%%%%%%%%%%%%%%%%%%%%%%%%%%%%


  \mytitle{\course\\Exercise \exnum}
  \myauthor{Marc Toussaint\\ \small\addressUSTT}
  \date{\coursedate}
  
  \begin{document}
  \onecolumn
  \maketitle
}

\newcommand{\exsection}[1]{\section{#1}}

\newcommand{\exerfoot}{
  \end{document}
}

\newenvironment{items}[1][9]{
  \par\setlength{\unitlength}{1pt}\fontsize{#1}{#1}\linespread{1.2}\selectfont
  \begin{list}{--}{\leftmargin4ex \rightmargin0ex \labelsep1ex \labelwidth2ex
      \topsep0pt \parsep0ex \itemsep3pt}
}{
  \end{list}
}

  \exerciseshead
}

\providecommand{\script}{
  \newcommand{\scripthead}{
  \documentclass[9pt,twoside]{article}
  \stdpackages

  \usepackage{palatino}
  \usepackage[envcountsect]{beamerarticle}
  \usepackage{makeidx}
  \makeindex

  \definecolor{bluecol}{rgb}{0,0,.5}
  \definecolor{greencol}{rgb}{0,.4,0}
  \definecolor{shadecolor}{gray}{0.9}
  \usepackage[
    %    pdftex%,
    %%    letterpaper,
    %bookmarks,
    bookmarksnumbered,
    colorlinks,
    urlcolor=bluecol,
    citecolor=black,
    linkcolor=bluecol,
    %    pagecolor=bluecol,
    pdfborder={0 0 0},
    %pdfborderstyle={/S/U/W 1},
    %%    backref,     %link from bibliography back to sections
    %%    pagebackref, %link from bibliography back to pages
    %%    pdfstartview=FitH, %fitwidth instead of fit window
    pdfpagemode=UseOutlines, %bookmarks are displayed by acrobat
    pdftitle={\course},
    pdfauthor={Marc Toussaint},
    pdfkeywords={}
  ]{hyperref}
  \DeclareGraphicsExtensions{.pdf,.png,.jpg,.eps}

  \usepackage{multimedia}
  %\setbeamercolor{background canvas}{bg=}

  \renewcommand{\r}{\varrho}
  \renewcommand{\l}{\lambda}
  \renewcommand{\L}{\Lambda}
  \renewcommand{\b}{\beta}
  \renewcommand{\d}{\delta}
  \renewcommand{\k}{\kappa}
  \renewcommand{\t}{\theta}
  \renewcommand{\O}{\Omega}
  \renewcommand{\o}{\omega}
  \renewcommand{\SS}{{\cal S}}
  \renewcommand{\=}{\!=\!}
  %\renewcommand{\boldsymbol}{}
  %\renewcommand{\Chapter}{\chapter}
  %\renewcommand{\Subsection}{\subsection}

  \renewcommand{\baselinestretch}{1.0}
  \geometry{a5paper,headsep=6mm,hdivide={10mm,*,10mm},vdivide={13mm,*,7mm}}

  \fancyhead[OL,ER]{\course, \textit{Marc Toussaint}}
  \fancyhead[OR,EL]{\thepage}
  \fancyhead[C]{}
  \fancyfoot{}
  \pagestyle{fancy}

%  \setcounter{tocdepth}{3}
  \setcounter{tocdepth}{2}

   \columnsep 6ex
  %  \renewcommand{\familydefault}{\sfdefault}
  \newcommand{\headerfont}{\large}%helvetica{12}{1}{b}{n}}
  \newcommand{\slidefont} {}%\helvetica{9}{1.3}{m}{n}}
  \newcommand{\storyfont} {}
  %  \renewcommand{\small}   {\helvetica{8}{1.2}{m}{n}}
  \renewcommand{\tiny}    {\footnotesize}%helvetica{7}{1.1}{m}{n}}
  \newcommand{\ttiny} {\fontsize{5}{5}\selectfont}
  \newcommand{\codefont}{\fontsize{6}{6}\selectfont}%helvetica{8}{1.2}{m}{n}}

  %auto-ignore
  \renewcommand{\a}{\alpha}
  \renewcommand{\b}{\beta}
  \renewcommand{\d}{\delta}
    \newcommand{\D}{\Delta}
    \newcommand{\e}{\epsilon}
    \newcommand{\g}{\gamma}
    \newcommand{\G}{\Gamma}
  \renewcommand{\l}{\lambda}
  \renewcommand{\L}{\Lambda}
    \newcommand{\m}{\mu}
    \newcommand{\n}{\nu}
    \newcommand{\N}{\nabla}
  \renewcommand{\k}{\kappa}
  \renewcommand{\o}{\omega}
  \renewcommand{\O}{\Omega}
    \newcommand{\p}{\phi}
    \newcommand{\ph}{\varphi}
  \renewcommand{\P}{\Phi}
  \renewcommand{\r}{\varrho}
    \newcommand{\s}{\sigma}
  \renewcommand{\S}{\Sigma}
  \renewcommand{\t}{\theta}
    \newcommand{\T}{\Theta}
  %\renewcommand{\v}{\vartheta}
    \newcommand{\x}{\xi}
    \newcommand{\X}{\Xi}
    \newcommand{\Y}{\Upsilon}
    \newcommand{\z}{\zeta}

  \renewcommand{\AA}{{\cal A}}
    \newcommand{\BB}{{\cal B}}
    \newcommand{\CC}{{\cal C}}
    \newcommand{\cc}{{\cal c}}
    \newcommand{\DD}{{\cal D}}
    \newcommand{\EE}{{\cal E}}
    \newcommand{\FF}{{\cal F}}
    \newcommand{\GG}{{\cal G}}
    \newcommand{\HH}{{\cal H}}
    \newcommand{\II}{{\cal I}}
    \newcommand{\KK}{{\cal K}}
    \newcommand{\LL}{{\cal L}}
    \newcommand{\MM}{{\cal M}}
    \newcommand{\NN}{{\cal N}}
    \newcommand{\oNN}{\overline\NN}
    \newcommand{\OO}{{\cal O}}
    \newcommand{\PP}{{\cal P}}
    \newcommand{\QQ}{{\cal Q}}
    \newcommand{\RR}{{\cal R}}
  \renewcommand{\SS}{{\cal S}}
    \newcommand{\TT}{{\cal T}}
    \newcommand{\uu}{{\cal u}}
    \newcommand{\UU}{{\cal U}}
    \newcommand{\VV}{{\cal V}}
    \newcommand{\XX}{{\cal X}}
    \newcommand{\xx}{\mathcal{x}}
    \newcommand{\YY}{{\cal Y}}
    \newcommand{\SOSO}{{\cal SO}}
    \newcommand{\GLGL}{{\cal GL}}

    \newcommand{\Ee}{{\rm E}}

  \newcommand{\NNN}{{\mathbb{N}}}
  \newcommand{\III}{{\mathbb{I}}}
  \newcommand{\ZZZ}{{\mathbb{Z}}}
  %\newcommand{\RRR}{{\mathrm{I\!R}}}
  \newcommand{\RRR}{{\mathbb{R}}}
  \newcommand{\SSS}{{\mathbb{S}}}
  \newcommand{\CCC}{{\mathbb{C}}}
  \newcommand{\DDD}{{\mathbb{D}}}
  \newcommand{\one}{{{\bf 1}}}
  \newcommand{\eee}{\text{e}}

  \newcommand{\NNNN}{{\overline{\cal N}}}

  \renewcommand{\[}{\Big[}
  \renewcommand{\]}{\Big]}
  \renewcommand{\(}{\Big(}
  \renewcommand{\)}{\Big)}
  \renewcommand{\|}{\,|\,}
  \renewcommand{\;}{\,;\,}
  \renewcommand{\=}{\!=\!}
    \newcommand{\<}{\left\langle}
  \renewcommand{\>}{\right\rangle}

  \newcommand{\na}[1][]{{\nabla_{\!\!#1}}}
  \newcommand{\he}[1][]{{\nabla_{\!\!#1}^2}}
  \newcommand{\Prob}{{\rm Prob}}
  \newcommand{\Dir}{{\rm Dir}}
  \newcommand{\Beta}{{\rm Beta}}
  \newcommand{\Bern}{{\rm Bern}}
  \newcommand{\Bin}{{\rm Bin}}
  \newcommand{\Mult}{{\rm Mult}}
  \newcommand{\Aut}{{\rm Aut}}
  \newcommand{\cor}{{\rm cor}}
  \newcommand{\corr}{{\rm corr}}
  \newcommand{\sd}{{\rm sd}}
  \newcommand{\tr}{{\rm tr}}
  \newcommand{\Tr}{{\rm Tr}}
  \newcommand{\rank}{{\rm rank}}
  \newcommand{\diag}{{\rm diag}}
  \newcommand{\dom}{{\rm dom}}
  \newcommand{\id}{{\rm id}}
  \newcommand{\Id}{{\rm\bf I}}
  \newcommand{\Gl}{{\rm Gl}}
  \renewcommand{\th}{\ensuremath{{}^\text{th}} }
  \newcommand{\lag}{\mathcal{L}}
  \newcommand{\inn}{\rfloor}
  \newcommand{\lie}{\pounds}
  \newcommand{\longto}{\longrightarrow}
  \newcommand{\speer}{\parbox{0.4ex}{\raisebox{0.8ex}{$\nearrow$}}}
  \renewcommand{\dag}{ {}^\dagger }
  \newcommand{\blbox}{\rule{1ex}{1ex}}
  \newcommand{\Ji}{J^\sharp}
  \newcommand{\h}{{}^\star}
  \newcommand{\w}{\wedge}
  \newcommand{\too}{\longrightarrow}
  \newcommand{\oot}{\longleftarrow}
  \newcommand{\To}{\Rightarrow}
  \newcommand{\oT}{\Leftarrow}
  \newcommand{\oTo}{\Leftrightarrow}
  \renewcommand{\iff}{~\Longleftrightarrow~}
  \newcommand{\Too}{\;\Longrightarrow\;}
  \newcommand{\oto}{\leftrightarrow}
  \newcommand{\ot}{\leftarrow}
  \newcommand{\ootoo}{\longleftrightarrow}
  \newcommand{\ow}{\stackrel{\circ}\wedge}
  \newcommand{\defeq}{\stackrel{\hspace{0.2ex}{}_\Delta}=}
%  \newcommand{\defeq}{{\overstack\Delta =}}
  \newcommand{\feed}{\nonumber \\}
  \newcommand{\comma}{~,\quad}
  \newcommand{\period}{~.\quad}
  \newcommand{\del}{\partial}
%  \newcommand{\quabla}{\Delta}
  \newcommand{\point}{$\bullet~~$}
  \newcommand{\doubletilde}{ ~ \raisebox{0.3ex}{$\widetilde {}$} \raisebox{0.6ex}{$\widetilde {}$} \!\! }
  \newcommand{\topcirc}{\parbox{0ex}{~\raisebox{2.5ex}{${}^\circ$}}}
  \newcommand{\topdot} {\parbox{0ex}{~\raisebox{2.5ex}{$\cdot$}}}
  \newcommand{\topddot} {\parbox{0ex}{~\raisebox{1.3ex}{$\ddot{~}$}}}
  \newcommand{\sym}{\topcirc}
  \newcommand{\tsum}{\textstyle\sum}
  \newcommand{\st}{\quad\text{s.t.}\quad}

  \newcommand{\half}{\ensuremath{\frac{1}{2}}}
  \newcommand{\third}{\ensuremath{\frac{1}{3}}}
  \newcommand{\fourth}{\ensuremath{\frac{1}{4}}}

  \newcommand{\ubar}{\underline}
  %\renewcommand{\vec}{\underline}
  \renewcommand{\vec}{\boldsymbol}
  %\renewcommand{\_}{\underset}
  %\renewcommand{\^}{\overset}
  %\renewcommand{\*}{{\rm\raisebox{-.6ex}{\text{*}}{}}}
  \renewcommand{\*}{\text{\footnotesize\raisebox{-.4ex}{*}{}}}

  \newcommand{\gto}{{\raisebox{.5ex}{${}_\rightarrow$}}}
  \newcommand{\gfrom}{{\raisebox{.5ex}{${}_\leftarrow$}}}
  \newcommand{\gnto}{{\raisebox{.5ex}{${}_\nrightarrow$}}}
  \newcommand{\gnfrom}{{\raisebox{.5ex}{${}_\nleftarrow$}}}

  %\newcommand{\RND}{{\SS}}
  %\newcommand{\IF}{\text{if }}
  %\newcommand{\AND}{\textsc{and }}
  %\newcommand{\OR}{\textsc{or }}
  %\newcommand{\XOR}{\textsc{xor }}
  %\newcommand{\NOT}{\textsc{not }}

  %\newcommand{\argmax}[1]{{\rm arg}\!\max_{#1}}
  %\newcommand{\argmin}[1]{{\rm arg}\!\min_{#1}}
  \DeclareMathOperator*{\argmax}{argmax}
  \DeclareMathOperator*{\argmin}{argmin}
  \DeclareMathOperator{\sign}{sign}
  \DeclareMathOperator{\acos}{acos}
  \DeclareMathOperator{\unifies}{unifies}
  \DeclareMathOperator{\Span}{span}
  \newcommand{\ortho}{\perp}
  %\newcommand{\argmax}[1]{\underset{~#1}{\text{argmax}}\;}
  %\newcommand{\argmin}[1]{\underset{~#1}{\text{argmin}}\;}
  \newcommand{\ee}[1]{\ensuremath{\cdot10^{#1}}}
  \newcommand{\sub}[1]{\ensuremath{_{\text{#1}}}}
  \newcommand{\up}[1]{\ensuremath{^{\text{#1}}}}
  \newcommand{\kld}[3][{}]{D_{#1}\big(#2\,\big|\!\big|\,#3\big)}
  %\newcommand{\kld}[2]{D\big(#1:#2\big)}
  \newcommand{\sprod}[2]{\big<#1\,,\,#2\big>}
  \newcommand{\End}{\text{End}}
  \newcommand{\txt}[1]{\quad\text{#1}\quad}
  \newcommand{\Over}[2]{\genfrac{}{}{0pt}{0}{#1}{#2}}
  %\newcommand{\mat}[1]{{\bf #1}}
  \newcommand{\arr}[2]{\hspace*{-.5ex}\begin{array}{#1}#2\end{array}\hspace*{-.5ex}}
  \newcommand{\mat}[3][.9]{
    \renewcommand{\arraystretch}{#1}{\scriptscriptstyle{\left(
      \hspace*{-1ex}\begin{array}{#2}#3\end{array}\hspace*{-1ex}
    \right)}}\renewcommand{\arraystretch}{1.2}
  }
  \newcommand{\Mat}[3][.9]{
    \renewcommand{\arraystretch}{#1}{\scriptscriptstyle{\left[
      \hspace*{-1ex}\begin{array}{#2}#3\end{array}\hspace*{-1ex}
    \right]}}\renewcommand{\arraystretch}{1.2}
  }
  \newcommand{\case}[2][ll]{\left\{\arr{#1}{#2}\right.}
  \newcommand{\seq}[1]{\textsf{\<#1\>}}
  \newcommand{\seqq}[1]{\textsf{#1}}
  \newcommand{\floor}[1]{\lfloor#1\rfloor}
  \newcommand{\Exp}[2][]{\text{E}_{#1}\{#2\}}
  \newcommand{\Var}[2][]{\text{Var}_{#1}\{#2\}}
  \newcommand{\cov}[2][]{\text{cov}_{#1}\{#2\}}

%\newcommand{\Exp}[2]{\left\langle{#2}\right\rangle_{#1}}
  \newcommand{\ex}{\setminus}

  \providecommand{\href}[2]{{\color{blue}USE PDFLATEX!}}
  \providecommand{\url}[2]{\href{#1}{{\color{blue}#2}}}
%  \newcommand{\link}[1]{\href{{\protect #1}}{\texttt{\protect #1}}}
  \newcommand{\anchor}[2]{\begin{picture}(0,0)\put(#1){#2}\end{picture}}
  \newcommand{\pagebox}{\begin{picture}(0,0)\put(-3,-23){
    \textcolor[rgb]{.5,1,.5}{\framebox[\textwidth]{\rule[-\textheight]{0pt}{0pt}}}}
    \end{picture}}

  \newcommand{\hide}[1]{
    \begin{list}{}{\leftmargin0ex \rightmargin0ex \topsep0ex \parsep0ex}
       \helvetica{5}{1}{m}{n}
       \renewcommand{\section}{\par SECTION: }
       \renewcommand{\subsection}{\par SUBSECTION: }
       \item[$~~\blacktriangleright$]
       #1%$\blacktriangleleft~~$
       \message{^^JHIDE--Warning!^^J}
    \end{list}
  }
  %\newcommand{\hide}[1]{{\tt[hide:~}{\footnotesize\sf #1}{\tt]}\message{^^JHIDE--Warning!^^J}}
  \newcommand{\Hide}{\renewcommand{\hide}[1]{\message{^^JHIDE--Warning (hidden)!^^J}}}
  \newcommand{\HIDE}{\renewcommand{\hide}[1]{}}
  \newcommand{\fullhide}[1]{\message{^^JHIDE--Warning (hidden)!^^J}}
  \newcommand{\todo}[1]{{\tt[TODO: #1]}\message{^^JTODO--Warning: #1^^J}}
  \newcommand{\Todo}{\renewcommand{\todo}[1]{\message{^^JTODO--Warning (hidden)!^^J}}}
  %\renewcommand{\title}[1]{\renewcommand{\thetitle}{#1}}
  \newcommand{\myauthor}[1]{\author{#1}\newcommand{\theauthor}{#1}}%\@author}
  \newcommand{\mytitle}[1]{\title{#1}\newcommand{\thetitle}{#1}}%\@title}
  \newcommand{\header}{\begin{document}\mytitle\cleardefs}
  \newcommand{\contents}{{\tableofcontents}\renewcommand{\contents}{}}
  \newcommand{\footer}{\small\bibliography{marc,bibs}\end{document}}
  \newcommand{\widepaper}{\usepackage{geometry}\geometry{a4paper,hdivide={25mm,*,25mm},vdivide={25mm,*,25mm}}}
  \newcommand{\moviex}[2]{\movie[externalviewer]{#1}{#2}} %\pdflatex\usepackage{multimedia}
  \newcommand{\rbox}[1]{\fboxrule2mm\fcolorbox[rgb]{1,.85,.85}{1,.85,.85}{#1}}
  \newcommand{\mpage}[2]{{\begin{minipage}{#1\columnwidth}#2\end{minipage}}}
  \newcommand{\redbox}[2]{\fboxrule1mm\fcolorbox[rgb]{1,.7,.7}{1,.7,.7}{\begin{minipage}{#1\columnwidth}\center#2\end{minipage}}}
  \newcommand{\onecol}[2]{
    \begin{minipage}[c]{#1\columnwidth}#2\end{minipage}}
  \newcommand{\twocol}[5][0]{
    \begin{minipage}[c]{#2\columnwidth}#4\end{minipage}\hspace*{#1\columnwidth}%
    \begin{minipage}[c]{#3\columnwidth}#5\end{minipage}}
  \newcommand{\threecol}[7][0]{%
    \begin{minipage}[c]{#2\columnwidth}#5\end{minipage}\hspace*{#1\columnwidth}%
    \begin{minipage}[c]{#3\columnwidth}#6\end{minipage}\hspace*{#1\columnwidth}%
    \begin{minipage}[c]{#4\columnwidth}#7\end{minipage}}
  \newcommand{\threecoltext}[7][c]{
    \begin{minipage}[#1]{#2\textwidth}#5\end{minipage}%
    \begin{minipage}[#1]{#3\textwidth}#6\end{minipage}%
    \begin{minipage}[#1]{#4\textwidth}#7\end{minipage}}
  \newcommand{\threecoltop}[7][0]{%
   \begin{minipage}[t]{#2\columnwidth}#5\end{minipage}\hspace*{#1\columnwidth}%
   \begin{minipage}[t]{#3\columnwidth}#6\end{minipage}\hspace*{#1\columnwidth}%
   \begin{minipage}[t]{#4\columnwidth}#7\end{minipage}}
  \newcommand{\fourcol}[9][0]{%
   \begin{minipage}[c]{#2\columnwidth}#6\end{minipage}\hspace*{#1\columnwidth}%
   \begin{minipage}[c]{#3\columnwidth}#7\end{minipage}\hspace*{#1\columnwidth}%
   \begin{minipage}[c]{#4\columnwidth}#8\end{minipage}\hspace*{#1\columnwidth}%
   \begin{minipage}[c]{#5\columnwidth}#9\end{minipage}}
  \newcommand{\helvetica}[4]{\setlength{\unitlength}{1pt}\fontsize{#1}{#1}\linespread{#2}\usefont{OT1}{phv}{#3}{#4}}
  \newcommand{\helve}[1]{\helvetica{#1}{1.5}{m}{n}}
  \newcommand{\german}{\usepackage[german]{babel}\usepackage[utf8]{inputenc}}

\newcommand{\norm}[1]{|\!|#1|\!|}
\newcommand{\expr}[1]{[\hspace{-.2ex}[#1]\hspace{-.2ex}]}

\newcommand{\Jwi}{J^\sharp_W}
\newcommand{\THi}{T^\sharp_H}
\newcommand{\Jci}{J^\natural_C}
\newcommand{\hJi}{{\bar J}^\sharp}
\renewcommand{\|}{\,|\,}
\renewcommand{\=}{\!=\!}
\newcommand{\myminus}{{\hspace*{-.0pt}\text{\rm -}\hspace*{-.5pt}}}
\newcommand{\myplus}{{\hspace*{-.0pt}\text{\rm +}\hspace*{-.5pt}}}
\newcommand{\1}{{\myminus1}}
\newcommand{\2}{{\myminus2}}
\newcommand{\3}{{\myminus3}}
\newcommand{\mT}{{\text{\rm -}\hspace*{-1pt}\top}}
\newcommand{\po}{{\myplus1}}
\newcommand{\pt}{{\myplus2}}
%\renewcommand{\-}{\myminus}
%\newcommand{\+}{\myplus}
\renewcommand{\T}{{\!\top\!}}
\newcommand{\xT}{{\underline x}}
\newcommand{\uT}{{\underline u}}
\newcommand{\zT}{{\underline z}}
\newcommand{\Sum}{\textstyle\sum}
\newcommand{\Int}{\textstyle\int}
\newcommand{\Prod}{\textstyle\prod}


\newenvironment{centy}{
\vspace{15mm}
\large
\hspace*{5mm}
\begin{minipage}{8cm}\it\color{blue}
}{
\end{minipage}
}

\newcommand{\old}{{\text{old}}}
\newcommand{\new}{{\text{new}}}
\newcommand{\MAP}{{\text{MAP}}}
\newcommand{\ML}{{\text{ML}}}

\newcommand{\redArrow}{\quad\anchor{0,-1}{\includegraphics[scale=.5]{figs/redArrow}}}
\newcommand{\pub}[1]{{\color{green}\helvetica{8}{1.3}{m}{n}#1\\}}
\DeclareMathOperator{\opKL}{KL}
\newcommand{\KL}[2]{\opKL\big(#1\,\big|\!\big|\,#2\big)} %\left(#1 |\!| #2\right)}

\renewcommand{\show}[2][.8]{\centerline{\includegraphics[width=#1\columnwidth]{#2}}}
\newcommand{\showh}[2][.8]{\includegraphics[width=#1\columnwidth]{#2}}
\newcommand{\shows}[2][.8]{\centerline{\includegraphics[scale=#1]{#2}}}
\newcommand{\showhs}[2][.8]{\includegraphics[scale=#1]{#2}}
\newcommand{\mov}[2]{\movie[externalviewer]{{\color{blue}\small #1}}{movies/#2}}
\newcommand{\movex}[2]{\movie[externalviewer]{#1}{#2}} %\pdflatex\usepackage{multimedia}
%\newcommand{\movgb}[1]{\hfill\movie[externalviewer]{\small[movie]}{/home/mtoussai/movies/10-goalDirectedBehavior/#1}}
\newcommand{\movh}[3][loop]{
\movie[#1]{\showh[#2]{movies/#3.png}}{movies/#3.avi}%
\movie[externalviewer]{$\circ$}{movies/#3.avi}
}
\newcommand{\movc}[3][loop]{\centerline{\movh[#1]{#2}{#3}}}
\newcommand{\cen}[1]{\centerline{#1}}

\newcommand{\citing}[1]{
{\color{citcol}\tiny#1\par}
}

\newcommand{\cit}[3]{
\par\smallskip
{\color{greencol}\tiny #1: \emph{#2}. #3 \par}
}

\newcommand{\citurl}[4]{
\par\smallskip
{\color{greencol}\tiny #1: \protect{\href{#4}{\color{blue}{#2.}}} #3 \par}
}

\newcommand{\cito}[3]{
\par\smallskip
{\color{bluecol}\tiny #1: \emph{#2}. #3 \par}
}

\newcommand{\redoMacrosInProof}{
  \renewcommand{\d}{\delta}
%  \renewcommand{\|}{\,|\,}
  \renewcommand{\=}{\!=\!}
}

%% \makeatletter
%% \newenvironment{code}{%
%%   \begin{lrbox}{\@tempboxa}\begin{minipage}{1\columnwidth}\codefont
%% }{
%%   \end{minipage}\end{lrbox}%
%%   \colorbox[rgb]{.95,.95,.95}{\usebox{\@tempboxa}}
%% }\makeatother

\newenvironment{code}{%
\codefont
\begin{shaded}
}{
\end{shaded}
}

%\newcommand{\refeq}[1]{(\ref{#1})}

\usepackage{algorithm}
\usepackage{algpseudocode}
\algrenewcommand{\algorithmicrequire}{\textbf{Input:~~}}
\algrenewcommand{\algorithmicensure}{\textbf{Output:}}
\algrenewcommand{\algorithmiccomment}[1]{\qquad\hfill~\hspace*{-5ex}\textit{// #1}}
\algrenewcommand{\alglinenumber}[1]{\helvetica{6}{1.3}{m}{n}#1:}

\newenvironment{algo}[1][8]{
\quad\begin{minipage}{.8\columnwidth}\helvetica{#1}{1.3}{m}{n}
\medskip\hrule\medskip
\begin{algorithmic}[1]
}{
\end{algorithmic}
\medskip\hrule\medskip
\end{minipage}
}

\usepackage{etoolbox}

%%%%%%%%%%%%%%%%%%%%%%%%%%%%%%%%%%%%%%%%%%%%%%%%%%%%%%%%%%%%%%%%%%%%%%%%%%%%%%%%

\usepackage{multirow}
\usepackage{colortbl}
%\setlength{\jot}{0pt}
%\setlength{\mathindent}{1ex}
\usepackage{empheq}

%%%%%%%%%%%%%%%%%%%%%%%%%%%%%%%%%%%%%%%%%%%%%%%%%%%%%%%%%%%%%%%%%%%%%%%%%%%%%%%

\newcommand{\mypause}{\pause}
%\newcommand{\dom}{{\text{dom}}}
\newcommand{\defi}[1]{\textbf{#1}}
\newcommand{\red}[1]{\emph{\color{red}#1}}
%\newcommand{\ul}{\underline}
\newcommand{\pos}{{\textsf{pos}}}
\newcommand{\eff}{{\textsf{eff}}}
\newcommand{\rot}{{\textsf{rot}}}
\newcommand{\veC}{{\textsf{vec}}}
\newcommand{\quat}{{\textsf{quat}}}
\newcommand{\col}{{\textsf{col}}}
\newcommand{\de}[2]{\frac{\partial #1}{\partial #2}}
\newcommand{\target}{{\text{target}}}
\newcommand{\near}{{\text{near}}}
\newcommand{\qfree}{Q_{\text{free}}}
\renewcommand{\vec}{\boldsymbol}
\newcommand{\lft}{\text{left}}
\newcommand{\rgh}{\text{right}}
\DeclareMathOperator{\real}{real}
\newcommand{\prev}{{\text{prev}}}
\newcommand{\TR}[2]{T_{{#1}\to{#2}}}
\newcommand{\RO}[2]{R_{{#1}\to{#2}}}
\newcommand{\liter}{\helvetica{8}{1.1}{m}{n}\parskip 1ex}
\newcommand{\Fc}{\color{green}F}
\newcommand{\muc}{\color{blue}\mu}
\newcommand{\Astar}{A$^*$}

%for AI course:
\newcommand{\defn}[1]{\textbf{#1}}

%for optimization course:
\newcommand{\adec}{\r_\a^-}
\newcommand{\ainc}{\r_\a^+}
\newcommand{\ldec}{\r_\l^-}
\newcommand{\linc}{\r_\l^+}
\newcommand{\minc}{\r_\m^+}
\newcommand{\mdec}{\r_\m^-}
\newcommand{\lsstop}{\r_{\text{ls}}}


\definecolor{boxcol}{rgb}{.85,.9,.92}
\newcommand{\eqbox}[1]{\centerline{\fboxrule0mm\fcolorbox{boxcol}{boxcol}{#1}}}
\newcommand{\movgb}[1]{\hfill\movie[externalviewer]{\small[movie]}{/home/mtoussai/movies/10-goalDirectedBehavior/#1}}
\newcommand{\demo}[1]{{{\color{blue}[\small #1]}}}

\graphicspath{{../pics-robotics/}{../pics-ML/}{../pics-all/}{../pics-all2/}{../pics-Optim/}}
\DeclareGraphicsExtensions{.pdf,.png,.jpg}

%\usepackage{pdfpages}
%\setbeamercolor{background canvas}{bg=}

\newcommand{\SUM}{\texttt{sum}}
\usepackage{float}

%% prevent pagebreaks before environment
\makeatletter
\newcommand{\NewParNoBreak}[1][\parskip]{\par\vspace*{-\parskip}\vspace*{#1}\nobreak\@afterheading}
\makeatother

%\newcommand{\idx}[2]{\label{IKgn}}

%%%%%%%%%%%%%%%%%%%%%%%%%%%%%%%%%%%%%%%%%%%%%%%%%%%%%%%%%%%%%%%%%%%%%%%%%%%%%%%%



%% \newwrite\tempfile
%% \immediate\openout\tempfile=z.keys.tex

%% \renewcommand{\key}[1]{
%% %%   \addtocounter{mypage}{1}
%% \makeatletter
%% \immediate\write\tempfile{\symbol{`\\}}
%% \makeatother
%%   \immediate\write\tempfile{hyperref[key:#1]{#1(\arabic{mypage})}}
%% %%  % \phantomsection\label{key:#1}
%% %%   %\index{#1@{\hyperref[key:#1]{#1 (\arabic{mysec}:\arabic{mypage})}}|phantom}
%% %%   \addtocounter{mypage}{-1}
%% }


  \DefineShortVerb{\@}

  \newcounter{solutions}
  \setcounter{solutions}{1}
  \renewenvironment{solution}{
    \small
    \begin{shaded}
  }{
    \end{shaded}
  }

  \graphicspath{{pics/}{../shared/pics/}}

%%%%%%%%%%%%%%%%%%%%%%%%%%%%%%%%%%%%%%%%%%%%%%%%%%%%%%%%%%%%%%%%%%%%%%%%%%%%%%%%

  \mytitle{\course\\Lecture Script}
  \myauthor{Marc Toussaint}
  \date{\coursedate}

  \begin{document}

  %% \vspace*{2cm}

  \maketitle
  %\anchor{100,10}{\includegraphics[width=4cm]{optim}}

%  \vspace*{1cm}

  \emph{This is a direct concatenation and reformatting of all lecture
    slides and exercises from this course, including indexing to help
    prepare for exams.}

  {\tableofcontents}
}

%%%%%%%%%%%%%%%%%%%%%%%%%%%%%%%%%%%%%%%%%%%%%%%%%%%%%%%%%%%%%%%%%%%%%%%%%%%%%%%%

%% \renewcommand{\keywords}{}
%% \newcommand{\topic}{}
%% \renewcommand{\mypause}{}

  \newcounter{mypage}
  \setcounter{mypage}{0}
  \newcounter{mysec}
  \setcounter{mysec}{0}
  \newcommand{\incpage}{\addtocounter{mypage}{1}}
  \newcommand{\incsec} {\addtocounter{mysec}{1}}

\newcommand{\beginTocMinipage}{
  \addtocontents{toc}{\smallskip\noindent\hspace*{.036\columnwidth}}
  \addtocontents{toc}{\protect\begin{minipage}{.914\columnwidth}\small}
}
\newcommand{\closeTocMinipage}{
  \addtocontents{toc}{\protect\end{minipage}}
  \addtocontents{toc}{}
  \addtocontents{toc}{\medskip}
}

\renewcommand{\slides}[1][]{
  \clearpage
  \incsec
  \section{\topic}
  {\small #1}
  \beginTocMinipage
  \setcounter{mypage}{0}
  \smallskip\nopagebreak\hrule\medskip
}

\newcommand{\slidesfoot}{
  \closeTocMinipage
  \bigskip
}

\newcommand{\sublecture}[2]{
  \pagebreak[3]
  \incpage
  \closeTocMinipage
  \subsection{#1}
  {\storyfont #2}
  \beginTocMinipage
  {\hfill\tiny \textsf{\arabic{mysec}:\arabic{mypage}}}\nopagebreak%
  \smallskip\nopagebreak\hrule
}

\newcommand{\key}[1]{
  \pagebreak[2]
  \addtocounter{mypage}{1}
  \addtocontents{toc}{\hyperref[key:#1]{#1 (\arabic{mysec}:\arabic{mypage})}}
  \phantomsection\label{key:#1}
  \index{#1@{\hyperref[key:#1]{#1 (\arabic{mysec}:\arabic{mypage})}}|phantom}
  \addtocounter{mypage}{-1}
}

\newenvironment{slidecore}[1]{
  \incpage
  \subsubsection*{#1}%{\headerfont\noindent\textbf{#1}\\}%
  \vspace{-6ex}%
  \begin{list}{$\bullet$}{\leftmargin4ex \rightmargin0ex \labelsep1ex
    \labelwidth2ex \partopsep0ex \topsep0ex \parsep.5ex \parskip0ex \itemsep0pt}\item[]~\\\nopagebreak%
}{
  \end{list}\nopagebreak%
  {\hfill\tiny \textsf{\arabic{mysec}:\arabic{mypage}}}\nopagebreak%
  \smallskip\nopagebreak\hrule
}

\newcommand{\slide}[2]{
  \begin{slidecore}{#1}
    #2
  \end{slidecore}
}

\newcommand{\exsection}[1]{
  \subsubsection{#1}
}

\renewcommand{\exercises}{
  \subsection{Exercise \exnum}
}

\newcommand{\exerfoot}{
  \bigskip
}

\newcommand{\story}[1]{
  \subsection*{Motivation \& Outline}
  \addtocontents{toc}{\hyperref[mot\arabic{mysec}]{Motivation \& Outline}}
  \phantomsection\label{mot\arabic{mysec}}
  {\storyfont\sf #1}
  \medskip\nopagebreak\hrule
}

\newcounter{savedsection}
\newcommand{\subappendix}{\setcounter{savedsection}{\arabic{section}}\appendix}
\newcommand{\noappendix}{
  \setcounter{section}{\arabic{savedsection}}% restore section number
  \setcounter{subsection}{0}% reset section counter
%  \gdef\@chapapp{\sectionname}% reset section name
  \renewcommand{\thesection}{\arabic{section}}% make section numbers arabic
}

\newenvironment{items}[1][9]{
\par\setlength{\unitlength}{1pt}\fontsize{#1}{#1}\linespread{1.2}\selectfont
\begin{list}{--}{\leftmargin4ex \rightmargin0ex \labelsep1ex \labelwidth2ex
\topsep0pt \parsep0ex \itemsep3pt}
}{
\end{list}
}

  \scripthead
}

\providecommand{\course}{NO COURSE}
\providecommand{\coursepicture}{NO PICTURE}
\providecommand{\coursedate}{NO DATE}
\providecommand{\topic}{NO TOPIC}
\providecommand{\keywords}{NO KEYWORDS}
\providecommand{\exnum}{NO NUMBER}


\providecommand{\stdpackages}{
  \usepackage{amsmath}
  \usepackage{amssymb}
  \usepackage{amsfonts}
  \allowdisplaybreaks
  \usepackage{amsthm}
  \usepackage{eucal}
  \usepackage{graphicx}
  \usepackage{color}
  \usepackage{geometry}
  \usepackage{framed}
%  \usecolor{xcolor}
  \definecolor{shadecolor}{gray}{0.9}
  \setlength{\FrameSep}{3pt}
  \usepackage{fancyvrb}
  \fvset{numbers=left,xleftmargin=5ex}

  \usepackage{multicol} 
  \usepackage{fancyhdr}
}

\providecommand{\addressUSTT}{
  Machine~Learning~\&~Robotics~lab, U~Stuttgart\\\small
  Universit{\"a}tsstra{\ss}e 38, 70569~Stuttgart, Germany
}


\renewcommand{\course}{Robotics}
\renewcommand{\coursepicture}{roboticsLecture}
\renewcommand{\coursedate}{Winter 2014}
\renewcommand{\topic}{Dynamics}

\slides

%%%%%%%%%%%%%%%%%%%%%%%%%%%%%%%%%%%%%%%%%%%%%%%%%%%%%%%%%%%%%%%%%%%%%%%%%%%%%%%%

\key{Dynamic vs. non-dynamic robots}
\slide{}{

\hspace*{-12mm}\begin{tabular}{p{.45\columnwidth}@{\qquad}|@{\qquad}p{.45\columnwidth}}
\cen{\large\textbf{Kinematic}} & \cen{\large\textbf{Dynamic}} \\[2ex]
 instantly change joint velocities $\dot q$:
&
 instantly change joint torques $u$:\\
\cen{$\d q_t \overset{!}= J^\sharp~ (y^* - \phi(q_t))$}
&
\cen{$u \overset{!}=$ ?} \\[2ex]

accounts for kinematic coupling of joints but
\textbf{ignores inertia, forces, torques}
&
accounts for dynamic coupling of joints and full
Newtonian physics\\[4ex]

gears, \textbf{stiff}, all of industrial robots\newline
\mov{\show[.15]{kukaStandardRobot.jpg}}{10-RoboticsLecture/11-Honda-All-New-ASIMO-hopping.mp4}
%\show[.9]{11-Honda-All-New-ASIMO-hopping2}
&
future robots, \textbf{compliant}, few research robots\newline
\show[.25]{barrett}

\end{tabular}

}

%%%%%%%%%%%%%%%%%%%%%%%%%%%%%%%%%%%%%%%%%%%%%%%%%%%%%%%%%%%%%%%%%%%%%%%%%%%%%%%%

\slide{When velocities cannot be changed/set arbitrarily}{

\item Examples:
\begin{items}
\item An air plane flying: You cannot command it to hold still in the
   air, or to move straight up.

\item A car: you cannot command it to move side-wards.

\item Your arm: you cannot command it to throw a ball with arbitrary
   speed (force limits).

\item A \emph{torque controlled} robot: You cannot command it to
   instantly change velocity (infinite acceleration/torque).
\end{items}

~

\item What all examples have in comment:
\begin{items}
\item One can set \textbf{controls} $u_t$ (air plane's control stick,
car's steering wheel, your muscles activations, torque/voltage/current
send to a robot's motors)

\item But these controls only indirectly influence the \textbf{dynamics of state}
\color{red}\Large\vspace*{-1ex}  $$x_{t\po} = f(x_t,u_t)$$
\end{items}
}

%%%%%%%%%%%%%%%%%%%%%%%%%%%%%%%%%%%%%%%%%%%%%%%%%%%%%%%%%%%%%%%%%%%%%%%%%%%%%%%%
\key{Definition of a dynamics equation}
\slide{Dynamics}{

\item The {dynamics} of a system describes how the controls
$u_t$ influence the change-of-state of the system

\eqbox{ $x_{t\po} = f(x_t,u_t)$ }

~

\begin{items}
\item The notation $x_t$ refers to the \emph{dynamic state} of the
   system: e.g., joint positions \emph{and velocities} $x_t =
   (q_t,\dot q_t)$.

\item $f$ is an arbitrary function, often smooth
\end{items}


}

%%%%%%%%%%%%%%%%%%%%%%%%%%%%%%%%%%%%%%%%%%%%%%%%%%%%%%%%%%%%%%%%%%%%%%%%%%%%%%%%

\slide{Outline}{

\item We start by discussing a \textbf{1D point mass} for 3 reasons:
\begin{items}
\item The most basic force-controlled system with inertia
\item We can introduce and understand \textbf{PID control}
\item The behavior of a point mass under PID control is
a \emph{reference} that we can also follow with arbitrary dynamic
robots (if the dynamics are known)
\end{items}

~

\item We discuss computing the dynamics of general robotic systems
\begin{items}
\item Euler-Lagrange equations
\item Euler-Newton method
\end{items}

\item We derive the dynamic equivalent of inverse kinematics:
\begin{items}
\item operational space control
\end{items}

}

%%%%%%%%%%%%%%%%%%%%%%%%%%%%%%%%%%%%%%%%%%%%%%%%%%%%%%%%%%%%%%%%%%%%%%%%%%%%%%%%

\sublecture{PID and a 1D point mass}

%%%%%%%%%%%%%%%%%%%%%%%%%%%%%%%%%%%%%%%%%%%%%%%%%%%%%%%%%%%%%%%%%%%%%%%%%%%%%%%%

\key{1D point mass}
\slide{The dynamics of a 1D point mass}{

\item Start with simplest possible example: 1D point mass

(no gravity, no friction, just a single mass)

~

\cen{\includegraphics[width=5cm]{mass-1}}

\item The {state} $x(t) = (q(t),\dot q(t))$ is described by:
\begin{items}
\item position $q(t) \in \RRR$
\item velocity $\dot q(t) \in \RRR$
\end{items}

~

\item The {controls} $u(t)$ is the force we apply on the mass point

~

\item The {system dynamics} is:
\begin{align*}
\ddot q(t) = u(t)/m
\end{align*}

}

%%%%%%%%%%%%%%%%%%%%%%%%%%%%%%%%%%%%%%%%%%%%%%%%%%%%%%%%%%%%%%%%%%%%%%%%%%%%%%%%
\key{P gain}
\slide{1D point mass -- proportional feedback}{

\item Assume current position is $q$.

The goal is to move it to the position $q^*$.

~

What can we do?

~\mypause

\item \textbf{\color{red}Idea 1:}

\emph{``Always pull the mass towards the goal $q^*$:''}
\begin{align*}
u = K_p~ (q^* - q)
\end{align*}

\cen{\includegraphics[width=5cm]{mass-2}}

}

%%%%%%%%%%%%%%%%%%%%%%%%%%%%%%%%%%%%%%%%%%%%%%%%%%%%%%%%%%%%%%%%%%%%%%%%%%%%%%%%
\slide{1D point mass -- proportional feedback}{

\item What's the effect of this control law?
\begin{align*}
m~ \ddot q
 & = u = K_p~ (q^* - q)
\end{align*}

$q = q(t)$ is a function of time, this is a second order differential equation

\mypause

\item Solution: ~ {\color{blue}assume $q(t) = a + b e^{\o t}$}

(a ``non-imaginary'' alternative would be $q(t) = a + b~ \e^{-\l
t}~ \cos(\o t)$)
\mypause%
\begin{align*}
&m~ b~ \o^2~ e^{\o t}
 = K_p~ q^* - K_p~ a - K_p~ b~ e^{\o t} \\
&(m~ b~ \o^2 + K_p~ b)~ e^{\o t}
  = K_p~ (q^* - a) \\
&\To
  (m~ b~ \o^2 + K_p~ b) = 0 ~\wedge~ (q^*-a) = 0 \\
& \To ~ \o = i \sqrt{K_p/m} \\
&q(t)
  = q^* + b~ e^{i \sqrt{K_p/m}~ t}
\end{align*}

This is an oscillation around $q^*$ with amplitude $b=q(0)-q^*$ and
frequency $\sqrt{K_p/m}$!

}

%%%%%%%%%%%%%%%%%%%%%%%%%%%%%%%%%%%%%%%%%%%%%%%%%%%%%%%%%%%%%%%%%%%%%%%%%%%%%%%%
\slide{1D point mass -- proportional feedback}{

\begin{align*}
&m~ \ddot q
  = u = K_p~ (q^* - q) \\
&q(t)
  = q^* + b~ e^{i \sqrt{K_p/m}~ t}
\end{align*}

Oscillation around $q^*$ with amplitude $b=q(0)-q^*$ and
frequency $\sqrt{K_p/m}$

~

\cen{\includegraphics[angle=-90,width=8cm]{cosine}}

%set terminal postscript
%set output 'cosine.eps'
%set size 1.0,.5
%plot[0:15][-1.1:1.1] cos(x) lw 3

}


%%%%%%%%%%%%%%%%%%%%%%%%%%%%%%%%%%%%%%%%%%%%%%%%%%%%%%%%%%%%%%%%%%%%%%%%%%%%%%%%
\key{D gain}
\slide{1D point mass -- derivative feedback}{

\item \textbf{\color{red}Idea 2}

\emph{``Pull less, when we're heading the right direction already:''}

\emph{``Damp the system:''}
\begin{align*}
u = K_p (q^* - q) + K_d (\dot q^* - \dot q)
\end{align*}

$\dot q^*$ is a desired goal velocity

For simplicity we set $\dot q^* = 0$ in the following.

~

~

\cen{\includegraphics[width=5cm]{mass-3}}

}

%%%%%%%%%%%%%%%%%%%%%%%%%%%%%%%%%%%%%%%%%%%%%%%%%%%%%%%%%%%%%%%%%%%%%%%%%%%%%%%%
\slide{1D point mass -- derivative feedback}{

\item What's the effect of this control law?
\begin{align*}
m \ddot q
 & = u
   = K_p (q^* - q) + K_d (0 - \dot q)
\end{align*}

\item Solution: ~ {\color{blue}again assume $q(t) = a + b e^{\o t}$}
\begin{align*}
&m~ b~ \o^2~ e^{\o t}
 = K_p~ q^* - K_p~ a - K_p~ b~ e^{\o t} - K_d~ b~ \o e^{\o t}\\
&(m~ b~ \o^2 + K_d~ b~ \o + K_p~ b)~ e^{\o t}
 = K_p~ (q^* - a) \\
&\To (m~ \o^2 + K_d~ \o + K_p) = 0 ~\wedge~ (q^*-a) = 0 \\
&\To \o = \frac{-K_d \pm \sqrt{K_d^2 - 4 m K_p}}{2m} \\
&q(t)
 = q^* + b~ e^{\o~ t}
\end{align*}

The term $-\frac{K_d}{2m}$ in $\o$ is real ~ $\oto$ ~ exponential decay (damping)

}

%%%%%%%%%%%%%%%%%%%%%%%%%%%%%%%%%%%%%%%%%%%%%%%%%%%%%%%%%%%%%%%%%%%%%%%%%%%%%%%%
\slide{1D point mass -- derivative feedback}{

$$q(t)
 = q^* + b~ e^{\o~ t}
\comma \o = \frac{-K_d \pm \sqrt{K_d^2 - 4 m K_p}}{2m}$$

~

\item Effect of the second term $\sqrt{K_d^2 - 4 m K_p}/2m$ in $\o$:

~

\begin{tabular}{c@{\quad$\To$\quad}p{.7\columnwidth}}
$K_d^2 < 4 m K_p$
&
$\o$ has imaginary part\newline
oscillating with frequency $\sqrt{K_p/m - K_d^2/4m^2}$\newline
$q(t) = q^* + b e^{-K_d/2m~ t}~ e^{i \sqrt{K_p/m - K_d^2/4m^2}~ t}$
\\
$K_d^2 > 4 m K_p$
&
$\o$ real\newline
strongly damped
\\
$K_d^2 = 4 m K_p$
&
second term zero\newline
only exponential decay
\end{tabular}

}


%%%%%%%%%%%%%%%%%%%%%%%%%%%%%%%%%%%%%%%%%%%%%%%%%%%%%%%%%%%%%%%%%%%%%%%%%%%%%%%%
\key{oscillatory-damped, critically damped, over-damped}
\slide{1D point mass -- derivative feedback}{

\show[.9]{dampenings}
\hfill{\tiny illustration from O. Brock's lecture}

%% $\xi^2 = \frac{K_d^2}{4m K_p}$

%% $\xi = \frac{K_d}{4\sqrt{m K_p}}$

%% $\xi <1$ $\To$


}

%%%%%%%%%%%%%%%%%%%%%%%%%%%%%%%%%%%%%%%%%%%%%%%%%%%%%%%%%%%%%%%%%%%%%%%%%%%%%%%%
\key{Damping ration, wave length}
\slide{1D point mass -- derivative feedback}{

Alternative parameterization:

Instead of the \emph{gains} $K_p$ and $K_d$ it is sometimes more
intuitive to set the

~

\item wave length $\l = \frac{1}{\o_0} = \frac{1}{\sqrt{K_p/m}} \comma K_p = m/\l^2$

~

\item damping ratio $\xi
 = \frac{K_d}{\sqrt{4m K_p}}
 = \frac{\l K_d}{2m}
 \comma K_d = 2m\xi/\l$

~

$\xi > 1$: over-damped

$\xi = 1$: critically dampled

$\xi < 1$: oscillatory-damped

$$q(t) = q^* + b e^{-\xi~ t/\l}~ e^{i \sqrt{1-\xi^2}~ t/\l}$$

}

%%%%%%%%%%%%%%%%%%%%%%%%%%%%%%%%%%%%%%%%%%%%%%%%%%%%%%%%%%%%%%%%%%%%%%%%%%%%%%%%
\key{I gain}
\slide{1D point mass -- integral feedback}{

\item \textbf{\color{red}Idea 3}

\emph{``Pull if the position error accumulated large in the past:''}

\begin{align*}
u = K_p (q^* - q) + K_d (\dot q^* - \dot q) + K_i \int_{s=0}^t (q^*(s) - q(s))~ ds
\end{align*}

~\tiny

\item This is not a linear ODE w.r.t.\ $x=(q,\dot q)$.

However, when we extend the state to $x=(q,\dot q,e)$ we have the ODE
\begin{align*}
\dot q
 &= \dot q \\
\ddot q
 &= u/m = K_p/m (q^* - q) + K_d/m (\dot q^* - \dot q) + K_i/m~ e \\
\dot e
 &= q^* - q
\end{align*}

(no explicit discussion here)

}

%%%%%%%%%%%%%%%%%%%%%%%%%%%%%%%%%%%%%%%%%%%%%%%%%%%%%%%%%%%%%%%%%%%%%%%%%%%%%%%%
\key{PID control}
\slide{1D point mass -- PID control}{

\begin{align*}
u = K_p (q^* - q) + K_d (\dot q^* - \dot q) + K_i \int_{s=0}^t (q^* - q(s))~ ds
\end{align*}

~

\item\textbf{PID control}

-- Proportional Control (``Position Control'')

~~ $u \propto K_p (q^* - q)$

~

-- Derivative Control (``Damping'')

~~ $u \propto K_d (\dot q^* - \dot q)$  ~~ ($\dot x^*=0$ $\to$ damping)

~

-- Integral Control (``Steady State Error'')

~~ $u \propto K_i \int_{s=0}^t (q^*(s) - q(s))~ ds$

}


%%%%%%%%%%%%%%%%%%%%%%%%%%%%%%%%%%%%%%%%%%%%%%%%%%%%%%%%%%%%%%%%%%%%%%%%%%%%%%%%

\slide{Controlling a 1D point mass -- lessons learnt}{

\item Proportional and derivative feedback (PD control) are like
adding a spring and damper to the point mass

~

\item PD control is a \emph{linear control law}

 $$(q,\dot q) \mapsto u = K_p (q^* - q) + K_d (\dot q^* - \dot
 q)$$

(linear in the \emph{dynamic system state} $x=(q,\dot q)$)

~

\item With such linear control laws we can design 
approach trajectories (by tuning the gains)

-- but no optimality principle behind such motions

}

%%%%%%%%%%%%%%%%%%%%%%%%%%%%%%%%%%%%%%%%%%%%%%%%%%%%%%%%%%%%%%%%%%%%%%%%%%%%%%%%
\sublecture{Dynamics of mechanical systems}

%%%%%%%%%%%%%%%%%%%%%%%%%%%%%%%%%%%%%%%%%%%%%%%%%%%%%%%%%%%%%%%%%%%%%%%%%%%%%%%%
\slide{Two ways to derive dynamics equations for mechanical systems}{

\item The Euler-Lagrange equation, $L=L(t, q(t), \dot q(t))$,
$$ \frac{d}{dt} \frac{\del L}{\del \dot q} - \frac{\del L}{\del q} = u $$
{\small Used when you want to derive analytic equations of
motion (``on paper'')}

~

\item The Newton-Euler recursion ~ (and related algorithms)
$$f_i = m \dot v_i \comma u_i = I_i \dot w + w\times I w$$ {\small
Algorithms that ``propagate'' forces through a kinematic tree and
numerically compute the \emph{inverse} dynamics $u=\text{NE}(q,\dot
q,\ddot q)$ or \emph{forward} dynamics $\ddot q = f(q,\dot q,u)$.

} 

}

%%%%%%%%%%%%%%%%%%%%%%%%%%%%%%%%%%%%%%%%%%%%%%%%%%%%%%%%%%%%%%%%%%%%%%%%%%%%%%%%
\key{Euler-Lagrange equation}
\slide{The Euler-Lagrange equation}{

$$\frac{d}{dt} \frac{\del L}{\del \dot q} - \frac{\del L}{\del q} = u$$

\item $L(q,\dot q)$ is called \textbf{Lagrangian} and defined as
$$L = T - U$$
where $T$=kinetic energy and $U$=potential energy.

\item $q$ is called generalized coordinate -- any coordinates such that
 $(q,\dot q)$ describes the state of the system. Joint angles in our
 case.

\item $u$ are external forces

}

%%%%%%%%%%%%%%%%%%%%%%%%%%%%%%%%%%%%%%%%%%%%%%%%%%%%%%%%%%%%%%%%%%%%%%%%%%%%%%%%
\key{Pendulum example for Euler-Lagrange}
\slide{Example: A pendulum}{

\shows[1]{pendulum}

\item Generalized coordinates: angle $q=(\t)$

\item Kinematics:
\begin{items}
\item velocity of the mass: $v = (l\dot\t\cos\t, 0, l\dot\t\sin\t)$
\item angular velocity of the mass: $w = (0, -\dot\t, 0)$
\end{items}

\item Energies:
$$T=\half m v^2 + \half w^\T I w = \half (ml^2 + I_2)\dot\t^2\comma U
= - m g l\cos\t$$

\item Euler-Lagrange equation:
\begin{align*}
u
&=\frac{d}{dt} \frac{\del L}{\del \dot q} - \frac{\del L}{\del q} \\
&=\frac{d}{dt} (ml^2 + I_2)\dot\t + m g l\sin\t 
 = (ml^2 + I_2)\ddot\t + m g l\sin\t
\end{align*}

}


%%%%%%%%%%%%%%%%%%%%%%%%%%%%%%%%%%%%%%%%%%%%%%%%%%%%%%%%%%%%%%%%%%%%%%%%%%%%%%%%
\key{General structure of Euler-Lagrange}
\slide{The Euler-Lagrange equation}{

\small

\item How is this typically done?

\item \textbf{First}, describe the \emph{kinematics and Jacobians} for
every link $i$:
$$(q,\dot q)\mapsto \{ T_{W\to i}(q), v_i, w_i \}$$

{\tiny Recall $T_{W\to i}(q)
 = T_{W\to A}~ {\color{red}T_{A\to A'}(q)}~
   T_{A'\to B}~ {\color{red}T_{B\to B'}(q)}~
   \cdots$

Further, we know that a link's velocity $v_i=J_i \dot q$ can be
described via its position Jacobian.

Similarly we can describe the link's \emph{angular velocity} $w_i = J^w_i \dot q$ as linear in $\dot q$.

}

\item \textbf{Second}, formulate the kinetic energy
$$T = \sum_i \half m_i v_i^2 + \half w_i^\T I_i w_i
= \sum_i \half \dot q^\T M_i \dot q
\comma M_i = \mat{c}{J_i\\J^w_i}^\T \mat{cc}{m_i\Id_3 & 0 \\ 0 & I_i} \mat{c}{J_i\\J^w_i} $$
where $I_i = R_i \bar I_i R_i^\T$ and $\bar I_i$ the inertia tensor in
link coordinates

\item \textbf{Third}, formulate the potential energies (typically independent of $\dot q$)
$$U = g m_i \text{height}(i)$$

\item \textbf{Fourth}, compute the partial derivatives analytically to
get something like
\begin{align*}
\underbrace{u}_\text{control} = \frac{d}{dt} \frac{\del L}{\del \dot q} - \frac{\del L}{\del q} 
 = \underbrace{M}_\text{inertia} \ddot q + \underbrace{\dot M \dot q
 - \frac{\del T}{\del q}}_\text{Coriolis} + \underbrace{\frac{\del U}{\del q}}_\text{gravity}
\end{align*}
which relates accelerations $\ddot q$ to the forces

}

%%%%%%%%%%%%%%%%%%%%%%%%%%%%%%%%%%%%%%%%%%%%%%%%%%%%%%%%%%%%%%%%%%%%%%%%%%%%%%%%

%% \slide{Computing $M$ and $F$}{

%% \cen{$M(q)~ \ddot q + F(q,\dot q) = u$}

%% \item Recall:

%% We have an algorithm to compute all positions and
%% orientations given $q$:

%% $T_{W\to i}(q)
%%  = T_{W\to A}~ {\color{red}T_{A\to A'}(q)}~
%%    T_{A'\to B}~ {\color{red}T_{B\to B'}(q)}~
%%    T_{B'\to C}~ {\color{red}T_{C\to C'}(q)}~
%%    T_{C'\to i}$

%% ~\mypause

%% \item This can be extended easily to compute also:

%% -- linear velocities $v_i$

%% -- angular velocities $w_i$

%% -- linear accelerations $\dot v_i$

%% -- angular accelerations $\dot w_i$

%% for any link $i$ in the robot ~ (via forward propagation; see
%% geometry notes for details).

%% }

%%%%%%%%%%%%%%%%%%%%%%%%%%%%%%%%%%%%%%%%%%%%%%%%%%%%%%%%%%%%%%%%%%%%%%%%%%%%%%%%
\key{Newton-Euler recursion}
\slide{Newton-Euler recursion}{

\item An algorithm that computes the \emph{inverse dynamics}
$$u=\text{NE}(q,\dot q,\ddot q^*)$$ by recursively computing force
balance at each joint:
\begin{items}
\item \textbf{Newton's equation} ~ expresses the force acting at the center of mass for an accelerated body:
$$f_i = m \dot v_i$$

\item \textbf{Euler's equation} ~ expresses the torque (=control) acting on a
rigid body given an angular velocity and angular acceleration:
$$u_i = I_i \dot w + w\times I w$$
\end{items}

~

\item \textbf{Forward recursion:} ($\approx$ kinematics)

Compute (angular) velocities $(v_i, w_i)$ \emph{and} accelerations
$(\dot v_i,\dot w_i)$ for every link {\tiny(via forward propagation; see
geometry notes for details)}

~

\item \textbf{Backward recursion:}

For the leaf links, we now know the desired accelerations $q^*$ and can
compute the necessary joint torques. Recurse backward.

}

%%%%%%%%%%%%%%%%%%%%%%%%%%%%%%%%%%%%%%%%%%%%%%%%%%%%%%%%%%%%%%%%%%%%%%%%%%%%%%%%

\slide{Numeric algorithms for forward and inverse dynamics}{

\item \textbf{Newton-Euler recursion}:~ very fast ($O(n)$) method to compute
\emph{inverse} dynamics $$u=\text{NE}(q,\dot q,\ddot q^*)$$
{\tiny Note that we can use this algorithm to also compute
\begin{items}
\item gravity terms:~ $u = \text{NE}(q,0,0) = G(q)$ 
\item Coriolis terms:~ $u = \text{NE}(q,\dot q,0) = C(q,\dot q)~ \dot q + G(q)$
\item column of Intertia matrix:~ $u = \text{NE}(q,0,e_i) = M(q)~ e_i$
\end{items}
}

~

\item \textbf{Articulated-Body-Dynamics}:~ fast method ($O(n)$) to compute \emph{forward}
dynamics $\ddot q = f(q,\dot q,u)$

}

%%%%%%%%%%%%%%%%%%%%%%%%%%%%%%%%%%%%%%%%%%%%%%%%%%%%%%%%%%%%%%%%%%%%%%%%%%%%%%%%

\slide{Some last practical comments}{

\item \texttt{[demo]}

\item Use \emph{energy conservation} to measure dynamic of physical
simulation

\item Physical simulation engines (developed for games):
\begin{items}
\item ODE (Open Dynamics Engine)
\item Bullet (originally focussed on collision only)
\item Physx (Nvidia)
\end{items}

Differences of these engines to Lagrange, NE or ABD:
\begin{items}
\item Game engine can model much more: Contacts, tissues,
particles, fog, etc
\item (The way they model contacts looks ok but is somewhat fictional)
\item On kinematic trees, NE or ABD are much
more precise than game engines
\item Game engines do not provide \emph{inverse} dynamics, $u=\text{NE}(q,\dot
q,\ddot q)$
\end{items}

~

\item Proper modelling of contacts is really really hard

}

%%%%%%%%%%%%%%%%%%%%%%%%%%%%%%%%%%%%%%%%%%%%%%%%%%%%%%%%%%%%%%%%%%%%%%%%%%%%%%%%
\sublecture{Controlling a dynamic robot}

%%%%%%%%%%%%%%%%%%%%%%%%%%%%%%%%%%%%%%%%%%%%%%%%%%%%%%%%%%%%%%%%%%%%%%%%%%%%%%%%

\slide{}{

\item We previously learnt the effect of PID control on a 1D point
mass

~

\item Robots are not a 1D point mass
\begin{items}
\item Neither is each joint a 1D point mass
\item Applying separate PD control in each joint neglects force
coupling

(Poor solution: Apply very high gains separately in each joint $\oto$
make joints stiff, as with gears.)
\end{items}

~

\item However, knowing the robot dynamics we can transfer our
understanding of PID control of a point mass to general systems

}

%%%%%%%%%%%%%%%%%%%%%%%%%%%%%%%%%%%%%%%%%%%%%%%%%%%%%%%%%%%%%%%%%%%%%%%%%%%%%%%%
\key{General structure of robot dynamics}
\slide{General robot dynamics}{

\item Let $(q, \dot q)$ be the dynamic state and $u\in\RRR^n$ the
controls (typically joint torques in each
motor) of a robot

\item Robot dynamics can generally be written as:
$$
M(q)~ \ddot q + C(q,\dot q)~ \dot q + G(q) = u
$$
{\small\begin{tabular}{cp{.7\columnwidth}}
$M(q) \in \RRR^{n\times n}$
& is positive definite intertia matrix\newline (can be inverted $\to$
forward simulation of dynamics)
\\
$C(q,\dot q) \in \RRR^n$
& are the centripetal and coriolis forces
\\
$G(q) \in \RRR^n$
& are the gravitational forces
\\
$u$
& are the joint torques
\end{tabular}

}

(cf.\ to the Euler-Lagrange equation on slide 22)


\item We often write more compactly:

\eqbox{$M(q)~ \ddot q + F(q,\dot q) = u$}

}

%%%%%%%%%%%%%%%%%%%%%%%%%%%%%%%%%%%%%%%%%%%%%%%%%%%%%%%%%%%%%%%%%%%%%%%%%%%%%%%%
\key{Inverse and forward dynamics}
\slide{Controlling a general robot}{

\item From now on we just assume that we have algorithms to
efficiently compute $M(q)$ and $F(q,\dot q)$ for any $(q,\dot q)$

\item \textbf{Inverse dynamics:}~ If we know the desired $\ddot q^*$
for each joint, 
$$u = M(q)~ \ddot q^* + F(q,\dot q)$$
gives the necessary torques

\item \textbf{Forward dynamics:}~ If we know which torques $u$ we
apply, use
$$\ddot q^* = M(q)^\1 (u-F(q,\dot q))$$
to simulate the dynamics of the system (e.g., using Runge-Kutta)

}

%%%%%%%%%%%%%%%%%%%%%%%%%%%%%%%%%%%%%%%%%%%%%%%%%%%%%%%%%%%%%%%%%%%%%%%%%%%%%%%%
\key{Following a reference trajectory in joint space}
\slide{Following a reference trajectory in joint space}{

\item Where could we get the desired $\ddot q^*$ from?

Assume we have a nice smooth \textbf{reference trajectory}
$q_{0:T}^\text{ref}$ (generated with some motion profile or alike), we
can at each $t$ read off the desired acceleration as
$$\ddot q_t^\text{ref}
 := \frac{1}{\tau}[(q_{t\po}-q_t)/\tau - (q_t-q_{t\1})/\tau]
  = (q_{t\1} + q_{t\po} - 2q_t)/\tau^2$$

However, tiny errors in acceleration will accumulate greatly over
time! This is Instable!!

\mypause

\item Choose a desired acceleration $\ddot q_t^*$ that implies
a \emph{PD-like behavior around the reference trajectory}!

\eqbox{$\ddot q_t^* = \ddot q_t^\text{ref} + K_p(q_t^\text{ref} - q_t) + K_d(\dot q_t^\text{ref} - \dot q_t)$}

\small
This is a standard and very convenient heuristic to track a reference
trajectory when the robot dynamics are known: \emph{All joints will exactly
behave like a 1D point particle around the reference trajectory!}

}

%%%%%%%%%%%%%%%%%%%%%%%%%%%%%%%%%%%%%%%%%%%%%%%%%%%%%%%%%%%%%%%%%%%%%%%%%%%%%%%%

%% \slide{Dynamic reaching}{

%% \item How can we accelerate all joints \emph{to reach an
%% endeffector target $y^*$}?

%% -- 1D point mass $\to$ PD control can generate approach
%%    trajectories

%% -- How can we combine things?

%% \mypause

%% \emph{2 ways of thinking:}

%% \item \textbf{Joint space approach:}

%% 1) Compute target joint angles $q^*$ from $y^*$ using inverse
%% kinematics.

%% 2) Compute $u$ to ``mimick'' PD behavior in all
%% joints:

%% ~~ $\ddot q^* = K_p(q^* - q) + K_d(0 - \dot q)$

%% ~~ $u = M(q)~ \ddot q^* + F(q,\dot q)$

%% (better than separate PD controllers for each joint! cancels all
%% natural forces; behaves like an $n$-dimensional point mass)

%% \mypause

%% \item \textbf{Operational space approach:}

%% 1) Compute a desired
%% effector acceleration $\ddot y^*$ mimicking PD behavior.

%% 2) Translate this ``somehow'' to desired acceleration $\ddot q^*$:

%% $\ddot y^* = K_p(y^* - y) + K_d(0 - \dot y)$

%% $u = \text{function of }(\ddot y^*)$

%% }

%%%%%%%%%%%%%%%%%%%%%%%%%%%%%%%%%%%%%%%%%%%%%%%%%%%%%%%%%%%%%%%%%%%%%%%%%%%%%%%%
\key{Following a reference trajectory in task space}
\slide{Following a reference trajectory in task space}{

\item Recall the inverse kinematics problem:
\begin{items}
\item We know the desired step $\d y^*$ (or velocity $\dot y^*$) of the \emph{endeffector}.

\item Which step $\d q$ (or velocities $\dot q$) should we make in the
joints?
\end{items}

~

\item Equivalent dynamic problem:
\begin{items}
\item We know how the desired acceleration $\ddot y^*$ of
   the \emph{endeffector}.
\item What controls $u$ should we apply?
\end{items}

}

%%%%%%%%%%%%%%%%%%%%%%%%%%%%%%%%%%%%%%%%%%%%%%%%%%%%%%%%%%%%%%%%%%%%%%%%%%%%%%%%

%% \slide{Linear(ized) system and linear control}{

%% System state is given as $\bar q_t=\mat{c}{q_t\\ \dot q_t}$
%% \begin{align}
%% &P(q_{t\po} \| \dot q_t, q_t)
%%  = \NN(q_{t\po} \| q_t + u \dot q_{t\po}, W^\1) ~,\\
%% &P(\dot q_{t\po} \| \dot q_t, u_t)
%%  = \NN(\dot q_{t\po} \| \dot q_t + u M^\1(u_t+F), Q) ~,\\
%% %
%% &\mat{c}{q_{t\po}\\ \dot q_{t\po}}
%%  = \mat{c@{~}c}{1&u\\0&1}~ \mat{c}{q_t\\ \dot q_t} +
%%  \mat{c}{u^2\\u} M^\1 (u_t+F) + \xi \comma \<d\xi d\xi^\T\> = \mat{c@{~}c}{W^\1&0\\0&Q} \\
%% %
%% & A=\mat{c@{~}c}{1&u\\0&1}
%%   \comma B=\mat{c}{u^2 M^\1\\u M^\1}
%%   \comma a=\mat{c}{u^2 M^\1 F\\u M^\1 F}
%% \end{align}

%% }

%%%%%%%%%%%%%%%%%%%%%%%%%%%%%%%%%%%%%%%%%%%%%%%%%%%%%%%%%%%%%%%%%%%%%%%%%%%%%%%%
\key{Operational space control}
\slide{Operational space control}{

~

\item Inverse kinematics:
$$q^* = \argmin_q \norm{\phi(q) - y^*}^2_C + \norm{q-q_0}^2_W $$

~

\item Operational space control (one might call it ``Inverse task space dynamics''):
$$u^* = \argmin_u \norm{\ddot\phi(q) - \ddot y^*}^2_C + \norm{u}^2_H$$

\label{pageObjFunc}

}

%%%%%%%%%%%%%%%%%%%%%%%%%%%%%%%%%%%%%%%%%%%%%%%%%%%%%%%%%%%%%%%%%%%%%%%%%%%%%%%%



%% ~

%% \item what is actually $\dot y_{t\po}$ and $y_{t\po}$ depending on
%% $u$?

%% Euler integration:
%% \begin{align*}
%% \mat{c}{q_{t\po}\\ \dot q_{t\po}}
%%  = \mat{c}{q_t + \tau \dot q_t  \\
%%            \dot q_t + \tau  M^\1 (u-F)}
%%  = \mat{cc}{1&\tau\\0&1}~ \mat{c}{q_t\\ \dot q_t} +
%%     \mat{c}{0\\\tau} M^\1 (u-F)
%% \end{align*}

%% Leap frog integration:
%% \begin{align*}
%% \mat{c}{q_{t\po}\\ \dot q_{t\po}}
%%  = \mat{c}{q_t + \tau \dot q_t + \tau^2/2~ M^\1 (u-F)
%%            \dot q_t + \tau  M^\1 (u-F)}
%%   = \mat{cc}{1&\tau\\0&1}~ \mat{c}{q_t\\ \dot q_t} +
%%     \mat{c}{\tau^2/2\\\tau} M^\1 (u-F)
%% \end{align*}

%% }

%%%%%%%%%%%%%%%%%%%%%%%%%%%%%%%%%%%%%%%%%%%%%%%%%%%%%%%%%%%%%%%%%%%%%%%%%%%%%%%%
\slide{Operational space control}{

%% discrete time:
%% \begin{align*}
%% f(u)
%%  &= \norm{u}_{H} \\
%%  &+ \norm{\dot y^* - J \dot q_t - \tau J M^\1 (u-F)}_{D}
%%  &+ \norm{y^* - \phi(q_t) - \tau J \dot q_t - \tau^2/2~ J M^\1 (u-F)}_{C}
%% \end{align*}
%% (here we approximate $\dot J=0$!)

~

\item We can derive the optimum perfectly analogous to inverse
kinematics

{\small We identify the minimum of a locally squared potential, using
the local linearization (and approx. $\ddot J=0$)
$$
\ddot \phi(q)
  = \frac{d}{dt} \dot \phi(q)
  \approx \frac{d}{dt}(J\dot q+\dot J q)
  \approx J \ddot q + 2\dot J \dot q
  = J M^\1 (u-F) + 2\dot J \dot q
$$

}

We get
\begin{align*}
u^*
 &= T^\sharp (\ddot y^* - 2\dot J \dot q + T F)\\
% &= T^\sharp (\dot y^* - J \dot q_t + \tau T F)\\
 &\text{ with } T = J M^\1 \comma T^\sharp = (T^\T C T + H)^\1 T^\T C
\end{align*}
($C\to\infty ~\To~ T^\sharp = H^\1 T^\T (T H^\1 T^\T)^\1$)

}

%%%%%%%%%%%%%%%%%%%%%%%%%%%%%%%%%%%%%%%%%%%%%%%%%%%%%%%%%%%%%%%%%%%%%%%%%%%%%%%%

\slide{Controlling a robot -- operational space approach}{

\item Where could we get the desired $\ddot y^*$ from?

-- \textbf{Reference trajectory} $y_{0:T}^\text{ref}$ in operational space

-- \textbf{PD-like behavior} in each operational space:

\cen{$\ddot y_t^* = \ddot y_t^\text{ref} + K_p(y_t^\text{ref} - y_t) + K_d(\dot y_t^\text{ref} - \dot y_t)$}

~

\show[.8]{jointSpace-operationalSpace}
\hfill{\tiny illustration from O. Brock's lecture}

~

\item Operational space control: \emph{Let the system behave as if we could directly
``apply a 1D point mass behavior'' to the endeffector}


}


%%%%%%%%%%%%%%%%%%%%%%%%%%%%%%%%%%%%%%%%%%%%%%%%%%%%%%%%%%%%%%%%%%%%%%%%%%%%%%%%

%% \slide{Summary}{

%% \renewcommand{\arraystretch}{1.8}

%% \hspace*{-5mm}\begin{tabular}{|p{.27\columnwidth}|c|c|c|}
%% \hline
%% & State & Controls & Dynamics \\
%% %\hline
%% general notation
%% & $x$
%% & $u$
%% & $\dot x = f(x,u)$ \\
%% \hline
%% \hline
%% Kinematic control
%% & $x=q$
%% & $u=\dot q$
%% & $\dot q = u$ (can be \emph{set}) \\
%% \hline
%% 1D Point mass
%% & $x = \mat{c}{q \\ \dot q}$
%% & $u = \text{force}$
%% & $\dot x = \mat{c}{\dot q \\
%% u/m}$ \\
%% \hline
%% General dynamic\newline system
%% & $x = \mat{c}{q \\ \dot q}$
%% & $u = \text{torque}$
%% & $\dot x = \mat{c}{\dot q \\ M^\1(u-F)}$ \\
%% \hline
%% Car
%% & $q = \mat{c}{x \\ y \\ \t}$
%% & $u = \mat{c}{v \\ \varphi}$
%% & $\dot q = \mat{c}{v~ \cos \t \\ v~ \sin \t \\ (v/L)~ \tan \varphi }$\\
%% \hline
%% \end{tabular}

%% ~

%% Only kinematic control has $\dim(u)=\dim(x)$, all others $\dim(u)<\dim(x)$

%% }





%%%%%%%%%%%%%%%%%%%%%%%%%%%%%%%%%%%%%%%%%%%%%%%%%%%%%%%%%%%%%%%%%%%%%%%%%%%%%%%%


%% \slide{Analogies:~ motion rate and dynamic control}{


%% \small
%% \hspace*{-11mm}\begin{tabular}{|p{.19\columnwidth}|p{.26\columnwidth}|p{.5\columnwidth}|}
%% \hline
%% & motion rate control
%% & dynamic control \\
%% \hline
%% control interface
%% & can set \emph{velocities} $\dot q$ of servo motors
%% & can set \emph{torques} $u$ for each
%% joint  (or other \emph{control signals} $u$)\\
%% \hline
%% effect after a\newline time step $\tau$
%% & a step $\d q = \tau \dot q$ in joint angles
%% & a change $\d \dot q = \tau \ddot q$ in velocity \\
%% \hline
%% system\newline equation %$x_{t\po} = f(x_t,u)$
%% & $q_{t\po} = q_t + \tau \dot q$
%% & $M(q)~ \ddot q + F(q,\dot q) = u$ \newline
%%   $\mat{c}{q_{t\po}\\ \dot q_{t\po}}
%%   = \mat{c@{~}c}{1&\tau\\0&1}~ \mat{c}{q_t\\ \dot q_t} +
%%     \mat{c}{\tau^2/2\\\tau} M^\1 (u-F)$ \\
%% \hline
%% optimality: & & \\
%% 1) lazyness
%% & %$\norm{\dot q}_{W}^2$ \newline
%%   $\norm{q_{t\po} - q_t}_{W}^2$
%% & $\norm{u}_{H}^2$ \\
%% 2) task
%% & %$\norm{\dot y^* - J \dot q}_{C}^2$ \newline
%%   $\norm{\phi(q_{t\po}) - y^*}_{C}^2$
%% & $\norm{J M^\1 (u-F) + \dot J\dot q - \ddot y^*}_{C}^2$ \\
%% \hline
%% optimal solution
%% & %$\dot q = J^\sharp~ \dot y^*$ \newline
%%   $\d q = J^\sharp~ (y^* - \phi(q_t))$
%% & $u = T^\sharp (\ddot y^* - \dot J \dot q + T F)$\\
%% \hline
%% \end{tabular}

%% }

%%%%%%%%%%%%%%%%%%%%%%%%%%%%%%%%%%%%%%%%%%%%%%%%%%%%%%%%%%%%%%%%%%%%%%%%%%%%%%%%
\key{Multiple tasks}
\slide{Multiple tasks}{

\item Recall trick last time: we defined a ``big kinematic map'' $\Phi(q)$ such
that 
$$ q^* = \argmin_q \norm{q-q_0}_W^2 + \norm{\Phi(q)}^2 $$

~

\item Works analogously in the dynamic case:
$$ u^* = \argmin_u \norm{u}_H^2  + \norm{\Phi(q)}^2 $$

}

%%%%%%%%%%%%%%%%%%%%%%%%%%%%%%%%%%%%%%%%%%%%%%%%%%%%%%%%%%%%%%%%%%%%%%%%%%%%%%%%

\slide{What have we learned? What not?}{\label{lastpage}

\item More theory
\begin{items}
\item Contacts $\to$ Inequality constraints on the dynamics
\item Switching dynamics ~ (e.g.\ for walking)
\item Controllling contact forces
\end{items}

~

\item \textbf{Hardware limits}
\begin{items}
\item I think: the current success stories on highly dynamic robots
are all anchored in novel hardware approaches
\end{items}

}

%% \slide{}{

%% \item more on dynamics:

%% -- big field, e.g.\ see Stefan Schaal's lecture

%% -- biological motor control research (e.g.\ google Daniel Wolpert)%
%% \anchor{-0,-100}{\includegraphics[width=2cm]{bioMotorNoise}}

%% ~~ Wolpert DM & Flanagan JR (2010):\\
%% ~~ \emph{Q&A: Robotics as a tool to understand the brain}

%% -- recall big dog!! ~ (there is a lot more to say about dynamics)

%% ~

%% \item contacts:

%% -- difficult to model ~ (think of time scale of physics!)

%% -- really dynamic contacts $\to$ require compliant hardware

%% ~

%% \item more on kinematics:

%% -- non-redundant kinematics $\to$ direct analytic
%%    inverse kinematics

%% -- closed loop kinematics (as in walking)

%% -- Denavit-Hartenberg conventions:\\
%% ~~ special 4D-parameterization of
%%    transformations $T_{A\to B}$

%% ~

%% \item etc.\

%% }

%% %%%%%%%%%%%%%%%%%%%%%%%%%%%%%%%%%%%%%%%%%%%%%%%%%%%%%%%%%%%%%%%%%%%%%%%%%%%%%%%%

%% \slide{Vocabulary}{

%% \item \textbf{Linear controller (control law)}

%% $\pi: x \mapsto u$ maps the state ($(q,\dot q)$ or $q$)
%% linearly to controls ($u$ or $\dot q$)

%% \item \textbf{Closed loop, open loop}
%% \begin{items}
%% \item Closed loop control: $u = \pi(x,t)$ (depends on current state $x$, also called feedback control)
%% \item Open loop control: $u = \pi(t)$ ~~ (independent of current
%% state $x$, \emph{we didn't discuss this here})
%% \end{items}

%% \item \textbf{PD control, gains, stiffness}

%% $\ddot y^* = K_p(y^*-y) + K_d(\dot y^* - \dot y)$ \qquad (here in operational space)

%% $K_p$ = position gain, $K_d$ = velocity gain

%% stiffness = high $K_p$

%% \item \textbf{Damping}

%% $K_d$ is damping the system, $\xi<1$ = oscillatory-damped
%% (``overshooting'')

%% $\xi=1$ critically damped, $\xi>1$ over-damped

%% }


%%%%%%%%%%%%%%%%%%%%%%%%%%%%%%%%%%%%%%%%%%%%%%%%%%%%%%%%%%%%%%%%%%%%%%%%%%%%%%%%

\slidesfoot

