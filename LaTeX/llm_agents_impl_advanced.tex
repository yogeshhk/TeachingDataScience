%%%%%%%%%%%%%%%%%%%%%%%%%%%%%%%%%%%%%%%%%%%%%%%%%%%%%%%%%%%%%%%%%%%%%%%%%%%%%%%%%%
\begin{frame}[fragile]\frametitle{}
\begin{center}
{\Large Multi Agents}
\end{center}
\end{frame}


%%%%%%%%%%%%%%%%%%%%%%%%%%%%%%%%%%%%%%%%%%%%%%%%%%%%%%%%%%%
\begin{frame}[fragile]\frametitle{Avoid Overloaded Agents}
    \begin{itemize}
        \item Don't overload a single AI agent with many MCP servers
        \item Leads to performance bottlenecks and poor scalability
        \item Use multiple agents for effective orchestration
    \end{itemize}
	
{\tiny (Ref: LinkedIn post by Rakesh Gohel)}
	
\end{frame}

%%%%%%%%%%%%%%%%%%%%%%%%%%%%%%%%%%%%%%%%%%%%%%%%%%%%%%%%%%%
\begin{frame}[fragile]\frametitle{Agents}
	
	\begin{center}
	\includegraphics[width=0.8\linewidth,keepaspectratio]{agents1}
	\end{center}
	
{\tiny (Ref: Meet Agentic AI: The Vanguard of Modern Enterprise - Multimodal)}

\end{frame}


%%%%%%%%%%%%%%%%%%%%%%%%%%%%%%%%%%%%%%%%%%%%%%%%%%%%%%%%%%%
\begin{frame}[fragile]\frametitle{Why Multi-Agent Systems?}
    \begin{itemize}
        \item Specialised agents enable scalable automation
        \item Collaboration improves decision-making
        \item Parallel agents deliver faster results
        \item Real-time adaptation to dynamic inputs
    \end{itemize}
\end{frame}

%%%%%%%%%%%%%%%%%%%%%%%%%%%%%%%%%%%%%%%%%%%%%%%%%%%%%%%%%%%
\begin{frame}[fragile]\frametitle{Benefits of Multi-Agent Workflow}
    \begin{itemize}
        \item Single-agent systems are limited in scalability
        \item Multi-agent systems are modular and efficient
        \item Better for solving complex, dynamic problems
        \item Mimics real-world team collaboration
    \end{itemize}
\end{frame}

%%%%%%%%%%%%%%%%%%%%%%%%%%%%%%%%%%%%%%%%%%%%%%%%%%%%%%%%%%%
\begin{frame}[fragile]\frametitle{Popular Multi-Agent Patterns}
    \begin{itemize}
        \item Choose design patterns based on task needs
        \item Six effective patterns streamline development
        \item Supports better orchestration and coordination
    \end{itemize}
\end{frame}

%%%%%%%%%%%%%%%%%%%%%%%%%%%%%%%%%%%%%%%%%%%%%%%%%%%%%%%%%%%
\begin{frame}[fragile]\frametitle{Multi-Agent Patterns}
	
	\begin{center}
	\includegraphics[width=0.8\linewidth,keepaspectratio]{agents2}
	\end{center}
	
{\tiny (Ref: LinkedIn post by Rakesh Gohel)}

\end{frame}


%%%%%%%%%%%%%%%%%%%%%%%%%%%%%%%%%%%%%%%%%%%%%%%%%%%%%%%%%%%
\begin{frame}[fragile]\frametitle{1. Sequential Pattern}
    \begin{itemize}
        \item Agents execute one after another in a chain
        \item Each refines or transforms the output
        \item Use cases: ETL pipelines, Q\&A verification
    \end{itemize}
\end{frame}

%%%%%%%%%%%%%%%%%%%%%%%%%%%%%%%%%%%%%%%%%%%%%%%%%%%%%%%%%%%
\begin{frame}[fragile]\frametitle{2. Router Pattern}
    \begin{itemize}
        \item Central router delegates tasks to specialists
        \item Acts like an API gateway
        \item Use cases: Customer support, service orchestration
    \end{itemize}
\end{frame}

%%%%%%%%%%%%%%%%%%%%%%%%%%%%%%%%%%%%%%%%%%%%%%%%%%%%%%%%%%%
\begin{frame}[fragile]\frametitle{3. Parallel Pattern}
    \begin{itemize}
        \item Divides tasks into independent parallel subtasks
        \item Aggregates results after parallel processing
        \item Use cases: Info retrieval, financial risk analysis
    \end{itemize}
\end{frame}

%%%%%%%%%%%%%%%%%%%%%%%%%%%%%%%%%%%%%%%%%%%%%%%%%%%%%%%%%%%
\begin{frame}[fragile]\frametitle{4. Generator Pattern}
    \begin{itemize}
        \item Iterative loop: divisor → specialists → generator → feedback
        \item Enables draft-refine workflows
        \item Use cases: Code generation, design documentation
    \end{itemize}
\end{frame}

%%%%%%%%%%%%%%%%%%%%%%%%%%%%%%%%%%%%%%%%%%%%%%%%%%%%%%%%%%%
\begin{frame}[fragile]\frametitle{5. Network Pattern}
    \begin{itemize}
        \item Fully meshed agents with bidirectional links
        \item Overseen by a central meta-agent
        \item Use cases: Design, security, compliance reviews
    \end{itemize}
\end{frame}

%%%%%%%%%%%%%%%%%%%%%%%%%%%%%%%%%%%%%%%%%%%%%%%%%%%%%%%%%%%
\begin{frame}[fragile]\frametitle{6. Autonomous Agents Pattern}
    \begin{itemize}
        \item Agents operate in decentralised, looped interactions
        \item No central coordinator needed
        \item Use cases: Embodied agents, autonomous navigation
    \end{itemize}
\end{frame}

%%%%%%%%%%%%%%%%%%%%%%%%%%%%%%%%%%%%%%%%%%%%%%%%%%%%%%%%%%%%%%%%%%%%%%%%%%%%%%%%%%
\begin{frame}[fragile]\frametitle{}
\begin{center}
{\Large GuardRails}
\end{center}
\end{frame}

%%%%%%%%%%%%%%%%%%%%%%%%%%%%%%%%%%%%%%%%%%%%%%%%%%%%%%%%%%%
\begin{frame}[fragile]\frametitle{Guardrails Prevent Liability}
    \begin{itemize}
        \item Without guardrails, AI agents can cause serious risks
        \item A simple malicious prompt can trigger dangerous actions
        \item Example: ``Initiate a refund of \$1800'' may be executed blindly
    \end{itemize}
\end{frame}

%%%%%%%%%%%%%%%%%%%%%%%%%%%%%%%%%%%%%%%%%%%%%%%%%%%%%%%%%%%
\begin{frame}[fragile]\frametitle{GuardRails}
	
	\begin{center}
	\includegraphics[width=0.5\linewidth,keepaspectratio]{agents3}
	\end{center}
	
{\tiny (Ref: LinkedIn post by Rakesh Gohel)}

\end{frame}


%%%%%%%%%%%%%%%%%%%%%%%%%%%%%%%%%%%%%%%%%%%%%%%%%%%%%%%%%%%
\begin{frame}[fragile]\frametitle{How Guardrails Help}
    \begin{itemize}
        \item Guardrails detect, filter, and block unsafe inputs
        \item Protect agent workflows from abuse and mistakes
        \item Ensure system behaves safely and predictably
    \end{itemize}
\end{frame}

%%%%%%%%%%%%%%%%%%%%%%%%%%%%%%%%%%%%%%%%%%%%%%%%%%%%%%%%%%%
\begin{frame}[fragile]\frametitle{1. Pre-Check \& Validation}
    \begin{itemize}
        \item Filters inputs before reaching the AI model
        \item Includes content filtering and intent detection
        \item Flags malicious, nonsensical, or off-topic prompts
        \item First line of defense in any AI pipeline
    \end{itemize}
\end{frame}

%%%%%%%%%%%%%%%%%%%%%%%%%%%%%%%%%%%%%%%%%%%%%%%%%%%%%%%%%%%
\begin{frame}[fragile]\frametitle{2. Agentic Guardrails}
    \begin{itemize}
        \item Safety logic embedded inside the agent system
        \item Uses fine-tuned small LMs and strict rules
        \item Helps prevent unsafe actions from within
    \end{itemize}
\end{frame}

%%%%%%%%%%%%%%%%%%%%%%%%%%%%%%%%%%%%%%%%%%%%%%%%%%%%%%%%%%%
\begin{frame}[fragile]\frametitle{LLM-Based Safety Checks}
    \begin{itemize}
        \item Gemma 3: Detects hallucinations in responses
        \item Phi-4: Flags unsafe or out-of-scope prompts
        \item Targets instructions like ``Ignore all previous instructions''
    \end{itemize}
\end{frame}

%%%%%%%%%%%%%%%%%%%%%%%%%%%%%%%%%%%%%%%%%%%%%%%%%%%%%%%%%%%
\begin{frame}[fragile]\frametitle{Moderation APIs}
    \begin{itemize}
        \item Use APIs from OpenAI, AWS, Azure, etc.
        \item Catch toxicity, PII, and policy violations
        \item Adds an additional moderation layer to the pipeline
    \end{itemize}
\end{frame}

%%%%%%%%%%%%%%%%%%%%%%%%%%%%%%%%%%%%%%%%%%%%%%%%%%%%%%%%%%%
\begin{frame}[fragile]\frametitle{Rule-Based Protections}
    \begin{itemize}
        \item Blacklists block known prompt injection phrases
        \item Regex filters catch dangerous patterns
        \item Input length limits prevent oversized payloads
    \end{itemize}
\end{frame}

%%%%%%%%%%%%%%%%%%%%%%%%%%%%%%%%%%%%%%%%%%%%%%%%%%%%%%%%%%%
\begin{frame}[fragile]\frametitle{3. Deepcheck Safety Validation}
    \begin{itemize}
        \item Central logic gate: \texttt{is\_safe}
        \item Routes safe prompts to AI agents
        \item Unsafe prompts are diverted to fallback agents
    \end{itemize}
\end{frame}

%%%%%%%%%%%%%%%%%%%%%%%%%%%%%%%%%%%%%%%%%%%%%%%%%%%%%%%%%%%
\begin{frame}[fragile]\frametitle{4. AI Agent Frameworks \& Handoffs}
    \begin{itemize}
        \item Once validated, input goes to correct agent
        \item Example: Refund Agent handles refund logic
        \item Ensures only safe instructions reach execution layer
    \end{itemize}
\end{frame}

%%%%%%%%%%%%%%%%%%%%%%%%%%%%%%%%%%%%%%%%%%%%%%%%%%%%%%%%%%%
\begin{frame}[fragile]\frametitle{5. Refund Agent Execution}
    \begin{itemize}
        \item Final agent in the chain performs the task
        \item Secure function call handles the refund logic
        \item Operates only after multilayer validation
    \end{itemize}
\end{frame}

%%%%%%%%%%%%%%%%%%%%%%%%%%%%%%%%%%%%%%%%%%%%%%%%%%%%%%%%%%%
\begin{frame}[fragile]\frametitle{6. Post-Check \& Output Validation}
    \begin{itemize}
        \item Output reviewed before being sent to user
        \item Checks formatting, style, and safety again
        \item Prevents accidental disclosure or unsafe responses
    \end{itemize}
\end{frame}

%%%%%%%%%%%%%%%%%%%%%%%%%%%%%%%%%%%%%%%%%%%%%%%%%%%%%%%%%%%
\begin{frame}[fragile]\frametitle{Observability Layer}
    \begin{itemize}
        \item Logs every step: input → logic → output
        \item Enables auditing, debugging, and improvement
        \item Critical for maintaining trust in AI systems
    \end{itemize}
\end{frame}

%%%%%%%%%%%%%%%%%%%%%%%%%%%%%%%%%%%%%%%%%%%%%%%%%%%%%%%%%%%
\begin{frame}[fragile]\frametitle{Key Takeaways}
    \begin{itemize}
        \item AI agents need more than good models
        \item Guardrails ensure safety, traceability, and fallbacks
        \item Systems thinking is essential for reliable automation
    \end{itemize}
\end{frame}

%%%%%%%%%%%%%%%%%%%%%%%%%%%%%%%%%%%%%%%%%%%%%%%%%%%%%%%%%%%%%%%%%%%%%%%%%%%%%%%%%%
\begin{frame}[fragile]\frametitle{}
\begin{center}
{\Large Agentic RAG}
\end{center}
\end{frame}

%%%%%%%%%%%%%%%%%%%%%%%%%%%%%%%%%%%%%%%%%%%%%%%%%%%%%%%%%%%
\begin{frame}[fragile]\frametitle{Agentic RAG: RAG is Here to Stay}
    \begin{itemize}
        \item Agentic RAG proves the lasting value of RAG systems
        \item Used by Glean AI, Perplexity, Harvey, and others
        \item Ideal for complex enterprise workflows
    \end{itemize}
\end{frame}

%%%%%%%%%%%%%%%%%%%%%%%%%%%%%%%%%%%%%%%%%%%%%%%%%%%%%%%%%%%
\begin{frame}[fragile]\frametitle{Comparison}
	
	\begin{center}
	\includegraphics[width=0.5\linewidth,keepaspectratio]{agents4}
	\end{center}
	
{\tiny (Ref: LinkedIn post by Rakesh Gohel)}

\end{frame}


%%%%%%%%%%%%%%%%%%%%%%%%%%%%%%%%%%%%%%%%%%%%%%%%%%%%%%%%%%%
\begin{frame}[fragile]\frametitle{What is RAG (Retrieval Augmented Generation)?}
    \begin{itemize}
        \item Combines external data retrieval with LLM generation
        \item Ensures grounded and up-to-date responses
        \item Enhances reliability and relevance of output
    \end{itemize}
\end{frame}

%%%%%%%%%%%%%%%%%%%%%%%%%%%%%%%%%%%%%%%%%%%%%%%%%%%%%%%%%%%
\begin{frame}[fragile]\frametitle{RAG Workflow Overview}
    \begin{itemize}
        \item \textbf{Retrieval:} Query is embedded and relevant data is fetched from vector DB
        \item \textbf{Augmentation:} Retrieved data merged with query + system prompt
        \item \textbf{Generation:} LLM generates final response using augmented prompt
    \end{itemize}
\end{frame}

%%%%%%%%%%%%%%%%%%%%%%%%%%%%%%%%%%%%%%%%%%%%%%%%%%%%%%%%%%%
\begin{frame}[fragile]\frametitle{AI Agents in the Loop}
    \begin{itemize}
        \item Handle incoming queries and analyze intent
        \item Use memory and planning (ReACT, Reflexion)
        \item Fetch real-time data using tools and APIs
        \item Generate answers using reasoning and context
    \end{itemize}
\end{frame}

%%%%%%%%%%%%%%%%%%%%%%%%%%%%%%%%%%%%%%%%%%%%%%%%%%%%%%%%%%%
\begin{frame}[fragile]\frametitle{How Agentic RAG Combines RAG + Agents}
    \begin{itemize}
        \item Agents manage RAG's embedding and retrieval steps
        \item Dynamically choose data sources based on query
        \item Augment RAG prompts with planning and external tool data
        \item Deliver more precise and contextual outputs
    \end{itemize}
\end{frame}

%%%%%%%%%%%%%%%%%%%%%%%%%%%%%%%%%%%%%%%%%%%%%%%%%%%%%%%%%%%
\begin{frame}[fragile]\frametitle{Operational Workflow of Agentic RAG}
    \begin{itemize}
        \item \textbf{1. Query Routing:} Directs query to right agent
        \item \textbf{2. Context Retention:} Maintains short and long-term memory
        \item \textbf{3. Task Planning:} Chooses tools and retrieval plan
        \item \textbf{4. Data Fetching:} Retrieves from KBs using tools (e.g., vector search)
        \item \textbf{5. Prompt Optimisation:} Merges retrieved info + prompt + reasoning
        \item \textbf{6. Response Generation:} Final LLM output is generated and returned
    \end{itemize}
\end{frame}

%%%%%%%%%%%%%%%%%%%%%%%%%%%%%%%%%%%%%%%%%%%%%%%%%%%%%%%%%%%
\begin{frame}[fragile]\frametitle{Why Agentic RAG Matters}
    \begin{itemize}
        \item Enables smarter, more adaptive responses
        \item Combines memory, planning, retrieval, and reasoning
        \item Revolutionizing AI in enterprise applications
    \end{itemize}
\end{frame}


%%%%%%%%%%%%%%%%%%%%%%%%%%%%%%%%%%%%%%%%%%%%%%%%%%%%%%%%%%%%%%%%%%%%%%%%%%%%%%%%%%
\begin{frame}[fragile]\frametitle{}
\begin{center}
{\Large Claude Research Multi Agents}
\end{center}
\end{frame}

%%%%%%%%%%%%%%%%%%%%%%%%%%%%%%%%%%%%%%%%%%%%%%%%%%%%%%%%%%%
\begin{frame}[fragile]\frametitle{Claude Research: Multi-Agent Architecture}
    \begin{itemize}
        \item Anthropic shared insights into Claude's multi-agent architecture
        \item Real-world example of production-grade agent systems
        \item Highlights challenges, benefits, and practical design
    \end{itemize}
\end{frame}

%%%%%%%%%%%%%%%%%%%%%%%%%%%%%%%%%%%%%%%%%%%%%%%%%%%%%%%%%%%
\begin{frame}[fragile]\frametitle{Architecture}
	
	\begin{center}
	\includegraphics[width=0.8\linewidth,keepaspectratio]{agents5}
	\end{center}
	
{\tiny (Ref: LinkedIn post by Jerry Liu)}

\end{frame}

%%%%%%%%%%%%%%%%%%%%%%%%%%%%%%%%%%%%%%%%%%%%%%%%%%%%%%%%%%%
\begin{frame}[fragile]\frametitle{Process diagram}
	
	\begin{center}
	\includegraphics[width=0.6\linewidth,keepaspectratio]{agents6}
	\end{center}
	
{\tiny (Ref: How we built our multi-agent research system - Anthropic)}

\end{frame}

%%%%%%%%%%%%%%%%%%%%%%%%%%%%%%%%%%%%%%%%%%%%%%%%%%%%%%%%%%%
\begin{frame}[fragile]\frametitle{Multi-Agent Research Workflow}
    \begin{itemize}
        \item User query spawns a LeadResearcher agent.
        \item LeadResearcher plans and saves context to Memory.
        \item Specialized subagents are created for subtasks.
        \item Subagents perform web search, analyze results, return findings.
        \item LeadResearcher synthesizes and iterates if needed.
        \item CitationAgent adds source citations to claims.
        \item Final report with citations is returned to user.
    \end{itemize}
\end{frame}

%%%%%%%%%%%%%%%%%%%%%%%%%%%%%%%%%%%%%%%%%%%%%%%%%%%%%%%%%%%
\begin{frame}[fragile]\frametitle{Challenges in Multi-Agent Coordination}
    \begin{itemize}
        \item Early agents over-delegated and created task redundancy.
        \item Coordination complexity grows with agent count.
        \item Prompt engineering was key to guiding agent behavior.
        \item Simulations helped reveal failure modes.
        \item Agents often misused tools or duplicated tasks.
    \end{itemize}
\end{frame}

%%%%%%%%%%%%%%%%%%%%%%%%%%%%%%%%%%%%%%%%%%%%%%%%%%%%%%%%%%%
\begin{frame}[fragile]\frametitle{Effective Prompt Engineering}
    \begin{itemize}
        \item Build mental models to improve prompt quality.
        \item Use simulations to observe and refine behavior.
        \item Prompts guide delegation, scope, and tool use.
        \item Poor task descriptions cause duplication and gaps.
        \item Prompts should scale agent effort to task complexity.
    \end{itemize}
\end{frame}

%%%%%%%%%%%%%%%%%%%%%%%%%%%%%%%%%%%%%%%%%%%%%%%%%%%%%%%%%%%
\begin{frame}[fragile]\frametitle{Optimizing Tool Usage}
    \begin{itemize}
        \item Tool choice is critical—must match user intent.
        \item Agents are taught to assess tool relevance first.
        \item Specialized tools are preferred over generic ones.
        \item Bad tool descriptions can derail agent behavior.
        \item Self-improving agents rewrite flawed tool descriptions.
    \end{itemize}
\end{frame}

%%%%%%%%%%%%%%%%%%%%%%%%%%%%%%%%%%%%%%%%%%%%%%%%%%%%%%%%%%%
\begin{frame}[fragile]\frametitle{Search Strategy and Thinking}
    \begin{itemize}
        \item Begin with broad queries, then narrow focus.
        \item Extended “thinking” improves planning and reasoning.
        \item Subagents evaluate results and refine iteratively.
        \item Heuristics guide when to explore vs. go deep.
        \item Avoid verbose or overly specific search queries initially.
    \end{itemize}
\end{frame}

%%%%%%%%%%%%%%%%%%%%%%%%%%%%%%%%%%%%%%%%%%%%%%%%%%%%%%%%%%%
\begin{frame}[fragile]\frametitle{Parallelism and Performance Gains}
    \begin{itemize}
        \item Lead agent spawns subagents in parallel.
        \item Subagents use multiple tools concurrently.
        \item Parallel execution cuts research time drastically.
        \item Enables broad exploration within short timeframes.
        \item Improves system responsiveness for complex queries.
    \end{itemize}
\end{frame}

%%%%%%%%%%%%%%%%%%%%%%%%%%%%%%%%%%%%%%%%%%%%%%%%%%%%%%%%%%%
\begin{frame}[fragile]\frametitle{Reliability in Production Systems}
    \begin{itemize}
        \item Agents must persist state across long tasks.
        \item System supports graceful error recovery and resumption.
        \item Observability helps trace agent failures without logging content.
        \item Debugging focuses on behavior patterns, not just outputs.
        \item Rainbow deployments prevent disruption during updates.
    \end{itemize}
\end{frame}

%%%%%%%%%%%%%%%%%%%%%%%%%%%%%%%%%%%%%%%%%%%%%%%%%%%%%%%%%%%
\begin{frame}[fragile]\frametitle{Sync vs. Async Agent Execution}
    \begin{itemize}
        \item Current systems use synchronous agent execution.
        \item Sync simplifies coordination but slows down progress.
        \item Async allows subagents to act independently.
        \item Adds complexity in managing state and errors.
        \item Anticipated performance gains justify the shift.
    \end{itemize}
\end{frame}

%%%%%%%%%%%%%%%%%%%%%%%%%%%%%%%%%%%%%%%%%%%%%%%%%%%%%%%%%%%
\begin{frame}[fragile]\frametitle{From Prototype to Production}
    \begin{itemize}
        \item Minor bugs can cause cascading behavioral failures.
        \item Agent systems need more engineering than expected.
        \item Testing, iteration, and collaboration are essential.
        \item Guardrails prevent runaway agent behavior.
        \item Final systems must handle edge cases gracefully.
    \end{itemize}
\end{frame}

%%%%%%%%%%%%%%%%%%%%%%%%%%%%%%%%%%%%%%%%%%%%%%%%%%%%%%%%%%%
\begin{frame}[fragile]\frametitle{Impact and User Value}
    \begin{itemize}
        \item Users save days by uncovering hidden connections.
        \item Agents assist in business, healthcare, and debugging.
        \item Multi-agent systems solve complex research tasks.
        \item Careful design makes these systems reliable at scale.
        \item They are transforming how people tackle hard problems.
    \end{itemize}
\end{frame}


%%%%%%%%%%%%%%%%%%%%%%%%%%%%%%%%%%%%%%%%%%%%%%%%%%%%%%%%%%%
\begin{frame}[fragile]\frametitle{Not All Use Cases Need Multi-Agents}
    \begin{itemize}
        \item Some domains require shared context among agents
        \item High interdependencies reduce multi-agent effectiveness
        \item Not every task benefits from parallel agent workflows
    \end{itemize}
\end{frame}

%%%%%%%%%%%%%%%%%%%%%%%%%%%%%%%%%%%%%%%%%%%%%%%%%%%%%%%%%%%
\begin{frame}[fragile]\frametitle{Single vs Multi-Agent Debate}
    \begin{itemize}
        \item Similar point made by Cognition's ``Don't Build Multi-Agents''
        \item Both agree: multi-agents fit a specific class of problems
        \item Focus should be on identifying those right-fit use cases
    \end{itemize}
\end{frame}

%%%%%%%%%%%%%%%%%%%%%%%%%%%%%%%%%%%%%%%%%%%%%%%%%%%%%%%%%%%
\begin{frame}[fragile]\frametitle{Sub-Agents as Tools, Not Peers}
    \begin{itemize}
        \item Claude's system treats sub-agents like tools
        \item No explicit agent-to-agent handoffs
        \item Simplifies control and orchestration
    \end{itemize}
\end{frame}

%%%%%%%%%%%%%%%%%%%%%%%%%%%%%%%%%%%%%%%%%%%%%%%%%%%%%%%%%%%
\begin{frame}[fragile]\frametitle{Agents Improve Tool Interfaces}
    \begin{itemize}
        \item Claude uses a tool testing agent to refine tool descriptions
        \item Agent rewrites unclear interfaces after testing failures
        \item Result: 40\% reduction in task time for future agents
    \end{itemize}
\end{frame}

%%%%%%%%%%%%%%%%%%%%%%%%%%%%%%%%%%%%%%%%%%%%%%%%%%%%%%%%%%%
\begin{frame}[fragile]\frametitle{Self Improving Agents in Practice}
    \begin{itemize}
        \item Tool ergonomics improved through feedback loops
        \item Agents help reduce integration complexity
        \item Smarter interface   fewer downstream errors
    \end{itemize}
\end{frame}

%%%%%%%%%%%%%%%%%%%%%%%%%%%%%%%%%%%%%%%%%%%%%%%%%%%%%%%%%%%
\begin{frame}[fragile]\frametitle{Synchronous Execution   Bottlenecks}
    \begin{itemize}
        \item Claude's agents wait synchronously for sub agent results
        \item Simplifies coordination, but delays execution
        \item Creates sequential bottlenecks in agent chains
    \end{itemize}
\end{frame}

%%%%%%%%%%%%%%%%%%%%%%%%%%%%%%%%%%%%%%%%%%%%%%%%%%%%%%%%%%%
\begin{frame}[fragile]\frametitle{The Case for Async Architectures}
    \begin{itemize}
        \item Event driven models allow async agent execution
        \item Each agent acts as events arrive—faster coordination
        \item Matches design in frameworks like LlamaIndex workflows
    \end{itemize}
\end{frame}

%%%%%%%%%%%%%%%%%%%%%%%%%%%%%%%%%%%%%%%%%%%%%%%%%%%%%%%%%%%
\begin{frame}[fragile]\frametitle{Key Lessons from Claude Research}
    \begin{itemize}
        \item Use multi agent design selectively and purposefully
        \item Let agents optimize tools and interfaces over time
        \item Consider async architectures to eliminate bottlenecks
    \end{itemize}
\end{frame}

%%%%%%%%%%%%%%%%%%%%%%%%%%%%%%%%%%%%%%%%%%%%%%%%%%%%%%%%%%%%%%%%%%%%%%%%%%%%%%%%%%
\begin{frame}[fragile]\frametitle{}
\begin{center}
{\Large Practical Tips}
\end{center}
\end{frame}

%%%%%%%%%%%%%%%%%%%%%%%%%%%%%%%%%%%%%%%%%%%%%%%%%%%%%%%%%%%
\begin{frame}[fragile]\frametitle{RAG vs. Fine-Tuning: Business Impact}
    \begin{itemize}
        \item Business-critical systems require the right LLM strategy.
        \item Fine-tuning feels powerful, but often adds avoidable complexity.
        \item Start by exploring simpler and more flexible approaches first.
    \end{itemize}
\end{frame}

%%%%%%%%%%%%%%%%%%%%%%%%%%%%%%%%%%%%%%%%%%%%%%%%%%%%%%%%%%%
\begin{frame}[fragile]\frametitle{RAG vs SFT}
	
	\begin{center}
	\includegraphics[width=0.8\linewidth,keepaspectratio]{vizuara1}
	\end{center}
	
{\tiny (Ref: LinkedIn post by Raj Dandekar)}

\end{frame}


%%%%%%%%%%%%%%%%%%%%%%%%%%%%%%%%%%%%%%%%%%%%%%%%%%%%%%%%%%%
\begin{frame}[fragile]\frametitle{Pre-Fine-Tuning Checklist}
    \begin{itemize}
        \item Can prompt engineering alone solve the task?
        \item Could Retrieval-Augmented Generation (RAG) help more?
        \item Is there a clear and testable system already in place?
    \end{itemize}
\end{frame}

%%%%%%%%%%%%%%%%%%%%%%%%%%%%%%%%%%%%%%%%%%%%%%%%%%%%%%%%%%%
\begin{frame}[fragile]\frametitle{Why RAG Outperformed Fine-Tuning}
    \begin{itemize}
        \item \textbf{Higher Accuracy}: Grounded answers from relevant context.
        \item \textbf{Fewer Hallucinations}: More reliable than fine-tuned outputs.
        \item \textbf{Dynamic Updates}: Easily update data without retraining models.
    \end{itemize}
\end{frame}

%%%%%%%%%%%%%%%%%%%%%%%%%%%%%%%%%%%%%%%%%%%%%%%%%%%%%%%%%%%
\begin{frame}[fragile]\frametitle{When to Use Fine-Tuning}
    \begin{itemize}
        \item RAG can't solve context window limitations.
        \item Domain-specific tone or behavior is required.
        \item The system is mature enough to absorb added complexity.
    \end{itemize}
\end{frame}

%%%%%%%%%%%%%%%%%%%%%%%%%%%%%%%%%%%%%%%%%%%%%%%%%%%%%%%%%%%
\begin{frame}[fragile]\frametitle{Strategy Summary}
    \begin{itemize}
        \item Prompt engineering solves 30--50\% of tasks.
        \item RAG adds power for another 30--40\%.
        \item Fine-tuning is best for the final 10\% of hard problems.
        \item Always choose the simplest effective approach first.
    \end{itemize}
\end{frame}

%%%%%%%%%%%%%%%%%%%%%%%%%%%%%%%%%%%%%%%%%%%%%%%%%%%%%%%%%%%
\begin{frame}[fragile]\frametitle{A2A MCP ADK}
	
	\begin{center}
	\includegraphics[width=0.4\linewidth,keepaspectratio]{agents7}
	\end{center}
	
{\tiny (Ref: LinkedIn post by Deepak Jaiswal)}

\end{frame}

%%%%%%%%%%%%%%%%%%%%%%%%%%%%%%%%%%%%%%%%%%%%%%%%%%%%%%%%%%%
\begin{frame}[fragile]\frametitle{Agentic AI Protocols Overview}
    \begin{itemize}
        \item A2A, MCP, and ADK are foundational for agentic systems.
        \item Each solves a unique challenge in building autonomous agents.
        \item They work together to enable scalable multi-agent architectures.
    \end{itemize}
\end{frame}

%%%%%%%%%%%%%%%%%%%%%%%%%%%%%%%%%%%%%%%%%%%%%%%%%%%%%%%%%%%
\begin{frame}[fragile]\frametitle{A2A: Agent-to-Agent Protocol}
    \begin{itemize}
        \item Enables agent discovery, delegation, and communication.
        \item Facilitates coordination in distributed multi-agent systems.
        \item Example: Agent A delegates a task to Agent B and receives results.
    \end{itemize}
\end{frame}

%%%%%%%%%%%%%%%%%%%%%%%%%%%%%%%%%%%%%%%%%%%%%%%%%%%%%%%%%%%
\begin{frame}[fragile]\frametitle{MCP: Model Context Protocol}
    \begin{itemize}
        \item Standardizes agent access to tools, APIs, and data.
        \item Ensures consistent, secure interaction with external systems.
        \item Example: Agent queries a database or triggers a payment via MCP.
    \end{itemize}
\end{frame}

%%%%%%%%%%%%%%%%%%%%%%%%%%%%%%%%%%%%%%%%%%%%%%%%%%%%%%%%%%%
\begin{frame}[fragile]\frametitle{ADK: Agent Development Kit}
    \begin{itemize}
        \item Toolkit for building A2A-compliant agents quickly.
        \item Includes libraries and scaffolds for rapid development.
        \item Compatible with frameworks like CrewAI, LangGraph, Semantic Kernel.
    \end{itemize}
\end{frame}

%%%%%%%%%%%%%%%%%%%%%%%%%%%%%%%%%%%%%%%%%%%%%%%%%%%%%%%%%%%
\begin{frame}[fragile]\frametitle{How They Work Together}
    \begin{itemize}
        \item \textbf{MCP}: Makes agents powerful with tool access.
        \item \textbf{A2A}: Enables inter-agent collaboration.
        \item \textbf{ADK}: Simplifies and accelerates agent development.
    \end{itemize}
\end{frame}

%%%%%%%%%%%%%%%%%%%%%%%%%%%%%%%%%%%%%%%%%%%%%%%%%%%%%%%%%%%
\begin{frame}[fragile]\frametitle{Real-World Analogy}
    \begin{itemize}
        \item \textbf{MCP}: Tools in a mechanic’s toolbox.
        \item \textbf{A2A}: Team communication and task-sharing.
        \item \textbf{ADK}: Blueprint for building each mechanic fast.
    \end{itemize}
\end{frame}

%%%%%%%%%%%%%%%%%%%%%%%%%%%%%%%%%%%%%%%%%%%%%%%%%%%%%%%%%%%
\begin{frame}[fragile]\frametitle{Why This Stack Matters}
    \begin{itemize}
        \item Forms the backbone of enterprise AI, robotics, and automation.
        \item Enables modular, scalable, and intelligent agentic systems.
        \item Quickly becoming a standard for future AI workflows.
    \end{itemize}
\end{frame}
