%%%%%%%%%%%%%%%%%%%%%%%%%%%%%%%%%%%%%%%%%%%%%%%%%%%%%%%%%%%%%%%%%%%%%%%%%%%%%%%%%%
\begin{frame}[fragile]\frametitle{}
\begin{center}
{\Large Introduction to Graph RAG}
\end{center}
\end{frame}

%%%%%%%%%%%%%%%%%%%%%%%%%%%%%%%%%%%%%%%%%%%%%%%%%%%%%%%%%%%
\begin{frame}[fragile]\frametitle{Challenges with LLMs}
    \begin{itemize}
        \item Learns random sentences from random people
        \item Talks like a person but doesn't really understand what it's saying
        \item Occasionally speaks absolute non sense
        \item Sensitive to question phrasing
        \item Limited to public ``knowledge''
    \end{itemize}
	
	{\tiny (Ref: The GenAI Stack - Andreas Kollegger - Neo4j)}
	
\end{frame}

%%%%%%%%%%%%%%%%%%%%%%%%%%%%%%%%%%%%%%%%%%%%%%%%%%%%%%%%%%%
\begin{frame}[fragile]\frametitle{Thematic RAG Classification}

	\begin{center}
	\includegraphics[width=\linewidth,keepaspectratio]{graphrag14}
	\end{center}
	
		{\tiny (Ref: https://graphrag.com/concepts/intro-to-graphrag/))}

	
\end{frame}

%%%%%%%%%%%%%%%%%%%%%%%%%%%%%%%%%%%%%%%%%%%%%%%%%%%%%%%%%%%
\begin{frame}[fragile]\frametitle{The phases of an advanced RAG system}
    \begin{itemize}
        \item Pre-retrieval-Query rewriting, query entity extraction, query expansion, etc.
        \item Retrieval of relevant context
        \item Post-retrieval: Reranking, pruning, etc.
        \item Answer generation
    \end{itemize}
	
		{\tiny (Ref: https://graphrag.com/concepts/intro-to-graphrag/))}
	
\end{frame}


%%%%%%%%%%%%%%%%%%%%%%%%%%%%%%%%%%%%%%%%%%%%%%%%%%%%%%%%%%%
\begin{frame}[fragile]\frametitle{Why Graph RAG?}
    \begin{itemize}
        \item Language models struggle with factual accuracy and real-world knowledge.
        \item Retrieval-Augmented Generation (RAG) improves accuracy using external text data.
        \item Traditional RAG has limitations in context understanding and scalability.
        \item GraphRAG leverages knowledge graphs for better retrieval and response generation.
    \end{itemize}
\end{frame}

%%%%%%%%%%%%%%%%%%%%%%%%%%%%%%%%%%%%%%%%%%%%%%%%%%%%%%%%%%%
\begin{frame}[fragile]\frametitle{Limitations of Traditional RAG}
    \begin{itemize}
        \item \textbf{Flat Retrieval:} Documents are treated as isolated entities.
        \item \textbf{Contextual Shortcomings:} Lacks deep semantic understanding.
        \item \textbf{Scalability Issues:} Slower retrieval with increasing data volume.
        \item Struggles with scalability due to flat data representation.
        \item Lacks global context over the entire data corpus.
        \item Inefficient for complex reasoning across multiple documents.
        \item \textbf{Lack of Explainability:} Hard to trace the source of retrieved information.
        \item \textbf{Local Window:} Limited chunk-level context leads to incomplete responses.
        \item \textbf{Loss of Structural Relationships:} Ignores hierarchical relationships in data.			
    \end{itemize}
\end{frame}


%%%%%%%%%%%%%%%%%%%%%%%%%%%%%%%%%%%%%%%%%%%%%%%%%%%%%%%%%%%
\begin{frame}[fragile]\frametitle{Challenges with Microsoft's GraphRAG}
    \begin{itemize}
        \item Too expensive and impractical for large-scale industrial use.
        \item Most companies prefer standard vector databases.
        \item Lack of widespread production adoption due to cost and complexity.
    \end{itemize}
\end{frame}

%%%%%%%%%%%%%%%%%%%%%%%%%%%%%%%%%%%%%%%%%%%%%%%%%%%%%%%%%%%
\begin{frame}[fragile]\frametitle{Text-To-Cypher/SPARQL Alternative}
    \begin{itemize}
        \item Effective alternative to Microsoft's GraphRAG.
        \item Requires costly LLM calls for query generation.
        \item Adds uncertainty due to prompt effectiveness and model performance.
        \item Increases response time and implementation complexity.
    \end{itemize}
\end{frame}

%%%%%%%%%%%%%%%%%%%%%%%%%%%%%%%%%%%%%%%%%%%%%%%%%%%%%%%%%%%
\begin{frame}[fragile]\frametitle{What is GraphRAG?}
    \begin{itemize}
        \item No universally accepted definition yet.
        \item Some associate it with Microsoft's graph-based search approach.
        \item Others define it as querying LPG or RDF graphs using LLM-generated queries (Cypher, SPARQL).	
        \item Uses knowledge graphs instead of unstructured text.
        \item Captures entities, relationships, and hierarchical structures.
        \item Enables accurate, context-aware retrieval and response generation.
        \item Supports complex query handling with enhanced explainability.
        \item Combines structured Knowledge Graphs (KGs) with semantic vectors.
        \item Enables LLMs to reason over multi-hop connections.
        \item Provides a holistic perspective by connecting different data sources.		
    \end{itemize}
\end{frame}

%%%%%%%%%%%%%%%%%%%%%%%%%%%%%%%%%%%%%%%%%%%%%%%%%%%%%%%%%%%
\begin{frame}[fragile]\frametitle{Basics of GraphRAG }

	\begin{center}
	\includegraphics[width=\linewidth,keepaspectratio]{graphrag13}
	\end{center}
	
		{\tiny (Ref: https://graphrag.com/concepts/intro-to-graphrag/))}

	
\end{frame}

%%%%%%%%%%%%%%%%%%%%%%%%%%%%%%%%%%%%%%%%%%%%%%%%%%%%%%%%%%%
\begin{frame}[fragile]\frametitle{Why RAG?}
    \begin{itemize}
        \item RAG is widely used for real-world enterprise applications.
        \item Augments external knowledge sources to query private corpora.
        \item Traditional vector-based RAG relies only on semantic similarity.
    \end{itemize}
\end{frame}

%%%%%%%%%%%%%%%%%%%%%%%%%%%%%%%%%%%%%%%%%%%%%%%%%%%%%%%%%%%
\begin{frame}[fragile]\frametitle{Key Features of GraphRAG}
    \begin{itemize}
        \item \textbf{Structured Representation:} Uses knowledge graphs.
        \item \textbf{Contextual Retrieval:} Understands semantic relationships.
        \item \textbf{Efficient Processing:} Reduces computational cost.
        \item \textbf{Multi-Faceted Queries:} Synthesizes data from multiple sources.
        \item \textbf{Explainability:} More transparent than black-box models.
        \item \textbf{Continuous Learning:} Expands knowledge over time.
    \end{itemize}
\end{frame}

%%%%%%%%%%%%%%%%%%%%%%%%%%%%%%%%%%%%%%%%%%%%%%%%%%%%%%%%%%%
\begin{frame}[fragile]\frametitle{}

	\begin{center}
	\includegraphics[width=\linewidth,keepaspectratio]{rag_vs_graphrag}
	\end{center}
	
\end{frame}


%%%%%%%%%%%%%%%%%%%%%%%%%%%%%%%%%%%%%%%%%%%%%%%%%%%%%%%%%%%
\begin{frame}[fragile]\frametitle{Limitations of Traditional RAG}
	
	\begin{center}
	\includegraphics[width=\linewidth,keepaspectratio]{graphrag16}
	
	{\tiny (Ref: GraphRAG: The Practical Guide for Cost-Effective Document Analysis with Knowledge Graphs -Jaykumaran)}
	\end{center}	
\end{frame}

%%%%%%%%%%%%%%%%%%%%%%%%%%%%%%%%%%%%%%%%%%%%%%%%%%%%%%%%%%%
\begin{frame}[fragile]\frametitle{Challenges in Standard RAG}
    \begin{itemize}
        \item \textbf{Lack of Explainability:} Hard to trace the source of retrieved information.
        \item \textbf{Local Window:} Limited chunk-level context leads to incomplete responses.
        \item \textbf{Scalability Issues:} Struggles with large-scale medical and legal corpora.
        \item \textbf{Loss of Structural Relationships:} Ignores hierarchical relationships in data.
    \end{itemize}
\end{frame}

%%%%%%%%%%%%%%%%%%%%%%%%%%%%%%%%%%%%%%%%%%%%%%%%%%%%%%%%%%%
\begin{frame}[fragile]\frametitle{GraphRAG: Evolution of Knowledge Graphs}
    \begin{itemize}
        \item Earlier, KGs were built using statistical and NLP techniques.
        \item GraphRAG scales efficiently by using LLMs for entity extraction.
        \item Entities in KG are nodes, linked by edges encoding relationships.
    \end{itemize}
\end{frame}

%%%%%%%%%%%%%%%%%%%%%%%%%%%%%%%%%%%%%%%%%%%%%%%%%%%%%%%%%%%
\begin{frame}[fragile]\frametitle{Advantages of Knowledge Graphs}
    \begin{itemize}
        \item Connects information from multiple sources for deeper insights.
        \item Boosts retrieval accuracy and enables multi-hop reasoning.
        \item Enhances LLM responses by integrating structured relationships.
    \end{itemize}
\end{frame}

%%%%%%%%%%%%%%%%%%%%%%%%%%%%%%%%%%%%%%%%%%%%%%%%%%%%%%%%%%%
\begin{frame}[fragile]\frametitle{Key Features of GraphRAG}
    \begin{itemize}
        \item Maintains hierarchical communities preserving local and global insights.
        \item Ensures source traceability down to the node level for citations.
        \item Aggregates information from multiple sources, reducing bias.
        \item Focuses on meaningful nodes, filtering out irrelevant information.
    \end{itemize}
\end{frame}


%%%%%%%%%%%%%%%%%%%%%%%%%%%%%%%%%%%%%%%%%%%%%%%%%%%%%%%%%%%
\begin{frame}[fragile]\frametitle{Example: Medical Query Challenge}
    \begin{itemize}
        \item Query: How does Medication A in Patient Record 1 affect Condition B in Patient Record 2?
        \item LLM needs to infer relationships across multiple records.
        \item Standard RAG struggles with such complex dependencies.
        \item Scaling this to millions of patient records is infeasible.
    \end{itemize}
	
	\begin{center}
	\includegraphics[width=0.8\linewidth,keepaspectratio]{graphrag15}
	
	{\tiny (Ref: GraphRAG: The Practical Guide for Cost-Effective Document Analysis with Knowledge Graphs -Jaykumaran)}
	\end{center}	
\end{frame}

%%%%%%%%%%%%%%%%%%%%%%%%%%%%%%%%%%%%%%%%%%%%%%%%%%%%%%%%%%%
\begin{frame}[fragile]\frametitle{Vector RAG vs GraphRAG}

	
	\begin{center}
	\includegraphics[width=0.5\linewidth,keepaspectratio]{graphrag17}
	
	{\tiny (Ref: GraphRAG: The Practical Guide for Cost-Effective Document Analysis with Knowledge Graphs -Jaykumaran)}
	\end{center}	
\end{frame}

%%%%%%%%%%%%%%%%%%%%%%%%%%%%%%%%%%%%%%%%%%%%%%%%%%%%%%%%%%%
\begin{frame}[fragile]\frametitle{Traditional RAG vs GraphRAG}
    \begin{table}[]
        \centering
        \begin{tabular}{|l|l|l|}
            \hline
            \textbf{Feature} & \textbf{Traditional RAG} & \textbf{GraphRAG} \\
            \hline
            Data Representation & Flat Vectors & Knowledge Graph \\
            \hline
            Query Scope & Local context & Global Reasoning \\
            \hline
            Scalability & Low & High \\
            \hline
            Citation Transparency & Low & High (Traceable sources) \\
            \hline
            Response Coherence & Fragmented & Relevant and Context-Rich \\
            \hline
        \end{tabular}
    \end{table}
\end{frame}

%%%%%%%%%%%%%%%%%%%%%%%%%%%%%%%%%%%%%%%%%%%%%%%%%%%%%%%%%%%
\begin{frame}[fragile]\frametitle{How GraphRAG Solves the Problem}
    \begin{itemize}
        \item GraphRAG combines graph structures with vector search.
        \item Traverses multi-hop connections to infer relationships.
        \item Offers more reliable responses than vector-only RAG.
    \end{itemize}

\textit{Graph-based RAG helps agentic AI make human-like decisions.} 
– May Habib, CEO of Writer.com
\end{frame}




%%%%%%%%%%%%%%%%%%%%%%%%%%%%%%%%%%%%%%%%%%%%%%%%%%%%%%%%%%%
\begin{frame}[fragile]\frametitle{Applications of GraphRAG}
    \begin{itemize}
        \item \textbf{Healthcare:} Assists in diagnoses and treatment decisions.
        \item \textbf{Banking:} Detects fraudulent transactions using knowledge graphs.
        \item \textbf{Customer Service :} quickly answer customer questions from thousands of pages of policy documentation
        \item \textbf{Recommendations:} understand customer behavior and preferences better, to provide personalized services.
        \item \textbf{Supply Chain:} product recall and associated quality control checking, internal documentation search
	
    \end{itemize}

\end{frame}

%%%%%%%%%%%%%%%%%%%%%%%%%%%%%%%%%%%%%%%%%%%%%%%%%%%%%%%%%%%
\begin{frame}[fragile]\frametitle{How GraphRAG Works?}
    \begin{itemize}
        \item \textbf{Knowledge Graph Construction:} Extracts entities and relationships.
        \item \textbf{Knowledge Graph Summarization:} Generates hierarchical summaries.
        \item \textbf{Retrieval-Augmented Generation:} Uses local and global searches for queries.
    \end{itemize}
\end{frame}

%%%%%%%%%%%%%%%%%%%%%%%%%%%%%%%%%%%%%%%%%%%%%%%%%%%%%%%%%%%
\begin{frame}[fragile]\frametitle{Example of GraphRAG Representation}
    \begin{lstlisting}
    # Entities
    Type 2 Diabetes (Condition)
    High Blood Sugar Levels (Symptom)
    Nerve Damage (Complication)
    Kidney Disease (Complication)
    Cardiovascular Problems (Complication)

    # Relationships
    Type 2 Diabetes -> has_symptom -> High Blood Sugar Levels
    Type 2 Diabetes -> can_lead_to -> Nerve Damage
    Type 2 Diabetes -> can_lead_to -> Kidney Disease
    Type 2 Diabetes -> can_lead_to -> Cardiovascular Problems
    \end{lstlisting}
\end{frame}


%%%%%%%%%%%%%%%%%%%%%%%%%%%%%%%%%%%%%%%%%%%%%%%%%%%%%%%%%%%
\begin{frame}[fragile]\frametitle{Advantages of GraphRAG}
    \begin{itemize}
        \item Structured knowledge representation.
        \item Context-aware and efficient retrieval.
        \item Handles complex queries effectively.
        \item Provides explainability and transparency.
    \end{itemize}
\end{frame}

%%%%%%%%%%%%%%%%%%%%%%%%%%%%%%%%%%%%%%%%%%%%%%%%%%%%%%%%%%%
\begin{frame}[fragile]\frametitle{}

	\begin{center}
	\includegraphics[width=\linewidth,keepaspectratio]{graphrag8}
	\end{center}
	
\end{frame}


%%%%%%%%%%%%%%%%%%%%%%%%%%%%%%%%%%%%%%%%%%%%%%%%%%%%%%%%%%%
\begin{frame}[fragile]\frametitle{}

	\begin{center}
	\includegraphics[width=\linewidth,keepaspectratio]{graphrag9}
	\end{center}
	
\end{frame}

%%%%%%%%%%%%%%%%%%%%%%%%%%%%%%%%%%%%%%%%%%%%%%%%%%%%%%%%%%%
\begin{frame}[fragile]\frametitle{}

	\begin{center}
	\includegraphics[width=\linewidth,keepaspectratio]{graphrag10}
	\end{center}
	
\end{frame}

%%%%%%%%%%%%%%%%%%%%%%%%%%%%%%%%%%%%%%%%%%%%%%%%%%%%%%%%%%%
\begin{frame}[fragile]\frametitle{Challenges of GraphRAG}
    \begin{itemize}
        \item \textbf{Complex Knowledge Graph Construction:} Requires sophisticated NLP techniques.
        \item \textbf{Data Dependency:} Performance relies on input data quality.
        \item \textbf{Scalability Issues:} Large graphs require significant computational resources.
    \end{itemize}
\end{frame}

%%%%%%%%%%%%%%%%%%%%%%%%%%%%%%%%%%%%%%%%%%%%%%%%%%%%%%%%%%%
\begin{frame}[fragile]\frametitle{Trend of GraphRAG Research}

	\begin{center}
	\includegraphics[width=\linewidth,keepaspectratio]{graphrag12}
	\end{center}
	
		{\tiny (Ref: Awesome-GraphRAG (GraphRAG Survey))}

	
\end{frame}

%%%%%%%%%%%%%%%%%%%%%%%%%%%%%%%%%%%%%%%%%%%%%%%%%%%%%%%%%%%
\begin{frame}[fragile]\frametitle{Conclusion}
    \begin{itemize}
        \item GraphRAG enhances traditional RAG models using structured knowledge.
        \item Improves accuracy, context-awareness, and efficiency.
        \item Useful in various domains like healthcare and banking.
        \item A promising approach for future AI-powered knowledge retrieval.
    \end{itemize}
\end{frame}
