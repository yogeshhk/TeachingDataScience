%%%%%%%%%%%%%%%%%%%%%%%%%%%%%%%%%%%%%%%%%%%%%%%%%%%%%%%%%%%%%%%%%%%%%%%%%%%%%%%%%%
\begin{frame}[fragile]\frametitle{}
\begin{center}
{\Large Introduction to LangChain}
\end{center}
\end{frame}

%%%%%%%%%%%%%%%%%%%%%%%%%%%%%%%%%%%%%%%%%%%%%%%%%%%%%%%%%%%%%%%%%%%%%%%%%%%%%%%%%%
\begin{frame}\frametitle{What is LangChain?}

\begin{columns}
    \begin{column}{0.5\textwidth}
        \textbf{A Framework for Building LLM Applications}
        
        \vspace{0.3cm}
        
        \textbf{Key Capabilities:}
        \begin{itemize}
        \item Connect LLMs to external data (RAG)
        \item Build intelligent agents
        \item Manage prompts \& memory
        \item Switch between models easily
        \end{itemize}
        
        \vspace{0.3cm}
        
        \textbf{Quick Facts:}
        \begin{itemize}
        \item Open Source (MIT License)
        \item Python \& JavaScript
        \item Created by Harrison Chase
        \end{itemize}
    \end{column}
    \begin{column}{0.5\textwidth}
        \begin{center}
        % SUGGESTED IMAGE: LangChain logo or architecture diagram
        % Use langchain15 or a simplified flowchart showing:
        % LLM → LangChain → Application
        \includegraphics[width=\linewidth,keepaspectratio]{langchain_logo}
        
        \vspace{0.5cm}
        
        \textbf{From Idea to Production}
        
        \includegraphics[width=0.8\linewidth,keepaspectratio]{langchain17}
        \end{center}
    \end{column}
\end{columns}

{\tiny (Ref: Getting Started with LangChain: A Beginner's Guide)}
\end{frame}

% %%%%%%%%%%%%%%%%%%%%%%%%%%%%%%%%%%%%%%%%%%%%%%%%%%%%%%%%%%%%%%%%%%%%%%%%%%%%%%%%%%
% \begin{frame}\frametitle{What is LangChain?}

% \textbf{LangChain}: A comprehensive framework for building LLM-powered applications

% \begin{itemize}
% \item \textbf{Core Purpose}: Simplify development of applications using LLMs
% \item \textbf{Key Solutions}:
    % \begin{itemize}
    % \item \textbf{RAG}: Connect language models to external data sources
    % \item \textbf{Agentic}: Allow language models to interact with their environment
    % \end{itemize}
% \item \textbf{Main Features}:
    % \begin{itemize}
    % \item Generic interface to various foundation models
    % \item Advanced prompt management framework
    % % \item Central interface to memory, external data, and agents
    % \end{itemize}
% \item \textbf{Availability}: Python and JavaScript libraries
% \item \textbf{Open Source}: MIT License, created by Harrison Chase
% \item \textbf{Repository}: https://github.com/langchain-ai/langchain
% \end{itemize}

% {\tiny (Ref: Getting Started with LangChain: A Beginner's Guide to Building LLM-Powered Applications)}
% \end{frame}

% %%%%%%%%%%%%%%%%%%%%%%%%%%%%%%%%%%%%%%%%%%%%%%%%%%%%%%%%%%%%%%%%%%%%%%%%%%%%%%%%%%
% \begin{frame}[fragile]\frametitle{LangChain vs LLMs vs ChatGPT}

% {\tiny (Ref: LangChain 101: Building Simple Q\&A App)}


% \begin{lstlisting}[language=python, basicstyle=\tiny]
% +==========+========================+====================+====================+
% |          | LangChain              | LLM                | ChatGPT            | 
% +==========+========================+====================+====================+
% | Type     | Framework              | Model              | Model              | 
% +----------+------------------------+--------------------+--------------------+
% | Purpose  | Build applications     | Generate text      | Generate chat      | 
% |          | with LLMs              |                    | conversations      | 
% +----------+------------------------+--------------------+--------------------+
% | Features | Chains, prompts, LLMs, | Large dataset of   | Large dataset of   | 
% |          | memory, index, agents  | text and code      | chat conversations | 
% +----------+------------------------+--------------------+--------------------+
% | Pros     | Composable, modular,   | Generates nearly   | Generates realistic| 
% |          | production-ready       | human-quality text | chat conversations | 
% +----------+------------------------+--------------------+--------------------+
% | Use Case | Building RAG apps,     | Direct text        | Conversational AI  | 
% |          | agents, workflows      | generation         | applications       | 
% +----------+------------------------+--------------------+--------------------+
% \end{lstlisting}

% \end{frame}

% %%%%%%%%%%%%%%%%%%%%%%%%%%%%%%%%%%%%%%%%%%%%%%%%%%%%%%%%%%%
% \begin{frame}[fragile]\frametitle{Why You Need LangChain}
    % \begin{itemize}
        % \item LLMs alone lack context, memory, and retrieval abilities.
        % \item Chatbots need to manage chat history and company knowledge bases.
        % \item Storing, retrieving, and reasoning over data is complex manually.
        % \item LangChain acts as an abstraction layer for building AI agents.
        % \item It integrates LLMs, memory, and tools seamlessly.
        % \item Enables vendor flexibility, easy switch between Open/Close models.
        % \item Reduces code complexity and accelerates AI app development.
        % \item Simplifies connecting models, databases, and APIs into a single framework.
    % \end{itemize}
% \end{frame}

%%%%%%%%%%%%%%%%%%%%%%%%%%%%%%%%%%%%%%%%%%%%%%%%%%%%%%%%%%%
\begin{frame}[fragile]\frametitle{Why You Need LangChain}

\begin{columns}
    \begin{column}{0.5\textwidth}
        \textbf{The Challenge:}
        \begin{itemize}
        \item LLMs lack memory
        \item No access to your data
        \item Can't use external tools
        \item Complex integration code
        \end{itemize}
        
        \vspace{0.5cm}
        
        \textbf{The Solution:}
        \begin{itemize}
        \item Memory management built-in
        \item Easy data integration
        \item Tool calling framework
        \item Simple, modular code
        \end{itemize}
    \end{column}
    \begin{column}{0.5\textwidth}
        \begin{center}
        % SUGGESTED IMAGE: Before/After comparison
        % Left side: Tangled mess of API calls
        % Right side: Clean LangChain pipeline
        \includegraphics[width=\linewidth,keepaspectratio]{before_after_langchain}
		
        {\tiny (Ref: https://www.leanware.co/insights/langchain-vs-pinecone-which-should-you-choose)}
		
        \vspace{0.5cm}
        
        \fcolorbox{blue}{blue!10}{
        \parbox{0.9\linewidth}{
        \centering
        \textbf{Bottom Line:}\\
        \vspace{0.2cm}
        Build AI apps faster\\
        with less code
        }
        }
        \end{center}
    \end{column}
\end{columns}

\end{frame}

% %%%%%%%%%%%%%%%%%%%%%%%%%%%%%%%%%%%%%%%%%%%%%%%%%%%%%%%%%%%
% \begin{frame}[fragile]\frametitle{LLMs vs Agentic Software}
    % \begin{itemize}
        % \item LLMs are static, respond from training data without awareness.
        % \item Agents have autonomy, memory, and tool use for dynamic actions.
        % \item Example: Refund query, agent retrieves policy, product, and chat history.
        % \item Agents can access vector databases and retrieve company knowledge.
        % \item LangChain enables memory persistence across user conversations.
        % \item Traditional software follows fixed logic; agents make adaptive decisions.
        % \item Agentic software uses modular components (LLMs, memory, retrievers).
        % \item LangChain provides prebuilt tools for these agentic capabilities.
    % \end{itemize}
% \end{frame}

% %%%%%%%%%%%%%%%%%%%%%%%%%%%%%%%%%%%%%%%%%%%%%%%%%%%%%%%%%%%
% \begin{frame}[fragile]\frametitle{LangChain Core Components and Labs}
    % \begin{itemize}
        % \item Core modules: LLM connectors, memory, embeddings, vector stores.
        % \item Simplifies LLM API setup, e.g., ChatOpenAI() or ChatGroq() %replaces manual integration.
        % \item Supports databases like Chroma or Pinecone for knowledge retrieval.
        % \item Prompt templates manage dynamic, reusable prompts and chat context.
        % \item LCEL (LangChain Expression Language) composes chains without legacy LLMChain wrappers
        % \item Enables async, streaming, and batch workflows with type safety.
    % \end{itemize}
% \end{frame}

%%%%%%%%%%%%%%%%%%%%%%%%%%%%%%%%%%%%%%%%%%%%%%%%%%%%%%%%%%%
\begin{frame}[fragile]\frametitle{Core Building Blocks}

\begin{center}
\includegraphics[width=0.7\linewidth,keepaspectratio]{langchain16}
\end{center}

\begin{columns}
    \begin{column}{0.5\textwidth}
        \textbf{Foundation:}
        \begin{itemize}
        \item \textbf{Models}: Connect to any LLM
        \item \textbf{Prompts}: Dynamic templates
        \item \textbf{Memory}: Conversation context
        \end{itemize}
    \end{column}
    \begin{column}{0.5\textwidth}
        \textbf{Advanced:}
        \begin{itemize}
        \item \textbf{Retrievers}: Search your data
        \item \textbf{Agents}: Autonomous reasoning
        \item \textbf{LCEL}: Chain components easily
        \end{itemize}
    \end{column}
\end{columns}

\vspace{0.3cm}

\begin{center}
\fcolorbox{green}{green!10}{
\parbox{0.8\linewidth}{
\centering
\textbf{Mix and match components to build your app}
}
}
\end{center}

\end{frame}

%%%%%%%%%%%%%%%%%%%%%%%%%%%%%%%%%%%%%%%%%%%%%%%%%%%%%%%%%%%%%%%%%%%%%%%%%%%%%%%%%%
\begin{frame}[fragile]\frametitle{So, Why LangChain?}

  \begin{columns}
    \begin{column}{0.5\textwidth}
      \begin{itemize}
    \item \textbf{RAG Applications}: Connect LLMs to your data
    \item \textbf{Intelligent Agents}: Autonomous tool usage
    \item \textbf{Production Ready}: Monitoring \& logging built-in
    \item \textbf{Model Flexibility}: Switch providers easily
    \item \textbf{Modular Design}: Use only what you need
      \end{itemize}
    \end{column}
    \begin{column}{0.5\textwidth}
			\begin{center}
			\includegraphics[width=\linewidth,keepaspectratio]{langchain15}
			\end{center}	  
			{\tiny (Ref: LangChain tutorial: Build an LLM-powered app)}
    \end{column}
  \end{columns}

\end{frame}

%%%%%%%%%%%%%%%%%%%%%%%%%%%%%%%%%%%%%%%%%%%%%%%%%%%%%%%%%%%%%%%%%%%%%%%%%%%%%%%%%%
\begin{frame}[fragile]\frametitle{The LangChain Community}


 
        \begin{itemize}
        \item 80,000+ GitHub stars
        \item 2,000+ contributors
        \item Millions of downloads/month
        \item Active Discord community
        \item MIT - Commercial friendly
        \end{itemize}
		
{\tiny (Ref: What is Langchain and why should I care as a developer? - Logan Kilpatrick)}
		



\end{frame}


%%%%%%%%%%%%%%%%%%%%%%%%%%%%%%%%%%%%%%%%%%%%%%%%%%%%%%%%%%%%%%%%%%%%%%%%%%%%%%%%%%
\begin{frame}[fragile]\frametitle{The LangChain Ecosystem}

        
        \begin{center}
        \includegraphics[width=0.9\linewidth,keepaspectratio]{langchain_ecosystem_diagram}
		
		{\tiny (Ref: Demystifying the LangChain Ecosystem for LLM-Powered Application Development - Wltsankalpa)}

        \end{center}
        
\end{frame}


% %%%%%%%%%%%%%%%%%%%%%%%%%%%%%%%%%%%%%%%%%%%%%%%%%%%%%%%%%%%%%%%%%%%%%%%%%%%%%%%%%%
% \begin{frame}[fragile]\frametitle{LangChain Community \& Ecosystem}

% \begin{itemize}
% \item \textbf{Community Growth}:
	% \begin{itemize}
	% \item Over 80,000+ GitHub stars (as of 2024)
	% \item 2,000+ contributors
	% \item Millions of monthly downloads
	% \item Active Discord, Twitter/X presence
	% \end{itemize}
	
% \item \textbf{License}: MIT License - freely modifiable and commercial-friendly

% \item \textbf{Ecosystem Components}:
	% \begin{itemize}
	% \item \textbf{LangChain Core}: Foundation library
	% \item \textbf{LangGraph}: Stateful, multi-actor applications
	% \item \textbf{LangServe}: Deploy as REST APIs
	% \item \textbf{LangSmith}: Debugging, testing, monitoring
	% \end{itemize}
% \end{itemize}

% {\tiny (Ref: What is Langchain and why should I care as a developer? - Logan Kilpatrick)}

% \end{frame}

%%%%%%%%%%%%%%%%%%%%%%%%%%%%%%%%%%%%%%%%%%%%%%%%%%%%%%%%%%%%%%%%%%%%%%%%%%%%%%%%%%
\begin{frame}[fragile]\frametitle{Installation \& Setup}

Python 3.10 or higher

\begin{lstlisting}[language=python, basicstyle=\tiny]
# Core packages (LangChain 1.0 focuses on the 'langchain' agent framework)
pip install -U langchain langchain-core

# Provider-specific packages (Groq is a first-class integration)
pip install -U langchain-groq

# Document processing (Now separated for better dependency management)
pip install -U langchain-community langchain-text-splitters

# Embeddings & Vector Stores
pip install -U langchain-huggingface langchain-chroma

# NEW: Legacy features
# Use this ONLY if you need old chains like LLMChain or RetrievalQA
pip install -U langchain-classic

//API Keys Setup:

import os
os.environ["GROQ_API_KEY"] = "your-groq-api-key-here"
# Or use .env file with python-dotenv

# LangSmith Tracing (Highly recommended for v1.0 Agents)
os.environ["LANGCHAIN_TRACING_V2"] = "true"
os.environ["LANGCHAIN_API_KEY"] = "your-langsmith-key"
os.environ["LANGCHAIN_PROJECT"] = "my-v1-project"
\end{lstlisting}


\end{frame}

%%%%%%%%%%%%%%%%%%%%%%%%%%%%%%%%%%%%%%%%%%%%%%%%%%%%%%%%%%%%%%%%%%%%%%%%%%%%%%%%%%
\begin{frame}[fragile]\frametitle{Core Components Overview}

\begin{center}
\includegraphics[width=0.7\linewidth,keepaspectratio]{langchain16}
\end{center}

\begin{center}
\begin{tabular}{|c|l|l|}
\hline
\textbf{Component} & \textbf{Purpose} & \textbf{Use Case} \\
\hline
Models & Connect to LLMs & Text generation, chat \\
\hline
Prompts & Template management & Dynamic prompts \\
\hline
LCEL & Chain components & Build workflows \\
\hline
Memory & Store conversations & Chatbots \\
\hline
Retrievers & Search data & RAG, Q\&A \\
\hline
Agents & Tool selection & Autonomous tasks \\
\hline
\end{tabular}
\end{center}

% \begin{itemize}
% \item \textbf{Models}: LLMs, Chat Models, Embeddings
% \item \textbf{Prompts}: Template management and optimization
% \item \textbf{Chains/LCEL}: Compose components with pipes
% \item \textbf{Memory}: Conversation state management
% \item \textbf{Retrievers}: Access external data
% \item \textbf{Agents}: Dynamic tool selection and execution
% \end{itemize}

{\tiny (Ref: How LangChain Makes Large Language Models More Powerful)}

\end{frame}

%%%%%%%%%%%%%%%%%%%%%%%%%%%%%%%%%%%%%%%%%%%%%%%%%%%%%%%%%%%%%%%%%%%%%%%%%%%%%%%%%%
\begin{frame}\frametitle{Key Concepts: Chains vs Agents}

\begin{itemize}
\item \textbf{Chains (LCEL)}:
    \begin{itemize}
    \item Predetermined sequence of operations
    \item Composed with pipe operator: \texttt{prompt | llm | parser}
    \item Fixed execution path
    \item Best for: Structured, predictable workflows
    \end{itemize}

\item \textbf{Agents}:
    \begin{itemize}
    \item Dynamic decision-making with LLM reasoning
    \item Choose tools based on input
    \item Adaptive execution path
    \item Best for: Complex, unpredictable scenarios
    \end{itemize}
\end{itemize}

\textbf{When to use what?}
\begin{itemize}
\item Use \textbf{Chains/LCEL} when: Steps are known, workflow is fixed
\item Use \textbf{Agents} when: Need dynamic tool selection, multi-step reasoning
\end{itemize}

{\tiny (Ref: Superpower LLMs with Conversational Agents)}
\end{frame}

%%%%%%%%%%%%%%%%%%%%%%%%%%%%%%%%%%%%%%%%%%%%%%%%%%%%%%%%%%%%%%%%%%%%%%%%%%%%%%%%%%
\begin{frame}\frametitle{What Can You Build?}

\begin{columns}
    \begin{column}{0.5\textwidth}
        \textbf{RAG Applications:}
        \begin{itemize}
        \item Document Q\&A
        \item Knowledge base search
        \item Semantic search
        \end{itemize}
        
        \vspace{0.3cm}
        
        \textbf{Conversational AI:}
        \begin{itemize}
        \item Chatbots with memory
        \item Customer support
        \item Personal assistants
        \end{itemize}
    \end{column}
    \begin{column}{0.5\textwidth}
        \textbf{Data Analysis:}
        \begin{itemize}
        \item SQL query generation
        \item Report generation
        \item Data insights
        \end{itemize}
        
        \vspace{0.3cm}
        
        \textbf{Autonomous Agents:}
        \begin{itemize}
        \item Web research
        \item API integration
        \item Multi-step workflows
        \end{itemize}
    \end{column}
\end{columns}

% \vspace{0.5cm}

% \begin{center}
% % SUGGESTED IMAGE: 4-quadrant diagram with icons
% \includegraphics[width=0.7\linewidth,keepaspectratio]{use_cases_diagram}
% \end{center}

{\tiny (Ref: LangChain Use Cases Documentation)}
\end{frame}

% %%%%%%%%%%%%%%%%%%%%%%%%%%%%%%%%%%%%%%%%%%%%%%%%%%%%%%%%%%%%%%%%%%%%%%%%%%%%%%%%%%
% \begin{frame}\frametitle{LangChain Use Cases}

% \begin{itemize}
% \item \textbf{Retrieval Augmented Generation (RAG)}:
    % \begin{itemize}
    % \item Document Q\&A systems
    % \item Knowledge base search
    % \item Semantic search applications
    % \end{itemize}

% \item \textbf{Conversational AI}:
    % \begin{itemize}
    % \item Chatbots with memory
    % \item Customer support agents
    % \item Personal assistants
    % \end{itemize}

% \item \textbf{Data Analysis}:
    % \begin{itemize}
    % \item SQL query generation
    % \item Tabular data Q\&A
    % \item Report generation
    % \end{itemize}

% \item \textbf{Autonomous Agents}:
    % \begin{itemize}
    % \item Web scraping and research
    % \item Multi-tool workflows
    % \item API integration
    % \end{itemize}
% \end{itemize}

% {\tiny (Ref: LangChain Use Cases Documentation)}
% \end{frame}

% %%%%%%%%%%%%%%%%%%%%%%%%%%%%%%%%%%%%%%%%%%%%%%%%%%%%%%%%%%%%%%%%%%%%%%%%%%%%%%%%%%
% \begin{frame}\frametitle{What's Next?}

% \textbf{In Following Sections:}
% \begin{itemize}
% \item \textbf{Framework Deep Dive}: Models, Prompts, Memory, Retrievers
% \item \textbf{LCEL Advanced Patterns}: Complex chains, fallbacks, error handling
% \item \textbf{Agents \& Tools}: Building intelligent agents
% \item \textbf{Vector Stores}: Document indexing and retrieval
% \item \textbf{LangChain Ecosystem}: LangGraph, LangServe, LangSmith
% \item \textbf{Best Practices}: Production deployment, monitoring, optimization
% \item \textbf{Real-world Applications}: Complete project examples
% \end{itemize}

% \end{frame}