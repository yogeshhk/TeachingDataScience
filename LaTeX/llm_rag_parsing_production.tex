%%%%%%%%%%%%%%%%%%%%%%%%%%%%%%%%%%%%%%%%%%%%%%%%%%%%%%%%%%%%%%%%%%%%%%%%%%%%%%%%%%
\begin{frame}[fragile]\frametitle{}
\begin{center}
{\Large Production Considerations}

\end{center}
\end{frame}

%%%%%%%%%%%%%%%%%%%%%%%%%%%%%%%%%%%%%%%%%%%%%%%%%%%%%%%%%%%
\begin{frame}[fragile]\frametitle{Modern Parsing Challenges (2024-2025)}
\begin{columns}
    \begin{column}[T]{0.5\linewidth}
      \begin{itemize}
        \item \textbf{Cloud-Native Documents:}
        \begin{itemize}
            \item Google Docs, Notion, Confluence formats
            \item Dynamic content and embedded widgets
            \item Real-time collaborative editing artifacts
            \item Version control and change tracking
        \end{itemize}
        \item \textbf{API-Based Extraction:}
        \begin{itemize}
            \item OAuth authentication requirements
            \item Rate limiting and pagination handling
            \item Incremental sync challenges
        \end{itemize}
      \end{itemize}
    \end{column}
    \begin{column}[T]{0.5\linewidth}
      \begin{itemize}
        \item \textbf{Real-Time Streaming:}
        \begin{itemize}
            \item Processing documents as they're created
            \item Handling partial/incomplete content
            \item Low-latency requirements for live systems
        \end{itemize}
        \item \textbf{Format Complexity:}
        \begin{itemize}
            \item Rich media embeds (videos, interactive charts)
            \item Cross-document references and links
            \item Complex nested structures
            \item Export format inconsistencies
        \end{itemize}
      \end{itemize}
    \end{column}
\end{columns}
\end{frame}

%%%%%%%%%%%%%%%%%%%%%%%%%%%%%%%%%%%%%%%%%%%%%%%%%%%%%%%%%%%
\begin{frame}[fragile]\frametitle{Handling Collaborative Document Formats}
      \begin{itemize}
        \item \textbf{Notion Documents:}
        \begin{itemize}
            \item Block-based structure requires specialized parsing
            \item Databases and relations need graph representation
            \item Toggle lists and callouts often lost in conversion
            \item Solution: Use official Notion API + custom post-processing
        \end{itemize}
        \item \textbf{Confluence Pages:}
        \begin{itemize}
            \item Macro expansions and dynamic content
            \item Nested page hierarchies and attachments
            \item Storage format vs. view format discrepancies
            \item Solution: REST API extraction + HTML parsing hybrid
        \end{itemize}
        \item \textbf{Google Workspace:}
        \begin{itemize}
            \item Suggested edits and comments metadata
            \item Real-time sync state management
            \item Permission-based content visibility
            \item Solution: Google Drive API with export format selection
        \end{itemize}
      \end{itemize}
\end{frame}

%%%%%%%%%%%%%%%%%%%%%%%%%%%%%%%%%%%%%%%%%%%%%%%%%%%%%%%%%%%
\begin{frame}[fragile]\frametitle{Multimodal Parsing: The Next Frontier}
      \begin{itemize}
        \item \textbf{Beyond Text: Understanding All Content Types}
        \begin{itemize}
            \item Text, tables, images, charts, diagrams, code blocks
            \item Each modality requires specialized extraction
            \item Traditional parsers lose 40-60\% of information
        \end{itemize}
        \item \textbf{Why Multimodal Matters for RAG:}
        \begin{itemize}
            \item Financial reports: Tables contain critical metrics
            \item Technical docs: Diagrams explain architecture
            \item Research papers: Figures show experimental results
            \item Code repositories: Mixed text and code context
        \end{itemize}
        \item \textbf{Key Technologies:}
        \begin{itemize}
            \item Vision-Language Models (VLMs): BLIP-2, LLaVA, GPT-4V
            \item Layout-aware parsers: Docling, LayoutLMv3
            \item Table extraction: Table Transformer models
            \item OCR + structure: Tesseract + layout analysis
        \end{itemize}
        \item \textit{Note: Detailed multimodal implementation covered in next section}
      \end{itemize}
\end{frame}

%%%%%%%%%%%%%%%%%%%%%%%%%%%%%%%%%%%%%%%%%%%%%%%%%%%%%%%%%%%
\begin{frame}[fragile]\frametitle{Parser Performance Benchmarks}
      \begin{itemize}
        \item \textbf{Benchmark Dataset:} 100 diverse documents (contracts, reports, papers)
        \item \textbf{Metrics:} Text accuracy, table preservation, layout fidelity
      \end{itemize}
      
\begin{table}[h]
\centering
\small
\begin{tabular}{|l|c|c|c|c|}
\hline
\textbf{Parser} & \textbf{Text Acc.} & \textbf{Table Acc.} & \textbf{Layout} & \textbf{Overall} \\
\hline
PyPDF & 85\% & 35\% & 40\% & 53\% \\
\hline
Tesseract OCR & 78\% & 62\% & 55\% & 65\% \\
\hline
Unstructured & 89\% & 71\% & 68\% & 76\% \\
\hline
LlamaParse & 92\% & 88\% & 85\% & 88\% \\
\hline
Docling & 94\% & 91\% & 89\% & 91\% \\
\hline
Azure Doc Intel & 96\% & 94\% & 92\% & 94\% \\
\hline
\end{tabular}
\end{table}

      \begin{itemize}
        \item \textbf{Key Insight:} 20-40 point difference between worst and best performers
        \item \textbf{Impact:} Parser choice often matters more than model selection
      \end{itemize}
\end{frame}

%%%%%%%%%%%%%%%%%%%%%%%%%%%%%%%%%%%%%%%%%%%%%%%%%%%%%%%%%%%
\begin{frame}[fragile]\frametitle{RAG Performance: Impact of Parser Quality}
      \begin{itemize}
        \item \textbf{Experiment Setup:}
        \begin{itemize}
            \item Same RAG pipeline (embedding model, retriever, LLM)
            \item Same 50 test documents (financial reports, technical docs)
            \item Same 200 evaluation questions with ground truth
            \item Only variable: document parser
        \end{itemize}
        \item \textbf{Results - Answer Accuracy:}
        \begin{itemize}
            \item PyPDF baseline: 58\% correct answers
            \item Tesseract OCR: 64\% (+6 points)
            \item Unstructured: 72\% (+14 points)
            \item LlamaParse: 78\% (+20 points)
            \item Docling: 81\% (+23 points)
        \end{itemize}
        \item \textbf{Key Finding:} Upgrading parser improved accuracy by 23 percentage points
        \item \textbf{Comparison:} Upgrading from GPT-3.5 to GPT-4: +12 points (with same parser)
        \item \textbf{Conclusion:} Parser quality = 2x impact of model upgrade
      \end{itemize}
\end{frame}

%%%%%%%%%%%%%%%%%%%%%%%%%%%%%%%%%%%%%%%%%%%%%%%%%%%%%%%%%%%
\begin{frame}[fragile]\frametitle{Performance Benchmarks: Processing Speed}
      \begin{itemize}
        \item \textbf{Test Setup:} Intel Xeon 8-core, 32GB RAM, NVIDIA RTX 3080
        \item \textbf{Document Types:} Research papers, business reports, technical manuals
      \end{itemize}

\begin{table}[h]
\centering
\small
\begin{tabular}{|l|c|c|c|c|}
\hline
\textbf{Doc Size} & \textbf{Pages} & \textbf{CPU Only} & \textbf{GPU} & \textbf{Throughput} \\
\hline
Small & 1-5 & 2-5s & 1-2s & 12-30 docs/min \\
\hline
Medium & 10-20 & 8-15s & 3-6s & 4-10 docs/min \\
\hline
Large & 50-100 & 45-90s & 15-30s & 1-2 docs/min \\
\hline
Very Large & 200+ & 3-6min & 1-2min & 0.3-0.5 docs/min \\
\hline
\end{tabular}
\end{table}

      \begin{itemize}
        \item \textbf{Factors Affecting Speed:}
        \begin{itemize}
            \item OCR requirement: +50-100\% processing time
            \item Complex tables: +20-30\% per page
            \item Image resolution: Higher DPI = slower processing
            \item Layout complexity: Multi-column layouts add 10-20\%
        \end{itemize}
      \end{itemize}
\end{frame}

%%%%%%%%%%%%%%%%%%%%%%%%%%%%%%%%%%%%%%%%%%%%%%%%%%%%%%%%%%%
\begin{frame}[fragile]\frametitle{Performance Benchmarks: Memory Requirements}
      \begin{itemize}
        \item \textbf{Base Memory Usage:}
        \begin{itemize}
            \item Python process: ~500 MB
            \item Model loading: ~1-2 GB (layout, table models)
            \item Per-document processing: 50-200 MB depending on size
        \end{itemize}
        \item \textbf{Memory Scaling by Document Size:}
      \end{itemize}

\begin{table}[h]
\centering
\small
\begin{tabular}{|l|c|c|c|}
\hline
\textbf{Document} & \textbf{Peak RAM} & \textbf{GPU VRAM} & \textbf{Disk Cache} \\
\hline
10-page PDF & 2.5 GB & 1 GB & 50 MB \\
\hline
50-page PDF & 4 GB & 2 GB & 200 MB \\
\hline
100-page PDF & 6 GB & 3 GB & 400 MB \\
\hline
500-page PDF & 12 GB & 6 GB & 2 GB \\
\hline
\end{tabular}
\end{table}

      \begin{itemize}
        \item \textbf{Memory Optimization Tips:}
        \begin{itemize}
            \item Process large documents in batches
            \item Clear cache between documents: \texttt{gc.collect()}
            \item Use document streaming for very large files
            \item Disable features not needed (OCR, table extraction)
        \end{itemize}
      \end{itemize}
\end{frame}

%%%%%%%%%%%%%%%%%%%%%%%%%%%%%%%%%%%%%%%%%%%%%%%%%%%%%%%%%%%
\begin{frame}[fragile]\frametitle{Comparison: Docling vs Other Parsers}

\begin{table}[h]
\centering
\tiny
\begin{tabular}{|l|c|c|c|c|c|c|}
\hline
\textbf{Feature} & \textbf{PyPDF} & \textbf{Tesseract} & \textbf{Unstructured} & \textbf{LlamaParse} & \textbf{Docling} & \textbf{Azure DI} \\
\hline
Text Extraction & $\checkmark$ & $\checkmark$ & $\checkmark$ & $\checkmark$ & $\checkmark$ & $\checkmark$ \\
\hline
Layout Analysis & $ \times $ & $ \times $ & $\checkmark$ & $\checkmark$ & $\checkmark$ & $\checkmark$ \\
\hline
Table Structure & $ \times $ & ~ & $\checkmark$ & $\checkmark$ & $\checkmark$ & $\checkmark$ \\
\hline
OCR Support & $ \times $ & $\checkmark$ & $\checkmark$ & $\checkmark$ & $\checkmark$ & $\checkmark$ \\
\hline
Reading Order & $ \times $ & $ \times $ & ~ & $\checkmark$ & $\checkmark$ & $\checkmark$ \\
\hline
Formulas/Math & $ \times $ & $ \times $ & $ \times $ & ~ & $\checkmark$ & ~ \\
\hline
Multi-format & PDF & Image & Many & PDF & Many & Many \\
\hline
Local Execution & $\checkmark$ & $\checkmark$ & $\checkmark$ & $ \times $ & $\checkmark$ & $ \times $ \\
\hline
Open Source & $\checkmark$ & $\checkmark$ & $\checkmark$ & $ \times $ & $\checkmark$ & $ \times $ \\
\hline
Confidence Scores & $ \times $ & ~ & $ \times $ & $ \times $ & $\checkmark$ & $\checkmark$ \\
\hline
Structured Output & $ \times $ & $ \times $ & $\checkmark$ & $\checkmark$ & $\checkmark$ & $\checkmark$ \\
\hline
Speed (10pg) & <1s & 5-10s & 3-5s & 8-12s & 2-4s & 4-8s \\
\hline
Cost/1K docs & \$0 & \$0 & \$0-50 & \$100-200 & \$0 & \$150-300 \\
\hline
\end{tabular}
\end{table}

      \begin{itemize}
        \item $\checkmark$ = Full support, ~ = Partial support, $ \times $ = No support
        \item \textbf{Docling's Unique Strengths:} Open source, local execution, confidence scores, unified document model
      \end{itemize}
\end{frame}

%%%%%%%%%%%%%%%%%%%%%%%%%%%%%%%%%%%%%%%%%%%%%%%%%%%%%%%%%%%
\begin{frame}[fragile]\frametitle{Docling Advantages: Key Differentiators}
      \begin{itemize}
        \item \textbf{1. Unified Document Representation}
        \begin{itemize}
            \item Single DoclingDocument format for all content types
            \item Consistent API across PDF, DOCX, PPTX, HTML
            \item Preserves hierarchical structure and metadata
        \end{itemize}
        \item \textbf{2. Advanced Layout Understanding}
        \begin{itemize}
            \item Deep learning models for layout analysis
            \item Accurate reading order detection
            \item Multi-column and complex layout support
        \end{itemize}
        \item \textbf{3. Built-in Quality Assessment}
        \begin{itemize}
            \item Confidence scores at page and document level
            \item Quality grades: POOR, FAIR, GOOD, EXCELLENT
            \item Enables automated quality control
        \end{itemize}
        \item \textbf{4. Production-Ready Features}
        \begin{itemize}
            \item Local execution for data privacy
            \item Plugin system for extensibility
            \item Integration with major RAG frameworks
            \item Comprehensive error handling
        \end{itemize}
      \end{itemize}
\end{frame}

%%%%%%%%%%%%%%%%%%%%%%%%%%%%%%%%%%%%%%%%%%%%%%%%%%%%%%%%%%%
\begin{frame}[fragile]\frametitle{Error Handling Patterns: Basic Error Handling}
      \begin{itemize}
        \item \textbf{Common Error Scenarios:}
        \begin{itemize}
            \item Corrupted or password-protected PDFs
            \item Unsupported file formats
            \item Out of memory errors for large documents
            \item Model loading failures
        \end{itemize}
      \end{itemize}

\begin{lstlisting}[language=Python, basicstyle=\tiny]
from docling.document_converter import DocumentConverter
from docling.exceptions import ConversionError
import logging

logging.basicConfig(level=logging.INFO)
logger = logging.getLogger(__name__)

\end{lstlisting}
\end{frame}


%%%%%%%%%%%%%%%%%%%%%%%%%%%%%%%%%%%%%%%%%%%%%%%%%%%%%%%%%%%
\begin{frame}[fragile]\frametitle{Error Handling Patterns: Basic Error Handling}

\begin{lstlisting}[language=Python, basicstyle=\tiny]

def safe_convert(source: str) -> Optional[DoclingDocument]:
    converter = DocumentConverter()
    
    try:
        result = converter.convert(source)
        
        # Check conversion quality
        if result.confidence.mean_grade < 0.5:
            logger.warning(f"Low quality conversion: {source}")
            
        return result.document
        
    except ConversionError as e:
        logger.error(f"Conversion failed for {source}: {e}")
        return None
    except MemoryError:
        logger.error(f"Out of memory processing: {source}")
        return None
    except Exception as e:
        logger.error(f"Unexpected error for {source}: {e}")
        return None
\end{lstlisting}
\end{frame}

%%%%%%%%%%%%%%%%%%%%%%%%%%%%%%%%%%%%%%%%%%%%%%%%%%%%%%%%%%%
\begin{frame}[fragile]\frametitle{Error Handling: Advanced Retry Logic}

\begin{lstlisting}[language=Python, basicstyle=\tiny]
from tenacity import retry, stop_after_attempt, wait_exponential
from docling.datamodel.pipeline_options import PdfPipelineOptions

class RobustDoclingConverter:
    def __init__(self):
        self.converter = DocumentConverter()
        
    @retry(
        stop=stop_after_attempt(3),
        wait=wait_exponential(multiplier=1, min=2, max=10)
    )
    def convert_with_retry(self, source: str):
        return self.converter.convert(source)
    
    def convert_with_fallback(self, source: str):
        # Try with full pipeline first
        try:
            return self.convert_with_retry(source)
        except Exception as e:
            logger.warning(f"Full pipeline failed, trying simplified: {e}")
            
            # Fallback: disable expensive features
            options = PdfPipelineOptions()
            options.do_ocr = False
            options.do_table_structure = False
            
            converter = DocumentConverter(
                format_options={InputFormat.PDF: PdfFormatOption(options)}
            )
            return converter.convert(source)
\end{lstlisting}
\end{frame}


%%%%%%%%%%%%%%%%%%%%%%%%%%%%%%%%%%%%%%%%%%%%%%%%%%%%%%%%%%%
\begin{frame}[fragile]\frametitle{Error Handling: Quality-Based Filtering}

\begin{lstlisting}[language=Python, basicstyle=\tiny]
from typing import List, Tuple
from docling.datamodel.document import DoclingDocument

class QualityFilteredConverter:
    def __init__(self, min_confidence: float = 0.7):
        self.converter = DocumentConverter()
        self.min_confidence = min_confidence
        
    def convert_and_filter(self, sources: List[str]) -> Tuple[List, List]:
        successful = []
        failed = []
        
        for source in sources:
            try:
                result = self.converter.convert(source)
                
                # Check confidence score
                if result.confidence.mean_grade >= self.min_confidence:
                    successful.append({
                        'source': source, 'document': result.document,
                        'confidence': result.confidence.mean_grade })
                else:
                    failed.append({
                        'source': source,'reason': 'low_confidence',
                        'confidence': result.confidence.mean_grade })
                    
            except Exception as e:
                failed.append({'source': source,
                    'reason': str(e), 'confidence': 0.0 })
        
        return successful, failed
\end{lstlisting}
\end{frame}

%%%%%%%%%%%%%%%%%%%%%%%%%%%%%%%%%%%%%%%%%%%%%%%%%%%%%%%%%%%
\begin{frame}[fragile]\frametitle{Batch Processing: Basic Batch Conversion}

        \begin{itemize}
            \item Initial document corpus ingestion
            \item Periodic document updates
            \item Large-scale document digitization
        \end{itemize}

\begin{lstlisting}[language=Python, basicstyle=\tiny]
class BatchDoclingProcessor:
    def __init__(self):
        self.converter = DocumentConverter()
        
    def process_directory(self, input_dir: str, output_dir: str):
        input_path = Path(input_dir)
        output_path = Path(output_dir)
        output_path.mkdir(exist_ok=True)

        files = list(input_path.glob("**/*.pdf"))
        files.extend(input_path.glob("**/*.docx"))
        files.extend(input_path.glob("**/*.pptx"))
        
        results = []
        for file in tqdm(files, desc="Processing documents"):
            try:
                result = self.converter.convert(str(file))
                output_file = output_path / f"{file.stem}.md"
                output_file.write_text(result.document.export_to_markdown())
                results.append({'file': file, 'status': 'success'})
            except Exception as e:
                results.append({'file': file, 'status': 'failed', 'error': str(e)})
        
        return results
\end{lstlisting}
\end{frame}

%%%%%%%%%%%%%%%%%%%%%%%%%%%%%%%%%%%%%%%%%%%%%%%%%%%%%%%%%%%
\begin{frame}[fragile]\frametitle{Batch Processing: Parallel Processing}

\begin{lstlisting}[language=Python, basicstyle=\tiny]
from concurrent.futures import ProcessPoolExecutor, as_completed
from multiprocessing import cpu_count
import os

class ParallelBatchProcessor:
    def __init__(self, max_workers: int = None):
        self.max_workers = max_workers or max(1, cpu_count() - 1)
        
    def process_single(self, file_path: str) -> dict:
        """Process single file - called in separate process"""
        converter = DocumentConverter()
        try:
            result = converter.convert(file_path)
            return {
                'file': file_path,
                'status': 'success',
                'pages': len(result.document.pages),
                'confidence': result.confidence.mean_grade
            }
        except Exception as e:
            return {'file': file_path, 'status': 'failed', 'error': str(e)}
    
    def process_batch(self, file_paths: List[str]) -> List[dict]:
        results = []
        with ProcessPoolExecutor(max_workers=self.max_workers) as executor:
            futures = {executor.submit(self.process_single, fp): fp 
                      for fp in file_paths}
            
            for future in tqdm(as_completed(futures), total=len(file_paths)):
                results.append(future.result())
        
        return results
\end{lstlisting}
\end{frame}

%%%%%%%%%%%%%%%%%%%%%%%%%%%%%%%%%%%%%%%%%%%%%%%%%%%%%%%%%%%
\begin{frame}[fragile]\frametitle{Batch Processing: Best Practices}
      \begin{itemize}
        \item \textbf{1. Resource Management}
        \begin{itemize}
            \item Limit concurrent processes to avoid memory exhaustion
            \item Use process pool (not thread pool) to avoid GIL
            \item Monitor memory usage: \texttt{psutil.virtual\_memory()}
            \item Implement back-pressure mechanisms for large batches
        \end{itemize}
        \item \textbf{2. Progress Tracking and Logging}
        \begin{itemize}
            \item Use tqdm for progress bars
            \item Log all failures with traceback for debugging
            \item Save intermediate results periodically
            \item Generate summary reports (success rate, avg time, errors)
        \end{itemize}
        \item \textbf{3. Fault Tolerance}
        \begin{itemize}
            \item Implement checkpointing to resume failed batches
            \item Skip already processed files (check output directory)
            \item Separate retry queue for failed documents
            \item Use exponential backoff for transient errors
        \end{itemize}
        \item \textbf{4. Output Management}
        \begin{itemize}
            \item Use consistent naming conventions
            \item Preserve directory structure in output
            \item Store metadata alongside converted documents
            \item Implement cleanup for failed conversions
        \end{itemize}
      \end{itemize}
\end{frame}

%%%%%%%%%%%%%%%%%%%%%%%%%%%%%%%%%%%%%%%%%%%%%%%%%%%%%%%%%%%
\begin{frame}[fragile]\frametitle{Batch Processing: Production Pipeline Example}

\begin{lstlisting}[language=Python, basicstyle=\tiny]
class ProductionBatchPipeline:
    def __init__(self, config: dict):
        self.converter = DocumentConverter()
        self.checkpoint_file = config.get('checkpoint', 'progress.json')
        self.max_retries = config.get('max_retries', 3)
        
    def load_checkpoint(self) -> set:
        if Path(self.checkpoint_file).exists():
            with open(self.checkpoint_file) as f:
                return set(json.load(f))
        return set()
    
    def save_checkpoint(self, processed: set):
        with open(self.checkpoint_file, 'w') as f:
            json.dump(list(processed), f)
    
    def process_with_monitoring(self, files: List[str]):
        processed = self.load_checkpoint()
        remaining = [f for f in files if f not in processed]
        
        stats = {'success': 0, 'failed': 0, 'skipped': len(processed)}
        
        for file in tqdm(remaining):
            result = self._process_with_retry(file)
            if result['status'] == 'success':
                stats['success'] += 1
                processed.add(file)
                self.save_checkpoint(processed)
            else:
                stats['failed'] += 1
                
        return stats
\end{lstlisting}
\end{frame}

%%%%%%%%%%%%%%%%%%%%%%%%%%%%%%%%%%%%%%%%%%%%%%%%%%%%%%%%%%%
\begin{frame}[fragile]\frametitle{Performance Optimization Tips}
      \begin{itemize}
        \item \textbf{1. Model Loading Optimization}
        \begin{itemize}
            \item Load models once, reuse DocumentConverter instance
            \item Pre-load models at startup: reduces first-document latency
            \item Use model caching for repeated conversions
        \end{itemize}
        \item \textbf{2. Memory Optimization}
        \begin{itemize}
            \item Process documents in streaming mode for large files
            \item Clear document cache after processing: \texttt{del result}
            \item Use garbage collection: \texttt{gc.collect()} between batches
            \item Limit image resolution for OCR when high quality not needed
        \end{itemize}
        \item \textbf{3. Speed Optimization}
        \begin{itemize}
            \item Disable unnecessary features (OCR, table extraction)
            \item Use GPU acceleration when available
            \item Batch similar documents together for better caching
            \item Pre-filter documents by type before processing
        \end{itemize}
        \item \textbf{4. Quality vs Performance Tradeoff}
        \begin{itemize}
            \item Use confidence scores to determine processing depth
            \item Implement fast-path for high-quality digital PDFs
            \item Reserve expensive processing for low-confidence documents
        \end{itemize}
      \end{itemize}
\end{frame}

%%%%%%%%%%%%%%%%%%%%%%%%%%%%%%%%%%%%%%%%%%%%%%%%%%%%%%%%%%%
\begin{frame}[fragile]\frametitle{Monitoring and Observability}

\begin{lstlisting}[language=Python, basicstyle=\tiny]
from dataclasses import dataclass
from datetime import datetime
import json

@dataclass
class ConversionMetrics:
    file_path: str
    start_time: datetime
    end_time: datetime
    duration_seconds: float
    pages: int
    confidence_score: float
    memory_peak_mb: float
    status: str
    error: str = None

class MonitoredConverter:
    def __init__(self):
        self.converter = DocumentConverter()
        self.metrics = []
        
    def convert_with_metrics(self, source: str) -> ConversionMetrics:
        import psutil
        import tracemalloc
        
        tracemalloc.start()
        process = psutil.Process()
        start_time = datetime.now()
        
		:
        )
\end{lstlisting}
\end{frame}

%%%%%%%%%%%%%%%%%%%%%%%%%%%%%%%%%%%%%%%%%%%%%%%%%%%%%%%%%%%
\begin{frame}[fragile]\frametitle{Monitoring and Observability}

\begin{lstlisting}[language=Python, basicstyle=\tiny]

class MonitoredConverter:
        
    def convert_with_metrics(self, source: str) -> ConversionMetrics:
        :
		
        try:
            result = self.converter.convert(source)
            status = 'success'
            confidence = result.confidence.mean_grade
            pages = len(result.document.pages)
            error = None
        except Exception as e:
            status = 'failed'
            confidence = 0.0
            pages = 0
            error = str(e)
        
        current, peak = tracemalloc.get_traced_memory()
        tracemalloc.stop()
        
        end_time = datetime.now()
        
        return ConversionMetrics(
            file_path=source, start_time=start_time, end_time=end_time,
            duration_seconds=(end_time - start_time).total_seconds(),
            pages=pages, confidence_score=confidence,
            memory_peak_mb=peak / 1024 / 1024, status=status, error=error
        )
\end{lstlisting}
\end{frame}


%%%%%%%%%%%%%%%%%%%%%%%%%%%%%%%%%%%%%%%%%%%%%%%%%%%%%%%%%%%
\begin{frame}[fragile]\frametitle{Cost Comparison: Parsing Solutions}
\begin{table}[h]
\centering
\small
\begin{tabular}{|l|c|c|c|c|}
\hline
\textbf{Solution} & \textbf{Type} & \textbf{Cost/1K docs} & \textbf{Setup} & \textbf{Latency} \\
\hline
PyPDF & Open Source & \$0 & Easy & 0.5s \\
\hline
Tesseract & Open Source & \$0* & Medium & 3-5s \\
\hline
Unstructured & Open/Hosted & \$0-\$50 & Easy & 2-4s \\
\hline
LlamaParse & API & \$100-\$200 & Easy & 5-8s \\
\hline
Docling & Open Source & \$0* & Medium & 2-3s \\
\hline
Azure Doc Intel & Cloud API & \$150-\$300 & Easy & 3-6s \\
\hline
AWS Textract & Cloud API & \$150-\$500 & Easy & 4-7s \\
\hline
\end{tabular}
\end{table}

      \begin{itemize}
        \item *Compute costs only (CPU/GPU hours)
        \item \textbf{Open source:} Higher setup, lower marginal cost, full control
        \item \textbf{API services:} Quick start, predictable pricing, limited customization
        \item \textbf{Recommendation:} Start with open source, switch to API at scale
      \end{itemize}
\end{frame}

%%%%%%%%%%%%%%%%%%%%%%%%%%%%%%%%%%%%%%%%%%%%%%%%%%%%%%%%%%%
\begin{frame}[fragile]\frametitle{Cost-Performance Tradeoff Analysis}
      \begin{itemize}
        \item \textbf{Total Cost of Ownership (TCO) for 1M documents/year:}
      \end{itemize}

\begin{table}[h]
\centering
\small
\begin{tabular}{|l|c|c|c|c|}
\hline
\textbf{Parser} & \textbf{API Cost} & \textbf{Compute} & \textbf{Dev/Ops} & \textbf{Total} \\
\hline
PyPDF & \$0 & \$500 & \$2,000 & \$2,500 \\
\hline
Docling & \$0 & \$2,000 & \$5,000 & \$7,000 \\
\hline
LlamaParse & \$100,000 & \$0 & \$1,000 & \$101,000 \\
\hline
Azure Doc Intel & \$200,000 & \$0 & \$1,000 & \$201,000 \\
\hline
\end{tabular}
\end{table}

      \begin{itemize}
        \item \textbf{Decision Matrix:}
        \begin{itemize}
            \item less than 100K docs/year: Any parser works, optimize for dev time
            \item 100K-1M docs/year: Open source (Docling) offers best ROI
            \item 1M-10M docs/year: Hybrid approach (open source + API fallback)
            \item More than 10M docs/year: Custom solution or negotiated enterprise API pricing
        \end{itemize}
        \item \textbf{Quality Factor:} Higher accuracy reduces downstream costs (fewer errors, less human review)
      \end{itemize}
\end{frame}

%%%%%%%%%%%%%%%%%%%%%%%%%%%%%%%%%%%%%%%%%%%%%%%%%%%%%%%%%%%
\begin{frame}[fragile]\frametitle{Hidden Costs in Production Parsing}
      \begin{itemize}
        \item \textbf{Infrastructure Costs:}
        \begin{itemize}
            \item GPU requirements for advanced parsers (Docling, LayoutLM)
            \item Storage for raw + processed documents
            \item Network bandwidth for cloud API calls
            \item Caching infrastructure to reduce redundant processing
        \end{itemize}
        \item \textbf{Operational Costs:}
        \begin{itemize}
            \item Monitoring and error tracking systems
            \item Human review for failed/low-confidence parses (10-20\% of docs)
            \item Retry logic and fallback parser costs
            \item Version upgrades and model retraining
        \end{itemize}
        \item \textbf{Quality Costs:}
        \begin{itemize}
            \item False negatives in retrieval due to poor parsing
            \item User trust loss from incorrect RAG responses
            \item Manual data cleaning and correction
        \end{itemize}
        \item \textbf{Rule of Thumb:} Budget 2-3x the direct parsing cost for full production system
      \end{itemize}
\end{frame}

%%%%%%%%%%%%%%%%%%%%%%%%%%%%%%%%%%%%%%%%%%%%%%%%%%%%%%%%%%%
\begin{frame}[fragile]\frametitle{Choosing a Parser: Decision Framework}
      \begin{itemize}
        \item \textbf{Step 1: Assess Your Document Types}
        \begin{itemize}
            \item Simple text PDFs → PyPDF, pypdf
            \item Scanned documents → Tesseract, Docling (with OCR)
            \item Complex layouts/tables → Docling, LlamaParse, Azure Doc Intel
            \item Cloud-native → API-based solutions (Notion API, etc.)
        \end{itemize}
        \item \textbf{Step 2: Define Quality Requirements}
        \begin{itemize}
            \item Acceptable error rate? (Text: <5\%, Tables: <10\%)
            \item Manual review budget available?
            \item Impact of parsing errors on downstream tasks
        \end{itemize}
        \item \textbf{Step 3: Calculate Total Cost}
        \begin{itemize}
            \item API pricing × volume
            \item Compute costs (GPU hours × rate)
            \item Development + operational overhead
        \end{itemize}
        \item \textbf{Step 4: Prototype and Benchmark}
        \begin{itemize}
            \item Test 2-3 parsers on representative sample (50-100 docs)
            \item Measure accuracy, speed, cost
            \item Run end-to-end RAG evaluation
        \end{itemize}
      \end{itemize}
\end{frame}