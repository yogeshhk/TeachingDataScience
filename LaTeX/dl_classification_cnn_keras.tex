%%%%%%%%%%%%%%%%%%%%%%%%%%%%%%%%%%%%%%%%%%%%%%%%%%%%%%%%%%%%%%%%%%%%%%%%%%%%%%%%%%
\begin{frame}[fragile]\frametitle{}
\begin{center}
{\Large Image Classification with Keras}

{\tiny (Ref: Handwritten Digit Recognition using Convolutional Neural Networks in Python with Keras - Jason Brownlee https://machinelearningmastery.com/handwritten-digit-recognition-using-convolutional-neural-networks-python-keras/)}
\end{center}
\end{frame}

%%%%%%%%%%%%%%%%%%%%%%%%%%%%%%%%%%%%%%%%%%%%%%%%%%%%%%%%%%%%%%%%%%%%%%%%%%%%%%%%%%
\begin{frame}[fragile]\frametitle{The Problem: Digit Recognition Task}
\begin{itemize}
\item The MNIST problem is a dataset developed by Yann LeCun, Corinna Cortes and Christopher Burges
\item Images of digits were taken from a variety of scanned documents, normalized in size and centered.
\item Each image is a 28 by 28 pixel square (784 pixels total). 
\item Generally, 60,000 images are used to train a model and a separate set of 10,000 images are used to test it.
\item As such there are 10 digits (0 to 9) or 10 classes to predict. 
\item Results are reported using prediction error
\end{itemize}
\end{frame}
