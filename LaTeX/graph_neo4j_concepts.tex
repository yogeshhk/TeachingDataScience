%%%%%%%%%%%%%%%%%%%%%%%%%%%%%%%%%%%%%%%%%%%%%%%%%%%%%%%%%%%%%%%%%%%%%%%%%%%%%%%%%%
\begin{frame}[fragile]\frametitle{}
\begin{center}
{\Large Neo4j}
\end{center}
\end{frame}


%%%%%%%%%%%%%%%%%%%%%%%%%%%%%%%%%%%%%%%%%%%%%%%%%%%%%%%%%%%%%%%%%%%%%%%%%%%%%%%%%%
\begin{frame}\frametitle{Introduction}

Neo4j: Network Exploration and Optimization for Java

\begin{itemize}
\item Open Source
\item Implemented in Java and Scala
\item Cypher : mature and rivals SQL
\item ACID (Atomic, Consistent, Isolated, Durable) compliant
\item Embedded Server
\item REST API
\end{itemize}

{\tiny (Ref: CIS 6930 - Advanced Databases - Neo4j )}


\begin{center}
\includegraphics[width=0.6\linewidth,keepaspectratio]{neo4j32}
\end{center}	

\end{frame}


%%%%%%%%%%%%%%%%%%%%%%%%%%%%%%%%%%%%%%%%%%%%%%%%%%%%%%%%%%%
\begin{frame}[fragile]\frametitle{Neo4j}

\begin{center}
\includegraphics[width=\linewidth,keepaspectratio]{neo4j14}
\end{center}	  

{\tiny (Ref: Introduction to Neo4j and Graph Databases
 - M David Allen)}

\end{frame}

%%%%%%%%%%%%%%%%%%%%%%%%%%%%%%%%%%%%%%%%%%%%%%%%%%%%%%%%%%%%%%%%%%%%%%%%%%%%%%%%%%
\begin{frame}\frametitle{Neo4j features}

\begin{itemize}
\item Capacity: Nodes/Relationships/Labels (all in Billions)
\item High data integrity
\item Native graph processing
\item High scalability
\item Data browser
\end{itemize}

{\tiny (Ref: CIS 6930 - Advanced Databases - Neo4j )}
\end{frame}

%%%%%%%%%%%%%%%%%%%%%%%%%%%%%%%%%%%%%%%%%%%%%%%%%%%%%%%%%%%%%%%%%%%%%%%%%%%%%%%%%%
\begin{frame}\frametitle{Index free adjacency }
With index free adjaceny, when a node or relationship is written to the database, it is stored in the database as connected and any subsequent access to the data is done using pointer navigation which is very fast. Since Neo4j is a native graph database, it supports very large graphs where connected data can be traversed in constant time without the need for an index.

\begin{center}
\includegraphics[width=0.6\linewidth,keepaspectratio]{neo4j79}
\end{center}	

{\tiny (Ref: Learning Neo4j - Wabri Github)}
\end{frame}

%%%%%%%%%%%%%%%%%%%%%%%%%%%%%%%%%%%%%%%%%%%%%%%%%%%%%%%%%%%%%%%%%%%%%%%%%%%%%%%%%%
\begin{frame}\frametitle{ACID }
Transactionality is very important for robust applications that require an atomicity, consistency, isolation, and durability guarantees for their data. If a relationship between nodes is created, not only is the relationship created, but the nodes are updated as connected. All of these updates to the database must all succeed or fail.


\begin{center}
\includegraphics[width=0.8\linewidth,keepaspectratio]{neo4j80}
\end{center}	

{\tiny (Ref: Learning Neo4j - Wabri Github)}
\end{frame}

%%%%%%%%%%%%%%%%%%%%%%%%%%%%%%%%%%%%%%%%%%%%%%%%%%%%%%%%%%%%%%%%%%%%%%%%%%%%%%%%%%
\begin{frame}\frametitle{Clusters }
Neo4j supports clusters that provide high availablity, scalability for read access to the data and failover which is important to many enterprises.

\begin{center}
\includegraphics[width=0.6\linewidth,keepaspectratio]{neo4j81}
\end{center}	

{\tiny (Ref: Learning Neo4j - Wabri Github)}
\end{frame}

%%%%%%%%%%%%%%%%%%%%%%%%%%%%%%%%%%%%%%%%%%%%%%%%%%%%%%%%%%%%%%%%%%%%%%%%%%%%%%%%%%
\begin{frame}\frametitle{Misc }

\begin{itemize}
\item The Neo4j graph engine is used to interpret Cypher statements and also executes kernel-level code to store and retrive data, whether it is on disk, or cached in memory.
\item Neo4j supports Java, JavaScript, Python, C\#, and Go drivers that use Neo4j's bolt protocol for binary access to the database layer. Bolt is an efficiant binary protocol that compresses data sent over the wire as well encrypting the data. It's possible to create a java application that uses the bolt driver to access the Neo4j database and the application may use other packages that allow data integration between Neo4j and other data stores or uses as common framework such as spring.
\end{itemize}


{\tiny (Ref: Learning Neo4j - Wabri Github)}
\end{frame}

%%%%%%%%%%%%%%%%%%%%%%%%%%%%%%%%%%%%%%%%%%%%%%%%%%%%%%%%%%%%%%%%%%%%%%%%%%%%%%%%%%
\begin{frame}\frametitle{Tools }

\begin{itemize}
\item Neo4j browser is an application that uses the JavaScript Bolt driver to access the graph engine of the Neo4j database server.
\item Bloom enables you to visualize a graph without knowing much about Cypher (youtube video).
\item  ETL used to importing and exporting data between flat files and a neo4j Database.
\end{itemize}

\begin{center}
\includegraphics[width=0.8\linewidth,keepaspectratio]{neo4j82}
\end{center}	

{\tiny (Ref: Learning Neo4j - Wabri Github)}
\end{frame}


%%%%%%%%%%%%%%%%%%%%%%%%%%%%%%%%%%%%%%%%%%%%%%%%%%%%%%%%%%%%%%%%%%%%%%%%%%%%%%%%%%
\begin{frame}\frametitle{Usage }

\begin{itemize}
\item Desktop:  Includes the Neo4j Database server which includes the graph engine and kernel so that Cypher statements can be executed to access a database on your system. It includes an application called Neo4j Browser. Neo4j Browser enables you to access a Neo4j database using Cypher. 
\item Browser Sandbox: Is a temporary, cloud-based instance of a Neo4j Server with its associated graph that you can access from any Web browser. The database in a Sandbox may be blank or it may be pre-populated. It is started automatically for you when you create the Sandbox.
\end{itemize}


{\tiny (Ref: Learning Neo4j - Wabri Github)}
\end{frame}


%%%%%%%%%%%%%%%%%%%%%%%%%%%%%%%%%%%%%%%%%%%%%%%%%%%%%%%%%%%%%%%%%%%%%%%%%%%%%%%%%%
\begin{frame}\frametitle{Good For}

\begin{itemize}
\item Highly connected data (social networks)
\item Recommendations (e-commerce)
\item Path Finding (how do I know you?)
\item A* (Least Cost path)
\item  Data First Schema (bottom-up, but you still need to design)
\end{itemize}

{\tiny (Ref: CIntroduction to Graph Databases - Max De Marzi )}
\end{frame}


%%%%%%%%%%%%%%%%%%%%%%%%%%%%%%%%%%%%%%%%%%%%%%%%%%%%%%%%%%%%%%%%%%%%%%%%%%%%%%%%%%
\begin{frame}\frametitle{Property Graph}

\begin{itemize}
\item Labels are node types. There can be multiple labels to a node, e.g. Person, Female, etc.
\item Properties are key-value attributes/fields of node or relationship, e.g. firstname, born.

\end{itemize}
\begin{center}
\includegraphics[width=\linewidth,keepaspectratio]{neo4j60}
\end{center}	

{\tiny (Ref: Introduction to Neo4j - a hands-on crash course - neo4j)}
\end{frame}


%%%%%%%%%%%%%%%%%%%%%%%%%%%%%%%%%%%%%%%%%%%%%%%%%%%%%%%%%%%%%%%%%%%%%%%%%%%%%%%%%%
\begin{frame}\frametitle{Property Graph}

\begin{center}
\includegraphics[width=\linewidth,keepaspectratio]{neo4j46}
\end{center}	

{\tiny (Ref: Introduction to Graph Databases - Max De Marzi )}
\end{frame}

%%%%%%%%%%%%%%%%%%%%%%%%%%%%%%%%%%%%%%%%%%%%%%%%%%%%%%%%%%%%%%%%%%%%%%%%%%%%%%%%%%
\begin{frame}\frametitle{Property Graph}

\begin{center}
\includegraphics[width=\linewidth,keepaspectratio]{neo4j47}
\end{center}	

{\tiny (Ref: Introduction to Graph Databases - Max De Marzi )}
\end{frame}

%%%%%%%%%%%%%%%%%%%%%%%%%%%%%%%%%%%%%%%%%%%%%%%%%%%%%%%%%%%%%%%%%%%%%%%%%%%%%%%%%%
\begin{frame}\frametitle{Native Graph}

\begin{itemize}
\item Index free adjacency.
\item Each node stores incoming and outgoing relationships.
\item So traversal is from node to node via edges, and not by iterating through linear index.
\end{itemize}

\begin{center}
\includegraphics[width=\linewidth,keepaspectratio]{neo4j61}
\end{center}	

{\tiny (Ref: Introduction to Neo4j - a hands-on crash course - neo4j)}
\end{frame}


%%%%%%%%%%%%%%%%%%%%%%%%%%%%%%%%%%%%%%%%%%%%%%%%%%%%%%%%%%%%%%%%%%%%%%%%%%%%%%%%%%
\begin{frame}\frametitle{Cypher}

Pattern Matching Query Language (like SQL for graphs)

\begin{center}
\includegraphics[width=\linewidth,keepaspectratio]{neo4j48}
\end{center}	

{\tiny (Ref: Introduction to Graph Databases - Max De Marzi )}
\end{frame}

%%%%%%%%%%%%%%%%%%%%%%%%%%%%%%%%%%%%%%%%%%%%%%%%%%%%%%%%%%%%%%%%%%%%%%%%%%%%%%%%%%
\begin{frame}\frametitle{Gremlin}

A Graph Scripting DSL (groovy-based)

\begin{center}
\includegraphics[width=0.8\linewidth,keepaspectratio]{neo4j49}
\end{center}	

{\tiny (Ref: Introduction to Graph Databases - Max De Marzi )}
\end{frame}

%%%%%%%%%%%%%%%%%%%%%%%%%%%%%%%%%%%%%%%%%%%%%%%%%%%%%%%%%%%%%%%%%%%%%%%%%%%%%%%%%%
\begin{frame}\frametitle{If you’ve ever}

\begin{itemize}
\item Joined more than 7 tables together
\item  Modeled a graph in a table
\item  Written a recursive CTE
\item Tried to write some crazy stored procedure with multiple recursive self and inner joins
\end{itemize}

{\bf You should use Neo4j}

{\tiny (Ref: Introduction to Graph Databases - Max De Marzi )}
\end{frame}


%%%%%%%%%%%%%%%%%%%%%%%%%%%%%%%%%%%%%%%%%%%%%%%%%%%%%%%%%%%
\begin{frame}[fragile]\frametitle{Windows Installation}

\begin{itemize}
\item Have Open JDK 11 ready, if not, go to https://learn.microsoft.com/en-us/java/openjdk/download (178 MB)
\item https://neo4j.com/download-center/\#community (4.4.11 129 MB)
\item Place the extracted files in a permanent home on your server, for example \lstinline|D:\neo4j\|. The top level directory is referred to as NEO4J\_HOME.
\item To run Neo4j as a console application, use: \lstinline|<NEO4J_HOME>\bin\neo4j console|

% \begin{lstlisting}
% C:\neo4j\bin>neo4j.bat console
% Directories in use:
% home:         C:\neo4j
% :
% Starting Neo4j.
% \end{lstlisting}

\item Visit http://localhost:7474 in your web browser.
\item Default login is username 'neo4j' and password 'neo4j' Change password. Got conncted to `neo4j://127.0.0.1:7687`
\end{itemize}

(Note: Btw, free, no-download sandbox option is available at sandbox.neo4j.com)
\end{frame}

%%%%%%%%%%%%%%%%%%%%%%%%%%%%%%%%%%%%%%%%%%%%%%%%%%%%%%%%%%%%%%%%%%%%%%%%%%%%%%%%%%
\begin{frame}\frametitle{Neo4j Driver API}


\begin{itemize}
\item  Bolt protocol
\item   Currently supports .NET, Java, JavaScript and Python
\item   Uniformity across languages
\end{itemize}


{\tiny (Ref: CIS 6930 - Advanced Databases - Neo4j )}
\end{frame}

%%%%%%%%%%%%%%%%%%%%%%%%%%%%%%%%%%%%%%%%%%%%%%%%%%%%%%%%%%%%%%%%%%%%%%%%%%%%%%%%%%
\begin{frame}\frametitle{Neo4j Browser}


\begin{itemize}
\item   Developer focused
\item    Export results
\item   Visualization
\end{itemize}

\begin{center}
\includegraphics[width=\linewidth,keepaspectratio]{neo4j40}
\end{center}	 

{\tiny (Ref: CIS 6930 - Advanced Databases - Neo4j )}
\end{frame}



%%%%%%%%%%%%%%%%%%%%%%%%%%%%%%%%%%%%%%%%%%%%%%%%%%%%%%%%%%%
\begin{frame}[fragile]\frametitle{Console}

\begin{center}
\includegraphics[width=\linewidth,keepaspectratio]{neo4j1}
\end{center}	  


\end{frame}

%%%%%%%%%%%%%%%%%%%%%%%%%%%%%%%%%%%%%%%%%%%%%%%%%%%%%%%%%%%
\begin{frame}[fragile]\frametitle{Interaction}

Type commands in top command window

\begin{itemize}
\item \lstinline|MATCH(n) RETURN n| Return nodes (type none). As there is nothing, nothing gets returned. So create something.
\item \lstinline|CREATE(n)| will create one empty node.
\item \lstinline|MATCH(n) RETURN n| will now return 1 node and show it. $n$ is the reference(s) or the object handle(s), which is returned.
\end{itemize}

\begin{center}
\includegraphics[width=0.8\linewidth,keepaspectratio]{neo4j2}
\end{center}

\end{frame}


%%%%%%%%%%%%%%%%%%%%%%%%%%%%%%%%%%%%%%%%%%%%%%%%%%%%%%%%%%%
\begin{frame}[fragile]\frametitle{Examples}

\begin{itemize}
\item \lstinline|CREATE(n:PERSON)| create a node (type PERSON). Type is actually the label of the node.
\item \lstinline|MATCH(n) DELETE(n)| will delete all the nodes. n, the object handled, returned by MATCH, is getting deleted in the delete function where n is argument. 
\item \lstinline|CREATE(n:PERSON{name:'chris', favoritecolor:'blue'})| create a node (type PERSON) along with some properties. SImilariy, can create different nodes of different type.
\item \lstinline|MATCH(n:PERSON) RETURN n|to selectively return only PERSONs.
\item \lstinline|MATCH(n) RETURN n LIMIT 4| to return only 4 nodes
\item Two create relationship find 2 nodes using \lstinline|MATCH| and \lstinline|WHERE| to restrict, then create relationship 'studied\_at'

\begin{lstlisting}
MATCH (s:School), (p:Person)
WHERE s.name = 'LSU' and p.name = 'jenny'
CREATE (p)-[stu:STUDIED_AT]->(s)
\end{lstlisting}


\end{itemize}

\end{frame}


%%%%%%%%%%%%%%%%%%%%%%%%%%%%%%%%%%%%%%%%%%%%%%%%%%%%%%%%%%%
\begin{frame}[fragile]\frametitle{Examples}


\begin{center}
\includegraphics[width=0.9\linewidth,keepaspectratio]{neo4j3}
\end{center}	

{\tiny (Ref: An introduction to neo4j (graph database tutorial for beginners) - Chris Hay)}

\end{frame}

%%%%%%%%%%%%%%%%%%%%%%%%%%%%%%%%%%%%%%%%%%%%%%%%%%%%%%%%%%%
\begin{frame}[fragile]\frametitle{Comparison}
Find all reports and how many people they manage upto 3 levels down.

\begin{center}
\includegraphics[width=0.8\linewidth,keepaspectratio]{neo4j20}
\end{center}	    

{\tiny (Ref: Introduction to Neo4j and Graph Databases
 - M David Allen)}

\end{frame}

%%%%%%%%%%%%%%%%%%%%%%%%%%%%%%%%%%%%%%%%%%%%%%%%%%%%%%%%%%%
\begin{frame}[fragile]\frametitle{Choice}
When to choose Neo4j over Relational databases? Sub-second response \ldots


\begin{center}
\includegraphics[width=\linewidth,keepaspectratio]{neo4j21}
\end{center}	    

{\tiny (Ref: Introduction to Neo4j and Graph Databases
 - M David Allen)}

\end{frame}


%%%%%%%%%%%%%%%%%%%%%%%%%%%%%%%%%%%%%%%%%%%%%%%%%%%%%%%%%%%
\begin{frame}[fragile]\frametitle{Indexes}
Used only to find the starting points for queries

\begin{center}
\includegraphics[width=\linewidth,keepaspectratio]{neo4j16}
\end{center}	  


{\tiny (Ref: Introduction to Neo4j and Graph Databases
 - M David Allen)}

\end{frame}

%%%%%%%%%%%%%%%%%%%%%%%%%%%%%%%%%%%%%%%%%%%%%%%%%%%%%%%%%%%%%%%%%%%%%%%%%%%%%%%%%%
\begin{frame}\frametitle{Neo4j Drawbacks}


\begin{itemize}
\item  Scalability
\item   Complex Domains
\item   Complex types
\item   Deleted Records
\end{itemize}

 

{\tiny (Ref: CIS 6930 - Advanced Databases - Neo4j )}
\end{frame}

%%%%%%%%%%%%%%%%%%%%%%%%%%%%%%%%%%%%%%%%%%%%%%%%%%%%%%%%%%%%%%%%%%%%%%%%%%%%%%%%%%
\begin{frame}[fragile]\frametitle{}
\begin{center}
{\Large Architecture}
\end{center}
\end{frame}

%%%%%%%%%%%%%%%%%%%%%%%%%%%%%%%%%%%%%%%%%%%%%%%%%%%%%%%%%%%%%%%%%%%%%%%%%%%%%%%%%%
\begin{frame}\frametitle{Native Graph Processing}



\begin{itemize}
\item Index-free adjacency
\item Each node maintains direct references to its adjacent nodes
\item Efficient query time
\end{itemize}

{\tiny (Ref: CIS 6930 - Advanced Databases - Neo4j )}
\end{frame}

%%%%%%%%%%%%%%%%%%%%%%%%%%%%%%%%%%%%%%%%%%%%%%%%%%%%%%%%%%%%%%%%%%%%%%%%%%%%%%%%%%
\begin{frame}\frametitle{Native Graph Storage}

\begin{center}
\includegraphics[width=0.5\linewidth,keepaspectratio]{neo4j37}
\end{center}	  


{\tiny (Ref: CIS 6930 - Advanced Databases - Neo4j )}
\end{frame}

%%%%%%%%%%%%%%%%%%%%%%%%%%%%%%%%%%%%%%%%%%%%%%%%%%%%%%%%%%%%%%%%%%%%%%%%%%%%%%%%%%
\begin{frame}\frametitle{Native Graph Storage}

\begin{center}
\includegraphics[width=\linewidth,keepaspectratio]{neo4j38}
\end{center}	  


{\tiny (Ref: CIS 6930 - Advanced Databases - Neo4j )}
\end{frame}

%%%%%%%%%%%%%%%%%%%%%%%%%%%%%%%%%%%%%%%%%%%%%%%%%%%%%%%%%%%%%%%%%%%%%%%%%%%%%%%%%%
\begin{frame}\frametitle{Cypher Query Language}

\begin{itemize}
\item Neo4j’s open graph query language
\item Uses patterns to describe graph data
\item Familiar SQL-like clauses
\item Describe what to find, not how to find it
\end{itemize}

{\tiny (Ref: CIS 6930 - Advanced Databases - Neo4j )}
\end{frame}
