%%%%%%%%%%%%%%%%%%%%%%%%%%%%%%%%%%%%%%%%%%%%%%%%%%%%%%%%%%%%%%%%%%%%%%%%%%%%%%%%%%
\begin{frame}[fragile]\frametitle{}
\begin{center}
{\Large Create a Blockchain}\\
{\small Ref: Learn with Udemy Courses : Blockchain A-z}
\end{center}
\end{frame}

%%%%%%%%%%%%%%%%%%%%%%%%%%%%%%%%%%%%%%%%%%%%%%%%%%%%%%%%%%%%%%%%%%%%%%%%%%%%%%%%%%
\begin{frame}[fragile]\frametitle{Installations}
\begin{itemize}
\item Install Anaconda-Python 3.6
\item Install packages: postman, Flask 0.12.2
\item Have some IDE like Spider, Pycharm etc.
\end{itemize}
\end{frame}

%%%%%%%%%%%%%%%%%%%%%%%%%%%%%%%%%%%%%%%%%%%%%%%%%%%%%%%%%%%%%%%%%%%%%%%%%%%%%%%%%%
\begin{frame}[fragile]\frametitle{Creating a Blockchain}
\begin{itemize}
\item Create class Blockchain
\item Initialize the chain as empty list and create first/genesis block with some `proof` and None (ie '0') previous hash.
\item A block can get added if they come up with has that is same as given proof.
\item Each block will be a dict with keys as: index, timestamp, proof, previous\_hash and data
\end{itemize}

\begin{lstlisting}
class Blockchain:
    def __init__(self):
        self.chain = []
        self.create_block(proof=1, previous_hash='0')

    def create_block(self, proof, previous_hash):
        block = {'index': len(self.chain)+1,
                 'timestamp': str(datetime.datetime.now()),
                 'proof': proof,
                 'previous_hash': previous_hash
                 }
        self.chain.append(block)
        return block
\end{lstlisting}
\end{frame}


%%%%%%%%%%%%%%%%%%%%%%%%%%%%%%%%%%%%%%%%%%%%%%%%%%%%%%%%%%%%%%%%%%%%%%%%%%%%%%%%%%
\begin{frame}[fragile]\frametitle{Proof of Work}
\begin{itemize}
\item To add a new block to the blockchain, the miner (the adder) needs to solve the `proof`. Needs to be non-symmetrical, order independent, square to make it 
\item Meaning s/he needs to generate a hash that matches with the string given as `proof`.
\item The proof problem has to be challenging to solve (or hard to find)  but easy to verify.
\item Its hard as not many can solve it so value of the blockchain is not diluted.

\end{itemize}

\begin{lstlisting}
def proof_of_work(self, previous_proof):
		new_proof = 1
		check_proof = False
		while check_proof is False:
				hash_operation = hashlib.sha256(str(new_proof**2 - previous_proof**2).encode()) .hexdigest()
				if hash_operation[:4] == '0000': # leading 0's ie 4
						check_proof = True
				else:
						new_proof += 1
				return new_proof
\end{lstlisting}
\end{frame}

%%%%%%%%%%%%%%%%%%%%%%%%%%%%%%%%%%%%%%%%%%%%%%%%%%%%%%%%%%%%%%%%%%%%%%%%%%%%%%%%%%
\begin{frame}[fragile]\frametitle{Chain Validity}
\begin{itemize}
\item Check ALL blocks of the chain to see if they are valid
\item Check: hash of previous block should match with the 'previous\_hash' field in the current block.
\item Check: hash of the whole current block has 4 leading 0's.
\end{itemize}

\begin{lstlisting}
def is_chain_valid(self, chain):
		block_index = 1
		previous_block = chain[0]
		while block_index < len(chain):
				block = chain[block_index]
				if block['previous_hash'] != self.hash(previous_block):
						return False
				previous_proof = previous_block['proof']
				proof = block['proof']
				hash_operation = hashlib.sha256(str(proof**2 - previous_proof**2).encode()) .hexdigest()
				if hash_operation[:4] != '0000':
						return False
				block_index += 1
				previous_block = block
		return True
\end{lstlisting}
\end{frame}

%%%%%%%%%%%%%%%%%%%%%%%%%%%%%%%%%%%%%%%%%%%%%%%%%%%%%%%%%%%%%%%%%%%%%%%%%%%%%%%%%%
\begin{frame}[fragile]\frametitle{Create Web App}
\begin{itemize}
\item Web App to interact with blockchain, on web
\item For now, interacting with local blockchain.
\item Start Flask server, put url + routing in browser, the callback gets called.
\item To mine a new block, get previous proof, create new proof, then create new block.
\end{itemize}

\begin{lstlisting}
@app.route('/mine_block', methods=['GET'])
def mine_block():
    previous_block = blockcahin.get_previous_block()
    previous_proof = previous_block['previous_proof']
    proof = blockcahin.proof_of_work(previous_proof)
    previous_hash = blockcahin.hash(previous_block)
    block = blockcahin.create_block(proof, previous_hash)
    response = {'message': 'Congrats, you mined a block',
                'index': block['index'],
                'timestamp': block['timestamp'],
                'proof': block['proof'],
                'previous_hash': block['previous_hash']}
    return jsonify(response), 200
\end{lstlisting}
\end{frame}

%%%%%%%%%%%%%%%%%%%%%%%%%%%%%%%%%%%%%%%%%%%%%%%%%%%%%%%%%%%%%%%%%%%%%%%%%%%%%%%%%%
\begin{frame}[fragile]\frametitle{Display Blockchain}
\begin{itemize}
\item Send GET request to get full blockchain
\item Display it in postman
\end{itemize}

\begin{lstlisting}
@app.route('/mine_block', methods=['GET'])
def mine_block():
    previous_block = blockcahin.get_previous_block()
    previous_proof = previous_block['previous_proof']
    proof = blockcahin.proof_of_work(previous_proof)
    previous_hash = blockcahin.hash(previous_block)
    block = blockcahin.create_block(proof, previous_hash)
    response = {'message': 'Congrats, you mined a block',
                'index': block['index'],
                'timestamp': block['timestamp'],
                'proof': block['proof'],
                'previous_hash': block['previous_hash']}
    return jsonify(response), 200
\end{lstlisting}
\end{frame}

%%%%%%%%%%%%%%%%%%%%%%%%%%%%%%%%%%%%%%%%%%%%%%%%%%%%%%%%%%%%%%%%%%%%%%%%%%%%%%%%%%
\begin{frame}[fragile]\frametitle{Demo}
\begin{itemize}
\item Run `blockchain.py` with localhost and port as 5000
\item Go to Postman and fire get\_chain first to see genesis block, then `mine\_block` to add block and then `get\_chain` again to see bigger chain.
\end{itemize}
\begin{lstlisting}
{
    "chain": [
        {
            "index": 1,
            "previous_hash": "0",
            "proof": 1,
            "timestamp": "2022-09-27 14:58:09.070311"
        },
        {
            "index": 2,
            "previous_hash": "32853ba4e5fa299f55ce64e0fdc14deafb3e4067985d5d6f919c348197f84768",
            "proof": 2,
            "timestamp": "2022-09-27 15:46:12.540411"
        }
    ],
    "length": 2
}
\end{lstlisting}

\end{frame}