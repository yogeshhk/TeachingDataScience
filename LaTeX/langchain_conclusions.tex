%%%%%%%%%%%%%%%%%%%%%%%%%%%%%%%%%%%%%%%%%%%%%%%%%%%%%%%%%%%%%%%%%%%%%%%%%%%%%%%%%%
\begin{frame}[fragile]\frametitle{}
\begin{center}
{\Large Conclusions}
\end{center}
\end{frame}

% %%%%%%%%%%%%%%%%%%%%%%%%%%%%%%%%%%%%%%%%%%%%%%%%%%%%%%%%%%%%%%%%%%%%%%%%%%%%%%%%%%
% \begin{frame}\frametitle{What is LangChain?}

% \textbf{LangChain}: A Comprehensive Framework for Building LLM-Powered Applications

% \begin{itemize}
% \item \textbf{Core Purpose}: Simplify development of complex applications using Large Language Models
% \item \textbf{Key Components}:
    % \begin{itemize}
        % \item Unified interface for multiple language models
        % \item Advanced prompt management
        % \item Flexible data integration
        % \item Intelligent agent and tool systems
    % \end{itemize}
% \item \textbf{Founding}: Created by Harrison Chase as an open-source project
% \item \textbf{Availability}: Python and JavaScript libraries
% \end{itemize}

% \begin{center}
% \begin{tabular}{|l|l|}
% \hline
% \textbf{Feature} & \textbf{Description} \\
% \hline
% Data Awareness & Connect LLMs to external data sources \\
% Agentic Capability & Enable dynamic interaction with environment \\
% Model Flexibility & Support for multiple LLM providers \\
% \hline
% \end{tabular}
% \end{center}

% {\tiny (Ref: LangChain Framework Overview)}
% \end{frame}

% %%%%%%%%%%%%%%%%%%%%%%%%%%%%%%%%%%%%%%%%%%%%%%%%%%%%%%%%%%%%%%%%%%%%%%%%%%%%%%%%%%
% \begin{frame}\frametitle{Why LangChain?}

% \textbf{Addressing Challenges in LLM Application Development}

% \begin{itemize}
% \item \textbf{Limitations in Current LLM Tooling}:
    % \begin{itemize}
        % \item Fragmented model interfaces
        % \item Complex prompt management
        % \item Limited external data integration
        % \item Lack of flexible reasoning mechanisms
    % \end{itemize}

% \item \textbf{LangChain Solutions}:
    % \begin{itemize}
        % \item Model-agnostic framework
        % \item Standardized prompt engineering
        % \item Seamless external data augmentation
        % \item Intelligent agent orchestration
    % \end{itemize}

% \item \textbf{Key Advantages}:
    % \begin{itemize}
        % \item Rapid prototyping of AI applications
        % \item Simplified model and tool integration
        % \item Scalable architecture for complex use cases
    % \end{itemize}
% \end{itemize}

% \begin{center}
% \textbf{Enabling Developers to Build Sophisticated AI Systems with Ease}
% \end{center}

% {\tiny (Ref: LangChain Development Principles)}
% \end{frame}


%%%%%%%%%%%%%%%%%%%%%%%%%%%%%%%%%%%%%%%%%%%%%%%%%%%%%%%%%%%%%%%%%%%%%%%%%%%%%%%%%%
\begin{frame}\frametitle{Key Takeaways}


\textbf{LangChain in Five Concepts:}
\begin{itemize}
\item \textbf{Model Abstraction}: One API for any LLM provider
\item \textbf{LCEL}: Compose any pipeline with the pipe operator
\item \textbf{RAG}: Load $\to$ Split $\to$ Embed $\to$ Retrieve $\to$ Generate
\item \textbf{Agents}: LLMs that choose and call tools dynamically
\item \textbf{LangGraph}: For stateful, multi-step agentic workflows
\end{itemize}


\vspace{0.5cm}

\begin{center}
\textbf{You're ready to build intelligent LLM applications!}
\end{center}

\end{frame}
% ```

% **IMAGE SUGGESTION:** A simple pyramid diagram:
% ```
        % Agents (Top)
       % /          \
    % Memory      Tools
      % |            |
    % Chains     Retrievers
       % \          /
         % Models
       % (Foundation)
	   
	   
%%%%%%%%%%%%%%%%%%%%%%%%%%%%%%%%%%%%%%%%%%%%%%%%%%%%%%%%%%%
\begin{frame}[fragile]\frametitle{LangChain at a Glance}

\begin{center}
\includegraphics[width=0.9\linewidth,keepaspectratio]{llm99}
\end{center}

\begin{columns}
    \begin{column}{0.5\textwidth}
        \begin{itemize}
        \item \textbf{Models}: LLMs, Chat, Embeddings
        \item \textbf{Prompts}: Dynamic templates
        \item \textbf{Chains}: Compose with LCEL
        \end{itemize}
    \end{column}
    \begin{column}{0.5\textwidth}
        \begin{itemize}
        \item \textbf{Memory}: Conversation state
        \item \textbf{Retrievers}: Document access
        \item \textbf{Agents}: Autonomous tools
        \end{itemize}
    \end{column}
\end{columns}

{\tiny (Ref: Building Generative AI applications - Anand Iyer, Rajesh Thallam)}

\end{frame}
	   
% %%%%%%%%%%%%%%%%%%%%%%%%%%%%%%%%%%%%%%%%%%%%%%%%%%%%%%%%%%%
% \begin{frame}[fragile]\frametitle{One pager}

% \begin{itemize}
% \item \textbf{Models:} Building blocks supporting different AI model types - LLMs, Chat, Text Embeddings.
% \item \textbf{Prompts:} Inputs constructed from various components. LangChain offers easy interfaces - Prompt Templates, Example Selectors, Output Parsers.
% \item \textbf{Memory:} Stores/retrieves messages, short or long term, in conversations.
% \item \textbf{Indexes:} Assist LLMs with documents - Document Loaders, Text Splitters, Vector Stores, Retrievers.
% \item \textbf{Chains:} Combine components or chains in order to accomplish tasks.
% \item \textbf{Agents:} Empower LLMs to interact with external systems, make decisions, and complete tasks using Tools.
% \end{itemize}

% \begin{center}
% \includegraphics[width=0.8\linewidth,keepaspectratio]{llm99}
% \end{center}


% {\tiny (Ref: Building Generative AI applications made easy with Vertex AI PaLM API and LangChain  - Anand Iyer, Rajesh Thallam)}

% \end{frame}

% %%%%%%%%%%%%%%%%%%%%%%%%%%%%%%%%%%%%%%%%%%%%%%%%%%%%%%%%%%%%%%%%%%%%%%%%%%%%%%%%%%
% \begin{frame}[fragile]\frametitle{Migration Quick Reference}

% \begin{center}
% \begin{tabular}{|p{5cm}|p{5cm}|}
% \hline
% \textbf{Old (Deprecated)} & \textbf{New } \\
% \hline
% \texttt{LLMChain} & LCEL with \texttt{|} operator \\
% \hline
% \texttt{SimpleSequentialChain} & Chained LCEL runnables \\
% \hline
% \texttt{ConversationChain} & LCEL + Memory + Prompt \\
% \hline
% \texttt{RetrievalQA} & Custom LCEL RAG chain \\
% \hline
% \texttt{AgentType.ZERO\_SHOT} & \texttt{create\_tool\_calling\_agent} \\
% \hline
% \texttt{chain.run()} & \texttt{chain.invoke()} \\
% \hline
% \hline
% \texttt{langchain.schema} & \texttt{langchain\_core.messages} \\
% \hline
% \texttt{gpt-3.5-turbo} & \texttt{gemma2-9b-it}, \texttt{meta-llama/llama-4-scout-17b-16e-instruct} (Groq) \\
% \hline
% OpenAI imports & \texttt{langchain\_groq} imports \\
% \hline
% \texttt{VectorstoreIndexCreator} & Manual vector store creation \\
% \hline
% \end{tabular}
% \end{center}

% \textbf{Migration Steps:}
% \begin{enumerate}
% \item Update imports to new package structure (e.g., \texttt{langchain\_groq})
% \item Replace \texttt{LLMChain} with LCEL pipes
% \item Update deprecated model names to modern ones
% \item Replace old agent types with modern agent constructors
% \item Test thoroughly - behavior may differ slightly
% \end{enumerate}

% \end{frame}

% %%%%%%%%%%%%%%%%%%%%%%%%%%%%%%%%%%%%%%%%%%%%%%%%%%%%%%%%%%%%%%%%%%%%%%%%%%%%%%%%%%
% \begin{frame}\frametitle{Summary: Modern LangChain}

% \textbf{Key Takeaways:}
% \begin{itemize}
% \item \textbf{LCEL is the Standard}: Use pipe operator for all chains
% \item \textbf{Modular Architecture}: Separate packages for different providers
% \item \textbf{Streaming First}: Built-in streaming and async support
% \item \textbf{Type Safety}: Pydantic models for structured outputs
% \item \textbf{Production Ready}: LangServe for deployment, LangSmith for monitoring
% \item \textbf{Stateful Workflows}: LangGraph for complex agent systems
% \item \textbf{Fast Inference}: Groq LPU for production-speed responses
% \end{itemize}

% \textbf{Best Practices Summary:}
% \begin{itemize}
% \item Use \texttt{ChatOpenAI} instead of \texttt{OpenAI}
% \item Implement fallbacks and retries
% \item Monitor token usage and costs
% \item Use async for parallel operations
% \item Leverage structured outputs
% \item Enable LangSmith tracing
% \end{itemize}

% \end{frame}

%%%%%%%%%%%%%%%%%%%%%%%%%%%%%%%%%%%%%%%%%%%%%%%%%%%%%%%%%%%%%%%%%%%%%%%%%%%%%%%%%%
\begin{frame}\frametitle{Best Practices to Remember}

\begin{columns}
    \begin{column}{0.5\textwidth}
        \textbf{Development:}
        \begin{itemize}
        \item Use LCEL for chains
        \item Start with simple models
        \item Add complexity gradually
        \item Test with small datasets
        \end{itemize}
        
        \vspace{0.3cm}
        
        \textbf{Production:}
        \begin{itemize}
        \item Enable LangSmith tracing
        \item Implement error handling
        \item Monitor token usage
        \item Use async for scale
        \end{itemize}
    \end{column}
    \begin{column}{0.5\textwidth}
        \begin{center}
        % SUGGESTED IMAGE: Checklist or roadmap
        \includegraphics[width=\linewidth,keepaspectratio]{langchain_best_practices}
        
		{\tiny (Ref: What is LangChain and Why Should You Care? - Saif Ali)}
        \vspace{0.5cm}
        
        \fcolorbox{green}{green!10}{
        \parbox{0.9\linewidth}{
        \centering
        \textbf{Golden Rule:}\\
        \vspace{0.2cm}
        Keep it simple,\\
        make it work,\\
        then optimize
        }
        }
        \end{center}
    \end{column}
\end{columns}

\end{frame}

%%%%%%%%%%%%%%%%%%%%%%%%%%%%%%%%%%%%%%%%%%%%%%%%%%%%%%%%%%%%%%%%%%%%%%%%%%%%%%%%%%
\begin{frame}\frametitle{Resources \& Next Steps}

\textbf{Official Documentation:}
\begin{itemize}
\item LangChain Docs: https://python.langchain.com
\item LangChain Blog: https://blog.langchain.dev
\item LangChain Academy: https://academy.langchain.com
\item API Reference: https://api.python.langchain.com
\end{itemize}

\textbf{GitHub Repositories:}
\begin{itemize}
\item Core: https://github.com/langchain-ai/langchain
\item Templates: https://github.com/langchain-ai/langchain/tree/master/templates
\item LangGraph: https://github.com/langchain-ai/langgraph
\end{itemize}

\textbf{Community:}
\begin{itemize}
\item Discord: https://discord.gg/langchain
\item Twitter: @LangChainAI
\item YouTube: LangChain official channel
\end{itemize}

\textbf{Practice:}
\begin{itemize}
\item Start with simple LCEL chains
\item Build a RAG application
\item Create custom tools and agents
\item Deploy with LangServe
\end{itemize}

\end{frame}

% %%%%%%%%%%%%%%%%%%%%%%%%%%%%%%%%%%%%%%%%%%%%%%%%%%%%%%%%%%%
% \begin{frame}[fragile]\frametitle{Closing Thoughts}

% \begin{itemize}
% \item Langchain's usefulness in solving problems today
% \item Possibility of LLM APIs expanding capabilities over time
% \item Potential for Langchain to become an interface to LLMs
% \item Acknowledgment of Langchain's valuable contributions and community efforts
% \item Appreciation for the work done by Harrison and the Langchain team
% \end{itemize}

% {\tiny (Ref: What is Langchain and why should I care as a developer? - Logan Kilpatrick)}

% \end{frame}

%%%%%%%%%%%%%%%%%%%%%%%%%%%%%%%%%%%%%%%%%%%%%%%%%%%%%%%%%%%
\begin{frame}[fragile]\frametitle{Final Thoughts}

\begin{center}
\includegraphics[width=0.5\linewidth,keepaspectratio]{langchain_logo}
\end{center}

\vspace{0.2cm}

\begin{center}
\Large
\textbf{LangChain makes building with LLMs accessible}
\end{center}

\vspace{0.2cm}

\begin{itemize}
\item Start experimenting today
\item Join the community
\item Share your projects
\item Keep learning and building
\end{itemize}

\vspace{0.2cm}

\begin{center}
\fcolorbox{purple}{purple!10}{
\parbox{0.8\linewidth}{
\centering
\Large
\textbf{Thank you!}\\
\vspace{0.2cm}
\normalsize
Questions? Let's discuss.
}
}
\end{center}

{\tiny (Ref: What is Langchain and why should I care as a developer? - Logan Kilpatrick)}

\end{frame}