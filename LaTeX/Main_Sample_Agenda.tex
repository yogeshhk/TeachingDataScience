\documentclass[a4paper, 11pt]{article}
\usepackage{comment} % enables the use of multi-line comments (\ifx \fi) 
\usepackage{fullpage} % changes the margin
\usepackage[swedish]{babel}
\usepackage[utf8]{inputenc}
\usepackage{graphicx}
\usepackage{multicol}
\usepackage{float}
\usepackage{fancyhdr}
\usepackage{enumitem}
\pagestyle{fancy} 
\usepackage{pdfpages}

%%%%%%%%%%%%%%%%%%%%%%%%%%
\title{Agenda}
\author{Yogesh Haribhau Kulkarni}
\usepackage{geometry}
\setlength{\footskip}{0.1pt}
\setlength{\headheight}{80pt}
\setlength{\topmargin}{0pt}
\setlength\parindent{0pt}

%%%%%%%%%%%%%%%%%%%%%%%%%%

\fancypagestyle{mystyle}{
\lhead{\includegraphics[width=4.5cm]{images/YHK_Color_OutOfTheBox.png}}

\rhead{
\textbf{\large Dr. Yogesh Haribhau Kulkarni}\\
E1-32 State Bank Nagar,\\
NCL, Pashan, Pune, India\\
\textbf{Mobile:}  +91 989 025 1406\\
\textbf{Email:} yogeshkulkarni@yahoo.com\\
\textbf{LinkedIn:} https://www.linkedin.com/in/yogeshkulkarni/\\
}
\renewcommand{\headrulewidth}{0pt}
\cfoot{}

}

%%%%%%%%%%%%%%%%%%%%%%%%%
\begin{document}
\pagestyle{mystyle}

\makebox[\linewidth]{}
\begin{center} 
\textbf{\Large Workshop Title}
\end{center}
\makebox[\linewidth]{} % \rule{\linewidth}{1pt}

Unlock the potential of WORKSHOP in this 4-hour workshop. 
Learn essential techniques to craft effective WORKSHOP for large language models. 
Explore strategies to control model behavior and optimize outputs for various applications.

\makebox[\linewidth]{}

\textbf{Prerequsites:} Computer usage literacy
\begin{itemize}[topsep=0pt, partopsep=0pt, itemsep=0pt, parsep=0pt]
\item aa
\item bb
\end{itemize}

\makebox[\linewidth]{}

\begin{center} 
\begin{tabular}{ |p{1cm}|p{1cm}||p{10cm}|  }
 \hline
 \multicolumn{3}{|c|}{\textbf{\large Agenda}} \\
 \hline
Day & Session & Topic\\
 \hline
 One   &  One    &  \textbf{Introduction to ChatGPT}
\begin{itemize}[topsep=0pt, partopsep=0pt, itemsep=0pt, parsep=0pt]
\item aa
\end{itemize} 
\\
One &   Two  &   \textbf{Introduction to Prompt Engineering}
 \begin{itemize}[topsep=0pt, partopsep=0pt, itemsep=0pt, parsep=0pt]
\item bb
\end{itemize} 
\\
Two   &  One    & \textbf{Advanced Prompt Design}
 \begin{itemize}[topsep=0pt, partopsep=0pt, itemsep=0pt, parsep=0pt]
\item cc
\end{itemize} 
\\
Two  &   Two  & \textbf{Applications to Marketing}
 \begin{itemize}[topsep=0pt, partopsep=0pt, itemsep=0pt, parsep=0pt]
\item cc
\end{itemize} 
\\
 \hline
\end{tabular}
\end{center}

\end{document}