\documentclass[11pt,paper=a4]{exam}
\usepackage{graphicx}
\addtolength{\topmargin}{-2cm}
\noprintanswers
\boxedpoints
\date{}

\begin{document}

\begin{center}
\includegraphics[width=\linewidth]{images/coep_new_logo.png}

\rule{\textwidth}{1pt}

\vspace{2ex}

{\huge{\bf {Test 2 Examination}}}
\end{center}



\thispagestyle{empty}

\begin{minipage}[t]{.5\textwidth}%
{\bf Programme}: M.Tech.\par
\vspace{1ex}
{\bf Course Code}: (DE) - 21002   \par
\vspace{1ex}
{\bf Course Name}: Data Science (Elective)\par
\vspace{1ex}
{\bf Branch}: Robotics \& Automation
\end{minipage}%
\hfill
\begin{minipage}[t]{.4\textwidth}%
{\bf Semester}:I \par
\vspace{1ex} 
{\bf Duration}: 1 hr \par
\vspace{1ex}
{\bf PRN No.}:\makebox[.5\textwidth]{\hrulefill}  \par
\vspace{1ex}
{\bf Faculty}:Dr. Yogesh H. Kulkarni
\end{minipage}

\vspace{2ex}
\rule{\textwidth}{1pt}

\section*{Instructions}

\begin{center}
\fbox{\fbox{\parbox{5.5in}{
\begin{itemize}
\item Figures in the box indicate the full marks.
%\item Mobile phones and programmable calculators are strictly prohibited.
%\item Writing anything on question paper is not allowed.
%\item Exchange/Sharing of stationery, calculator etc. not allowed.
\item Write your PRN Number on Question Paper.  
\end{itemize}
}}}
\end{center}

\rule{\textwidth}{1pt}
\section*{Questions}

\begin{questions}
\question[4] Describe Exploratory Data Analysis (EDA) and types of analysis done in it.

  % - Variable Identification
  % - Univariate Analysis
  % - Bi-variate Analysis
  % - Missing values treatment
  % - Outlier treatment
  % - Variable transformation
  % - Variable creation
  
\question[6] Describe types of data with examples.

 % - Categorical types (Qualitative): Nominal and Ordinal
		% - Nominal (numbers do not give sense of order/rank): eye color, zip codes
		% - Ordinal (numbers give sense of order/rank): rankings, size in small-	medium-large
 % - Numeric types (Quantitative): Interval and Ratio
		% - Discrete: A discrete attribute has a fi nite or countably infinite set of values. 
		% - Binary attributes are a special case of discrete attributes.
		% - Continuous: A continuous attribute is one whose values are real numbers.
		% - Interval: calendar dates
		% - Ratio: counts, time
		
\question[4] Describe types and subtypes of Machine Learning algorithms 

		% Supervised: 
			% - Prediction: predicting a continuous variable from data
			% - Classification: assigning records to predefi ned groups
		% Unsupervised:
			% - Clustering: splitting records into groups based on similarity
			
			
\question[6] Write Pandas Python function for the following:

				\begin{itemize}
				\item Create Series using dictionary, with keys as index 
				\item Find top 10 rows of dataframe
				\item Find columns of dataframe
				\item Find statistical summary of dataframe
				\item Select single column in dataframe
				\item Find sub-dataframe from 1st to 3rd row, all columns
				\end{itemize}

 % - Create Series using dictionary, with keys as index cities = pd.Series(d)
 % - Find top 10 rows of dataframe df.head(10)
 % - Find columns of dataframe df.columns
 % - Find statistical summary of dataframe df.describe()
 % - Select single column in dataframe df['A']
 % - Find sub-dataframe from 1st to 3rd row, all columns df[0:3]
 
\end{questions}
\end{document}

