%%%%%%%%%%%%%%%%%%%%%%%%%%%%%%%%%%%%%%%%%%%%%%%%%%%%%%%%%%%%%%%%%%%%%%%%%%%%%%%%%%
\begin{frame}[fragile]\frametitle{}
\begin{center}
{\Large Introduction to Descriptive Statistics}
\end{center}
\end{frame}


%%%%%%%%%%%%%%%%%%%%%%%%%%%%%%%%%%%%%%%%%%%%%%%%%%%%%%%%%%%
\begin{frame}[fragile]\frametitle{Descriptive Statistics}
\begin{itemize}
\item Describes the data characteristics
\item To make sense of the data
\item To make rational decisions
\item E.g. Demographics, clinical data.
\item Measures of Central Tendencies
\item Measures of Variability
\item Measures of Shape
\end{itemize}
\end{frame}

%%%%%%%%%%%%%%%%%%%%%%%%%%%%%%%%%%%%%%%%%%%%%%%%%%%%%%%%%%%
\begin{frame}[fragile]\frametitle{Why Descriptive Statistics?}
\begin{itemize}
\item Population: the whole
\item Sample: small subset of the population
\item Gauging Population by examining traits of Sample.
\item Example Question: Finding height of Americans?
\item Not going to measure everyone height, but that in a {\bf representative} sample.
\item Example: Election sampling?
\end{itemize}
\end{frame}



%%%%%%%%%%%%%%%%%%%%%%%%%%%%%%%%%%%%%%%%%%%%%%%%%%%%%%%%%%%
\begin{frame}[fragile]\frametitle{Why Descriptive Statistics?}
\begin{itemize}
\item To check the accuracy and precision of the process
\item To reduce variability and improve process capability
\item To know the truth about the real world
\end{itemize}
\end{frame}


%%%%%%%%%%%%%%%%%%%%%%%%%%%%%%%%%%%%%%%%%%%%%%%%%%%%%%%%%%%%%%%%%%%%%%%%%%%%%%%%%%
\begin{frame}[fragile]\frametitle{}
\begin{center}
{\Large Basic Terms}
\end{center}
\end{frame}

%%%%%%%%%%%%%%%%%%%%%%%%%%%%%%%%%%%%%%%%%%%%%%%%%%%%%%%%%%%%%%%%%%%%%%%%
\begin{frame}[fragile]\frametitle{Histogram}

\begin{columns}
    \begin{column}[T]{0.6\linewidth}
Example
	\begin{itemize}
	\item Say, we are measuring height of people.
	\item Plotting them on X axis.
	\item The dots would look very crowded where there are many close or repetitive observations.
	\item Some dots get hidden.
	\item We can improve the visualization, by plotting frequency (number of occurrences) on Y axis.
	\item But in case of contiguous variable, like, exact measurements are rare. So we `bin' them and measure occurrences.
	\item Thats Histogram.
	\end{itemize}

    \end{column}
    \begin{column}[T]{0.4\linewidth}
      \begin{center}
      \includegraphics[width=\linewidth,keepaspectratio]{statq1}
	  
	  \includegraphics[width=\linewidth,keepaspectratio]{statq2}
	  
	  \includegraphics[width=\linewidth,keepaspectratio]{statq3}	  
	  	\end{center}
    \end{column}

  \end{columns}
  

\tiny{(Ref: StatQuest: What is a Histogram? - Josh Starmer )}
\end{frame}

%%%%%%%%%%%%%%%%%%%%%%%%%%%%%%%%%%%%%%%%%%%%%%%%%%%%%%%%%%%%%%%%%%%%%%%%
\begin{frame}[fragile]\frametitle{Histogram}

\begin{columns}
    \begin{column}[T]{0.6\linewidth}

	\begin{itemize}
	\item Histogram can be used to predict probability of getting (future) measurements.
	\item Getting measurement (as shown in the box) in the middle region is more likely.
	\item Measurements at both the ends are rare.
	\item We can approximate this histogram of observations by a `distribution'.
	\item Looks like `Normal' distribution, or a bell-curve
	\end{itemize}

    \end{column}
    \begin{column}[T]{0.4\linewidth}
      \begin{center}
      \includegraphics[width=\linewidth,keepaspectratio]{statq4}
	  
	  \includegraphics[width=\linewidth,keepaspectratio]{statq5}
	  
	  \includegraphics[width=\linewidth,keepaspectratio]{statq6}	  
	  	\end{center}
    \end{column}

  \end{columns}
  

\tiny{(Ref: StatQuest: What is a Histogram? - Josh Starmer )}
\end{frame}

%%%%%%%%%%%%%%%%%%%%%%%%%%%%%%%%%%%%%%%%%%%%%%%%%%%%%%%%%%%%%%%%%%%%%%%%
\begin{frame}[fragile]\frametitle{Histogram}

\begin{columns}
    \begin{column}[T]{0.6\linewidth}

	\begin{itemize}
	\item If the frequency of measurements seem decreasing, it may be an exponential distribution.
	\item Binning criterion is critical. They can not be too narrow or too wide.
	\item Try different bin widths/formulas to plot a histogram.
	\end{itemize}

    \end{column}
    \begin{column}[T]{0.4\linewidth}
      \begin{center}
      \includegraphics[width=\linewidth,keepaspectratio]{statq7}
	  
	  \includegraphics[width=\linewidth,keepaspectratio]{statq8}
	  
	  \includegraphics[width=\linewidth,keepaspectratio]{statq9}	  
	  	\end{center}
    \end{column}

  \end{columns}
  

\tiny{(Ref: StatQuest: What is a Histogram? - Josh Starmer )}
\end{frame}


%%%%%%%%%%%%%%%%%%%%%%%%%%%%%%%%%%%%%%%%%%%%%%%%%%%%%%%%%%%%%%%%%%%%%%%%%%%%%%%%%%
\begin{frame}[fragile]\frametitle{}
\begin{center}
{\Large Descriptive Statistics Example}
\end{center}
\end{frame}


%%%%%%%%%%%%%%%%%%%%%%%%%%%%%%%%%%%%%%%%%%%%%%%%%%%%%%%%%%%
\begin{frame}[fragile]\frametitle{Descriptive Statistics}
\begin{itemize}
\item Describes features of data sets using numbers
\item Individual row: Data
\item Full table: Dataset
\item Purpose: Answer questions.
\end{itemize}
\begin{center}
\includegraphics[width=0.35\linewidth,keepaspectratio]{da1}
\end{center}
\end{frame}

%%%%%%%%%%%%%%%%%%%%%%%%%%%%%%%%%%%%%%%%%%%%%%%%%%%%%%%%%%%
\begin{frame}[fragile]\frametitle{Questions}
\begin{center}
\includegraphics[width=0.35\linewidth,keepaspectratio]{da1}
\end{center}
\begin{itemize}
\item What is Bobby's score?
\item Out of? (Total \# entries)
\item Highest/Lowest scores?
\end{itemize}
\end{frame}

%%%%%%%%%%%%%%%%%%%%%%%%%%%%%%%%%%%%%%%%%%%%%%%%%%%%%%%%%%%
\begin{frame}[fragile]\frametitle{Questions}
\begin{center}
\includegraphics[width=0.35\linewidth,keepaspectratio]{da1}
\end{center}
\begin{itemize}
\item Class average?
\item Most Common/frequent Score?
\item Any other questions?
\end{itemize}
\end{frame}

%%%%%%%%%%%%%%%%%%%%%%%%%%%%%%%%%%%%%%%%%%%%%%%%%%%%%%%%%%%
\begin{frame}[fragile]\frametitle{Numerical Measures}
\begin{itemize}
\item Highest to Lowest Score: RANGE
\item Most Common Score: MODE
\item Average Score: MEAN
\item Any other measures?
\end{itemize}
\end{frame}

%%%%%%%%%%%%%%%%%%%%%%%%%%%%%%%%%%%%%%%%%%%%%%%%%%%%%%%%%%%
\begin{frame}[fragile]\frametitle{Descriptive Statistics}
\begin{itemize}
\item Examines ALL data (not sample)
\item Cannot generalize to other datasets
\end{itemize}
\end{frame}

%%%%%%%%%%%%%%%%%%%%%%%%%%%%%%%%%%%%%%%%%%%%%%%%%%%%%%%%%%%%%%%%%%%%%%%%%%%%%%%%%%
\begin{frame}[fragile]\frametitle{}
\begin{center}
{\Large Descriptive Statistics Example}
\end{center}
\end{frame}


%%%%%%%%%%%%%%%%%%%%%%%%%%%%%%%%%%%%%%%%%%%%%%%%%%%
\begin{frame}[fragile] \frametitle{Descriptive Tasks}

\adjustbox{valign=t}{
\begin{minipage}{0.45\linewidth}
\begin{center}
\includegraphics[width=\linewidth,keepaspectratio]{descriptivetable}
\end{center}
\end{minipage}
}
\hfill
\adjustbox{valign=t}{
\begin{minipage}{0.45\linewidth}
\begin{itemize}
\item Objective: Derive patterns, summarize underlying relationships
\item More exploratory of current state
\end{itemize}
\end{minipage}
}

\end{frame}



%%%%%%%%%%%%%%%%%%%%%%%%%%%%%%%%%%%%%%%%%%%%%%%%%%%%%%%%%%
\begin{frame}[fragile]\frametitle{Data Example}	
\begin{center}
\includegraphics[width=0.8\linewidth,keepaspectratio]{da2}
\end{center}
What sense it makes?
Any pattern?


%\code{Store as list of integers}
\end{frame}

%%%%%%%%%%%%%%%%%%%%%%%%%%%%%%%%%%%%%%%%%%%%%%%%%%%%%%%%%%
\begin{frame}[fragile]\frametitle{Visualize}	
\begin{center}
\includegraphics[width=0.8\linewidth,keepaspectratio]{da3}
\end{center}
Makes sense?


%\code{Plot the data points, not the curve}
\end{frame}

%%%%%%%%%%%%%%%%%%%%%%%%%%%%%%%%%%%%%%%%%%%%%%%%%%%%%%%%%%
\begin{frame}[fragile]\frametitle{The Shape of The Distribution}	
Better to see
\begin{center}
\includegraphics[width=0.8\linewidth,keepaspectratio]{da4}
\end{center}
Symmetric? Skewed right/left?
%\code{Plot the curve}
\end{frame}
