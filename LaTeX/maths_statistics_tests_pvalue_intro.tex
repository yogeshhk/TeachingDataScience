%%%%%%%%%%%%%%%%%%%%%%%%%%%%%%%%%%%%%%%%%%%%%%%%%%%%%%%%%%%
\begin{frame}
\begin{center}
{\Large P-Value}
\end{center}
\end{frame}



%%%%%%%%%%%%%%%%%%%%%%%%%%%%%%%%%%%%%%%%%%%%%%%%%%%%%%%%%%%%%%%%%%%%%%%%
\begin{frame}[fragile]\frametitle{P-Value}


	\begin{itemize}

	\item Do not confuse this p with probability. They are related but not the same.
	\item Experiment: What is the probability of getting 2 heads in a row? Whats the p-value of getting 2 heads in a row?
	\item Probability of getting 1 heads and 1 tails is $0.25 + 0.25 = 0.5$. Here order did not matter, ie HT and TH is same.
	
	\end{itemize}


      \begin{center}
      \includegraphics[width=\linewidth,keepaspectratio]{statq39}
	  	\end{center}

  
 
\tiny{(Ref: StatQuest: P Values, clearly explained - Josh Starmer )}
\end{frame}

%%%%%%%%%%%%%%%%%%%%%%%%%%%%%%%%%%%%%%%%%%%%%%%%%%%%%%%%%%%%%%%%%%%%%%%%
\begin{frame}[fragile]\frametitle{P-Value}

\begin{itemize}

	\item P-Value of getting HH?
	\item P-value is the probability that random chance generated data (which is 0.25) is equal (ie of TT, which has same probability, ie 0.25) or rarer (there are none here)
	\item So P-value for HH is 0.5
	
	\end{itemize}

      \begin{center}
      \includegraphics[width=\linewidth,keepaspectratio]{statq40}
	  	\end{center}

  
  
 
\tiny{(Ref: StatQuest: P Values, clearly explained - Josh Starmer )}
\end{frame}

%%%%%%%%%%%%%%%%%%%%%%%%%%%%%%%%%%%%%%%%%%%%%%%%%%%%%%%%%%%
\begin{frame}[fragile]\frametitle{Level of Significance: p-value}
\begin{itemize}
\item Level of Significance: Probability of rejecting null  
hypothesis when it is true. Represented by Greek letter `alpha'. 
\item if $p < \alpha$ : there is statistically significant difference between groups
\item if $p > \alpha$ : there is NOT MUCH statistically significant difference between groups
\item $\alpha$ is generally 0.05
\item So, only incorrectly rejecting $H_0$ is ok upto 5\%. 
\item Type I error only upto 5\%
\end{itemize}
\end{frame}

%%%%%%%%%%%%%%%%%%%%%%%%%%%%%%%%%%%%%%%%%%%%%%%%%%%%%%%%%%%
\begin{frame}[fragile]\frametitle{Level of Significance: p-value: Example}
 Why p-value is the deciding factor for accepting or rejecting a hypothesis we
develop before any experiment?
\begin{itemize}
\item You have launched a product (e.g. a phone) in the market. 
\item And you get customer feedback that the phone has over heating problem. 
\item As the phone is already launched in the market you can't recall all of them to test if the majority of the phones have overheating problem due to some manufacturing problem.
\end{itemize}

(Ref: https://www.datasciencecentral.com/profiles/blogs/significance-of-p-value)
\end{frame}


%%%%%%%%%%%%%%%%%%%%%%%%%%%%%%%%%%%%%%%%%%%%%%%%%%%%%%%%%%%
\begin{frame}[fragile]\frametitle{Level of Significance: p-value: Example}

\begin{itemize}
\item Hence to address the issue you have decided to take surveys
\item Lets do a statistical test to overrule your apprehension regarding the manufacturing issue
\item Now you have a random sample of 500 feedbacks against the total number of 250000 phones you have sold.
\item Population size = 25000, Sample size = 500
\end{itemize}


\end{frame}

%%%%%%%%%%%%%%%%%%%%%%%%%%%%%%%%%%%%%%%%%%%%%%%%%%%%%%%%%%%
\begin{frame}[fragile]\frametitle{Level of Significance: p-value: Example}

\begin{itemize}
\item The test conducted within the factory says that at maximum 3\% of the phone may have the overheating problem which is due to some random event (nothing to do with manufacturing as such), may be due to overcharging or overusing.
\item This is acceptable to your company. 
\item Otherwise you have to recall all the phones from market to do a re-evaluation.
\end{itemize}

\begin{center}
\includegraphics[width=0.35\linewidth,keepaspectratio]{pval1}
\end{center}

Now you have to take a decision whether you will recall the phone from the market or not.


\end{frame}

%%%%%%%%%%%%%%%%%%%%%%%%%%%%%%%%%%%%%%%%%%%%%%%%%%%%%%%%%%%
\begin{frame}[fragile]\frametitle{Level of Significance: p-value: Example}
Hypothesis: Set up null hypothesis and alternate hypothesis first
\begin{itemize}
\item Null hypothesis (H0):= Overheating of phones are as expected and due the some random events which was observed during the production process.
\item Alternate hypothesis (H1):= Overheating of the phones are not due to some random events. There must be some strong reason behind the overheating.
\item If p value is large you accept null hypothesis.
\item If p value is small you fail to accept null hypothesis. You believe that the alternate hypothesis is somewhat acceptable. Your test is statistically significant.
\end{itemize}



\end{frame}

%%%%%%%%%%%%%%%%%%%%%%%%%%%%%%%%%%%%%%%%%%%%%%%%%%%%%%%%%%%
\begin{frame}[fragile]\frametitle{Level of Significance: p-value: Example}
Data: Scenario 1:
\begin{center}
\includegraphics[width=0.35\linewidth,keepaspectratio]{pval2}
\end{center}

Data: Scenario 2:
\begin{center}
\includegraphics[width=0.35\linewidth,keepaspectratio]{pval3}
\end{center}

The sample size n = 500.

m = 2. (Number of categorical values (Here they are Overheating \& Non-overheating OR Yes \& No))


\end{frame}

%%%%%%%%%%%%%%%%%%%%%%%%%%%%%%%%%%%%%%%%%%%%%%%%%%%%%%%%%%%
\begin{frame}[fragile]\frametitle{Level of Significance: p-value: Example}
Experiments and Results: let’s set the confidence interval first and know about the types of error.

\begin{itemize}
\item H1 Error: - We reject null hypothesis even though it is true. (In our example, even after observing that the overheating of phones happen due to random events we still reject null hypothesis and assume that the overheating happens due to some manufacturing issue.)

\item H2 Error: - We retain null hypothesis even though it is false. (In our example, even after observing that the overheating of phones happen due to some manufacturing issue we still accept null hypothesis and assume that the overheating happens due to random events.)

\end{itemize}

\end{frame}

%%%%%%%%%%%%%%%%%%%%%%%%%%%%%%%%%%%%%%%%%%%%%%%%%%%%%%%%%%%
\begin{frame}[fragile]\frametitle{Level of Significance: p-value: Example}
Experiments and Results: let’s set the confidence interval first and know about the types of error.

\begin{itemize}

\item CI: Confidence interval for our test will be 95\%. This means we are 95\% confident that the test results of our sample will fall 95\% close to the population.

\item $\alpha$ (Significance level) is the probability of H1 error. Here $\alpha = 0.05$.
\end{itemize}

\end{frame}

%%%%%%%%%%%%%%%%%%%%%%%%%%%%%%%%%%%%%%%%%%%%%%%%%%%%%%%%%%%
\begin{frame}[fragile]\frametitle{Level of Significance: p-value: Example}
Scenario 1:

\begin{center}
\includegraphics[width=\linewidth,keepaspectratio]{pval4}
\end{center}

\begin{itemize}
\item From the experiment we saw the Chi Square (X2) value is 0.3436 and p-value is 0.56 (calculated).
\item  p-value is greater than the $\alpha$ (=0.05).
\end{itemize}
\end{frame}

%%%%%%%%%%%%%%%%%%%%%%%%%%%%%%%%%%%%%%%%%%%%%%%%%%%%%%%%%%%
\begin{frame}[fragile]\frametitle{Level of Significance: p-value: Example}
Scenario 1:
Search the critical value of X2 from the table. 
\begin{center}
\includegraphics[width=0.8\linewidth,keepaspectratio]{pval5}
\end{center}

\begin{itemize}
\item The next critical value after X2=0.3436 is 2.706 and its corresponding p-value is 0.1. 
\item And our X2 value lies between X20.10 and X20.90. 
\item This means our p-value (though we have already calculated) calculated above is between 0.1 to 0.9 and it is not smaller than 0.05.
\end{itemize}
\end{frame}

%%%%%%%%%%%%%%%%%%%%%%%%%%%%%%%%%%%%%%%%%%%%%%%%%%%%%%%%%%%
\begin{frame}[fragile]\frametitle{Level of Significance: p-value: Example}
Scenario 1:
\begin{itemize}
\item We fail to prove any evidence against null hypothesis. We can’t reject null hypothesis. 
\item This means the number of overheating phones we found from the survey is not significantly different than what we observed during our production process.
\end{itemize}
\end{frame}

%%%%%%%%%%%%%%%%%%%%%%%%%%%%%%%%%%%%%%%%%%%%%%%%%%%%%%%%%%%
\begin{frame}[fragile]\frametitle{Level of Significance: p-value: Example}
Scenario 2:

\begin{center}
\includegraphics[width=\linewidth,keepaspectratio]{pval6}
\end{center}

\begin{itemize}
\item From the experiment we saw the Chi Square (X2) value is 16.8384 and p –value is 4.10E-05 (calculated).
\item  p-value is smaller than the $\alpha$ (=0.05).
\end{itemize}
\end{frame}

%%%%%%%%%%%%%%%%%%%%%%%%%%%%%%%%%%%%%%%%%%%%%%%%%%%%%%%%%%%
\begin{frame}[fragile]\frametitle{Level of Significance: p-value: Example}
Scenario 2:
We can also search the critical value of X2 from the table. 
\begin{center}
\includegraphics[width=0.8\linewidth,keepaspectratio]{pval7}
\end{center}

\begin{itemize}
\item The next critical value after X2=X2=16.83 is 2.706 and its corresponding p-value is 0.01. 
\item This means our p-value (though we have already calculated) should be less than 0.01 and it is obviously less than 0.05 s well.
\end{itemize}
\end{frame}

%%%%%%%%%%%%%%%%%%%%%%%%%%%%%%%%%%%%%%%%%%%%%%%%%%%%%%%%%%%
\begin{frame}[fragile]\frametitle{Level of Significance: p-value: Example}
Scenario 2:
\begin{itemize}
\item We have to reject null hypothesis. 
\item This means the number of overheating phones we found from the survey is significantly different than what we observed during our production process. And it is not due to some random events.
\end{itemize}
\end{frame}

%%%%%%%%%%%%%%%%%%%%%%%%%%%%%%%%%%%%%%%%%%%%%%%%%%%%%%%%%%%
\begin{frame}[fragile]\frametitle{Level of Significance: p-value: Example}
Outcome of the Problem Statement:
\begin{itemize}
\item For the 1st scenario we will accept the null hypothesis and our phones don’t have any manufacturing problem.
\item For the 2nd scenario we have to check the phones for their manufacturing problem as we have strong evidence against the null hypothesis.
\end{itemize}
\end{frame}


%%%%%%%%%%%%%%%%%%%%%%%%%%%%%%%%%%%%%%%%%%%%%%%%%%%%%%%%%%%
\begin{frame}[fragile]\frametitle{Level of Significance: p-value}
\begin{itemize}


\item P-value:  The  probability  of  getting  test  statistics  as  extreme  as 
observed, under the null hypothesis is P-value. It is the observed 
level of significance. 
 
\item E.g. study placebo group vs new medication group to lower BP.  
\item Mean BP in the treatment group was less by 20mm. Assuming that $H_0$ is true (ie medication has no effect). 
\item If we repeated the study, then the difference in means can go AT MAX to 20mm.
\item P value is how much data disagrees with the Null Hypothesis.
\item  High similarity High P
\item Low p= reject $H_0$, as there is much diff
\item High p= fail to reject $H_0$. no real diff exists.
\item Whether P is low or high is determined by Level of Significations
\end{itemize}
\end{frame}


%%%%%%%%%%%%%%%%%%%%%%%%%%%%%%%%%%%%%%%%%%%%%%%%%%%
\begin{frame}
\frametitle{Graphical: Critical values}
At 95\% $\alpha$
\begin{center}
\includegraphics[width=0.45\linewidth,keepaspectratio]{alp1}
\end{center}

\begin{itemize}
\item  Compare if calculate z score of 2.6 is it in rejection region? 
\item Yes, so reject $H_0$
\item Basically, if we are in non-reject region, $H_0$ holds, so no real diff.
\item Only when you cross the band, you are making some diff.
\item That band is controlled by $\alpha$.
\end{itemize}
\end{frame}

%%%%%%%%%%%%%%%%%%%%%%%%%%%%%%%%%%%%%%%%%%%%%%%%%%%
\begin{frame}
\frametitle{Graphical:P Value}
How significant my result is.
\begin{center}
\includegraphics[width=0.5\linewidth,keepaspectratio]{alp2}
\end{center}

\begin{itemize}
\item  In one tail, we have 0.025 area
\item Given z score of 2.6, is above critical value of +1.96.
\item P is area above 2.6
\item It can be calculated from area under normal curve' table to be 0.0047
\item $ p < 0.025$. So, reject $H_0$
\end{itemize}
\end{frame}
