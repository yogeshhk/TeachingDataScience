%%%%%%%%%%%%%%%%%%%%%%%%%%%%%%%%%%%%%%%%%%%%%%%%%%%%%%%%%%%%%%%%%%%%%%%%%%%%%%%%%%
\begin{frame}[fragile]\frametitle{}
\begin{center}
{\Large Date and Time}
\end{center}
\end{frame}


%%%%%%%%%%%%%%%%%%%%%%%%%%%%%%%%%%%%%%%%%%%%%%%%%%%%%%%%%%%%%%%%%%%%%%%%%%%%%%%%%%%
\begin{frame}[fragile]\frametitle{Date Time}

    \begin{itemize}
    \item  The datetime module includes functions and classes for doing date and time parsing, formatting, and arithmetic.
    \item Calendar date values are represented with the date class. Instances have attributes for year, month, and day. 
    \item Time values are represented with the time class. Times have attributes for hour, minute, second, and microsecond. 
    \item  You can use datetime to perform basic arithmetic on date values via the timedelta class. 
    \end{itemize}
\end{frame}

%%%%%%%%%%%%%%%%%%%%%%%%%%%%%%%%%%%%%%%%%%%%%%%%%%%%%%%%%%%%%%%%%%%%%%%%%%%%%%%%%%%
\begin{frame}[fragile]\frametitle{Time}
The arguments to initialize a time instance are optional, but the default of 0 is unlikely to be what you want.
\begin{lstlisting}
import datetime

t = datetime.time(1, 2, 3)
print t
print 'hour  :', t.hour
print 'minute:', t.minute
print 'second:', t.second
print 'microsecond:', t.microsecond
print 'tzinfo:', t.tzinfo

01:02:03
hour  : 1
minute: 2
second: 3
microsecond: 0
tzinfo: None
\end{lstlisting}

\end{frame}

%%%%%%%%%%%%%%%%%%%%%%%%%%%%%%%%%%%%%%%%%%%%%%%%%%%%%%%%%%%%%%%%%%%%%%%%%%%%%%%%%%%
\begin{frame}[fragile]\frametitle{ Time}
A time instance only holds values of time, and not a date associated with the time.
\begin{lstlisting}
import datetime

print 'Earliest  :', datetime.time.min
print 'Latest    :', datetime.time.max
print 'Resolution:', datetime.time.resolution

Earliest  : 00:00:00
Latest    : 23:59:59.999999
Resolution: 0:00:00.000001
\end{lstlisting}

\end{frame}

%%%%%%%%%%%%%%%%%%%%%%%%%%%%%%%%%%%%%%%%%%%%%%%%%%%%%%%%%%%%%%%%%%%%%%%%%%%%%%%%%%%
\begin{frame}[fragile]\frametitle{ Time}
The resolution for time is limited to whole microseconds.
\begin{lstlisting}
import datetime

for m in [ 1, 0, 0.1, 0.6 ]:
    try:
        print '%02.1f :' % m, datetime.time(0, 0, 0, microsecond=m)
    except TypeError, err:
        print 'ERROR:', err

1.0 : 00:00:00.000001
0.0 : 00:00:00
0.1 : ERROR: integer argument expected, got float
0.6 : ERROR: integer argument expected, got float        
\end{lstlisting}
\end{frame}

%%%%%%%%%%%%%%%%%%%%%%%%%%%%%%%%%%%%%%%%%%%%%%%%%%%%%%%%%%%%%%%%%%%%%%%%%%%%%%%%%%%
\begin{frame}[fragile]\frametitle{ Dates}
It is easy to create a date representing today's date using the today() class method.
\begin{lstlisting}
import datetime

today = datetime.date.today()
print today
print 'ctime:', today.ctime()
print 'tuple:', today.timetuple()
print 'ordinal:', today.toordinal()
print 'Year:', today.year
print 'Mon :', today.month
print 'Day :', today.day

2013-02-21
ctime: Thu Feb 21 00:00:00 2013
tuple: time.struct_time(tm_year=2013, tm_mon=2, tm_mday=21, tm_hour=0, tm_min=0, tm_sec=0, tm_wday=3, tm_yday=52, tm_isdst=-1)
ordinal: 734920
Year: 2013
Mon : 2
Day : 21
\end{lstlisting}
\end{frame}

%%%%%%%%%%%%%%%%%%%%%%%%%%%%%%%%%%%%%%%%%%%%%%%%%%%%%%%%%%%%%%%%%%%%%%%%%%%%%%%%%%%
\begin{frame}[fragile]\frametitle{ Dates}
The range of date values supported can be determined using the min and max attributes.
\begin{lstlisting}
import datetime

print 'Earliest  :', datetime.date.min
print 'Latest    :', datetime.date.max
print 'Resolution:', datetime.date.resolution

Earliest  : 0001-01-01
Latest    : 9999-12-31
Resolution: 1 day, 0:00:00
\end{lstlisting}
\end{frame}

%%%%%%%%%%%%%%%%%%%%%%%%%%%%%%%%%%%%%%%%%%%%%%%%%%%%%%%%%%%%%%%%%%%%%%%%%%%%%%%%%%%
\begin{frame}[fragile]\frametitle{ Dates}
Another way to create new date instances uses the replace() method of an existing date. For example, you can change the year, leaving the day and month alone.
\begin{lstlisting}
import datetime

d1 = datetime.date(2008, 3, 12)
print 'd1:', d1

d2 = d1.replace(year=2009)
print 'd2:', d2

d1: 2008-03-12
d2: 2009-03-12
\end{lstlisting}
\end{frame}

%%%%%%%%%%%%%%%%%%%%%%%%%%%%%%%%%%%%%%%%%%%%%%%%%%%%%%%%%%%%%%%%%%%%%%%%%%%%%%%%%%%
\begin{frame}[fragile]\frametitle{ timedeltas}
Subtracting dates produces a timedelta, and a timedelta can be added or subtracted from a date to produce another date. The internal values for a timedelta are stored in days, seconds, and microseconds.
\begin{lstlisting}
import datetime

print "microseconds:", datetime.timedelta(microseconds=1)
print "milliseconds:", datetime.timedelta(milliseconds=1)
print "seconds     :", datetime.timedelta(seconds=1)
print "minutes     :", datetime.timedelta(minutes=1)
print "hours       :", datetime.timedelta(hours=1)
print "days        :", datetime.timedelta(days=1)
print "weeks       :", datetime.timedelta(weeks=1)

microseconds: 0:00:00.000001
milliseconds: 0:00:00.001000
seconds     : 0:00:01
minutes     : 0:01:00
hours       : 1:00:00
days        : 1 day, 0:00:00
weeks       : 7 days, 0:00:00
\end{lstlisting}
\end{frame}

%%%%%%%%%%%%%%%%%%%%%%%%%%%%%%%%%%%%%%%%%%%%%%%%%%%%%%%%%%%%%%%%%%%%%%%%%%%%%%%%%%%
\begin{frame}[fragile]\frametitle{ Date Arithmetic}
Date math uses the standard arithmetic operators. 
\begin{lstlisting}
import datetime

today = datetime.date.today()
print 'Today    :', today
one_day = datetime.timedelta(days=1)
print 'One day  :', one_day
yesterday = today - one_day
print 'Yesterday:', yesterday
tomorrow = today + one_day
print 'Tomorrow :', tomorrow
print 'tomorrow - yesterday:', tomorrow - yesterday
print 'yesterday - tomorrow:', yesterday - tomorrow

Today    : 2013-02-21
One day  : 1 day, 0:00:00
Yesterday: 2013-02-20
Tomorrow : 2013-02-22
tomorrow - yesterday: 2 days, 0:00:00
yesterday - tomorrow: -2 days, 0:00:00
\end{lstlisting}
\end{frame}

%%%%%%%%%%%%%%%%%%%%%%%%%%%%%%%%%%%%%%%%%%%%%%%%%%%%%%%%%%%%%%%%%%%%%%%%%%%%%%%%%%%
\begin{frame}[fragile]\frametitle{Comparing Values}
Both date and time values can be compared using the standard operators to determine which is earlier or later.
\begin{lstlisting}
import datetime
import time

print 'Times:'
t1 = datetime.time(12, 55, 0)
print '\tt1:', t1
t2 = datetime.time(13, 5, 0)
print '\tt2:', t2
print '\tt1 < t2:', t1 < t2

print 'Dates:'
d1 = datetime.date.today()
print '\td1:', d1
d2 = datetime.date.today() + datetime.timedelta(days=1)
print '\td2:', d2
print '\td1 > d2:', d1 > d2
\end{lstlisting}
\end{frame}


%%%%%%%%%%%%%%%%%%%%%%%%%%%%%%%%%%%%%%%%%%%%%%%%%%%%%%%%%%%%%%%%%%%%%%%%%%%%%%%%%%%
\begin{frame}[fragile]\frametitle{Comparing Values}
\begin{lstlisting}
Times:
        t1: 12:55:00
        t2: 13:05:00
        t1 < t2: True
Dates:
        d1: 2013-02-21
        d2: 2013-02-22
        d1 > d2: False
\end{lstlisting}
\end{frame}


%%%%%%%%%%%%%%%%%%%%%%%%%%%%%%%%%%%%%%%%%%%%%%%%%%%%%%%%%%%%%%%%%%%%%%%%%%%%%%%%%%%
\begin{frame}[fragile]\frametitle{Combining Dates and Times}
As with date, there are several convenient class methods to make creating datetime instances from other common values.
\begin{lstlisting}
import datetime

print 'Now    :', datetime.datetime.now()
print 'Today  :', datetime.datetime.today()
print 'UTC Now:', datetime.datetime.utcnow()

d = datetime.datetime.now()
for attr in [ 'year', 'month', 'day', 'hour', 'minute', 'second', 'microsecond']:
    print attr, ':', getattr(d, attr)
\end{lstlisting}
\end{frame}


%%%%%%%%%%%%%%%%%%%%%%%%%%%%%%%%%%%%%%%%%%%%%%%%%%%%%%%%%%%%%%%%%%%%%%%%%%%%%%%%%%%
\begin{frame}[fragile]\frametitle{Combining Dates and Times}
\begin{lstlisting}
Now    : 2013-02-21 06:35:45.658505
Today  : 2013-02-21 06:35:45.659381
UTC Now: 2013-02-21 11:35:45.659396
year : 2013
month : 2
day : 21
hour : 6
minute : 35
second : 45
microsecond : 659677
\end{lstlisting}
\end{frame}


%%%%%%%%%%%%%%%%%%%%%%%%%%%%%%%%%%%%%%%%%%%%%%%%%%%%%%%%%%%%%%%%%%%%%%%%%%%%%%%%%%%
\begin{frame}[fragile]\frametitle{Formatting and Parsing}
\begin{itemize}
\item The default string representation of a datetime object uses the ISO 8601 format (YYYY-MM-DDTHH:MM:SS.mmmmmm). 
\item Alternate formats can be generated using strftime(). 
\item Similarly, if your input data includes timestamp values parsable with time.strptime(), then datetime.strptime() is a convenient way to convert them to datetime instances.

\end{itemize}
\end{frame}

%%%%%%%%%%%%%%%%%%%%%%%%%%%%%%%%%%%%%%%%%%%%%%%%%%%%%%%%%%%%%%%%%%%%%%%%%%%%%%%%%%%
\begin{frame}[fragile]\frametitle{Formatting and Parsing}
\begin{lstlisting}
import datetime

format = "%a %b %d %H:%M:%S %Y"

today = datetime.datetime.today()
print 'ISO     :', today

s = today.strftime(format)
print 'strftime:', s

d = datetime.datetime.strptime(s, format)
print 'strptime:', d.strftime(format)

ISO     : 2013-02-21 06:35:45.707450
strftime: Thu Feb 21 06:35:45 2013
strptime: Thu Feb 21 06:35:45 2013
\end{lstlisting}
\end{frame}




