\documentclass[8pt,landscape]{article}
\usepackage{beamerarticle} % makes slides into article

%% xcolor Option Clash issue
%	Do not include xcolor,, tikz-qtree, todonotes, here do it after beamerarticle
\usepackage{multicol}
\usepackage{booktabs}
\usepackage{calc}
\usepackage{ifthen}
\usepackage[landscape]{geometry}
\usepackage{hyperref}
\usepackage{color}
% \usepackage{enumitem}
\usepackage{textcomp} 				% copyleft symbol
\usepackage{verbatim}
\usepackage{adjustbox} 				% for resizebox to adjust table figure content
% \usepackage{enumitem}				% margin free lists
\usepackage{amsmath}
\usepackage{mathrsfs}
\usepackage{csvsimple}				% importing csv as table
\usepackage{textcomp} 				% copyleft symbol
\usepackage{graphicx}
\usepackage{etoolbox} % conditional inclusions
\usepackage{algorithmic}
\usepackage{tikz-qtree} 

\usepackage{media9}
 \usepackage{multimedia}
 \usepackage{makecell}
 \usepackage{listings}
%  \usepackage{color}
\usepackage{algorithm,algorithmic}

 
\definecolor{codegreen}{rgb}{0,0.6,0}
\definecolor{codegray}{rgb}{0.5,0.5,0.5}
\definecolor{codepurple}{rgb}{0.58,0,0.82}
\definecolor{backcolour}{rgb}{.914, .89, .957} % pale purple

\definecolor{mygreen}{rgb}{0,0.6,0}
\definecolor{mygray}{rgb}{0.5,0.5,0.5}
\definecolor{mymauve}{rgb}{0.58,0,0.82}

\lstdefinestyle{mystyle}{
  backgroundcolor=\color{backcolour},   % choose the background color; you must add \usepackage{color} or \usepackage{xcolor}; should come as last argument
  basicstyle=\footnotesize\ttfamily,       % the size of the fonts that are used for the code
  breakatwhitespace=true,          % sets if automatic breaks should only happen at whitespace
  breaklines=true,                 % sets automatic line breaking
  captionpos=b,                    % sets the caption-position to bottom
  commentstyle=\color{mygreen},    % comment style
  deletekeywords={...},            % if you want to delete keywords from the given language
  escapeinside={\%*}{*)},          % if you want to add LaTeX within your code
  extendedchars=true,              % lets you use non-ASCII characters; for 8-bits encodings only, does not work with UTF-8
  frame=single,	                   % adds a frame around the code
  keepspaces=true,                 % keeps spaces in text, useful for keeping indentation of code (possibly needs columns=flexible)
  keywordstyle=\color{blue},       % keyword style
  language=Python,                 % the language of the code
  morekeywords={*,...},            % if you want to add more keywords to the set
  % numbers=left,                    % where to put the line-numbers; possible values are (none, left, right)
  % numbersep=5pt,                   % how far the line-numbers are from the code
  % numberstyle=\tiny\color{mygray}, % the style that is used for the line-numbers
  rulecolor=\color{black},         % if not set, the frame-color may be changed on line-breaks within not-black text (e.g. comments (green here))
  showspaces=false,                % show spaces everywhere adding particular underscores; it overrides 'showstringspaces'
  showstringspaces=false,          % underline spaces within strings only
  showtabs=false,                  % show tabs within strings adding particular underscores
  stepnumber=2,                    % the step between two line-numbers. If it's 1, each line will be numbered
  stringstyle=\color{codepurple},  % string literal style
  tabsize=2,	                   % sets default tabsize to 2 spaces
  columns=fullflexible,
  linewidth=0.98\linewidth,        % Box width
  aboveskip=10pt,	   			   % Space before listing 
  belowskip=5pt,	   			   % Space after listing  
  xleftmargin=.02\linewidth,  
  %title=\lstname                   % show the filename of files included with \lstinputlisting; also try caption instead of title
}


% \definecolor{codegreen}{rgb}{0,0.6,0}
% \definecolor{codegray}{rgb}{0.5,0.5,0.5}
% \definecolor{codepurple}{rgb}{0.58,0,0.82}
% %\definecolor{backcolour}{rgb}{0.95,0.95,0.92} % faint postman color
% \definecolor{backcolour}{rgb}{.914, .89, .957} % pale purple
% %\lstset{basicstyle=\footnotesize\ttfamily}

% \lstdefinestyle{mystyle}{
    % backgroundcolor=\color{backcolour},   
    % commentstyle=\color{codegreen},
    % keywordstyle=\color{magenta},
    % numberstyle=\tiny\color{codegray},
    % stringstyle=\color{codepurple},
    % basicstyle= \tiny\ttfamily %\scriptsize\ttfamily, %\footnotesize,  % the size of the fonts that are used for the code
    % breakatwhitespace=true,  % sets if automatic breaks should only happen at whitespace        
    % breaklines=true, % sets automatic line breaking   
    % linewidth=\linewidth,	
    % captionpos=b,                    
    % keepspaces=true,% keeps spaces in text, useful for keeping indentation                
% %    numbers=left,                  
    % numbers=none,  
% %    numbersep=5pt,                  
    % showspaces=false,                
    % showstringspaces=false,
    % showtabs=false,                  
    % tabsize=2
% }
\lstset{style=mystyle}


%\lstset{basicstyle=\footnotesize\ttfamily}

\newtoggle{VideoFrames}
\togglefalse{VideoFrames}


\newtoggle{CopyrightPictures}
\togglefalse{CopyrightPictures}

\hypersetup{ % remove ugly hyperlink boxes
    colorlinks,
    linkcolor={red!50!black},
    citecolor={blue!50!black}%,
    %urlcolor={green!80!black}
}

% This sets page margins to .5 inch if using letter paper, and to 1cm
% if using A4 paper. (This probably isn't strictly necessary.)
% If using another size paper, use default 1cm margins.
\ifthenelse{\lengthtest { \paperwidth = 8.5in}}
	{ \geometry{top=.2in,left=.25in,right=.25in,bottom=.5in} }
	{\ifthenelse{ \lengthtest{ \paperwidth = 290mm}}
		{\geometry{top=1cm,left=1cm,right=1cm,bottom=2cm} }
		{\geometry{top=1cm,left=1cm,right=1cm,bottom=2cm} }
	}

% Turn off header and footer
\pagestyle{empty}
 

% Redefine section commands to use less space
\makeatletter
\renewcommand{\section}{\@startsection{section}{1}{0mm}%
                                {-1ex plus -.5ex minus -.2ex}%
                                {0.5ex plus .2ex}%x
                                {\normalfont\large\bfseries}}
\renewcommand{\subsection}{\@startsection{subsection}{2}{0mm}%
                                {-1explus -.5ex minus -.2ex}%
                                {0.5ex plus .2ex}%
                                {\normalfont\normalsize\bfseries}}
\renewcommand{\subsubsection}{\@startsection{subsubsection}{3}{0mm}%
                                {-1ex plus -.5ex minus -.2ex}%
                                {1ex plus .2ex}%
                                {\normalfont\small\bfseries}}
\makeatother

% Define BibTeX command
\def\BibTeX{{\rm B\kern-.05em{\sc i\kern-.025em b}\kern-.08em
    T\kern-.1667em\lower.7ex\hbox{E}\kern-.125emX}}

% Don't print section numbers
\setcounter{secnumdepth}{0}

\newcommand{\code}[1]{\par\vskip0pt plus 1filll \footnotesize Code:~\itshape#1}
\date{}

\setlength{\parindent}{0pt}
\setlength{\parskip}{0pt plus 0.5ex}
\setlength\columnsep{30pt}

\lstdefinelanguage{markdown}{
    basicstyle=\ttfamily,
    sensitive=false
}