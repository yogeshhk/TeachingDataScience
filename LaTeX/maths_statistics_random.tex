%%%%%%%%%%%%%%%%%%%%%%%%%%%%%%%%%%%%%%%%%%%%%%%%%%%%%%%%%%%%%%%%%%%%%%%%%%%%%%%%%%
\begin{frame}[fragile]\frametitle{}
\begin{center}
{\Large Random Variables}
\end{center}
\end{frame}

%%%%%%%%%%%%%%%%%%%%%%%%%%%%%%%%%%%%%%%%%%%%%%%%%%%%%%%%%%
\begin{frame}\frametitle{Random Variables}
\begin{itemize}
\item Those variables which can take different value randomly are called
random variables.
\item If the variables are discrete in nature, they are called discrete random
variables.
\item Similarly, if the variables are continuous in nature, then it is called
continuous random variable.
\end{itemize}
\end{frame}



%%%%%%%%%%%%%%%%%%%%%%%%%%%%%%%%%%%%%%%%%%%%%%%%%%%%%%%%%%
\begin{frame} \frametitle{Random variables}
A {\bf random variable} is generated by a ``random'' function. 

Types:
\begin{itemize}
\item Nominal
\item Ordinal
\item Quantitative
\end{itemize}
\end{frame}


%%%%%%%%%%%%%%%%%%%%%%%%%%%%%%%%%%%%%%%%%%%%%%%%%%%%%%%%%%
\begin{frame}\frametitle{Nominal Random Variable}
\begin{itemize}
\item A {\bf nominal} random variable, also called {\bf qualitative} or {\bf  categorical}, takes on values in a set of names or labels.
\item Examples of nominal sample spaces include geographical locations, biological species, and ethnic categories.
\item Data from a nominal random variable cannot be {\bf strictly ordered}.
\end{itemize}
\end{frame}

%%%%%%%%%%%%%%%%%%%%%%%%%%%%%%%%%%%%%%%%%%%%%%%%%%%%%%%%%%
\begin{frame}\frametitle{Ordinal Random Variable}

\begin{itemize}
\item An {\bf ordinal} measurement is usually either a {\bf ranking}
($1^{\rm st}$, $2^{\rm nd}$, etc.) or a {\bf rating} (good, bad,
favorable, strongly agree, etc.).  

\item Ordinal values can be ordered, but do not have {\bf measurement units}
\item It doesn't always make sense to subtract or divide their numerical
values.
\end{itemize}
\end{frame}

%%%%%%%%%%%%%%%%%%%%%%%%%%%%%%%%%%%%%%%%%%%%%%%%%%%%%%%%%%
\begin{frame}\frametitle{Quantitative Random Variable}

\begin{itemize}
\item A {\bf quantitative} random variable represents a magnitude, for
example the result of counting something, or the result of measuring a
physical quantity such as length, width, or volume.

\item Quantitative measurements can always be ordered.
\end{itemize}
\end{frame}

%%%%%%%%%%%%%%%%%%%%%%%%%%%%%%%%%%%%%%%%%%%%%%%%%%%%%%%%%%%
%\begin{frame}
%\frametitle{Measurement scales}
%
%\begin{itemize}
%\item Celsius temperature is an interval scale -- it makes
%perfect sense to compare two temperatures by subtracting them.
%\item Is Celsius temperature a ratio scale?  Is the coffee twice as hot as
%the tea?
%\item 
%The answer is no.  The freezing point of water defines the origin of
%the Celsius scale.  This is arbitrary.  Pure ethanol freezes at
%$-114^\circ$C, so if we used the freezing point of ethanol as the zero
%point of the scale, the ratio between the two objects' temperatures
%would be only $(50+114)/(25+114)=1.2$.
%
%\textcolor{blue}{Note:} There is some ambiguity and controversy about
%these designations.
%
%\end{frame}

%%%%%%%%%%%%%%%%%%%%%%%%%%%%%%%%%%%%%%%%%%%%%%%%%%%%%%%%%%%
\begin{frame}
\frametitle{ Quantitative Random Variable }
Types:
\begin{itemize}
\item Discrete: Number of days that it rains yearly 
\item Continuous:  Amount of rain on a given day 
\end{itemize}
Note that these designations aren't used in the case of nominal or
ordinal random variables.
\end{frame}

%%%%%%%%%%%%%%%%%%%%%%%%%%%%%%%%%%%%%%%%%%%%%%%%%%%%%%%%%%%
\begin{frame}
\frametitle{ Quantitative Random Variable }

\begin{itemize}
\item A {\bf discrete} random variable takes on finitely many values, or
infinitely many separated values (e.g.\ $0,1,2,\ldots$).

\item A {\bf continuous} random variable can take on any real number in some
interval (e.g.\ $[0,1]$ or $[0,\infty)$).  

\item Gray area between discrete and continuous random variables:
e.g.\ monetary values recorded in dollars or dollars and cents
\end{itemize}
\end{frame}

%%%%%%%%%%%%%%%%%%%%%%%%%%%%%%%%%%%%%%%%%%%%%%%%%%%%%%%%%%%
\begin{frame}
\frametitle{Units, populations and samples}

\begin{itemize}
\item A {\bf unit} (more specifically {\bf ``sampling unit''}) is one member
of the collection.
\item A {\bf population} is the set of all units of interest. 
\item In practice, we cannot observe the whole population, so we usually
work with a {\bf sample} of units
\end{itemize}
\end{frame}

%%%%%%%%%%%%%%%%%%%%%%%%%%%%%%%%%%%%%%%%%%%%%%%%%%%%%%%%%%%
\begin{frame}
\frametitle{Units, populations and samples}

%\textcolor{blue}{\bf Example:} Suppose we are interested in childrens'
%diets, and we sample 100 seven year old
%girls in the US and record their food intake for one day.

\begin{itemize}
\item Properties of a sample will generally differ from properties of the
population.  
\item For example, the 100 children in our sample may have
somewhat higher or somewhat lower average calorie intake than the
population at large. 
\item 
Statistics: how samples relate to
populations, and what we can confidently state about a population
given the limited information in a sample.

\end{itemize}

\end{frame}

%%%%%%%%%%%%%%%%%%%%%%%%%%%%%%%%%%%%%%%%%%%%%%%%%%%%%%%%%%
\begin{frame}\frametitle{Random Variables}
\begin{itemize}
\item  Random variable can take on randomly different
value. 
\item But are all the values that it takes can be equal or likely be equal
too, or is it more likely that the random variable take a particular value
more often than other?
\item Depends on the way (function) using which random numbers are generated.
\end{itemize}
\end{frame}

%%%%%%%%%%%%%%%%%%%%%%%%%%%%%%%%%%%%%%%%%%%%%%%%%%%%%%%%%%
\begin{frame}\frametitle{Random Variable Generation}
\begin{itemize}
\item  Experiment: records numbers from a SINGLE six face die.
\item X axis shows 6 possible outcomes.
\item Y axis shows their probabilities.
\item For each of the 6 possible values on X axis, the Y value is same ie 1/6
\item Uniform distribution
\end{itemize}
\end{frame}

%%%%%%%%%%%%%%%%%%%%%%%%%%%%%%%%%%%%%%%%%%%%%%%%%%%%%%%%%%
\begin{frame}\frametitle{Random Variable Generation}
\begin{itemize}
\item  Experiment: records sum of numbers from TWO six face dies.
\item X axis shows 12 possible outcomes.
\item Y axis shows their probabilities.
\item For each of the 12 possible values on X axis, the Y value is NOT SAME.
\end{itemize}

\begin{center}
\includegraphics[width=0.5\linewidth,keepaspectratio]{pmf}
\end{center}
\end{frame}

%%%%%%%%%%%%%%%%%%%%%%%%%%%%%%%%%%%%%%%%%%%%%%%%%%%%%%%%%%
\begin{frame}\frametitle{Random Variable Generation}
\begin{itemize}
\item  If the random variable is discrete in nature, we use Probability Mass
Function to describe its probability distribution
\item If the random variable is continuous in nature, we use Probability
Density Function to describe its probability distribution.
\end{itemize}
\end{frame}
