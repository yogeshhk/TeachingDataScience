%%%%%%%%%%%%%%%%%%%%%%%%%%%%%%%%%%%%%%%%%%%%%%%%%%%%%%%%%%%%%%%%%%%%%%%%%%%%%%%%%%
\begin{frame}[fragile]\frametitle{}
\begin{center}
{\Large Concepts}
\end{center}
\end{frame}

%%%%%%%%%%%%%%%%%%%%%%%%%%%%%%%%%%%%%%%%%%%%%%%%%%%%%%%%%%%
\begin{frame}\frametitle{Core Concepts: Natural Language Understanding (NLU)}
      \begin{itemize}
        \item Interpreting user's intended meaning from natural language query
        \item Tokenization: Breaking down queries into individual words/tokens
        \item Intent recognition: Identifying the user's goal
        \item Entity extraction: Detecting key information relevant to the database
        \item Context understanding: Resolving ambiguities in language
        \item Coreference resolution: Linking pronouns to their antecedents
        \item Critical for handling complex, colloquial language in real-world scenarios
      \end{itemize}
\end{frame}

%%%%%%%%%%%%%%%%%%%%%%%%%%%%%%%%%%%%%%%%%%%%%%%%%%%%%%%%%%%
\begin{frame}\frametitle{Core Concepts: SQL Syntax and Database Schema}
      \begin{itemize}
        \item SQL Syntax: Rules governing structure of SQL queries (SELECT, FROM, WHERE)
        \item SQL Semantics: Logical operations determining how queries interact with data
        \item Database Schema: Blueprint of data structure
          \begin{itemize}
            \item Tables, columns, and their relationships
            \item Primary and foreign keys
            \item Constraints on data
          \end{itemize}
        \item Schema representation techniques
          \begin{itemize}
            \item Table and column names
            \item Natural language descriptions
            \item Explicit relationship definitions
          \end{itemize}
      \end{itemize}
\end{frame}

%%%%%%%%%%%%%%%%%%%%%%%%%%%%%%%%%%%%%%%%%%%%%%%%%%%%%%%%%%%
\begin{frame}\frametitle{Traditional Approaches to Text-to-SQL}
      \begin{itemize}
        \item Keyword Matching
          \begin{itemize}
            \item Maps keywords in query to database schema elements
            \item Limited to basic queries with direct term correspondence
            \item Struggles with synonyms and context
          \end{itemize}
        \item Rule-Based Systems
          \begin{itemize}
            \item Predefined linguistic and database-specific rules
            \item Effective in limited domains with predictable queries
            \item Lacks flexibility for complex or varied expressions
          \end{itemize}
        \item Syntax-Based Translation
          \begin{itemize}
            \item Parses query into syntactic structure
            \item Maps to SQL syntax using grammatical rules
            \item Struggles with semantic gaps between natural language and SQL
          \end{itemize}
      \end{itemize}
\end{frame}

%%%%%%%%%%%%%%%%%%%%%%%%%%%%%%%%%%%%%%%%%%%%%%%%%%%%%%%%%%%
\begin{frame}\frametitle{The Generative AI Revolution in Text-to-SQL}
      \begin{itemize}
        \item Large Language Models (LLMs) transform query generation capabilities
          \begin{itemize}
            \item GPT-3/4, Gemini, Llama models show remarkable SQL generation
            \item Leverage knowledge from vast training data
            \item Superior at understanding context, nuances, and relationships
          \end{itemize}
        \item Contextual understanding through self-attention mechanisms
        \item Can interpret ambiguous queries and prompt for clarification
        \item Handles business jargon and organization-specific terminology
        \item Generates complex SQL with joins, filters, aggregations, and advanced features
        \item Better transferability across different database schemas
      \end{itemize}
\end{frame}

%%%%%%%%%%%%%%%%%%%%%%%%%%%%%%%%%%%%%%%%%%%%%%%%%%%%%%%%%%%
\begin{frame}\frametitle{Enhancing LLM Performance for Text-to-SQL}
      \begin{itemize}
        \item Prompt Engineering
          \begin{itemize}
            \item Carefully crafting instructions and context
            \item Including schema details and example query pairs
          \end{itemize}
        \item Fine-tuning
          \begin{itemize}
            \item Further training pre-trained LLMs on text-to-SQL datasets
            \item Adapting to domain-specific requirements
          \end{itemize}
        \item Retrieval-Augmented Generation (RAG)
          \begin{itemize}
            \item Retrieves relevant schema information and examples
            \item Incorporates retrieved context before generating SQL
            \item Improves accuracy and context-awareness
          \end{itemize}
        \item Task decomposition for complex queries
        \item Chain-of-thought prompting to encourage step-by-step reasoning
      \end{itemize}
\end{frame}

%%%%%%%%%%%%%%%%%%%%%%%%%%%%%%%%%%%%%%%%%%%%%%%%%%%%%%%%%%%
\begin{frame}\frametitle{LLM Comparison for Text-to-SQL}
      \begin{itemize}
        \item GPT-3 \& GPT-4 (OpenAI)
          \begin{itemize}
            \item Strong natural language understanding and code generation
            \item Excellent handling of complex queries
            \item Can be prone to hallucinations
          \end{itemize}
        \item Gemini (Google)
          \begin{itemize}
            \item Multimodal capabilities
            \item Strong context understanding
            \item Relatively newer compared to GPT models
          \end{itemize}
        \item Llama2, Mixtral \& Code Llama (Meta)
          \begin{itemize}
            \item Open-source options
            \item Can be fine-tuned for specific tasks
            \item May require more fine-tuning than closed-source models
          \end{itemize}
      \end{itemize}
\end{frame}
