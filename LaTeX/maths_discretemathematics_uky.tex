%%%%%%%%%%%%%%%%%%%%%%%%%%%%%%%%%%%%%%%%%%%%%%%%%%%%%%%%%%%
 \begin{frame}[fragile] \frametitle{Notes from Avinash Sathaye}
https://http://www.ms.uky.edu/~sohum/ma162/fa\_09/lectures/
\end{frame}

%%%%%%%%%%%%%%%%%%%%%%%%%%%%%%%%%%%%%%%%%%%%%%%%%%%%%%%%%%%%%%%%%%%%%%%%%%%%%%%%%%
 \begin{frame}[fragile]\frametitle{}
\begin{center}
{\Large Lectures on Sets}

\end{center}
\end{frame}

\begin{frame} %2

  \frametitle{Set Notations.}
 \begin{itemize}%[<+-| alert@+>]  
\item \mbl{Sets}  are collections of objects.
For convenience and to avoid some logical problems, we always agree
to a universal set $U$ and understand that all sets under discussion
are contained in it.

\item An empty set is denoted by the symbol $\emptyset$.
A set can be described in essentially one of two ways:
\begin{itemize}
\item 
An explicit list \mbl{(roster notation)} as in $A=\{1,2,3\}$
\item or  by a defining property \mbl{(set builder notation)}  
$$A = \{x ~|~ x \mbox{ is one of the first three natural numbers.} \}.$$
\end{itemize}

\item 
We can build new sets from given sets, say $A,B$  by repeated application
of one of these operations:
$$\mbl{\mbox{ Union }} A \bigcup B \mbl{\mbox{ Intersection }} A \bigcap B
\mbl{\mbox{ Subtraction }}
A \setminus B.$$

\end{itemize}
\pause 
\end{frame}

%2
%


\begin{frame}%3
  \frametitle{Union}
  \begin{itemize}%[<+-| alert@+>]

\item
Let $U=\{1,2,3,4,5,6,7,8\}$ and $A=\{1,2,3\}, B=\{2,3,4,5,6\}$.
We define and illustrate the various set operations.

\item \mbl{$A\bigcup B$} is the set obtained by putting elements of $A$
as well as $B$ together. By convention, repeated elements are written
only once.

So $A\bigcup B = \{ 1,2,3,4,5,6\}$.

\item It is worth noting that  $X\bigcup U=U$ and $X\bigcup
\emptyset = X$ for all sets $X$.

\item We can also take the union of several sets and the final result
does not depend on the order of taking the unions.

In fancy words, this says that the union is an associative and  commutative
binary operation.

Thus, if $C=\{2,4,8\}$, then
$$A\bigcup B \bigcup C = \{1,2,3,4,5,6,8\}.$$


\end{itemize}
\pause
\end{frame}
%3
%


\begin{frame}%4
  \frametitle{Intersection and Subtraction.}
  \begin{itemize}%[<+-| alert@+>]

\item
Let $U=\{1,2,3,4,5,6,7,8\}$ and $A=\{1,2,3\}$, $B=\{2,3,4,5,6\}$.
We define and illustrate the various set operations.

\item \mbl{$A\bigcap B$} is defined as the set of common elements of $A$
and $B$. Thus, for our example, it gives $A\bigcap B = \{2,3\}$.

\item \mbl{ $A \setminus B$ } is defined by listing elements of $A$  \mbl{
except for  } elements of $B$.
This may also be \mbl{ written as $A-B$.}

Thus, for our example, \mbl{$A\setminus B = \{1\}.$}
Note that  by the same definition \mbl{$B\setminus A = \{4,5,6\}.$}

\item The set $U\setminus A$ is also called \mbl{ the complement} of $A$
and has a special notation $A^C$.
Thus, for our example $$A^C = \{1,2,3,4,5,6,7,8\} \setminus
\{1,2,3\} = \{4,5,6,7,8\}.$$


\end{itemize}
\pause
\end{frame}

%4
%


\begin{frame}%5
  \frametitle{Calculations in Set Builder notation.}
  \begin{itemize}%[<+-| alert@+>]
 
\item For finite sets, the lister notation is easy to work with and set
operations are easily handled. For large sets or infinite sets, one has
to use the set builder notation and the calculation of set operations
also has to be understood in the same way. Here is an illustration.

\item Let $U$ be the set of all integers $\{ \cdots,
-3,-2,-1,0,1,2,3,\cdots \}.$

Let
$$A = \{ x ~|~ x=2t \mbox{ for some $t\in U$} \},
B= \{x ~|~ x=5t \mbox{ for some $t\in U$} \}.$$

\item We claim the following:
$$A \bigcup B = \{ x ~|~ x = 2t \mbl{\mbox{ or }} x=5t  \mbox{ for some $t\in
U$}\}.$$

A little thought shows that the set consists of all even numbers as well
as all elements of $\{\pm 5, \pm 15, \pm 25, \pm 35 ,\cdots \}$.

\end{itemize}
\pause
\end{frame}

%5
%


\begin{frame}%6
  \frametitle{Continued Set Builder Notation.}
  \begin{itemize}%[<+-| alert@+>]
 
\item With the above definition of $U,A,B$ we see that
$$A\bigcap B = \{x ~|~ x = 2t \mbl{\mbox{ and }} x=5t  \mbox{ for some $t\in
U$}\}.$$

 This set is easily seen to be all multiples of $10$.

\item
The complement $A^C$ is easily seen as the set of all odd integers and
can be described as $\{2t+1~|~ t\in U\}$.

\item Also:
$$A\setminus  B = \{x ~|~ x = 2t \mbl{\mbox{ and }} \mbl{x\neq 5t}
\mbox{ for \mbl{any} $t\in U$}\}.$$

These are all even integers which don't end in a zero!
\end{itemize}
\pause
\end{frame}

%5
%

\begin{frame}%7
  \frametitle{Venn Diagrams.}
  \begin{itemize}%[<+-| alert@+>]

\item Often a graphical technique helps us understand the set operations and
helps in deducing properties of calculated sets.

\item If one is working with two sets $A,B$,  then it is easy to see
that any combination of operations between them results in a set
which is a union of some of the following sets in a Venn diagram:
\centerline{\pict{1.2}{1.2}{venn1_lec12.jpg}}

\item We shall find it convenient to number them so that
the set $1$ is $A\setminus B$, set $2$ is $B\setminus A$, set $3$ is
$A\bigcap B$ and set $0$ is $(A \bigcup B)^C$.

\item This is useful to prove properties of two set combinations.


\end{itemize}
\pause
\end{frame}

%6
%


\begin{frame}%7
  \frametitle{Using the Two-set Venn Diagram.}
  \begin{itemize}%[<+-| alert@+>]
 

\item
We can use the set numbering to ``prove theorems'' about two sets.
Thus, we show how to prove:

$$(A\bigcup B)^C = A^C \bigcap B^C.$$

\item Note from the Venn diagram that $A\bigcup B$ is composed of the pieces
$1,2,3$ so its complement is the piece $0$. This says that the set on
the left hand side is the $0$ piece.

\item From the Venn diagram, we also see that $A^C$ and $B^C$ are
composed of sets ``$2$, $0$'' and ``$1$, $0$'' respectively. Thus, there
intersection is also the piece ``$0$''.

\item The same technique can be used to prove or disprove any statements
about two sets.


\end{itemize}
\pause
\end{frame}

%7
%


\begin{frame}%8
 \frametitle{Counting Formula 1.}
  \begin{itemize}%[<+-| alert@+>]
 
\item 
For any \mbl{finite set $X$}, let $n(X)$ denote the number of elements
in $X$. For two sets $A,B$ we denote by $d_1,d_2,d_3,d_0$ the respective
number of elements in the Venn diagram pieces $1,2,3,0$.

\item Then from the Venn diagram, we note the following:
\item $$n(A) = d_1+d_3 ,~~~ n(B)=d_2+d_3.$$
$$n(A\bigcup B) = d_1+d_2+d_3,~~~ n(A\bigcap B) = d_3$$

\item It follows that
$$ n(A) + n(B) - n(A\bigcap B) = (d_1+d_3)+(d_2+d_3) - (d_3) =d_1+d_2+d_3.$$
Hence:
$$n(A\bigcup B) = n(A) + n(B) - n(A\bigcap B).$$


\item This is one of the important counting formula that we shall use.

\end{itemize}
\pause
\end{frame}

%8
%




\begin{frame}%9
 \frametitle{Venn Diagram of three sets.}
  \begin{itemize}%[<+-| alert@+>]
 
\item
Here is  similar idea to handle three sets $A,B,C$.
\begin{tabular}{cl}
\pict{1.2}{1.2}{venn2_lec12.jpg}
 &
\begin{minipage}[b]{2in}
Note the numbering of the pieces from $0$ to $7$.
Here is a sample theorem:
\mbl{$$A\bigcup(B\bigcap C) =(A\bigcup B)\bigcap (A\bigcup C).$$}
\vspace*{\fill}
\end{minipage}\\
\end{tabular}
\item The second part on LHS has pieces $4,7$ and the first part has
pieces $1,5,6,7$, so the union is composed of $1,4,5,6,7$.


\item On the RHS, the two parts are respectively composed of
$1,5,6,7,2,4$ and $1,5,6,7,3,4$. So the intersection gives,
$1,5,6,7,4$ which is the same as the LHS.

\item This proves the theorem!


\end{itemize}
%\pause
\end{frame}

%9
%


\begin{frame}%10
  \frametitle{Counting Formula 3.}
  \begin{itemize}%[<+-| alert@+>]

\item
We can make a three set counting formula using the Venn diagram as
before.

\item The formula is:
$n(A \bigcup B \bigcup C) = n(A)+n(B) + n(C) -n(A\bigcap B)-n(B\bigcap
C)-n(C\bigcap A) + n(A\bigcap B \bigcap C).$

\item Using a notation similar to the previous calculation, the LHS is
clearly the sum of $d_1+d_2+d_3+d_4+d_5+d_6+d_7$.
\item The first three terms on the RHS give the  sum
\mbl{ $d_1+d_2+d_3+d_4+d_5+d_6+d_7 + (d_4+d_5+d_6+2d_7)$.}
\item The next three terms on the RHS subtract off
$d_6+d_7+d_4+d_7+d_5+d_7 =d_4+d_5+d_6+3d_7 $, so we are now left with
\mbl{ $d_1+d_2+d_3+d_4+d_5+d_6+d_7 + (-d_7)$.}
\item The last term on RHS is exactly $d_7$ and adding it matches the
LHS. This finishes the proof!

\end{itemize}
%\pause
\end{frame}
%10
%

\begin{frame}%11
  \frametitle{More on Venn Diagrams.}
  \begin{itemize}%[<+-| alert@+>]
\item
It is tempting to hope that we can have Venn diagrams to handle $4$ or
more sets. However, this does not work. If you have four sets, then you
need $16$ pieces  and these cannot be made by intersecting four circles!

\item So, for four or more sets, one has to learn how to use use the set
builder notation carefully and resolve set operations.

\item As a result, some people prefer to do the two and three set
calculations also using formulas and set builder notation.

\item The two basic counting formulas that one should remember are:
$$n(A\bigcup B) = \mbl{n(A) + n(B)} -\mgr{ n(A\bigcap B)}$$
and
$$
\begin{array}{lcl}
n(A \bigcup B \bigcup C) & = & \mbl{n(A)+n(B) + n(C)} \\
& & -\mgr{(n(A\bigcap B)+n(b\bigcap C)+n(C\bigcap A))}\\
& &  + \mbl{n(A\bigcap B \bigcap C)}
\end{array}
.$$


\end{itemize}
%\pause
\end{frame}

\begin{frame} %2

  \frametitle{Product Sets.}
 \begin{itemize}%[<+-| alert@+>]  
\item One of the important skills about sets is to be able to count the
number of elements in a set. Quite often, a set is a table of
information.

\item Given sets $A,B$ we can form the product set
$$A\times B ~=~ \{(a,b) ~|~ a\in A, b\in B\}.$$

\item A little thought shows that if $A,B$ are finite, then:
$$n(A\times B) = n(A)\cdot n(B).$$

\item For example, we can have a class roll consisting of student names
followed by letters giving the class grade. It will be convenient to
consider the pair   (Student Name, Grade) and call it an assigned grade.


\item So, if $S$ is the set of all students and $G$ is the set
$\{A,B,C,D,E,W\}$ then the assigned grades in a semester form
the set $S\times G$.

For a class of $180$ students, we get  $n(S\times G)=180\cdot 6=1080$.

\end{itemize}
%\pause 
\end{frame}

%2
%


\begin{frame}%3
  \frametitle{Multiple Products.}
  \begin{itemize}%[<+-| alert@+>]



\item To handle the whole grade book, we can let $N$ be the set of integers
between  $0$ and $100$ and consider the product set
$$S\times N\times N\times N\times N\times G \mbox{ or }
S\times N^4 \times G$$
whose members are 6-tuples consisting of student name followed by a
sequence of four exam scores followed by the final grade.

\item Naturally, we have \mbl{an extended product formula}
$$n(A_1 \times A_2 \times \cdots A_r) = n(A_1)\cdot n(A_2)\cdots
n(A_r).$$

\item We now illustrate how we can count the number of elements in
various sets using this formula.

\item For the same class of $180$ students, we get
$$n(S\times N^4 \times G) = 180\cdot 100^4 \cdot 6 = 108,000,000,000.$$



\end{itemize}
%\pause
\end{frame}
%3
%


\begin{frame}%4
  \frametitle{Counting Specific Sets.}
  \begin{itemize}%[<+-| alert@+>]
\item The above number only gives the possible student records for a
class of $180$. The actual class records are, of course only $180$,
since there is exactly one per student.

We will often need to identify and count specific subsets of the full
product sets in this way.

\item We cast a die $5$ times and note the number on top, which would be
a member of the set $S=\{1,2,3,4,5,6\}$. If we record the $5$ castings
and record the top numbers in order, then we get members of the product
set $S^5$ which has $6^5=7776$ elements.

\item A telephone company assigns nine digit telephone numbers. How many
different phones can if handle?

We can imagine the telephone numbers as members of the set $T^9$ where
$T=\{0,1,2,3,4,5,6,7,8,9\}$. Thus $n(T)=10$, so $n(T^9)=10^9 =
1,000,000,000$.



\end{itemize}
%\pause
\end{frame}

%4
%


\begin{frame}%5
  \frametitle{Further Counting.}
  \begin{itemize}%[<+-| alert@+>]
 
\item Suppose we now require that the first digit of a telephone number
\mgr{cannot be $0$ or $5$}, then we get a modified count $8\cdot 10^8$ or
$800,000,000$.

\item This is typical. We often start restricting the elements of a
product set and count the new subset.

\item As another example, if besides the beginning $0$ or $5$, we also
disallow all numbers with the same digit repeated $9$-times, then we get
to disallow these $10$ numbers $000,000,000,
111,111,111,\cdots , 999,999,999$.

\item Is the new answer $800,000,000 -10$?
A little thought will show that it is actually
$800,000,000 -8 = 799,999,992$.

\end{itemize}
%\pause
\end{frame}

%5
%


\begin{frame}%6
  \frametitle{More Restricted Products: Permutations.}
  \begin{itemize}%[<+-| alert@+>]
 
\item A natural question about the telephone numbers can be the
following. Suppose, we wish to count those telephone number which do not
repeat any digits. We could try and calculate the numbers with some
repeated digits and try to subtract them off.

\item A little thought will show that this is not practical. There are
too many ways to repeat some digit and it would be hard to keep track of
double counting.
\mbl{A better strategy is the following:}
Imagine the permissible numbers as sequences of digits to be filled in.
\mbl{The very first digit} can be any one of the 10 digits in $T$.
\mbl{The second} can now be one of $9$, since the first digit is now used up!
\mbl{The third} has now $8$ choices and continuing, the last (ninth)
digit has only two choices left.
\item Thus, the total count is $10\cdot 9 \cdot 8 \cdot 7\cdot 6 \cdot
5\cdot 4 \cdot 3 \cdot 2 = 3,628,800$.

\item We end up with a much smaller answer!
\end{itemize}
%\pause
\end{frame}

%6
%



\begin{frame}%7
  \frametitle{Permutations Continued.}
  \begin{itemize}%[<+-| alert@+>]

\item
In general, many problems can be solved by this idea of filling in slots
as we did above. The problem can be formulated as asking

\item
\mbl{``in how many ways, can we seat $n$ people in $r$ chairs''}
or in a more neutral language, `
\mgr{`in how many ways can we arrange $n$
objects in $r$ positions''?}

\item The general answer is:
$$P(n,r) = n \cdot (n-1) \cdot (n-2) \cdots (n-r+1) .$$

It is instructive to try various values of $r$ and check this out!

\item It is also clear that $r\leq n$ for otherwise the number is zero
and there is no solution. \mbl{After all $4$ people cannot fill up $5$
chairs!}

\item Using the factorial notation, we can conveniently rewrite this as
$P(n,r) = \frac{n!}{(n-r)!}$.

\mbl{This formula should be memorized.}


\end{itemize}
%\pause
\end{frame}

%7
%
\begin{frame}%7a
  \frametitle{Circular permutations.}
  \begin{itemize}%[<+-| alert@+>]
\item Sometimes the problem is to seat a number of people in a circle
rather than in a row of chairs.

\item We can ask for the number of ways of arranging some $r$ of $n$
people at a circular table with $r$ chairs.

\item The simplest way to do this problem is to note that if the chairs
are in a row, we know the answer $P(n,r)$. Now we move the chairs in a
circle. We note that $r$ different arrangements in a row can give the
same arrangement in a circle since shifting everybody to the right in
the circle does not give a new arrangement.

\item So, the simple formula for the circular arrangement is:
$$\frac{P(n,r)}{r} = \frac{n!}{(n-r)!r}.$$

  
\end{itemize}
%\pause
\end{frame}

%7a
%


\begin{frame}%8
  \frametitle{Set Problems.}
  \begin{itemize}%[<+-| alert@+>]
 
\item
Given that $A, B$ and $C$ are sets with $94, 67$ and $84$
members respectively, answer the following.

\item  If $B\bigcap A$ has $45$ members, then $B\bigcup A$ has
$\_\_\_$   members.

{\bf Answer:} Use the formula
$$n(B\bigcup A) = n(B) + n(A) - n(B\bigcap A) = 94+67-45 = 116.$$

\item If it is further known that $C\bigcap A$ has $57$ members, then
$C\bigcup A$ has $\_\_\_$   members.

\item
{\bf Answer:} By an identical calculation, we get: $94+84-57=121$.

\item If, in addition $B-C$ has $40$ members, then $C\bigcap B$ has 
$\_\_\_$   members.

\item {\bf Answer.} This time, we note that the sets $B-C$ and
$C\bigcap B$ have no common elements and their union is $B$.
So, we get
$$n(B) = 67 = n(B-C) + n(C\bigcap B) = 40+ n(C\bigcap B).$$
Hence $n(C\bigcap B) = 67-40 =27$.



\end{itemize}
%\pause
\end{frame}

%8
%

\begin{frame}%9
 \frametitle{Some Counting Problems}
  \begin{itemize}%[<+-| alert@+>]
 
\item 
Finally, if we are given that  the intersection of all three
sets $A, B,$ and $C$  has $17$ members, then the union of all three sets has 
$\_\_\_$   members.

\item {\bf Answer:}
Using the three set formula:
$n(A \bigcup B \bigcup C) = n(A)+n(B) + n(C) -n(A\bigcap B)-n(B\bigcap
C)-n(C\bigcap A) + n(A\bigcap B \bigcap C)$

and the above calculations, we get:
\item 
$n(A \bigcup B \bigcup C) = 94+67+84-(45+57+27)+17 = 133$.





\end{itemize}
%\pause
\end{frame}

%9
%





\begin{frame}%10
 \frametitle{Combinations.}
  \begin{itemize}%[<+-| alert@+>]
 
\item Let us revisit our formula for permutations. Consider a problem of
selecting a delegation of three students from a class of $25$ students
to go to a state convention.

\item We could try to count the number of possible delegations thus.
Set up three chairs and seat the students randomly in them one after the
other.

\item As we saw before, the possible number of such selections appears
to be $25\cdot 24 \cdot 23 = 13,800$.
\mbl{Do remember this as $P(25,3)=\frac{25!}{22!}$}.

\item But we need to think some more. Once the team is selected, the
order does not matter. A little thought shows that the same $3$-student
team can appear in $3!=6$ different ways in our selections, depending on
the order of choosing.

\item Since we are only interested in the team and not the order, the
correct answer  should be $13,800/6 = 2,300$.



\end{itemize}
%\pause
\end{frame}

%10
%


\begin{frame}%11
  \frametitle{More Combinations.}
  \begin{itemize}%[<+-| alert@+>]
\item We record the formula for choosing $r$ element sets out of $n$
objects as:
$$C(n,r) = \frac{n!}{(n-r)!r!} =
\frac{n(n-1)\cdots (n-r+1)}{r(r-1)\cdots (1)}.$$

\item This is also described as the number of ways of choosing $r$
objects out of $n$ objects.

\item We gave two forms of the formula. While the first is easy to
remember, the second is easier to work with especially if $r<(n-r)$.

\item Here are some observations and hints.

\item $C(n,r)=C(n,n-r)$. Just check the first formula.

Alternatively, think thus: Choosing $r$ objects from $n$ is the same as
choosing $(n-r)$ for rejection!

\item $C(n+1,r) = C(n,r)+c(n,r-1)$. Use simple algebraic manipulation.


\end{itemize}
%\pause
\end{frame}
%11
%

\begin{frame}%12
  \frametitle{Binomial Theorem.}
  \begin{itemize}%[<+-| alert@+>]
\item \mbl{Binomial Theorem:}
$$(1+x)^n = 1 + C(n,1)x+C(n,2)x^2+\cdots +C(n,r)x^r+\cdots +x^n.$$
\item {\bf Idea of the proof.} Think of $(1+x)^n$ as product of $n$
terms $(1+x)$. To get a term $x^r$ out of this product, we simply have
to choose $x$ from $r$ of the terms and $1$ from the remaining $(n-r)$
terms. Hence the term $x^r$ must occur $C(n,r)$ times!

\item Note that the above expression suggests why $C(n,0)=C(n,n)=1$.

\item Also, we must clearly have $C(n,r)=0$ if $r>n$, since we cannot
choose $r>n$ objects from among the $n$ objects.

\item $C(n,r)$ gives the number of ways of splitting an $n$-element set
into two pieces - an $r$ element set and an $(n-r)$ element set. We can
ask for the number of ways of splitting into three sets of sizes
$a,b,(n-a-b)$. 


\end{itemize}
%\pause
\end{frame}
%


%12

\begin{frame}%13
  \frametitle{Multinomial Theorm.}
  \begin{itemize}%[<+-| alert@+>]

\item A similar argument can show that the answer is:
$$C(n,a,b)=\frac{n!}{(n-a-b)!a!b!}.$$

\item There is a corresponding multinomial theorem:
$$(1+x+y)^n = \sum C(n,a,b)x^ay^b .$$

\item We can thus state the Multinomial Theorem:
$$(1+x+y)^n = \sum C(n,a,b)x^ay^b.$$

For example:
$$(1+x+y)^3 = 1+3\,x+3\,y+3\,{x}^{2}+6\,xy+3\,{y}^{2}+{x}^{3}+3\,{x}^{2}y+3\,x{y}^{2
}+{y}^{3}.$$

Verify these terms.

\item {\bf Food for thought!} Make suitable definitions and prove the
\mbl{full Multinomial Theorem.}

$$(1+x_1+x_2+\cdots x_r )^n = \sum C(n,a_1,a_2,\cdots
,a_r)x_1^{a_1}x_2^{a_2}\cdots x_r^{a_r}.$$




\end{itemize}
%\pause
\end{frame}

%13


\begin{frame} %2
\frametitle{Counting formulas.}
  \begin{itemize}%[<+-| alert@+>]

\item {\bf Q.1.}
Given that $A, B$ and $C$ are sets with $95, 69$ and $85$ members respectively,

If $B\bigcap A$  has $44$ members, then $B\bigcup A$ has  ???  members.

\item {\bf Answer:} Use the two set formula
$n(B \bigcup A) = n(B) + n(A) - n(B\bigcap A) = 69 + 95 - 44 = 120.$


\item
ii) If it is further known that $C\bigcap A$ has $57$ members, then $C\bigcup A$ has   ??? members.

\item {\bf Answer:} Similar formula gives: $n(C\bigcup A) = n(C) + n(A)
- n(c\bigcap A) = 85 + 95 - 57 = 123$.

\item 
iii) If, in addition $B-C$ has $39$ members, then $C\bigcap B$ has ???  members.

\item Note that $n(B) = n(B-C) + n(C\bigcap B)$, so $69 = 39 + n(C\bigcap B)$.
Hence the answer is $30$.




\end{itemize}

\end{frame}

%2
%


\begin{frame}%3
  \frametitle{Q.1. continued.}
  \begin{itemize}%[<+-| alert@+>]

\item iv) Finally, if we are given that  the intersection of all three sets $A, B$ and $C$
has $17$ members, then the union of all three sets has  ??? members.
\item {\bf Answer:} Apply the three set formula and numbers known from
above to get:
$n(A \bigcup B \bigcup C) = 95+69+85 -(44+30+57) + 17 = 135.$




\end{itemize}
%\pause
\end{frame}
%3
%



\begin{frame}%4
  \frametitle{Question 2.}
  \begin{itemize}%[<+-| alert@+>]
\item {\bf Q.2.}
Your favorite restaurant, Wally's Fine Dining, has a dinner menu with 6 appetizers,
11 entrees, and 8 desserts.
A dinner at Wally's consists of 1 appetizer, 1 entree, and 1 dessert.
What is the largest number of  days could you eat dinner at Wally's without ever
ordering the exact same meal?

\item
{\bf Answer:}
Imagine a choice card with three boxes marked appetizer, entree and
dessert.

They can be respectively filled with $6,11,8$ choices and by
multiplication principle, the answer is the product: $(6)(11)(8) = 528$.


\end{itemize}
%\pause
\end{frame}

%4
%

\begin{frame}%5
  \frametitle{Further Counting.}
  \begin{itemize}%[<+-| alert@+>]
\item {\bf Q.3.}
Manjula is extremely fashionable; so she can't stand wearing the same outfit
twice to one job.  She owns 6 shirts, 3 pairs of pants, and 9 pairs of shoes.
If she works 270 days at her current job, how many more shirts must she get to have
enough so that she will never have to wear the same outfit twice?

\item
{\bf Answer:} Suppose she buys $x$ new shirts. Then by multiplication
principle, she would be good for $(6+x)(3)(9) = 162 + 27x$ days.

\item We want this
answer to become at least $270$. Clearly $x=4 $ works!
$(6+4)(3)(9)=270$. 

\end{itemize}
%\pause
\end{frame}

%5
%



\begin{frame}%6
  \frametitle{Question 4.}
  \begin{itemize}%[<+-| alert@+>]
 
\item {\bf Q.4.} Three friends are playing a murder mystery game.
To win, they must correctly guess which suspect murdered someone,
what room they committed the crime in, and what weapon was used to do it.
One of the friends starts guessing combinations (without repeating any he already guessed).
What is the maximum number of guesses this friend must make to win the game
if there are 7 possible weapons, 4 possible suspects, and 5 possible rooms to commit the murder in?

\item {\bf Amswer:}
The problem is a simple multiplication principle again, so
$(7)(4)(5)=140$.

\end{itemize}
%\pause
\end{frame}

%6
%



\begin{frame}%7
  \frametitle{Question 5.}
  \begin{itemize}%[<+-| alert@+>]

\item {\bf Q.5.}

In a local dog show, there are prizes for first, second, and third place.
If there are 14 dogs
competing, how many ways are there to award the prizes (assuming no dog can win two prizes)?

\item {\bf Answer:}
A naive guess is that it is a simple multiplication problem, make three
boxes marked No. 1, 2 3 and fill in names of one othe 14 dogs in them in
different ways.

\item
Howveer, a little thought shows that while the No. 1 box has $14$
choices, the No. 2 box can be filled in only $13$ ways and the No. 3 box
has only $12$ choices left. The answer is, therefore $(14)(13)(12)=
2184$.


\end{itemize}
%\pause
\end{frame}

%7
%


\begin{frame}%8
  \frametitle{Question 6.}
  \begin{itemize}%[<+-| alert@+>]
\item {\bf Q.6.}
Commercial radio station call signs (i.e. WZZQ) are assigned following certain rules.
The first letter must be either "W" or "K" and the last three letters can be anything
as long as they aren't all the same.  How many possible radio station call signs are possible?

\item {\bf Answer:}
We imagine four slots to be filled with letters. The first slot  has only $2$
choices $W,K$.

We ignore the restriction for a minute and count the choices for the
remaining slots as $26$ each. So this tentative answer is $(2)26^3$.

\item Bringing in the restriction, we realize that for each starting
letter ($W$ or $K$) we throw out $26$ slots choices like $AAA$, $BBB$
etc. Thus the number of discards is $(2)(26)$.

Final answer $(2)(26^3) - (2)(26) = 35100$.
      

\end{itemize}
%\pause
\end{frame}

%8
%



\begin{frame}%8a
  \frametitle{Q. 6 continued.}
  \begin{itemize}%[<+-| alert@+>]
 
\item
               
How many different radio station call signs can you have with the first letter W
and the second letter different from B?

\item {\bf Answer:} We follow the same idea. The first slot has a single
choice $W$, the second has $25$ (other than $B$ and if we ignore the
restriction, then we have $26$ each for the last slots.
This gives $(25)(26^2)$. We subtract off the count of the signs $WAAA,
WCCC, WDDD, \cdots $ etc and this number is $25$.

Final answer is: $25(26^2) - 25 = 16875$.


\end{itemize}
%\pause
\end{frame}

%8a
%


\begin{frame}%9
 \frametitle{Question 7.}
  \begin{itemize}%[<+-| alert@+>]
 
\item {\bf Q.7.}

Some prisoners are trying to make a jail break.
To open the door to their freedom, they have to put a series of 10 switches
in the one configuration that will open the door.
Each switch can be either "OFF" or "ON" .(All of them are currently "OFF"
and the door is locked.)  It takes them 1 minute to test each new configuration.
If they have 17 hours 1 minutes before they are discovered missing and the alarm sounds,
are they guaranteed to open the door before the alarm sounds?
\item
In either case, calculate the exact number of spare minutes left when they open the
door or the number of minutes they are short by: Ans: 
        

\item {\bf Answer:} First we compute the number of choices they have to make. There vare $10$
switches and each has two choices, so the answer is $2^10=1024$.




               


\end{itemize}
%\pause
\end{frame}

%9
%





\begin{frame}%10
 \frametitle{Questions 7,8.}
  \begin{itemize}%[<+-| alert@+>]

\item {\bf Q.7 continued.}

But note that one of these is already noticed and is a wrong choices, so
there are $1023$ choices left.

They have $(17)(60)+1 = 1021$ minutes, so they cannot be guaranteed to
escape.

they fall short by $2$ minutes!


\item {\bf Q.8.}
License plates in a certain state all have the same format.

They all have 3 numbers  followed by 2 letters.

How many license plates can the state issue before being forced to reuse numbers
from older plates?

\item {\bf Answer:}
This is simple  multiplication principle problem:

There are five slots to fill with $10$ choices each for the first three
slots and $26$ each for the last two.

So the answer is $(10^3)(26^2) = 676000 $.



\end{itemize}
%\pause
\end{frame}

%10


\begin{frame}%11
  \frametitle{More Combinations.}
  \begin{itemize}%[<+-| alert@+>]
\item We record the formula for choosing $r$ element sets out of $n$
objects as:
$$C(n,r) = \frac{n!}{(n-r)!r!} =
\frac{n(n-1)\cdots (n-r+1)}{r(r-1)\cdots (1)}.$$

\item This is also described as the number of ways of choosing $r$
objects out of $n$ objects.

\item We gave two forms of the formula. While the first is easy to
remember, the second is easier to work with especially if $r<(n-r)$.

\item Here are some observations and hints.

\item $C(n,r)=C(n,n-r)$. Just check the first formula.

Alternatively, think thus: Choosing $r$ objects from $n$ is the same as
choosing $(n-r)$ for rejection!

\item $C(n+1,r) = C(n,r)+c(n,r-1)$. Use simple algebraic manipulation.


\end{itemize}
%\pause
\end{frame}
%11
%

\begin{frame}%12
  \frametitle{Binomial Theorem.}
  \begin{itemize}%[<+-| alert@+>]
\item \mbl{Binomial Theorem:}
$$(1+x)^n = 1 + C(n,1)x+C(n,2)x^2+\cdots +C(n,r)x^r+\cdots +x^n.$$
\item {\bf Idea of the proof.} Think of $(1+x)^n$ as product of $n$
terms $(1+x)$. To get a term $x^r$ out of this product, we simply have
to choose $x$ from $r$ of the terms and $1$ from the remaining $(n-r)$
terms. Hence the term $x^r$ must occur $C(n,r)$ times!

\item Note that the above expression suggests why $C(n,0)=C(n,n)=1$.

\item Also, we must clearly have $C(n,r)=0$ if $r>n$, since we cannot
choose $r>n$ objects from among the $n$ objects.

\item $C(n,r)$ gives the number of ways of splitting an $n$-element set
into two pieces - an $r$ element set and an $(n-r)$ element set. We can
ask for the number of ways of splitting into three sets of sizes
$a,b,(n-a-b)$. 


\end{itemize}
%\pause
\end{frame}
%


%12

\begin{frame}%13
  \frametitle{Multinomial Theorm.}
  \begin{itemize}%[<+-| alert@+>]

\item A similar argument can show that the answer is:
$$C(n,a,b)=\frac{n!}{(n-a-b)!a!b!}.$$

\item There is a corresponding multinomial theorem:
$$(1+x+y)^n = \sum C(n,a,b)x^ay^b .$$

\item We can thus state the Multinomial Theorem:
$$(1+x+y)^n = \sum C(n,a,b)x^ay^b.$$

For example:
$$(1+x+y)^3 = 1+3\,x+3\,y+3\,{x}^{2}+6\,xy+3\,{y}^{2}+{x}^{3}+3\,{x}^{2}y+3\,x{y}^{2
}+{y}^{3}.$$

Verify these terms.

\item {\bf Food for thought!} Make suitable definitions and prove the
\mbl{full Multinomial Theorem.}

$$(1+x_1+x_2+\cdots x_r )^n = \sum C(n,a_1,a_2,\cdots
,a_r)x_1^{a_1}x_2^{a_2}\cdots x_r^{a_r}.$$




\end{itemize}
%\pause
\end{frame}


