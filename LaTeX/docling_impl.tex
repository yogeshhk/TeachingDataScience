%%%%%%%%%%%%%%%%%%%%%%%%%%%%%%%%%%%%%%%%%%%%%%%%%%%%%%%%%%%%%%%%%%%%%%%%%%%%%%%%%%
\begin{frame}[fragile]\frametitle{}
\begin{center}
{\Large Building a Production RAG Pipeline with Docling}

\end{center}
\end{frame}

%%%%%%%%%%%%%%%%%%%%%%%%%%%%%%%%%%%%%%%%%%%%%%%%%%%%%%%%%%%
\begin{frame}[fragile]\frametitle{Overview: RAG with Docling and LlamaIndex}
      \begin{itemize}
          \item This implementation demonstrates Retrieval-Augmented Generation (RAG) using Docling for document processing
          \item Combines document parsing, vector storage, and language model querying in a single pipeline
          \item Uses ChromaDB as vector store and HuggingFace models for embeddings and text generation
          \item Docling specializes in parsing complex PDF documents while preserving structure and metadata
          \item LlamaIndex provides the orchestration layer for building the RAG pipeline
          \item The system can answer questions about PDF content by retrieving relevant passages and generating responses
      \end{itemize}
\end{frame}

%%%%%%%%%%%%%%%%%%%%%%%%%%%%%%%%%%%%%%%%%%%%%%%%%%%%%%%%%%%
\begin{frame}[fragile]\frametitle{Import Dependencies and Setup}
      \begin{itemize}
          \item Import essential libraries for document processing, vector storage, and language models
          \item Set up environment variables and warning filters for clean execution
          \item Load environment variables from .env file for API tokens
          \item Configure tokenizer parallelism to avoid threading conflicts
          \item Import specialized readers and parsers for document processing
      \end{itemize}
      
\begin{lstlisting}[language=Python, basicstyle=\tiny]
import os
from pathlib import Path
from tempfile import mkdtemp
from warnings import filterwarnings
import torch
from dotenv import load_dotenv
from llama_index.embeddings.huggingface import HuggingFaceEmbedding
:
from llama_index.vector_stores.chroma import ChromaVectorStore

load_dotenv()

filterwarnings(action="ignore", category=UserWarning)
os.environ["TOKENIZERS_PARALLELISM"] = "false"
\end{lstlisting}
\end{frame}

%%%%%%%%%%%%%%%%%%%%%%%%%%%%%%%%%%%%%%%%%%%%%%%%%%%%%%%%%%%
\begin{frame}[fragile]\frametitle{Embedding Model Configuration}
      \begin{itemize}
          \item Configure HuggingFace embedding model for converting text to vector representations
          \item BAAI/bge-m3 is a popular choice for semantic similarity tasks, supports multi-linguality and multi-modality
          \item Small model size balances performance with resource requirements
          \item Embeddings enable semantic search by measuring similarity in vector space
          \item This model supports English text and produces 384-dimensional vectors
          \item The embedding model is crucial for the retrieval component of RAG
      \end{itemize}
      
\begin{lstlisting}[language=Python, basicstyle=\tiny]
EMBED_MODEL = HuggingFaceEmbedding(
    model_name="BAAI/bge-m3",  # 384-dim
    # Consider: "sentence-transformers/all-MiniLM-L6-v2" (384-dim)
    # or "BAAI/bge-base-en-v1.5" (768-dim, better quality)
)
\end{lstlisting}
\end{frame}

%%%%%%%%%%%%%%%%%%%%%%%%%%%%%%%%%%%%%%%%%%%%%%%%%%%%%%%%%%%
\begin{frame}[fragile]\frametitle{Language Model Setup}
      \begin{itemize}
          \item Configure local language model using HuggingFace Transformers
          \item Local execution avoids API rate limits and ensures data privacy
          \item Context window of 1024 tokens 
          \item Temperature and top-p parameters control creativity, coherence
          \item Padding configuration ensures proper token handling during generation
      \end{itemize}
      
\begin{lstlisting}[language=Python, basicstyle=\tiny]
from llama_index.llms.huggingface import HuggingFaceLLM

GEN_MODEL = HuggingFaceLLM(
    model_name="google/gemma-2-9b-it",  # or "microsoft/phi-3-mini-4k-instruct"
    tokenizer_name="google/gemma-2-9b-it",
    context_window=8192,  # Updated
    max_new_tokens=512,
    model_kwargs={
        "torch_dtype": torch.bfloat16,  # Use bfloat16 for better performance on modern GPUs
        "trust_remote_code": True
    },
    device_map="auto",  # Better than "cpu"
)
\end{lstlisting}
\end{frame}

%%%%%%%%%%%%%%%%%%%%%%%%%%%%%%%%%%%%%%%%%%%%%%%%%%%%%%%%%%%
\begin{frame}[fragile]\frametitle{Document Source and Query Definition}
      \begin{itemize}
          \item Define the PDF document to be processed and indexed
          \item Specify the query that will be answered using the RAG system
          \item Document path points to a technical report about NVIDIA and AI models
          \item Query focuses on identifying main AI models mentioned in the document
          \item This setup demonstrates domain-specific question answering capabilities
          \item The system will retrieve relevant sections and generate comprehensive answers
      \end{itemize}
      
\begin{lstlisting}[language=Python, basicstyle=\tiny]
SOURCE = "data/NVIDIAAn.pdf"  # Docling Technical Report
QUERY = "What is NVIDIA's outlook for the first quarter of fiscal 2026?"
\end{lstlisting}
\end{frame}

%%%%%%%%%%%%%%%%%%%%%%%%%%%%%%%%%%%%%%%%%%%%%%%%%%%%%%%%%%%
\begin{frame}[fragile]\frametitle{Document Processing Components}
      \begin{itemize}
          \item Initialize DoclingReader for intelligent PDF parsing and content extraction
          \item SemanticSplitterNodeParser converts document structure into processable nodes
          \item Docling preserves document layout, tables, and hierarchical structure
          \item Node parsing breaks documents into semantic chunks for better retrieval
          \item Each node contains text content plus metadata about source location
          \item This preprocessing step is crucial for effective information retrieval
      \end{itemize}
      
\begin{lstlisting}[language=Python, basicstyle=\tiny]
reader = DoclingReader()
node_parser = SemanticSplitterNodeParser()
\end{lstlisting}
\end{frame}

%%%%%%%%%%%%%%%%%%%%%%%%%%%%%%%%%%%%%%%%%%%%%%%%%%%%%%%%%%%
\begin{frame}[fragile]\frametitle{Vector Store Setup with ChromaDB}
      \begin{itemize}
          \item Initialize ChromaDB as an in-memory vector database for storing document embeddings
          \item EphemeralClient stores vectors in memory for fast access during the session
          \item Create a collection named "quickstart" to organize the vector embeddings
          \item ChromaVectorStore provides LlamaIndex integration with ChromaDB
          \item Vector store enables similarity search across document chunks
          \item In-memory storage is suitable for development and small-scale applications
      \end{itemize}
      
\begin{lstlisting}[language=Python, basicstyle=\tiny]
# Add persistent storage option
import chromadb
from chromadb.config import Settings

# Option 1: In-memory (existing)
chroma_client = chromadb.EphemeralClient()

# Option 2: Persistent storage (ADD THIS)
chroma_client = chromadb.PersistentClient(
    path="./chroma_db",
    settings=Settings(anonymized_telemetry=False)
)
\end{lstlisting}
\end{frame}

%%%%%%%%%%%%%%%%%%%%%%%%%%%%%%%%%%%%%%%%%%%%%%%%%%%%%%%%%%%
\begin{frame}[fragile]\frametitle{Vector Index Creation and Document Processing}
      \begin{itemize}
          \item Create VectorStoreIndex from documents using the configured components
          \item DoclingReader loads and parses the PDF while preserving structure
          \item SemanticSplitterNodeParser transforms content into searchable nodes
          \item StorageContext integrates the ChromaDB vector store
          \item Embedding model converts text chunks into vector representations
          \item This step builds the searchable knowledge base for the RAG system
      \end{itemize}
      
\begin{lstlisting}[language=Python, basicstyle=\tiny]
index = VectorStoreIndex.from_documents(
    documents=reader.load_data(SOURCE),
    transformations=[node_parser],
    storage_context=StorageContext.from_defaults(
        vector_store=vector_store
    ),
    embed_model=EMBED_MODEL,
)
\end{lstlisting}
\end{frame}

%%%%%%%%%%%%%%%%%%%%%%%%%%%%%%%%%%%%%%%%%%%%%%%%%%%%%%%%%%%
\begin{frame}[fragile]\frametitle{Query Execution and Results Display}
      \begin{itemize}
          \item Convert the vector index into a query engine with the language model
          \item Execute the predefined query against the indexed document content
          \item Query engine performs similarity search to find relevant document chunks
          \item Language model generates a coherent response based on retrieved context
          \item Display both the generated answer and the source nodes for transparency
          \item Source nodes show which parts of the document informed the answer
      \end{itemize}
      
\begin{lstlisting}[language=Python, basicstyle=\tiny]
result = index.as_query_engine(llm=GEN_MODEL).query(QUERY)

print(f"Q: {QUERY}")
print(f"A: {result.response.strip()}")
print(f"\nSources:")
print([(n.text, n.metadata) for n in result.source_nodes])
\end{lstlisting}
\end{frame}

%%%%%%%%%%%%%%%%%%%%%%%%%%%%%%%%%%%%%%%%%%%%%%%%%%%%%%%%%%%
\begin{frame}[fragile]\frametitle{RAG Pipeline Architecture Summary}
      \begin{itemize}
          \item Document Processing: Docling extracts and structures PDF content
          \item Chunking: SemanticSplitterNodeParser creates semantic text segments
          \item Embedding: HuggingFace model converts chunks to vector representations
          \item Storage: ChromaDB stores vectors for efficient similarity search
          \item Retrieval: Query embeddings find most relevant document chunks
          \item Generation: Language model synthesizes answers from retrieved context
          \item Transparency: System returns source information for answer verification
          \item This architecture enables accurate, contextual responses from document content
      \end{itemize}
\end{frame}