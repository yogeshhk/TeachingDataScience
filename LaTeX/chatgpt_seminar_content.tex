%%%%%%%%%%%%%%%%%%%%%%%%%%%%%%%%%%%%%%%%%%%%%%%%%%%%%%%%%%%%%%%%%%%%%%%%%%%%%%%%%%
\begin{frame}[fragile]\frametitle{}
\begin{center}
{\Large Overview of ChatGPT}
\end{center}
\end{frame}

%%%%%%%%%%%%%%%%%%%%%%%%%%%%%%%%%%%%%%%%%%%%%%%%%%%%%%%%%%%
\begin{frame}[fragile]\frametitle{What is ChatGPT?}

ChatGPT is powered by a Large Language Model (GPT-3.5) optimized for 
dialogue, developed by OpenAI

\begin{itemize}
\item Link: https://chat.openai.com/chat
\item Details: https://openai.com/blog/chatgpt/
\end{itemize}	 

\begin{center}
\includegraphics[width=0.6\linewidth,keepaspectratio]{chatgpt33}
\end{center}		
		
{\tiny (Ref: OpenAI, creator of ChatGPT, casts spell on Microsoft
- The Jakarta Post}

\end{frame}

%%%%%%%%%%%%%%%%%%%%%%%%%%%%%%%%%%%%%%%%%%%%%%%%%%%%%%%%%%%
\begin{frame}[fragile]\frametitle{What is a Language Models?}

\begin{itemize}
\item While typing SMS, have you seen it suggests next word?
\item While typing email, have you seen next few words are suggested?
\item How does it suggest? (suggestions are not random, right?)
\item In the past, for ``Lets go for a \ldots', if you have typed 'coffee' 15 times, 'movie' say 4 times, then it learns that. Machine/Statistical Learning.
\item Next time, when you type ``Lets go for a '', what will be suggested? why?
\item This is called Language Model. Predicting the next word. When done continuously, one after other, it spits sentence, called Generative Model.
\end{itemize}	

\begin{center}
\includegraphics[width=0.6\linewidth,keepaspectratio]{chatgpt34}
\end{center}		

\end{frame}

%%%%%%%%%%%%%%%%%%%%%%%%%%%%%%%%%%%%%%%%%%%%%%%%%%%%%%%%%%%
\begin{frame}[fragile]\frametitle{Evolution of Language Models}

Language Models can be statistical (frequency based) or Machine/Deep Learning (supervised) based. Simple to complex.

\begin{center}
\includegraphics[width=\linewidth,keepaspectratio]{chatgpt30}
\end{center}				
{\tiny (Ref: Analytics Vidhya https://editor.analyticsvidhya.com/uploads/59483evolution\_of\_NLP.png)}

\end{frame}

%%%%%%%%%%%%%%%%%%%%%%%%%%%%%%%%%%%%%%%%%%%%%%%%%%%%%%%%%%%
\begin{frame}[fragile]\frametitle{Large Language Models - Comparison}

\begin{center}
\includegraphics[width=\linewidth,keepaspectratio]{chatgpt31}
\end{center}				
{\tiny (Ref: Deus.ai https://www.deus.ai/post/gpt-3-what-is-all-the-excitement-about)}

\end{frame}



%%%%%%%%%%%%%%%%%%%%%%%%%%%%%%%%%%%%%%%%%%%%%%%%%%%%%%%%%%%
\begin{frame}[fragile]\frametitle{What is ChatGPT?}


\begin{itemize}
\item A Chatbot
\item A language model, that takes 'prompt' from user and generates a response.
\item Built by OpenAI
\item Released in Nov 2022
\item Got 1m users in 5 days  (Insta 2.5 months, Spotify 5m, Facebook 10m, Netflix 3.5yrs)
\end{itemize}	 

\begin{center}
\includegraphics[width=0.6\linewidth,keepaspectratio]{chatgpt8}
\end{center}				
{\tiny (Ref: ChatGPT - Explained! - CodeEmporium)}
			
			
\end{frame}

%%%%%%%%%%%%%%%%%%%%%%%%%%%%%%%%%%%%%%%%%%%%%%%%%%%%%%%%%%%
\begin{frame}[fragile]\frametitle{What is ChatGPT?}


\begin{itemize}
\item GPT based
\item Trained on large amount of data
\item Uses Supervised Learning and Reinforcement Learning.
\item Gives answers like a human
\item Can ask Follow-up questions and even admit mistakes.
\end{itemize}	 

\end{frame}


%%%%%%%%%%%%%%%%%%%%%%%%%%%%%%%%%%%%%%%%%%%%%%%%%%%%%%%%%%%
\begin{frame}[fragile]\frametitle{Who are these guys?}

Open AI:
\begin{itemize}
\item San Francisco-based artificial intelligence company
\item Famous for its well-known DALL·E, a deep-learning model that generates images from text instructions called prompts.
\item OpenAI Inc. is the non-profit parent company of the for-profit OpenAI LP.
\item Initially supported by Elon Musk
\item The CEO is Sam Altman, who previously was president of Y Combinator.
\item Microsoft is a partner and investor in the amount of \$1 billion dollars. They jointly developed the Azure AI Platform.
\item Mission: To ensure that artificial general intelligence benefits all of humanity
\item Prevent misuse of AI
\end{itemize}	 

\begin{center}
\includegraphics[width=0.4\linewidth,keepaspectratio]{chatgpt25}
\end{center}				
{\tiny (Ref: https://www.slideegg.com/open-ai-chat-gpt)}


\end{frame}

%%%%%%%%%%%%%%%%%%%%%%%%%%%%%%%%%%%%%%%%%%%%%%%%%%%%%%%%%%%
\begin{frame}[fragile]\frametitle{How to Use ChatGPT?}


\begin{center}
\includegraphics[width=\linewidth,keepaspectratio]{chatgpt26}
\end{center}				
{\tiny (Ref: https://www.slideegg.com/open-ai-chat-gpt)}


\end{frame}


%%%%%%%%%%%%%%%%%%%%%%%%%%%%%%%%%%%%%%%%%%%%%%%%%%%%%%%%%%%
\begin{frame}[fragile]\frametitle{GPTs Series}


\begin{itemize}
\item GPT + Generative Pre-trained Transformers
\item GPT-1, GPT-2 and GPT-3 are similar in terms of architecture 
\item Differ on the data and the number of transformer blocks with the number of incoming tokens.   
\item GPT-1 : 12 blocks, encoded using a Byte pair encoding, 512 tokens: 117 million parameters 
\item GPT-2 : 48 blocks, 1024 tokens: 1.5 billion parameters 
\item GPT-3 : 96 blocks, 2048 tokens: 175 billion parameters 
\end{itemize}	 

\end{frame}

%%%%%%%%%%%%%%%%%%%%%%%%%%%%%%%%%%%%%%%%%%%%%%%%%%%%%%%%%%%
\begin{frame}[fragile]\frametitle{GPTs Trainings}


\begin{itemize}
\item GPT-1 is trained in a self-supervised manner (learn to predict the next word in text data) and fine-tuned in a supervised learning manner. 
\item GPT-2 is trained in a fully self supervised way, focusing on zero-shot transfer
\item  GPT-3 is pre-trained in a self supervised manner exploring a bit more the few-shots fine-tuning.
\item GPT-1 is pre-trained on the BooksCorpus dataset, containing ~7000 books amounting to ~5GB of data
\item GPT-2 is pre-trained using the WebText dataset which is a more diverse set of internet data containing ~8M documents for about ~40 GB of data
\item GPT-3 uses an expanded version of the WebText dataset, two internet-based books corpora that are not disclosed and the English-language Wikipedia which constituted ~600 GB (45TB?) of data
\end{itemize}	 

\end{frame}

%%%%%%%%%%%%%%%%%%%%%%%%%%%%%%%%%%%%%%%%%%%%%%%%%%%%%%%%%%%
\begin{frame}[fragile]\frametitle{GPT3 vs ChatGPT}


\begin{center}
\includegraphics[width=0.8\linewidth,keepaspectratio]{chatgpt27}
\end{center}		

\tiny{(Ref:ChatGPT Explained: Complete A-Z Guide - Kripesh Adwani)}
\end{frame}

%%%%%%%%%%%%%%%%%%%%%%%%%%%%%%%%%%%%%%%%%%%%%%%%%%%%%%%%%%%
\begin{frame}[fragile]\frametitle{GPT Training: Overview}


Task: predicting next word

\begin{center}
\includegraphics[width=0.8\linewidth,keepaspectratio]{chatgpt13}
\end{center}		


{\tiny (Ref: How GPT3 Works - Easily Explained with Animations - Jay Alammar)}

\end{frame}

%%%%%%%%%%%%%%%%%%%%%%%%%%%%%%%%%%%%%%%%%%%%%%%%%%%%%%%%%%%
\begin{frame}[fragile]\frametitle{GPT Inferencing: Overview}


GPT being a decoder only model, takes words vectors (does this conversion internally), then through set of Decoder blocks, to spit out the next word one by one

\begin{center}
\includegraphics[width=0.8\linewidth,keepaspectratio]{chatgpt14}
\end{center}		


{\tiny (Ref: How GPT3 Works - Easily Explained with Animations - Jay Alammar)}

\end{frame}

%%%%%%%%%%%%%%%%%%%%%%%%%%%%%%%%%%%%%%%%%%%%%%%%%%%%%%%%%%%
\begin{frame}[fragile]\frametitle{Misalignment Issue}


\begin{itemize}
\item GPT-3 was trained to predict the next word. Not much of context there.
\item Introducing prompting : provide with certain samples and context

\end{itemize}	 

			\begin{center}
			\includegraphics[width=0.6\linewidth,keepaspectratio]{chatgpt2}
			
			{\tiny "Example of misalignment between text to be predicted and final use"}
			\end{center}		
			
			
			{\tiny (Ref: ChatGPT: training process, advantages, and limitations - By Sergio Soage, Machine Learning Engineer at Aivo)}
			
\end{frame}

%%%%%%%%%%%%%%%%%%%%%%%%%%%%%%%%%%%%%%%%%%%%%%%%%%%%%%%%%%%
\begin{frame}[fragile]\frametitle{Curated Training Data}


\begin{itemize}
\item To fix this misalignment, humans must be involved in teaching GPT, and in this way, GPT will be able to generate better questions as it evolves.
\item Reinforcement Learning based on Human Input (rewards if aligns) was leveraged.
\end{itemize}	 

\begin{center}
\includegraphics[width=0.6\linewidth,keepaspectratio]{chatgpt3}

\end{center}		

{\tiny (Ref: ChatGPT: training process, advantages, and limitations - By Sergio Soage, Machine Learning Engineer at Aivo)}
			
\end{frame}


%%%%%%%%%%%%%%%%%%%%%%%%%%%%%%%%%%%%%%%%%%%%%%%%%%%%%%%%%%%
\begin{frame}[fragile]\frametitle{Prerequisites: What is Reinforcement Learning (RL)?}


\begin{itemize}
\item RL is a method of achieving a goal via rewards. Find best path to get maximum points (game below)
\item Agent: Makes the moves
\item Reward: Positive or Negative (as shown)
\item State: Current position, location.
\item Action: Move done by the Agent
\item Policy: Sequence of Actions that Agent takes to achieve the goal, say, Down-Down-Right-Right-Right (reward 6)..there could be other paths and one(or more) of then can be the best.
\end{itemize}	 


\begin{center}
\includegraphics[width=0.6\linewidth,keepaspectratio]{chatgpt9}
\end{center}		


{\tiny (Ref: ChatGPT - Explained! - CodeEmporium)}
\end{frame}

%%%%%%%%%%%%%%%%%%%%%%%%%%%%%%%%%%%%%%%%%%%%%%%%%%%%%%%%%%%
\begin{frame}[fragile]\frametitle{RL in context of ChatGPT}


\begin{itemize}
\item Agent: the model, it spits out words
\item Reward: score at predicting each next word
\item State: prompt + sentence generated so far
\item Action: Predicting the next word
\item Policy: Sequence of words and then total rewards for them.
\item Manual person can chose between such sequences (or polices) and label which is best. Then PPO can happen to adjust backwardly the reward function.
\item So, the Reinforcement Learning is used to capture human preferences and reduce misalignment.
\end{itemize}	 
\end{frame}

%%%%%%%%%%%%%%%%%%%%%%%%%%%%%%%%%%%%%%%%%%%%%%%%%%%%%%%%%%%
\begin{frame}[fragile]\frametitle{Better GPT 3.5}

\begin{itemize}
\item ChatGPT is slight deviation to InstructGPT model (GPT 3.5)
\item InstructGPT Paper: ``Training language models to follow instructions with human feedback''
\end{itemize}	 

\begin{center}
\includegraphics[width=0.8\linewidth,keepaspectratio]{chatgpt1}
\end{center}		

\tiny{(Ref: https://openai.com/blog/chatgpt/)}
\end{frame}

%%%%%%%%%%%%%%%%%%%%%%%%%%%%%%%%%%%%%%%%%%%%%%%%%%%%%%%%%%%
\begin{frame}[fragile]\frametitle{Step 1}

\begin{itemize}
\item Have a prompts dataset 
\item Let human laborer get the correct response for them
\item Do fine-tuning of GPT 3.5 using this data via supervised learning (SFT). 
\item Simple but very expensive.
\end{itemize}	 

			\begin{center}
			\includegraphics[width=0.8\linewidth,keepaspectratio]{chatgpt4}
			
			\end{center}		
			
			{\tiny (Ref: ChatGPT: training process, advantages, and limitations - By Sergio Soage, Machine Learning Engineer at Aivo)}
			
\end{frame}

%%%%%%%%%%%%%%%%%%%%%%%%%%%%%%%%%%%%%%%%%%%%%%%%%%%%%%%%%%%
\begin{frame}[fragile]\frametitle{Step 2}

\begin{itemize}
\item For a prompt, generate multiple responses from the fine-tuned model, one after another.
\item Human feedback provides a ranking of responses
\item This data is used to build Rewards Model (RM)
\end{itemize}	 

			\begin{center}
			\includegraphics[width=0.8\linewidth,keepaspectratio]{chatgpt5}
			
			\end{center}		
			
			{\tiny (Ref: ChatGPT: training process, advantages, and limitations - By Sergio Soage, Machine Learning Engineer at Aivo)}
			
\end{frame}


% %%%%%%%%%%%%%%%%%%%%%%%%%%%%%%%%%%%%%%%%%%%%%%%%%%%%%%%%%%%
% \begin{frame}[fragile]\frametitle{Step 2: Manual Ranking}


			% \begin{center}
			% \includegraphics[width=\linewidth,keepaspectratio]{chatgpt10}
			
			% \end{center}		
			
			% {\tiny (Ref: ChatGPT - Explained! - CodeEmporium)}
			
% \end{frame}


%%%%%%%%%%%%%%%%%%%%%%%%%%%%%%%%%%%%%%%%%%%%%%%%%%%%%%%%%%%
\begin{frame}[fragile]\frametitle{Step 3}

\begin{itemize}
\item Take unseen prompt. Pass it through Proximal Policy Optimization (PPO) model which has been initialized with SFT. It generates a response (ie inferencing via SFT).
\item Calculate reward for that response via RM model. This reward is used to update the PPO model (loss function, back-propagation)
\end{itemize}	 

			\begin{center}
			\includegraphics[width=0.8\linewidth,keepaspectratio]{chatgpt6}
			
			\end{center}		
			
			{\tiny (Ref: ChatGPT: training process, advantages, and limitations - By Sergio Soage, Machine Learning Engineer at Aivo)}
			
\end{frame}

%%%%%%%%%%%%%%%%%%%%%%%%%%%%%%%%%%%%%%%%%%%%%%%%%%%%%%%%%%%%%%%%%%%%%%%%%%%%%%%%%%
\begin{frame}[fragile]\frametitle{As a Python Program?}
Pip install openai library, have KEY from your own account and then try various APIs available

			\begin{center}
			\includegraphics[width=0.8\linewidth,keepaspectratio]{chatgpt12}
			
			\end{center}		
			
			{\tiny (Ref: Getting Started with OpenAI API and GPT-3 -Assembly AI)}
			

\end{frame}

%%%%%%%%%%%%%%%%%%%%%%%%%%%%%%%%%%%%%%%%%%%%%%%%%%%%%%%%%%%%%%%%%%%%%%%%%%%%%%%%%%
\begin{frame}[fragile]\frametitle{}
\begin{center}
{\Large Applications}
\end{center}
\end{frame}




%%%%%%%%%%%%%%%%%%%%%%%%%%%%%%%%%%%%%%%%%%%%%%%%%%%%%%%%%%%%%%%%%%%%%%%%%%%%%%%%%%
\begin{frame}[fragile]\frametitle{Explain Complex Subjects}
\begin{center}
\includegraphics[width=0.8\linewidth,keepaspectratio]{chatgpt15}
\end{center}
	
{\tiny (Ref: Top 10 Chat GPT Use Cases - Simplilearn)}
\end{frame}

%%%%%%%%%%%%%%%%%%%%%%%%%%%%%%%%%%%%%%%%%%%%%%%%%%%%%%%%%%%%%%%%%%%%%%%%%%%%%%%%%%
\begin{frame}[fragile]\frametitle{Write Code}
\begin{center}
\includegraphics[width=0.8\linewidth,keepaspectratio]{chatgpt16}
\end{center}

it was still writing but I stopped it, just to show this pic

{\tiny (Ref: Top 10 Chat GPT Use Cases - Simplilearn)}
\end{frame}

%%%%%%%%%%%%%%%%%%%%%%%%%%%%%%%%%%%%%%%%%%%%%%%%%%%%%%%%%%%%%%%%%%%%%%%%%%%%%%%%%%
\begin{frame}[fragile]\frametitle{Debug Code}
\begin{center}
\includegraphics[width=0.8\linewidth,keepaspectratio]{chatgpt17}
\end{center}

{\tiny (Ref: Top 10 Chat GPT Use Cases - Simplilearn)}
\end{frame}

%%%%%%%%%%%%%%%%%%%%%%%%%%%%%%%%%%%%%%%%%%%%%%%%%%%%%%%%%%%%%%%%%%%%%%%%%%%%%%%%%%
\begin{frame}[fragile]\frametitle{Get Custom Marketing Strategy}
\begin{center}
\includegraphics[width=0.8\linewidth,keepaspectratio]{chatgpt18}
\end{center}

{\tiny (Ref: Top 10 Chat GPT Use Cases - Simplilearn)}
\end{frame}

%%%%%%%%%%%%%%%%%%%%%%%%%%%%%%%%%%%%%%%%%%%%%%%%%%%%%%%%%%%%%%%%%%%%%%%%%%%%%%%%%%
\begin{frame}[fragile]\frametitle{Write Articles}
\begin{center}
\includegraphics[width=0.8\linewidth,keepaspectratio]{chatgpt19}
\end{center}

{\tiny (Ref: Top 10 Chat GPT Use Cases - Simplilearn)}
\end{frame}

%%%%%%%%%%%%%%%%%%%%%%%%%%%%%%%%%%%%%%%%%%%%%%%%%%%%%%%%%%%%%%%%%%%%%%%%%%%%%%%%%%
\begin{frame}[fragile]\frametitle{Summarize Book}
\begin{center}
\includegraphics[width=0.8\linewidth,keepaspectratio]{chatgpt20}
\end{center}

{\tiny (Ref: Top 10 Chat GPT Use Cases - Simplilearn)}
\end{frame}

%%%%%%%%%%%%%%%%%%%%%%%%%%%%%%%%%%%%%%%%%%%%%%%%%%%%%%%%%%%%%%%%%%%%%%%%%%%%%%%%%%
\begin{frame}[fragile]\frametitle{Answer Interview Questions}
\begin{center}
\includegraphics[width=0.8\linewidth,keepaspectratio]{chatgpt21}
\end{center}

{\tiny (Ref: Top 10 Chat GPT Use Cases - Simplilearn)}
\end{frame}

%%%%%%%%%%%%%%%%%%%%%%%%%%%%%%%%%%%%%%%%%%%%%%%%%%%%%%%%%%%%%%%%%%%%%%%%%%%%%%%%%%
\begin{frame}[fragile]\frametitle{Develop Apps}
\begin{center}
\includegraphics[width=0.8\linewidth,keepaspectratio]{chatgpt22}
\end{center}

{\tiny (Ref: Top 10 Chat GPT Use Cases - Simplilearn)}
\end{frame}

%%%%%%%%%%%%%%%%%%%%%%%%%%%%%%%%%%%%%%%%%%%%%%%%%%%%%%%%%%%%%%%%%%%%%%%%%%%%%%%%%%
\begin{frame}[fragile]\frametitle{Create Health Plan}
\begin{center}
\includegraphics[width=0.8\linewidth,keepaspectratio]{chatgpt23}
\end{center}

{\tiny (Ref: Top 10 Chat GPT Use Cases - Simplilearn)}
\end{frame}

%%%%%%%%%%%%%%%%%%%%%%%%%%%%%%%%%%%%%%%%%%%%%%%%%%%%%%%%%%%%%%%%%%%%%%%%%%%%%%%%%%
\begin{frame}[fragile]\frametitle{Answer General Knowledge Questions}
\begin{center}
\includegraphics[width=0.8\linewidth,keepaspectratio]{chatgpt24}
\end{center}

{\tiny (Ref: Top 10 Chat GPT Use Cases - Simplilearn)}
\end{frame}

%%%%%%%%%%%%%%%%%%%%%%%%%%%%%%%%%%%%%%%%%%%%%%%%%%%%%%%%%%%
\begin{frame}[fragile]\frametitle{In Other Industries?}


\begin{itemize}
\item Generating automated responses to frequently asked questions 
\item Can help improve customer service as well as
train technicians
\item Helping write professional and effective emails, quotes, and proposals;
\item Generating marketing and advertising content
\item Providing technical assistance, such as troubleshooting
\item Helping generate training material for employees.
\end{itemize}	 

\tiny{(Ref:ChatGPT Could Be the ‘Write’ Choice for HVACR Contractors - Joanna R. Turpin)}
\end{frame}


%%%%%%%%%%%%%%%%%%%%%%%%%%%%%%%%%%%%%%%%%%%%%%%%%%%%%%%%%%%%%%%%%%%%%%%%%%%%%%%%%%
\begin{frame}[fragile]\frametitle{HVAC QnA}
Prompt: ``Diagnose a frosted evaporator coil''

Result: 
``There are several reasons why an evaporator coil may become frosted:
\begin{itemize}
\item Low refrigerant levels: If the refrigerant levels in the system are too low, it can cause the
evaporator coil to freeze. This is because the refrigerant is responsible for absorbing heat
from the air, and without enough refrigerant, the coil will not be able to absorb enough heat
to prevent freezing.
\item Dirty air filter: A dirty air filter can restrict airflow to the evaporator coil, causing the coil
to freeze.
\item Blocked or restricted airflow: If the airflow to the evaporator coil is blocked or
restricted, the coil may freeze. This can be caused by a number of things such as dirty ducts or
registers, closed or blocked vents, or a malfunctioning blower fan.
\item Obstruction in the drain line: A blockage in the drain line can cause water to back up and freeze on the coil.
\end{itemize}	 
It's important to fix the issue as soon as possible, as a frozen evaporator coil can lead to decreased
efficiency and can cause damage to the unit.
''

\tiny{(Ref:ChatGPT Could Be the ‘Write’ Choice for HVACR Contractors - Joanna R. Turpin)}
\end{frame}

%%%%%%%%%%%%%%%%%%%%%%%%%%%%%%%%%%%%%%%%%%%%%%%%%%%%%%%%%%%%%%%%%%%%%%%%%%%%%%%%%%
\begin{frame}[fragile]\frametitle{HVAC QnA}
Prompt: ``How to respond to a customer who thinks an air conditioner is
too expensive''

Result: ``
\begin{itemize}
\item Understand their concerns: Find out why they think the air conditioner is too expensive.
Are they comparing it to a lower-quality unit? Are they on a tight budget? Understanding
their concerns will help you address them more effectively.
\item Provide a detailed explanation of the costs: Explain the costs associated with the air
conditioner, including the cost of the unit, installation, and any additional materials or labor
that may be required.
\item Compare costs with the benefits: Explain the benefits of the air conditioner, such as
energy efficiency, durability, and long-term cost savings. Explain how the cost of the unit may
be offset by energy savings in the long run.
\item Offer financing options: If the customer is unable to afford the air conditioner upfront,
offer financing options that can help them spread the cost over time.

\end{itemize}	 

\ldots

\tiny{(Ref:ChatGPT Could Be the ‘Write’ Choice for HVACR Contractors - Joanna R. Turpin)}
\end{frame}

%%%%%%%%%%%%%%%%%%%%%%%%%%%%%%%%%%%%%%%%%%%%%%%%%%%%%%%%%%%%%%%%%%%%%%%%%%%%%%%%%%
\begin{frame}[fragile]\frametitle{HVAC QnA}

\ldots 

\begin{itemize}
\item Provide a written estimate: Provide the customer with a detailed, written estimate that
outlines the costs associated with the air conditioner and the benefits it will provide. This will
help them make an informed decision.
\item Provide a guarantee or warranty on the unit: Provide the customer with a guarantee
or warranty on the unit to give them peace of mind in their investment.
\item Provide an alternative solution: If the customer cannot afford the recommended
solution, offer an alternative solution that fits their budget while still meeting their needs.
\end{itemize}	 
It's important to be patient, understanding and provide all the information customers need to
make a decision that best fits their needs and budget.
''

\tiny{(Ref:ChatGPT Could Be the ‘Write’ Choice for HVACR Contractors - Joanna R. Turpin)}
\end{frame}
%%%%%%%%%%%%%%%%%%%%%%%%%%%%%%%%%%%%%%%%%%%%%%%%%%%%%%%%%%%%%%%%%%%%%%%%%%%%%%%%%%
\begin{frame}[fragile]\frametitle{}
\begin{center}
{\Large The Effect}
\end{center}
\end{frame}



%%%%%%%%%%%%%%%%%%%%%%%%%%%%%%%%%%%%%%%%%%%%%%%%%%%%%%%%%%%
\begin{frame}[fragile]\frametitle{ChatGPT vs Google}


\begin{center}
\includegraphics[width=0.8\linewidth,keepaspectratio]{chatgpt28}
\end{center}		

Google: almost latest data, gives source, not trainable, looks more accurate
ChatGPT: a bit old data, no source, trainable, hallucinates

\tiny{(Ref:ChatGPT Explained: Complete A-Z Guide - Kripesh Adwani)}
\end{frame}


%%%%%%%%%%%%%%%%%%%%%%%%%%%%%%%%%%%%%%%%%%%%%%%%%%%%%%%%%%%
\begin{frame}[fragile]\frametitle{Will ChatGPT Kill Jobs?}


\begin{center}
\includegraphics[width=0.8\linewidth,keepaspectratio]{chatgpt29}
\end{center}		

Repetitive, boring and standard, language based jobs, for sure.
Need to be more creative, experiential to stand against ChatGPT.

\tiny{(Ref:ChatGPT Explained: Complete A-Z Guide - Kripesh Adwani)}
\end{frame}


%%%%%%%%%%%%%%%%%%%%%%%%%%%%%%%%%%%%%%%%%%%%%%%%%%%%%%%%%%%%%%%%%%%%%%%%%%%%%%%%%%
\begin{frame}[fragile]\frametitle{New Job Roles?}
Prompt Engineer: Preparing input to AI effectively to get the desired answer. Will need to AI works in the background plus domain knowledge. Give context, examples etc to prime the model to give short specific answers than the usual page-long ones (davinci GPT3 in this case)

			\begin{center}
			\includegraphics[width=0.8\linewidth,keepaspectratio]{chatgpt11}
			
			\end{center}		
			
			{\tiny (Ref: Advanced ChatGPT Guide - How to build your own Chat GPT Site - Drian Twarog)}
			

\end{frame}

%%%%%%%%%%%%%%%%%%%%%%%%%%%%%%%%%%%%%%%%%%%%%%%%%%%%%%%%%%%%%%%%%%%%%%%%%%%%%%%%%%
\begin{frame}[fragile]\frametitle{}
\begin{center}
{\Large Conclusions}
\end{center}
\end{frame}




%%%%%%%%%%%%%%%%%%%%%%%%%%%%%%%%%%%%%%%%%%%%%%%%%%%%%%%%%%%
\begin{frame}[fragile]\frametitle{Advantages}


\begin{itemize}
\item Conversational Abilities
\item Solving Complex Problems
\item Retaining Previous Information
\item Creative Assistant, random-mix-ideas
\item Will replace mundane language tasks, How to articles, homeworks, etc
\item Supports more complex instructions, ``reasoning'' tasks
\end{itemize}	 

\end{frame}

%%%%%%%%%%%%%%%%%%%%%%%%%%%%%%%%%%%%%%%%%%%%%%%%%%%%%%%%%%%
\begin{frame}[fragile]\frametitle{Dis-advantages}


\begin{itemize}
\item Sensitive to Input Phrasing
\item Incorrect Answers at Times
\item Cannot replace humans for innovation, for which data does not exist already
\item Keeps ``hallucinating''
\item Tends to write plausible but incorrect content with confidence
\item Cannot get language structure right all the time, e.g try getting ghazal written
\end{itemize}	 

\end{frame}



%%%%%%%%%%%%%%%%%%%%%%%%%%%%%%%%%%%%%%%%%%%%%%%%%%%%%%%%%%%
\begin{frame}[fragile]\frametitle{My Sketchnote}

\begin{center}
\includegraphics[width=0.45\linewidth,keepaspectratio]{ChatGPT_Sketchnote_Medium}
\end{center}		

{\tiny (Ref: https://medium.com/technology-hits/overview-of-chatgpt-95f4b43645c0)}
			

\end{frame}


% %%%%%%%%%%%%%%%%%%%%%%%%%%%%%%%%%%%%%%%%%%%%%%%%%%%%%%%%%%%
% \begin{frame}[fragile]\frametitle{Whats happening around, now?}


% \begin{itemize}
% \item ``Science journals ban listing of ChatGPT as co-author on papers'' - The Guardian
% \item OpenAI’s ChatGPT Took An IQ Test! - Two Minutes Papers
% \item Discussing internally as one of the Editors on a Medium Publication, that Do we allow AI generated articles? and if yes, how?
% \item Google’s DeepMind might release ChatGPT competitor Sparrow this year
% \end{itemize}	 

% \end{frame}

%%%%%%%%%%%%%%%%%%%%%%%%%%%%%%%%%%%%%%%%%%%%%%%%%%%%%%%%%%%
\begin{frame}[fragile]\frametitle{The Hype}


\begin{center}
\includegraphics[width=\linewidth,keepaspectratio]{chatgpt32}
\end{center}		

{\tiny (Ref: ChatGPT - Intro \& Potential Impact Sudalai Rajkumar, SRK)}
			

\end{frame}

%%%%%%%%%%%%%%%%%%%%%%%%%%%%%%%%%%%%%%%%%%%%%%%%%%%%%%%%%%%
\begin{frame}[fragile]\frametitle{Finally, from horses mouth!!}


\begin{center}
\includegraphics[width=\linewidth,keepaspectratio]{chatgpt7}
\end{center}		

{\tiny (Ref: ChatGPT: training process, advantages, and limitations - By Sergio Soage, Machine Learning Engineer at Aivo)}
			

\end{frame}

% %%%%%%%%%%%%%%%%%%%%%%%%%%%%%%%%%%%%%%%%%%%%%%%%%%%%%%%%%%%
% \begin{frame}[fragile]\frametitle{Whats next? - GPT4 }


% \begin{center}
% \includegraphics[width=\linewidth,keepaspectratio]{chatgpt33}
% \end{center}		

% *not confirmed yet

% {\tiny (Ref: https://twitter.com/WolfofBaldSt/status/1596768686051921923)}
			

% \end{frame}


%%%%%%%%%%%%%%%%%%%%%%%%%%%%%%%%%%%%%%%%%%%%%%%%%%%%%%%%%%%
\begin{frame}[fragile]\frametitle{References}
		\begin{itemize}
		\item Let's build GPT: from scratch, in code, spelled out: Andrej Karpathy
		\item ChatGPT and Reinforcement Learning - CodeEmporium
		\end{itemize}
\end{frame}
