%%%%%%%%%%%%%%%%%%%%%%%%%%%%%%%%%%%%%%%%%%%%%%%%%%%%%%%%%%%%%%%%%%%%%%%%%%%%%%%%%%
\begin{frame}[fragile]\frametitle{}
\begin{center}
{\Large Examples of Probability distributions}
\end{center}
\end{frame}


%%%%%%%%%%%%%%%%%%%%%%%%%%%%%%%%%%%%%%%%%%%%%%%%%%%%%%%%%%%
\begin{frame}
\frametitle{Discrete Probability Distributions }
Some examples:
\begin{itemize}
\item Bernoulli distribution :$ P(X = 1) = p,           P(X = 0) = 1- p$
 
\item Binomial distribution : -Two possible outcomes, probability constant = $(\frac{n}{x})p^x (1-p)^{n-x}$
 
\item Poisson distribution 
 
\item Negative Binomial 
\end{itemize}
\end{frame}

%%%%%%%%%%%%%%%%%%%%%%%%%%%%%%%%%%%%%%%%%%%%%%%%%%%%%%%%%%
\begin{frame}{Bernoulli}

Bernoulli random variable is used when the experiment results in
either success or failure where success is represented as 1 and failure as
0. If P is probability of success, then probability mass function of
Bernoulli variable x is given as:

\begin{center}
\includegraphics[width=0.45\linewidth,keepaspectratio]{bernoulli}
\end{center}
\end{frame}

%%%%%%%%%%%%%%%%%%%%%%%%%%%%%%%%%%%%%%%%%%%%%%%%%%%%%%%%%%
\begin{frame}{Binomial}

Binomial distribution is used when we want to count how many success we
have when we repeat an experiment for n numbers of time
independently. The probability mass function of binomial variable x,
where p is probability of success, is given as:

\begin{center}
\includegraphics[width=0.45\linewidth,keepaspectratio]{binom}
\end{center}
\end{frame}

%%%%%%%%%%%%%%%%%%%%%%%%%%%%%%%%%%%%%%%%%%%%%%%%%%%%%%%%%%
\begin{frame}{Poisson}

Poisson variable x is used when we are counting the number of
occurrences of an event in a unit of time such that the occurrences are
independent and rarely simultaneous. For an average number of
occurrence $\lambda$ the probability mass function is given as:
\begin{center}
\includegraphics[width=0.45\linewidth,keepaspectratio]{poisson}
\end{center}
\end{frame}


%%%%%%%%%%%%%%%%%%%%%%%%%%%%%%%%%%%%%%%%%%%%%%%%%%%%%%%%%%%
\begin{frame}
\frametitle{Continuous Probability Distributions }
Some examples:
\begin{itemize}
\item  Normal distribution 
 
\item  Uniform distribution 
 
\item  Exponential distribution 
 
\item  Weibull distribution 
 
\end{itemize}
\end{frame}

%%%%%%%%%%%%%%%%%%%%%%%%%%%%%%%%%%%%%%%%%%%%%%%%%%%%%%%%%%
\begin{frame}{Normal}

Normal random variable x comes handy for different situations. We can
use it to model physical measurements like weight, height etc. We can
also use it to model error made by measuring instruments. In general, it
can be used when the average and variance of the quantity being
measured is known. The probability density function is given as:
\begin{center}
\includegraphics[width=0.65\linewidth,keepaspectratio]{normdist1}
\end{center}
\end{frame}

%%%%%%%%%%%%%%%%%%%%%%%%%%%%%%%%%%%%%%%%%%%%%%%%%%%%%%%%%%
\begin{frame}{Uniform}

We use uniform random variable x when the probability density for
every value between an interval (starting from a and ending at b) is
equal. The probability density function of x is given as:
\begin{center}
\includegraphics[width=0.65\linewidth,keepaspectratio]{uniform}
\end{center}
\end{frame}

%%%%%%%%%%%%%%%%%%%%%%%%%%%%%%%%%%%%%%%%%%%%%%%%%%%%%%%%%%
\begin{frame}{Exponential}

Exponential variable x are used when we are measuring the time until
the  rst occurrence of an event, such that the occurrences in disjoint
time intervals are independent and rarely simultaneous. For the
average number of occurrences per unit time a, the probability density
function is given as:
\begin{center}
\includegraphics[width=0.65\linewidth,keepaspectratio]{expo}
\end{center}
\end{frame}

%%%%%%%%%%%%%%%%%%%%%%%%%%%%%%%%%%%%%%%%%%%%%%%%%%%%%%%%%%%%%%
%%\begin{frame}
%%\frametitle{Distributions of continuous random variables}
%%
%%\begin{itemize}
%%
%%\item For a continuous random variable, $P(X=x)=0$ for all $x$.  
%%
%%\item It takes on values in an interval, for example
%%
%%$$
%%P(X\le 3),\; P(X>-1),\; P(2 < X \le 3).
%%$$
%%
%%\end{itemize}
%%\end{frame}


