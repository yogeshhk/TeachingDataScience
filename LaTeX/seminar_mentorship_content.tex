%%%%%%%%%%%%%%%%%%%%%%%%%%%%%%%%%%%%%%%%%%%%%%%%%%%%%%%%%%%%%%%%%%%%%%%%%%%%%%%%%%
\begin{frame}[fragile]\frametitle{}
\begin{center}
{\Large Mentorship Overview}
\end{center}
\end{frame}

%%%%%%%%%%%%%%%%%%%%%%%%%%%%%%%%%%%%%%%%%%%%%%%%%%%%%%%%%%%
\begin{frame}[fragile]\frametitle{AI Community Projects}
\begin{itemize}
  \item Driven by Pune AI Community (PAIC) Team
  \item Project-based, hands-on AI mentorship and collaboration
  \item Focus on real-world systems: RAG, Knowledge Graphs, LLMs, Geometry
  \item Voluntary, unpaid, non-commercial contribution
  \item No job or internship guarantees
  \item Contributions via GitHub pull requests with minimal hand-holding
  \item Opportunities for talks and research papers
\end{itemize}
\end{frame}

%%%%%%%%%%%%%%%%%%%%%%%%%%%%%%%%%%%%%%%%%%%%%%%%%%%%%%%%%%%
\begin{frame}[fragile]\frametitle{How to Join}
\begin{itemize}
  \item Email yogeshkulkarni@yahoo.com cc puneaicommunity@gmail.com with subject line "AI Mentoring"
  \item Include GitHub profile or relevant work
  \item You will get added to the 'PAIC Mentoring' Google chat-group
  \item Match skills or willingness to learn required stack
  \item Learn by building and collaborating
\end{itemize}
\end{frame}

%%%%%%%%%%%%%%%%%%%%%%%%%%%%%%%%%%%%%%%%%%%%%%%%%%%%%%%%%%%%%%%%%%%%%%%%%%%%%%%%%%
\begin{frame}[fragile]\frametitle{}
\begin{center}
{\Large Projects}
\end{center}
\end{frame}


%%%%%%%%%%%%%%%%%%%%%%%%%%%%%%%%%%%%%%%%%%%%%%%%%%%%%%%%%%%%%%%%%%%%%%%%%%%%%%%%%%
\begin{frame}[fragile]\frametitle{}
\begin{center}
{\Large Ask Yogasutra}

{\textbf{Repo: https://github.com/yogeshhk/Sarvadnya/tree/master/src/ask\_yogasutra}}
\end{center}
\end{frame}

%%%%%%%%%%%%%%%%%%%%%%%%%%%%%%%%%%%%%%%%%%%%%%%%%%%%%%%%%%%
\begin{frame}[fragile]\frametitle{Ask Yogasutra}
\begin{itemize}
  \item AI system to explore Yogasutra of Patanjali
  \item Combines original sutras, translations, commentaries
  \item Knowledge Graph: Sutras as nodes, relations as edges
  \item Streamlit-based graph visualization
  \item RAG-powered chatbot for Q\&A
  \item Planned GraphRAG for multi-hop reasoning
  \item \textbf{Skills:} Python, LangChain/LlamaIndex, Cypher, Neo4j, Streamlit
  \item \textbf{Focus:} RAG systems, GraphRAG, Graph modeling, Knowledge Graphs, research papers
\end{itemize}
\end{frame}

%%%%%%%%%%%%%%%%%%%%%%%%%%%%%%%%%%%%%%%%%%%%%%%%%%%%%%%%%%%%%%%%%%%%%%%%%%%%%%%%%%
\begin{frame}[fragile]\frametitle{}
\begin{center}
{\Large MidcurveNN/LLM}

{\textbf{Repo: https://github.com/yogeshhk/MidcurveNN}}
\end{center}
\end{frame}

%%%%%%%%%%%%%%%%%%%%%%%%%%%%%%%%%%%%%%%%%%%%%%%%%%%%%%%%%%%
\begin{frame}[fragile]\frametitle{MidcurveNN: Neural Midcurve Generation}
\begin{itemize}
  \item Neural network for midcurve generation of 2D closed polygon shapes
  \item Geometric dimension reduction using deep learning
  \item Input: polygon points/lines; Output: midcurve representation
  \item Architecture: encoder-decoder style models
  \item Modeling variable-length graph input/output
  \item LLM fine-tuning for geometry using text-based BREP format
  \item \textbf{Skills:} Python, neural networks, CAD geometry, transformers, fine-tuning
  \item \textbf{Focus:} Algorithms, training data, metrics, benchmarking, research analysis
\end{itemize}
\end{frame}

%%%%%%%%%%%%%%%%%%%%%%%%%%%%%%%%%%%%%%%%%%%%%%%%%%%%%%%%%%%%%%%%%%%%%%%%%%%%%%%%%%
\begin{frame}[fragile]\frametitle{}
\begin{center}
{\Large Mining Resume}

{\textbf{Repo: https://github.com/yogeshhk/MiningResume}}
\end{center}
\end{frame}

%%%%%%%%%%%%%%%%%%%%%%%%%%%%%%%%%%%%%%%%%%%%%%%%%%%%%%%%%%%
\begin{frame}[fragile]\frametitle{Mining Resume}
\begin{itemize}
  \item Extract structured fields from resumes using text mining
  \item Parse contact info, skills, education, work experience
  \item Optional: Build knowledge graph for resume data \& querying
  \item NLP parsing pipelines for unstructured documents
  \item Entity extraction: name, email, skills, companies
  \item Dataset cleaning and text normalization
  \item \textbf{Skills:} Python, NLP (spaCy/NLTK), regex, LLMs
  \item \textbf{Focus:} Data extraction workflows, text processing, visualizations
\end{itemize}
\end{frame}

%%%%%%%%%%%%%%%%%%%%%%%%%%%%%%%%%%%%%%%%%%%%%%%%%%%%%%%%%%%%%%%%%%%%%%%%%%%%%%%%%%
\begin{frame}[fragile]\frametitle{}
\begin{center}
{\Large Sarvadnya RAG Systems}

{\textbf{Repo: https://github.com/yogeshhk/Sarvadnya}}
\end{center}
\end{frame}

%%%%%%%%%%%%%%%%%%%%%%%%%%%%%%%%%%%%%%%%%%%%%%%%%%%%%%%%%%%
\begin{frame}[fragile]\frametitle{Sarvadnya: RAG Systems Collection}
\begin{itemize}
  \item Collection of proof-of-concept RAG systems and integrations
  \item RAG over diverse document modalities \& corpora
  \item Implement RAG for PDFs, text collections, multi-modal content
  \item Handle chunking, embeddings, semantic search
  \item Build UI for query \& response (chat, dashboards)
  \item Evaluate retrieval quality and LLM grounding
  \item \textbf{Skills:} Python, LangChain/LlamaIndex, vector stores, embeddings
  \item \textbf{Focus:} Retrieval pipelines, extensions, improved retrieval quality
\end{itemize}
\end{frame}

%%%%%%%%%%%%%%%%%%%%%%%%%%%%%%%%%%%%%%%%%%%%%%%%%%%%%%%%%%%%%%%%%%%%%%%%%%%%%%%%%%
\begin{frame}[fragile]\frametitle{}
\begin{center}
{\Large The Nature of Code – Python}

{\textbf{Repo: https://github.com/yogeshhk/TheNatureOfCode}}
\end{center}
\end{frame}

%%%%%%%%%%%%%%%%%%%%%%%%%%%%%%%%%%%%%%%%%%%%%%%%%%%%%%%%%%%
\begin{frame}[fragile]\frametitle{The Nature of Code – Python}
\begin{itemize}
  \item Python implementation inspired by The Nature of Code series
  \item Physics simulations (forces, motion, interactions)
  \item Generative animations and visual outputs
  \item Mix of mathematical models and creative coding
  \item Useful for ML-augmented animation prototypes
  \item Educational and exploratory codebase
  \item \textbf{Skills:} Python, NumPy, animation frameworks, physics modeling
  \item \textbf{Focus:} Add physics models, improve visuals, link with ML datasets
\end{itemize}
\end{frame}

%%%%%%%%%%%%%%%%%%%%%%%%%%%%%%%%%%%%%%%%%%%%%%%%%%%%%%%%%%%%%%%%%%%%%%%%%%%%%%%%%%
\begin{frame}[fragile]\frametitle{}
\begin{center}
{\Large Thanks}
\end{center}
\end{frame}