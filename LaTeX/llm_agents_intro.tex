%%%%%%%%%%%%%%%%%%%%%%%%%%%%%%%%%%%%%%%%%%%%%%%%%%%%%%%%%%%%%%%%%%%%%%%%%%%%%%%%%%
\begin{frame}[fragile]\frametitle{}
\begin{center}
{\Large Introduction}
\end{center}
\end{frame}

%%%%%%%%%%%%%%%%%%%%%%%%%%%%%%%%%%%%%%%%%%%%%%%%%%%%%%%%%%%
\begin{frame}[fragile]\frametitle{Classical idea of Agents}
      \begin{itemize}
        \item Agents are autonomous
        \item Can look at their environment and analyze the situation
        \item Make comprehensive plans to achieve specific goals
        \item Actually take action to execute those plans
        \item Agents bridge the gap between answering and doing
      \end{itemize}
\end{frame}

%%%%%%%%%%%%%%%%%%%%%%%%%%%%%%%%%%%%%%%%%%%%%%%%%%%%%%%%%%%
\begin{frame}[fragile]\frametitle{Welcome to AI Agents}
      \begin{itemize}
		\item Agents were there from 1950's but they are effective because of LLMs.
        \item Agents are systems where LLMs dynamically direct their own paths and tool usage
        \item Agents can have autonomous-ness or predefined workflow paths
        \item Essential component in modern AI systems with varying degrees of autonomy	  
		\item Unlike LLMs that just respond to prompts, agents do things
        \item Not just everyday chatbots, systems that reason, plan, and take action
        \item Can take on complex multi-step tasks autonomously or with human-in-loop
        \item Technology is advancing rapidly from conversational to agentic AI
        \item AI agents represent one of the most exciting frontiers in AI
		\item 2025 is the year of AI agents.
      \end{itemize}
\end{frame}


%%%%%%%%%%%%%%%%%%%%%%%%%%%%%%%%%%%%%%%%%%%%%%%%%%%%%%%%%%%
\begin{frame}[fragile]\frametitle{The Impossible Job}
      \begin{itemize}
        \item If your boss tasks you with planning a massive get-together
        \item Must prepare a guest list and plan fancy menu 
        \item Needs to find entertainment for the event
        \item A simple chatbot (with answering LLM, not reasoning LLM) cannot handle these complex requirements
        \item You need an AI agent, not just responses, but actions
		\item Agents have a Reasoning LLM as brain and tools as its hands-legs.
        \item AI agents improve with reasoning, acting, and memory components. (ReAct = Reasoning + Acting)
      \end{itemize}
\end{frame}

%%%%%%%%%%%%%%%%%%%%%%%%%%%%%%%%%%%%%%%%%%%%%%%%%%%%%%%%%%%
\begin{frame}[fragile]\frametitle{ReAct (Reasoning + Acting) Agent}
		\begin{center}
		\includegraphics[width=\linewidth,keepaspectratio]{aiagents122}
		
		{\tiny (Ref: Python + AI Agents - Microsoft)}
		\end{center}	
\end{frame}
%%%%%%%%%%%%%%%%%%%%%%%%%%%%%%%%%%%%%%%%%%%%%%%%%%%%%%%%%%%%%%%%%%%%%%%%%%%%%%%%%%
\begin{frame}[fragile]\frametitle{The Evolution of AI Capabilities}
\begin{itemize}
    \item \textbf{Traditional Programming:} Needed code to operate
    \item \textbf{Traditional ML:} Needed feature engineering
    \item \textbf{Deep Learning:} ML with Neural networks, no feature engineering
    \item \textbf{Foundational Models:} Many tasks single model
    \item \textbf{Agents (2024 \ldots):} Can actually \textbf{do things}, not just talk
\end{itemize}
\end{frame}

%%%%%%%%%%%%%%%%%%%%%%%%%%%%%%%%%%%%%%%%%%%%%%%%%%%%%%%%%%%%%%%%%%%%%%%%%%%%%%%%%%
\begin{frame}[fragile]\frametitle{Why Does ``Taking Action'' Matter?}
\begin{itemize}
    \item In 2022, ChatGPT was revolutionary because AI felt conversational
    \item By 2024, people wanted more than conversation, they wanted \textbf{execution}
    \item Examples of what users now expect:
    \begin{itemize}
        \item Instead of listing leads ? \textbf{email them directly}
        \item Instead of summarizing docs ? \textbf{file and create workflow tasks}
        \item Instead of suggesting products ? \textbf{customize landing pages}
    \end{itemize}
    \item This shift from \textbf{information} to \textbf{action} defines the agent era
\end{itemize}
\end{frame}


%%%%%%%%%%%%%%%%%%%%%%%%%%%%%%%%%%%%%%%%%%%%%%%%%%%%%%%%%%%%%%%%%%%%%%%%%%%%%%%%%%
\begin{frame}[fragile]\frametitle{How Agents Work?}
    \begin{itemize}
        \item Agent acts, take you from one state to the other state(ReAct paper: Reasoning and Action), 
		\item It can plan and make decisions, provides value by workflow automation. 
		\item Agents have access to tools (ToolFormer paper) e.g. Search APIs, booking, send email etc.
		\item Interacting of external environment and other Agents, etc.
		\item Memory to keep the history of conversations/actions done so far.
		\item May have human-in-loop to keep it sane in the wild-world.
    \end{itemize}
\end{frame}

%%%%%%%%%%%%%%%%%%%%%%%%%%%%%%%%%%%%%%%%%%%%%%%%%%%%%%%%%%%
\begin{frame}[fragile]\frametitle{The Agent's Fundamental Game Loop}
      \begin{itemize}
        \item Not a one-and-done action, but a continuous reasoning loop
        \item Similar to a programming while loop that keeps iterating
        \item \textbf{Thought}: Analyzes situation and plans next step
        \item \textbf{Action}: Calls specific tools to execute the plan
        \item \textbf{Observation}: Examines results of the action taken
        \item Cycle repeats: Thought → Action → Observation
        \item Continues until the task is completely accomplished
        \item This loop enables continuous adaptation and problem-solving
      \end{itemize}
\end{frame}

%%%%%%%%%%%%%%%%%%%%%%%%%%%%%%%%%%%%%%%%%%%%%%%%%%%%%%%%%%%
\begin{frame}[fragile]\frametitle{Agent's Inner Monologue}
      \begin{itemize}
        \item Agents have visible thought processes before taking action
        \item Example: ``User wants weather in New York. I have a tool for that''.
        \item ``My first move is to call the weather API''.
        \item Internal planning step makes agents more than reactive programs
        \item Reasoning through problems before execution
        \item This deliberation distinguishes agents from simple scripts
        \item Shows intelligent decision-making rather than blind execution
      \end{itemize}
\end{frame}

%%%%%%%%%%%%%%%%%%%%%%%%%%%%%%%%%%%%%%%%%%%%%%%%%%%%%%%%%%%%%%%%%%%%%%%%%%%%%%%%%%
\begin{frame}[fragile]\frametitle{How Do Agents Take Action?}
\begin{itemize}
    \item The magic lies in \textbf{tools} and \textbf{function calling}
    \item Agents are paired with APIs, plugins, or external systems
    \item Instead of just text responses, LLMs output structured commands:
    \begin{itemize}
        \item ``Call the send\_email() function with these inputs...''
        \item ``Fetch records from CRM using this query...''
        \item ``Schedule a meeting for Tuesday at 2PM...''
    \end{itemize}
    \item \textbf{Mental model:} LLM = brain, Tools = hands
    \item Without tools, agents just talk. With tools, they act.
\end{itemize}
\end{frame}


%%%%%%%%%%%%%%%%%%%%%%%%%%%%%%%%%%%%%%%%%%%%%%%%%%%%%%%%%%%
\begin{frame}[fragile]\frametitle{Defining AI Agents with an example}

      \begin{itemize}
		\item Planning a trip involves many complex tasks
		\item Point A: Just discussing the trip
		\item Point B: All bookings and itinerary ready
		\item AI Agents aim to take you from A to B
		\item First requirement: Agent adds value by saving time/money
      \end{itemize}

		\begin{center}
		\includegraphics[width=0.6\linewidth,keepaspectratio]{aiagents25}
		
		{\tiny (Ref: Vizuara AI Agents Bootcamp)}
		\end{center}	

\end{frame}

%%%%%%%%%%%%%%%%%%%%%%%%%%%%%%%%%%%%%%%%%%%%%%%%%%%%%%%%%%%
\begin{frame}[fragile]\frametitle{Evolving Definition of Agents}

      \begin{itemize}
		\item Not all tools from A to B are agents (e.g., cars)
		\item Agents must plan and make decisions
		\item Example: Choosing flights based on budget
		\item Planning daily itinerary needs contextual judgment
		\item Second requirement: Agent includes decision-making ability
      \end{itemize}

		\begin{center}
		\includegraphics[width=0.6\linewidth,keepaspectratio]{aiagents26}
		
		{\tiny (Ref: Vizuara AI Agents Bootcamp)}
		\end{center}	

\end{frame}

%%%%%%%%%%%%%%%%%%%%%%%%%%%%%%%%%%%%%%%%%%%%%%%%%%%%%%%%%%%
\begin{frame}[fragile]\frametitle{Agents Need Tools}

      \begin{itemize}
        \item Even self-driving cars plan but are not agents
        \item Agents need access to external tools
        \item Tools = Access to services (e.g., Gmail, Booking)
        \item Agents perform tasks using these tools
        \item Third requirement: Agent adds tool access to capabilities
      \end{itemize}

		\begin{center}
		\includegraphics[width=0.6\linewidth,keepaspectratio]{aiagents27}
		
		{\tiny (Ref: Vizuara AI Agents Bootcamp)}
		\end{center}	

\end{frame}

%%%%%%%%%%%%%%%%%%%%%%%%%%%%%%%%%%%%%%%%%%%%%%%%%%%%%%%%%%%
\begin{frame}[fragile]\frametitle{Rise of LLMs in Agents}

      \begin{itemize}
        \item Transformers (2017) enabled powerful LLMs
        \item LLMs understand and generate human language
        \item Agents use LLMs for reasoning and planning
        \item LLMs enable understanding of webpages and writing emails
        \item Fourth requirement: Agents are LLMs with tools and planning ability
      \end{itemize}

		\begin{center}
		\includegraphics[width=0.6\linewidth,keepaspectratio]{aiagents28}
		
		{\tiny (Ref: Vizuara AI Agents Bootcamp)}
		\end{center}	

\end{frame}



%%%%%%%%%%%%%%%%%%%%%%%%%%%%%%%%%%%%%%%%%%%%%%%%%%%%%%%%%%%%%%%%%%%%%%%%%%%%%%%%%%
\begin{frame}[fragile]\frametitle{What Is an Agent? (Technical Definition)}
\begin{itemize}
    \item Agent acts and takes you from one state to another, providing value through workflow automation
    \item Based on ReAct paradigm: \textbf{Reasoning + Acting}
    \item Key capabilities:
    \begin{itemize}
        \item Can plan and make decisions
        \item Has access to tools (search APIs, booking, email, etc.)
        \item Interacts with external environments and other agents
        \item Maintains memory of conversations and actions
        \item May include human-in-the-loop for safety
    \end{itemize}
    \item Agents existed since the 1950s but are now effective because of LLMs
\end{itemize}
\end{frame}


%%%%%%%%%%%%%%%%%%%%%%%%%%%%%%%%%%%%%%%%%%%%%%%%%%%%%%%%%%%%%%%%%%%%%%%%%%%%%%%%%%
\begin{frame}[fragile]\frametitle{Two Ways to Define Agents}
\begin{columns}
    \begin{column}[T]{0.5\linewidth}
        \textbf{Technical View:}
        \begin{itemize}
            \item LLM (brain)
            \item + Tools (hands)
            \item + Planning (strategy)
            \item + Memory (context)
            \item + State management
        \end{itemize}
    \end{column}
    \begin{column}[T]{0.5\linewidth}
        \textbf{Business View:}
        \begin{itemize}
            \item Systems that complete tasks end-to-end
            \item Focus on outcomes, not components
            \item Solve real-world problems
            \item Provide measurable value
        \end{itemize}
    \end{column}
\end{columns}

\vspace{0.5cm}
\textbf{Important:} Today's agents are \textbf{engineering wrappers} around AI models, the intelligence comes from the LLMs, agents help act on that intelligence.
\end{frame}


%%%%%%%%%%%%%%%%%%%%%%%%%%%%%%%%%%%%%%%%%%%%%%%%%%%%%%%%%%%
\begin{frame}[fragile]\frametitle{Final Definition of Agents}

      \begin{itemize}
        \item Agents can learn from feedback and environment
        \item Agents interact with tools, humans, and websites
        \item They improve with experience (memory)
        \item Agents evolve over time via memory and feedback
        \item Fifth requirement: LLMs + Tools + Planning + Learning
	
      \end{itemize}

		\begin{center}
		\includegraphics[width=0.6\linewidth,keepaspectratio]{aiagents29}
		
		{\tiny (Ref: Vizuara AI Agents Bootcamp)}
		\end{center}	

\end{frame}

%%%%%%%%%%%%%%%%%%%%%%%%%%%%%%%%%%%%%%%%%%%%%%%%%%%%%%%%%%%
\begin{frame}[fragile]\frametitle{Understanding Agency}

      \begin{itemize}
        \item Agency = Level of autonomy an agent has
        \item Low agency → less value
        \item High agency → high value
        \item More autonomous agents can handle complex tasks
        \item Agency is key to measuring agent usefulness
      \end{itemize}

		\begin{center}
		\includegraphics[width=0.8\linewidth,keepaspectratio]{aiagents30}
		
		{\tiny (Ref: Vizuara AI Agents Bootcamp)}
		\end{center}	

\end{frame}

%%%%%%%%%%%%%%%%%%%%%%%%%%%%%%%%%%%%%%%%%%%%%%%%%%%%%%%%%%%
\begin{frame}[fragile]\frametitle{Agentic Systems: Degrees of Autonomy}
      \begin{itemize}
        \item AI community debated what qualifies as a ``true agent''.
        \item Binary classification (agent vs. non-agent) was seen as unnecessary.
        \item Term \textbf{agentic} introduced to describe autonomy as a spectrum.
        \item Systems can exhibit different levels of agent-like behavior.
        \item Focus should shift from definitions to building effective systems.
        \item Idea popularized through talks, articles, and social media.
        \item Helped reduce unproductive debates in the AI community.
        \item Encourages practical progress in agent development.
      \end{itemize}
	  
		\begin{center}
		\includegraphics[width=0.8\linewidth,keepaspectratio]{agents10}
		
		{\tiny (Ref: Agentic AI - Deep Learning AI)}
		\end{center}		  
\end{frame}

%%%%%%%%%%%%%%%%%%%%%%%%%%%%%%%%%%%%%%%%%%%%%%%%%%%%%%%%%%%
\begin{frame}[fragile]\frametitle{Less Autonomous Agents}
      \begin{itemize}
        \item Example task: write an essay about black holes.
        \item Agent generates predefined web search queries.
        \item Hard-coded workflow calls search engine and fetches pages.
        \item Retrieved content is used to compose the essay.
        \item Execution follows a deterministic sequence of steps.
        \item Limited decision-making capability.
        \item Works adequately for structured workflows.
        \item Represents a lower degree of autonomy.
      \end{itemize}
\end{frame}

%%%%%%%%%%%%%%%%%%%%%%%%%%%%%%%%%%%%%%%%%%%%%%%%%%%%%%%%%%%
\begin{frame}[fragile]\frametitle{More Autonomous Agents}
      \begin{itemize}
        \item LLM decides how to fulfill the essay request.
        \item Chooses between sources like web, news, or research archives.
        \item Determines which tools to call without human direction.
        \item Selects how many pages or documents to fetch.
        \item May convert PDFs to text using additional tools.
        \item Writes the essay after gathering information.
        \item Can reflect, refine, and improve the output.
        \item Iteratively gathers more data if needed.
        \item Produces a final, higher-quality response.
      \end{itemize}
\end{frame}


%%%%%%%%%%%%%%%%%%%%%%%%%%%%%%%%%%%%%%%%%%%%%%%%%%%%%%%%%%%
\begin{frame}[fragile]\frametitle{Tools: The Agent's Hands}
      \begin{itemize}
        \item LLM is the agent's brain; tools are its hands
        \item Tools are functions agents call to interact with the world
        \item Can search web, run calculations, or query databases
        \item Bridge between thinking and doing in the real world
        \item Enable agents to move from planning to execution
        \item Without tools, agent thoughts would be useless
        \item Tools provide the interface to external systems and data
      \end{itemize}
\end{frame}

%%%%%%%%%%%%%%%%%%%%%%%%%%%%%%%%%%%%%%%%%%%%%%%%%%%%%%%%%%%
\begin{frame}[fragile]\frametitle{Two Ways Agents Use Tools}
      \begin{itemize}
        \item \textbf{JSON Agent}: Writes structured work orders for other systems
        \item JSON approach requires external system to read and execute
        \item \textbf{Code Agent}: Directly writes and runs code blocks
        \item Code approach is more direct and powerful
        \item Code is naturally more expressive than JSON
        \item Can handle complex logic like loops and conditionals
        \item Modular, easier to debug, and taps into existing libraries
        \item Code agents can access thousands of APIs directly
      \end{itemize}
\end{frame}

%%%%%%%%%%%%%%%%%%%%%%%%%%%%%%%%%%%%%%%%%%%%%%%%%%%%%%%%%%%
\begin{frame}[fragile]\frametitle{Code Agent in Action}
      \begin{itemize}
        \item Alfred needs a gala menu - agent has ``suggest\_menu'' tool
        \item Agent doesn't just make up suggestions randomly
        \item Generates and runs actual code to call the specific tool
        \item Gets real results from the tool execution
        \item Super direct, efficient, and powerful way to take action
        \item Code generation enables precise tool interaction
        \item Results are based on actual tool capabilities, not hallucination
      \end{itemize}
\end{frame}

%%%%%%%%%%%%%%%%%%%%%%%%%%%%%%%%%%%%%%%%%%%%%%%%%%%%%%%%%%%
\begin{frame}[fragile]\frametitle{Advanced Pattern: Agentic RAG}
      \begin{itemize}
        \item Traditional RAG: Retrieval Augmented Generation fetches info before answering
        \item Agentic RAG supercharges this with intelligent multi-step processes
        \item Turns retrieval itself into an agent-driven task
        \item Like having a master researcher on staff
        \item Doesn't just do one search - runs complete research processes
        \item Rewrites queries for better results and runs multiple searches
        \item Uses findings to inform next searches and validates accuracy
        \item Pulls from both private data and public web sources
      \end{itemize}
\end{frame}

%%%%%%%%%%%%%%%%%%%%%%%%%%%%%%%%%%%%%%%%%%%%%%%%%%%%%%%%%%%
\begin{frame}[fragile]\frametitle{Multi-Agent Systems: Digital Teams}
      \begin{itemize}
        \item Complex problems like finding the missing Batmobile need teams
        \item Single agents can't handle web searches, calculations, and visualization
        \item Solution: Build teams of specialized agents
        \item Manager agent acts as project lead breaking down big tasks
        \item Delegates work to specialist agents with specific skills
        \item Web agent handles online searching while manager coordinates
        \item Manager focuses on big picture and final integration
        \item Digital division of labor for complex problem solving
      \end{itemize}
\end{frame}

%%%%%%%%%%%%%%%%%%%%%%%%%%%%%%%%%%%%%%%%%%%%%%%%%%%%%%%%%%%
\begin{frame}[fragile]\frametitle{Agentic Workflow Design Patterns}
      \begin{itemize}
        \item Agentic workflows are built by sequencing reusable building blocks.
        \item Design patterns guide how blocks combine into complex systems.
        \item Four key patterns: reflection, tool use, planning, multi-agent collaboration.
        \item Patterns reduce ad-hoc workflow design.
        \item Enable more capable and flexible agents.
        \item Covered at a high level here.
        \item Explored in depth later in the course.
      \end{itemize}
\end{frame}

%%%%%%%%%%%%%%%%%%%%%%%%%%%%%%%%%%%%%%%%%%%%%%%%%%%%%%%%%%%
\begin{frame}[fragile]\frametitle{Reflection and Tool Use}
      \begin{itemize}
        \item \textbf{Reflection}: LLM evaluates and improves its own outputs.
        \item Example: generate code, then critique correctness and style.
        \item Feedback loop produces improved versions (v2, v3, etc.).
        \item External feedback (e.g., runtime errors) can enhance reflection.
        \item Reflection gives incremental performance gains, not guarantees.
        \item Can use a separate critique agent instead of self-critique.
        \item \textbf{Tool use}: LLMs call external functions or APIs.
        \item Examples: web search, code execution, data analysis.
        \item Tool choice enables LLMs to solve richer tasks.
      \end{itemize}
	  
		\begin{center}
		\includegraphics[width=0.8\linewidth,keepaspectratio]{agents11}
		
		{\tiny (Ref: Agentic AI - Deep Learning AI)}
		\end{center}		  
\end{frame}

%%%%%%%%%%%%%%%%%%%%%%%%%%%%%%%%%%%%%%%%%%%%%%%%%%%%%%%%%%%
\begin{frame}[fragile]\frametitle{Planning }
      \begin{itemize}
        \item \textbf{Planning}: LLM decides the sequence of actions to take.
        \item Example: HuggingGPT selects and orders multiple model calls.
        \item Planning replaces hard-coded workflows.
        \item More powerful but harder to control.
        \item Can yield surprising and delightful results.
      \end{itemize}
	  
		\begin{center}
		\includegraphics[width=0.8\linewidth,keepaspectratio]{agents12}
		
		{\tiny (Ref: Agentic AI - Deep Learning AI)}
		\end{center}	  
\end{frame}

%%%%%%%%%%%%%%%%%%%%%%%%%%%%%%%%%%%%%%%%%%%%%%%%%%%%%%%%%%%
\begin{frame}[fragile]\frametitle{Multi-Agent Collaboration}
      \begin{itemize}
        \item Multiple specialized agents collaborate.
        \item Agents take roles (e.g., researcher, writer, editor).
        \item Inspired by human team structures.
        \item Improves outcomes on complex tasks.
        \item Introduces higher coordination and control challenges.
      \end{itemize}
	  
		\begin{center}
		\includegraphics[width=0.8\linewidth,keepaspectratio]{agents13}
		
		{\tiny (Ref: Agentic AI - Deep Learning AI)}
		\end{center}	  
\end{frame}

% %%%%%%%%%%%%%%%%%%%%%%%%%%%%%%%%%%%%%%%%%%%%%%%%%%%%%%%%%%%
% \begin{frame}[fragile]\frametitle{The GAIA Benchmark Reality Check}
      % \begin{itemize}
        % \item GAIA LLM benchmark tests real-world multi-step problems
        % \item Measures how well systems handle tricky, complex tasks
        % \item Results are eye-opening and show current limitations
        % \item Humans solve these tasks with 92\% accuracy
        % \item Today's most advanced AI models: only 15\% accuracy
        % \item Massive 77\% gap between human and AI performance
        % \item This gap is exactly what agentic systems aim to close
        % \item Shows the enormous potential for improvement
      % \end{itemize}
% \end{frame}

% %%%%%%%%%%%%%%%%%%%%%%%%%%%%%%%%%%%%%%%%%%%%%%%%%%%%%%%%%%%
% \begin{frame}[fragile]\frametitle{The Fundamental Shift}
      % \begin{itemize}
        % \item Moving from conversational AI to agentic AI era
        % \item Old paradigm: Ask questions, get answers
        % \item New paradigm: State goals, systems plan and accomplish them
        % \item Represents fundamental change in human-computer interaction
        % \item Technology for building personal AI assistants advancing rapidly
        % \item Not a question of ``if'' but ``when'' this becomes reality
        % \item The future Alfred is closer than we think
        % \item Prepare for AI that can handle ``impossible'' complex tasks
      % \end{itemize}
% \end{frame}


%%%%%%%%%%%%%%%%%%%%%%%%%%%%%%%%%%%%%%%%%%%%%%%%%%%%%%%%%%%
\begin{frame}[fragile]\frametitle{Importance of Evaluation in Agentic Workflows}
      \begin{itemize}
        \item Experience across many teams building agentic workflows.
        \item Biggest predictor of success is disciplined evaluation (evals).
        \item Strong eval practices lead to more effective agents.
        \item Weak evals result in inefficient development.
        \item Ability to design and run evals is a core skill.
        \item This module gives a high-level overview.
        \item Deeper treatment appears in later course modules.
      \end{itemize}
\end{frame}

%%%%%%%%%%%%%%%%%%%%%%%%%%%%%%%%%%%%%%%%%%%%%%%%%%%%%%%%%%%
\begin{frame}[fragile]\frametitle{Why Evals Are Hard to Design Upfront}
      \begin{itemize}
        \item Agentic workflows have many unforeseen failure modes.
        \item Difficult to anticipate issues before deployment.
        \item Example: customer order inquiry agent.
        \item Problems often surface only after observing outputs.
        \item Predefining all eval criteria is unrealistic.
        \item Best practice: build first, evaluate later.
        \item Iterative inspection drives improvement.
      \end{itemize}
\end{frame}

% %%%%%%%%%%%%%%%%%%%%%%%%%%%%%%%%%%%%%%%%%%%%%%%%%%%%%%%%%%%
% \begin{frame}[fragile]\frametitle{Example Failure: Competitor Mentions}
      % \begin{itemize}
        % \item Agent unexpectedly mentions competitors.
        % \item Creates awkward or undesirable business situations.
        % \item Example competitor names: ComproCo, RivalCo.
        % \item Issue is hard to predict in advance.
        % \item Identified only by reviewing real outputs.
        % \item Classified as an error by many businesses.
        % \item Motivates creation of targeted evals.
      % \end{itemize}
% \end{frame}

% %%%%%%%%%%%%%%%%%%%%%%%%%%%%%%%%%%%%%%%%%%%%%%%%%%%%%%%%%%%
% \begin{frame}[fragile]\frametitle{Objective Evals Using Code}
      % \begin{itemize}
        % \item Competitor mentions are objectively measurable.
        % \item Either a competitor is mentioned or not.
        % \item Use a predefined list of competitor names.
        % \item Write code to scan agent outputs.
        % \item Count frequency of errors across responses.
        % \item Track error rate as a fraction of outputs.
        % \item Enables quantitative progress tracking.
      % \end{itemize}
% \end{frame}

%%%%%%%%%%%%%%%%%%%%%%%%%%%%%%%%%%%%%%%%%%%%%%%%%%%%%%%%%%%
\begin{frame}[fragile]\frametitle{Subjective Evals with LLMs as Judges}
      \begin{itemize}
        \item Many evaluation criteria are subjective.
        \item Free-text outputs resist simple rule-based scoring.
        \item Use an LLM as a judge for quality assessment.
        \item Example: rate essay quality from 1 to 5.
        \item Judge LLM reads and scores generated content.
        \item Applied across multiple generated reports.
        \item Scores can indicate improvement over time.
      \end{itemize}
\end{frame}

%%%%%%%%%%%%%%%%%%%%%%%%%%%%%%%%%%%%%%%%%%%%%%%%%%%%%%%%%%%
\begin{frame}[fragile]\frametitle{Limitations of LLM-Based Scoring}
      \begin{itemize}
        \item LLMs are weak at fine-grained numeric ratings.
        \item 1–5 scale scores are often unreliable.
        \item Technique may be used as an initial baseline.
        \item Better evaluation methods exist.
        \item Advanced techniques covered in later modules.
        \item Still useful for rough comparisons early on.
      \end{itemize}
\end{frame}

%%%%%%%%%%%%%%%%%%%%%%%%%%%%%%%%%%%%%%%%%%%%%%%%%%%%%%%%%%%
\begin{frame}[fragile]\frametitle{Types of Agentic AI Evaluations}
      \begin{itemize}
        \item Objective evals using code-based checks.
        \item Subjective evals using LLM judges.
        \item End-to-end evals measure full agent output quality.
        \item Component-level evals assess individual steps.
        \item Different evals support different development goals.
        \item Helps isolate and fix specific weaknesses.
      \end{itemize}
\end{frame}

%%%%%%%%%%%%%%%%%%%%%%%%%%%%%%%%%%%%%%%%%%%%%%%%%%%%%%%%%%%
\begin{frame}[fragile]\frametitle{Error Analysis and Traces}
      \begin{itemize}
        \item Inspect intermediate agent outputs (traces).
        \item Review each step of the workflow.
        \item Identify where behavior deviates from expectations.
        \item Manual reading reveals improvement opportunities.
        \item Known as error analysis.
        \item Essential skill for agent builders.
        \item Strongly linked to effective eval design.
      \end{itemize}
\end{frame}



%%%%%%%%%%%%%%%%%%%%%%%%%%%%%%%%%%%%%%%%%%%%%%%%%%%%%%%%%%%
\begin{frame}[fragile]\frametitle{Papers that Shaped AI Agents}

      \begin{itemize}
        \item Core research papers laid the foundation
        \item Introduced key frameworks and architectures
        \item Sparked recent boom in agent development
        \item Include Transformer and Agentic frameworks
        \item Major driving force in LLM-based agent systems
      \end{itemize}

		\begin{center}
		\includegraphics[width=0.45\linewidth,keepaspectratio]{aiagents31}
		\includegraphics[width=0.45\linewidth,keepaspectratio]{aiagents32}
		\includegraphics[width=0.45\linewidth,keepaspectratio]{aiagents33}
		\includegraphics[width=0.45\linewidth,keepaspectratio]{aiagents34}
		
		{\tiny (Ref: Vizuara AI Agents Bootcamp)}
		\end{center}	
 
\end{frame}

%%%%%%%%%%%%%%%%%%%%%%%%%%%%%%%%%%%%%%%%%%%%%%%%%%%%%%%%%%%%%%%%%%%%%%%%%%%%%%%%%%
\begin{frame}[fragile]\frametitle{}
\begin{center}
{\Large Frameworks}
\end{center}
\end{frame}

%%%%%%%%%%%%%%%%%%%%%%%%%%%%%%%%%%%%%%%%%%%%%%%%%%%%%%%%%%%
\begin{frame}[fragile]\frametitle{Python AI Agent Frameworks Overview}
    \begin{itemize}
        \item Python offers several frameworks for building single and multi-agent AI systems.
        \item These frameworks support workflows, tool use, reasoning, and human-in-the-loop capabilities.
        \item Common applications include autonomous assistants, research copilots, and workflow orchestration.
    \end{itemize}
\end{frame}

%%%%%%%%%%%%%%%%%%%%%%%%%%%%%%%%%%%%%%%%%%%%%%%%%%%%%%%%%%%
\begin{frame}[fragile]\frametitle{LangChain v1}
    \begin{itemize}
        \item \textbf{Type:} Multi-agent, graph-based framework.
        \item \textbf{Built on:} LangGraph for agent state management and orchestration.
        \item \textbf{Features:}
              \begin{itemize}
                  \item Agent-centric architecture.
                  \item Workflow graphs for reasoning and tool execution.
                  \item Human-in-the-loop integration via LangSmith.
                  \item Supports OpenAI, Anthropic, Gemini, and local models.
              \end{itemize}
    \end{itemize}
\end{frame}

%%%%%%%%%%%%%%%%%%%%%%%%%%%%%%%%%%%%%%%%%%%%%%%%%%%%%%%%%%%
\begin{frame}[fragile]\frametitle{Pydantic-AI}
    \begin{itemize}
        \item \textbf{Type:} Lightweight single/multi-agent framework.
        \item \textbf{Focus:} Strong type safety using Pydantic for structured inputs/outputs.
        \item \textbf{Features:}
              \begin{itemize}
                  \item Declarative agent definition with validation.
                  \item Easy LLM orchestration and chaining.
                  \item Built for developers needing data integrity.
              \end{itemize}
        \item \textbf{Ideal for:} AI applications requiring schema consistency and typed reasoning.
    \end{itemize}
\end{frame}

%%%%%%%%%%%%%%%%%%%%%%%%%%%%%%%%%%%%%%%%%%%%%%%%%%%%%%%%%%%
\begin{frame}[fragile]\frametitle{Crew AI}
    \begin{itemize}
        \item \textbf{Type:} Multi-agent coordination framework.
        \item \textbf{Concept:} “Crew” of specialized agents working collaboratively.
        \item \textbf{Features:}
              \begin{itemize}
                  \item Role-based agent interactions.
                  \item Shared memory and task assignment.
                  \item Supports autonomous workflow creation.
                  \item Integrates with LangChain, OpenAI, and custom tools.
              \end{itemize}
        \item \textbf{Use Case:} Complex multi-role task solving (research, planning, content generation).
    \end{itemize}
\end{frame}

%%%%%%%%%%%%%%%%%%%%%%%%%%%%%%%%%%%%%%%%%%%%%%%%%%%%%%%%%%%
\begin{frame}[fragile]\frametitle{Autogen}
    \begin{itemize}
        \item \textbf{Developed by:} Microsoft Research.
        \item \textbf{Type:} Multi-agent conversational framework.
        \item \textbf{Features:}
              \begin{itemize}
                  \item Agent-to-agent and human-in-loop communication.
                  \item Supports function calling and code execution.
                  \item Built-in memory and orchestration logic.
                  \item Extendable with custom agents and tools.
              \end{itemize}
        \item \textbf{Example Code:}
\begin{lstlisting}[language=Python]
from autogen import AssistantAgent, UserProxyAgent

assistant = AssistantAgent("assistant", llm="gpt-4")
user = UserProxyAgent("user", code_execution=True)

user.initiate_chat(assistant, message="Build a weather app.")
\end{lstlisting}
    \end{itemize}
\end{frame}

%%%%%%%%%%%%%%%%%%%%%%%%%%%%%%%%%%%%%%%%%%%%%%%%%%%%%%%%%%%
\begin{frame}[fragile]\frametitle{OpenAI Agents}
    \begin{itemize}
        \item \textbf{Type:} Hosted agent platform by OpenAI (2024+).
        \item \textbf{Features:}
              \begin{itemize}
                  \item Create, configure, and deploy GPT-based agents.
                  \item Access via OpenAI API and GPTs interface.
                  \item Integrates with files, APIs, tools, and memory.
                  \item Supports human feedback and evaluation.
              \end{itemize}
        \item \textbf{Example Use:} Deploy a custom assistant with knowledge base and tools.
    \end{itemize}
\end{frame}

%%%%%%%%%%%%%%%%%%%%%%%%%%%%%%%%%%%%%%%%%%%%%%%%%%%%%%%%%%%
\begin{frame}[fragile]\frametitle{Google ADK (Agent Development Kit)}
    \begin{itemize}
        \item \textbf{Developed by:} Google DeepMind / Gemini ecosystem.
        \item \textbf{Type:} Multi-agent and workflow orchestration framework.
        \item \textbf{Features:}
              \begin{itemize}
                  \item Integrates Gemini 1.5 models with tool-use capabilities.
                  \item Supports perception, memory, planning, and dialogue.
                  \item Human-in-the-loop and autonomous modes.
                  \item Built for scalable, production-grade AI agents.
              \end{itemize}
        \item \textbf{Use Case:} Building Gemini-powered assistants and reasoning systems.
    \end{itemize}
\end{frame}

%%%%%%%%%%%%%%%%%%%%%%%%%%%%%%%%%%%%%%%%%%%%%%%%%%%%%%%%%%%
\begin{frame}[fragile]\frametitle{Comparison Summary}
    \begin{itemize}
        \item \textbf{LangChain v1:} Graph-based, modular, integrates with LangSmith.
        \item \textbf{Pydantic-AI:} Strong typing, data-safe orchestration.
        \item \textbf{Crew AI:} Role-based multi-agent collaboration.
        \item \textbf{Autogen:} Conversational multi-agent system by Microsoft.
        \item \textbf{OpenAI Agents:} Hosted GPT-based no-code agent deployment.
        \item \textbf{Google ADK:} Gemini-native multi-agent orchestration suite.
    \end{itemize}
\end{frame}

%%%%%%%%%%%%%%%%%%%%%%%%%%%%%%%%%%%%%%%%%%%%%%%%%%%%%%%%%%%%%%%%%%%%%%%%%%%%%%%%%%
\begin{frame}[fragile]\frametitle{}
\begin{center}
{\Large Final Thoughts}
\end{center}
\end{frame}

%%%%%%%%%%%%%%%%%%%%%%%%%%%%%%%%%%%%%%%%%%%%%%%%%%%%%%%%%%%%%%%%%%%%%%%%%%%%%%%%%%
\begin{frame}[fragile]\frametitle{Why Agents?}
		\begin{center}
		\includegraphics[width=\linewidth,keepaspectratio]{aiagents123}
		
		{\tiny (Ref: Google Cloud Labgs H2 India Agents for All)}
		\end{center}
\end{frame}



%%%%%%%%%%%%%%%%%%%%%%%%%%%%%%%%%%%%%%%%%%%%%%%%%%%%%%%%%%%%%%%%%%%%%%%%%%%%%%%%%%
\begin{frame}[fragile]\frametitle{When to Use Agents?}
    \begin{itemize}
        \item Best suited for tasks requiring flexibility and model-driven decision-making
        \item Consider tradeoffs: agents increase latency and cost for better task performance
        \item Recommended for open-ended problems with unpredictable steps
        \item Simple solutions preferred - single LLM calls with retrieval often sufficient
    \end{itemize}
	
		\begin{center}
		\includegraphics[width=0.8\linewidth,keepaspectratio]{aiagents124}
		
		{\tiny (Ref: Google Cloud Labgs H2 India Agents for All)}
		\end{center}	
\end{frame}

%%%%%%%%%%%%%%%%%%%%%%%%%%%%%%%%%%%%%%%%%%%%%%%%%%%%%%%%%%%%%%%%%%%%%%%%%%%%%%%%%%
\begin{frame}[fragile]\frametitle{When NOT to Use Agents?}
	
		\begin{center}
		\includegraphics[width=\linewidth,keepaspectratio]{aiagents125}
		
		{\tiny (Ref: Google Cloud Labgs H2 India Agents for All)}
		\end{center}	
\end{frame}

%%%%%%%%%%%%%%%%%%%%%%%%%%%%%%%%%%%%%%%%%%%%%%%%%%%%%%%%%%%%%%%%%%%%%%%%%%%%%%%%%%
\begin{frame}[fragile]\frametitle{Agent Challenges}
	
		\begin{center}
		\includegraphics[width=\linewidth,keepaspectratio]{aiagents126}
		
		{\tiny (Ref: Google Cloud Labgs H2 India Agents for All)}
		\end{center}	
\end{frame}

%%%%%%%%%%%%%%%%%%%%%%%%%%%%%%%%%%%%%%%%%%%%%%%%%%%%%%%%%%%%%%%%%%%%%%%%%%%%%%%%%%
\begin{frame}[fragile]\frametitle{Mitigating Predictability challenges}
    \begin{itemize}
        \item choose reasoning models, when possible. (eg. gemini 2.5 flash)
        \item ground LLM in trusted data to reduce hallucinations (RAG, search)
        \item guide your agent to think step by step (chain of thought or few-shot prompting) 
        \item use state management and memory. 
        \item implement checks in the agent’s workflow: tool-call request validation, safety filters, logic checks...
        \item test and evaluate your agent.
    \end{itemize}
	
		{\tiny (Ref: Google Cloud Labgs H2 India Agents for All)}

\end{frame}

%%%%%%%%%%%%%%%%%%%%%%%%%%%%%%%%%%%%%%%%%%%%%%%%%%%%%%%%%%%%%%%%%%%%%%%%%%%%%%%%%%
\begin{frame}[fragile]\frametitle{Mitigating Stability challenges}
    \begin{itemize}
        \item use consistent tool abstractions via Model Context Protocol (MCP)
        \item when writing your agent’s system prompt, be as detailed as possible. (tool descriptions)
        \item implement or leverage error-handling in your agent (circuit-breakers, retries, hand off to a human in the loop...)
        \item utilize state management, especially the workflow status field (recover from interruptions, prevent infinite loops) 
        \item be willing to refactor and adjust with new framework features, tool interfaces... 
    \end{itemize}
	
		{\tiny (Ref: Google Cloud Labgs H2 India Agents for All)}

\end{frame}

%%%%%%%%%%%%%%%%%%%%%%%%%%%%%%%%%%%%%%%%%%%%%%%%%%%%%%%%%%%%%%%%%%%%%%%%%%%%%%%%%%
\begin{frame}[fragile]\frametitle{Mitigating Operations challenges}
    \begin{itemize}
        \item in the dev phase, log verbosely.
        \item build robust test and evaluation datasets. automate these tests on every commit.
        \item gather necessary metrics, like token throughput to your agent’s models, and response code distribution. create alerts if these trip certain thresholds.
        \item leverage agent framework’s tracing features to track model and tool calls, find + address latency bottlenecks.
    \end{itemize}
	
		{\tiny (Ref: Google Cloud Labgs H2 India Agents for All)}

\end{frame}

%%%%%%%%%%%%%%%%%%%%%%%%%%%%%%%%%%%%%%%%%%%%%%%%%%%%%%%%%%%%%%%%%%%%%%%%%%%%%%%%%%
\begin{frame}[fragile]\frametitle{Future AI Applications}
\begin{itemize}
    \item What are future AI applications like?
    \begin{itemize}
        \item \textbf{Generative:} Generate content like text and images
        \item \textbf{Agentic:} Execute complex tasks on behalf of humans
    \end{itemize}
    \item How do we empower every developer to build them?
    \begin{itemize}
        \item \textbf{Co-Pilots:} Human-AI collaboration
        \item \textbf{Autonomous:} Independent task execution
    \end{itemize}
    \item 2024 is expected to be the year of AI agents
\end{itemize}
\end{frame}

%%%%%%%%%%%%%%%%%%%%%%%%%%%%%%%%%%%%%%%%%%%%%%%%%%%%%%%%%%%
\begin{frame}[fragile]\frametitle{The Big Question}
      \begin{itemize}
        \item Agentic AI technology is moving incredibly fast
        \item Personal AI assistants will soon be capable of complex tasks
        \item Think beyond simple queries to multi-step accomplishments
        \item Consider what ``impossible'' tasks you want to delegate
        \item What complex, party-of-the-century level challenge will you tackle?
        \item The era of AI that truly does rather than just discusses
        \item Prepare for AI assistants that can handle your biggest challenges
      \end{itemize}
\end{frame}


