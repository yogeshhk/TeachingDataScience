%%%%%%%%%%%%%%%%%%%%%%%%%%%%%%%%%%%%%%%%%%%%%%%%%%%%%%%%%%%%%%%%%%%%%%%%%%%%%%%%%%
\begin{frame}[fragile]\frametitle{}
\begin{center}
{\Large Thermal Engineering: AC Tonnage Calculation }
\end{center}
\end{frame}

%%%%%%%%%%%%%%%%%%%%%%%%%%%%%%%%%%%%%%%%%%%%%%%%%%%%%%%%%%%
\begin{frame}[fragile]\frametitle{Basic}

Calculate the air conditioning (AC) tonnage required for a small residential house.

\end{frame}

%%%%%%%%%%%%%%%%%%%%%%%%%%%%%%%%%%%%%%%%%%%%%%%%%%%%%%%%%%%
\begin{frame}[fragile]\frametitle{Basic}

\textbf{Prompt:} Calculate the air conditioning (AC) tonnage required for a small residential house.

\vspace{6pt}
\textbf{Assistant response (simple, rule-of-thumb):}

Use a common thumb-rule: \emph{about 1 ton of cooling capacity per 400–600 ($ft^2$)}, depending on climate and building quality.
Steps (concise):

\begin{itemize}
\item Measure conditioned floor area ($ft^2$).
\item Choose a rule factor: use 1 ton for 500 ($ft^2$) as a neutral starting point.
\item Divide area by the chosen factor and round up to the nearest 0.5 ton to provide margin.
\end{itemize}

\textbf{Quick example:} 1200 ($ft^2$)
 house 
$\rightarrow$
$\rightarrow$ 1200 / 500 = 2.4 $\rightarrow$ choose 2.5 or 3.0 tons depending on climate (2.5 for moderate, 3.0 for hotter/humid).

\vspace{4pt}
\emph{Note:} This is a high-level estimate. Final selection should consider insulation, windows, orientation, occupancy, and internal gains.

\end{frame}

%%%%%%%%%%%%%%%%%%%%%%%%%%%%%%%%%%%%%%%%%%%%%%%%%%%%%%%%%%%
\begin{frame}[fragile]\frametitle{Adding Roles}

As a HVAC engineer, explain to a junior student how to estimate the AC tonnage required for a residential house.

\end{frame}


%%%%%%%%%%%%%%%%%%%%%%%%%%%%%%%%%%%%%%%%%%%%%%%%%%%%%%%%%%%
\begin{frame}[fragile]\frametitle{Adding Roles}

\textbf{Prompt:} As a HVAC engineer, explain to a junior student how to estimate the AC tonnage required for a residential house.

\vspace{6pt}
\textbf{Assistant response (mentoring tone, stepwise):}

\textbf{Measure and list inputs:} conditioned floor area, ceiling height (standard ~8 ft; if higher, note it), number and type of windows, wall/attic insulation quality, typical occupancy, major heat-producing appliances, and climate (warm/humid vs temperate).

\textbf{Pick a baseline rule:} start with 1 ton per $500 ft^2$
 for a conservative neutral case.

\textbf{Adjust qualitatively:}
\begin{itemize}
\item Poor insulation or many west/south windows $\rightarrow$ increase capacity.
\item Well insulated, shaded, or cooler climate $\rightarrow$ reduce slightly.
\item High occupancy/appliances or south-facing glass $\rightarrow$ add margin.
\end{itemize}

\textbf{Round and select equipment:} round the calculated value up to the next common size (e.g., 2.5, 3.0 tons). Explain trade-offs: oversizing reduces dehumidification and can increase cycling; undersizing reduces comfort.

\textbf{Document assumptions:} write down the rule used and all qualitative adjustments so the estimate can be checked later.

\end{frame}
%%%%%%%%%%%%%%%%%%%%%%%%%%%%%%%%%%%%%%%%%%%%%%%%%%%%%%%%%%%
\begin{frame}[fragile]\frametitle{Adding Constraints}

Estimate the AC tonnage using simple thumb rules only. Do not use detailed equations or advanced heat load software.

\end{frame}

%%%%%%%%%%%%%%%%%%%%%%%%%%%%%%%%%%%%%%%%%%%%%%%%%%%%%%%%%%%
\begin{frame}[fragile]\frametitle{Adding Constraints}

\textbf{Prompt:} Estimate the AC tonnage using simple thumb rules only. Do not use detailed equations or advanced heat load software.

\vspace{6pt}
\textbf{Assistant response (constrained, practical):}

\begin{enumerate}
\item \textbf{Baseline factor:} use 1 ton per 500 ($ft^2$) for typical homes with average insulation.
\item \textbf{Simple modifiers (apply qualitatively):}
\begin{itemize}
\item $+20$\% capacity for poor insulation or many sun-exposed windows.
\item $-10-15$\% for excellent insulation and heavy shading.
\item $+0.5$ ton if household often has 4+ occupants or many heat-generating appliances.
\end{itemize}
\item \textbf{Procedure:} compute base tons = area / 500, then adjust by one of the simple modifiers, finally round up to nearest 0.5 ton.
\end{enumerate}

\textbf{Constrained example (no equations beyond division):} For a $1,200 ft^2$ house with average insulation: base = 1200 / 500 = 2.4 $\rightarrow$ round up $\rightarrow$ 2.5 ton. If insulation is poor, add 20\%: 2.4 × 1.2 = 2.88 $\rightarrow$ round up $\rightarrow$ 3.0 ton.

\end{frame}

%%%%%%%%%%%%%%%%%%%%%%%%%%%%%%%%%%%%%%%%%%%%%%%%%%%%%%%%%%%
\begin{frame}[fragile]\frametitle{Adding Examples}

Provide two example cases: \\
Example 1: A 1000 sq.ft. house with average insulation. \\
Example 2: A 1500 sq.ft. house with poor insulation. 

\end{frame}

%%%%%%%%%%%%%%%%%%%%%%%%%%%%%%%%%%%%%%%%%%%%%%%%%%%%%%%%%%%
\begin{frame}[fragile]\frametitle{Adding Examples}

\textbf{Prompt:} Provide two example cases: Example 1: A 1000 sq.ft. house with average insulation. Example 2: A 1500 sq.ft. house with poor insulation.

\vspace{6pt}
\textbf{Assistant response (worked examples, step-by-step arithmetic):}

\textbf{Rule used:} 1 ton per 500 ($ft^2$)
 baseline. Poor insulation $\rightarrow$ +20\% capacity.

\medskip
\textbf{Example 1 - 1000($ft^2$)
, average insulation:}
\begin{itemize}
\item Base = 1000 / 500 = 2.0 tons.
\item No upward adjustment (average insulation).
\item  \textbf{Select 2.0 tons} (or 2.5 tons if you want extra margin in hot climates).
\end{itemize}

\medskip
\textbf{Example 2 - 1500 f($ft^2$), poor insulation:}
\begin{itemize}
\item Base = 1500 / 500 = 3.0 tons.
\item Apply poor-insulation modifier: 3.0 × 1.20 = 3.6 tons.
\item Round up to common size $\rightarrow$ \textbf{Select 4.0 tons} (3.5 might be available; 4.0 gives conservative margin).
\end{itemize}

\medskip
\emph{Practical note:} When capacity after adjustment is close to a common nominal size (e.g., 3.6 $\rightarrow$ 3.5 or 4.0), consider climate and humidity: pick the larger if high humidity or frequent heat waves are expected.

\end{frame}

%%%%%%%%%%%%%%%%%%%%%%%%%%%%%%%%%%%%%%%%%%%%%%%%%%%%%%%%%%%
\begin{frame}[fragile]\frametitle{Adding Contextual Information}

You are designing HVAC for a middle-income residential building in a warm climate. Explain how you would estimate AC tonnage for one apartment.

\end{frame}

%%%%%%%%%%%%%%%%%%%%%%%%%%%%%%%%%%%%%%%%%%%%%%%%%%%%%%%%%%%
\begin{frame}[fragile]\frametitle{Incorporating Feedback}

The homeowner reports that previous AC units felt insufficient during peak summer. Improve the tonnage estimation by considering practical real-world factors.

\end{frame}

%%%%%%%%%%%%%%%%%%%%%%%%%%%%%%%%%%%%%%%%%%%%%%%%%%%%%%%%%%%
\begin{frame}[fragile]\frametitle{Time Constraints and Prioritization}

Estimate the required AC tonnage quickly for an urgent site visit. Focus only on the most influential factors.

\end{frame}

%%%%%%%%%%%%%%%%%%%%%%%%%%%%%%%%%%%%%%%%%%%%%%%%%%%%%%%%%%%
\begin{frame}[fragile]\frametitle{Incorporating Multidisciplinary Knowledge}

Include effects from building orientation, sunlight exposure, occupancy, and electrical appliance heat gains while keeping the explanation simple and non-mathematical.

\end{frame}

%%%%%%%%%%%%%%%%%%%%%%%%%%%%%%%%%%%%%%%%%%%%%%%%%%%%%%%%%%%
\begin{frame}[fragile]\frametitle{Addressing Regional and Climate Factors}

Modify the tonnage estimation method for hot–humid, hot–dry, and moderate climate conditions.

\end{frame}

%%%%%%%%%%%%%%%%%%%%%%%%%%%%%%%%%%%%%%%%%%%%%%%%%%%%%%%%%%%
\begin{frame}[fragile]\frametitle{Reflection and Iteration}

Reflect on the estimation process and suggest how it could be improved for better comfort and energy efficiency.

\end{frame}

%%%%%%%%%%%%%%%%%%%%%%%%%%%%%%%%%%%%%%%%%%%%%%%%%%%%%%%%%%%
\begin{frame}[fragile]\frametitle{Self-Criticism}

Estimate the AC tonnage for a house. Then re-read your method, identify possible errors or oversimplifications, and refine it.

\end{frame}

%%%%%%%%%%%%%%%%%%%%%%%%%%%%%%%%%%%%%%%%%%%%%%%%%%%%%%%%%%%
\begin{frame}[fragile]\frametitle{Chain-of-Thought}

Estimate the AC tonnage required for a house. Let’s think step by step from room size, heat gains, and climate to the final tonnage.

\end{frame}

%%%%%%%%%%%%%%%%%%%%%%%%%%%%%%%%%%%%%%%%%%%%%%%%%%%%%%%%%%%
\begin{frame}[fragile]\frametitle{Self-Consistency}

Here are three estimation methods for AC tonnage: Method A (area-based rule), Method B (volume-based rule), Method C (experience-based rule). Combine them into one consistent guideline.

\end{frame}
