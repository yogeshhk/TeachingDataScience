%%%%%%%%%%%%%%%%%%%%%%%%%%%%%%%%%%%%%%%%%%%%%%%%%%%%%%%%%%%
\begin{frame}[fragile]\frametitle{}
\begin{center}
{\Large Assignments for Mechanical Engineering}
\end{center}
\end{frame}

%%%%%%%%%%%%%%%%%%%%%%%%%%%%%%%%%%%%%%%%%%%%%%%%%%%%%%%%%%%
% Slide 1: Predictive Maintenance using Regression Analysis
%%%%%%%%%%%%%%%%%%%%%%%%%%%%%%%%%%%%%%%%%%%%%%%%%%%%%%%%%%%
\begin{frame}[fragile]\frametitle{Predictive Maintenance using Regression Analysis}
    \begin{itemize}
        \item \textbf{Dataset}: NASA Turbofan Engine Degradation Simulation Data Set
        \item \textbf{Source}: \href{https://www.nasa.gov/content/prognostics-center-of-excellence-data-set-repository}{NASA Prognostics Data Repository}
        \item \textbf{Description}: Predict the remaining useful life of turbofan engines using regression analysis based on sensor data.
    \end{itemize}
    \begin{lstlisting}[language=Python]
import pandas as pd
from sklearn.model_selection import train_test_split
from sklearn.linear_model import LinearRegression

data = pd.read_csv('turbofan_data.csv')
X = data[['feature1', 'feature2', 'feature3']]
y = data['remaining_useful_life']

X_train, X_test, y_train, y_test = train_test_split(X, y, test_size=0.2)
model = LinearRegression()
model.fit(X_train, y_train)
predictions = model.predict(X_test)
    \end{lstlisting}
\end{frame}

%%%%%%%%%%%%%%%%%%%%%%%%%%%%%%%%%%%%%%%%%%%%%%%%%%%%%%%%%%%
% Slide 2: Structural Health Monitoring using Anomaly Detection
%%%%%%%%%%%%%%%%%%%%%%%%%%%%%%%%%%%%%%%%%%%%%%%%%%%%%%%%%%%
\begin{frame}[fragile]\frametitle{Structural Health Monitoring using Anomaly Detection}
    \begin{itemize}
        \item \textbf{Dataset}: Bridge Health Monitoring Dataset
        \item \textbf{Source}: \href{https://archive.ics.uci.edu/ml/datasets/Bridge+Health+Monitoring}{UCI Machine Learning Repository}
        \item \textbf{Description}: Detect anomalies indicating potential structural failures in bridges based on sensor readings.
    \end{itemize}
    \begin{lstlisting}[language=Python]
import pandas as pd
from sklearn.ensemble import IsolationForest

data = pd.read_csv('bridge_health_data.csv')
model = IsolationForest(contamination=0.1)
model.fit(data)
anomalies = model.predict(data)
    \end{lstlisting}
\end{frame}

%%%%%%%%%%%%%%%%%%%%%%%%%%%%%%%%%%%%%%%%%%%%%%%%%%%%%%%%%%%
% Slide 3: Design Optimization using Genetic Algorithms
%%%%%%%%%%%%%%%%%%%%%%%%%%%%%%%%%%%%%%%%%%%%%%%%%%%%%%%%%%%
\begin{frame}[fragile]\frametitle{Design Optimization using Genetic Algorithms}
    \begin{itemize}
        \item \textbf{Dataset}: Engineering Design Optimization Dataset
        \item \textbf{Source}: \href{https://www.kaggle.com/datasets/sagnik1511/design-optimization-dataset}{Kaggle}
        \item \textbf{Description}: Optimize engineering designs using genetic algorithms to evaluate performance metrics.
    \end{itemize}
    \begin{lstlisting}[language=Python]
from deap import base, creator, tools, algorithms

creator.create("FitnessMin", base.Fitness, weights=(-1.0,))
creator.create("Individual", list, fitness=creator.FitnessMin)

toolbox = base.Toolbox()
# Define operations for crossover, mutation, etc.
population = toolbox.population(n=50)
algorithms.eaSimple(population, toolbox, cxpb=0.5, mutpb=0.2, ngen=40)
    \end{lstlisting}
\end{frame}

%%%%%%%%%%%%%%%%%%%%%%%%%%%%%%%%%%%%%%%%%%%%%%%%%%%%%%%%%%%
% Slide 4: Fault Detection in Rotating Machinery using SVM
%%%%%%%%%%%%%%%%%%%%%%%%%%%%%%%%%%%%%%%%%%%%%%%%%%%%%%%%%%%
\begin{frame}[fragile]\frametitle{Fault Detection in Rotating Machinery using SVM}
    \begin{itemize}
        \item \textbf{Dataset}: Machinery Fault Diagnosis Dataset
        \item \textbf{Source}: \href{https://archive.ics.uci.edu/ml/datasets/Machinery+Fault+Diagnosis}{UCI Machine Learning Repository}
        \item \textbf{Description}: Classify faults in rotating machinery using support vector machines (SVM).
    \end{itemize}
    \begin{lstlisting}[language=Python]
import pandas as pd
from sklearn.svm import SVC
from sklearn.model_selection import train_test_split

data = pd.read_csv('machinery_fault_data.csv')
X = data.drop('fault_type', axis=1)
y = data['fault_type']

X_train, X_test, y_train, y_test = train_test_split(X, y, test_size=0.25)
model = SVC()
model.fit(X_train, y_train)
accuracy = model.score(X_test, y_test)
    \end{lstlisting}
\end{frame}

%%%%%%%%%%%%%%%%%%%%%%%%%%%%%%%%%%%%%%%%%%%%%%%%%%%%%%%%%%%
% Slide 5: Thermal Analysis using Neural Networks
%%%%%%%%%%%%%%%%%%%%%%%%%%%%%%%%%%%%%%%%%%%%%%%%%%%%%%%%%%%
\begin{frame}[fragile]\frametitle{Thermal Analysis using Neural Networks}
    \begin{itemize}
        \item \textbf{Dataset}: Thermal Conductivity Dataset
        \item \textbf{Source}: \href{https://www.kaggle.com/datasets/sagnik1511/thermal-conductivity-dataset}{Kaggle}
        \item \textbf{Description}: Predict thermal conductivity of materials using neural networks.
    \end{itemize}
    \begin{lstlisting}[language=Python]
from keras.models import Sequential
from keras.layers import Dense

model = Sequential()
model.add(Dense(64, activation='relu', input_dim=10))
model.add(Dense(32, activation='relu'))
model.add(Dense(1))

model.compile(optimizer='adam', loss='mean_squared_error')
    \end{lstlisting}
\end{frame}

%%%%%%%%%%%%%%%%%%%%%%%%%%%%%%%%%%%%%%%%%%%%%%%%%%%%%%%%%%%
% Slide 6: Quality Control in Manufacturing using CNNs
%%%%%%%%%%%%%%%%%%%%%%%%%%%%%%%%%%%%%%%%%%%%%%%%%%%%%%%%%%%
\begin{frame}[fragile]\frametitle{Quality Control in Manufacturing using CNNs}
    \begin{itemize}
        \item \textbf{Dataset}: Manufacturing Defect Dataset
        \item \textbf{Source}: \href{https://www.kaggle.com/datasets/yangjianxin1/defect-detection-dataset}{Kaggle}
        \item \textbf{Description}: Train a CNN to classify manufacturing defects based on part images.
    \end{itemize}
    \begin{lstlisting}[language=Python]
from keras.models import Sequential
from keras.layers import Conv2D, MaxPooling2D, Flatten, Dense

model = Sequential()
model.add(Conv2D(32, (3,3), activation='relu', input_shape=(128,128,3)))
model.add(MaxPooling2D(pool_size=(2,2)))
model.add(Flatten())
model.add(Dense(128, activation='relu'))
model.add(Dense(10, activation='softmax'))

model.compile(optimizer='adam', loss='categorical_crossentropy', metrics=['accuracy'])
    \end{lstlisting}
\end{frame}

%%%%%%%%%%%%%%%%%%%%%%%%%%%%%%%%%%%%%%%%%%%%%%%%%%%%%%%%%%%
% Slide 7: Energy Consumption Prediction using Time Series Analysis
%%%%%%%%%%%%%%%%%%%%%%%%%%%%%%%%%%%%%%%%%%%%%%%%%%%%%%%%%%%
\begin{frame}[fragile]\frametitle{Energy Consumption Prediction using Time Series Analysis}
    \begin{itemize}
        \item \textbf{Dataset}: Energy Consumption Dataset
        \item \textbf{Source}: \href{https://archive.ics.uci.edu/ml/datasets/Energy+consumption+of+residential+buildings}{UCI Machine Learning Repository}
        \item \textbf{Description}: Forecast energy consumption patterns using ARIMA models.
    \end{itemize}
    \begin{lstlisting}[language=Python]
import pandas as pd
from statsmodels.tsa.arima.model import ARIMA

data = pd.read_csv('energy_consumption.csv')
model = ARIMA(data['consumption'], order=(5,1,0))
model_fit = model.fit()
forecast = model_fit.forecast(steps=10)
    \end{lstlisting}
\end{frame}

%%%%%%%%%%%%%%%%%%%%%%%%%%%%%%%%%%%%%%%%%%%%%%%%%%%%%%%%%%%
% Slide 8: Material Property Prediction using Gaussian Processes
%%%%%%%%%%%%%%%%%%%%%%%%%%%%%%%%%%%%%%%%%%%%%%%%%%%%%%%%%%%
\begin{frame}[fragile]\frametitle{Material Property Prediction using Gaussian Processes}
    \begin{itemize}
        \item \textbf{Dataset}: Materials Project Database
        \item \textbf{Source}: \href{https://materialsproject.org/}{Materials Project}
        \item \textbf{Description}: Predict material properties using Gaussian process regression.
    \end{itemize}
    \begin{lstlisting}[language=Python]
from sklearn.gaussian_process import GaussianProcessRegressor

gp = GaussianProcessRegressor()
gp.fit(X_train, y_train)
predictions = gp.predict(X_test)
    \end{lstlisting}
\end{frame}

%%%%%%%%%%%%%%%%%%%%%%%%%%%%%%%%%%%%%%%%%%%%%%%%%%%%%%%%%%%
% Slide 9: Load Prediction in Structural Engineering using Random Forests
%%%%%%%%%%%%%%%%%%%%%%%%%%%%%%%%%%%%%%%%%%%%%%%%%%%%%%%%%%%
\begin{frame}[fragile]\frametitle{Load Prediction in Structural Engineering using Random Forests}
    \begin{itemize}
        \item \textbf{Dataset}: Structural Load Data
        \item \textbf{Source}: \href{https://archive.ics.uci.edu/ml/datasets/Structural+Load+Prediction}{UCI Machine Learning Repository}
        \item \textbf{Description}: Predict structural loads using random forests.
    \end{itemize}
    \begin{lstlisting}[language=Python]
import pandas as pd
from sklearn.ensemble import RandomForestRegressor

data = pd.read_csv('structural_load_data.csv')
X = data.drop('load', axis=1)
y = data['load']

model = RandomForestRegressor()
model.fit(X, y)
predictions = model.predict(X)
    \end{lstlisting}
\end{frame}

%%%%%%%%%%%%%%%%%%%%%%%%%%%%%%%%%%%%%%%%%%%%%%%%%%%%%%%%%%%
% Slide 10: Simulation of Fluid Dynamics using Deep Learning
%%%%%%%%%%%%%%%%%%%%%%%%%%%%%%%%%%%%%%%%%%%%%%%%%%%%%%%%%%%
\begin{frame}[fragile]\frametitle{Simulation of Fluid Dynamics using Deep Learning}
    \begin{itemize}
        \item \textbf{Dataset}: Fluid Dynamics Simulation Data
        \item \textbf{Source}: \href{https://www.openfoam.com/}{OpenFOAM}
        \item \textbf{Description}: Simulate fluid dynamics using deep learning models.
    \end{itemize}
    \begin{lstlisting}[language=Python]
import tensorflow as tf

model = tf.keras.Sequential([
    tf.keras.layers.Conv2D(32, (3,3), activation='relu', input_shape=(64,64,1)),
    tf.keras.layers.Flatten(),
    tf.keras.layers.Dense(64, activation='relu'),
    tf.keras.layers.Dense(1)
])

model.compile(optimizer='adam', loss='mean_squared_error')
    \end{lstlisting}
\end{frame}
