
%%%%%%%%%%%%%%%%%%%%%%%%%%%%%%%%%%%%%%%%%%%%%%%%%%%%%%%%%%%%%%%%%%%%%%%%%%%%%%%%%%
\begin{frame}[fragile]\frametitle{}
\begin{center}
{\Large Conclusion}
\end{center}
\end{frame}


%%%%%%%%%%%%%%%%%%%%%%%%%%%%%%%%%%%%%%%%%%%%%%%%%%%%%%%%%%%%%%%%%%%%%%%%%%%%%%%%%%
\begin{frame}[fragile]\frametitle{Conclusion}

\begin{itemize}
\item RL helps us to discover which action could yield the highest reward for a longer time. 
\item Realistic environments can have partial observability and be non-stationary as well. 
\item It isn’t very useful to apply when you have hands-on enough data to solve the problem using supervised learning. \item The main challenge of this method is that parameters could affect the speed of the learning.
\end{itemize}



\end{frame}

%%%%%%%%%%%%%%%%%%%%%%%%%%%%%%%%%%%%%%%%%%%%%%%%%%%%%%%%%%%%%%%%%%%%%%%%%%%%%%%%%%
\begin{frame}[fragile]\frametitle{How to get started?}

\begin{itemize}
\item Reinforcement Learning-An Introduction, a book by the father of Reinforcement Learning- Richard Sutton and his doctoral advisor Andrew Barto. An online draft available.
\item Course by David Silver including video lectures.
\item Technical tutorial by Pieter Abbeel and John Schulman (Open AI/ Berkeley AI Research Lab).
\item "Deep Reinforcement Learning: Pong from Pixels" - Andrej Karpathy blog
\end{itemize}

\end{frame}
