%%%%%%%%%%%%%%%%%%%%%%%%%%%%%%%%%%%%%%%%%%%%%%%%%%%%%%%%%%%%%%%%%%%%%%%%%%%%%%%%%%
  \begin{frame}[fragile]\frametitle{}
\begin{center}
{\Large Linear Equations }
\end{center}
\end{frame}


%%%%%%%%%%%%%%%%%%%%%%%%%%%%%%%%%%%%%%%%%%%%%%%%%%%%%%%%%%%
  \begin{frame}[fragile]\frametitle{Linear Equations}

Consider the simultaneous equations
\begin{eqnarray*}
x + 2y &=& 4\\
3x-5y &=& 1 
\end{eqnarray*}
Provided you understand how matrices are multiplied together you will realize
that these can be written in matrix form as
$$
\left[
\begin{array}{cc}
1 & 2 \\
3 & -5 \\
\end{array}
\right]\left[
\begin{array}{c}
x \\
y\\
\end{array}\right] =\left[
\begin{array}{c}
4 \\
1\\
\end{array} 
\right]
$$
Writing
$$
A = \left[
\begin{array}{cc}
1 & 2 \\
3 & -5 \\
\end{array}
\right], X = \left[
\begin{array}{c}
x \\
y\\
\end{array}\right], \hbox { and } B=\left[
\begin{array}{c}
4 \\
1\\
\end{array} 
\right]
$$
we have
$$
AX=B
$$
\end{frame}


%%%%%%%%%%%%%%%%%%%%%%%%%%%%%%%%%%%%%%%%%%%%%%%%%%%%%%%%%%%
  \begin{frame}[fragile]\frametitle{Linear Equations}

We need to calculate the inverse of $A=\left[
\begin{array}{cc}
1 & 2 \\
3 & -5 \\
\end{array}
\right]$.

\begin{eqnarray*}
A^{-1} &=& \frac{1}{(1)(-5)-(2)(3)}\left[
\begin{array}{cc}
-5 & -2 \\
-3 & 1 \\
\end{array}
\right] \\
&=& -\frac{1}{11}\left[
\begin{array}{cc}
-5 & -2 \\
-3 & 1 \\
\end{array}
\right] 
\end{eqnarray*}
Then $X$ is given by
\begin{eqnarray*}
X = A^{-1}B &=&-\frac{1}{11}\left[
\begin{array}{cc}
-5 & -2 \\
-3 & 1 \\
\end{array}
\right]  \left[
\begin{array}{c}
4 \\
1\\
\end{array} 
\right] \\
&=& -\frac{1}{11}\left[
\begin{array}{c}
-22\\
-11\\
\end{array} 
\right]
= \left[
\begin{array}{c}
2\\
1\\
\end{array} 
\right]
\end{eqnarray*}
Hence $x=2$, $y=1$ is the solution of the simultaneous equations.

\end{frame}

%%%%%%%%%%%%%%%%%%%%%%%%%%%%%%%%%%%%%%%%%%%%%%%%%%%%%%%%%%%
  \begin{frame}[fragile]\frametitle{Linear Equations}
$m$ linear equations of $n$ variables.

$a_{11} x_1 + a_{12}x_2 + \ldots + a_{1n} = u_1$\\
$a_{21} x_1 + a_{22}x_2 + \ldots + a_{2n} = u_2$\\
\ldots \\
$a_{m1} x_1 + a_{m2}x_2 + \ldots + a_{mn} = u_m$ \\

Can be written in matrix form as :

$$ \left[ \begin{array}{ccc} a_{11} \hspace{1mm} a_{12} \hspace{1mm}
                                    \ldots \hspace{1mm} a_{1n} \\
                              a_{21} \hspace{1mm} a_{22} \hspace{1mm} 
                                     \ldots \hspace{1mm} a_{2n} \\
                              \vdots \hspace{6mm} \vdots \hspace{11mm} \vdots \\
                              a_{m1} \hspace{1mm} a_{m2} \hspace{1mm}
                                     \ldots \hspace{1mm} a_{mn}
              \end{array}  \right]   \left[ \begin{array}{c} x_{1} \\ x_2 \\ \vdots \\ x_m               \end{array}  \right] 
                =  \left[ \begin{array}{c} u_{1} \\ u_2 \\ \vdots \\ u_m               \end{array}  \right]                $$
\end{frame}

%%%%%%%%%%%%%%%%%%%%%%%%%%%%%%%%%%%%%%%%%%%%%%%%%%%%%%%%%%%
  \begin{frame}[fragile]\frametitle{Example}
\begin{itemize}
\item $x$: price of pencil
\item $y$: price of pen
\item $u$: box 1
\item $v$: box 2
\item Set I: \begin{align} 2x + 3y = u \\ 3x+4y = v\end{align}
\item Set II: \begin{align} 2u + v = 57 \\ 3u+2v = 97\end{align}
\end{itemize}
\end{frame}

%%%%%%%%%%%%%%%%%%%%%%%%%%%%%%%%%%%%%%%%%%%%%%%%%%%%%%%%%%%
  \begin{frame}[fragile]\frametitle{Example}
II by putting $u,v$ from I

\begin{align} 2(2x + 3y) + 1(3x+4y) = 57 \\  3(2x + 3y) + 2(3x+4y) = 97\end{align}
Collecting $x,y$
\begin{align} [(2x2) + (3x1)]x + [(2x3) + (4x1)]y = 57 \\   [(2x2) + (3x1)]x + [(2x3) + (4x1)]y = 97\end{align}

$$
\left[
\begin{array}{cc}
2 & 1 \\
3 & 2 \\
\end{array}
\right]
\left[
\begin{array}{cc}
2 & 3 \\
3 & 4 \\
\end{array}
\right]\left[
\begin{array}{c}
x \\
y\\
\end{array}\right] =\left[
\begin{array}{c}
57 \\
97\\
\end{array} 
\right]
$$
\end{frame}


%%%%%%%%%%%%%%%%%%%%%%%%%%%%%%%%%%%%%%%%%%%%%%%%%%%%%%%%%%%
  \begin{frame}[fragile]\frametitle{Solving Simultaneous Equations}
$$ \left[ \begin{array}{ccc} a_1 \hspace{1mm} a_2 \hspace{1mm} \hspace{1mm} a_3 \\
                              a_4 \hspace{1mm} a_5 \hspace{1mm} \hspace{1mm} a_6 \\
                             a_7 \hspace{1mm} a_8 \hspace{1mm} \hspace{1mm} a_9\end{array}  \right]   
                             \left[ \begin{array}{c} x \\ y\\z\end{array}  \right] 
                =  \left[ \begin{array}{c} c_1\\ c_2  \\ c_3              \end{array}  \right]                $$
                
Make augmented matrix
$$ \left[ \begin{array}{cccc} a_1 \hspace{1mm} a_2 \hspace{1mm} \hspace{1mm} a_3  \hspace{1mm} |c_1\\
                              a_4 \hspace{1mm} a_5 \hspace{1mm} \hspace{1mm} a_6  \hspace{1mm} |c_2 \\
                             a_7 \hspace{1mm} a_8 \hspace{1mm} \hspace{1mm} a_9  \hspace{1mm} |c_3 \end{array}  \right]   $$

Row operations permitted
\begin{itemize}
\item Interchange of rows
\item Multiply by scalar
\item Add rows
\end{itemize}     
Using this make lower triable 0, and get the answers.           
\end{frame}

%%%%%%%%%%%%%%%%%%%%%%%%%%%%%%%%%%%%%%%%%%%%%%%%%%%%%%%%%%%
  \begin{frame}[fragile]\frametitle{Assignment}
\begin{itemize}
\item Q: Similar to row transformations can we apply column transformations as well? Difference?
\item A: We can use column transformations as well but in that case the $x$ vector changes. Eg. In case of $C_1 \Longleftrightarrow C_3$, $\left[
\begin{array}{c}
x_3 \\ x_2\\ x_1
\end{array} 
\right]$ is the $x$ vector.
\end{itemize}
\end{frame}

%%%%%%%%%%%%%%%%%%%%%%%%%%%%%%%%%%%%%%%%%%%%%%%%%%%%%%%%%%%
  \begin{frame}[fragile]\frametitle{}
A system of linear equations is a collection of equations in the same set of variables. 

 For example,
\begin{equation*}
\begin{cases}
 x_1 + 3 x_2 &=5\\
 2x_1 -x_2 &= -4
\end{cases}
\end{equation*}

\begin{itemize}
\item Pair of equations in two variables
\item Think of this as the intersection of two lines.
 \item Sketch these lines and find their intersection.
\end{itemize}
\end{frame}

%%%%%%%%%%%%%%%%%%%%%%%%%%%%%%%%%%%%%%%%%%%%%%%%%%%%%%%%%%%
  \begin{frame}[fragile] \frametitle{Solution Sets} 
The solution set for a linear system of equations can be:

\begin{itemize}
 \item Empty. (there is no solution.)
 \item Exactly one solution
 \item More than one solution
\end{itemize}

\end{frame}

%%%%%%%%%%%%%%%%%%%%%%%%%%%%%%%%%%%%%%%%%%%%%%%%%%%%%%%%%%%
  \begin{frame}[fragile] \frametitle{Solution Sets} 
\begin{itemize}
 \item It is easy to come up with examples of the first two circumstances.  
 \item The third possibility actually means that something much stronger is true.
\item If we have two distinct solutions then we must have infinitely many solutions.
\item Why is it that if we have at least two solutions then there are infinitely many?
\end{itemize}

\textbf {To Do}
Draw two dimensional examples illustrating each of the three possible outcomes. 


\end{frame}




%%%%%%%%%%%%%%%%%%%%%%%%%%%%%%%%%%%%%%%%%%%%%%%%%%%%%%%%%%%
  \begin{frame}[fragile]  \frametitle{Matrix Notation}
The system of equations above has coefficient matrix
$$
\begin{bmatrix}
 1 & 3 \\
 2 & -1
\end{bmatrix}
$$


and augmented matrix 
$$
\begin{bmatrix}
 1 & 3 & 5\\
 2 & -1 & -4
\end{bmatrix}
$$

\end{frame}




%%%%%%%%%%%%%%%%%%%%%%%%%%%%%%%%%%%%%%%%%%%%%%%%%%%%%%%%%%%
  \begin{frame}[fragile]\frametitle{Elementary row operations}

\textbf {To Do}
 Subtract twice the first row from the second row of the augmented matrix.
 Explain why the new system of equations has precisely the same set of solutions
 as the old set of equations.



\textbf{Operations}
here are 3 types of elementary row operations.
\begin{itemize}
 \item (Replacement) Replace a row by the sum of {\em the same row} and a multiple of {\em different} row. 
 \item (Interchange)  Interchange two rows.
 \item (Scaling) Multiply a row by a non-zero constant.
\end{itemize}


\end{frame}


%%%%%%%%%%%%%%%%%%%%%%%%%%%%%%%%%%%%%%%%%%%%%%%%%%%%%%%%%%%
  \begin{frame}[fragile]\frametitle{}
\textbf{Notation}
\begin{itemize}
\item (Replacement) The operation of replacing row $i$ by itself plus  $c$ times row $j$ will be denoted $R_i + cR_j$.
\item (Interchange)  The operation of interchanging rows $i$ and $j$ will be denoted by $R_i \leftrightarrow R_j$.
\item (Scaling) The operation of scaling row $i$ by a nonzero constant $c$ will be denoted by $cR_i$.
\end{itemize}


\end{frame}


%%%%%%%%%%%%%%%%%%%%%%%%%%%%%%%%%%%%%%%%%%%%%%%%%%%%%%%%%%%
  \begin{frame}[fragile]\frametitle{}

It is clear that the set of solutions is not changed by the second and third operations.  



\textbf{To Do}
 Explain why the set of solutions is unchanged by the first operation.



\end{frame}

%%%%%%%%%%%%%%%%%%%%%%%%%%%%%%%%%%%%%%%%%%%%%%%%%%%%%%%%%%%
  \begin{frame}[fragile]\frametitle{Equivalence of Linear Systems}
\textbf{Definition}
\begin{itemize}
 \item Two linear systems are \textbf{equivalent} if they have the same solution set.
 \item Two matrices are \textbf{row equivalent} if one matrix can be obtained from the other
 by elementary row operations.
\end{itemize}



\textbf{Principle}  Suppose $A$ and $B$ are $m \times n$ matrices: explain why the following 
 is true: if $A$ can be reduced via elementary row operations to $B$, then $B$ can be reduced
 to $A$. 
 



The key fact to notice is that these operations can be undone or inverted.

\end{frame}

%%%%%%%%%%%%%%%%%%%%%%%%%%%%%%%%%%%%%%%%%%%%%%%%%%%%%%%%%%%
  \begin{frame}[fragile]\frametitle{}
Two linear systems are equivalent if and only if their augmented matrices
are row-equivalent.



 Write the following system of equations as an augmented matrix and solve the system:
 \[
\begin{array}{rcrcrcr}
  x_1  & + &       & + & -3 x_3 & =  & 8 \\
  2x_1 & + & 2 x_2 & + & 9 x_3  & =  & 7 \\
       & + &  x_2  & + & 5 x_3  & =  & -2 
\end{array}
\]

\end{frame}


%%%%%%%%%%%%%%%%%%%%%%%%%%%%%%%%%%%%%%%%%%%%%%%%%%%%%%%%%%%
  \begin{frame}[fragile]\frametitle{Fundamental Questions}

We can rephrase our question about how many solutions there are to a system in the following way.


\begin{itemize}
 \item Is the system consistent, that is, do there exist {\em any} solutions?
 \item If there is at least one solution, is it unique?
\end{itemize}

\end{frame}

%%%%%%%%%%%%%%%%%%%%%%%%%%%%%%%%%%%%%%%%%%%%%%%%%%%%%%%%%
  \begin{frame}[fragile]\frametitle{Fundamental Questions}

For example, we've seen that the system 
\begin{eqnarray*}
 x_1 + 3 x_2 &=&5\\
 2x_1 -x_2 &=& -4
\end{eqnarray*}
has a unique solution. 
On the other hand, the system 
\begin{eqnarray*}
 x_1 + 3 x_2 &=&5\\
 2x_1 +6 x_2 &=& -4
\end{eqnarray*}
has no solutions,
 and the system
\begin{eqnarray*}
 x_1 + 3 x_2 &=&5\\
 2x_1 +6 x_2 &=& 10
\end{eqnarray*}
has infinitely many.

\end{frame}

%%%%%%%%%%%%%%%%%%%%%%%%%%%%%%%%%%%%%%%%%%%%%%%%%%%%%%%%%%%
  \begin{frame}[fragile]\frametitle{Row Reduction and Row Echelon Form}
 A rectangular matrix is in \textbf{echelon form}
(or \textbf{row echelon form}) if it has the following properties
\begin{itemize}
 \item All non-zero rows are above any rows of all zeros.
 \item Each leading entry of a row is in a column to the right of the leading
 entry of the row above it.
 \item All entries in a column below a leading entry are zeros.
\end{itemize}

\end{frame}

%%%%%%%%%%%%%%%%%%%%%%%%%%%%%%%%%%%%%%%%%%%%%%%%%%%%%%%%%%%
  \begin{frame}[fragile]\frametitle{Row Reduction and Row Echelon Form}
\begin{itemize}
 \item Sometimes we want more than echelon form. 
 \item  We can make all the leading 
entries 1 by multiplying by a constant, and we can subtract from rows 
above to zero out their entries in that column.
\end{itemize}

\textbf{Definition} A rectangular matrix is in \textbf{reduced echelon form}
(or \textbf{reduced row echelon form}) if it is in row echelon form, and 
 has the following additional properties
\begin{itemize}
 \item The leading non-zero term of every non-zero row is 1
 \item Each leading 1 is the only non-zero entry in its column.
\end{itemize}

\end{frame}

%%%%%%%%%%%%%%%%%%%%%%%%%%%%%%%%%%%%%%%%%%%%%%%%%%%%%%%%%%%
  \begin{frame}[fragile]\frametitle{Row Reduction and Row Echelon Form}
\noindent{\bf Examples} 
\[
\begin{bmatrix}
 1 & 2 & 5 \\
 0 & 2 & 6 \\
 0 & 0 & 0
\end{bmatrix}
\]

is in row echelon form but is not reduced:
\[
\begin{bmatrix}
 1 & 2 & 5 \\
 0 & 1 & 3 \\
 0 & 0 & 0
\end{bmatrix}
\] 
is in row echelon form but is not reduced:
\end{frame}

%%%%%%%%%%%%%%%%%%%%%%%%%%%%%%%%%%%%%%%%%%%%%%%%%%%%%%%%%%%
  \begin{frame}[fragile]\frametitle{Row Reduction and Row Echelon Form}
\noindent{\bf Examples} 
\[
\begin{bmatrix}
 1 & 0 & -1 \\
 0 & 1 & 3 \\
 0 & 0 & 0
\end{bmatrix}
\] 
is in reduced row echelon form:
\[
\begin{bmatrix}
 1 & 2 & 5 \\
 0 & 1 & 3 \\
 2 & 0 & 0
\end{bmatrix}
\] 
is not in row echelon form.
\end{frame}

%%%%%%%%%%%%%%%%%%%%%%%%%%%%%%%%%%%%%%%%%%%%%%%%%%%%%%%%%%%
  \begin{frame}[fragile]\frametitle{Row Reduction and Row Echelon Form}

\textbf{Solving a system in reduced row echelon form}
Suppose we have a system of equations, we've written them as an 
augmented matrix, we've performed elementary row operations, and arrived
at the following reduced row echelon form matrix.  
\[
 \begin{bmatrix}
  1 & 6 & 0 & 3 & 0 & 0 \\
  0 & 0 & 1 & -4 & 0 & 5\\
  0 & 0 & 0 & 0 & 1 & 7 
 \end{bmatrix}
 \]
 
 
\begin{itemize}
 \item  Write down the corresponding equations.
 \item  Pair up the variables and the pivot columns: these are the \textbf{basic variables}.
 \item  The remaining variables are the \textbf{free variables}.
\end{itemize}

\end{frame}

%%%%%%%%%%%%%%%%%%%%%%%%%%%%%%%%%%%%%%%%%%%%%%%%%%%%%%%%%%%
  \begin{frame}[fragile]\frametitle{Great News}

\begin{itemize}
 \item  Each matrix is row equivalent to one and only one reduced echelon matrix.
\item Gaussian elimination: How do we get a matrix into row echelon or reduced row echelon form?
\end{itemize}
\end{frame}
