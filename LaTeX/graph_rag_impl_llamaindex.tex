%%%%%%%%%%%%%%%%%%%%%%%%%%%%%%%%%%%%%%%%%%%%%%%%%%%%%%%%%%%%%%%%%%%%%%%%%%%%%%%%%%
\begin{frame}[fragile]\frametitle{}
\begin{center}
{\Large Implementation using LlamaIndex}
\end{center}
\end{frame}

%%%%%%%%%%%%%%%%%%%%%%%%%%%%%%%%%%%%%%%%%%%%%%%%%%%%%%%%%%%
\begin{frame}[fragile]\frametitle{What is Graph RAG?}
  \begin{itemize}
    \item Enhances traditional RAG by incorporating knowledge graphs.
    \item Moves beyond vector similarity to structured entity relationships.
    \item Enables LLMs to reason over graph-based data structures.
  \end{itemize}
\end{frame}



























%%%%%%%%%%%%%%%%%%%%%%%%%%%%%%%%%%%%%%%%%%%%%%%%%%%%%%%%%%%
\begin{frame}[fragile]\frametitle{Conclusion}
  \begin{itemize}
    \item Graph RAG integrates structured knowledge into LLM responses.
    \item Enhances the depth and accuracy of information retrieval.
    \item Suitable for complex, relationship-rich data domains.
  \end{itemize}
\end{frame}



