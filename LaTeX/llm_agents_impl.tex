%%%%%%%%%%%%%%%%%%%%%%%%%%%%%%%%%%%%%%%%%%%%%%%%%%%%%%%%%%%%%%%%%%%%%%%%%%%%%%%%%%
\begin{frame}[fragile]\frametitle{}
\begin{center}
{\Large Implementation}
\end{center}
\end{frame}

%%%%%%%%%%%%%%%%%%%%%%%%%%%%%%%%%%%%%%%%%%%%%%%%%%%%%%%%%%%%%%%%%%%%%%%%%%%%%%%%%%
\begin{frame}[fragile]\frametitle{}
\begin{center}
{\Large Frameworks}

{\tiny (Ref: Vizuara AI Agents Bootcamp)}
\end{center}
\end{frame}


%%%%%%%%%%%%%%%%%%%%%%%%%%%%%%%%%%%%%%%%%%%%%%%%%%%%%%%%%%%
\begin{frame}[fragile]\frametitle{}

\begin{center}
\includegraphics[width=\linewidth,keepaspectratio]{aiagents35}

{\tiny (Ref: Vizuara AI Agents Bootcamp)}
\end{center}	
  
\end{frame}

%%%%%%%%%%%%%%%%%%%%%%%%%%%%%%%%%%%%%%%%%%%%%%%%%%%%%%%%%%%
\begin{frame}[fragile]\frametitle{LangChain: The OG Agent Framework}
      \begin{itemize}
	  \item Open-source library connecting LLMs with data and APIs
	  \item Launched late 2022, fastest-growing GitHub project by mid-2023
	  \item Provides modular components: prompts, memory, tools, interfaces
	  \item Enables building chatbots to autonomous agents
	  \item General-purpose toolkit for LLM applications
	  \item Used by thousands of developers worldwide
	  \item Chain-based approach for sequential operations
	  \item Foundation for many other agent frameworks
	  \end{itemize}
\end{frame}

%%%%%%%%%%%%%%%%%%%%%%%%%%%%%%%%%%%%%%%%%%%%%%%%%%%%%%%%%%%
\begin{frame}[fragile]\frametitle{LangGraph: Graph-Based Agent Orchestration}
      \begin{itemize}
	  \item Newer framework from LangChain team with graph approach
	  \item Defines graphs of steps and decision points vs linear chains
	  \item Enables complex workflows with conditional branching
	  \item More low-level and controllable than LangChain agents
	  \item Charts exactly how AI navigates tasks like a map
	  \item Handles multiple possible sub-tasks efficiently
	  \item Ideal for complex decision flows and routing
	  \item Provides granular control over agent behavior
	  \end{itemize}
\end{frame}

%%%%%%%%%%%%%%%%%%%%%%%%%%%%%%%%%%%%%%%%%%%%%%%%%%%%%%%%%%%
\begin{frame}[fragile]\frametitle{LlamaIndex: LLMs + Your Data}
      \begin{itemize}
	  \item Formerly GPT Index, specializes in retrieval-augmented generation
	  \item Go-to framework for connecting LLMs with external datasets
	  \item Indexes knowledge bases for intelligent fact retrieval
	  \item Pre-built agents for question-answering scenarios
	  \item Integrates with databases, vector stores, and APIs
	  \item Enables agents to access specific data beyond base LLM knowledge
	  \item Powers document-based and enterprise knowledge systems
	  \item Essential for data-grounded AI applications
	  \end{itemize}
\end{frame}

%%%%%%%%%%%%%%%%%%%%%%%%%%%%%%%%%%%%%%%%%%%%%%%%%%%%%%%%%%%
\begin{frame}[fragile]\frametitle{SmolAgents: Minimalist Code-First Approach}
      \begin{itemize}
	  \item Hugging Face framework emphasizing simplicity and efficiency
	  \item Powerful agents in just a few lines of code
	  \item "Code-as-actions" philosophy: agents write Python directly
	  \item Up to 30\% fewer steps and API calls vs verbose instructions
	  \item Core logic only ~1000 lines of code
	  \item Lightweight and developer-friendly
	  \item Agents execute code for math, web scraping, etc.
	  \item Quick deployment without overhead
	  \end{itemize}
\end{frame}

%%%%%%%%%%%%%%%%%%%%%%%%%%%%%%%%%%%%%%%%%%%%%%%%%%%%%%%%%%%
\begin{frame}[fragile]\frametitle{Autogen: Multi-Agent Communication}
      \begin{itemize}
	  \item Microsoft framework for multi-agent conversations
	  \item Models applications as conversations between specialized agents
	  \item Supports group chats, hierarchical chats, human-in-the-loop
	  \item Agents can execute code in various environments
	  \item Enables AI pair-programming scenarios
	  \item Multiple agents critique and refine solutions collectively
	  \item Ideal for tasks requiring multiple perspectives
	  \item Pushes envelope on agent collaboration
	  \end{itemize}
\end{frame}

%%%%%%%%%%%%%%%%%%%%%%%%%%%%%%%%%%%%%%%%%%%%%%%%%%%%%%%%%%%
\begin{frame}[fragile]\frametitle{LangFlow: Visual LangChain Interface}
      \begin{itemize}
	  \item Web-based graphical UI for building LangChain flows
	  \item Drag-and-drop components instead of coding
	  \item Visualizes agent "thought" processes interactively
	  \item Lowers entry barrier for non-developers
	  \item Rapid prototyping and experimentation tool
	  \item Abstracts LangChain complexity for convenience
	  \item Configure prompts, models, memory through UI
	  \item Same capabilities as LangChain, visual interface
	  \end{itemize}
\end{frame}

%%%%%%%%%%%%%%%%%%%%%%%%%%%%%%%%%%%%%%%%%%%%%%%%%%%%%%%%%%%
\begin{frame}[fragile]\frametitle{CrewAI: Team-Based Agent Collaboration}
      \begin{itemize}
	  \item Framework inspired by human team collaboration
	  \item Sets up multiple agents with different roles in "crews"
	  \item High-level configuration over coding approach
	  \item Example: Researcher agent + Writer agent collaboration
	  \item Abstractions for tasks, planning, memory types
	  \item Developer-friendly with common patterns handled
	  \item Semi-finished assembly approach vs raw toolkits
	  \item Excellent for workflows breaking into sub-tasks
	  \end{itemize}
\end{frame}




%%%%%%%%%%%%%%%%%%%%%%%%%%%%%%%%%%%%%%%%%%%%%%%%%%%%%%%%%%%
\begin{frame}[fragile]\frametitle{n8n: Visual Workflow Automation with AI}
      \begin{itemize}
	  \item Open-source automation tool with LLM agent integration
	  \item Drag-and-drop workflow editor like visual Zapier
	  \item AI agent nodes operate within automated workflows
	  \item Maintains conversation context and calls tools
	  \item Visual orchestration mixing AI with enterprise tools
	  \item Example: Auto-analyze support tickets and recommend responses
	  \item Human review integration for quality control
	  \item Conductor for LLM agents in enterprise environments
	  \end{itemize}
\end{frame}

%%%%%%%%%%%%%%%%%%%%%%%%%%%%%%%%%%%%%%%%%%%%%%%%%%%%%%%%%%%
\begin{frame}[fragile]\frametitle{Manus: General-Purpose Autonomous Agent}
      \begin{itemize}
	  \item No-code platform claiming "AI employee" capabilities
	  \item Operates through chat interface with high-level goals
	  \item Plans and executes tasks autonomously over hours/days
	  \item Internal cognition engine with API calls, web browsing
	  \item Continuous operation without supervision
	  \item Internal feedback loop and memory for progress tracking
	  \item Actively uses tools: code execution, web queries, file operations
	  \item Beyond typical ChatGPT: keeps going until task completion
	  \end{itemize}
\end{frame}

%%%%%%%%%%%%%%%%%%%%%%%%%%%%%%%%%%%%%%%%%%%%%%%%%%%%%%%%%%%
\begin{frame}[fragile]\frametitle{Framework Recommendations by Category}
      \begin{itemize}
	  \item Code-level: LangGraph (graph orchestration), LangChain (general toolkit)
	  \item Code-level: LlamaIndex (data integration), SmolAgents (minimalist)
	  \item Low-code: CrewAI (team collaboration), LangFlow (visual workflows)
	  \item Low-code: n8n (enterprise automation), Agno (full-stack performance)
	  \item No-code: Manus (autonomous agents), OpenAI Deep Research
	  \item No-code: Gemini Deep Research for specialized tasks
	  \item Choose based on technical expertise and use case complexity
	  \item Consider team skills, maintenance requirements, and scalability needs
	  \end{itemize}
\end{frame}


%%%%%%%%%%%%%%%%%%%%%%%%%%%%%%%%%%%%%%%%%%%%%%%%%%%%%%%%%%%
\begin{frame}[fragile]\frametitle{Langfuse: Tracing and Evaluation}
% \begin{columns}
    % \begin{column}[T]{0.6\linewidth}
      \begin{itemize}
		\item Complete Execution Tracing: Records every agent step and tool usage
		\item Timeline Visualization: Shows sequence of operations and model calls
		\item Agent Message Logging: Captures prompts and responses between agents
		\item Tool Usage Tracking: Monitors exact queries and results from each tool
		\item Performance Metrics: Measures latency, token usage, and quality scores
		\item Production Monitoring: Essential for reliable agent deployment
	  \end{itemize}
    % \end{column}
    % \begin{column}[T]{0.4\linewidth}
		\begin{center}
		\includegraphics[width=0.8\linewidth,keepaspectratio]{aiagents58}
		
		{\tiny (Ref: Vizuara AI Agents Bootcamp)}
		\end{center}	
    % \end{column}
  % \end{columns}
\end{frame}


%%%%%%%%%%%%%%%%%%%%%%%%%%%%%%%%%%%%%%%%%%%%%%%%%%%%%%%%%%%
\begin{frame}[fragile]\frametitle{Arize Phoenix: Production Observability}
% \begin{columns}
    % \begin{column}[T]{0.6\linewidth}
      \begin{itemize}
		\item Open-Source Platform: LLM observability and evaluation for production systems
		\item Trace Visualization: Tree view of agent actions and reasoning steps
		\item Performance Metrics: Latency, token count, and quality score tracking
		\item Error Detection: Identifies problematic spans and execution failures
		\item Callback Integration: Easy setup with LlamaIndex callback handlers
		\item Production Monitoring: Essential for trust and reliability in deployed agents
	  \end{itemize}
    % \end{column}
    % \begin{column}[T]{0.4\linewidth}
		\begin{center}
		\includegraphics[width=0.6\linewidth,keepaspectratio]{aiagents63}
		\end{center}	
    % \end{column}
  % \end{columns}
\end{frame}

%%%%%%%%%%%%%%%%%%%%%%%%%%%%%%%%%%%%%%%%%%%%%%%%%%%%%%%%%%%
\begin{frame}[fragile]\frametitle{Future Directions and Extensions}
% \begin{columns}
    % \begin{column}[T]{0.6\linewidth}
      \begin{itemize}
		\item Multi-Agent Orchestration: Multiple specialized agents working together
		\item Real-Time Data Integration: Live APIs and streaming data sources
		\item Advanced Tool Ecosystem: Expanding LlamaHub with custom tools
		\item Workflow Automation: Complex business process automation with agent chains
		\item Quality Assurance: Automated evaluation and continuous improvement loops
		\item Production Scaling: Enterprise deployment with monitoring and governance
	  \end{itemize}
    % \end{column}
    % \begin{column}[T]{0.4\linewidth}
		% \begin{center}
		% \includegraphics[width=0.8\linewidth,keepaspectratio]{future_directions}
		% \end{center}	
    % \end{column}
  % \end{columns}
\end{frame}

%%%%%%%%%%%%%%%%%%%%%%%%%%%%%%%%%%%%%%%%%%%%%%%%%%%%%%%%%%%
\begin{frame}[fragile]\frametitle{}

\begin{center}
\includegraphics[width=\linewidth,keepaspectratio]{aiagents5}

{\tiny (Ref: The Ultimate Guide to AI Agents for PMs - Pawl Huryn)}
\end{center}	
  
\end{frame}

%%%%%%%%%%%%%%%%%%%%%%%%%%%%%%%%%%%%%%%%%%%%%%%%%%%%%%%%%%%
\begin{frame}[fragile]\frametitle{Build Your First AI Agent}
      \begin{itemize}
        \item Takes just 30-60 minutes to get started
        \item Follow a clear step-by-step process
        \item Focus on functionality, not theory
      \end{itemize}
\end{frame}

%%%%%%%%%%%%%%%%%%%%%%%%%%%%%%%%%%%%%%%%%%%%%%%%%%%%%%%%%%%%%%%%%%%%%%%%%%%%%%%%%%
\begin{frame}[fragile]\frametitle{How to Build an AI Agent?}

	\begin{center}
	\includegraphics[width=0.8\linewidth,keepaspectratio]{aiagents7}
	\end{center}
	
	{\tiny (Ref: The Ultimate Guide to AI Agents for PMs - Pawl Huryn)}
\end{frame}


%%%%%%%%%%%%%%%%%%%%%%%%%%%%%%%%%%%%%%%%%%%%%%%%%%%%%%%%%%%
\begin{frame}[fragile]\frametitle{Step 1: Define a System Prompt}
      \begin{itemize}
        \item Set goals, logic, and expectations
        \item Use structured prompting principles
        \item Refer to expert guides for inspiration
      \end{itemize}
\end{frame}

%%%%%%%%%%%%%%%%%%%%%%%%%%%%%%%%%%%%%%%%%%%%%%%%%%%%%%%%%%%
\begin{frame}[fragile]\frametitle{Step 2: Select an LLM}
      \begin{itemize}
        \item Choose a reasoning-capable model (e.g., o1-mini)
        \item Frameworks like n8n may handle iterations
        \item Pick based on your use case complexity
      \end{itemize}
\end{frame}

%%%%%%%%%%%%%%%%%%%%%%%%%%%%%%%%%%%%%%%%%%%%%%%%%%%%%%%%%%%
\begin{frame}[fragile]\frametitle{Step 3: Connect Tools}
      \begin{itemize}
        \item Add tools based on agent goals
        \item Use calculators, functions, data sources
        \item Optional: MCP servers for integration
      \end{itemize}
\end{frame}

%%%%%%%%%%%%%%%%%%%%%%%%%%%%%%%%%%%%%%%%%%%%%%%%%%%%%%%%%%%
\begin{frame}[fragile]\frametitle{Step 4: Connect Memory}
      \begin{itemize}
        \item Enable short-term memory for local state
        \item Use long-term memory: vector, SQL, graph
        \item Essential for tracking progress and context
      \end{itemize}
\end{frame}

%%%%%%%%%%%%%%%%%%%%%%%%%%%%%%%%%%%%%%%%%%%%%%%%%%%%%%%%%%%
\begin{frame}[fragile]\frametitle{Step 5: Orchestrate the Logic}
      \begin{itemize}
        \item Map core logic not tied to a single agent
        \item Enable agent-to-agent communication
        \item Static or dynamic flows via orchestration
      \end{itemize}
\end{frame}

%%%%%%%%%%%%%%%%%%%%%%%%%%%%%%%%%%%%%%%%%%%%%%%%%%%%%%%%%%%
\begin{frame}[fragile]\frametitle{Step 6: Add User Interface}
      \begin{itemize}
        \item Use no-code tools like Lovable or Bolt
        \item Easily create interfaces without coding
        \item Great for user-facing agents and SaaS apps
      \end{itemize}
\end{frame}

%%%%%%%%%%%%%%%%%%%%%%%%%%%%%%%%%%%%%%%%%%%%%%%%%%%%%%%%%%%
\begin{frame}[fragile]\frametitle{Step 7: Evaluate the Agent}
      \begin{itemize}
        \item Skip fixed metrics-do error analysis
        \item Let performance metrics emerge naturally
        \item For RAG: evaluate retrieval and generation separately
      \end{itemize}
\end{frame}

%%%%%%%%%%%%%%%%%%%%%%%%%%%%%%%%%%%%%%%%%%%%%%%%%%%%%%%%%%%
\begin{frame}[fragile]\frametitle{Bottom Line: Just Start}
      \begin{itemize}
        \item Follow practical guides-many require no coding
        \item Use frameworks like n8n and tools like MCP
        \item Stop theorizing-start shipping real systems
      \end{itemize}
\end{frame}


%%%%%%%%%%%%%%%%%%%%%%%%%%%%%%%%%%%%%%%%%%%%%%%%%%%%%%%%%%%%%%%%%%%%%%%%%%%%%%%%%%
\begin{frame}[fragile]\frametitle{}
\begin{center}
{\Large Practical Tips}
\end{center}
\end{frame}

%%%%%%%%%%%%%%%%%%%%%%%%%%%%%%%%%%%%%%%%%%%%%%%%%%%%%%%%%%%
\begin{frame}[fragile]\frametitle{RAG vs. Fine-Tuning: Business Impact}
    \begin{itemize}
        \item Business-critical systems require the right LLM strategy.
        \item Fine-tuning feels powerful, but often adds avoidable complexity.
        \item Start by exploring simpler and more flexible approaches first.
    \end{itemize}
\end{frame}

%%%%%%%%%%%%%%%%%%%%%%%%%%%%%%%%%%%%%%%%%%%%%%%%%%%%%%%%%%%
\begin{frame}[fragile]\frametitle{RAG vs SFT}
	
	\begin{center}
	\includegraphics[width=0.8\linewidth,keepaspectratio]{vizuara1}
	\end{center}
	
{\tiny (Ref: LinkedIn post by Raj Dandekar)}

\end{frame}


%%%%%%%%%%%%%%%%%%%%%%%%%%%%%%%%%%%%%%%%%%%%%%%%%%%%%%%%%%%
\begin{frame}[fragile]\frametitle{Pre-Fine-Tuning Checklist}
    \begin{itemize}
        \item Can prompt engineering alone solve the task?
        \item Could Retrieval-Augmented Generation (RAG) help more?
        \item Is there a clear and testable system already in place?
    \end{itemize}
\end{frame}

%%%%%%%%%%%%%%%%%%%%%%%%%%%%%%%%%%%%%%%%%%%%%%%%%%%%%%%%%%%
\begin{frame}[fragile]\frametitle{Why RAG Outperformed Fine-Tuning}
    \begin{itemize}
        \item \textbf{Higher Accuracy}: Grounded answers from relevant context.
        \item \textbf{Fewer Hallucinations}: More reliable than fine-tuned outputs.
        \item \textbf{Dynamic Updates}: Easily update data without retraining models.
    \end{itemize}
\end{frame}

%%%%%%%%%%%%%%%%%%%%%%%%%%%%%%%%%%%%%%%%%%%%%%%%%%%%%%%%%%%
\begin{frame}[fragile]\frametitle{When to Use Fine-Tuning}
    \begin{itemize}
        \item RAG can't solve context window limitations.
        \item Domain-specific tone or behavior is required.
        \item The system is mature enough to absorb added complexity.
    \end{itemize}
\end{frame}

%%%%%%%%%%%%%%%%%%%%%%%%%%%%%%%%%%%%%%%%%%%%%%%%%%%%%%%%%%%
\begin{frame}[fragile]\frametitle{Strategy Summary}
    \begin{itemize}
        \item Prompt engineering solves 30--50\% of tasks.
        \item RAG adds power for another 30--40\%.
        \item Fine-tuning is best for the final 10\% of hard problems.
        \item Always choose the simplest effective approach first.
    \end{itemize}
\end{frame}

%%%%%%%%%%%%%%%%%%%%%%%%%%%%%%%%%%%%%%%%%%%%%%%%%%%%%%%%%%%
\begin{frame}[fragile]\frametitle{A2A MCP ADK}
	
	\begin{center}
	\includegraphics[width=0.4\linewidth,keepaspectratio]{agents7}
	\end{center}
	
{\tiny (Ref: LinkedIn post by Deepak Jaiswal)}

\end{frame}

%%%%%%%%%%%%%%%%%%%%%%%%%%%%%%%%%%%%%%%%%%%%%%%%%%%%%%%%%%%
\begin{frame}[fragile]\frametitle{Agentic AI Protocols Overview}
    \begin{itemize}
        \item A2A, MCP, and ADK are foundational for agentic systems.
        \item Each solves a unique challenge in building autonomous agents.
        \item They work together to enable scalable multi-agent architectures.
    \end{itemize}
\end{frame}

%%%%%%%%%%%%%%%%%%%%%%%%%%%%%%%%%%%%%%%%%%%%%%%%%%%%%%%%%%%
\begin{frame}[fragile]\frametitle{A2A: Agent-to-Agent Protocol}
    \begin{itemize}
        \item Enables agent discovery, delegation, and communication.
        \item Facilitates coordination in distributed multi-agent systems.
        \item Example: Agent A delegates a task to Agent B and receives results.
    \end{itemize}
\end{frame}

%%%%%%%%%%%%%%%%%%%%%%%%%%%%%%%%%%%%%%%%%%%%%%%%%%%%%%%%%%%
\begin{frame}[fragile]\frametitle{MCP: Model Context Protocol}
    \begin{itemize}
        \item Standardizes agent access to tools, APIs, and data.
        \item Ensures consistent, secure interaction with external systems.
        \item Example: Agent queries a database or triggers a payment via MCP.
    \end{itemize}
\end{frame}

%%%%%%%%%%%%%%%%%%%%%%%%%%%%%%%%%%%%%%%%%%%%%%%%%%%%%%%%%%%
\begin{frame}[fragile]\frametitle{ADK: Agent Development Kit}
    \begin{itemize}
        \item Toolkit for building A2A-compliant agents quickly.
        \item Includes libraries and scaffolds for rapid development.
        \item Compatible with frameworks like CrewAI, LangGraph, Semantic Kernel.
    \end{itemize}
\end{frame}

%%%%%%%%%%%%%%%%%%%%%%%%%%%%%%%%%%%%%%%%%%%%%%%%%%%%%%%%%%%
\begin{frame}[fragile]\frametitle{How They Work Together}
    \begin{itemize}
        \item \textbf{MCP}: Makes agents powerful with tool access.
        \item \textbf{A2A}: Enables inter-agent collaboration.
        \item \textbf{ADK}: Simplifies and accelerates agent development.
    \end{itemize}
\end{frame}

%%%%%%%%%%%%%%%%%%%%%%%%%%%%%%%%%%%%%%%%%%%%%%%%%%%%%%%%%%%
\begin{frame}[fragile]\frametitle{Real-World Analogy}
    \begin{itemize}
        \item \textbf{MCP}: Tools in a mechanic's toolbox.
        \item \textbf{A2A}: Team communication and task-sharing.
        \item \textbf{ADK}: Blueprint for building each mechanic fast.
    \end{itemize}
\end{frame}

%%%%%%%%%%%%%%%%%%%%%%%%%%%%%%%%%%%%%%%%%%%%%%%%%%%%%%%%%%%
\begin{frame}[fragile]\frametitle{Why This Stack Matters}
    \begin{itemize}
        \item Forms the backbone of enterprise AI, robotics, and automation.
        \item Enables modular, scalable, and intelligent agentic systems.
        \item Quickly becoming a standard for future AI workflows.
    \end{itemize}
\end{frame}


%%%%%%%%%%%%%%%%%%%%%%%%%%%%%%%%%%%%%%%%%%%%%%%%%%%%%%%%%%%%%%%%%%%%%%%%%%%%%%%%%%
\begin{frame}[fragile]\frametitle{}
\begin{center}
{\Large Anthropic :  How to Build AI Agents}

{\tiny (Ref; LinkedIn post by Maryam Miradi)}
\end{center}
\end{frame}

%%%%%%%%%%%%%%%%%%%%%%%%%%%%%%%%%%%%%%%%%%%%%%%%%%%%%%%%%%%
\begin{frame}[fragile]\frametitle{When to Build AI Agents}
    \begin{itemize}
        \item Don’t build agents for every task.
        \item Ideal for complex, ambiguous, high-value problems.
        \item Use workflows when decision paths are clear.
        \item Agents consume tokens—ensure value justifies cost.
        \item Avoid if error discovery is slow or high-risk.
        \item Limit autonomy when safety is critical.
        \item Use checklist: complexity, value, bottlenecks, risks.
        \item Coding is a great fit: complex but verifiable.
    \end{itemize}
\end{frame}

%%%%%%%%%%%%%%%%%%%%%%%%%%%%%%%%%%%%%%%%%%%%%%%%%%%%%%%%%%%
\begin{frame}[fragile]\frametitle{Designing Simple, Scalable Agents}
    \begin{itemize}
        \item Every agent = Model + Tools + Environment.
        \item Start with extremely simple components.
        \item Avoid early complexity—hurts iteration speed.
        \item Reuse agent backbones for many use cases.
        \item Recombine code, tools, and prompts easily.
        \item Don’t optimize until behavior is stable.
        \item Visual clarity builds trust in the agent.
    \end{itemize}
\end{frame}

%%%%%%%%%%%%%%%%%%%%%%%%%%%%%%%%%%%%%%%%%%%%%%%%%%%%%%%%%%%
\begin{frame}[fragile]\frametitle{Optimization \& Performance}
    \begin{itemize}
        \item Parallelize tools to lower latency.
        \item Cache action paths in coding agents to save tokens.
        \item Show step-by-step progress to build trust.
        \item Optimize costs only after validating core loop.
        \item Keep environments simple before scaling up.
    \end{itemize}
\end{frame}

%%%%%%%%%%%%%%%%%%%%%%%%%%%%%%%%%%%%%%%%%%%%%%%%%%%%%%%%%%%
\begin{frame}[fragile]\frametitle{Think Like Your Agent}
    \begin{itemize}
        \item Agent only sees its limited context window.
        \item Don’t expect magic—expect bounded reasoning.
        \item Weird actions often mean missing context.
        \item Simulate tasks from the agent’s view.
        \item Debug using only the agent’s available info.
        \item Poor UI? Add metadata or improve resolution.
        \item Replay full trajectory—ask the model ``why?''
    \end{itemize}
\end{frame}

%%%%%%%%%%%%%%%%%%%%%%%%%%%%%%%%%%%%%%%%%%%%%%%%%%%%%%%%%%%
\begin{frame}[fragile]\frametitle{Tools \& Self-Improvement}
    \begin{itemize}
        \item Define tools with clear inputs and expected outputs.
        \item Use the LLM to review tool clarity.
        \item Let agents self-critique prompts and tools.
        \item Build meta-tools that improve agent tooling.
        \item Better ergonomics reduce errors and retries.
    \end{itemize}
\end{frame}

%%%%%%%%%%%%%%%%%%%%%%%%%%%%%%%%%%%%%%%%%%%%%%%%%%%%%%%%%%%
\begin{frame}[fragile]\frametitle{The Future: Multi-Agent \& Budget-Aware}
    \begin{itemize}
        \item Solo agents dominate now—but not for long.
        \item Multi-agent = modular reasoning + parallel tasks.
        \item Sub-agents help preserve main context window.
        \item Async interactions beat synchronous limitations.
        \item Role-based coordination is the next big step.
        \item Budget-awareness = tokens, time, latency constraints.
        \item Define strict resource limits before deployment.
    \end{itemize}
\end{frame}

%%%%%%%%%%%%%%%%%%%%%%%%%%%%%%%%%%%%%%%%%%%%%%%%%%%%%%%%%%%%%%%%%%%%%%%%%%%%%%%%%%
\begin{frame}[fragile]\frametitle{}
\begin{center}
{\Large Tips \& Tricks}

{\tiny (Ref: Principles of Building AI Agents - Pawel Huryn)}
\end{center}
\end{frame}

%%%%%%%%%%%%%%%%%%%%%%%%%%%%%%%%%%%%%%%%%%%%%%%%%%%%%%%%%%%
\begin{frame}[fragile]\frametitle{Don't Use Agents If You Don't Have To}
\begin{columns}
    \begin{column}[T]{0.6\linewidth}
      \begin{itemize}
		\item Simple scripts and if/else statements are faster and cheaper than AI agents
		\item Traditional solutions are more reliable and predictable for basic tasks
		\item Users care about functionality, not whether it's powered by AI
		\item Agents can become a liability when overused or misapplied
		\item Reserve agents for complex tasks that truly require intelligence
		\item Cost-effectiveness should drive your architectural decisions
		\item Maintenance overhead is significantly lower with simple solutions
	  \end{itemize}
    \end{column}
    \begin{column}[T]{0.4\linewidth}
		\begin{center}
		\includegraphics[width=0.8\linewidth,keepaspectratio]{aiagents81}
		
		{\tiny (Ref: Principles of Building AI Agents - Pawel Huryn)}
		\end{center}	
    \end{column}
  \end{columns}
\end{frame}

%%%%%%%%%%%%%%%%%%%%%%%%%%%%%%%%%%%%%%%%%%%%%%%%%%%%%%%%%%%
\begin{frame}[fragile]\frametitle{Small, Specialized, and Decoupled}
\begin{columns}
    \begin{column}[T]{0.6\linewidth}
      \begin{itemize}
		\item Build a team of specialist agents rather than one monolithic agent
		\item Each agent should have a single, well-defined responsibility
		\item Planners plan, summarizers summarize, verifiers check
		\item Decoupled architecture reduces operational costs significantly
		\item Individual agents are easier to test, debug, and maintain
		\item Predictable behavior emerges from focused specialization
		\item Failure isolation prevents cascading system breakdowns
		\item Scale individual components based on specific needs
	  \end{itemize}
    \end{column}
    \begin{column}[T]{0.4\linewidth}
		\begin{center}
		\includegraphics[width=0.8\linewidth,keepaspectratio]{aiagents82}
		
		{\tiny (Ref: Principles of Building AI Agents - Pawel Huryn)}
		\end{center}		
    \end{column}
  \end{columns}
\end{frame}

%%%%%%%%%%%%%%%%%%%%%%%%%%%%%%%%%%%%%%%%%%%%%%%%%%%%%%%%%%%
\begin{frame}[fragile]\frametitle{Enforce Structured Output}
\begin{columns}
    \begin{column}[T]{0.6\linewidth}
      \begin{itemize}
		\item JSON format is easier to debug than unstructured text
		\item Parsing costs are significantly lower with structured data
		\item Acts as a contract between different agents in the system
		\item Enables automatic validation of agent responses
		\item Prevents errors from propagating through the system
		\item Facilitates integration with downstream processes
		\item Reduces ambiguity in inter-agent communication
		\item Enables better monitoring and quality control
	  \end{itemize}
    \end{column}
    \begin{column}[T]{0.4\linewidth}
		\begin{center}
		\includegraphics[width=0.8\linewidth,keepaspectratio]{aiagents83}
		
		{\tiny (Ref: Principles of Building AI Agents - Pawel Huryn)}
		\end{center}	
    \end{column}
  \end{columns}
\end{frame}

%%%%%%%%%%%%%%%%%%%%%%%%%%%%%%%%%%%%%%%%%%%%%%%%%%%%%%%%%%%
\begin{frame}[fragile]\frametitle{Explain the Why, Not Just the What}
\begin{columns}
    \begin{column}[T]{0.6\linewidth}
      \begin{itemize}
		\item Anthropomorphizing AI improves performance in many contexts
		\item Lead with context rather than rigid control mechanisms
		\item Explain the importance and purpose behind each task
		\item Provide situational context for better decision-making
		\item Helps agents make autonomous decisions with shorter prompts
		\item Reduces the need for extensive prompt engineering
		\item Enables more nuanced and appropriate responses
		\item Improves alignment between agent actions and user intent
	  \end{itemize}
    \end{column}
    \begin{column}[T]{0.4\linewidth}
		\begin{center}
		\includegraphics[width=0.8\linewidth,keepaspectratio]{aiagents84}
		
		{\tiny (Ref: Principles of Building AI Agents - Pawel Huryn)}
		\end{center}	
    \end{column}
  \end{columns}
\end{frame}

%%%%%%%%%%%%%%%%%%%%%%%%%%%%%%%%%%%%%%%%%%%%%%%%%%%%%%%%%%%
\begin{frame}[fragile]\frametitle{Orchestration > Autonomy}
\begin{columns}
    \begin{column}[T]{0.6\linewidth}
      \begin{itemize}
		\item Predictability is more valuable than complete autonomy
		\item Move deterministic logic out of agent prompts
		\item Handle if/then conditions in the orchestration layer
		\item Implement loops, retries, and procedures outside agents
		\item Known business rules should be hardcoded, not learned
		\item Reduces prompt complexity and improves reliability
		\item Enables better debugging and system maintenance
		\item Maintains control while leveraging AI capabilities
	  \end{itemize}
    \end{column}
    \begin{column}[T]{0.4\linewidth}
		\begin{center}
		\includegraphics[width=0.8\linewidth,keepaspectratio]{aiagents85}
		
		{\tiny (Ref: Principles of Building AI Agents - Pawel Huryn)}
		\end{center}	
    \end{column}
  \end{columns}
\end{frame}

%%%%%%%%%%%%%%%%%%%%%%%%%%%%%%%%%%%%%%%%%%%%%%%%%%%%%%%%%%%
\begin{frame}[fragile]\frametitle{Prompt Engineering > Fine Tuning}
\begin{columns}
    \begin{column}[T]{0.6\linewidth}
      \begin{itemize}
		\item Diagnose why the model is failing before considering fine-tuning
		\item Missing facts indicate a need for RAG implementation
		\item Formatting issues may require fine-tuning for brand consistency
		\item 80\% of problems are actually prompt engineering issues
		\item Fine-tuning is expensive and time-consuming to implement
		\item Prompt improvements offer faster iteration cycles
		\item Most performance issues can be solved with better prompts
		\item Fine-tuning should be a last resort, not first choice
	  \end{itemize}
    \end{column}
    \begin{column}[T]{0.4\linewidth}
		\begin{center}
		\includegraphics[width=0.8\linewidth,keepaspectratio]{aiagents86}
		
		{\tiny (Ref: Principles of Building AI Agents - Pawel Huryn)}
		\end{center}	
    \end{column}
  \end{columns}
\end{frame}

%%%%%%%%%%%%%%%%%%%%%%%%%%%%%%%%%%%%%%%%%%%%%%%%%%%%%%%%%%%
\begin{frame}[fragile]\frametitle{Double Down on Tool Descriptions}
\begin{columns}
    \begin{column}[T]{0.6\linewidth}
      \begin{itemize}
		\item Treat tool descriptions as micro-prompts for agent reasoning
		\item Default MCP server descriptions are often insufficient
		\item Customize descriptions for your specific domain context
		\item Specify when and why to use each tool
		\item Include clear examples and usage scenarios
		\item Explain what to avoid when using tools
		\item Document how tools work in combination
		\item Place detailed instructions in agent prompts
	  \end{itemize}
    \end{column}
    \begin{column}[T]{0.4\linewidth}
		\begin{center}
		\includegraphics[width=0.8\linewidth,keepaspectratio]{aiagents87}
		
		{\tiny (Ref: Principles of Building AI Agents - Pawel Huryn)}
		\end{center}	
    \end{column}
  \end{columns}
\end{frame}

%%%%%%%%%%%%%%%%%%%%%%%%%%%%%%%%%%%%%%%%%%%%%%%%%%%%%%%%%%%
\begin{frame}[fragile]\frametitle{Cache Like You Mean It}
\begin{columns}
    \begin{column}[T]{0.6\linewidth}
      \begin{itemize}
		\item Agents often repeat the same tasks on identical data
		\item Web scraping scenarios particularly benefit from caching
		\item Use hash of agent ID plus input as cache key
		\item Dramatically reduces API costs and latency
		\item Implement intelligent cache invalidation strategies
		\item Monitor cache hit rates for optimization opportunities
		\item Consider distributed caching for multi-agent systems
		\item Balance cache storage costs against API savings
	  \end{itemize}
    \end{column}
    \begin{column}[T]{0.4\linewidth}
		\begin{center}
		\includegraphics[width=0.8\linewidth,keepaspectratio]{aiagents88}
		
		{\tiny (Ref: Principles of Building AI Agents - Pawel Huryn)}
		\end{center}	
    \end{column}
  \end{columns}
\end{frame}

%%%%%%%%%%%%%%%%%%%%%%%%%%%%%%%%%%%%%%%%%%%%%%%%%%%%%%%%%%%
\begin{frame}[fragile]\frametitle{Use Shared Artefacts}
\begin{columns}
    \begin{column}[T]{0.6\linewidth}
      \begin{itemize}
		\item Enable agent collaboration through shared documents
		\item Avoid sending attachments between agents
		\item Implement co-editing capabilities for plans and code
		\item Agents often need references, not full content
		\item Reduces data transfer overhead between agents
		\item Enables version control and change tracking
		\item Facilitates parallel work on the same project
		\item Improves overall system efficiency and coordination
	  \end{itemize}
    \end{column}
    \begin{column}[T]{0.4\linewidth}
		\begin{center}
		\includegraphics[width=0.8\linewidth,keepaspectratio]{aiagents89}
		
		{\tiny (Ref: Principles of Building AI Agents - Pawel Huryn)}
		\end{center}	
    \end{column}
  \end{columns}
\end{frame}

%%%%%%%%%%%%%%%%%%%%%%%%%%%%%%%%%%%%%%%%%%%%%%%%%%%%%%%%%%%
\begin{frame}[fragile]\frametitle{Log Everything (Seriously)}
\begin{columns}
    \begin{column}[T]{0.6\linewidth}
      \begin{itemize}
		\item No logs means no learning or improvement opportunities
		\item Track inputs, outputs, retries, and tool calls
		\item Capture agent reasoning and decision-making processes
		\item Add application-specific dimensions like customer type
		\item Include use case categorization in your logging
		\item Analyze error patterns to improve system design
		\item Design evaluators based on logged performance data
		\item Enable continuous system optimization through data
	  \end{itemize}
    \end{column}
    \begin{column}[T]{0.4\linewidth}
		\begin{center}
		\includegraphics[width=0.8\linewidth,keepaspectratio]{aiagents90}
		
		{\tiny (Ref: Principles of Building AI Agents - Pawel Huryn)}
		\end{center}	
    \end{column}
  \end{columns}
\end{frame}

%%%%%%%%%%%%%%%%%%%%%%%%%%%%%%%%%%%%%%%%%%%%%%%%%%%%%%%%%%%%%%%%%%%%%%%%%%%%%%%%%
\begin{frame}[fragile]\frametitle{}
\begin{center}
{\Large Autonomous Agents}
\end{center}
\end{frame}


%%%%%%%%%%%%%%%%%%%%%%%%%%%%%%%%%%%%%%%%%%%%%%%%%%%%%%%%%%%%%%%%%%%%%%%%%%%%%%%%%
\begin{frame}[fragile]
\frametitle{AutoGPT and BabyAGI: Autonomous Agents}

\textbf{Introduction}
\begin{itemize}
    \item AutoGPT and BabyAGI are AI systems designed to work autonomously without constant human guidance.
    \item These agents are creating excitement and hype in the AI community with over 100k stars on GitHub.
    \item AutoGPT uses GPT-4 to sift through the internet, formulate subtasks, and create new agents.
    \item BabyAGI integrates GPT-4, a vector store, and LangChain to create tasks based on prior outcomes and set goals.
\end{itemize}

\textbf{Key Factors}
\begin{itemize}
    \item Limited human involvement: Autonomous agents require minimal human intervention compared to traditional systems like ChatGPT.
    \item Diverse applications: Potential use cases include personal assistants, problem solvers, email management, and prospecting automation.
    \item Swift progress: The rapid growth and interest in these projects showcase their significant potential to revolutionize AI and beyond.
\end{itemize}

\end{frame}

%%%%%%%%%%%%%%%%%%%%%%%%%%%%%%%%%%%%%%%%%%%%%%%%%%%%%%%%%%%%%%%%%%%%%%%%%%%%%%%%%
\begin{frame}[fragile]
\frametitle{Effectively Utilizing Autonomous Agents}

\textbf{Long-Term Goals}
\begin{itemize}
    \item Set specific long-term goals tailored to the project's needs.
    \item Goals might include generating high-quality natural language text, accurate question-answering with context, and continuous performance improvement.
\end{itemize}

\textbf{Challenges and Opportunities}
\begin{itemize}
    \item Rapid Evolution: AutoGPT and similar technologies are evolving quickly, providing developers with new challenges and opportunities.
    \item Ongoing Improvements: Continuous efforts to build and improve these models enhance their capabilities.
    \item Potential Impact: The intrigue surrounding autonomous agents lies in their diverse applications and transformative potential.
\end{itemize}

\textbf{Conclusion}
\begin{itemize}
    \item AutoGPT and BabyAGI represent promising advancements in autonomous agents.
    \item Their ability to work independently and diverse applications make them valuable tools in the AI landscape.
    \item Developers can harness their potential by setting clear goals and embracing continuous improvements.
\end{itemize}

\end{frame}

%%%%%%%%%%%%%%%%%%%%%%%%%%%%%%%%%%%%%%%%%%%%%%%%%%%%%%%%%%%%%%%%%%%%%%%%%%%%%%%%%
\begin{frame}[fragile]
\frametitle{AutoGPT: An Autonomous AI Agent}

\textbf{What is AutoGPT?}
\begin{itemize}
    \item AutoGPT is an autonomous AI agent designed to autonomously carry out tasks until they are solved.
    \item It brings three key features to the table:
    \begin{itemize}
        \item Connected to the internet for real-time research and information retrieval.
        \item Can self-prompt and generate sub-tasks to accomplish a given task.
        \item Capable of executing tasks, including spinning up other AI agents.
    \end{itemize}
\end{itemize}

\end{frame}

%%%%%%%%%%%%%%%%%%%%%%%%%%%%%%%%%%%%%%%%%%%%%%%%%%%%%%%%%%%%%%%%%%%%%%%%%%%%%%%%%
\begin{frame}[fragile]
\frametitle{Challenges and Evolution of AutoGPT}

\textbf{Challenges in Execution}
\begin{itemize}
    \item AutoGPT has faced challenges in executing tasks, including getting caught in loops and incorrectly assuming task completion.
\end{itemize}

\textbf{Evolution of AutoGPT}
\begin{itemize}
    \item Initially conceived as a general autonomous agent with broad application.
    \item Developers observed dilution of effectiveness due to wide breadth of tasks.
    \item Shift in AutoGPT development towards building specialized agents for specific tasks.
    \item Specialized agents designed to perform specific tasks effectively and efficiently.
\end{itemize}

\textbf{Practical Usefulness}
\begin{itemize}
    \item Specialized agents offer more practical usefulness in focused tasks.
    \item Shift towards building task-specific agents enhances AutoGPT's capabilities.
\end{itemize}

\end{frame}

%%%%%%%%%%%%%%%%%%%%%%%%%%%%%%%%%%%%%%%%%%%%%%%%%%%%%%%%%%%%%%%%%%%%%%%%%%%%%%%%%
\begin{frame}[fragile]
\frametitle{How AutoGPT Works}

\textbf{Concept behind AutoGPT}
\begin{itemize}
    \item AutoGPT goes beyond simple text generation like ChatGPT and GPT-4.
    \item It generates, prioritizes, and executes tasks, not limited to text generation.
\end{itemize}

\textbf{Task Generation and Execution}
\begin{itemize}
    \item AutoGPT understands the overall goal and breaks it into subtasks.
    \item It can dynamically adjust actions based on ongoing context.
    \item Uses plugins for internet browsing and access to gather information.
    \item Outside memory serves as a context-aware module for evaluation and task management.
\end{itemize}

\textbf{Active Goal-Oriented Agent}
\begin{itemize}
    \item Transforms from a passive text generator to an active, goal-oriented agent.
    \item Constantly reprioritizes and executes tasks based on the context and situation.
\end{itemize}

\end{frame}

%%%%%%%%%%%%%%%%%%%%%%%%%%%%%%%%%%%%%%%%%%%%%%%%%%%%%%%%%%%%%%%%%%%%%%%%%%%%%%%%%
\begin{frame}[fragile]
\frametitle{Challenges and Implications}

\textbf{New Vistas of AI-Powered Productivity}
\begin{itemize}
    \item AutoGPT opens up new possibilities for AI-powered productivity and problem-solving.
\end{itemize}

\textbf{New Challenges}
\begin{itemize}
    \item Control: Ensuring the agent behaves as intended and avoiding unintended outcomes.
    \item Misuse: Addressing potential misuse of AutoGPT for harmful or unethical purposes.
    \item Unforeseen Consequences: Anticipating and managing unexpected outcomes.
\end{itemize}

\textbf{Ethical Considerations}
\begin{itemize}
    \item Development and deployment of AutoGPT require careful consideration of ethical implications.
    \item Balancing the potential benefits with responsible use and safeguards.
\end{itemize}

\end{frame}

%%%%%%%%%%%%%%%%%%%%%%%%%%%%%%%%%%%%%%%%%%%%%%%%%%%%%%%%%%%%%%%%%%%%%%%%%%%%%%%%%
\begin{frame}[fragile]
\frametitle{What is BabyAGI?}

\textbf{Overview}
\begin{itemize}
    \item BabyAGI works similarly to AutoGPT.
    \item Operates in an infinite loop, executing tasks, enriching results, and generating new tasks based on previous outcomes.
    \item Implementation may differ from AutoGPT.
\end{itemize}

\end{frame}

%%%%%%%%%%%%%%%%%%%%%%%%%%%%%%%%%%%%%%%%%%%%%%%%%%%%%%%%%%%%%%%%%%%%%%%%%%%%%%%%%
\begin{frame}[fragile]
\frametitle{How BabyAGI Works}

\textbf{Sub-Agents}
\begin{itemize}
    \item BabyAGI operates with four main sub-agents:
    \item Execution Agent: Executes tasks by feeding objective and task parameters to LLM (e.g., GPT-4).
    \item Task Creation Agent: Creates new tasks based on previous task objective and results.
    \item Prioritization Agent: Responsible for prioritizing tasks in the task list.
    \item Context Agent: Collects Execution Agent results and merges them with previous intermediate results.
\end{itemize}

\end{frame}

%%%%%%%%%%%%%%%%%%%%%%%%%%%%%%%%%%%%%%%%%%%%%%%%%%%%%%%%%%%%%%%%%%%%%%%%%%%%%%%%%
\begin{frame}[fragile]
\frametitle{Conclusions about BabyAGI}

\begin{itemize}
    \item BabyAGI is an autonomous AI agent designed to execute tasks and generate new ones based on prior outcomes.
    \item Utilizes GPT-4, vector database, and LangChain framework for efficient decision-making.
    \item Adaptable task management with autonomous task generation and prioritization.
    \item GPT-4 and LangChain allow BabyAGI to enrich and store results, making it a learning system.
\end{itemize}

\end{frame}

%%%%%%%%%%%%%%%%%%%%%%%%%%%%%%%%%%%%%%%%%%%%%%%%%%%%%%%%%%%%%%%%%%%%%%%%%%%%%%%%%
\begin{frame}[fragile]
\frametitle{Future Possibilities}

\textbf{AI Agent Advancements}
\begin{itemize}
    \item Exciting future possibilities for BabyAGI and AutoGPT.
    \item Each agent has unique strengths and challenges.
    \item AutoGPT: Powerful for complex tasks, but steeper learning curve.
    \item BabyAGI: Excellent at providing detailed task lists, faces implementation hurdles.
    \item Agents are improving with open-source community efforts.
\end{itemize}

\textbf{AI Autonomy}
\begin{itemize}
    \item AI agents showcasing autonomy previously reserved for human intellect.
    \item Navigating tasks and problems independently.
    \item Future developments hold immense potential in AI landscape.
\end{itemize}

\end{frame}