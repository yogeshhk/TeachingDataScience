\section{Step-by Step Progress}

\begin{itemize}
\itemsep-0.25em % Set a negative itemsep value to reduce spacing between items
\item \textbf{Basic}: ``Explain how to make a peanut butter and jelly sandwich''
\item \textbf{Adding Roles}: As a chef, explain to your assistant how to make a peanut butter and jelly sandwich''
\item \textbf{Adding Constraints}: ``Make a nut-free version of the sandwich due to a customer's nut allergy''
\item \textbf{Adding Examples}: ``Create two unique variations of the classic sandwich. Banana Nut Crunch: \ldots. Triple Berry Blast: \ldots ''
\item \textbf{Adding Contextual Information}: ``As the head chef at 'The Sandwich Haven,' guide your new assistant to create specials for the menu''
\item \textbf{Incorporating Feedback}: ``Improve the sandwich based on customer feedback for less sweetness and a creative twist''
\item \textbf{Time Constraints and Prioritization}: ``Prepare an alternative fruit version for testing within a tight deadline''
\item \textbf{Incorporating Multidisciplinary Knowledge}: ``Use food presentation and garnishing techniques for a visually appealing sandwich''
\item \textbf{Addressing Dietary Preferences}: ``Prepare a vegan version using plant-based alternatives for all ingredients''
\item \textbf{Reflection and Iteration}: ``Reflect on feedback and iteratively refine the sandwich for better taste and appeal''
\item \textbf{Self-Criticism}: ``Explain how to make a peanut butter and jelly sandwich. Please re-read your above response. Any mistakes? If so, please identify and make the necessary edits.''
\item \textbf{Chain-of-Thought}: ``Explain how to make a peanut butter and jelly sandwich. Let’s think step by step.''
\item \textbf{Self-Consistency}: ``Here are recipes of multiple sandwiches. Sandwich 1: recipe 1. Sandwich 2: recipe 2. \ldots. Explain how to make a peanut butter jelly sandwich. ''

\end{itemize}

\section{NLP}

\begin{itemize}
\itemsep-0.25em % Set a negative itemsep value to reduce spacing between items
\item \textbf{Text Generation}: ``Write a conclusion paragraph for a science fiction story.''
\item \textbf{Summarization}: ``Summarize the following content. content: \ldots''
\item \textbf{Question Answering}: ``What is the capital of France?''
\item \textbf{Paraphrasing}: ``Rewrite the following content. content: \ldots''
\item \textbf{Sentiment Analysis (zero-shot)}: ``Find if the following content is positive or negative. content: \ldots''
\item \textbf{Text to Table}: ``Create a table from this text: create a 2 column table where the first column contains the stock ticker symbol for Apple, Google, Amazon, Meta, and the other column contains the names of the companies.''
\item \textbf{Token Classification (zero-shot)}: ``Classify the named entities in the following content. content: \ldots''
\item \textbf{Translation}: ``Translate the following content from English to Spanish. content: \ldots''
\item \textbf{Zero-Shot}: ``Generate 10 possible names for my new cat.''
\item \textbf{One-Shot}: ``Generate 10 possible names for my new cat. Say like mani''
\item \textbf{Few-Shots}: ``Generate 10 possible names for my new cat. Say like mani, manjar, mau''

\end{itemize}

\section{Coding}
\begin{itemize}
\itemsep-0.25em % Set a negative itemsep value to reduce spacing between items
\item \textbf{Code Generation}: ``Show me how to find the factorial of a number in Python.''
\item \textbf{Code Explanation}: ``Act as an technical expert and an educator. Explain the following python code understandable by junior software engineer. Explain working of the functions and reasons for doing so. Embed code snippets and explain them just below each code block. code: \ldots''
\item \textbf{Docstrings Generation}: ``Write a docstring description for the following python function. function: \ldots''
\item \textbf{Code Translation}: ``Convert the following Python code to Javascript. Python code: \ldots''
\item \textbf{Data Object Translation}: ``Convert the following Json object to XML. Json object: \ldots''
\item \textbf{Knowledge Graph Generation}: ``Convert this text into nodes and edges: Babe Ruth joined the New York Yankees in 1920 \ldots''
\item \textbf{HTML to Text (Web Scraping)}: ``Convert the following HTML to Text. HTML: \ldots''
\item \textbf{Extract Email}: ``Extract all email addresses from the given email. email: \ldots''

\end{itemize}

\section{Creativity}
\begin{itemize}
\itemsep-0.25em % Set a negative itemsep value to reduce spacing between items
\item \textbf{Twitter}: ``Write a tweet about the latest movie release.''
\item \textbf{Blog}: ``Write a [adjective] [type of content] on [goal]. Explain
why: [topic1, topic2, ...]''
\item \textbf{Email}: ``Write an email inviting colleagues to a team-building event.''
\item \textbf{Copywriting}: ``Write a [type of content] for [subject] showcasing the [benefit1, benefit2, ...]''

\item \textbf{Poems}: ``Write a haiku about the beauty of nature.''
\item \textbf{Songs}: ``Write a love song with a catchy chorus.''
\item \textbf{Resume}: `` Write a marketing manager resume.''
\item \textbf{Product}: ``Write a product description for a home security system.''
\item \textbf{Specify the emotion or tone}: ``Write an inspiring paragraph about the potential of AI to transform society for the better.''
\item \textbf{Avoiding Inappropriate Content}: ``Write an engaging blog post about AI ethics, ensuring that the content is appropriate for all ages and backgrounds.''
\item \textbf{Marketing}: ``Give me some tips on how to increase open and click-through rates for my email campaigns.''
\item \textbf{Ideas}: ``Can you give me some meme ideas for [topic]?.''
''
\item \textbf{Planning}: ``Create a [n]-day [type of plan] for [topic]?.''
\item \textbf{Learning}: ``Provide [n] ways to understand [subject] with examples.''
\item \textbf{Names}: ``Suggest a [adjective] name for my [type] using the words [word1, word2, ...]''
\item \textbf{Summary, for Medium Story and a Sketchnote for Book/Topic}: ``Step 1: I want you to act as a Medium content creator. Your goal is to craft engaging, informative, and relevant Medium Story based on the given 'Book/Topic' below, for various professionals across different industries. You will focus on [summarizing the 'Book'|covering all the salient aspects of the 'Topic'] in the effective manner.  [Keep the structure of the 'Book' as is, while summarizing the chapter wise content.]

Step 2: Now I want you to be an experienced sketchnote artist. Understanding Medium story created in the first step, suggest visual representations of salient points in there, using visual symbols, clip arts, cartoons, etc. all to be part of a one page sketchnote. The whole essence of the 'Book/Topic' should come in the sketchnote suggested.

Book: <title and author> | Topic: <subject>''
\item \textbf{Medium Story on ready content/summary}: ``I want you to act as a Medium content creator. Your goal is to craft engaging, informative, and relevant Medium Story for various professionals across different industries. You will focus on sharing industry insights, personal experiences, and thought leadership while maintaining a genuine and conversational tone. My first suggestion request is to write an impressive, 400 words long Medium story based on the points below.  The content is point wise summary of an article or book or blogpost. Suggest appealing, witty but terse title. 

---
<content>''. <Add unsplash pic, In the subtitle, mention the original source>
\item \textbf{Reformat content for LinkedIn post}: ``I want you to act as a LinkedIn content creator. Your goal is to craft engaging, informative, and relevant LinkedIn posts for various professionals across different industries, while maintaining a genuine, non-exciting, non-exuberant, sincere and conversational tone. My first suggestion request is to reformat following content as a technical and non-trivial LinkedIn post with minimal emojis, relevant handles and keywords with hashes

<content>''
\item \textbf{Reformat content for Reddit post}: ``Act as a Reddit expert content creator. Your goal is to craft engaging, informative, and relevant subreddit posts for various professionals across different industries, while maintaining a genuine, non-exciting, non-exuberant, sincere and conversational tone. My first suggestion request is to reformat following content as a technical and non-trivial subreddit post with minimal emojis, relevant handles and keywords with hashes

<content>''
\end{itemize}

\section{Productivity}

\begin{itemize}
\itemsep-0.25em % Set a negative itemsep value to reduce spacing between items
\item \textbf{Prompts Generation}: ``You are [GPT-4, OpenAI’s] advanced language model. Today, your job is to generate prompts for [GPT-4]. Can you generate the best prompts on ways to [what you want]''

\item \textbf{Resume Generation}: ``Based on this job description for a [JOB TITLE] role at [COMPANY], write a resume for my past [X] years of work experience with 3-5 bullet points per role that include metrics and the most important 10 keywords from the job description. My past titles and companies were [X, Y, and Z]. No need to include an objective statement. [Copy/paste the job description.]''

\item \textbf{Resume Adjustment}: ``Here's my current resume. How would you rewrite it if you were applying to this [TITLE] role at [COMPANY]? Include metrics in the achievements. [Copy/paste your resume and the job description.]''

\item \textbf{Directives}: Words that direct the AI
\begin{tabular}{lll}
List           & Draft           & Investigate \\
Describe       & Suggest         & Justify \\
Explain        & Analyse         & Measure \\
Compare        & Assess          & Predict \\
Contrast       & Calculate       & Prioritise \\
Generate       & Define          & Recommend \\
Summarise      & Determine       & Report \\
Translate      & Develop         & Review \\
Plan           & Evaluate        & Validate \\
Interpret      & Explore         & Simulate \\
Investigate    & Identify        & Prototype \\
Benchmark      & Use examples    & Forecast \\
Estimate       & Strategise      & Align \\
Collaborate    & Coordinate      & Delegate \\
Facilitate     & Integrate       & Make this easier \\
Simplify       & Negotiate       & Organise \\
Standardise    & Monitor         & Track \\
Optimise       & Streamline      & Iterate \\
\end{tabular}
 
\end{itemize}


\section{\LaTeX}

\begin{itemize}
\itemsep-0.25em % Set a negative itemsep value to reduce spacing between items

\item \textbf{Beamer Slides for Subject}: ``I want you to act as a Technical Educator expert in AI, Machine Learning and Data Science. Your goal is to craft engaging, informative, and relevant slide deck for various professionals across different industries. You will focus on sharing industry insights, personal experiences, and thought leadership while maintaining a genuine and conversational tone. The slide deck should have  sections like 'Introduction', 'Concepts',  'Framework', ''Workflows', 'Architecture', 'Use cases', 'Applications', 'Conclusion' and 'References'.  The whole presentation should be in the format of LaTeX beamer. Each section should have at least 5 slides.

My first suggestion request is to prepare a comprehensive presentation for <Subject>''

\item \textbf{Code Slides for Subject}: ``You are an expert in <subject>. Your goal is to craft engaging, informative, and relevant slides as part of a presentation to be given to the professionals from <domain> industry. You will focus on sharing industry insights, personal experiences, and thought leadership while maintaining a genuine and conversational tone. Please prepare at least 10 slides only specific to the 'Topics' given below. Do not prepare slides on any other topic or even slides like 'Welcome' or 'Thank you'. All the slides should be in python programming language.

Topics: Python implementations of [topic1, topic2, \ldots]''


\item \textbf{Slide by Reformatting Content}: ``Reformat following text as LaTeX beamer slides, each with list of terse bulleted points without losing meaning. ''
\item \textbf{Syllabus/Agenda for Training}: ``Prepare a comprehensive agenda for an 8 hour workshop on the topic of <Subject> with hourly breakup. It should have sections like 'Introduction', 'Background',  'How It Works?', 'sub-topic 1', 'sub-topic 2', \ldots, 'Applications', 'Conclusion' and 'References'.  The whole agenda should be in the tabular format of LaTeX beamer.''

\end{itemize}

\section{Tuning Parameters}

\begin{itemize}
\itemsep-0.25em % Set a negative itemsep value to reduce spacing between items
\item \textbf{Temperature}:
	\begin{itemize}
	\itemsep-0.25em % Set a negative itemsep value to reduce spacing between items

	\item Controls the creativity of AI-generated text.
	\item High temperature (e.g., 0.8) results in more randomness and diversity in responses.
	\item Low temperature (e.g., 0.2) produces more focused and deterministic responses.
	\end{itemize}
\item \textbf{Top-p and Top-k}:
	\begin{itemize}
	\itemsep-0.25em % Set a negative itemsep value to reduce spacing between items

	\item Used to fine-tune response length and diversity.
	\item Top-p considers the most probable words until the cumulative probability exceeds a threshold.
	\item Top-k retains the top-k most likely words as options.
	\item Both techniques prevent unrealistic or excessively long outputs.
	\end{itemize}
\end{itemize}

\section{References and More Info}

\begin{itemize}
\itemsep-0.25em % Set a negative itemsep value to reduce spacing between items
\item KD Nuggets: The ChatGPT Cheat Sheet
\item QuickRef: ChatGPT cheatsheet
\item Digital Strategies: ChatGPT Cheat Sheet
\item Prompt Engineering for Effective Interaction with ChatGPT
\item dair-ai: Prompt-Engineering-Guide
\end{itemize}