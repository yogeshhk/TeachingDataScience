%\documentclass[english,serif,mathserif,xcolor=pdftex,dvipsnames,table]{beamer}
%\usetheme[informal]{s3it}
%\usepackage{s3it}
%
%\title[Introduction]{%
%  Introduction to the Python programming language
%}
%\author[S3IT]{%
%  S3IT: Services and Support for Science IT, \\
%  University of Zurich
%}
%\date{June~23--24, 2014}
%
%\begin{document}
%
%% title frame
%\maketitle

%\begin{frame}
%  \begin{center}
%    {\Large Welcome!}
%  \end{center}
%\end{frame}


\begin{frame}\frametitle{References}

Many publicly available resources have been refereed for making this presentation. Some of the notable ones are:
\tiny
\begin{itemize}
\item CS 20SI: TensorFlow for Deep Learning Research.
\item Deep Learning Tutorial Hung yi Lee.
\item STAT 365/665: Data Mining and Machine Learning - Taylor Arnold.
\item CSC 600: Data Mining - Burns.
\item Using the TensorFlow API: An Introductory Tutorial Series - Erik Hallström.
\item Deep Learning Project in NLP - Yu Wang, Yale University
\item ``Anyone Can Learn To Code an LSTM'' - i Am Trask.
\item ``A friendly Introduction to Deep Learning and Neural Networks'': Luis Serrano.
\item Machine Learning Crash Course by Google using TensorFlow.
\item Sung Kim: https://github.com/hunkim/PyTorchZeroToAll
%\item Ian Goodfellow, Aaron Courville and Yoshua Bengio. \textit{Deep Learning}. Book in preparation for MIT Press. \url{http://www.deeplearningbook.org/}.
%\item Andrej Karpathy's Hacker's guide to Neural Networks: \url{http://karpathy.github.io/neuralnets/}
%\item Andrej Karpathy's lecture notes: \url{http://cs231n.github.io/}
%\item Geoffrey E. Hinton, Yann LeCun, and Yoshua Bengio (video; NIPS 2015): \url{http://research.microsoft.com/apps/video/default.aspx?id=259574}
\item Michael Nielsen's Neural Networks and Deep Learning: \url{http://neuralnetworksanddeeplearning.com/}
\end{itemize}


\end{frame}

%=======================================================================%
%
%
%\end{document}
%
%%%% Local Variables:
%%%% mode: latex
%%%% TeX-master: t
%%%% End:
