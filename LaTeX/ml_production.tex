%%%%%%%%%%%%%%%%%%%%%%%%%%%%%%%%%%%%%%%%%%%%%%%%%%%%%%%%%%%%%%%%%%%%%%%%%%%%%%%%%%
\begin{frame}[fragile]\frametitle{}
\begin{center}
{\Large ML in Production - Deployment}


{\tiny (Ref: The most under-taught skill in machine learning - George Seif )}
\end{center}


\end{frame}

%%%%%%%%%%%%%%%%%%%%%%%%%%%%%%%%%%%%%%%%%%%%%%%%%%%%%%%%%%%%%%%%%%%%%%%%%%%%%%%%%%
\begin{frame}\frametitle{After Learning Machine Learning}

\begin{itemize}
\item Once you create your machine learning algorithm, that research part is done. 
\item Then you really start the bulk of the work. 
\item How will the results of your model be delivered to the end user?
\item In today’s world you’ll need some powerful hardware to be able to run it at a reasonable speed; that means running your machine learning API on the cloud.
\item Thats called ``putting it in Production'' or ``Deployment''
\end{itemize}
\end{frame}

%%%%%%%%%%%%%%%%%%%%%%%%%%%%%%%%%%%%%%%%%%%%%%%%%%%%%%%%%%%%%%%%%%%%%%%%%%%%%%%%%%
\begin{frame}\frametitle{Whats needed}

\begin{itemize}
\item In today’s world you'll need some powerful hardware to be able to run it at a reasonable speed; 
\item That means running your machine learning API on the cloud.
\item You run it on a cloud server and send the results back to the user! 
\item You automate your system pipeline and have it ready to automatically scale based on your user traffic!
\item Cloud computing is the workhorse behind the real-world machine learning applications.
\end{itemize}
\end{frame}

%%%%%%%%%%%%%%%%%%%%%%%%%%%%%%%%%%%%%%%%%%%%%%%%%%%%%%%%%%%%%%%%%%%%%%%%%%%%%%%%%%
\begin{frame}\frametitle{Cloud computing for machine learning}
\begin{center}
\includegraphics[width=\linewidth]{prod1}
\end{center}
\end{frame}

%%%%%%%%%%%%%%%%%%%%%%%%%%%%%%%%%%%%%%%%%%%%%%%%%%%%%%%%%%%%%%%%%%%%%%%%%%%%%%%%%%
\begin{frame}\frametitle{Cloud Players}
\begin{itemize}
\item AWS: most popular, allows you to control and customize 
\item Azure: Offers you easy of use at the expense of a bit of control and customization. 
\item GCP is somewhere in the middle of the two with some abstraction but not too much.
\end{itemize}
\end{frame}

%%%%%%%%%%%%%%%%%%%%%%%%%%%%%%%%%%%%%%%%%%%%%%%%%%%%%%%%%%%%%%%%%%%%%%%%%%%%%%%%%%
\begin{frame}\frametitle{AWS EC2}
\begin{itemize}
\item Houses your machine learning servers. 
\item Set up your machine learning model on the server. 
\item When you want to run something on your model, you send the data you want processed to the server, your model processes it, and sends it back to the user! 
\item EC2 also offers auto-scaling so that you can automatically spawn more or less servers based on the demand
\end{itemize}

\begin{center}
\includegraphics[width=0.6\linewidth]{prod2}
\end{center}

{\tiny (Ref: AWS EC2 Instance Types: The Definitive Guide - nOps)}
\end{frame}

%%%%%%%%%%%%%%%%%%%%%%%%%%%%%%%%%%%%%%%%%%%%%%%%%%%%%%%%%%%%%%%%%%%%%%%%%%%%%%%%%%
\begin{frame}\frametitle{AWS Lambda}
\begin{itemize}
\item With lambda, you can basically set up automated trigger functions which will only run when a certain condition is met. 
\item For example, have your lambda function send an email to your user only when a certain results comes up from your machine learning module, such as some critical situation
\end{itemize}

\begin{center}
\includegraphics[width=\linewidth]{prod3}
\end{center}

{\tiny (Ref: https://aws.amazon.com/lambda/)}


\end{frame}

%%%%%%%%%%%%%%%%%%%%%%%%%%%%%%%%%%%%%%%%%%%%%%%%%%%%%%%%%%%%%%%%%%%%%%%%%%%%%%%%%%
\begin{frame}\frametitle{AWS S3}
Very cheap, 99.9999999\% up-time, with fast download and upload speeds!

\begin{center}
\includegraphics[width=\linewidth]{prod4}
\end{center}

{\tiny (Ref: https://aws.amazon.com/s3/)}

\end{frame}

%%%%%%%%%%%%%%%%%%%%%%%%%%%%%%%%%%%%%%%%%%%%%%%%%%%%%%%%%%%%%%%%%%%%%%%%%%%%%%%%%%
\begin{frame}\frametitle{AWS RDS}
Managed Relational Database Service for MySQL, PostgreSQL, Oracle, SQL Server, and MariaDB. Organize all of your important data for your machine learning data, API, infrastructure, and model results here.

\begin{center}
\includegraphics[width=\linewidth]{prod5}
\end{center}

{\tiny (Ref: https://aws.amazon.com/rds/)}

\end{frame}

%%%%%%%%%%%%%%%%%%%%%%%%%%%%%%%%%%%%%%%%%%%%%%%%%%%%%%%%%%%%%%%%%%%%%%%%%%%%%%%%%%
\begin{frame}\frametitle{AWS CodeDeploy}
Automatically have your code and new machine learning models deployed to your servers as soon as you commit them to GitHub

\begin{center}
\includegraphics[width=\linewidth]{prod6}
\end{center}

{\tiny (Ref: https://aws.amazon.com/codedeploy/)}

\end{frame}


%%%%%%%%%%%%%%%%%%%%%%%%%%%%%%%%%%%%%%%%%%%%%%%%%%%%%%%%%%%%%%%%%%%%%%%%%%%%%%%%%%
\begin{frame}\frametitle{AWS Cloudwatch}
Online logs to constantly monitor your machine learning system

\begin{center}
\includegraphics[width=\linewidth]{prod7}
\end{center}

{\tiny (Ref: https://aws.amazon.com/cloudwatch/)}
\end{frame}

%%%%%%%%%%%%%%%%%%%%%%%%%%%%%%%%%%%%%%%%%%%%%%%%%%%%%%%%%%%%%%%%%%%%%%%%%%%%%%%%%%
\begin{frame}\frametitle{Amazon Simple Queue Service (SQS)}
A queue hosted in the cloud. Keep your machine learning jobs organised and in order using a cloud queue


\begin{center}
\includegraphics[width=\linewidth]{prod8}
\end{center}

{\tiny (Ref: https://aws.amazon.com/sqs/)}
\end{frame}

%%%%%%%%%%%%%%%%%%%%%%%%%%%%%%%%%%%%%%%%%%%%%%%%%%%%%%%%%%%%%%%%%%%%%%%%%%%%%%%%%%
\begin{frame}\frametitle{AWS Mobile Hub}
Build, test, and monitor your apps remotely using the cloud. Just log in to your AWS account without the hassle of pulling data from your app manually

\begin{center}
\includegraphics[width=\linewidth]{prod9}
\end{center}

{\tiny (Ref: https://aws.amazon.com/products/frontend-web-mobile/)}
\end{frame}


%%%%%%%%%%%%%%%%%%%%%%%%%%%%%%%%%%%%%%%%%%%%%%%%%%%%%%%%%%%%%%%%%%%%%%%%%%%%%%%%%%
\begin{frame}\frametitle{Amazon API Gateway}
Build, deploy, and manage your API at any scale in the cloud. Have all the information you need for this in one simple place

\begin{center}
\includegraphics[width=\linewidth]{prod10}
\end{center}

{\tiny (Ref: https://aws.amazon.com/api-gateway/?nc2=type\_a)}


\end{frame}

%%%%%%%%%%%%%%%%%%%%%%%%%%%%%%%%%%%%%%%%%%%%%%%%%%%%%%%%%%%%%%%%%%%%%%%%%%%%%%%%%%
\begin{frame}\frametitle{Amazon Sagemaker}
Build, train, and test your machine learning models using a high-level easy to use interface

\begin{center}
\includegraphics[width=\linewidth]{prod11}
\end{center}

{\tiny (Ref: https://aws.amazon.com/blogs/aws/sagemaker/}
\end{frame}

%%%%%%%%%%%%%%%%%%%%%%%%%%%%%%%%%%%%%%%%%%%%%%%%%%%%%%%%%%%%%%%%%%%%%%%%%%%%%%%%%%
\begin{frame}\frametitle{What Next?}
\begin{itemize}
\item Coursera has a great one on GCP and Udemy has one on AWS! 
\item As always, it’s a great idea to actually use the platform to learn it best. 
\item AWS offers a free tier for a year and their service aren’t too expensive if you would like to play around with some of the non-free ones. 
\item GCP offers \$300 of free credits for new accounts too!
\end{itemize}

\end{frame}