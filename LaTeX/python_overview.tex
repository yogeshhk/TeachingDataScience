%%%%%%%%%%%%%%%%%%%%%%%%%%%%%%%%%%%%%%%%%%%%%%%%%%%%%%%%%%%%%%%%%%%%%%%%%%%%%%%%%%
\begin{frame}[fragile]\frametitle{}
\begin{center}
{\Large Overview}
\end{center}
\end{frame}

%%%%%%%%%%%%%%%%%%%%%%%%%%%%%%%%%%%%%%%%%%%%%%%%%%%%%%%%%%%
% Slide 1: Why
%%%%%%%%%%%%%%%%%%%%%%%%%%%%%%%%%%%%%%%%%%%%%%%%%%%%%%%%%%%
\begin{frame} \frametitle{Why Python?}
\begin{itemize}
\item  Readability.
\item Ease of use.
\item Fits in your head.
\item Gets things done.
\item Good libraries.
\item Lookie what I did!.
\end{itemize}
\end{frame}


%%%%%%%%%%%%%%%%%%%%%%%%%%%%%%%%%%%%%%%%%%%%%%%%%%%%%%%%%%%
% Slide 2: Course Overview
%%%%%%%%%%%%%%%%%%%%%%%%%%%%%%%%%%%%%%%%%%%%%%%%%%%%%%%%%%%
\begin{frame}[fragile]\frametitle{Course Overview}

      \begin{itemize}
        \item Python Fundamentals
        \item Object-Oriented Programming
        \item File Handling \& I/O Operations
        \item Advanced Python Features
        \item Asynchronous Programming
        \item Real-world Projects
      \end{itemize}

\end{frame}

%%%%%%%%%%%%%%%%%%%%%%%%%%%%%%%%%%%%%%%%%%%%%%%%%%%%%%%%%%%%%%%%%%%%%%%%%%%%%%%%%%%
\begin{frame}[fragile]  \frametitle{Introduction}
\begin{itemize}
\item Python is a simple, yet powerful interpreted language.
\item  Numerous libraries: NumPy, SciPy, Matplotlib \ldots.
\item Named after Monty Python.
\item Open Source and Free
\item Invented by Guido van Rossum.
\end{itemize}
\end{frame}

%%%%%%%%%%%%%%%%%%%%%%%%%%%%%%%%%%%%%%%%%%%%%%%%%%%%%%%%%%%%%%%%%%%%%%%%%%%%%%%%%%%
\begin{frame}[fragile]\frametitle{Python's Benevolent Dictator For Life}

\adjustbox{valign=t}{
\begin{minipage}{0.5\linewidth}
{\em ``Python is an experiment in how  much freedom programmers need.  Too much freedom and nobody can read another's code; too little and expressiveness is endangered.''}

      - Guido van Rossum 
\end{minipage}
}
\hfill
\adjustbox{valign=t}{
\begin{minipage}{0.4\linewidth}
\begin{center}
\includegraphics[width=0.8\linewidth,keepaspectratio]{rossum}
\end{center}
\tiny{(Reference: https://en.wikipedia.org/ wiki/Guido\_van\_Rossum)}
\end{minipage}
}
\end{frame}



%%%%%%%%%%%%%%%%%%%%%%%%%%%%%%%%%%%%%%%%%%%%%%%%%%%%%%%%%%%%%%%%%%%%%%%%%%%%%%%%%%%
\begin{frame}[fragile]  \frametitle{What is Python?}
\begin{itemize}
\item Interpreted
\item Object-oriented 
\item High-level
\item Dynamic semantics
\item Cross-platform
\item Readability.
\end{itemize}


\end{frame}


%%%%%%%%%%%%%%%%%%%%%%%%%%%%%%%%%%%%%%%%%%%%%%%%%%%%%%%%%%%%%%%%%%%%%%%%%%%%%%%%%%%
\begin{frame}[fragile]  \frametitle{Compiled Languages}
\begin{itemize}
\item Needs entire program
\item Translates directly to machine codes 
\item Exe native and fast
\item Usually statically typed
\item Types known during compilation
\item Change in type : recompilation
\item Ideal for compute-heavy tasks
\item E.g. C, C++, FORTRAN
\end{itemize}
\end{frame}

%%%%%%%%%%%%%%%%%%%%%%%%%%%%%%%%%%%%%%%%%%%%%%%%%%%%%%%%%%%%%%%%%%%%%%%%%%%%%%%%%%%
\begin{frame}[fragile]  \frametitle{Interpreted Languages}
\begin{itemize}
\item  Interpreted on the fly
\item No need to compile: can execute right away
\item Usually dynamically typed
\item Non-syntax errors are detected only in run-time
\item Slower than compiled languages
\item Ideal for small tasks
\item  E.g. Python, Perl, PHP, Bash
\end{itemize}

Note: Python, at the beginning, loosely checks the program. Only at run time, line-by-line, it checks for errors. So, if the error statements are not in the running path, their error does not get reported. So, its a bit relaxed. Thats the objection for its use in production code, where strong type checking and error checking, upfront is essential.
\end{frame}

%%%%%%%%%%%%%%%%%%%%%%%%%%%%%%%%%%%%%%%%%%%%%%%%%%%%%%%%%%%%%%%%%%%%%%%%%%%%%%%%%%%
\begin{frame}[fragile]  \frametitle{JIT-Compiled Languages}
\begin{itemize}
\item Between compiled and interpreted
\item Code is initially interpreted, hotspots compiled
\item Deduce types during compilation, can change in run-time
\item Slower than compiled
\item E.g. Java, C\#
\end{itemize}
\end{frame}

%%%%%%%%%%%%%%%%%%%%%%%%%%%%%%%%%%%%%%%%%%%%%%%%%%%%%%%%%%%%%%%%%%%%%%%%%%%%%%%%%%%
\begin{frame}[fragile]  \frametitle{``C'' guys to take pride in}
\begin{itemize}
\item Python interpreter written in  ``C''
\item Source code at www.python.org
\item So, Python Interpreter is compiled exe
%\item On windows, its \lstinline|C:\Python35\python.exe|
\end{itemize}
\end{frame}


%%%%%%%%%%%%%%%%%%%%%%%%%%%%%%%%%%%%%%%%%%%%%%%%%%%%%%%%%%%%%%%%%%%%%%%%%%%%%%%%%%%
\begin{frame}[fragile]  \frametitle{Completely Controversial Observations about Languages}
\begin{center}
\includegraphics[width=0.6\linewidth,keepaspectratio]{langs}
\end{center}
\end{frame}



%%%%%%%%%%%%%%%%%%%%%%%%%%%%%%%%%%%%%%%%%%%%%%%%%%%%%%%%%%%%%%%%%%%%%%%%%%%%%%%%%%%
\begin{frame}[fragile]  \frametitle{Important features}
\begin{itemize}
\item Built-in high level data types: \lstinline{strings, lists, dictionaries}, etc.
\item Usual control structures: \lstinline{if, if-else, if-elif-else, while, for}
\item Levels of organization: \lstinline{functions, classes, modules, packages}
\item Extensions in C and C++ possible
\end{itemize}
\end{frame}

%%%%%%%%%%%%%%%%%%%%%%%%%%%%%%%%%%%%%%%%%%%%%%%%%%%%%%%%%%%%%%%%%%%%%%%%%%%%%%%%%%%
\begin{frame}[fragile]  \frametitle{Python 2 \emph{vs} Python 3}
\begin{itemize}
\item Two major versions 2.*, 3.*
\item Python 2.7: Latest release in 2.x series
\item Python 3.5: More polished syntax, removed inconsistencies
\end{itemize}
\end{frame}



%%%%%%%%%%%%%%%%%%%%%%%%%%%%%%%%%%%%%%%%%%%%%%%%%%%%%%%%%%%
% Slide 3: Variables & Data Types
%%%%%%%%%%%%%%%%%%%%%%%%%%%%%%%%%%%%%%%%%%%%%%%%%%%%%%%%%%%
\begin{frame}[fragile]\frametitle{Variables \& Data Types}
\begin{columns}
    \begin{column}[T]{0.6\linewidth}
      \begin{itemize}
        \item Dynamic typing - no declaration needed
        \item Basic types: int, float, str, bool
        \item Type conversion and checking
        \item Variable naming conventions
        \item Multiple assignment
      \end{itemize}
    \end{column}
    \begin{column}[T]{0.4\linewidth}
      \begin{lstlisting}[language=python, basicstyle=\tiny]
# Variable assignment
name = "Alice"
age = 25
height = 5.6
is_student = True

# Type checking
print(type(age))  # <class 'int'>

# Type conversion
age_str = str(age)
price = float("19.99")

# Multiple assignment
x, y, z = 1, 2, 3
      \end{lstlisting}
    \end{column}
  \end{columns}
\end{frame}

%%%%%%%%%%%%%%%%%%%%%%%%%%%%%%%%%%%%%%%%%%%%%%%%%%%%%%%%%%%
% Slide 4: Data Structures
%%%%%%%%%%%%%%%%%%%%%%%%%%%%%%%%%%%%%%%%%%%%%%%%%%%%%%%%%%%
\begin{frame}[fragile]\frametitle{Data Structures - Lists, Tuples, Dicts, Sets}
\begin{columns}
    \begin{column}[T]{0.6\linewidth}
      \begin{itemize}
        \item Lists: ordered, mutable sequences
        \item Tuples: ordered, immutable sequences
        \item Dictionaries: key-value pairs
        \item Sets: unordered, unique elements
        \item Indexing and slicing
      \end{itemize}
    \end{column}
    \begin{column}[T]{0.4\linewidth}
      \begin{lstlisting}[language=python, basicstyle=\tiny]
# List
fruits = ["apple", "banana", "cherry"]
fruits.append("date")

# Tuple
coordinates = (10, 20)

# Dictionary
person = {"name": "Bob", "age": 30}
person["city"] = "NYC"

# Set
unique_nums = {1, 2, 3, 3, 2}
print(unique_nums)  # {1, 2, 3}
      \end{lstlisting}
    \end{column}
  \end{columns}
\end{frame}

%%%%%%%%%%%%%%%%%%%%%%%%%%%%%%%%%%%%%%%%%%%%%%%%%%%%%%%%%%%
% Slide 5: Control Flow - If Statements
%%%%%%%%%%%%%%%%%%%%%%%%%%%%%%%%%%%%%%%%%%%%%%%%%%%%%%%%%%%
\begin{frame}[fragile]\frametitle{Control Flow - Conditional Statements}
\begin{columns}
    \begin{column}[T]{0.6\linewidth}
      \begin{itemize}
        \item if, elif, else statements
        \item Comparison operators $(==, !=, <, >, <=, >=)$
        \item Logical operators (and, or, not)
        \item Ternary operator
        \item Truthiness in Python
      \end{itemize}
    \end{column}
    \begin{column}[T]{0.4\linewidth}
      \begin{lstlisting}[language=python, basicstyle=\tiny]
score = 85

if score >= 90:
    grade = "A"
elif score >= 80:
    grade = "B"
elif score >= 70:
    grade = "C"
else:
    grade = "F"

# Ternary operator
status = "Pass" if score >= 60 else "Fail"

# Logical operators
if score > 70 and score < 90:
    print("Good performance")
      \end{lstlisting}
    \end{column}
  \end{columns}
\end{frame}

%%%%%%%%%%%%%%%%%%%%%%%%%%%%%%%%%%%%%%%%%%%%%%%%%%%%%%%%%%%
% Slide 6: Control Flow - Loops
%%%%%%%%%%%%%%%%%%%%%%%%%%%%%%%%%%%%%%%%%%%%%%%%%%%%%%%%%%%
\begin{frame}[fragile]\frametitle{Control Flow - Loops}
\begin{columns}
    \begin{column}[T]{0.6\linewidth}
      \begin{itemize}
        \item for loops - iterate over sequences
        \item while loops - condition-based iteration
        \item break and continue statements
        \item range() function
        \item enumerate() for index tracking
        \item Loop with else clause
      \end{itemize}
    \end{column}
    \begin{column}[T]{0.4\linewidth}
      \begin{lstlisting}[language=python, basicstyle=\tiny]
# For loop with range
for i in range(5):
    print(i)

# Iterate over list
fruits = ["apple", "banana", "cherry"]
for fruit in fruits:
    print(fruit)

# While loop
count = 0
while count < 3:
    print(count)
    count += 1

# Enumerate
for idx, fruit in enumerate(fruits):
    print(f"{idx}: {fruit}")
      \end{lstlisting}
    \end{column}
  \end{columns}
\end{frame}

%%%%%%%%%%%%%%%%%%%%%%%%%%%%%%%%%%%%%%%%%%%%%%%%%%%%%%%%%%%
% Slide 7: Functions - Basics
%%%%%%%%%%%%%%%%%%%%%%%%%%%%%%%%%%%%%%%%%%%%%%%%%%%%%%%%%%%
\begin{frame}[fragile]\frametitle{Functions - Basics}
\begin{columns}
    \begin{column}[T]{0.6\linewidth}
      \begin{itemize}
        \item Function definition with def keyword
        \item Parameters and arguments
        \item Return values
        \item Default parameter values
        \item Docstrings for documentation
        \item Function as first-class objects
      \end{itemize}
    \end{column}
    \begin{column}[T]{0.4\linewidth}
      \begin{lstlisting}[language=python, basicstyle=\tiny]
# Basic function
def greet(name):
    """Greet a person by name"""
    return f"Hello, {name}!"

# Default parameters
def power(base, exp=2):
    return base ** exp

print(power(3))      # 9
print(power(3, 3))   # 27

# Multiple return values
def get_stats(numbers):
    return min(numbers), max(numbers)

min_val, max_val = get_stats([1, 5, 3])
      \end{lstlisting}
    \end{column}
  \end{columns}
\end{frame}

%%%%%%%%%%%%%%%%%%%%%%%%%%%%%%%%%%%%%%%%%%%%%%%%%%%%%%%%%%%
% Slide 8: Functions - Arguments
%%%%%%%%%%%%%%%%%%%%%%%%%%%%%%%%%%%%%%%%%%%%%%%%%%%%%%%%%%%
\begin{frame}[fragile]\frametitle{Functions - Arguments \& Scope}
\begin{columns}
    \begin{column}[T]{0.6\linewidth}
      \begin{itemize}
        \item Positional arguments
        \item Keyword arguments
        \item *args for variable arguments
        \item **kwargs for keyword arguments
        \item Local vs global scope
        \item LEGB rule (Local, Enclosing, Global, Built-in)
      \end{itemize}
    \end{column}
    \begin{column}[T]{0.4\linewidth}
      \begin{lstlisting}[language=python, basicstyle=\tiny]
# *args and **kwargs
def calculate(*args, **kwargs):
    total = sum(args)
    operation = kwargs.get('op', 'sum')
    return f"{operation}: {total}"

result = calculate(1, 2, 3, op="total")

# Scope example
x = 10  # Global

def modify():
    x = 5  # Local
    print(x)  # 5

modify()
print(x)  # 10 (global unchanged)
      \end{lstlisting}
    \end{column}
  \end{columns}
\end{frame}

%%%%%%%%%%%%%%%%%%%%%%%%%%%%%%%%%%%%%%%%%%%%%%%%%%%%%%%%%%%
% Slide 9: Lambda Functions
%%%%%%%%%%%%%%%%%%%%%%%%%%%%%%%%%%%%%%%%%%%%%%%%%%%%%%%%%%%
\begin{frame}[fragile]\frametitle{Lambda Functions \& Built-in Functions}
\begin{columns}
    \begin{column}[T]{0.6\linewidth}
      \begin{itemize}
        \item Anonymous functions with lambda
        \item Single expression functions
        \item map() - apply function to iterable
        \item filter() - filter items by condition
        \item reduce() - aggregate values
        \item sorted() with key parameter
      \end{itemize}
    \end{column}
    \begin{column}[T]{0.4\linewidth}
      \begin{lstlisting}[language=python, basicstyle=\tiny]
# Lambda function
square = lambda x: x ** 2
print(square(5))  # 25

# map() - transform elements
numbers = [1, 2, 3, 4]
squared = list(map(lambda x: x**2, numbers))

# filter() - select elements
evens = list(filter(lambda x: x % 2 == 0, numbers))

# sorted with key
students = [("Alice", 85), ("Bob", 92)]
sorted_students = sorted(students, 
                         key=lambda x: x[1], 
                         reverse=True)
      \end{lstlisting}
    \end{column}
  \end{columns}
\end{frame}

%%%%%%%%%%%%%%%%%%%%%%%%%%%%%%%%%%%%%%%%%%%%%%%%%%%%%%%%%%%
% Slide 10: Object-Oriented Programming - Classes
%%%%%%%%%%%%%%%%%%%%%%%%%%%%%%%%%%%%%%%%%%%%%%%%%%%%%%%%%%%
\begin{frame}[fragile]\frametitle{OOP - Classes \& Objects}
\begin{columns}
    \begin{column}[T]{0.6\linewidth}
      \begin{itemize}
        \item Class definition with class keyword
        \item \_\_init\_\_ constructor method
        \item Instance variables (self)
        \item Instance methods
        \item Class vs instance attributes
        \item Encapsulation concept
      \end{itemize}
    \end{column}
    \begin{column}[T]{0.4\linewidth}
      \begin{lstlisting}[language=python, basicstyle=\tiny]
class Dog:
    # Class attribute
    species = "Canis familiaris"
    
    def __init__(self, name, age):
        # Instance attributes
        self.name = name
        self.age = age
    
    def bark(self):
        return f"{self.name} says Woof!"
    
    def description(self):
        return f"{self.name} is {self.age} years old"

# Create objects
buddy = Dog("Buddy", 3)
print(buddy.bark())
      \end{lstlisting}
    \end{column}
  \end{columns}
\end{frame}

%%%%%%%%%%%%%%%%%%%%%%%%%%%%%%%%%%%%%%%%%%%%%%%%%%%%%%%%%%%
% Slide 11: OOP - Inheritance
%%%%%%%%%%%%%%%%%%%%%%%%%%%%%%%%%%%%%%%%%%%%%%%%%%%%%%%%%%%
\begin{frame}[fragile]\frametitle{OOP - Inheritance}
\begin{columns}
    \begin{column}[T]{0.6\linewidth}
      \begin{itemize}
        \item Inheriting from parent classes
        \item super() to call parent methods
        \item Method overriding
        \item Multiple inheritance
        \item isinstance() and issubclass()
        \item Code reusability
      \end{itemize}
    \end{column}
    \begin{column}[T]{0.4\linewidth}
      \begin{lstlisting}[language=python, basicstyle=\tiny]
class Animal:
    def __init__(self, name):
        self.name = name
    
    def speak(self):
        pass

class Dog(Animal):
    def __init__(self, name, breed):
        super().__init__(name)
        self.breed = breed
    
    def speak(self):
        return f"{self.name} barks!"

class Cat(Animal):
    def speak(self):
        return f"{self.name} meows!"

dog = Dog("Max", "Labrador")
print(dog.speak())
      \end{lstlisting}
    \end{column}
  \end{columns}
\end{frame}

%%%%%%%%%%%%%%%%%%%%%%%%%%%%%%%%%%%%%%%%%%%%%%%%%%%%%%%%%%%
% Slide 12: OOP - Polymorphism
%%%%%%%%%%%%%%%%%%%%%%%%%%%%%%%%%%%%%%%%%%%%%%%%%%%%%%%%%%%
\begin{frame}[fragile]\frametitle{OOP - Polymorphism \& Special Methods}
\begin{columns}
    \begin{column}[T]{0.6\linewidth}
      \begin{itemize}
        \item Method overriding for polymorphism
        \item Duck typing in Python
        \item Special/magic methods (\_\_str\_\_, \_\_repr\_\_)
        \item Operator overloading (\_\_add\_\_, \_\_eq\_\_)
        \item \_\_len\_\_, \_\_getitem\_\_ for collections
      \end{itemize}
    \end{column}
    \begin{column}[T]{0.4\linewidth}
      \begin{lstlisting}[language=python, basicstyle=\tiny]
class Vector:
    def __init__(self, x, y):
        self.x = x
        self.y = y
    
    def __str__(self):
        return f"Vector({self.x}, {self.y})"
    
    def __add__(self, other):
        return Vector(self.x + other.x, 
                      self.y + other.y)
    
    def __eq__(self, other):
        return self.x == other.x and \
               self.y == other.y

v1 = Vector(1, 2)
v2 = Vector(3, 4)
v3 = v1 + v2  # Calls __add__
print(v3)     # Calls __str__
      \end{lstlisting}
    \end{column}
  \end{columns}
\end{frame}

%%%%%%%%%%%%%%%%%%%%%%%%%%%%%%%%%%%%%%%%%%%%%%%%%%%%%%%%%%%
% Slide 13: OOP - Encapsulation & Properties
%%%%%%%%%%%%%%%%%%%%%%%%%%%%%%%%%%%%%%%%%%%%%%%%%%%%%%%%%%%
\begin{frame}[fragile]\frametitle{OOP - Encapsulation \& Properties}
\begin{columns}
    \begin{column}[T]{0.6\linewidth}
      \begin{itemize}
        \item Private attributes with \_\_ prefix
        \item Protected attributes with \_ prefix
        \item Property decorators (@property)
        \item Getters and setters
        \item Computed properties
        \item Data validation
      \end{itemize}
    \end{column}
    \begin{column}[T]{0.4\linewidth}
      \begin{lstlisting}[language=python, basicstyle=\tiny]
class BankAccount:
    def __init__(self, balance):
        self.__balance = balance  # Private
    
    @property
    def balance(self):
        return self.__balance
    
    @balance.setter
    def balance(self, amount):
        if amount < 0:
            raise ValueError("Balance cannot be negative")
        self.__balance = amount
    
    def deposit(self, amount):
        self.balance += amount

account = BankAccount(100)
account.deposit(50)
print(account.balance)  # 150
      \end{lstlisting}
    \end{column}
  \end{columns}
\end{frame}

%%%%%%%%%%%%%%%%%%%%%%%%%%%%%%%%%%%%%%%%%%%%%%%%%%%%%%%%%%%
% Slide 14: Project - Library Management System
%%%%%%%%%%%%%%%%%%%%%%%%%%%%%%%%%%%%%%%%%%%%%%%%%%%%%%%%%%%
\begin{frame}[fragile]\frametitle{Project: Library Management System}
\begin{lstlisting}[language=python, basicstyle=\tiny]
class Book:
    def __init__(self, title, author, isbn):
        self.title = title
        self.author = author
        self.isbn = isbn
        self.is_available = True

class Member:
    def __init__(self, name, member_id):
        self.name = name
        self.member_id = member_id
        self.borrowed_books = []

class Library:
    def __init__(self):
        self.books = []
        self.members = []
    
    def add_book(self, book): # TODO: Implement adding book to library
        pass
    
    def lend_book(self, member, book): # TODO: Check availability, update records
        pass
    
    def return_book(self, member, book): # TODO: Process book return
        pass
\end{lstlisting}
\end{frame}

%%%%%%%%%%%%%%%%%%%%%%%%%%%%%%%%%%%%%%%%%%%%%%%%%%%%%%%%%%%
% Slide 15: Project - Shopping Cart System
%%%%%%%%%%%%%%%%%%%%%%%%%%%%%%%%%%%%%%%%%%%%%%%%%%%%%%%%%%%
\begin{frame}[fragile]\frametitle{Project: Shopping Cart System}
\begin{lstlisting}[language=python, basicstyle=\tiny]
class Product:
    def __init__(self, name, price, category):
        self.name = name
        self.price = price
        self.category = category

class ShoppingCart:
    def __init__(self):
        self.items = {}  # {product: quantity}
    
    def add_item(self, product, quantity=1): # TODO: Add product to cart
        pass
    
    def remove_item(self, product): # TODO: Remove product from cart
        pass
    
    def calculate_total(self, discount_code=None):
        # TODO: Calculate total with optional discount
        # Apply percentage discount if code valid
        pass
    
    def checkout(self): # TODO: Process payment and clear cart
        pass
\end{lstlisting}
\end{frame}

%%%%%%%%%%%%%%%%%%%%%%%%%%%%%%%%%%%%%%%%%%%%%%%%%%%%%%%%%%%
% Slide 16: File I/O - Reading & Writing
%%%%%%%%%%%%%%%%%%%%%%%%%%%%%%%%%%%%%%%%%%%%%%%%%%%%%%%%%%%
\begin{frame}[fragile]\frametitle{File I/O - Reading \& Writing Text Files}
\begin{columns}
    \begin{column}[T]{0.6\linewidth}
      \begin{itemize}
        \item open() function with modes (r, w, a)
        \item Reading: read(), readline(), readlines()
        \item Writing: write(), writelines()
        \item Context managers (with statement)
        \item File paths - absolute vs relative
        \item Automatic file closing
      \end{itemize}
    \end{column}
    \begin{column}[T]{0.4\linewidth}
      \begin{lstlisting}[language=python, basicstyle=\tiny]
# Reading a file
with open('data.txt', 'r') as file:
    content = file.read()
    print(content)

# Reading line by line
with open('data.txt', 'r') as file:
    for line in file:
        print(line.strip())

# Writing to a file
data = ["Line 1\n", "Line 2\n", "Line 3\n"]
with open('output.txt', 'w') as file:
    file.writelines(data)

# Appending to a file
with open('log.txt', 'a') as file:
    file.write("New log entry\n")
      \end{lstlisting}
    \end{column}
  \end{columns}
\end{frame}

%%%%%%%%%%%%%%%%%%%%%%%%%%%%%%%%%%%%%%%%%%%%%%%%%%%%%%%%%%%
% Slide 17: JSON Handling
%%%%%%%%%%%%%%%%%%%%%%%%%%%%%%%%%%%%%%%%%%%%%%%%%%%%%%%%%%%
\begin{frame}[fragile]\frametitle{JSON Handling}
\begin{columns}
    \begin{column}[T]{0.6\linewidth}
      \begin{itemize}
        \item json module for JSON operations
        \item json.dump() - write to file
        \item json.dumps() - convert to string
        \item json.load() - read from file
        \item json.loads() - parse from string
        \item Handling nested structures
      \end{itemize}
    \end{column}
    \begin{column}[T]{0.4\linewidth}
      \begin{lstlisting}[language=python, basicstyle=\tiny]
import json

# Python dict to JSON
data = {
    "name": "Alice",
    "age": 30,
    "skills": ["Python", "SQL"]
}

# Write to file
with open('data.json', 'w') as f:
    json.dump(data, f, indent=4)

# Read from file
with open('data.json', 'r') as f:
    loaded_data = json.load(f)

# Convert to/from string
json_string = json.dumps(data)
parsed_data = json.loads(json_string)
      \end{lstlisting}
    \end{column}
  \end{columns}
\end{frame}

%%%%%%%%%%%%%%%%%%%%%%%%%%%%%%%%%%%%%%%%%%%%%%%%%%%%%%%%%%%
% Slide 18: CSV Processing
%%%%%%%%%%%%%%%%%%%%%%%%%%%%%%%%%%%%%%%%%%%%%%%%%%%%%%%%%%%
\begin{frame}[fragile]\frametitle{CSV Processing}
\begin{columns}
    \begin{column}[T]{0.6\linewidth}
      \begin{itemize}
        \item csv module for CSV operations
        \item csv.reader() for reading
        \item csv.writer() for writing
        \item DictReader for dictionary access
        \item DictWriter for writing dicts
        \item Handling different delimiters
      \end{itemize}
    \end{column}
    \begin{column}[T]{0.4\linewidth}
      \begin{lstlisting}[language=python, basicstyle=\tiny]
import csv

# Reading CSV
with open('data.csv', 'r') as file:
    reader = csv.reader(file)
    for row in reader:
        print(row)

# Reading as dictionaries
with open('data.csv', 'r') as file:
    reader = csv.DictReader(file)
    for row in reader:
        print(row['name'], row['age'])

# Writing CSV
data = [['Name', 'Age'], 
        ['Alice', 30], 
        ['Bob', 25]]
with open('output.csv', 'w', newline='') as file:
    writer = csv.writer(file)
    writer.writerows(data)
      \end{lstlisting}
    \end{column}
  \end{columns}
\end{frame}

%%%%%%%%%%%%%%%%%%%%%%%%%%%%%%%%%%%%%%%%%%%%%%%%%%%%%%%%%%%
% Slide 19: Exception Handling
%%%%%%%%%%%%%%%%%%%%%%%%%%%%%%%%%%%%%%%%%%%%%%%%%%%%%%%%%%%
\begin{frame}[fragile]\frametitle{Exception Handling}
\begin{columns}
    \begin{column}[T]{0.6\linewidth}
      \begin{itemize}
        \item try-except blocks
        \item Multiple except clauses
        \item else clause - executes if no exception
        \item finally clause - always executes
        \item Raising exceptions with raise
        \item Custom exception classes
      \end{itemize}
    \end{column}
    \begin{column}[T]{0.4\linewidth}
      \begin{lstlisting}[language=python, basicstyle=\tiny]
# Basic exception handling
try:
    result = 10 / 0
except ZeroDivisionError as e:
    print(f"Error: {e}")
except Exception as e:
    print(f"Unexpected error: {e}")
else:
    print("Success!")
finally:
    print("Cleanup code")

# Custom exception
class InvalidAgeError(Exception):
    pass

def set_age(age):
    if age < 0:
        raise InvalidAgeError("Age cannot be negative")
    return age

try:
    set_age(-5)
except InvalidAgeError as e:
    print(e)
      \end{lstlisting}
    \end{column}
  \end{columns}
\end{frame}

%%%%%%%%%%%%%%%%%%%%%%%%%%%%%%%%%%%%%%%%%%%%%%%%%%%%%%%%%%%
% Slide 20: Decorators
%%%%%%%%%%%%%%%%%%%%%%%%%%%%%%%%%%%%%%%%%%%%%%%%%%%%%%%%%%%
\begin{frame}[fragile]\frametitle{Decorators Basics}
\begin{columns}
    \begin{column}[T]{0.6\linewidth}
      \begin{itemize}
        \item Functions as first-class objects
        \item Wrapper functions
        \item @ syntax for applying decorators
        \item functools.wraps for metadata
        \item Common use cases: logging, timing, caching
        \item Parameterized decorators
      \end{itemize}
    \end{column}
    \begin{column}[T]{0.4\linewidth}
      \begin{lstlisting}[language=python, basicstyle=\tiny]
import time
from functools import wraps

def timer(func):
    @wraps(func)
    def wrapper(*args, **kwargs):
        start = time.time()
        result = func(*args, **kwargs)
        end = time.time()
        print(f"{func.__name__} took {end-start:.2f}s")
        return result
    return wrapper

def log_call(func):
    @wraps(func)
    def wrapper(*args, **kwargs):
        print(f"Calling {func.__name__}")
        return func(*args, **kwargs)
    return wrapper

@timer
@log_call
def process_data(n):
    return sum(range(n))
      \end{lstlisting}
    \end{column}
  \end{columns}
\end{frame}

%%%%%%%%%%%%%%%%%%%%%%%%%%%%%%%%%%%%%%%%%%%%%%%%%%%%%%%%%%%
% Slide 21: Context Managers
%%%%%%%%%%%%%%%%%%%%%%%%%%%%%%%%%%%%%%%%%%%%%%%%%%%%%%%%%%%
\begin{frame}[fragile]\frametitle{Context Managers}
\begin{columns}
    \begin{column}[T]{0.5\linewidth}
      \begin{itemize}
        \item with statement for resource management
        \item \_\_enter\_\_ and \_\_exit\_\_ methods
        \item contextlib module
        \item @contextmanager decorator
        \item Automatic cleanup of resources
        \item Custom context managers
      \end{itemize}
    \end{column}
    \begin{column}[T]{0.5\linewidth}
      \begin{lstlisting}[language=python, basicstyle=\tiny]
from contextlib import contextmanager

# Custom context manager class
class FileManager:
    def __init__(self, filename, mode):
        self.filename = filename
        self.mode = mode
    
    def __enter__(self):
        self.file = open(self.filename, self.mode)
        return self.file
    
    def __exit__(self, exc_type, exc_val, exc_tb):
        self.file.close()

# Using decorator
@contextmanager
def timer_context():
    start = time.time()
    yield
    print(f"Elapsed: {time.time()-start:.2f}s")

with timer_context():
    # Code to time
    time.sleep(1)
      \end{lstlisting}
    \end{column}
  \end{columns}
\end{frame}

%%%%%%%%%%%%%%%%%%%%%%%%%%%%%%%%%%%%%%%%%%%%%%%%%%%%%%%%%%%
% Slide 22: Project - CSV Data Processor
%%%%%%%%%%%%%%%%%%%%%%%%%%%%%%%%%%%%%%%%%%%%%%%%%%%%%%%%%%%
\begin{frame}[fragile]\frametitle{Project: CSV Data Processor}
\begin{lstlisting}[language=python, basicstyle=\tiny]
import csv
import logging

def log_operation(func):
    """Decorator to log operations"""
    def wrapper(*args, **kwargs):
        logging.info(f"Starting {func.__name__}")
        try:
            result = func(*args, **kwargs)
            logging.info(f"Completed {func.__name__}")
            return result
        except Exception as e:
            logging.error(f"Error in {func.__name__}: {e}")
            raise
    return wrapper

class CSVProcessor:
    def __init__(self, input_file):
        self.input_file = input_file
    @log_operation
    def read_data(self): # TODO: Read CSV with error handling
        pass
    @log_operation
    def filter_data(self, condition): # TODO: Filter rows based on condition
        pass
    @log_operation
    def transform_data(self, transformation): # TODO: Apply transformation to columns
        pass
    @log_operation
    def export_results(self, output_file): # TODO: Write processed data to new CSV
        pass
\end{lstlisting}
\end{frame}

%%%%%%%%%%%%%%%%%%%%%%%%%%%%%%%%%%%%%%%%%%%%%%%%%%%%%%%%%%%
% Slide 23: Project - Config File Manager
%%%%%%%%%%%%%%%%%%%%%%%%%%%%%%%%%%%%%%%%%%%%%%%%%%%%%%%%%%%
\begin{frame}[fragile]\frametitle{Project: Config File Manager}
\begin{lstlisting}[language=python, basicstyle=\tiny]
import json
import yaml
from contextlib import contextmanager
from datetime import datetime

class ConfigManager:
    def __init__(self, config_file):
        self.config_file = config_file
        self.config = {}
    
    @contextmanager 
    def config_context(self): # TODO: Backup current config
        """Context manager for safe config operations"""
        try:
            yield self
        except Exception as e: # TODO: Restore from backup
            raise
        finally: # TODO: Cleanup
            pass
    
    def load_config(self): # TODO: Load JSON/YAML config
        pass
    def validate_config(self, schema): # TODO: Validate against schema
        pass
    def save_config(self): # TODO: Save config with backup
        pass
    def get(self, key, default=None): # TODO: Get config value with dot notation
        pass
\end{lstlisting}
\end{frame}

%%%%%%%%%%%%%%%%%%%%%%%%%%%%%%%%%%%%%%%%%%%%%%%%%%%%%%%%%%%
% Slide 24: List/Dict/Set Comprehensions
%%%%%%%%%%%%%%%%%%%%%%%%%%%%%%%%%%%%%%%%%%%%%%%%%%%%%%%%%%%
\begin{frame}[fragile]\frametitle{Comprehensions}
\begin{columns}
    \begin{column}[T]{0.6\linewidth}
      \begin{itemize}
        \item List comprehensions for concise lists
        \item Dict comprehensions for dictionaries
        \item Set comprehensions for unique values
        \item Conditional logic in comprehensions
        \item Nested comprehensions
        \item Performance benefits
      \end{itemize}
    \end{column}
    \begin{column}[T]{0.4\linewidth}
      \begin{lstlisting}[language=python, basicstyle=\tiny]
# List comprehension
squares = [x**2 for x in range(10)]
evens = [x for x in range(20) if x % 2 == 0]

# Dict comprehension
word_lengths = {word: len(word) 
                for word in ['apple', 'banana']}

# Set comprehension
unique_squares = {x**2 for x in range(-5, 6)}

# Nested comprehension
matrix = [[i*j for j in range(3)] 
          for i in range(3)]

# Conditional expression
results = [x if x > 0 else 0 
           for x in range(-5, 5)]
      \end{lstlisting}
    \end{column}
  \end{columns}
\end{frame}

%%%%%%%%%%%%%%%%%%%%%%%%%%%%%%%%%%%%%%%%%%%%%%%%%%%%%%%%%%%
% Slide 25: Generators
%%%%%%%%%%%%%%%%%%%%%%%%%%%%%%%%%%%%%%%%%%%%%%%%%%%%%%%%%%%
\begin{frame}[fragile]\frametitle{Generators \& Generator Expressions}
\begin{columns}
    \begin{column}[T]{0.6\linewidth}
      \begin{itemize}
        \item Memory-efficient iteration
        \item yield keyword for generators
        \item Generator expressions (lazy evaluation)
        \item next() function
        \item StopIteration exception
        \item Infinite sequences
      \end{itemize}
    \end{column}
    \begin{column}[T]{0.4\linewidth}
      \begin{lstlisting}[language=python, basicstyle=\tiny]
# Generator function
def fibonacci(n):
    a, b = 0, 1
    for _ in range(n):
        yield a
        a, b = b, a + b

# Using generator
for num in fibonacci(10):
    print(num)

# Generator expression
squares_gen = (x**2 for x in range(1000000))
print(next(squares_gen))  # 0
print(next(squares_gen))  # 1

# Infinite generator
def infinite_counter():
    n = 0
    while True:
        yield n
        n += 1
      \end{lstlisting}
    \end{column}
  \end{columns}
\end{frame}

%%%%%%%%%%%%%%%%%%%%%%%%%%%%%%%%%%%%%%%%%%%%%%%%%%%%%%%%%%%
% Slide 26: Iterators & Advanced Iteration
%%%%%%%%%%%%%%%%%%%%%%%%%%%%%%%%%%%%%%%%%%%%%%%%%%%%%%%%%%%
\begin{frame}[fragile]\frametitle{Iterators \& Advanced Iteration}
\begin{columns}
    \begin{column}[T]{0.5\linewidth}
      \begin{itemize}
        \item Iterator protocol: \_\_iter\_\_ and \_\_next\_\_
        \item itertools module
        \item Chain, cycle, repeat functions
        \item Combinations and permutations
        \item zip() and zip\_longest()
        \item Creating custom iterators
      \end{itemize}
    \end{column}
    \begin{column}[T]{0.5\linewidth}
      \begin{lstlisting}[language=python, basicstyle=\tiny]
from itertools import islice, chain, cycle

# Custom iterator
class CountDown:
    def __init__(self, start):
        self.current = start
    
    def __iter__(self):
        return self
    
    def __next__(self):
        if self.current <= 0:
            raise StopIteration
        self.current -= 1
        return self.current + 1

# Using itertools
combined = chain([1, 2], [3, 4])
first_five = islice(range(100), 5)

# Zip multiple iterables
names = ['Alice', 'Bob']
ages = [30, 25]
for name, age in zip(names, ages):
    print(f"{name}: {age}")
      \end{lstlisting}
    \end{column}
  \end{columns}
\end{frame}

%%%%%%%%%%%%%%%%%%%%%%%%%%%%%%%%%%%%%%%%%%%%%%%%%%%%%%%%%%%
% Slide 27: Threading & Multiprocessing
%%%%%%%%%%%%%%%%%%%%%%%%%%%%%%%%%%%%%%%%%%%%%%%%%%%%%%%%%%%
\begin{frame}[fragile]\frametitle{Concurrent Operations - Threading}
\begin{columns}
    \begin{column}[T]{0.5\linewidth}
      \begin{itemize}
        \item threading module for concurrency
        \item Thread creation and management
        \item Global Interpreter Lock (GIL)
        \item Thread-safe operations
        \item concurrent.futures.ThreadPoolExecutor
        \item Use for I/O-bound tasks
      \end{itemize}
    \end{column}
    \begin{column}[T]{0.5\linewidth}
      \begin{lstlisting}[language=python, basicstyle=\tiny]
import threading
from concurrent.futures import ThreadPoolExecutor
import time

# Basic threading
def task(name):
    print(f"Task {name} starting")
    time.sleep(1)
    print(f"Task {name} complete")

threads = []
for i in range(3):
    t = threading.Thread(target=task, args=(i,))
    threads.append(t)
    t.start()

for t in threads:
    t.join()

# ThreadPoolExecutor
with ThreadPoolExecutor(max_workers=3) as executor:
    futures = [executor.submit(task, i) for i in range(3)]
      \end{lstlisting}
    \end{column}
  \end{columns}
\end{frame}

%%%%%%%%%%%%%%%%%%%%%%%%%%%%%%%%%%%%%%%%%%%%%%%%%%%%%%%%%%%
% Slide 28: Async/Await Basics
%%%%%%%%%%%%%%%%%%%%%%%%%%%%%%%%%%%%%%%%%%%%%%%%%%%%%%%%%%%
\begin{frame}[fragile]\frametitle{Async/Await Basics}
\begin{columns}
    \begin{column}[T]{0.5\linewidth}
      \begin{itemize}
        \item Asynchronous programming concepts
        \item async def for coroutines
        \item await keyword for awaiting
        \item Event loop with asyncio
        \item Concurrent execution of coroutines
        \item Non-blocking I/O operations
      \end{itemize}
    \end{column}
    \begin{column}[T]{0.5\linewidth}
      \begin{lstlisting}[language=python, basicstyle=\tiny]
import asyncio

# Basic async function
async def fetch_data(id):
    print(f"Fetching {id}...")
    await asyncio.sleep(1)  # Simulates I/O
    print(f"Fetched {id}")
    return f"Data {id}"

# Running coroutines
async def main():
    # Sequential
    result1 = await fetch_data(1)
    
    # Concurrent
    results = await asyncio.gather(
        fetch_data(2),
        fetch_data(3),
        fetch_data(4)
    )
    print(results)

# Run event loop
asyncio.run(main())
      \end{lstlisting}
    \end{column}
  \end{columns}
\end{frame}

%%%%%%%%%%%%%%%%%%%%%%%%%%%%%%%%%%%%%%%%%%%%%%%%%%%%%%%%%%%
% Slide 29: Project - Async Web Scraper
%%%%%%%%%%%%%%%%%%%%%%%%%%%%%%%%%%%%%%%%%%%%%%%%%%%%%%%%%%%
\begin{frame}[fragile]\frametitle{Project: Async Web Scraper}
\begin{lstlisting}[language=python, basicstyle=\tiny]
import asyncio
import aiohttp
from bs4 import BeautifulSoup

class AsyncWebScraper:
    def __init__(self, urls):
        self.urls = urls
        self.results = []
    
    async def fetch_url(self, session, url):
        """Fetch single URL asynchronously"""
        try: # TODO: Get HTML content, Parse with BeautifulSoup, Extract required data
            async with session.get(url) as response:
                pass
        except Exception as e:
            print(f"Error fetching {url}: {e}")
            return None
    
    async def scrape_all(self): # TODO: Gather all results concurrently
        async with aiohttp.ClientSession() as session:
            tasks = [self.fetch_url(session, url) for url in self.urls]
            pass
    
    async def save_results(self, filename): # TODO: Write results to file
        pass

# asyncio.run(scraper.scrape_all())
\end{lstlisting}
\end{frame}

%%%%%%%%%%%%%%%%%%%%%%%%%%%%%%%%%%%%%%%%%%%%%%%%%%%%%%%%%%%
% Slide 30: Project - Concurrent File Processor
%%%%%%%%%%%%%%%%%%%%%%%%%%%%%%%%%%%%%%%%%%%%%%%%%%%%%%%%%%%
\begin{frame}[fragile]\frametitle{Project: Concurrent File Processor}
\begin{lstlisting}[language=python, basicstyle=\tiny]
import asyncio
import aiofiles
from pathlib import Path

class ConcurrentFileProcessor:
    def __init__(self, file_paths):
        self.file_paths = file_paths
        self.progress = {}
    
    async def process_file(self, file_path):
        try: # TODO: Read file in chunks, Process, Update progress
            async with aiofiles.open(file_path, 'r') as f:
                pass
        except Exception as e:
            print(f"Error processing {file_path}: {e}")
    
    async def track_progress(self):
        while True: # TODO: Display progress for all files
            await asyncio.sleep(0.5) # TODO: Break when all complete
            pass
    
    async def process_all(self): # TODO: Run all tasks concurrently
        tasks = [self.process_file(fp) for fp in self.file_paths]
        progress_task = asyncio.create_task(self.track_progress())
        pass

# processor = ConcurrentFileProcessor(file_list)
# asyncio.run(processor.process_all())
\end{lstlisting}
\end{frame}

%%%%%%%%%%%%%%%%%%%%%%%%%%%%%%%%%%%%%%%%%%%%%%%%%%%%%%%%%%%
% Slide 31: Conclusion & Best Practices
%%%%%%%%%%%%%%%%%%%%%%%%%%%%%%%%%%%%%%%%%%%%%%%%%%%%%%%%%%%
\begin{frame}[fragile]\frametitle{Conclusion \& Best Practices}

      \begin{itemize}
        \item Follow PEP 8 style guide
        \item Write readable, self-documenting code
        \item Use type hints for clarity
        \item Test your code thoroughly
        \item Handle exceptions appropriately
        \item Document with docstrings
        \item Use virtual environments
        \item Version control with Git
      \end{itemize}

\end{frame}

%%%%%%%%%%%%%%%%%%%%%%%%%%%%%%%%%%%%%%%%%%%%%%%%%%%%%%%%%%%
% Slide 32: Further Learning Resources
%%%%%%%%%%%%%%%%%%%%%%%%%%%%%%%%%%%%%%%%%%%%%%%%%%%%%%%%%%%
\begin{frame}\frametitle{Further Learning Resources}
      \begin{itemize}
		\item Home page -- http://www.python.org
		\item Wiki -- http://wiki.python.org/
		\item  Packages -- https://pypi.python.org/pypi
		\item Projects at http://sourceforge.net and github.org
        \item Real Python tutorials
        \item Python Package Index (PyPI)
        \item Stack Overflow community
        \item GitHub open source projects
        \item Python Enhancement Proposals (PEPs)
        \item Online courses and certifications
      \end{itemize}

\end{frame}


